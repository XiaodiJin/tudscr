%\documentclass{minimal}
%\usepackage{scrfontsizes}
%\generatefontfile{tudscrsize}[13.6pt]{11pt}
%\begin{document}\end{document}
\RequirePackage[ngerman=ngerman-x-latest]{hyphsubst}
\documentclass[%
  ngerman,automark,%
%  twoside,open=right,%
%  cd=litecolor,
%  widehead,
%  cd=no,%cdfont=heavy,
%  cdhead=wide,%cdfont=nodin,cdfont=sansmath,cdfont=upgreek,
%  cdtitle=no,
%  cdfoot=-5mm,automark,
%   ddcfoot,%color=pale,
%  geometry=no,DIV=9,%twoside,
%  widehead,
%  BCOR=10mm,
%  headinclude,
%  footinclude,
%  twoside,
%  draft,
%  cd=color,
%paper=200mm:287mm,pagesize,
%  paper=a3,%twoside,
%  cdfont=no,
%  headheight=45pt,
%  footheight=45pt,
%  footlines=3,%2.5,
%  extrabottommargin=50pt,
%  cd=no,
%  toc=listofnumbered,
%  fontsize=7pt,%7.25pt,
%  tudscrver=2.02,%ddcfoot,
%  ddcfoot,
%  version=3.12,
%  listof=leveldown,
%  listof=totoc,
%  listof=numbered,
%  twocolumn,
%  titlepage=no,
%fontsize=36,
%sansmath=no,
%footsepline,footbotline,
%cdfont=nodin,
%fontspec,
%cdfonts=no,%cdmath,
%headings=small,
cdfoot,
chapterpage,
%geometry=false,
]%
%{tudscrartcl}
{tudscrreprtnew}

%\AfterPackage!{layout}{\typeout{+++++}}

%{tudscrreprt}
\usepackage{selinput}\SelectInputMappings{adieresis={ä},germandbls={ß}}
\usepackage[T1]{fontenc}
\usepackage{babel}
\usepackage{blindtext}
%\usepackage{showframe}
\usepackage{layout}
\usepackage{hyperref}
%\usepackage[RGB]{tudscrcolor}

%\usepackage{scrlayer-scrpage}
%\TUDoptions{cdfoot=on}

\begin{document}
\date{16.12.2014}
\author{Mickey Mouse}
\title{Die Klasse tudscrposter}
\subtitle{Eine \LaTeX-Klasse für die Evaluationsposter}
\ifcsdef{tudcls@name}{
\faculty{Juristische Fakultät}
\department{Fachrichtung Strafrecht}
\institute{Institut für Kriminologie}
\chair{Lehrstuhl für Kriminalprognose und noch etwas}
\extraheadline{Lehrstuhl für Kriminalprognose}
\subject{text}
\thesis{master}
\advisor{Dipl.-Ing. Irgendwas}
\professor{Dr.-Ing. Irgendwer}
}{}

\blinddocument
\TUDoptions{geometry=true}
\blinddocument


\makeatletter
\newcommand{\bla}{\Blindtext[2][2]}
\makeatother

\pagestyle{tudheadings}

\bla
\TUDoptions{cdhead=nocolor,ddcfoot,cdfoot=no}
\bla
%
\part{test}\bla\chapter{test}\bla
\TUDoptions{cd=lite}
\part{test}\bla\chapter{test}\bla
\TUDoptions{cd=bicolor}
\part{test}\bla\chapter{test}\bla
\TUDoptions{cd=color}
\part{test}\bla\chapter{test}\bla
\TUDoptions{cd=fullcolor}
\part{test}\bla\chapter{test}\bla

\end{document}
