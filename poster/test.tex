\RequirePackage[ngerman=ngerman-x-latest]{hyphsubst}
\documentclass[%
  ngerman,automark,parskip=full,%
%  twoside,open=right,%
%  color,
%  widehead,
%cdfont=no,
%cdtitle=no,
%  cdfoot,automark,
%  ddcfoot,%color=pale,
%  geometry=no,
%  DIV=13,%twoside,
%  headinclude,
%  footinclude,
%  twoside,
%  draft,
%%  cd=color,
%  paper=a3,
%cdfont=no,
%footinclude,
%headheight=45pt,
%footheight=45pt,
%footlines=3,%2.5,
%extrabottommargin=2cm,
%cd=no,
%toc=listofnumbered,
%fontsize=9pt,
%tudscrver=2.02,
%listof=leveldown,
%listof=totoc,
%listof=numbered,
]%
%{tudscrreprt}
{tudscrreprtnew}
%{scrreprt}
\usepackage{selinput}\SelectInputMappings{adieresis={ä},germandbls={ß}}
\usepackage[T1]{fontenc}
\usepackage{babel}
\usepackage{blindtext}
\usepackage{showframe}
%\usepackage{layout}
\usepackage{twocolfix,multicol}
\usepackage{etoolbox}
\usepackage{scrlayer-scrpage}
\usepackage{tudscrx,twocolfix}
\usepackage{hyperref,afterpage}

\begin{document}

\date{16.12.2014}
\author{Mickey Mouse}
\title{Die Klasse tudscrposter}
\subtitle{Eine \LaTeX-Klasse für die Evaluationsposter}

\ifcsdef{tudcls@name}{%
\faculty{Juristische Fakultät}
\department{Fachrichtung Strafrecht}
\institute{Institut für Kriminologie}
\chair{Lehrstuhl für Kriminalprognose und noch etwas}
\extraheadline{Lehrstuhl für Kriminalprognose}
%\subject{text}
\thesis{master}
\advisor{Dipl.-Ing. Irgendwas}
\professor{Dr.-Ing. Irgendwer}
}{}


\maketitle
\TUDoptions{parttitle}
\part{Die Klasse tudscrposter}
\addpart{Die Klasse tudscrposter}
\title{}
\part*{Die Klasse tudscrposter}
\addpart*{Die Klasse tudscrposter}
\title{Die Klasse tudscrposter}
\part{Die Klasse tudscrposter}
\TUDoptions{parttitle=no}
\part{Die Klasse tudscrposter aaa}
\addpart{Die Klasse tudscrposter}
\part*{Die Klasse tudscrposter}
\addpart*{Die Klasse tudscrposter}
\KOMAoptions{chapterprefix=no}
\chapter{Die Klasse tudscrposter aaa}
\Blindtext[3][1]
\addchap{Die Klasse tudscrposter bbb}
\Blindtext[3][1]
\chapter*{Die Klasse tudscrposter ccc}
\Blindtext[3][1]
\addchap*{Die Klasse tudscrposter ddd}
\Blindtext[3][1]
\KOMAoptions{chapterprefix}
\TUDoptions{parttitle}
\part{Die Klasse tudscrposter}
%\setcounter{secnumdepth}{-2}
\part{Die Klasse tudscrposter}
\chapter{Die Klasse tudscrposter aaa}
\Blindtext[3][1]
\addchap{Die Klasse tudscrposter bbb}
\Blindtext[3][1]
\chapter*{Die Klasse tudscrposter ccc}
\Blindtext[3][1]
\addchap*{Die Klasse tudscrposter ddd}
\Blindtext[3][1]
\chapter{Die Klasse tudscrposter aaa}
\Blindtext[3][1]

\end{document}

\RequirePackage[ngerman=ngerman-x-latest]{hyphsubst}
\documentclass[ngerman,automark]{scrreprt}
\usepackage{selinput}\SelectInputMappings{adieresis={ä},germandbls={ß}}
\usepackage[T1]{fontenc}
\usepackage{babel}
\usepackage{blindtext}
\usepackage[automark]{scrlayer-scrpage}
\usepackage{xcolor}
\usepackage{blindtext}

\definecolor{HKS41}{cmyk}{1.00,0.70,0.10,0.50}
\DeclareLayer[
  background,page,%
  contents={%
    \color{HKS41}%
    \rule[-\dp\strutbox]{\layerwidth}{\layerheight}%
  },%
]{mylayer}
\DeclareNewPageStyleByLayers[%
  onbackground=\color{white},
]{mypagestyle}{%
  scrheadings.head.odd,%
  scrheadings.head.even,%
  scrheadings.head.oneside,%
  scrheadings.head.above.line,%
  scrheadings.head.below.line,%
  scrheadings.foot.odd,%
  scrheadings.foot.even,%
  scrheadings.foot.oneside,%
  scrheadings.foot.above.line,%
  scrheadings.foot.below.line,%
  mylayer,%
}
\begin{document}
\Blindtext[4][2]
\clearpage

\pagestyle{mypagestyle}
{
\color{white}
\Blindtext[4][2]
\clearpage
}

\pagestyle{scrheadings}
\Blindtext[4][2]

\end{document}

