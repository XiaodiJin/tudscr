% Beispiel für Klasse tudposter.cls (Version 0.5).
%%
%% Autor: Martin Zabel (martin.zabel@tu-dresden.de)
%%        Tobias schlemmer (tobias.schlemmer@mailbox.tu-dresden.de)
%%
%% Klassenoptionen siehe tudposter.cls.
%%
%%
%% Wir bevorzugen PDFLaTeX. Für den Fall, dass die CD-Schriften nicht
%% in der Standardkonfiguration erwähnt sind laden wir die .map-Datei
%% Wichtig: das muss vor dem ersten Font-Befehl (also vor der
%% Dokumentklasse geschehen.
%
% Diese Zeilen sind jetzt auskommentiert, da dies das Paket „tudfonts“
% jetzt übernimmt
%
%\pdfmapfile{+univers.map}
%\pdfmapfile{+dinbold.map}
%%
\documentclass[a0paper,noDIN,Mathematik]{tudmathposter}
\listfiles
% Bei Bedarf: dinBold für Section. Dann aber auch Überschrift nur in
% Großbuchstaben.
%\addtokomafont{section}{\dinBold}

% In diesem Poster benötigte Pakete.
% Jedem wie es beliebt.
\usepackage[T1]{fontenc}
\usepackage[utf8]{inputenc} % latin1 (Linux) oder ansinew (Win) wenn
                            % nicht UTF 8
\usepackage[ngerman]{babel}
\usepackage{eso-pic}
\usepackage{multicol}
\usepackage{listings}
\usepackage{tikz}
\usepackage{url}
\usepackage[pdfborder={0 0 0}]{hyperref}
%\usepackage{tabularx}
\usetikzlibrary{calc}
\usetikzlibrary{shapes.multipart}
\usetikzlibrary{positioning}


\makeatletter
\newcommand\nopt[1]{\strip@pt#1}%
\makeatother

\newlength{\zielbreite}%
\newlength{\zielhoehe}%
\newlength{\printmml}%

% Beschriftung im Seitenkopf und Seitenfuß
\institut{Institut für Algebra}
\institutslogofile{mathelogo}
\professur{Professur für \TeX nische Algebra}
\author{T. Schlemmer, M. Behrisch und D. Borchmann}% Zum Beispiel für Kontaktdaten.
\telefon{0351 463-35078}
\fax{0351 463-\dots}
\homepage{\hbox{\url{http://www.math.tu-dresden.de/~schlemme/tudmathposter/}\hspace{-1cm}}}
% Folgende Zeile ist nicht CD-konform
%\zweitlogofile{mathelogo}

\title{Die Dokumentklasse „tudmathposter“}
\subtitle{Eine \LaTeX klasse für die Evaluationsposter}
\grautabelle
\addto\captionsngerman{%
  \def\figurename{Abb.}%
  \def\tablename{Tab.}%
}%

\makeatletter
\newcommand\bemaszung{%
\begin{tikzpicture}[line width=8pt,draw=HKS92K100,text=HKS92K100,baseline=0pt]
\useasboundingbox (0,0) rectangle (\paperwidth,\paperheight);

%Hilfslinien
\draw[red!10!white,line width=1pt] (\oddsidemargin+1in+0.5\textwidth,0) -- +(0,\paperheight);

\fill[fill=HKS41K10] (0,\paperheight-2cm) rectangle
  +(\paperwidth,-\tudposter@headheight-\tudposter@cutborder+2cm);
\fill[fill=HKS41K10!50!white] (0,\paperheight-\tudposter@headheight-\tudposter@cutborder)
  rectangle +(\paperwidth,-\tudposter@cbheight);
\fill[fill=HKS41K10] (0,0) rectangle (\paperwidth,\tudposter@footheight+\tudposter@cutborder);

\draw[HKS92K30] (\oddsidemargin+1in,\paperheight-\headheight-\headsep)
  rectangle +(\textwidth,-\textheight);
\draw[HKS92K30,line width=4pt] (\oddsidemargin+1in,0) -- +(0,\paperheight);
\draw[HKS92K30,line width=4pt] (\oddsidemargin+1in+\textwidth,0) -- +(0,\paperheight);
\draw (\tudposter@logospacex+\tudposter@cutborder,\paperheight-\tudposter@logospacey-\tudposter@cutborder)
  node[anchor=north west,inner sep=0pt,outer
  sep=0pt](logo){\color{white}\includegraphics[width=\tudposter@logowidth]{\tudposter@logo}};
\draw[HKS92K30] (logo.south east) rectangle (logo.north west);

\begingroup
  \fontsize{\tudposter@cbfontsize}\tudposter@cbfontsize\selectfont
  \setlength\@tempdima{\depthof{g}}%
  \draw
    (\oddsidemargin+1in,\paperheight-\headheight-\@tempdima+\tudposter@cbtextraise)
    node [anchor=south west,text=white,inner sep=0pt,outer sep=0pt] (fachrichtung)
    {\raisebox{-\depth}{\textbf{Fakultät für Mathematik und
          Naturwissenschaften}~~Fachrichtung Mathematik}};
  \draw[HKS92K30,line width=4pt] (fachrichtung.north west) -- +(\textwidth-\oddsidemargin -1in,0);
  \draw[HKS92K30,line width=2pt]
    (\oddsidemargin+1in,\paperheight-\headheight+\tudposter@cbtextraise)--
    +(\textwidth-\oddsidemargin -1in,0);
  \draw[HKS92K30,line width=4pt] (fachrichtung.south west) -- +(\textwidth-\oddsidemargin -1in,0);
\endgroup
\draw (\oddsidemargin+1in,0.5\tudposter@footheight+\tudposter@cutborder)
node [draw=HKS92K30,anchor=west,color=HKS92K10,text width=\textwidth,inner sep=0pt,outer sep=0pt]{\@fusszeile};
\draw [dotted,line width=2pt] (\tudposter@cutborder,\tudposter@cutborder) --
  ++ (\paperwidth-2\tudposter@cutborder,0) --
  ++(0,\paperheight-2cm-2\tudposter@cutborder) node [pos=0.3,sloped,above]
  {tudposter@cutborder +(\textbackslash schnittrand) – Papiergröße}--
  +(-\paperwidth+2\tudposter@cutborder,0) -- cycle;

\draw[<->,HKS92K30] (0,0.5\paperheight) -- +(\paperwidth,0)
  node[pos=0.46,above] {paperwidth};
\draw[<->] (0.5\paperwidth,0) -- (0.5\paperwidth,\paperheight)
  node[pos=0.6,sloped,below]{paperheight};
\draw[<->] (0.6\paperwidth,\paperheight) -- +(0,-2cm) node [pos=0.5,auto,xshift=0.5cm]
  {topmargin};
\draw[<->] (0.6\paperwidth,\paperheight-\tudposter@headheight-\tudposter@cutborder) --
  +(0,\tudposter@headheight-2cm)
  node[pos=0.5,sloped,below]{tudposter@headheight};
\draw[<->] (0.6\paperwidth,\paperheight-\tudposter@headheight-\tudposter@cutborder) --
  +(0,-\tudposter@cbheight) node[pos=0.5,anchor=west]
  {tudposter@cbheight};
\draw[<->] (0.6\paperwidth,\tudposter@cutborder) -- +(0,\tudposter@footheight)
  node[pos=0.5,sloped,below] {tudposter@footheight};
\draw[line cap=butt]
  (0.17\paperwidth,\paperheight-\tudposter@headheight-\tudposter@cbheight-\tudposter@cutborder)
  -- +(0,\tudposter@cbtextraise) node[pos=0.5,anchor=west] {tudposter@cbtextraise};
\draw [<-]
  (0.17\paperwidth,\paperheight-\tudposter@headheight-\tudposter@cbheight)
  -- +(0,-1cm);
\draw [<-]
  (0.17\paperwidth,\paperheight-\tudposter@headheight-\tudposter@cbheight+\tudposter@cbtextraise)
  -- +(0,1cm);

\begingroup
  \fontsize{\tudposter@cbfontsize}\tudposter@cbfontsize\selectfont
  \setlength\@tempdima{\depthof{g}}%
  \draw [<->]
    (0.3\paperwidth,\paperheight-\tudposter@headheight-\tudposter@cbheight+\tudposter@cbtextraise-\@tempdima-\tudposter@cutborder)
    -- +(0,\tudposter@cbfontsize) node [pos=0.5,anchor=west] {\small tudposter@cbfontsize};
\endgroup

\draw[<->] (0.91\paperwidth,\paperheight-\headheight) --
  +(0,\headheight-2cm) node[pos=0.5,sloped,below]{headheight};
\draw[<->] (0.91\paperwidth,\paperheight-\headheight-\headsep) --
  +(0,\headsep) node [pos=0.5,sloped,below] {headsep};
\draw[<->] (0.91\paperwidth,\paperheight-\headheight-\headsep-\textheight)
  -- +(0,\textheight) node [pos=0.1,sloped,below] {textheight};
\draw[<->] (0.91\paperwidth,\paperheight-\headheight-\headsep-\textheight-\footskip)
  -- +(0,\footskip) node [pos=0.5,sloped,below] {footskip};


\draw[<->] ($(logo.west)!0.5!(logo.north west)$) -- +(-\tudposter@logospacex,0)
  node[pos=0.9,anchor = south west,sloped] {tudposter@logospacex};
\draw[<-] ($(logo.west)!0.5!(logo.north west)$) -- +(-\tudposter@logoposx,0)
  node[pos=0.8,anchor = south west,sloped,below] {tudposter@logoposx};
\draw[<->] (logo.north west) -- +(0,\tudposter@logospacey-2cm)
  node[pos=0.5,anchor = west] {tudposter@logospacey};
\draw[<->] (logo.south west) -- +(0,-\tudposter@logoposy)
  node[pos=0.5,anchor = west] {tudposter@logoposy};
\draw[<->] ($(logo.west)!0.5!(logo.south west)$) -- +(\tudposter@logowidth,0)
  node[pos=0.5,sloped,below] {tudposter@logoswidth};

\draw[<->] (0,0.9\footskip) -- +(\oddsidemargin+1in,0) node
  [pos=0.5,sloped,above] {oddsidemargin} node[pos=0.5,sloped,below]
  {evensidemargin};
\draw[<->] (\oddsidemargin+1in,0.9\footskip) -- +(\textwidth,0) node
  [pos=0.40,sloped,above,anchor=north east] {textwidth};

\draw [HKS92K30,line width=4pt](\oddsidemargin+1in+\tudposter@footcolumnwidth,0) --
+(0,\tudposter@footheight+\tudposter@cutborder);
\draw [HKS92K30,line width=4pt](\oddsidemargin+1in+\tudposter@footcolumnwidth+\tudposter@footcolumnsep,0) --
+(0,\tudposter@footheight+\tudposter@cutborder);
\draw [HKS92K30,line width=4pt](\oddsidemargin+1in+2\tudposter@footcolumnwidth+\tudposter@footcolumnsep,0) --
+(0,\tudposter@footheight+\tudposter@cutborder);
\draw [HKS92K30,line width=4pt](\oddsidemargin+1in+\tudposter@institutslogopos,0) --
+(0,\tudposter@footheight+\tudposter@cutborder);
\draw[<->] (\oddsidemargin+1in,0.6\tudposter@footheight+\tudposter@cutborder) -- +(\tudposter@footcolumnwidth,0) node
  [pos=0.50,sloped,above] {tudposter@footcolumnwidth};
\draw[<->] (\oddsidemargin+1in+\tudposter@footcolumnwidth,0.6\tudposter@footheight+\tudposter@cutborder) -- +(\tudposter@footcolumnsep,0) node
  [pos=0.50,sloped,above] {tudposter@footcolumnsep};
\draw[<->] (\oddsidemargin+1in,0.3\tudposter@footheight+\tudposter@cutborder) -- +(\tudposter@institutslogopos,0) node
  [pos=0.30,sloped,above] {tudposter@institutslogopos};
\setlength\fusszeilenhoehe{\heightof{\parbox{1em}{\normalsize\strut\\\strut\\\strut\\\strut\\\strut}}
+\depthof{\parbox{1em}{\normalsize\strut\\\strut\\\strut\\\strut\\\strut}}}%
\begingroup\normalsize
\draw[<->] (\oddsidemargin+1in+\tudposter@institutslogopos+1cm,0.5\tudposter@footheight+\tudposter@cutborder+0.5\fusszeilenhoehe) -- +(0,-55mm) node
  [pos=0.30,sloped,above] {55mm (kleines Logo)};
\draw[<->] (\oddsidemargin+1in+\tudposter@institutslogopos+3cm,0.5\tudposter@footheight+\tudposter@cutborder+0.5\fusszeilenhoehe) -- +(0,-4\baselineskip) node
  [pos=0.30,sloped,above] {4 Zeilen};
\draw[<->] (\oddsidemargin+1in+\tudposter@institutslogopos+5cm,0.5\tudposter@footheight+\tudposter@cutborder+0.5\fusszeilenhoehe) -- +(0,-5\baselineskip) node
  [pos=0.30,sloped,above] {5 Zeilen};
\draw[<->] (\oddsidemargin+1in+\tudposter@institutslogopos+7cm,0.5\tudposter@footheight+\tudposter@cutborder+0.5\fusszeilenhoehe) -- +(0,-69mm) node
  [pos=0.30,sloped,above] {69mm};
\endgroup
\end{tikzpicture}

}
\newlength{\fusszeilenhoehe}
\newlength\myfs
\newlength\mybs
\newcommand\printfontsize[1]{%
  \def\jetztfontsize{#1}%
  \csname #1\endcsname
  \setlength\myfs{\f@size pt}%
  \setlength\mybs{\f@baselineskip}%
  \expandafter\let\csname #1gr.\endcsname\myfs
  \expandafter\let\csname #1 Z.-abst.\endcsname\mybs
  \small
  \printmm{#1gr.}%
  \printmm{#1 Z.-abst.}%
}
\makeatother
\begin{document}\thispagestyle{empty}\AddToShipoutPicture{\bemaszung}%
\small
\parskip\baselineskip
\newcommand\bscale[1]{\setlength{#1}{#1*\ratio{\zielbreite}{\textwidth}}}
\newcommand\hscale[1]{\setlength{#1}{#1*\ratio{\zielhoehe}{\textheight}}}
\newcommand\printmm[1]{%
  \item
  \expandafter\printmml\csname #1\endcsname
 #1 = \nopt\printmml\,pt =
 \setlength\printmml{\printmml*\ratio{1pt}{1mm}}%
  \nopt\printmml\,mm\\
 \expandafter\printmml\csname #1\endcsname
  \bscale{\printmml}
  #1 (b-skaliert) = \nopt\printmml\,pt =
  \setlength\printmml{\printmml*\ratio{1pt}{1mm}}%
  \nopt\printmml\,mm\\
  \expandafter\printmml\csname #1\endcsname
  \hscale{\printmml}
  #1 (h-skaliert) = \nopt\printmml\,pt =
  \setlength\printmml{\printmml*\ratio{1pt}{1mm}}%
  \nopt\printmml\,mm\par
}%
% Papiergrößen
%\setlength{\zielbreite}{851mm}%
%\setlength{\zielhoehe}{1199mm}%
\setlength{\zielbreite}{680.mm}%
\setlength{\zielhoehe}{885.98906mm}%

\begin{multicols}{2}
Die Register werden wie folgt skaliert:
b-skaliert: \[\text{Wert}*\frac{\text{\textbackslash
    zielbreite}}{\text{\textbackslash textwidth (Breite des Textblockes)}}\]
h-skaliert: \[\text{Wert}*\frac{\text{\textbackslash
    zielhoehe}}{\text{\textbackslash textheight (Höhe des Textblocks)}}\]

\makeatletter
\@tempdima=1cm
1cm = $\the\@tempdima$\\
\begin{itemize}
\begingroup
\addtolength\topmargin{1in}%
\addtolength\evensidemargin{1in}%
\addtolength\oddsidemargin{1in}%
\printmm{zielhoehe}
\printmm{zielbreite}
\printmm{paperheight}
\printmm{paperwidth}
\printmm{topmargin}
\printmm{headheight}
\printmm{headsep}
\printmm{topskip}
\printmm{textheight}
\printmm{footskip}
\printmm{oddsidemargin}
\printmm{evensidemargin}
\printmm{textwidth}
\printmm{tudposter@cutborder}
\printmm{tudposter@headheight}
\printmm{tudposter@logospacex}
\printmm{tudposter@logospacey}
\printmm{tudposter@logoposx}
\printmm{tudposter@logoposy}
\printmm{tudposter@logowidth}
\printmm{tudposter@cbheight}
\printmm{tudposter@cbfontsize}
\printmm{tudposter@cbtextraise}
\printmm{tudposter@footheight}
\printmm{tudposter@lmargin}
\printmm{tudposter@footcolumnwidth}
\printmm{tudposter@footcolumnsep}
\printmm{tudposter@institutslogopos}
\setlength\fusszeilenhoehe{\heightof{\parbox{1em}{\normalsize\strut\\\strut\\\strut\\\strut\\\strut}}}%
\printmm{fusszeilenhoehe}
% rechter Rand: lt. M. Kaden: Achsenabstand minus zweimal Querbalken
\printmm{tudposter@rmargin}
\printmm{baselineskip}
\makeatother
\endgroup
\end{itemize}
\end{multicols}
\pagebreak
\section{Schriftgrößen und Zeilenabstände}
\begin{multicols}{2}
\begin{itemize}
\printfontsize{tiny}
\printfontsize{scriptsize}
\printfontsize{footnotesize}
\printfontsize{small}
\printfontsize{normalsize}
\printfontsize{large}
\printfontsize{Large}
\printfontsize{LARGE}
\printfontsize{huge}
\printfontsize{Huge}
\end{itemize}
\end{multicols}
\newskip\printskips
\newcommand\printskip[1]{%
  \item
  \expandafter\printskips\csname #1\endcsname
 #1 = \nopt\printskips\,pt =
 \setlength\printskips{\printskips*\ratio{1pt}{1mm}}%
  \nopt\printskips\,mm\\
 \expandafter\printskips\csname #1\endcsname
  \bscale{\printskips}
  #1 (b-skaliert) = \nopt\printskips\,pt =
  \setlength\printskips{\printskips*\ratio{1pt}{1mm}}%
  \nopt\printskips\,mm\\
  \expandafter\printskips\csname #1\endcsname
  \hscale{\printskips}
  #1 (h-skaliert) = \nopt\printskips\,pt =
  \setlength\printskips{\printskips*\ratio{1pt}{1mm}}%
  \nopt\printskips\,mm\par
}%
\makeatletter
\newcommand*\printmasz[1]{%
  \@tempskipa#1\relax
  \printskip{@tempskipa}%
}
\makeatother
\pagebreak
\homepage{\hbox{\url{http://www.math.tu-dresden.de/~schlemme/}}}
\author{Tobias Schlemmer}% Zum Beispiel für Kontaktdaten.

\section{weitere Maße}
\begin{multicols}2
\begin{itemize}
  \printskip{smallskipamount}
  \printskip{medskipamount}
  \printskip{bigskipamount}

  \printskip{abovecaptionskip}
  \printskip{belowcaptionskip}
  \printskip{abovedisplayskip}
  \printskip{abovedisplayshortskip}
  \printskip{belowdisplayshortskip}

  \foreach \i in {1pt,2pt,3pt,4pt,5pt,6pt,7pt,8pt,68pt,26.3pt,43.2pt,13.4pt,907.3pt}
    \printmasz{\i};
  \end{itemize}
\end{multicols}
\pagebreak \homepage{\hbox{\url{http://www.math.tu-dresden.de/}}}
\vfill
Diese Hilfslinie zeigt das Seitenende an. \textbf{ZU ENTFERNEN}!
\hrule
\pagebreak
kleines Logo mit kurzer URL
\author{Dipl.-Math. mat.-in\v z. Tobias Schlemmer}
\homepage{\hbox{\url{http://www.math.tu-dresden.de/}\hphantom{schlemmemm}}}
\end{document}
