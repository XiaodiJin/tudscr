\RequirePackage[ngerman=ngerman-x-latest]{hyphsubst}
\documentclass[%
  ngerman,%
%  cdmath=false,
%  12pt,%
]%
{tudscrreprtnew}
%{scrreprt}
\usepackage{selinput}\SelectInputMappings{adieresis={ä},germandbls={ß}}
\usepackage[T1]{fontenc}
\usepackage{babel}
%\usepackage{libertine}
 
% gewünschtes Paket einkommentieren
%\usepackage{mathptmx}
%\usepackage{mathpple}
\usepackage{mathtools,amssymb}

%\usepackage{newtxmath}
%\usepackage{eulervm}

%matcal aus eulervm
%mathfrac und mathbb aus newtxmath

\usepackage[math]{blindtext}

% für Umgebung notice, welche das Datum automatisch rechts oben setzt
%\usepackage{tudscrsupervisor}
%\AtEndPreamble{%
%  \SetMathAlphabet{\mathcal}{univers}{OMS}{ntxsy}{b}{n}\tud@font@math@set%
%}%

\begin{document}

%\faculty{Fakultät}
%\department{Fachrichtung}
%\institute{Institut}
%\chair{Professur}
\date{19.02.2015}


\begin{align}
RQSZC\\
\mathcal{RQSZC}\\
\mathfrak{RQSZC}\\
\mathbb{RQSZC}
\end{align}
\boldmath
\begin{align}
RQSZC\\
\mathcal{RQSZC}\\
\mathfrak{RQSZC}\\
\mathbb{RQSZC}
\end{align}

%\begin{notice}[headline=Dokument mit Gleichung]
Hier eine Gleichung. Wir nennen sie einfach Zustands DGL.

\[\overbrace{ABC}\quad\overbrace{ABC}\quad\underbrace{ABC}\quad
\underbrace{ABC}\quad \underbrace{ABCD}\]

\begin{equation}\label{eq:ZustandsDGL}
  \mathbf{\dot x}(t)=\mathsf{\dot x}(t)=\mathit{\dot x}(t)=\mathtt{\dot x}(t)
  f \left(\mathbf{x}(t), \mathbf{u}(t), \mathbf{w}(t), 
  t\right) \qquad \mathbf{x}(t_0) = \mathbf{x_0}
\end{equation}
%
Es sind hierin $\mathbf{x}(t)$ der Zustandsvektor, $\mathbf{u}(t)$ der
Steuervektor, $\mathbf{w}(t)$ der Störvektor und $f$ die Systemfunktion. 
Anfangszeitpunkt sowie Anfangszustand sind durch $t_0$ bzw. $\mathbf{x_0}$ 
gegeben.

Der Dijkstra-Algorithmus ist optimal. Die Laufzeit der Implementierung mit 
Fibonacci-Heaps beträgt $\mathcal{O}(m+n\log n)\mathcal{L}$.

\blindtext

%\end{notice}
\end{document}
