\RequirePackage[ngerman=ngerman-x-latest]{hyphsubst}
\documentclass[english,ngerman]{tudscrman3}
\usepackage{selinput}\SelectInputMappings{adieresis={ä},germandbls={ß}}
\usepackage[T1]{fontenc}
\usepackage{blindtext}
\lstset{%
  inputencoding=utf8,extendedchars=true,
  literate=%
    {ä}{{\"a}}1 {ö}{{\"o}}1 {ü}{{\"u}}1
    {Ä}{{\"A}}1 {Ö}{{\"O}}1 {Ü}{{\"U}}1
    {~}{{\textasciitilde}}1 {ß}{{\ss}}1
}
% §§§ überarbeiten
%\tracinglabels
\begin{document}

  identisch zu \Option*{cdmath}[false]%
  identisch zu \Option{cdmath}[false]%
  
  

\begin{Declaration}{\Environment{evaluation}[\OLParameter{Überschrift}]}{%
  \Environment{tudpage}'auto'%
}
\begin{Declaration}{%
  \Key{\Environment{evaluation}}{headline}[\PName{Überschrift}]%
}
\begin{Declaration}{\Key{\Environment{evaluation}}{grade}[\PName{Note}]}
\printdeclarationlist%
\index{Gutachten|!(}%
%
Diese Umgebung wird für das Erstellen eines Gutachtens einer 
wissenschaftlichen 
Arbeit bereitgestellt. Auch diese unterstützt alle Parameter, welche für die 
Umgebung \Environment{tudpage}'full' beschrieben wurden.

Für ein Gutachten wird gewöhnlich eine Überschrift aus \Term{evaluationname} 
und~-- falls der Abschlussarbeitstyp angegeben wurde~-- \Term{evaluationtext} 
sowie \Macro{thesis} generiert. Diese automatisch generierte Überschrift kann 
mit dem Parameter \Key{\Environment{evaluation}}{headline} ersetzt werden. Am 
Ende des Gutachtens wird die mit \Key{\Environment{evaluation}}{grade} 
gegebene Note in fetter Schrift ausgezeichnet.

Am Anfang der \Environment{evaluation}"=Umgebung wird die gleiche Tabelle mit 
Autorenangaben ausgegeben, wie dies bei der \Environment{task}"=Umgebung der 
Fall ist. Nach dem Tabellenkopf folgt auch hier der eigentliche Inhalt, sprich 
das Gutachten der Abschlussarbeit. Abgeschlossen wird die Umgebung mit der 
gegebenen Note~-- welche innerhalb von \Term{gradetext} ausgegeben wird~-- 
sowie der Orts- und Datumsangabe (\Macro{place}, \Macro{date}) und der 
darauffolgenden Unterschriftzeile für den oder die Gutachter 
(\Macro{referee}), 
welche wiederum mit den entsprechenden sprachabhängigen Bezeichner 
(\Term{refereename}, \Term{refereeothername}) ergänzt werden.
\end{Declaration}
\end{Declaration}
\end{Declaration}

\begin{Declaration}{\Macro{evaluationform}\LParameter%
  \Parameter{Aufgabe}\Parameter{Inhalt}\Parameter{Bewertung}\Parameter{Note}%
}
\printdeclarationlist%
%
Neben der individuell nutzbaren Umgebung \Environment{evaluation} wird ein 
separater Befehl zur Erstellung eines standardisierten Gutachtens 
bereitgestellt. Dieser strukturiert die Ausgabe in die vier Bereiche 
\emph{Aufgabe}, \emph{Inhalt}, \emph{Bewertung} und \emph{Note} und versieht 
diese jeweils mit der dazugehörigen Überschrift beziehungsweise Textausgabe 
(\Term{taskname}, \Term{contentname}, \Term{assessmentname} und 
\Term{gradetext}). Das optionale Argument unterstützt alle Parameter der 
\Environment{evaluation}"=Umgebung.
\index{Gutachten|!)}%
\end{Declaration}
%
\begin{Example}
Die empfohlene Verwendung des Befehls \Macro{evaluationform} ist wie folgt:
\begin{Code}[escapechar=§]
\evaluationform{%
  Kurzbeschreibung der Aufgabenstellung§\dots§
}{%
  Zusammenfassung von Inhalt und Struktur§\dots§
}{%
  Bewertung der schriftlichen Abschlussarbeit§\dots§
}{%
  Zahl (Note)
}
\end{Code}
Hierzu sei auch auf das Minimalbeispiel in \autoref{sec:exmpl:evaluation} 
verwiesen.
\end{Example}

\begin{Declaration}{\Macro{grade}\Parameter{Note}}
\printdeclarationlist%
%
Neben der Angabe der Note für ein Gutachten über den Parameter 
\Key{\Environment{evaluation}}{grade} kann dafür auch dieser global wirkende 
Befehl verwendet werden.
\end{Declaration}

%\bla

%%\begin{Declaration*}{\Package{tudscrsupervisor}}!
%%\begin{Declaration}{\Environment{evaluation}[\OLParameter{Überschrift}]}{%
%%  \Environment{tudpage}'auto'%
%%}
%%\begin{Declaration}{%
%%  \Key{\Environment{evaluation}}{headline}[\PName{Überschrift}]%
%%}
%%\begin{Declaration}{\Key{\Environment{evaluation}}{grade}[\PName{Note}]}
%%\printdeclarationlist%
%%\index{Gutachten|!(}%
%%%
%%Diese Umgebung wird für das Erstellen eines Gutachtens einer 
%%wissenschaftlichen 
%%Arbeit bereitgestellt. Auch diese unterstützt alle Parameter, welche für die 
%%Umgebung \Environment{tudpage}'full' beschrieben wurden.
%%
%%Für ein Gutachten wird gewöhnlich eine Überschrift aus \Term{evaluationname} 
%%und~-- falls der Abschlussarbeitstyp angegeben wurde~-- \Term{evaluationtext} 
%%sowie \Macro{thesis} generiert. Diese automatisch generierte Überschrift kann 
%%mit dem Parameter \Key{\Environment{evaluation}}{headline} ersetzt werden. Am 
%%Ende des Gutachtens wird die mit \Key{\Environment{evaluation}}{grade} 
%%gegebene Note in fetter Schrift ausgezeichnet.
%%
%%Am Anfang der \Environment{evaluation}"=Umgebung wird die gleiche Tabelle mit 
%%Autorenangaben ausgegeben, wie dies bei der \Environment{task}"=Umgebung der 
%%Fall ist. Nach dem Tabellenkopf folgt auch hier der eigentliche Inhalt, 
%%sprich 
%%das Gutachten der Abschlussarbeit. Abgeschlossen wird die Umgebung mit der 
%%gegebenen Note~-- welche innerhalb von \Term{gradetext} ausgegeben wird~-- 
%%sowie der Orts- und Datumsangabe (\Macro{place}, \Macro{date}) und der 
%%darauffolgenden Unterschriftzeile für den oder die Gutachter 
%%(\Macro{referee}), 
%%welche wiederum mit den entsprechenden sprachabhängigen Bezeichner 
%%(\Term{refereename}, \Term{refereeothername}) ergänzt werden.
%%\end{Declaration}
%%\end{Declaration}
%%\end{Declaration}
%%
%%\begin{Declaration}{\Macro{evaluationform}\LParameter%
%%  \Parameter{Aufgabe}\Parameter{Inhalt}\Parameter{Bewertung}\Parameter{Note}%
%%}
%%\printdeclarationlist%
%%%
%%Neben der individuell nutzbaren Umgebung \Environment{evaluation} wird ein 
%%separater Befehl zur Erstellung eines standardisierten Gutachtens 
%%bereitgestellt. Dieser strukturiert die Ausgabe in die vier Bereiche 
%%\emph{Aufgabe}, \emph{Inhalt}, \emph{Bewertung} und \emph{Note} und versieht 
%%diese jeweils mit der dazugehörigen Überschrift beziehungsweise Textausgabe 
%%(\Term{taskname}, \Term{contentname}, \Term{assessmentname} und 
%%\Term{gradetext}). Das optionale Argument unterstützt alle Parameter der 
%%\Environment{evaluation}"=Umgebung.
%%\index{Gutachten|!)}%
%%\end{Declaration}
%%%
%%\begin{Example}
%%Die empfohlene Verwendung des Befehls \Macro{evaluationform} ist wie folgt:
%%\begin{Code}[escapechar=§]
%%\evaluationform{%
%%  Kurzbeschreibung der Aufgabenstellung§\dots§
%%}{%
%%  Zusammenfassung von Inhalt und Struktur§\dots§
%%}{%
%%  Bewertung der schriftlichen Abschlussarbeit§\dots§
%%}{%
%%  Zahl (Note)
%%}
%%\end{Code}
%%Hierzu sei auch auf das Minimalbeispiel in \autoref{sec:exmpl:evaluation} 
%%verwiesen.
%%\end{Example}
%%
%%\begin{Declaration}{\Macro{grade}\Parameter{Note}}
%%\printdeclarationlist%
%%%
%%Neben der Angabe der Note für ein Gutachten über den Parameter 
%%\Key{\Environment{evaluation}}{grade} kann dafür auch dieser global wirkende 
%%Befehl verwendet werden.
%%\end{Declaration}
%%
%%\begin{Declaration}{\Option{grade}\Parameter{Note}}
%%\begin{Declaration}{\Option{gradetwo}[bbb]\Parameter{Note}}
%%\printdeclarationlist%
%%%
%%Neben der Angabe der Note für ein Gutachten über den Parameter 
%%\Key{\Environment{evaluation}}{grade} kann dafür auch dieser global wirkende 
%%Befehl verwendet werden.
%%\end{Declaration}
%%\end{Declaration}
%%\end{Declaration*}


\index{bla}

\index{\texttt{blubb}}

\addchap{\indexname}
\makeatletter
  \begingroup%
    \@printindex%
    \@printindex[options]%
    \@printindex[macros]%
    \@printindex[keys]%
%    \@printindex[terms]%
%    \@printindex[fonts]%
%    \begingroup%
%      \let\lettergroup\@gobble%
%      \let\indexspace\par%
%      \@printindex[colors]%
%    \endgroup%
%    \@printindex[files]%
  \endgroup%
\makeatother
\PrintChangelog
\end{document}
