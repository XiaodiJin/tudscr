\documentclass[ddcfooter]{tudbeamer}
\usepackage{german}

\begin{document}

\einrichtung{Fakult"at Sprach-, Literatur- und Kulturwissenschaften}
\institut{Institut f"ur Germanistik}

\title[Kurze Pr"asentation]{Eine ziemlich kurze Pr"asentation}
\subtitle{\ldots in der auch wieder Goethe zitiert wird.}
\author{Donald Duck}

\maketitle

\section{Novelle}
\begin{frame}
    \frametitle{Novelle}
    \framesubtitle{Johann Wolfgang von Goethe}
    \begin{itemize}
        \item Ein dichter Herbstnebel verh"ullte noch in der Fr"uhe die weiten R"aume des
              f"urstlichen Schlo"shofes, als man schon mehr oder weniger durch den sich
              lichtenden Schleier die ganze J"agerei zu Pferde und zu Fu"s durcheinander
              bewegt sah. 
        \item Die eiligen Besch"aftigungen der N"achsten lie"sen sich erkennen:
              man verl"angerte, man verk"urzte die Steigb"ugel, man reichte sich B"uchse
              und Patront"aschchen, man schob die Dachsranzen zurecht, indes die Hunde
              ungeduldig am Riemen den Zur"uckhaltenden mit fortzuschleppen drohten.
    \end{itemize}
\end{frame}

\section{Literatur}
\begin{frame}
    \frametitle*{Literatur}
    \begin{itemize}
        \item Goethe: Novelle. DB Sonderband: Meisterwerke deutscher Dichter und Denker,
              S. 12111 (vgl. Goethe-HA Bd. 6, S. 491)
    \end{itemize}
\end{frame}

\end{document}
