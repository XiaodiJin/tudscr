\documentclass[ddc]{tudform}
\usepackage{german}

\einrichtung{Fakult"at Maschinenwesen}
\institut{Institut f"ur Verfahrenstechnik und Umwelttechnik}
\professur{Professur f"ur Thermische Verfahrenstechnik und Umwelttechnik}

\subject{Novelle}

\begin{document}
Ein dichter Herbstnebel verh"ullte noch in der Fr"uhe die weiten R"aume des
f"urstlichen Schlo"shofes, als man schon mehr oder weniger durch den sich
lichtenden Schleier die ganze J"agerei zu Pferde und zu Fu"s durcheinander
bewegt sah. Die eiligen Besch"aftigungen der N"achsten lie"sen sich erkennen:
man verl"angerte, man verk"urzte die Steigb"ugel, man reichte sich B"uchse
und Patront"aschchen, man schob die Dachsranzen zurecht, indes die Hunde
ungeduldig am Riemen den Zur"uckhaltenden mit fortzuschleppen drohten. Auch
hie und da geb"ardete ein Pferd sich mutiger, von feuriger Natur getrieben
oder von dem Sporn des Reiters angeregt, der selbst hier in der Halbhelle
eine gewisse Eitelkeit, sich zu zeigen, nicht verleugnen konnte. Alle jedoch
warteten auf den F"ursten, der, von seiner jungen Gemahlin Abschied nehmend,
allzulange zauderte.
    
Erst vor kurzer Zeit zusammen getraut, empfanden sie schon das Gl"uck
"ubereinstimmender Gem"uter; beide waren von t"atig lebhaftem Charakter, eines
nahm gern an des andern Neigungen und Bestrebungen Anteil. Des F"ursten Vater
hatte noch den Zeitpunkt erlebt und genutzt, wo es deutlich wurde, da"s alle
Staatsglieder in gleicher Betriebsamkeit ihre Tage zubringen, in gleichem
Wirken und Schaffen jeder nach seiner Art erst gewinnen und dann genie"sen 
sollte.

Wie sehr dieses gelungen war, lie"s sich in diesen Tagen gewahr werden, als eben
der Hauptmarkt sich versammelte, den man gar wohl eine Messe nennen konnte. Der
F"urst hatte seine Gemahlin gestern durch das Gewimmel der aufgeh"auften Waren 
zu
Pferde gef"uhrt und sie bemerken lassen, wie gerade hier das Gebirgsland mit dem
flachen Lande einen gl"ucklichen Umtausch treffe; er wu"ste sie an Ort und 
Stelle
auf die Betriebsamkeit seines L"anderkreises aufmerksam zu machen.

Wenn sich nun der F"urst fast ausschlie"slich in diesen Tagen mit den Seinigen 
"uber
diese zudringenden Gegenst"ande unterhielt, auch besonders mit dem 
Finanzminister
anhaltend arbeitete, so behielt doch auch der Landj"agermeister sein Recht, auf
dessen Vorstellung es unm"oglich war, der Versuchung zu widerstehen, an diesen
g"unstigen Herbsttagen eine schon verschobene Jagd zu unternehmen, sich selbst 
und
den vielen angekommenen Fremden ein eignes und seltnes Fest zu er"offnen.

Die F"urstin blieb ungern zur"uck; man hatte sich vorgenommen, weit in das 
Gebirg
hineinzudringen, um die friedlichen Bewohner der dortigen W"alder durch einen
unerwarteten Kriegszug zu beunruhigen.

Scheidend vers"aumte der Gemahl nicht, einen Spazierritt vorzuschlagen, den sie 
im
Geleit Friedrichs, des f"urstlichen Oheims, unternehmen sollte. ">Auch lasse 
ich"<,
sagte er, ">dir unsern Honorio als Stall- und Hofjunker, der f"ur alles sorgen 
wird."<
Und im Gefolg dieser Worte gab er im Hinabsteigen einem wohlgebildeten jungen 
Mann die
n"otigen Auftr"age, verschwand sodann bald mit G"asten und Gefolge.

Die F"urstin, die ihrem Gemahl noch in den Schlo"shof hinab mit dem Schnupftuch
nachgewinkt hatte, begab sich in die hintern Zimmer, welche nach dem Gebirg 
eine freie
Aussicht lie"sen, die um desto sch"oner war, als das Schlo"s selbst von dem 
Flusse herauf
in einiger H"ohe stand und so vor- als hinterw"arts mannigfaltige bedeutende 
Ansichten
gew"ahrte. Sie fand das treffliche Teleskop noch in der Stellung, wo man es 
gestern abend
gelassen hatte, als man, "uber Busch, Berg und Waldgipfel die hohen Ruinen der 
uralten
Stammburg betrachtend, sich unterhielt, die in der Abendbeleuchtung merkw"urdig
hervortraten, indem alsdann die gr"o"sten Licht- und Schattenmassen den 
deutlichsten
Begriff von einem so ansehnlichen Denkmal alter Zeit verleihen konnten. Auch 
zeigte sich
heute fr"uh durch die ann"ahernden Gl"aser recht auffallend die herbstliche 
F"arbung jener
mannigfaltigen Baumarten, die zwischen dem Gem"auer ungehindert und ungest"ort 
durch lange
Jahre emporstrebten. Die sch"one Dame richtete jedoch das Fernrohr etwas tiefer 
nach einer
"oden, steinigen Fl"ache, "uber welche der Jagdzug weggehen mu"ste. [\ldots]
\rightline{\footnotesize aus \itshape Goethe: Novelle. DB Sonderband: 
Meisterwerke
           deutscher Dichter und Denker, S. 12111 (vgl. Goethe-HA Bd. 6, S. 
           491)}
\end{document}
