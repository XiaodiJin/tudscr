\documentclass[%
%  paper=a4,pagesize,%
%  cdgeometry=custom,%
  cdgeometry=no,%
  cdfoot,%
%  cdhead=full,
%  bleedmargin=0mm,%
%  cdfoot=2ex,%
%  BCOR=30pt,
  open=right,
  DIV=8,%usegeometry
%  cdfont=false,
]{tudscrreprt}
\usepackage{selinput}
\SelectInputMappings{adieresis={ä},germandbls={ß}}
\usepackage[T1]{fontenc}
\usepackage[english,ngerman]{babel}
\providecommand*\TUDoptions[1]{}
\usepackage{blindtext,showframe}

\usepackage{geometry}
\makeatletter
\let\showgeometry\Gm@showparams%
\makeatother

%\geometry{%
%%  layout=a5paper,paper=a4paper,
%  layoutoffset=.25in,
%%  marginparwidth=0pt,marginparsep=0pt,
%%  margin=0in,
%%%  footskip=1cm,
%%  nomarginpar,
%%%  marginratio=1:3,centering,
%  showframe,showcrop
%}
%%\recalctypearea
\TUDoptions{cdgeometry=no}
\begin{document}
% TODO: Testen mit tudheadings-Seitenstil
\chapter{Überschrift Eins}
\blindtext
\showgeometry{1}
\cleardoublepage

%\storeareas\PotraitArea% speichert den aktuellen Satzspiegel
%\KOMAoptions{paper=A3,paper=landscape,DIV=current}
%\newgeometry{margin=40pt,layoutoffset=20pt}
%\chapter{Überschrift Zwei}
%\Blindtext
%\cleardoublepage
%\typeout{*geometry* reset}
\TUDoptions{cdgeometry=custom}
\newgeometry{%
%  layout=a5paper,%paper=a4paper,
  layoutoffset=5pt,
%  marginparwidth=0pt,marginparsep=0pt,
  margin=.5cm,
%  footskip=1cm,
%  showframe,showcrop
}
%\pagestyle{tudheadings}
\chapter{Überschrift Zwei}
\blindtext
\showgeometry{2}
\cleardoublepage

\makeatletter
%\def\tud@cdgeometry@num{0}
\TUDoptions{cdgeometry=no}
%\TUDoptions{cdgeometry=no}
%\recalctypearea
%\TUDoptions{cdgeometry=no}
\makeatother
%\restoregeometry[false]
\pagestyle{tudheadings}
\chapter{Überschrift Drei}
\blindtext
\showgeometry{3}
\cleardoublepage

\end{document}


%% TODO Anpassungen an Schlüssel
%% - paper auf KOMAoptions
%% - paperwidth, paperheight, papersize abfangen
%% - landscape und portrait auf paper=... abbilden
%% - layout in anlehnung an paper, gleiche Werte
%% - layoutwidth, layoutheight, layoutsize abfangen
%% - landscape und portrait auf layout=... abbilden
%% - layout/paper=a4paper etc. möglich machen
%% - bleedmargin erkennen
%%\makeatletter
%%\TUD@parameter@family{cdgeometry}{%
%%  \TUD@parameter@define{paper}{\appto\@tempa{#1,}}%
%%  \TUD@parameter@define{paperwidth}{}%
%%  \TUD@parameter@define{paperheight}{}%
%%  \TUD@parameter@define{papersize}{}%
%%  \TUD@parameter@define{landscape}{%
%%    %if else fi
%%  }%
%%  \TUD@parameter@define{portrait}{%
%%    %if else fi
%%  }%
%%  \TUD@parameter@define{layout}{\appto\@tempb{#1,}}%
%%  \TUD@parameter@define{layoutwidth}{}%
%%  \TUD@parameter@define{layoutheight}{}%
%%  \TUD@parameter@define{layoutsize}{}%
%%%  \TUD@parameter@handler{%
%%%    \if\relax\detokenize{#2}\relax%
%%%      \appto\@tempc{#1,}%
%%%    \else%
%%%      \appto\@tempc{#1=#2,}%
%%%    \fi%
%%%  }%
%%}
%%\newcommand*\tudgeometry[1]{%
%%  \def\@tempa{}
%%  \def\@tempb{}
%%  \def\@tempc{}
%%  \TUD@parameter@set{cdgeometry}{#1}%
%%  \typeout{+++++ \meaning\@tempa}%
%%  \typeout{+++++ \meaning\@tempb}%
%%  \typeout{+++++ \meaning\@tempc}%
%%}
%%\makeatother
%%\tudgeometry{paper=b4,paper=seascape,layout=a4,margin=15mm,showframe,bottom=25mm}