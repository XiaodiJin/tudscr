\documentclass[%
  paper=a4,pagesize,%
  cdgeometry=false,%
  cdfoot,%
  cdhead=full,
%  bleedmargin=0mm,%
%  cdfoot=2ex,%
  open=right,
%  cdfont=false,
]{tudscrreprt}
\usepackage{selinput}
\SelectInputMappings{adieresis={ä},germandbls={ß}}
\usepackage[T1]{fontenc}
\usepackage[english,ngerman]{babel}

\usepackage{blindtext}

\usepackage{geometry}
\geometry{margin=10mm,showframe}

% TODO Anpassungen an Schlüssel
% - paper auf KOMAoptions
% - paperwidth, paperheight, papersize abfangen
% - landscape und portrait auf paper=... abbilden
% - layout in anlehnung an paper, gleiche Werte
% - layoutwidth, layoutheight, layoutsize abfangen
% - landscape und portrait auf layout=... abbilden
% - layout/paper=a4paper etc. möglich machen
% - bleedmargin erkennen
%\makeatletter
%\TUD@parameter@family{cdgeometry}{%
%  \TUD@parameter@define{paper}{\appto\@tempa{#1,}}%
%  \TUD@parameter@define{paperwidth}{}%
%  \TUD@parameter@define{paperheight}{}%
%  \TUD@parameter@define{papersize}{}%
%  \TUD@parameter@define{landscape}{%
%    %if else fi
%  }%
%  \TUD@parameter@define{portrait}{%
%    %if else fi
%  }%
%  \TUD@parameter@define{layout}{\appto\@tempb{#1,}}%
%  \TUD@parameter@define{layoutwidth}{}%
%  \TUD@parameter@define{layoutheight}{}%
%  \TUD@parameter@define{layoutsize}{}%
%%  \TUD@parameter@handler{%
%%    \if\relax\detokenize{#2}\relax%
%%      \appto\@tempc{#1,}%
%%    \else%
%%      \appto\@tempc{#1=#2,}%
%%    \fi%
%%  }%
%}
%\newcommand*\tudgeometry[1]{%
%  \def\@tempa{}
%  \def\@tempb{}
%  \def\@tempc{}
%  \TUD@parameter@set{cdgeometry}{#1}%
%  \typeout{+++++ \meaning\@tempa}%
%  \typeout{+++++ \meaning\@tempb}%
%  \typeout{+++++ \meaning\@tempc}%
%}
%\makeatother
%\tudgeometry{paper=b4,paper=seascape,layout=a4,margin=15mm,showframe,bottom=25mm}

\begin{document}
% TODO: Testen mit tudheadings-Seitenstil
%\pagestyle{tudheadings} 

\chapter{Überschrift Eins}
\Blindtext

\typeout{+++++!}
% TODO FamilyOptions innerhalb von Optionen vs AtEndOfFamilyOptions
\TUDoptions{%
  cdfont=true,cdfont=cdhead,%
%%  vspacing=false,%
%%  cdmath=false,
%%  cd=lite,%
%%  cdgeometry=false,
%%  extrabottommargin=33pt,
}

%\KOMAoptions{headings=small}
%\recalctypearea
\Blindtext

\cleardoublepage

%\storeareas\PotraitArea% speichert den aktuellen Satzspiegel
%\KOMAoptions{paper=A3,paper=landscape,DIV=current}

\newgeometry{margin=45mm}
\chapter{Überschrift Zwei}
\Blindtext


\cleardoublepage

\restoregeometry

%\PotraitArea% lädt den gespeicherten Satzspiegel
\chapter{Überschrift Drei}

\texttt{\meaning\PotraitArea}

\Blindtext
\end{document}

\documentclass[
  ngerman,
  cdhead=bicolor,
]{scrreprt}
\usepackage{selinput}\SelectInputMappings{adieresis={ä},germandbls={ß}}
\usepackage[T1]{fontenc}
\usepackage{babel}
\usepackage{blindtext}
\begin{document}

%\pagestyle{tudheadings}

\Blindtext

\clearpage
\KOMAoptions{paper=a3}
\recalctypearea

\Blindtext

\end{document}

\documentclass[english,ngerman,class=tudscrreprt]{standalone}
\usepackage[T1]{fontenc}
\usepackage[utf8]{inputenc}
\usepackage{tikz}
\usetikzlibrary{chains}
\usetikzlibrary{decorations.markings}
\tikzset{on grid}
\newlength{\tikzunit}
\setlength{\tikzunit}{.01\textwidth}
\tikzset{x=\tikzunit,y=\tikzunit}
\begin{document}
\begin{tikzpicture}
  \tikzstyle{inner box}=[%
    text width=17\tikzunit,
    align=center,
    rectangle,
    inner sep=.5\tikzunit,
    minimum height=8\tikzunit,
    font=\hspace{0pt},
    draw
  ]
  \tikzstyle{inner label}=[align=center, font=\scriptsize]
  \tikzstyle{inner box chain}=[every node/.style={on chain}]
  \tikzstyle{inner box chain below}=[%
    inner box chain, node distance=8\tikzunit,continue chain=going below
  ]
  \tikzstyle{inner box chain right}=[%
    inner box chain,node distance=35\tikzunit,continue chain=going right
  ]
  \tikzstyle{inner box chain above}=[%
    inner box chain,node distance=16\tikzunit,continue chain=going above
  ]
  \tikzstyle{pstarrow->}=[%
    decoration={markings,
      mark=at position 1 with {\arrow[xscale=1.5]{stealth}};
    },
    postaction={decorate},
    shorten >=0.7pt
  ]
  \newcommand{\tikzparbox}[2][9]{%
    \parbox{#1\tikzunit}{\centering\hspace{0pt}#2}%
  }
  \begin{scope}[start chain]
    \begin{scope}[inner box chain below]
      \node(NE)[inner box]{Navigations\-ebene};
      \node(NB)[inner label]{gewählte Fahrtroute\\ zeitlicher Ablauf};
      \node(BE)[inner box]{{Bahnführungs\-ebene}};
      \node(BS)[inner label]{%
        gewählte Führungsgrößen:\\ Sollspur, Sollgeschwindigkeit%
      };
      \node(SE)[inner box]{Stabilisierungs\-ebene};
    \end{scope}
    \begin{scope}[inner box chain right]
      \node(LQ)[inner box]{Längs- und Querdynamik};
      \node(FO)[inner box]{Fahrbahn\-oberfläche};
    \end{scope}
    \begin{scope}[inner box chain above]
      \node(FR)[inner box]{Fahrraum\\ \smallskip{\scriptsize Straße 
      und\\ \vspace{-1.5ex}Verkehrssituation}};
      \node(SN)[inner box]{Straßennetz};
    \end{scope}
  \end{scope}
  \begin{scope}[inner label,minimum size=0pt]
    \draw [pstarrow->] (FO) -| ++(13.5,-12) to
      node [above]{Istspur, Istgeschwindigkeit} ++(-97,0) |- (SE);
    \draw [pstarrow->] (FR) -| ++(14  ,-32) to 
      node [above]{Bereich sicherer Führungsgrößen} ++(-98,0) |- (BE);
    \draw [pstarrow->] (SN) -| ++(14.5  ,-52) to 
      node [above]{mögliche Fahrtroute} ++(-99,0) |- (NE);
  \end{scope}
  \begin{scope}[inner label]
    \draw              (NE) to (NB);
    \draw [pstarrow->] (NB) to (BE);
    \draw              (BE) to (BS);
    \draw [pstarrow->] (BS) to (SE);
    \draw [pstarrow->] (SE) to
      node[above] {\tikzparbox{Stell\-größen}}
      node[below] {\tikzparbox{Lenken Gasgeben Bremsen}}
    (LQ);
    \draw [pstarrow->] (LQ) to
      node[above]{\tikzparbox{Regel\-größen}}
      node[below]{\tikzparbox{Fahrzeugbewegung}}
    (FO);
    \draw [pstarrow->] (LQ)+(24,0) |- (FR);
    \draw [pstarrow->] (LQ)+(24,0) |- (SN);
  \end{scope}
  \begin{scope}[very thick,rounded corners=5\tikzunit]
    \draw (-12.5,-40) rectangle (12.5,14);
    \draw ( 22.5,-40) rectangle (47.5,-18);
    \draw ( 57.5,-40) rectangle (82.5,14);
  \end{scope}
  \begin{scope}[font=\bfseries]
    \node at (0,9) {Fahrer};
    \node at (35,-23) {Fahrzeug};
    \node at (70,9) {Umwelt};
  \end{scope}
\end{tikzpicture}
\end{document}