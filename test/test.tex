% Option zum automatischen Setzen der Boxen
%https://groups.google.com/forum/#!topic/de.comp.text.tex/1gouJ0Zov6o
%http://tex.stackexchange.com/questions/248902/
%https://wiki.scribus.net/canvas/PDF_Boxes_:_mediabox,_cropbox,_bleedbox,_trimbox,_artbox
%http://www.prepressure.com/pdf/basics/page-boxes
\documentclass[
  cdgeometry=custom,
  paper=a4,
  DIV=12,
  ngerman,
%  oneside,
%  tudscrver=2.02,
  cd=fullcolor,
]{tudscrreprt}
%\usepackage{geometry}
\usepackage{babel}
\usepackage{blindtext}
\usepackage{multicol}
%\usepackage{graphicx}
%\usepackage[a3,cam,center]{crop}
%\recalctypearea
%\KOMAoption{DIV=10}
%\recalctypearea
%\edef\Gm@restore{}%
%\savegeometry{typearea}

%\geometry{paper=b3paper,layout=a4paper,layoutoffset=3cm,showcrop}

\usepackage[
a3,cross,center,off
]{crop}
\usepackage{hyperref}

%\geometry{pass}
%\geometry{textwidth=12cm}

%\recalctypearea
%\setlength{\headingsvskip}{-4cm}
\begin{document}
\blindtext

\pagestyle{tudheadings}
\blindtext

\blinddocument
\end{document}
%
%
%%\PassOptionsToPackage{driver=none}{geometry}
%%\RequirePackage{scrbase}
%\documentclass[ddcfoot,%
%  cd=fullcolor,
%  cdhead=fullcolor,
%  cdfoot=fullcolor,
%%  tudscrver=2.02,
%%  cdgeometry=false,
%%  paper=b4,DIV=14,
%%  fontsize=12pt,
%%  a3paper,
%%  cdgeometry=false,
%]{tudscrreprt}
%\usepackage[T1]{fontenc}
%\usepackage[ngerman]{babel}
%\usepackage[automark]{scrlayer-scrpage}
%\usepackage[math]{blindtext}
%
%%\usepackage{showframe}
%
%
%% TODO Zusammenspiel von geometry mit layout-Einstellungen und crop 
%%implementieren.
%
%\usepackage{geometry}
%\geometry{%
%%  showframe,%
%  showcrop,%
%  paper=c4paper,%
%%%  left=3cm,right=2cm,top=2cm,bottom=3cm,marginparwidth=1cm,%
%  layout=a4paper,%
%  layouthoffset=1cm,
%  layoutvoffset=1cm,%
%}
%
%
%%\usepackage[b3,center]{crop}
%%\crop[cam]
%
%\makeatletter
%\providecommand*\ps@tudheadings{\ps@headings}
%\makeatother
%\pagestyle{tudheadings}
%\begin{document}
%\faculty{text}
%\department{text}
%\chair{text}
%\institute{text}
%
%
%\blinddocument
%
%\pagestyle{scrheadings}
%\blinddocument
%\end{document}
%
%
%
%\RequirePackage[ngerman=ngerman-x-latest]{hyphsubst}
%\RequirePackage{fix-cm}
%\documentclass[english,ngerman,
%%ddcfoot,automark
%%paper=a5,
%cdfoot=5pt,
%cd=litecolor,
%cdfoot=bicolor,
%cdsection=on,
%abstract=sec,
%titlepage=no,
%%cdtitle=no,
%%cdhead=litecolor,
%%fontsize=4pt%
%%twocolumn,
%]{tudscrposter}
%\usepackage{selinput}\SelectInputMappings{adieresis={ä},germandbls={ß}}
%\usepackage[T1]{fontenc}
%\usepackage{babel}
%\usepackage{blindtext}
%\usepackage{showframe}
%\usepackage{multicol}
%
%
%\begin{document}
%\author{names}
%%\footlogo{TUD-white,TUD-black,}
%\title{text}
%%\subject{text}
%%\titlehead{head}
%%
%
%%\makecover
%
%
%%\author{<Autor(en)>}
%%\authormore{<Autorenzusatz>}
%\professor{<Name>}
%\supervisor{<Name(n)>\and blunn}
%\webpage{aaa}
%
%%
%%\meaning\titlepagestyle
%%\maketitle%onecolumn{aaaa}
%\maketitle[cd=no]
%
%\blindtext
%
%%\begin{multicols}{2}[%
%%%{\maketitle[cd=no]}
%%\begin{abstract}
%%\blindtext
%%\end{abstract}
%%]
%%\blinddocument
%%\end{multicols}
%
%
%\end{document}

\documentclass[english,ngerman,class=tudscrreprt]{standalone}
\usepackage[T1]{fontenc}
\usepackage[utf8]{inputenc}
\usepackage{tikz}
\usetikzlibrary{chains}
\usetikzlibrary{decorations.markings}
\tikzset{on grid}
\newlength{\tikzunit}
\setlength{\tikzunit}{.01\textwidth}
\tikzset{x=\tikzunit,y=\tikzunit}
\begin{document}
\begin{tikzpicture}
  \tikzstyle{inner box}=[%
    text width=17\tikzunit,
    align=center,
    rectangle,
    inner sep=.5\tikzunit,
    minimum height=8\tikzunit,
    font=\hspace{0pt},
    draw
  ]
  \tikzstyle{inner label}=[align=center, font=\scriptsize]
  \tikzstyle{inner box chain}=[every node/.style={on chain}]
  \tikzstyle{inner box chain below}=[%
    inner box chain, node distance=8\tikzunit,continue chain=going below
  ]
  \tikzstyle{inner box chain right}=[%
    inner box chain,node distance=35\tikzunit,continue chain=going right
  ]
  \tikzstyle{inner box chain above}=[%
    inner box chain,node distance=16\tikzunit,continue chain=going above
  ]
  \tikzstyle{pstarrow->}=[%
    decoration={markings,
      mark=at position 1 with {\arrow[xscale=1.5]{stealth}};
    },
    postaction={decorate},
    shorten >=0.7pt
  ]
  \newcommand{\tikzparbox}[2][9]{%
    \parbox{#1\tikzunit}{\centering\hspace{0pt}#2}%
  }
  \begin{scope}[start chain]
    \begin{scope}[inner box chain below]
      \node(NE)[inner box]{Navigations\-ebene};
      \node(NB)[inner label]{gewählte Fahrtroute\\ zeitlicher Ablauf};
      \node(BE)[inner box]{{Bahnführungs\-ebene}};
      \node(BS)[inner label]{%
        gewählte Führungsgrößen:\\ Sollspur, Sollgeschwindigkeit%
      };
      \node(SE)[inner box]{Stabilisierungs\-ebene};
    \end{scope}
    \begin{scope}[inner box chain right]
      \node(LQ)[inner box]{Längs- und Querdynamik};
      \node(FO)[inner box]{Fahrbahn\-oberfläche};
    \end{scope}
    \begin{scope}[inner box chain above]
      \node(FR)[inner box]{Fahrraum\\ \smallskip{\scriptsize Straße 
      und\\ \vspace{-1.5ex}Verkehrssituation}};
      \node(SN)[inner box]{Straßennetz};
    \end{scope}
  \end{scope}
  \begin{scope}[inner label,minimum size=0pt]
    \draw [pstarrow->] (FO) -| ++(13.5,-12) to
      node [above]{Istspur, Istgeschwindigkeit} ++(-97,0) |- (SE);
    \draw [pstarrow->] (FR) -| ++(14  ,-32) to 
      node [above]{Bereich sicherer Führungsgrößen} ++(-98,0) |- (BE);
    \draw [pstarrow->] (SN) -| ++(14.5  ,-52) to 
      node [above]{mögliche Fahrtroute} ++(-99,0) |- (NE);
  \end{scope}
  \begin{scope}[inner label]
    \draw              (NE) to (NB);
    \draw [pstarrow->] (NB) to (BE);
    \draw              (BE) to (BS);
    \draw [pstarrow->] (BS) to (SE);
    \draw [pstarrow->] (SE) to
      node[above] {\tikzparbox{Stell\-größen}}
      node[below] {\tikzparbox{Lenken Gasgeben Bremsen}}
    (LQ);
    \draw [pstarrow->] (LQ) to
      node[above]{\tikzparbox{Regel\-größen}}
      node[below]{\tikzparbox{Fahrzeugbewegung}}
    (FO);
    \draw [pstarrow->] (LQ)+(24,0) |- (FR);
    \draw [pstarrow->] (LQ)+(24,0) |- (SN);
  \end{scope}
  \begin{scope}[very thick,rounded corners=5\tikzunit]
    \draw (-12.5,-40) rectangle (12.5,14);
    \draw ( 22.5,-40) rectangle (47.5,-18);
    \draw ( 57.5,-40) rectangle (82.5,14);
  \end{scope}
  \begin{scope}[font=\bfseries]
    \node at (0,9) {Fahrer};
    \node at (35,-23) {Fahrzeug};
    \node at (70,9) {Umwelt};
  \end{scope}
\end{tikzpicture}
\makeatletter
%\the\tud@dim@layoutheight
\end{document}