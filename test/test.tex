\documentclass[
  ngerman,
]{tudscrreprt}
\usepackage{selinput}\SelectInputMappings{adieresis={ä},germandbls={ß}}
\usepackage[T1]{fontenc}
\usepackage{babel}
\usepackage{blindtext}
\begin{document}

\begin{description}
  \item[interword space] \the\fontdimen2\font
  \item[interword stretch] \the\fontdimen3\font
  \item[interword shrink] \the\fontdimen4\font
  \item[xheight] \the\fontdimen5\font
  \item[quad] \the\fontdimen6\font
  \item[extra space] \the\fontdimen7\font
  \item[spaceskip] \the\spaceskip
  \item[xspaceskip] \the\xspaceskip
  \item[hyphenchar] \the\hyphenchar\font
\end{description}

\end{document}

\documentclass[english,ngerman,class=tudscrreprt]{standalone}
\usepackage[T1]{fontenc}
\usepackage[utf8]{inputenc}
\usepackage{tikz}
\usetikzlibrary{chains}
\usetikzlibrary{decorations.markings}
\tikzset{on grid}
\newlength{\tikzunit}
\setlength{\tikzunit}{.01\textwidth}
\tikzset{x=\tikzunit,y=\tikzunit}
\begin{document}
\begin{tikzpicture}
  \tikzstyle{inner box}=[%
    text width=17\tikzunit,
    align=center,
    rectangle,
    inner sep=.5\tikzunit,
    minimum height=8\tikzunit,
    font=\hspace{0pt},
    draw
  ]
  \tikzstyle{inner label}=[align=center, font=\scriptsize]
  \tikzstyle{inner box chain}=[every node/.style={on chain}]
  \tikzstyle{inner box chain below}=[%
    inner box chain, node distance=8\tikzunit,continue chain=going below
  ]
  \tikzstyle{inner box chain right}=[%
    inner box chain,node distance=35\tikzunit,continue chain=going right
  ]
  \tikzstyle{inner box chain above}=[%
    inner box chain,node distance=16\tikzunit,continue chain=going above
  ]
  \tikzstyle{pstarrow->}=[%
    decoration={markings,
      mark=at position 1 with {\arrow[xscale=1.5]{stealth}};
    },
    postaction={decorate},
    shorten >=0.7pt
  ]
  \newcommand{\tikzparbox}[2][9]{%
    \parbox{#1\tikzunit}{\centering\hspace{0pt}#2}%
  }
  \begin{scope}[start chain]
    \begin{scope}[inner box chain below]
      \node(NE)[inner box]{Navigations\-ebene};
      \node(NB)[inner label]{gewählte Fahrtroute\\ zeitlicher Ablauf};
      \node(BE)[inner box]{{Bahnführungs\-ebene}};
      \node(BS)[inner label]{%
        gewählte Führungsgrößen:\\ Sollspur, Sollgeschwindigkeit%
      };
      \node(SE)[inner box]{Stabilisierungs\-ebene};
    \end{scope}
    \begin{scope}[inner box chain right]
      \node(LQ)[inner box]{Längs- und Querdynamik};
      \node(FO)[inner box]{Fahrbahn\-oberfläche};
    \end{scope}
    \begin{scope}[inner box chain above]
      \node(FR)[inner box]{Fahrraum\\ \smallskip{\scriptsize Straße 
      und\\ \vspace{-1.5ex}Verkehrssituation}};
      \node(SN)[inner box]{Straßennetz};
    \end{scope}
  \end{scope}
  \begin{scope}[inner label,minimum size=0pt]
    \draw [pstarrow->] (FO) -| ++(13.5,-12) to
      node [above]{Istspur, Istgeschwindigkeit} ++(-97,0) |- (SE);
    \draw [pstarrow->] (FR) -| ++(14  ,-32) to 
      node [above]{Bereich sicherer Führungsgrößen} ++(-98,0) |- (BE);
    \draw [pstarrow->] (SN) -| ++(14.5  ,-52) to 
      node [above]{mögliche Fahrtroute} ++(-99,0) |- (NE);
  \end{scope}
  \begin{scope}[inner label]
    \draw              (NE) to (NB);
    \draw [pstarrow->] (NB) to (BE);
    \draw              (BE) to (BS);
    \draw [pstarrow->] (BS) to (SE);
    \draw [pstarrow->] (SE) to
      node[above] {\tikzparbox{Stell\-größen}}
      node[below] {\tikzparbox{Lenken Gasgeben Bremsen}}
    (LQ);
    \draw [pstarrow->] (LQ) to
      node[above]{\tikzparbox{Regel\-größen}}
      node[below]{\tikzparbox{Fahrzeugbewegung}}
    (FO);
    \draw [pstarrow->] (LQ)+(24,0) |- (FR);
    \draw [pstarrow->] (LQ)+(24,0) |- (SN);
  \end{scope}
  \begin{scope}[very thick,rounded corners=5\tikzunit]
    \draw (-12.5,-40) rectangle (12.5,14);
    \draw ( 22.5,-40) rectangle (47.5,-18);
    \draw ( 57.5,-40) rectangle (82.5,14);
  \end{scope}
  \begin{scope}[font=\bfseries]
    \node at (0,9) {Fahrer};
    \node at (35,-23) {Fahrzeug};
    \node at (70,9) {Umwelt};
  \end{scope}
\end{tikzpicture}
\end{document}