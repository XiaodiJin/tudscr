% \CheckSum{335}
% \iffalse meta-comment
% ======================================================================
%
% Das Corporate Design der TU Dresden auf Basis der KOMA-Script-Klassen
%
% ======================================================================
% This work may be distributed and/or modified under the conditions of
% the LaTeX Project Public License, version 1.3c of the license.
% The latest version of this license is in
%     http://www.latex-project.org/lppl.txt
% and version 1.3c or later is part of all distributions of LaTeX
% version 2005/12/01 or later and of this work.
% This work has the LPPL maintenance status "author-maintained".
% The current maintainer and author of this work is Falk Hanisch.
% ----------------------------------------------------------------------
% Dieses Werk darf nach den Bedingungen der LaTeX Project Public Lizenz,
% Version 1.3c, verteilt und/oder veraendert werden.
% Die neuste Version dieser Lizenz ist
%     http://www.latex-project.org/lppl.txt
% und Version 1.3c ist Teil aller Verteilungen von LaTeX
% Version 2005/12/01 oder spaeter und dieses Werks.
% Dieses Werk hat den LPPL-Verwaltungs-Status "author-maintained"
% (allein durch den Autor verwaltet).
% Der aktuelle Verwalter und Autor dieses Werkes ist Falk Hanisch.
% ======================================================================
% \fi
%
% \CharacterTable
%  {Upper-case    \A\B\C\D\E\F\G\H\I\J\K\L\M\N\O\P\Q\R\S\T\U\V\W\X\Y\Z
%   Lower-case    \a\b\c\d\e\f\g\h\i\j\k\l\m\n\o\p\q\r\s\t\u\v\w\x\y\z
%   Digits        \0\1\2\3\4\5\6\7\8\9
%   Exclamation   \!     Double quote  \"     Hash (number) \#
%   Dollar        \$     Percent       \%     Ampersand     \&
%   Acute accent  \'     Left paren    \(     Right paren   \)
%   Asterisk      \*     Plus          \+     Comma         \,
%   Minus         \-     Point         \.     Solidus       \/
%   Colon         \:     Semicolon     \;     Less than     \<
%   Equals        \=     Greater than  \>     Question mark \?
%   Commercial at \@     Left bracket  \[     Backslash     \\
%   Right bracket \]     Circumflex    \^     Underscore    \_
%   Grave accent  \`     Left brace    \{     Vertical bar  \|
%   Right brace   \}     Tilde         \~}
%
% \iffalse
%%% From File: tudscr-comp.dtx
%<*driver>
\ifx\ProvidesFile\undefined\def\ProvidesFile#1[#2]{}\fi
\ProvidesFile{tudscr-comp.dtx}[%
  2014/10/22 v2.02 TUD-KOMA-Script\space%
%</driver>
%<package>\NeedsTeXFormat{LaTeX2e}[2011/06/27]
%<package>\ProvidesPackage{tudscrcomp}[%
%<*driver|package>
%!TUDVersion
%<package>  package
  (compatibility for tudbook)%
]
%</driver|package>
%<*driver>
\RequirePackage[ngerman=ngerman-x-latest]{hyphsubst}
\documentclass[english,ngerman]{tudscrdoc}
\usepackage{selinput}\SelectInputMappings{adieresis={ä},germandbls={ß}}
\usepackage[T1]{fontenc}
\usepackage{babel}
\usepackage{tudscrfonts} % only load this package, if the fonts are installed
\KOMAoptions{parskip=half-}
\CodelineIndex
\RecordChanges
\GetFileInfo{tudscr-comp.dtx}
\begin{document}
  \maketitle
  \DocInput{\filename}
\end{document}
%</driver>
% \fi
%
% \selectlanguage{ngerman}
%
% \section{Kompatibilität für die \cls{tudbook}-Klasse und frühere Versionen}
%
% Diese Paket stellt für die \cls{tudscr}-Klassen eine Schnittstelle bereit,
% die es ermöglicht, die in der alten \cls{tudbook}-Klasse und dem dazugehörigen
% \pkg{tudthesis}-Paket definierten Befehle hier zu benutzen, um alte Dokumente
% mit den neuen Klassen zu setzen.
%
% \StopEventually{\PrintIndex\PrintChanges}
%
% \iffalse
%<*package>
% \fi
%
% \subsection{Das Paket \pkg{tudscrcomp} -- \cls{tudbook}-Kompatibilität}
%
% \begin{macro}{\einrichtung}
% \begin{macro}{\fachrichtung}
% \begin{macro}{\institut}
% \begin{macro}{\professur}
% \begin{macro}{\moreauthor}
% \begin{macro}{\submitdate}
% \begin{macro}{\supervisorII}
% \begin{macro}{\supervisedby}
% \begin{macro}{\supervisedIIby}
% \begin{macro}{\submittedon}
% Es werden Aliasbefehle für die Eingabefelder definiert.
%    \begin{macrocode}
\newcommand*\einrichtung{}
\let\einrichtung\faculty
\newcommand*\fachrichtung{}
\let\fachrichtung\department
\newcommand*\institut{}
\let\institut\institute
\newcommand*\professur{}
\let\professur\chair
\newcommand*\moreauthor{}
\let\moreauthor\authormore
\newcommand*\submitdate{}
\let\submitdate\date
\newcommand*\supervisorII[1]{%
  \expandafter\gdef\expandafter\@supervisor\expandafter{\@supervisor\and #1}%
}
\newcommand*\supervisedby[1]{\gdef\supervisorname{#1}}
\newcommand*\supervisedIIby[1]{\gdef\supervisorothername{#1}}
\newcommand*\submittedon[1]{\gdef\datetext{#1}}
%    \end{macrocode}
% \end{macro}^^A \submittedon
% \end{macro}^^A \supervisedIIby
% \end{macro}^^A \supervisedby
% \end{macro}^^A \supervisorII
% \end{macro}^^A \submitdate
% \end{macro}^^A \moreauthor
% \end{macro}^^A \professur
% \end{macro}^^A \institut
% \end{macro}^^A \fachrichtung
% \end{macro}^^A \einrichtung}
% \begin{macro}{\dissertation}
% Bei der Definition des Typs der Abschlussarbeit mit \cs{dissertation} wird
% die Lokalisierungsvariable \cs{dissertationname} verwendet und die Feldnamen
% angepasst.
%    \begin{macrocode}
\newcommand*\dissertation{%
  \thesis{\dissertationname}%
  \let\supervisorname\refereename%
  \let\supervisorothername\refereeothername%
}
%    \end{macrocode}
% \end{macro}^^A \dissertation
% \begin{option}{colortitle}
% \begin{option}{nocolortitle}
% Für farbige Einstellungen wird von \cls{tudbook} die Option \opt{color} 
% definiert. Soll die Titelseite konträr dazu gesetzt werden, muss sich mit den
% Schlüsseln \opt{colortitle} und \opt{nocolortitle} beholfen werden.
%    \begin{macrocode}
\TUD@key{colortitle}[true]{%
  \TUD@set@ifkey{colortitle}{@tempswa}{#1}%
  \if@tempswa%
    \TUDoptions{cdtitle=color}%
  \else%
    \TUDoptions{cdtitle=true}%
  \fi%
}
\TUD@key{nocolortitle}[true]{%
  \TUD@set@ifkey{nocolortitle}{@tempswa}{#1}%
  \if@tempswa%
    \TUDoptions{cdtitle=true}%
  \else%
    \TUDoptions{cdtitle=color}%
  \fi%
}
%    \end{macrocode}
% \end{option}^^A nocolortitle
% \end{option}^^A colortitle
% \begin{option}{ddcfooter}
% Außer der Option \opt{ddc} gibt es bei der alten \cls{tudbook}-Klasse noch
% den Schlüssel \opt{ddcfooter}. Dieser wird auf die Option \opt{ddcfoot} 
% gelegt.
%    \begin{macrocode}
\TUD@key{ddcfooter}[true]{%
  \TUD@set@ifkey{ddcfooter}{@tempswa}{#1}%
  \if@tempswa%
    \TUDoptions{ddcfoot=true}%
  \else%
    \TUDoptions{ddcfoot=false}%
  \fi%
}
%    \end{macrocode}
% \end{option}^^A ddcfooter
% \begin{environment}{theglossary}
% \begin{macro}{\glossaryname}
% \begin{macro}{\glossitem}
% Eine rudimentäre Umgebung für ein Glossar.
%    \begin{macrocode}
\AtBeginDocument{%
  \ifdef{\theglossary}{}{%
    \providecommand*{\glossaryname}{Glossar}
    \newenvironment{theglossary}[1][]{%
      \ClassWarning{\tudcls@name}{%
        Using the environment theglossary is not\MessageBreak%
        recommended. You should rather use an appropriate\MessageBreak%
        package such as glossaries%
      }%
      \let\bibname\glossaryname%
      \bib@heading%
      #1%
      \list{}{%
        \setlength{\labelsep}{\z@}%
        \setlength{\labelwidth}{\z@}%
        \setlength{\itemindent}{-\leftmargin}%
      }%
    }{\endlist}
    \newcommand\glossitem[1]{\item[] #1\par}%
  }%
}
%    \end{macrocode}
% \end{macro}^^A \glossitem
% \end{macro}^^A \glossaryname
% \end{environment}^^A theglossary
% \begin{option}{serifmath}
% Die alte \cls{tudbook}-Klasse hat neben der Option \opt{sansmath}
% außerdem den zusätzlichen Schlüssel \opt{serifmath} definiert, welcher aus
% Gründen der Kompatibilität hier ebenfalls vorgehalten wird.
%    \begin{macrocode}
\TUD@key{serifmath}[true]{%
  \TUD@set@ifkey{serifmath}{@tempswa}{#1}%
  \if@tempswa%
    \TUDoptions{sansmath=false}%
  \else%
    \TUDoptions{sansmath=true}%
  \fi%
}
%    \end{macrocode}
% \end{option}^^A serifmath
% \begin{macro}{\chapterpage}
% \begin{macro}{\if@tud@chapterpage@temp}
% \begin{macro}{\tud@chapterpage@set}
% \begin{macro}{\tud@chapterpage@unset}
% \begin{macro}{\tud@chapterpage@wrn}
% Die alte \cls{tudbook}-Klasse stellt den Befehl \cs{chapterpage} bereit.
% Mit diesem ist es möglich, das Verhalten der Kapitelseiten~-- welches durch
% die Option \opt{chapterpage} gesteuert wird~-- temporär umzuschalten, also
% statt Kapitelseiten lediglich Überschriften zu setzen und umgekehrt. Dies ist
% typographisch durchaus zu hinterfragen, allerdings sollen die neuen Klassen
% möglichst kompatibel zu der alten sein, weshalb diese Funktionalität trotzdem
% implementiert wird. Der Befehl \cs{chapterpage} setzt den globalen Schalter
% \cs{if@tud@chapterpage@temp}. Der Befehl \cs{tud@chapterpage@set} setzt für 
% Kapitel das komplementäre Verhalten zur eigentlich gewählten
% \opt{chapterpage}"=Option. Nach dem Setzen der entsprechenden Überschrift
% wird \cs{tud@chapterpage@set} nochmals aufgerufen, das Verhalten auf den
% ursprünglichen Zustand geschaltet und der globale Schalter
% \cs{if@tud@chapterpage@temp} zurückgesetzt.
%    \begin{macrocode}
\newif\if@tud@chapterpage@temp
\newcommand*\chapterpage{%
  \global\@tud@chapterpage@temptrue%
  \tud@chapterpage@wrn%
}
\newcommand*\tud@chapterpage@set[1][]{%
  \if@tud@chapterpage@temp%
    \if@tud@chapterpage%
      \TUDoptions{chapterpage=false}%
    \else%
      \TUDoptions{chapterpage=true}%
    \fi%
  \fi%
}
\newcommand*\tud@chapterpage@unset[1][]{%
  \tud@chapterpage@set%
  \global\@tud@chapterpage@tempfalse%
}
\AtEndPreamble{%
  \pretocmd{\tud@chapter}{\tud@chapterpage@set}%
    {}{\tud@patch@wrn{tud@nchapter}}%
  \apptocmd{\tud@chapter}{\tud@chapterpage@unset}%
    {}{\tud@patch@wrn{tud@nchapter}}%
  \pretocmd{\tud@schapter}{\tud@chapterpage@set}%
    {}{\tud@patch@wrn{tud@schapter}}%
  \apptocmd{\tud@schapter}{\tud@chapterpage@unset}%
    {}{\tud@patch@wrn{tud@schapter}}%
  \pretocmd{\tud@addchap}{\tud@chapterpage@set}%
    {}{\tud@patch@wrn{tud@naddchap}}%
  \apptocmd{\tud@addchap}{\tud@chapterpage@unset}%
    {}{\tud@patch@wrn{tud@naddchap}}%
  \pretocmd{\tud@saddchap}{\tud@chapterpage@set}%
    {}{\tud@patch@wrn{tud@saddchap}}%
  \apptocmd{\tud@saddchap}{\tud@chapterpage@unset}%
    {}{\tud@patch@wrn{tud@saddchap}}%
}
%    \end{macrocode}
% Da wie bereits beschrieben das Vorgehen äußerst fragwürdig ist, wird bei der
% Verwendung von \cs{chapterpage} zumindest einmalig eine Warnung ausgegeben.
%    \begin{macrocode}
\newcommand*\tud@chapterpage@wrn{%
  \ClassWarning{\tudcls@name}{%
    The command \string\chapterpage\space is not\MessageBreak%
    recommended. You should use the same style for\MessageBreak%
    chapters throughout the document
  }%
  \global\let\tud@chapterpage@wrn\relax%
}
%    \end{macrocode}
% \end{macro}^^A \tud@chapterpage@wrn
% \end{macro}^^A \tud@chapterpage@unset
% \end{macro}^^A \tud@chapterpage@set
% \end{macro}^^A \if@tud@chapterpage@temp
% \end{macro}^^A \chapterpage
%
% \iffalse
%</package>
% \fi
%
% \subsection{Kompatibilität zu früheren \TUDScript-Versionen}
%
% Neben den Optionen und Befehlen im Paket \pkg{tudscrcomp} gibt es weitere 
% veraltete, die jedoch direkt in den Klassen vorgehalten werden. Deren Maßgabe 
% ist es, die Kompatibilität zu älteren \TUDScript-Versionen zu erhalten. 
%
% Mit der Version v2.02 wurde eine Menge~-- teilweise sehr tiefgreifend~-- 
% geändert. Die Kompatibilität zu früheren Versionen wird nicht verfolgt.
% In Zukunft wird es wohl noch die ein oder andere Änderung geben. Hierfür soll 
% jedoch allerdings sehr wohl ein Kompatibilitätsmodus zur Verfügung stehen. 
% Hiermit wird alles dafür vorbereitet.
%
% \iffalse
%<*class&option>
% \fi
%
% \begin{option}{color}
% \begin{option}{colour}
% Die alte \cls{tudbook}-Klasse hat die Option \opt{color} definiert, mit
% welcher ein Umschalten auf farbige Titel- und Kapitelseiten möglich ist. Aus
% Kompatibilitätsgründen wird diese hier ebenfalls vorgehalten.
%    \begin{macrocode}
\TUD@key{color}[true]{%
  \TUD@set@numkey{color}{@tempa}{%
    {false}{0},{off}{0},{no}{0},%
    {simple}{0},{std}{0},{standard}{0},{mono}{0},{monochrom}{0},%
    {true}{1},{on}{1},{yes}{1},{color}{1},{colour}{1},%
    {full}{1},{colorfull}{1},{fullcolor}{1},{colourfull}{1},{fullcolour}{1},%
    {lite}{2},{colorlite}{2},{litecolor}{2},{colourlite}{2},{litecolour}{2},%
    {light}{2},{colorlight}{2},{lightcolor}{2},{colorpale}{2},{palecolor}{2},%
    {pale}{2},{colourlight}{2},{lightcolour}{2},{colourpale}{2},{palecolour}{2}%
  }{#1}%
  \ifx\FamilyKeyState\FamilyKeyStateProcessed%
    \ifcase \@tempa\relax%
      \TUDoptions{cd=true}%
    \or%
      \TUDoptions{cd=color}%
    \or%
      \TUDoptions{cd=lite}%
    \fi%
  \fi%
}
\TUD@key{colour}[true]{\TUDoptions{color=#1}\FamilyKeyStateProcessed}
%    \end{macrocode}
% \end{option}^^A colour
% \end{option}^^A color
% \begin{option}{tudfonts}
% Diese Option wird nur aus Gründen der Kompatibilität zu v1.0 definiert.
%    \begin{macrocode}
\TUD@key{tudfonts}[true]{%
  \TUD@set@ifkey{tudfonts}{@tempswa}{#1}%
  \ifx\FamilyKeyState\FamilyKeyStateProcessed%
    \if@tempswa%
      \TUDoptions{cdfont=true}%
    \else%
      \TUDoptions{cdfont=false}%
    \fi%
  \fi%
}
%    \end{macrocode}
% \end{option}^^A tudfonts
% \begin{option}{tudfoot}
% Diese Option wird nur aus Gründen der Kompatibilität zu v1.0 definiert.
%    \begin{macrocode}
\TUD@key{tudfoot}[true]{%
  \TUD@set@ifkey{tudfoot}{@tempswa}{#1}%
  \ifx\FamilyKeyState\FamilyKeyStateProcessed%
    \if@tempswa%
      \TUDoptions{cdfoot=true}%
    \else%
      \TUDoptions{cdfoot=false}%
    \fi%
  \fi%
}
%    \end{macrocode}
% \end{option}^^A tudfoot
% \begin{option}{tudscrver}
% \changes{v2.02}{2014/08/22}{neu}%^^A
% \begin{macro}{\tudscr@v@comp}
% \changes{v2.02}{2014/08/22}{neu}%^^A
% \begin{macro}{\tudscr@v@first}
% \changes{v2.02}{2014/08/22}{neu}%^^A
% \begin{macro}{\tudscr@v@2.02}
% \changes{v2.02}{2014/08/22}{neu}%^^A
% \begin{macro}{\tudscr@v@last}
% \changes{v2.02}{2014/08/22}{neu}%^^A
% In einigen Fällen sind Änderungen mit früheren Versionen nicht kompatibel 
% oder unerwünscht, weil diese beispielsweise das Ausgabeergebnis verändern.
% Standardmäßig werden die Klassen in der aktuellen Version geladen. Mit 
% \opt{tudscrver=\meta{Version}} kann auf das Verhalten einer früheren
% Version geschaltet werden. Die eingestellte Kompatibilität wird als Zahl in 
% \cs{tudscr@v@comp} gespeichert. In den Makros \cs{tudscr@v@\meta{Version}}
% werden die zugehörigen Nummern gespeichert.
%    \begin{macrocode}
\newcommand*\tudscr@v@comp{\tudscr@v@last}
\TUD@key{tudscrver}[last]{%
  \scr@ifundefinedorrelax{tudscr@v@#1}{%
    \def\tudscr@v@comp{0}%
    \ClassWarningNoLine{\tudcls@name}{%
      You have set option `tudscrver' to `#1', but\MessageBreak%
      this value of version is not supported.\MessageBreak%
      Because of this, version was set to `first'%
    }%
    \FamilyKeyStateProcessed%
  }{%
    \ClassInfoNoLine{\tudcls@name}{%
      Switching compatibility level to `#1'%
    }%
    \edef\tudscr@v@comp{\@nameuse{tudscr@v@#1}}%
    \FamilyKeyStateProcessed%
  }%
}
%    \end{macrocode} 
% Eine zusätzliche Bedingung gibt es noch: Die Kompatibilität kann nur beim
% Laden der Klasse gesetzt werden. Danach geht dies nicht mehr.
%    \begin{macrocode}
\AtEndOfClass{%
  \TUD@key{tudscrver}[]{%
    \ClassError{\tudcls@name}{Option `tudscrver' too late}{%
      Option `tudscrver' can only be set while loading the\MessageBreak%
      class `\tudcls@name' but you have tried to set it up later.%
    }%
    \FamilyKeyStateProcessed%
  }%
% Außerdem wird darauf geachtet, dass mindestens \KOMAScript-Version~v3.12
% vorhanden ist, andernfalls wird ein Fehler erzeugt.
%    \begin{macrocode}
  \ifcsdef{scr@v@3.12}{%
    \ifnum\scr@compatibility<\@nameuse{scr@v@3.12}\relax%
      \ClassError{\tudcls@name}{%
        KOMA-Script compatibility level too low%
      }{%
        \TUDVersion\space has to be used at least with\MessageBreak%
        KOMA-Script `version=v3.12'%
      }%
    \fi%
  }{%
    \ClassError{\tudcls@name}{%
      outdated version of KOMA-Script%
    }{%
      \TUDVersion\space has to be used with KOMA-Script v3.12 or newer%
    }%
  }%
}
\@namedef{tudscr@v@first}{0}
\@namedef{tudscr@v@2.02}{0}
\@namedef{tudscr@v@last}{0}
%    \end{macrocode}
% \end{macro}^^A \tudscr@v@last
% \end{macro}^^A \tudscr@v@2.02
% \end{macro}^^A \tudscr@v@first
% \end{macro}^^A \tudscr@v@comp
% \end{option}^^A tudscrver
%
% \iffalse
%</class&option>
%<*class&body>
% \fi
%
% \begin{length}{\chapterheadingvskip}
% Die Länge \cs{chapterheadingvskip} wird aus Gründen der Kompatibilität zu
% älteren Versionen definiert.
%    \begin{macrocode}
%<*book|report>
\newlength{\chapterheadingvskip}
\let\chapterheadingvskip\headingsvskip
%</book|report>
%    \end{macrocode}
% \end{length}^^A \chapterheadingvskip
% \begin{macro}{\professorship}
% Für die Angabe des Lehrstuhls bzw. der Professur mit kann anstelle von 
% \cs{chair} als Aliasbefehl auch \cs{professorship} genutzt werden.
%    \begin{macrocode}
\newcommand*\professorship{}
\let\professorship\chair
%    \end{macrocode}
% \end{macro}^^A \professorship
% \begin{macro}{\student}
% Der Befehl \cs{student} kann als Alias für \cs{author} genutzt werden.
%    \begin{macrocode}
\newcommand*\student{}
\let\student\author
%    \end{macrocode}
% \end{macro}^^A \student
% \begin{macro}{\studentid}
% \begin{macro}{\matriculationid}
% Zur Angabe von Matrikelnummer kann auch \cs{studentid} oder
% \cs{matriculationnumber} genutzt werden.
%    \begin{macrocode}
\newrobustcmd*\studentid{}
\newrobustcmd*\matriculationid{}
\let\studentid\matriculationnumber
\let\matriculationid\matriculationnumber
%    \end{macrocode}
% \end{macro}^^A \matriculationid
% \end{macro}^^A \studentid
% \begin{macro}{\enrolmentyear}
% Das Immatrikulationsjahr kann auch mit \cs{enrolmentyear} angegeben werden.
%    \begin{macrocode}
\newrobustcmd*\enrolmentyear{}
\let\enrolmentyear\matriculationyear
%    \end{macrocode}
% \end{macro}^^A \enrolmentyear
% \begin{macro}{\birthplace}
% Zur Angabe des Geburtsortes kann auch \cs{birthplace} verwendet werden.
%    \begin{macrocode}
\newrobustcmd*\birthplace{}
\let\birthplace\placeofbirth
%    \end{macrocode}
% \end{macro}^^A \birthplace
% \begin{macro}{\location}
% Für die Angabe des Ortes kann auch \cs{location} genutzt werden.
%    \begin{macrocode}
\newcommand*\location{}
\let\location\place
%    \end{macrocode}
% \end{macro}^^A \location
% \begin{macro}{\submissiondate}
% Der Befehl \cs{submissiondate} kann als Aliasbefehl für den Standardbefehl 
% \cs{date} zur Datumsangabe genutzt werden.
%    \begin{macrocode}
\newcommand*\submissiondate{}
\let\submissiondate\date
%    \end{macrocode}
% \end{macro}^^A \submissiondate
% \begin{macro}{\oralexaminationdate}
% Für \cs{defensedate} kann als Aliasbefehl auch \cs{oralexaminationdate}
% verwendet werden.
%    \begin{macrocode}
\newcommand*\oralexaminationdate{}
\let\oralexaminationdate\defensedate
%    \end{macrocode}
% \end{macro}^^A \oralexaminationdate
% \begin{macro}{\birthday}
% Der Geburtstag kann auch mit \cs{birthday} angegeben werden.
%    \begin{macrocode}
\newrobustcmd*\birthday{}
\let\birthday\dateofbirth
%    \end{macrocode}
% \end{macro}^^A \birthday
% \begin{macro}{\logofile}
% \begin{macro}{\logofilename}
% Diese beiden Befehle können anstelle von \cs{headlogo} eingesetzt werden.
%    \begin{macrocode}
\newcommand*\logofile{}
\let\logofile\headlogo
\newcommand*\logofilename{}
\let\logofilename\headlogo
%    \end{macrocode}
% \end{macro}^^A \logofilename
% \end{macro}^^A \logofile
%
% \iffalse
%</class&body>
% \fi
%
% \Finale
%
\endinput
