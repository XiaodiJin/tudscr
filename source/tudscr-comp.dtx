% \CheckSum{781}
% \iffalse meta-comment
% 
% ============================================================================
% 
%  TUD-KOMA-Script
%  Copyright (c) Falk Hanisch <tudscr@gmail.com>, 2012-2015
% 
% ============================================================================
% 
%  This work may be distributed and/or modified under the conditions of the
%  LaTeX Project Public License, version 1.3c of the license. The latest
%  version of this license is in http://www.latex-project.org/lppl.txt and 
%  version 1.3c or later is part of all distributions of LaTeX 2005/12/01
%  or later and of this work. This work has the LPPL maintenance status 
%  "author-maintained". The current maintainer and author of this work
%  is Falk Hanisch.
% 
% ----------------------------------------------------------------------------
% 
% Dieses Werk darf nach den Bedingungen der LaTeX Project Public Lizenz
% in der Version 1.3c, verteilt und/oder veraendert werden. Die aktuelle 
% Version dieser Lizenz ist http://www.latex-project.org/lppl.txt und 
% Version 1.3c oder spaeter ist Teil aller Verteilungen von LaTeX 2005/12/01 
% oder spaeter und dieses Werks. Dieses Werk hat den LPPL-Verwaltungs-Status 
% "author-maintained", wird somit allein durch den Autor verwaltet. Der 
% aktuelle Verwalter und Autor dieses Werkes ist Falk Hanisch.
% 
% ============================================================================
%
% \fi
%
% \CharacterTable
%  {Upper-case    \A\B\C\D\E\F\G\H\I\J\K\L\M\N\O\P\Q\R\S\T\U\V\W\X\Y\Z
%   Lower-case    \a\b\c\d\e\f\g\h\i\j\k\l\m\n\o\p\q\r\s\t\u\v\w\x\y\z
%   Digits        \0\1\2\3\4\5\6\7\8\9
%   Exclamation   \!     Double quote  \"     Hash (number) \#
%   Dollar        \$     Percent       \%     Ampersand     \&
%   Acute accent  \'     Left paren    \(     Right paren   \)
%   Asterisk      \*     Plus          \+     Comma         \,
%   Minus         \-     Point         \.     Solidus       \/
%   Colon         \:     Semicolon     \;     Less than     \<
%   Equals        \=     Greater than  \>     Question mark \?
%   Commercial at \@     Left bracket  \[     Backslash     \\
%   Right bracket \]     Circumflex    \^     Underscore    \_
%   Grave accent  \`     Left brace    \{     Vertical bar  \|
%   Right brace   \}     Tilde         \~}
%
% \iffalse
%%% From File: tudscr-comp.dtx
%<*driver>
\ifx\ProvidesFile\undefined\def\ProvidesFile#1[#2]{}\fi
\ProvidesFile{tudscr-comp.dtx}[%
  2015/04/28 v2.04 TUD-KOMA-Script\space%
%</driver>
%<package&identify>\NeedsTeXFormat{LaTeX2e}[2011/06/27]
%<package&identify>\ProvidesPackage{tudscrcomp}[%
%<*driver|package&identify>
%!TUDVersion
%<package>  package
  (compatibility for old tud(scr) classes)%
]
%</driver|package&identify>
%<*driver>
\RequirePackage[ngerman=ngerman-x-latest]{hyphsubst}
\documentclass[english,ngerman]{tudscrdoc}
\usepackage{selinput}\SelectInputMappings{adieresis={ä},germandbls={ß}}
\usepackage[T1]{fontenc}
\usepackage{babel}
\usepackage{tudscrfonts} % only load this package, if the fonts are installed
\KOMAoptions{parskip=half-}
\CodelineIndex
\RecordChanges
\GetFileInfo{tudscr-comp.dtx}
\begin{document}
  \maketitle
  \DocInput{\filename}
\end{document}
%</driver>
% \fi
%
% \selectlanguage{ngerman}
%
% \changes{v2.02}{2014/09/30}{\pkg{tudscrfonts}: Nutzung alter tud-Klassen 
%   mit neuen Schriftfamilien ermöglicht}^^A
% \changes{v2.04}{2015/04/22}{\pkg{tudscrfonts}: Unterstützung veralteter 
%   Posterklassen}^^A
%
% \section{Kompatibilität zu früheren Versionen und Klassen}
%
% Hiermit werden sowohl veraltete Befehle und Optionen aus früheren Versionen
% von \TUDScript als auch \enquote{veraltete} Klassen und Pakete, welche das 
% \CD der \TnUD mehr oder weniger gut umsetzen, unterstützt.
%
% \StopEventually{\PrintIndex\PrintChanges}
%
% \subsection{Kompatibilität zu früheren \TUDScript-Versionen}
%
% Mit der Version v2.02 wurde eine Menge~-- teilweise sehr tiefgreifend~-- an 
% der Benutzerschnittstelle in \TUDScript geändert. Dabei wird versucht, die
% Kompatibilität zu früheren Versionen so weit es geht aufrecht zu erhalten und
% veraltete Optionen und Befehle zumindest weiter bereitzustellen, wobei bei
% der Verwendung dieser der Anwender darüber informiert wird. Die Vorkehrungen 
% betreffen sowohl die Klassen selber als auch die zusätzlich bereitgestellten 
% Pakete.
%
% \iffalse
%<*class|(package&fonts)>
% \fi
%
% \begin{macro}{\TUD@deprecated@key}
% \changes{v2.03}{2015/02/05}{neu}^^A
% \begin{macro}{\tud@deprecated@cmd}
% \changes{v2.03}{2015/02/05}{neu}^^A
% Um alte Optionen und Befehle dem Anwender bei der Verwendung kenntlich zu 
% machen, werden entsprechende Warnungen definiert. Für Optionen wird dabei
% lediglich die empfohlene Option ausgegeben. Das Ausführen dieser wird später
% definert.
%    \begin{macrocode}
%<*option>
\newcommand*\TUD@deprecated@key[2]{%
%<*!fonts>
  \ClassWarning{\tudcls@name}%
%</!fonts>
%<*fonts>
  \PackageWarning{tudscrfonts}%
%</fonts>
  {%
    The option key `#1' is deprecated.\MessageBreak%
    It's recommended to use `#2'\MessageBreak%
    instead%
  }%
  \TUDoptions{#2}%
}
%</option>
%    \end{macrocode}
% Bei alten Befehlen wird nach der Warnung die Definition des neuen Befehls auf 
% den alten überschrieben.
%    \begin{macrocode}
%<*class&body>
\newcommand*\tud@deprecated@cmd[2]{%
  \AtBeginDocument{%
    \ifcsdef{#1}{}{%
      \expandafter\newrobustcmd\csname #1\endcsname{%
        \ClassWarning{\tudcls@name}{%
          The command `\@backslashchar#1' is deprecated.\MessageBreak%
          It's recommended to use `\@backslashchar#2' instead%
        }%
        \csletcs{#1}{#2}%
        \@nameuse{#2}%
      }%
    }%
  }%
}
%</class&body>
%    \end{macrocode}
% \end{macro}^^A \tud@deprecated@cmd
% \end{macro}^^A \TUD@deprecated@key
%
% \iffalse
%<*option>
% \fi
%
% \begin{option}{cdfonts}
% \begin{option}{tudfonts}
% Diese Optionen werden nur aus Gründen der Kompatibilität zu v1.0 definiert.
%    \begin{macrocode}
\TUD@key{cdfonts}[true]{%
  \TUD@set@ifkey{cdfonts}{@tempswa}{#1}%
  \ifx\FamilyKeyState\FamilyKeyStateProcessed%
    \if@tempswa%
      \TUD@deprecated@key{cdfonts=#1}{cdfont=true}%
    \else%
      \TUD@deprecated@key{cdfonts=#1}{cdfont=false}%
    \fi%
  \fi%
}
\TUD@key{tudfonts}[true]{%
  \TUD@set@ifkey{tudfonts}{@tempswa}{#1}%
  \ifx\FamilyKeyState\FamilyKeyStateProcessed%
    \if@tempswa%
      \TUD@deprecated@key{tudfonts=#1}{cdfont=true}%
    \else%
      \TUD@deprecated@key{tudfonts=#1}{cdfont=false}%
    \fi%
  \fi%
}
%    \end{macrocode}
% \end{option}^^A tudfonts
% \end{option}^^A cdfonts
% \begin{option}{heavyfont}
% Hiermit lässt sich die Schriftstärke im Dokument beeinflussen.
%    \begin{macrocode}
\TUD@key{heavyfont}[true]{%
  \TUD@set@ifkey{heavyfont}{@tempswa}{#1}%
  \ifx\FamilyKeyState\FamilyKeyStateProcessed%
    \if@tempswa%
      \TUD@deprecated@key{heavyfont}{cdfont=heavy}%
    \else%
      \TUD@deprecated@key{heavyfont=#1}{cdfont=true}%
    \fi%
  \fi%
}
%    \end{macrocode}
% \end{option}^^A heavyfont
% \begin{option}{din}
% \begin{option}{nodin}
% \begin{option}{noDIN}
% \changes{v2.04}{2015/04/22}{neu}^^A
% Diese Optionen dienten zur Auswahl, ob die Schrift DIN~Bold für die
% Überschriften verwendet werden soll.
%    \begin{macrocode}
\TUD@key{din}[true]{%
  \TUD@set@ifkey{din}{@tempswa}{#1}%
  \ifx\FamilyKeyState\FamilyKeyStateProcessed%
    \if@tempswa%
      \TUD@deprecated@key{din}{cdfont=din}%
    \else%
      \TUD@deprecated@key{din=#1}{cdfont=nodin}%
    \fi%
  \fi%
}
\TUD@key{nodin}[true]{%
  \TUD@set@ifkey{nodin}{@tempswa}{#1}%
  \ifx\FamilyKeyState\FamilyKeyStateProcessed%
    \if@tempswa%
      \TUD@deprecated@key{nodin}{cdfont=nodin}%
    \else%
      \TUD@deprecated@key{nodin=#1}{cdfont=din}%
    \fi%
  \fi%
}
\TUD@key{noDIN}[true]{%
  \TUD@set@ifkey{noDIN}{@tempswa}{#1}%
  \ifx\FamilyKeyState\FamilyKeyStateProcessed%
    \if@tempswa%
      \TUD@deprecated@key{noDIN}{cdfont=nodin}%
    \else%
      \TUD@deprecated@key{noDIN=#1}{cdfont=din}%
    \fi%
  \fi%
}
%    \end{macrocode}
% \end{option}^^A noDIN
% \end{option}^^A nodin
% \end{option}^^A din
% \begin{option}{sansmath}
% \begin{option}{serifmath}
% Mit dieser Option kann die genutzte Standardschrift für den Mathematiksatz
% für das gesamte Dokument umgestellt werden. Die \cls{tudbook}-Klasse hat 
% neben der Option \opt{sansmath} außerdem den zusätzlichen Schlüssel
% \opt{serifmath} definiert, welcher aus Gründen der Kompatibilität hier
% ebenfalls vorgehalten wird.
%    \begin{macrocode}
\TUD@key{sansmath}[true]{%
  \TUD@set@ifkey{sansmath}{@tempswa}{#1}%
  \ifx\FamilyKeyState\FamilyKeyStateProcessed%
    \if@tempswa%
      \TUD@deprecated@key{sansmath}{cdmath=true}%
    \else%
      \TUD@deprecated@key{sansmath=#1}{cdmath=false}%
    \fi%
  \fi%
}
\TUD@key{serifmath}[true]{%
  \TUD@set@ifkey{serifmath}{@tempswa}{#1}%
  \ifx\FamilyKeyState\FamilyKeyStateProcessed%
    \if@tempswa%
      \TUD@deprecated@key{serifmath}{cdmath=false}%
    \else%
      \TUD@deprecated@key{serifmath=#1}{cdmath=true}%
    \fi%
  \fi%
}
%    \end{macrocode}
% \end{option}^^A serifmath
% \end{option}^^A sansmath
%
% \iffalse
%<*class>
% \fi
%
% \begin{option}{geometry}
% \changes{v2.02}{2014/07/08}{neu}^^A
% Umbennennung der zu allgemein benannten Option, um zukünftigen Konflikten mit
% \KOMAScript{} aus dem Weg zu gehen.
%    \begin{macrocode}
\TUD@key{geometry}[true]{\TUD@deprecated@key{geometry=#1}{cdgeometry=#1}}
%    \end{macrocode}
% \end{option}^^A geometry
% \begin{option}{barfont}
% \changes{v2.02}{2014/07/08}{neu}^^A
% Mit dieser Option kann die Schrift des \CDs und deren Schriftstärke in der
% TUD-Kopfzeile unabhängig von der gewählten Schriftart für den Fließtext 
% aktiviert werden.
%    \begin{macrocode}
\TUD@key{barfont}[true]{%
  \TUD@set@numkey{barfont}{@tempa}{%
    \TUD@numkey@bool,%
    {nocd}{0},{nocdfont}{0},{nocdfonts}{0},{notudfonts}{0},%
    {cd}{1},{cdfont}{1},{cdfonts}{1},{tudfonts}{1},%
    {light}{1},{lightfont}{1},{lite}{1},{litefont}{1},{noheavyfont}{1},%
    {heavy}{2},{heavyfont}{2}%
  }{#1}%
  \ifx\FamilyKeyState\FamilyKeyStateProcessed%
    \ifcase\@tempa\relax% false
      \TUD@deprecated@key{barfont=#1}{cdhead=false}%
    \or% true
      \TUD@deprecated@key{barfont=#1}{cdhead=true}%
    \or% heavy
      \TUD@deprecated@key{barfont=#1}{cdhead=heavy}%
    \fi%
  \fi%
}
%    \end{macrocode}
% \end{option}^^A barfont
% \begin{option}{widehead}
% Der Schalter dient zur Steuerung der Breite des Querbalkens im Kopf der
% \env{tudpage}-Seite. Entweder der Querbalken liegt im Satzspiegel oder
% aber über die komplette Papierbreite.
%    \begin{macrocode}
\TUD@key{widehead}[true]{%
  \TUD@set@ifkey{widehead}{@tempswa}{#1}%
  \ifx\FamilyKeyState\FamilyKeyStateProcessed%
    \if@tempswa%
      \TUD@deprecated@key{widehead}{cdhead=paperwidth}%
    \else%
      \TUD@deprecated@key{widehead=#1}{cdhead=textwidth}%
    \fi%
  \fi%
}
%    \end{macrocode}
% \end{option}^^A widehead
% \begin{option}{tudfoot}
% Diese Option wird nur aus Gründen der Kompatibilität zu v1.0 definiert.
%    \begin{macrocode}
\TUD@key{tudfoot}[true]{%
  \TUD@set@ifkey{tudfoot}{@tempswa}{#1}%
  \ifx\FamilyKeyState\FamilyKeyStateProcessed%
    \if@tempswa%
      \TUD@deprecated@key{tudfoot}{cdfoot=true}%
    \else%
      \TUD@deprecated@key{tudfoot=#1}{cdfoot=false}%
    \fi%
  \fi%
}
%    \end{macrocode}
% \end{option}^^A tudfoot
% \begin{option}{color}
% \begin{option}{colour}
% Die alte \cls{tudbook}-Klasse hat die Option \opt{color} definiert, mit
% welcher ein Umschalten auf farbige Titel- und Kapitelseiten möglich ist. Aus
% Kompatibilitätsgründen wird diese hier ebenfalls vorgehalten.
%    \begin{macrocode}
\TUD@key{color}[true]{%
  \TUD@set@numkey{color}{@tempa}{%
    \TUD@numkey@bool,%
    {nocolor}{0},{nocolour}{0},{monochrome}{0},{monochromatic}{0},%
    {color}{1},{colour}{1},%
    {lite}{2},{light}{2},{pale}{2},%
    {colorlite}{2},{litecolor}{2},{colourlite}{2},{litecolour}{2},%
    {colorlight}{2},{lightcolor}{2},{colourlight}{2},{lightcolour}{2},%
    {colorpale}{2},{palecolor}{2},{colourpale}{2},{palecolour}{2},%
    {bicolor}{3},{bicolour}{3},{twocolor}{3},{twocolour}{3},%
    {bichrome}{3},{bichromatic}{3},{dichrome}{3},{dichromatic}{3},%
    {full}{4},{colorfull}{4},{fullcolor}{4},{colourfull}{4},{fullcolour}{4}%
  }{#1}%
  \ifx\FamilyKeyState\FamilyKeyStateProcessed%
    \ifcase\@tempa\relax% false
      \TUD@deprecated@key{color=#1}{cd=true}%
    \or% true
      \TUD@deprecated@key{color=#1}{cd=color}%
    \or% lite
      \TUD@deprecated@key{color=#1}{cd=pale}%
    \or% bicolor
      \TUD@deprecated@key{color=#1}{cd=bicolor}%
    \or% full
      \TUD@deprecated@key{color=#1}{cd=fullcolor}%
    \fi%
  \fi%
}
\TUD@key{colour}[true]{\TUDoptions{color=#1}}
%    \end{macrocode}
% \end{option}^^A colour
% \end{option}^^A color
%
% \iffalse
%</class>
% \fi
%
% \begin{option}{tudscrver}
% \changes{v2.02}{2014/08/22}{neu}^^A
% \begin{macro}{\tud@v@comp}
% \changes{v2.02}{2014/08/22}{neu}^^A
% \begin{macro}{\tud@v@first}
% \changes{v2.02}{2014/08/22}{neu}^^A
% \begin{macro}{\tud@v@2.02}
% \changes{v2.02}{2014/08/22}{neu}^^A
% \begin{macro}{\tud@v@2.03}
% \changes{v2.03}{2015/01/13}{neu}^^A
% \begin{macro}{\tud@v@2.04}
% \changes{v2.04}{2015/02/18}{neu}^^A
% \begin{macro}{\tud@v@last}
% \changes{v2.02}{2014/08/22}{neu}^^A
% \changes{v2.03}{2015/01/13}{angepasst}^^A
% \changes{v2.04}{2015/02/18}{angepasst}^^A
% In einigen Fällen sind Änderungen mit früheren Versionen nicht kompatibel 
% oder unerwünscht, weil diese beispielsweise das Ausgabeergebnis verändern.
% Standardmäßig werden die Klassen in der aktuellen Version geladen. Mit 
% \opt{tudscrver=\meta{Version}} kann auf das Verhalten einer früheren
% Version geschaltet werden. Die eingestellte Kompatibilität wird als Zahl in 
% \cs{tud@v@comp} gespeichert. In den Makros \cs{tud@v@\meta{Version}} werden
% die zugehörigen Nummern gespeichert.
%    \begin{macrocode}
\newcommand*\tud@v@comp{\tud@v@last}
\TUD@key{tudscrver}[last]{%
  \ifcsvoid{tud@v@#1}{%
    \def\tud@v@comp{0}%
%<*class>
    \ClassWarningNoLine{\tudcls@name}%
%</class>
%<*package&fonts>
    \PackageWarningNoLine{tudscrfonts}%
%</package&fonts>
    {%
      You have set option `tudscrver' to `#1', but\MessageBreak%
      this value is not supported. Because of this, the\MessageBreak%
      value for `tudscrver' was set to `first'%
    }%
    \FamilyKeyStateProcessed%
  }{%
%<*class>
    \ClassInfoNoLine{\tudcls@name}%
%</class>
%<*package&fonts>
    \PackageInfoNoLine{tudscrfonts}%
%</package&fonts>
    {%
      Switching compatibility level to `#1'%
    }%
    \edef\tud@v@comp{\@nameuse{tud@v@#1}}%
    \FamilyKeyStateProcessed%
  }%
}
%    \end{macrocode}
% Eine zusätzliche Bedingung gibt es noch: Die Kompatibilität kann nur beim
% Laden der Klasse bzw. des Paketes gesetzt werden, danach nicht mehr.
%    \begin{macrocode}
%<*class>
\AtEndOfClass%
%</class>
%<*package>
\AtEndOfPackage%
%</package>
{%
  \TUD@key{tudscrver}[]{%
%<*class>
    \ClassError{\tudcls@name}%
%</class>
%<*package&fonts>
    \PackageError{tudscrfonts}%
%</package&fonts>
    {Option `tudscrver' too late}{%
      Option `tudscrver' can only be set while loading the\MessageBreak%
%<*class>
      class `\tudcls@name'
%</class>
%<*package&fonts>
      package `tudscrfonts'
%</package&fonts>
      but you have tried to set it up later.%
    }%
    \FamilyKeyStateProcessed%
  }%
%    \end{macrocode}
% Außerdem wird darauf geachtet, dass mindestens \KOMAScript-Version~v3.12
% vorhanden ist, andernfalls wird ein Fehler erzeugt.
%    \begin{macrocode}
%<*class>
  \ifcsdef{scr@v@3.12}{%
    \ifnum\scr@compatibility<\@nameuse{scr@v@3.12}\relax%
      \ClassError{\tudcls@name}{%
        KOMA-Script compatibility level too low%
      }{%
        \TUDVersion\space has to be used at least with\MessageBreak%
        KOMA-Script `version=v3.12'%
      }%
    \fi%
  }{%
    \ClassError{\tudcls@name}{%
      outdated version of KOMA-Script%
    }{%
      \TUDVersion\space has to be used with KOMA-Script v3.12 or newer%
    }%
  }%
%</class>
}
%    \end{macrocode}
% Die numerischen Werte zu den einzelnen Versionen.
% \ToDo{Für jede neue Version der entsprechende Befehl}[Release]
%    \begin{macrocode}
\@namedef{tud@v@first}{0}
\@namedef{tud@v@2.02}{0}
\@namedef{tud@v@2.03}{1}
\@namedef{tud@v@2.04}{2}
\@namedef{tud@v@last}{2}
%    \end{macrocode}
% \end{macro}^^A \tud@v@last
% \end{macro}^^A \tud@v@2.04
% \end{macro}^^A \tud@v@2.03
% \end{macro}^^A \tud@v@2.02
% \end{macro}^^A \tud@v@first
% \end{macro}^^A \tud@v@comp
% \end{option}^^A tudscrver
% \begin{macro}{\tud@ifverlower}
% \changes{v2.03}{2015/01/13}{neu}^^A
% Mit diesem Befehl können abhängig von der gegebenen Kompatibilitätsversion in 
% den Klassen unterschiedliche Einstellungen vorgenommen werden.
%    \begin{macrocode}
\newcommand*\tud@ifverlower[1]{%
  \ifnum\tud@v@comp<\@nameuse{tud@v@#1}\relax%
    \expandafter\@firstoftwo%
  \else%
    \expandafter\@secondoftwo%
  \fi%
}
%    \end{macrocode}
% \end{macro}^^A \tud@ifverlower
%
% \iffalse
%</option>
%</class|(package&fonts)>
% \fi
%
% Jetzt folgen noch einige veraltete Befehle für die \TUDScript-Klassen.
%
% \iffalse
%<*class&body>
% \fi
%
% \begin{length}{\chapterheadingvskip}
% Die Länge \cs{chapterheadingvskip} wird aus Gründen der Kompatibilität zu
% älteren Versionen definiert.
%    \begin{macrocode}
%<*book|report>
\newlength\chapterheadingvskip
\let\chapterheadingvskip\headingsvskip
%</book|report>
%    \end{macrocode}
% \end{length}^^A \chapterheadingvskip
% \begin{macro}{\professorship}
% Für die Angabe des Lehrstuhls bzw. der Professur mit kann anstelle von 
% \cs{chair} als Aliasbefehl auch \cs{professorship} genutzt werden.
%    \begin{macrocode}
\tud@deprecated@cmd{professorship}{chair}
%    \end{macrocode}
% \end{macro}^^A \professorship
% \begin{macro}{\student}
% Der Befehl \cs{student} kann als Alias für \cs{author} genutzt werden.
%    \begin{macrocode}
\tud@deprecated@cmd{student}{author}
%    \end{macrocode}
% \end{macro}^^A \student
% \begin{macro}{\studentid}
% \begin{macro}{\matriculationid}
% Zur Angabe von Matrikelnummer kann auch \cs{studentid} oder
% \cs{matriculationnumber} genutzt werden.
%    \begin{macrocode}
\tud@deprecated@cmd{studentid}{matriculationnumber}
\tud@deprecated@cmd{matriculationid}{matriculationnumber}
%    \end{macrocode}
% \end{macro}^^A \matriculationid
% \end{macro}^^A \studentid
% \begin{macro}{\enrolmentyear}
% Das Immatrikulationsjahr kann auch mit \cs{enrolmentyear} angegeben werden.
%    \begin{macrocode}
\tud@deprecated@cmd{enrolmentyear}{matriculationyear}%
%    \end{macrocode}
% \end{macro}^^A \enrolmentyear
% \begin{macro}{\birthplace}
% Zur Angabe des Geburtsortes kann auch \cs{birthplace} verwendet werden.
%    \begin{macrocode}
\tud@deprecated@cmd{birthplace}{placeofbirth}
%    \end{macrocode}
% \end{macro}^^A \birthplace
% \begin{macro}{\location}
% Für die Angabe des Ortes kann auch \cs{location} genutzt werden.
%    \begin{macrocode}
\tud@deprecated@cmd{location}{place}
%    \end{macrocode}
% \end{macro}^^A \location
% \begin{macro}{\submissiondate}
% Der Befehl \cs{submissiondate} kann als Aliasbefehl für den Standardbefehl 
% \cs{date} zur Datumsangabe genutzt werden.
%    \begin{macrocode}
\tud@deprecated@cmd{submissiondate}{date}
%    \end{macrocode}
% \end{macro}^^A \submissiondate
% \begin{macro}{\oralexaminationdate}
% Für \cs{defensedate} kann als Aliasbefehl auch \cs{oralexaminationdate}
% verwendet werden.
%    \begin{macrocode}
\tud@deprecated@cmd{oralexaminationdate}{defensedate}%
%    \end{macrocode}
% \end{macro}^^A \oralexaminationdate
% \begin{macro}{\birthday}
% Der Geburtstag kann auch mit \cs{birthday} angegeben werden.
%    \begin{macrocode}
\tud@deprecated@cmd{birthday}{dateofbirth}
%    \end{macrocode}
% \end{macro}^^A \birthday
% \begin{macro}{\logofile}
% \begin{macro}{\logofilename}
% Diese beiden Befehle können anstelle von \cs{headlogo} eingesetzt werden.
%    \begin{macrocode}
\tud@deprecated@cmd{logofile}{headlogo}
\tud@deprecated@cmd{logofilename}{headlogo}
%    \end{macrocode}
% \end{macro}^^A \logofilename
% \end{macro}^^A \logofile
% \begin{macro}{\startdate}
% \begin{macro}{\finaldate}
% \begin{macro}{\maturitydate}
% Für das Paket \pkg{tudscrsupervisor} sind auch einige Befehle als veraltet 
% anzusehen.
%    \begin{macrocode}
\AfterPackage{tudscrsupervisor}{%
  \tud@deprecated@cmd{startdate}{issuedate}%
  \tud@deprecated@cmd{finaldate}{duedate}%
  \tud@deprecated@cmd{maturitydate}{duedate}%
}
%    \end{macrocode}
% \end{macro}^^A \maturitydate
% \end{macro}^^A \finaldate
% \end{macro}^^A \startdate
%
%
% \iffalse
%</class&body>
%<*package&option&!fonts>
% \fi
%
% \subsection{Das Paket \pkg{tudscrcomp} -- Kompatibilität veralteter Klassen}
%
% Neben den \TUDScript-Klassen gibt es zahlreiche andere \LaTeX-Umsetzungen des 
% \CDs der \TnUD. Um eine Migration von diesen \enquote{veralteten} Klassen
% zu erleichtern, wird das Paket \pkg{tudscrcomp} bereitgestellt. Mit diesem 
% können bei der Verwendung von \TUDScript die meisten Optionen und Befehle 
% genutzt werden, welche durch die Klasse \cls{tudbook} und teilweise vormals 
% durch die Klassen die \TUDScript-Klassen in der Version~v1.0 sowie die 
% Posterklassen \cls{tudmathposter} bzw. \cls{tudposter} bereitgestellt wurden.
%
% \ToDo{%
%   Optionen für \cls{tudbook} und \cls{tudmathposter}\val{=Mathematik(A0)}%
% }[v2.04]
%
% \subsubsection{Kompatibilität zu \cls{tudbook}}
%
% Die folgenden Optionen und Befehle werden durch die Klasse \cls{tudbook} und
% teilweise durch \TUDScript~v1.0 definiert.
%
% \begin{option}{colortitle}
% \begin{option}{nocolortitle}
% Für farbige Einstellungen wird von \cls{tudbook} die Option \opt{color} 
% definiert. Soll die Titelseite konträr dazu gesetzt werden, muss sich mit den
% Schlüsseln \opt{colortitle} und \opt{nocolortitle} beholfen werden.
%    \begin{macrocode}
\TUD@key{colortitle}[true]{%
  \TUD@set@ifkey{colortitle}{@tempswa}{#1}%
  \ifx\FamilyKeyState\FamilyKeyStateProcessed%
    \if@tempswa%
      \TUDoptions{cdtitle=color}%
    \else%
      \TUDoptions{cdtitle=true}%
    \fi%
  \fi%
}
\TUD@key{nocolortitle}[true]{%
  \TUD@set@ifkey{nocolortitle}{@tempswa}{#1}%
  \ifx\FamilyKeyState\FamilyKeyStateProcessed%
    \if@tempswa%
      \TUDoptions{cdtitle=true}%
    \else%
      \TUDoptions{cdtitle=color}%
    \fi%
  \fi%
}
%    \end{macrocode}
% \end{option}^^A nocolortitle
% \end{option}^^A colortitle
% \begin{option}{ddcfooter}
% Außer der Option \opt{ddc} gibt es bei der alten \cls{tudbook}-Klasse noch
% den Schlüssel \opt{ddcfooter}. Dieser wird auf die Option \opt{ddcfoot} 
% gelegt.
%    \begin{macrocode}
\TUD@key{ddcfooter}[true]{%
  \TUD@set@ifkey{ddcfooter}{@tempswa}{#1}%
  \ifx\FamilyKeyState\FamilyKeyStateProcessed%
    \if@tempswa%
      \TUDoptions{ddcfoot=true}%
    \else%
      \TUDoptions{ddcfoot=false}%
    \fi%
  \fi%
}
%    \end{macrocode}
% \end{option}^^A ddcfooter
%
% \iffalse
%</package&option&!fonts>
%<*package&body&!fonts>
% \fi
%
% \begin{macro}{\einrichtung}
% \begin{macro}{\fachrichtung}
% \begin{macro}{\institut}
% \begin{macro}{\professur}
% \begin{macro}{\moreauthor}
% \begin{macro}{\submitdate}
% \begin{macro}{\supervisorII}
% \begin{macro}{\supervisedby}
% \begin{macro}{\supervisedIIby}
% \begin{macro}{\submittedon}
% Es werden Aliasbefehle für die Eingabefelder definiert.
%    \begin{macrocode}
\newcommand*\einrichtung{}
\let\einrichtung\faculty
\newcommand*\fachrichtung{}
\let\fachrichtung\department
\newcommand*\institut{}
\let\institut\institute
\newcommand*\professur{}
\let\professur\chair
\newcommand*\moreauthor{}
\let\moreauthor\authormore
\newcommand*\submitdate{}
\let\submitdate\date
\newcommand*\supervisorII[1]{%
  \expandafter\gdef\expandafter\@supervisor\expandafter{\@supervisor\and #1}%
}
\newcommand*\supervisedby[1]{\gdef\supervisorname{#1}}
\newcommand*\supervisedIIby[1]{\gdef\supervisorothername{#1}}
\newcommand*\submittedon[1]{\gdef\datetext{#1}}
%    \end{macrocode}
% \end{macro}^^A \submittedon
% \end{macro}^^A \supervisedIIby
% \end{macro}^^A \supervisedby
% \end{macro}^^A \supervisorII
% \end{macro}^^A \submitdate
% \end{macro}^^A \moreauthor
% \end{macro}^^A \professur
% \end{macro}^^A \institut
% \end{macro}^^A \fachrichtung
% \end{macro}^^A \einrichtung}
% \begin{macro}{\dissertation}
% Bei der Definition des Typs der Abschlussarbeit mit \cs{dissertation} wird
% die Lokalisierungsvariable \cs{dissertationname} verwendet und die Feldnamen
% angepasst.
%    \begin{macrocode}
\newcommand*\dissertation{%
  \thesis{\dissertationname}%
  \let\supervisorname\refereename%
  \let\supervisorothername\refereeothername%
}
%    \end{macrocode}
% \end{macro}^^A \dissertation
% \begin{environment}{theglossary}
% \begin{macro}{\glossaryname}
% \begin{macro}{\glossitem}
% Eine rudimentäre Umgebung für ein Glossar.
%    \begin{macrocode}
\AtBeginDocument{%
  \ifdef{\theglossary}{}{%
    \providecommand*{\glossaryname}{Glossar}
    \newenvironment{theglossary}[1][]{%
      \ClassWarning{\tudcls@name}{%
        Using the environment `theglossary' is not\MessageBreak%
        recommended. You should rather use an appropriate\MessageBreak%
        package such as glossaries%
      }%
      \let\bibname\glossaryname%
      \bib@heading%
      #1%
      \list{}{%
        \setlength\labelsep{\z@}%
        \setlength\labelwidth{\z@}%
        \setlength\itemindent{-\leftmargin}%
      }%
    }{\endlist}
    \newcommand\glossitem[1]{\item[] #1\par}%
  }%
}
%    \end{macrocode}
% \end{macro}^^A \glossitem
% \end{macro}^^A \glossaryname
% \end{environment}^^A theglossary
% \begin{macro}{\chapterpage}
% \begin{macro}{\if@tud@chapterpage@temp}
% \begin{macro}{\tud@chapterpage@set}
% \begin{macro}{\tud@chapterpage@unset}
% \begin{macro}{\tud@chapterpage@wrn}
% Die alte \cls{tudbook}-Klasse stellt den Befehl \cs{chapterpage} bereit.
% Mit diesem ist es möglich, das Verhalten der Kapitelseiten~-- welches durch
% die Option \opt{chapterpage} gesteuert wird~-- temporär umzuschalten, also
% statt Kapitelseiten lediglich Überschriften zu setzen und umgekehrt. Dies ist
% typographisch durchaus zu hinterfragen, allerdings sollen die neuen Klassen
% möglichst kompatibel zu der alten sein, weshalb diese Funktionalität trotzdem
% implementiert wird. Der Befehl \cs{chapterpage} setzt den globalen Schalter
% \cs{if@tud@chapterpage@temp}. Der Befehl \cs{tud@chapterpage@set} setzt für 
% Kapitel das komplementäre Verhalten zur eigentlich gewählten
% \opt{chapterpage}"=Option. Nach dem Setzen der entsprechenden Überschrift
% wird \cs{tud@chapterpage@set} nochmals aufgerufen, das Verhalten auf den
% ursprünglichen Zustand geschaltet und der globale Schalter
% \cs{if@tud@chapterpage@temp} zurückgesetzt.
%    \begin{macrocode}
\newif\if@tud@chapterpage@temp
\newcommand*\chapterpage{%
  \global\@tud@chapterpage@temptrue%
  \tud@chapterpage@wrn%
}
\newcommand*\tud@chapterpage@set[1][]{%
  \if@tud@chapterpage@temp%
    \if@tud@chapterpage%
      \TUDoptions{chapterpage=false}%
    \else%
      \TUDoptions{chapterpage=true}%
    \fi%
  \fi%
}
\newcommand*\tud@chapterpage@unset[1][]{%
  \tud@chapterpage@set%
  \global\@tud@chapterpage@tempfalse%
}
%    \end{macrocode}
% Da wie bereits beschrieben das Vorgehen äußerst fragwürdig ist, wird bei der
% Verwendung von \cs{chapterpage} zumindest einmalig eine Warnung ausgegeben.
%    \begin{macrocode}
\newcommand*\tud@chapterpage@wrn{%
  \ClassWarning{\tudcls@name}{%
    The command \string\chapterpage\space is not\MessageBreak%
    recommended. You should use the same style for\MessageBreak%
    chapters throughout the document
  }%
  \global\let\tud@chapterpage@wrn\relax%
}
%    \end{macrocode}
% \end{macro}^^A \tud@chapterpage@wrn
% \end{macro}^^A \tud@chapterpage@unset
% \end{macro}^^A \tud@chapterpage@set
% \end{macro}^^A \if@tud@chapterpage@temp
% \end{macro}^^A \chapterpage
% \begin{macro}{\tud@chapter}
% \begin{macro}{\tud@schapter}
% \begin{macro}{\tud@addchap}
% \begin{macro}{\tud@saddchap}
% Hier erfolgt die notwendige Anpassungen der internen Gliederungsbefehle.
%    \begin{macrocode}
\AtEndPreamble{%
  \pretocmd{\tud@chapter}{\tud@chapterpage@set}%
    {}{\tud@patch@wrn{tud@nchapter}}%
  \apptocmd{\tud@chapter}{\tud@chapterpage@unset}%
    {}{\tud@patch@wrn{tud@nchapter}}%
  \pretocmd{\tud@schapter}{\tud@chapterpage@set}%
    {}{\tud@patch@wrn{tud@schapter}}%
  \apptocmd{\tud@schapter}{\tud@chapterpage@unset}%
    {}{\tud@patch@wrn{tud@schapter}}%
  \pretocmd{\tud@addchap}{\tud@chapterpage@set}%
    {}{\tud@patch@wrn{tud@naddchap}}%
  \apptocmd{\tud@addchap}{\tud@chapterpage@unset}%
    {}{\tud@patch@wrn{tud@naddchap}}%
  \pretocmd{\tud@saddchap}{\tud@chapterpage@set}%
    {}{\tud@patch@wrn{tud@saddchap}}%
  \apptocmd{\tud@saddchap}{\tud@chapterpage@unset}%
    {}{\tud@patch@wrn{tud@saddchap}}%
}
%    \end{macrocode}
% \end{macro}^^A \tud@saddchap
% \end{macro}^^A \tud@addchap
% \end{macro}^^A \tud@schapter
% \end{macro}^^A \tud@chapter
%
% \subsubsection{Kompatibilität zu \cls{tudmathposter}}
%
% \ToDo{% 
% Folgende Befehle sollten für maximale Kompatibilität unterstützt werden:
% \begin{itemize}
%   \item Umgebung \env{farbtabellen} mit \cs{grautabelle} und \cs{blautabelle}
%   \item \cs{telefon}, \cs{fax}, \cs{homepage}, \cs{email}
%   \item \cs{fusszeile} \cs{footcolumn} Die linke Spalte enthält Hochschule,
%      Einrichtung, Fachrichtung, Institut und Professur. Die rechte Spalte ist
%      frei wählbar, und kann mit den Variablen \cs{author}, \cs{telefon}, 
%      \cs{email} und \cs{homepage} oder mit einem frei gewählten Absatz 
%      (\cs{footcolumn2}) gefüllt werden.
%   \item \cs{zweitlogo}, \cs{drittlogo}, \cs{institutslogo}
%   \item \cs{maketitle}
%   \item \cs{schnittrand} mit \pkg{crop}?
%   \item Abstände der Gliederungsebenen?
%   \item Umgebungen \env{tablehere} und \env{figurehere}
%   \item \cs{multicolsep}
%   \item Optionen \opt{Mathematik} und \opt{MathematikA0}
% \end{itemize}%
% Testen mit \texttt{beispiel-utf8-serifmath.tex}
% }[v2.04]
%
% \iffalse
%</package&body&!fonts>
%<*package&body&fonts>
% \fi
%
% \subsection{Neue Schriftfamilien für alte TUD-Klassen}
%
% Wird das Paket \pkg{tudscrfonts} mit den alten TUD-Klassen \cls{tudbook}, 
% \cls{tudmathposter} etc. oder \TUDScript in der Version~v1.0 verwendet,
% müssen einige Kompatibilitätseinstellungen vorgenmommen werden. Hierfür sind
% einige Patches notwendig.
%
% \begin{macro}{\if@tud@oldclass}
% \changes{v2.02}{2014/08/29}{neu}^^A
% Wenn eine der alten Klassen aus den Vorlagen von Klaus Bergmann geladen 
% wurde, wird dies für spätere Anpassungen registriert. Die alten Posterklassen
% werden ebenfalls unterstützt.
% \ToDo{Klasse tudbeamer und beamer-Stil tud einpflegen}[v2.xx]
%    \begin{macrocode}
\newif\if@tud@oldclass
\@ifclassloaded{tudbook}{\@tud@oldclasstrue}{}%
\@ifclassloaded{tudbeamer}{\@tud@oldclasstrue}{}%
\@ifclassloaded{tudfax}{\@tud@oldclasstrue}{}%
\@ifclassloaded{tudform}{\@tud@oldclasstrue}{}%
\@ifclassloaded{tudhaus}{\@tud@oldclasstrue}{}%
\@ifclassloaded{tudletter}{\@tud@oldclasstrue}{}%
\@ifclassloaded{tudposter}{\@tud@oldclasstrue}{}%
\@ifclassloaded{tudmathposter}{\@tud@oldclasstrue}{}%
%    \end{macrocode}
% \end{macro}^^A if@tud@oldclass
% Wird das Paket \pkg{tudscrfonts} mit einer der alten TUD-Klassen verwendet, 
% müssen einige Kompatibilitätseinstellungen vorgenmommen werden. Dies betrifft
% insbesondere die Mathematikschriften sowie allerhand darin definierter, fixer 
% Schriften.
%    \begin{macrocode}
\ifbool{@tud@oldclass}{%
  \AtBeginDocument{%
    \if@tud@fontspec%
      \DeclareFixedFont{\univLightVII}{\encodingdefault}%
        {UniversLight}{m}{n}{7}%
      \DeclareFixedFont{\univLightObliqueVII}{\encodingdefault}%
        {UniversLightOblique}{m}{n}{7}%
      \DeclareFixedFont{\univBoldVII}{\encodingdefault}%
        {UniversBold}{m}{n}{7}%
      \DeclareFixedFont{\univLightIX}{\encodingdefault}%
        {UniversLight}{m}{n}{9}%
      \DeclareFixedFont{\univLightObliqueIX}{\encodingdefault}%
        {UniversLightOblique}{m}{n}{9}%
      \DeclareFixedFont{\univBoldIX}{\encodingdefault}%
        {UniversBold}{m}{n}{9}%
      \DeclareFixedFont{\univLightXI}{\encodingdefault}%
        {UniversLight}{m}{n}{11}%
      \DeclareFixedFont{\univLightObliqueXI}{\encodingdefault}%
        {UniversLightOblique}{m}{n}{11}%
      \DeclareFixedFont{\univBoldXI}{\encodingdefault}%
        {UniversBold}{m}{n}{11}%
      \DeclareFixedFont{\univLightXV}{\encodingdefault}%
        {UniversLight}{m}{n}{15}%
      \DeclareFixedFont{\univLightObliqueXV}{\encodingdefault}%
        {UniversLightOblique}{m}{n}{15}%
      \DeclareFixedFont{\univBoldXV}{\encodingdefault}%
        {UniversBold}{m}{n}{15}%
    \else%
      \if@tud@univers%
        \DeclareFixedFont{\univLightVII}{\encodingdefault}%
          {\tud@univers}{l}{n}{7}%
        \DeclareFixedFont{\univLightObliqueVII}{\encodingdefault}%
          {\tud@univers}{l}{sl}{7}%
        \DeclareFixedFont{\univBoldVII}{\encodingdefault}%
          {\tud@univers}{b}{n}{7}%
        \DeclareFixedFont{\univLightIX}{\encodingdefault}%
          {\tud@univers}{l}{n}{9}%
        \DeclareFixedFont{\univLightObliqueIX}{\encodingdefault}%
          {\tud@univers}{l}{sl}{9}%
        \DeclareFixedFont{\univBoldIX}{\encodingdefault}%
          {\tud@univers}{b}{n}{9}%
        \DeclareFixedFont{\univLightXI}{\encodingdefault}%
          {\tud@univers}{l}{n}{11}%
        \DeclareFixedFont{\univLightObliqueXI}{\encodingdefault}%
          {\tud@univers}{l}{sl}{11}%
        \DeclareFixedFont{\univBoldXI}{\encodingdefault}%
          {\tud@univers}{b}{n}{11}%
        \DeclareFixedFont{\univLightXV}{\encodingdefault}%
          {\tud@univers}{l}{n}{15}%
        \DeclareFixedFont{\univLightObliqueXV}{\encodingdefault}%
          {\tud@univers}{l}{sl}{15}%
        \DeclareFixedFont{\univBoldXV}{\encodingdefault}%
          {\tud@univers}{b}{n}{15}%
      \fi%
    \fi%
    \if@sansmath%
      \DeclareSymbolFont{aunLetters}{OT1}{\tud@univers}{l}{sl}%
      \SetSymbolFont{aunLetters}{bold}{OT1}{\tud@univers}{b}{sl}%
      \DeclareSymbolFont{aunOperators}{OT1}{\tud@univers}{l}{n}%
      \SetSymbolFont{aunOperators}{bold}{OT1}{\tud@univers}{b}{n}%
      \DeclareSymbolFont{aunUpright}{OT1}{\tud@univers}{l}{n}%
      \SetSymbolFont{aunUpright}{bold}{OT1}{\tud@univers}{b}{n}%
      \DeclareSymbolFontAlphabet{\mathsf}{aunOperators}%
      \DeclareSymbolFontAlphabet{\mathnormal}{aunLetters}%
      \DeclareMathAlphabet\mathbf{OT1}{\tud@univers}{b}{n}%
      \SetMathAlphabet\mathbf{bold}{OT1}{\tud@univers}{b}{n}%
      \DeclareMathAlphabet\mathit{OT1}{\tud@univers}{l}{sl}%
      \SetMathAlphabet\mathit{bold}{OT1}{\tud@univers}{l}{sl}%
      \@ifpackageloaded{tudfonts}{%
        \let\mv@normal\mv@univers%
        \let\mv@bold\mv@universbold%
      }{}%
    \fi%
    \let\dinBold\dinbn%
    \let\dinfamily\dinbn%
  }%
}{}
%    \end{macrocode}
%
% \subsubsection{Patches für die Klasse \cls{tudbook}}
%
% Die notwendigen Anpassungen der Klasse \cls{tudbook} betreffen die
% Schriftauswahl bei Überschriften.
%
% \begin{macro}{\@makechapterhead}
% \begin{macro}{\@makeschapterhead}
% \begin{macro}{\section}
% \begin{macro}{\showtitle}
% Die Überschriften sollen abhängig von der Option \opt{cdfont} und nicht immer 
% zwingend in DIN~Bold gesetzt werden.
%    \begin{macrocode}
\@ifclassloaded{tudbook}{%
  \patchcmd{\@makechapterhead}{\dinBold\Huge\bfseries}%
    {\bfseries\tud@font@heading\Huge}{}{\tud@patch@wrn{@makechapterhead}}%
  \patchcmd{\@makeschapterhead}{\dinBold\Huge\bfseries}%
    {\bfseries\tud@font@heading\Huge}{}{\tud@patch@wrn{@makeschapterhead}}%
  \patchcmd{\section}{\dinBold\Large\bfseries}%
    {\bfseries\tud@font@heading\Large}{}{\tud@patch@wrn{section}}%
  \patchcmd{\showtitle}{\dinBold\Huge\bfseries}%
    {\bfseries\tud@font@heading\Huge}{}{\tud@patch@wrn{showtitle}}%
}{}
%    \end{macrocode}
% \end{macro}^^A \showtitle
% \end{macro}^^A \section
% \end{macro}^^A \@makeschapterhead
% \end{macro}^^A \@makechapterhead
%
% \subsubsection{Patches für \cls{tudposter} und \cls{tudmathposter}}
%
% Auch für die beiden Posterklassen \cls{tudmathposter} und \cls{tudposter}
% sowie das Paket \pkg{tudfonts} sind kleine Anpassungen notwendig.
%
% \begin{KOMAfont}{title}
% \changes{v2.04}{2015/04/24}{\cls{tudmathposter} wird unterstützt}^^A
% \begin{macro}{\subtitlefont}
% \changes{v2.04}{2015/04/24}{\cls{tudmathposter} wird unterstützt}^^A
% \begin{macro}{\sectionfont}
% \changes{v2.04}{2015/04/24}{\cls{tudmathposter} wird unterstützt}^^A
% \begin{macro}{\subsectionfont}
% \changes{v2.04}{2015/04/24}{\cls{tudmathposter} wird unterstützt}^^A
% \begin{macro}{\if@sansmath}
% \changes{v2.04}{2015/04/24}{\cls{tudmathposter} wird unterstützt}^^A
% \begin{macro}{\tudfont}
% \changes{v2.04}{2015/04/24}{\cls{tudmathposter} wird unterstützt}^^A
% Es werden einige Einstellungen für die Überschriften angepasst.
%    \begin{macrocode}
% \begin{macro}{\tudfont}
\providecommand*\tudfont[1]{\cdfont{#1}}
\@ifclassloaded{tudmathposter}{%
  \setkomafont{title}{\bfseries\tud@font@heading\Huge}%
  \renewcommand*\subtitlefont{\fontseries{m}\tud@font@heading\huge}%
  \if@mathevorgabe%
    \renewcommand*\sectionfont{\univbn\LARGE}%
    \renewcommand*\subsectionfont{\univbn\large}%
  \fi%
}{}
\@ifpackageloaded{tudfonts}{%
  \let\if@sansmath\if@tudfonts@sansmath%
  \renewcommand*\tudfont[1]{\cdfont{#1}}%
  \if@sansmath%
    \if@tudfonts@eulermath%
      \PackageWarningNoLine{tudscrfonts}{%
        You're using non-serif fonts in math mode. It's\MessageBreak%
        recommended to pass the option `noeulermath' to\MessageBreak%
        package `tudfonts'%
      }%
    \fi%
  \fi%
}{%
  \newif\if@tudfonts@nodin%
  \@ifundefined{if@sansmath}{\newif\if@sansmath}{}%
}
%    \end{macrocode}
% \end{macro}^^A \tudfont
% \end{macro}^^A \if@sansmath
% \end{macro}^^A \subsectionfont
% \end{macro}^^A \sectionfont
% \end{macro}^^A \subtitlefont
% \end{KOMAfont}^^A title
%
% Für die Klasse \cls{tudposter} muss lediglich ein Schalter übernommen werden.
%
%    \begin{macrocode}
\@ifclassloaded{tudposter}{%
  \let\if@sansmath\if@tudposter@sansmath%
}{}
%    \end{macrocode}
%
% \subsubsection{Patches für die \TUDScript-Klassen~v1.0}
%
% Für die Verwendung des Paketes \pkg{tudscrfonts} mit den \TUDScript-Klassen
% in der Version~v1.0 sind ebenfalls einige Anpassungen notwendig.
%
% \begin{macro}{\if@tud@oldscrclass}
% \changes{v2.02}{2014/08/29}{neu}^^A
% Soll das Paket \pkg{tudscrfonts} mit den \TUDScript-Klassen der Version~v1.0
% genutzt werden, sind ebenfalls einige Anpassungen notwendig.
%    \begin{macrocode}
\newif\if@tud@oldscrclass
\@ifclassloaded{tudscrbookold}{\@tud@oldscrclasstrue}{}%
\@ifclassloaded{tudscrreprtold}{\@tud@oldscrclasstrue}{}%
\@ifclassloaded{tudscrartclold}{\@tud@oldscrclasstrue}{}%
%    \end{macrocode}
% \end{macro}^^A if@tud@oldclass
% Es muss Sorge dafür getragen werden, dass einige veraltete Befehle
% aktualisiert werden. Hierfür müssen diese \enquote{unschädlich} gemacht
% werden, noch bevor das Paket selbst in Erscheinung tritt.
%    \begin{macrocode}
\ifbool{@tud@oldscrclass}{%
  \let\univln\relax%
  \let\univrn\relax%
  \let\univbn\relax%
  \let\univxn\relax%
  \let\univls\relax%
  \let\univrs\relax%
  \let\univbs\relax%
  \let\univxs\relax%
  \let\dinbn\relax%
  \let\textuln\relax%
  \let\texturn\relax%
  \let\textubn\relax%
  \let\textuxn\relax%
  \let\textuls\relax%
  \let\texturs\relax%
  \let\textubs\relax%
  \let\textuxs\relax%
  \let\textdbn\relax%
  \AtBeginDocument{%
    \cmd@restore{rmdefault}%
    \cmd@restore{sfdefault}%
    \cmd@restore{bfdefault}%
    \cmd@restore{mddefault}%
    \cmd@restore{familydefault}%
    \cmd@restore{seriesdefault}%
    \cmd@restore{shapedefault}%
    \cmd@restore{rmfamily}%
    \cmd@restore{sffamily}%
    \cmd@restore{ttfamily}%
    \DeclareSymbolFontAlphabet{\mathrm}{operators}
    \if@tud@fontspec%
      \DeclareFixedFont{\univLightHead}{\encodingdefault}%
        {UniversLight}{m}{n}{\tud@head@fntsz}%
      \DeclareFixedFont{\univBoldHead}{\encodingdefault}%
        {UniversBold}{m}{n}{\tud@head@fntsz}%
    \else%
      \if@tud@univers%
        \DeclareFixedFont{\univLightHead}{\encodingdefault}%
          {\tud@univers}{l}{n}{\tud@head@fntsz}%
        \DeclareFixedFont{\univBoldHead}{\encodingdefault}%
          {\tud@univers}{b}{n}{\tud@head@fntsz}%
      \fi%
    \fi%
  }%
}{}
%    \end{macrocode}
%
% \iffalse
%</package&body&fonts>
% \fi
%
% \Finale
%
\endinput
