% \CheckSum{959}
% \iffalse meta-comment
% 
% ============================================================================
% 
%  TUD-KOMA-Script
%  Copyright (c) Falk Hanisch <tudscr@gmail.com>, 2012-2016
% 
% ============================================================================
% 
%  This work may be distributed and/or modified under the conditions of the
%  LaTeX Project Public License, version 1.3c of the license. The latest
%  version of this license is in http://www.latex-project.org/lppl.txt and 
%  version 1.3c or later is part of all distributions of LaTeX 2005/12/01
%  or later and of this work. This work has the LPPL maintenance status 
%  "author-maintained". The current maintainer and author of this work
%  is Falk Hanisch.
% 
% ----------------------------------------------------------------------------
% 
% Dieses Werk darf nach den Bedingungen der LaTeX Project Public Lizenz
% in der Version 1.3c, verteilt und/oder veraendert werden. Die aktuelle 
% Version dieser Lizenz ist http://www.latex-project.org/lppl.txt und 
% Version 1.3c oder spaeter ist Teil aller Verteilungen von LaTeX 2005/12/01 
% oder spaeter und dieses Werks. Dieses Werk hat den LPPL-Verwaltungs-Status 
% "author-maintained", wird somit allein durch den Autor verwaltet. Der 
% aktuelle Verwalter und Autor dieses Werkes ist Falk Hanisch.
% 
% ============================================================================
%
% \fi
%
% \CharacterTable
%  {Upper-case    \A\B\C\D\E\F\G\H\I\J\K\L\M\N\O\P\Q\R\S\T\U\V\W\X\Y\Z
%   Lower-case    \a\b\c\d\e\f\g\h\i\j\k\l\m\n\o\p\q\r\s\t\u\v\w\x\y\z
%   Digits        \0\1\2\3\4\5\6\7\8\9
%   Exclamation   \!     Double quote  \"     Hash (number) \#
%   Dollar        \$     Percent       \%     Ampersand     \&
%   Acute accent  \'     Left paren    \(     Right paren   \)
%   Asterisk      \*     Plus          \+     Comma         \,
%   Minus         \-     Point         \.     Solidus       \/
%   Colon         \:     Semicolon     \;     Less than     \<
%   Equals        \=     Greater than  \>     Question mark \?
%   Commercial at \@     Left bracket  \[     Backslash     \\
%   Right bracket \]     Circumflex    \^     Underscore    \_
%   Grave accent  \`     Left brace    \{     Vertical bar  \|
%   Right brace   \}     Tilde         \~}
%
% \iffalse
%%% From File: tudscr-comp.dtx
%<*driver>
\ifx\ProvidesFile\undefined\def\ProvidesFile#1[#2]{}\fi
\ProvidesFile{tudscr-comp.dtx}[%
  2016/04/04 v2.05 TUD-KOMA-Script\space%
%</driver>
%<package&identify&comp>\ProvidesPackage{tudscrcomp}[%
%<package&identify&fontsfix>\ProvidesPackage{fix-tudscrfonts}[%
%<*driver|package&identify>
%!TUDVersion
%<package&comp>  package
%<package&fontsfix>  package for font fixes
  (compatibility for old tud(scr) classes)%
]
%</driver|package&identify>
%<*driver>
\RequirePackage[ngerman=ngerman-x-latest]{hyphsubst}
\documentclass[english,ngerman]{tudscrdoc}
\usepackage{selinput}\SelectInputMappings{adieresis={ä},germandbls={ß}}
\usepackage[T1]{fontenc}
\usepackage{babel}
\usepackage{tudscrfonts} % only load this package, if the fonts are installed
\KOMAoptions{parskip=half-}
\CodelineIndex
\RecordChanges
\GetFileInfo{tudscr-comp.dtx}
\begin{document}
  \maketitle
  \DocInput{\filename}
\end{document}
%</driver>
% \fi
%
% \selectlanguage{ngerman}
%
% \changes{v2.02}{2014/09/30}{\pkg{tudscrfonts}: Nutzung alter tud-Klassen 
%   mit neuen Schriftfamilien ermöglicht}^^A
% \changes{v2.04}{2015/04/22}{\pkg{tudscrfonts}: Unterstützung veralteter 
%   Posterklassen}^^A
% \changes{v2.05}{2015/12/08}{\pkg{fix-tudscrfonts}: Dieses Paket übernimmt die 
%   Bereitstellung der Schriften für veraltete Klassen}^^A
%
% \section{Kompatibilität zu früheren Versionen und Klassen}
%
% Hiermit werden sowohl veraltete Befehle und Optionen aus früheren Versionen
% von \TUDScript als auch \enquote{veraltete} Klassen und Pakete, welche das 
% \CD der \TnUD mehr oder weniger gut umsetzen, unterstützt.
%
% \StopEventually{\PrintIndex\PrintChanges}
%
% \subsection{Kompatibilität zu früheren \TUDScript-Versionen}
%
% Mit der Version v2.02 wurde eine Menge~-- teilweise sehr tiefgreifend~-- an 
% der Benutzerschnittstelle in \TUDScript geändert. Dabei wird versucht, die
% Kompatibilität zu früheren Versionen so weit es geht aufrecht zu erhalten und
% veraltete Optionen und Befehle zumindest weiter bereitzustellen, wobei bei
% der Verwendung dieser der Anwender darüber informiert wird. Die Vorkehrungen 
% betreffen sowohl die Klassen selber als auch die zusätzlich bereitgestellten 
% Pakete.
%
% \iffalse
%<*class|(package&fonts)>
% \fi
%
% \begin{macro}{\TUD@deprecated@key}
% \changes{v2.03}{2015/02/05}{neu}^^A
% \begin{macro}{\tud@deprecated@cmd}
% \changes{v2.03}{2015/02/05}{neu}^^A
% Um alte Optionen und Befehle dem Anwender bei der Verwendung kenntlich zu 
% machen, werden entsprechende Warnungen definiert. Für Optionen wird dabei
% lediglich die empfohlene Option ausgegeben. Das Ausführen dieser wird später
% definert.
%    \begin{macrocode}
%<*option>
\newcommand*\TUD@deprecated@key[2]{%
%<*class>
  \ClassWarning{\TUDClassName}%
%</class>
%<*package&fonts>
  \PackageWarning{tudscrfonts}%
%</package&fonts>
  {%
    The option key `#1' is deprecated\MessageBreak%
    It's recommended to use `#2'\MessageBreak%
    instead%
  }%
  \TUDoptions{#2}%
}
%</option>
%    \end{macrocode}
% Bei alten Befehlen wird nach der Warnung die Definition des neuen Befehls auf 
% den alten überschrieben.
%    \begin{macrocode}
%<*class&body>
\newcommand*\tud@deprecated@cmd[2]{%
  \AtBeginDocument{%
    \ifcsdef{#1}{}{%
      \expandafter\newrobustcmd\csname #1\endcsname{%
        \ClassWarning{\TUDClassName}{%
          The command `\@backslashchar#1' is deprecated\MessageBreak%
          It's recommended to use `\@backslashchar#2' instead%
        }%
        \csletcs{#1}{#2}%
        \@nameuse{#2}%
      }%
    }%
  }%
}
%</class&body>
%    \end{macrocode}
% \end{macro}^^A \tud@deprecated@cmd
% \end{macro}^^A \TUD@deprecated@key
%
% \iffalse
%<*option>
% \fi
%
% \begin{option}{fontspec}
% \changes{v2.02}{2014/08/29}{neu}^^A
% \changes{v2.05}{2015/07/06}{veraltet}^^A
% Früher musste die Unterstützung für die Schriftauswahl via \pkg{fontspec} 
% manuell über die folgende Option aktiviert werden.
%    \begin{macrocode}
\TUD@key{fontspec}[true]{%
%<*class>
  \ClassWarning{\TUDClassName}%
%</class>
%<*package&fonts>
  \PackageWarning{tudscrfonts}%
%</package&fonts>
  {%
    The option key `fontspec' is deprecated\MessageBreak%
    You should load the package `fontspec' manually%
  }%
  \TUD@set@ifkey{fontspec}{@tud@fontspec}{#1}%
}
%    \end{macrocode}
% Da dieser Prozess sehr tief in die Schriftauswahl eingreift und das Laden des
% Paketes die Schriftauswahl für das ganze Dokument verändert, wird diese 
% Option nur beim Laden der Klasse dzw. des Paketes zugelassen.
%    \begin{macrocode}
%<*class>
\AtEndOfClass%
%</class>
%<*package&fonts>
\AtEndOfPackage%
%</package&fonts>
{%
  \TUD@key{fontspec}[]%
  {%
%<*class>
    \ClassError{\TUDClassName}%
%</class>
%<*package&fonts>
    \PackageError{tudscrfonts}%
%</package&fonts>
      {Option `fontspec' too late}{%
      Option `fontspec' can only be set while loading the\MessageBreak%
%<*class>
      class `\TUDClassName'
%</class>
%<*package&fonts>
      package `tudscrfonts'
%</package&fonts>
      but you have tried to set it up later.%
    }%
    \FamilyKeyStateProcessed%
  }%
  \if@tud@fontspec%
    \RequirePackage{fontspec}[2014/06/21]%
  \fi%
}
%    \end{macrocode}
% \end{option}^^A fontspec
% \begin{option}{cdfonts}
% \begin{option}{tudfonts}
% Diese Optionen werden nur aus Gründen der Kompatibilität zu v1.0 definiert.
%    \begin{macrocode}
\TUD@key{cdfonts}[true]{%
  \TUD@set@ifkey{cdfonts}{@tempswa}{#1}%
  \ifx\FamilyKeyState\FamilyKeyStateProcessed%
    \if@tempswa%
      \TUD@deprecated@key{cdfonts=#1}{cdfont=true}%
    \else%
      \TUD@deprecated@key{cdfonts=#1}{cdfont=false}%
    \fi%
  \fi%
}
\TUD@key{tudfonts}[true]{%
  \TUD@set@ifkey{tudfonts}{@tempswa}{#1}%
  \ifx\FamilyKeyState\FamilyKeyStateProcessed%
    \if@tempswa%
      \TUD@deprecated@key{tudfonts=#1}{cdfont=true}%
    \else%
      \TUD@deprecated@key{tudfonts=#1}{cdfont=false}%
    \fi%
  \fi%
}
%    \end{macrocode}
% \end{option}^^A tudfonts
% \end{option}^^A cdfonts
% \begin{option}{heavyfont}
% Hiermit lässt sich die Schriftstärke im Dokument beeinflussen.
%    \begin{macrocode}
\TUD@key{heavyfont}[true]{%
  \TUD@set@ifkey{heavyfont}{@tempswa}{#1}%
  \ifx\FamilyKeyState\FamilyKeyStateProcessed%
    \if@tempswa%
      \TUD@deprecated@key{heavyfont}{cdfont=heavy}%
    \else%
      \TUD@deprecated@key{heavyfont=#1}{cdfont=true}%
    \fi%
  \fi%
}
%    \end{macrocode}
% \end{option}^^A heavyfont
% \begin{option}{din}
% \begin{option}{nodin}
% \begin{option}{noDIN}
% \changes{v2.04}{2015/04/22}{neu}^^A
% Diese Optionen dienten zur Auswahl, ob die Schrift \DIN für die Überschriften
% verwendet werden soll.
%    \begin{macrocode}
\TUD@key{din}[true]{%
  \TUD@set@ifkey{din}{@tempswa}{#1}%
  \ifx\FamilyKeyState\FamilyKeyStateProcessed%
    \if@tempswa%
      \TUD@deprecated@key{din}{cdfont=din}%
    \else%
      \TUD@deprecated@key{din=#1}{cdfont=nodin}%
    \fi%
  \fi%
}
\TUD@key{nodin}[true]{%
  \TUD@set@ifkey{nodin}{@tempswa}{#1}%
  \ifx\FamilyKeyState\FamilyKeyStateProcessed%
    \if@tempswa%
      \TUD@deprecated@key{nodin}{cdfont=nodin}%
    \else%
      \TUD@deprecated@key{nodin=#1}{cdfont=din}%
    \fi%
  \fi%
}
\TUD@key{noDIN}[true]{%
  \TUD@set@ifkey{noDIN}{@tempswa}{#1}%
  \ifx\FamilyKeyState\FamilyKeyStateProcessed%
    \if@tempswa%
      \TUD@deprecated@key{noDIN}{cdfont=nodin}%
    \else%
      \TUD@deprecated@key{noDIN=#1}{cdfont=din}%
    \fi%
  \fi%
}
%    \end{macrocode}
% \end{option}^^A noDIN
% \end{option}^^A nodin
% \end{option}^^A din
% \begin{option}{sansmath}
% \begin{option}{serifmath}
% Mit dieser Option kann die genutzte Standardschrift für den Mathematiksatz
% für das gesamte Dokument umgestellt werden. Die \cls{tudbook}-Klasse hat 
% neben der Option \opt{sansmath} außerdem den zusätzlichen Schlüssel
% \opt{serifmath} definiert, welcher aus Gründen der Kompatibilität hier
% ebenfalls vorgehalten wird.
%    \begin{macrocode}
\TUD@key{sansmath}[true]{%
  \TUD@set@ifkey{sansmath}{@tempswa}{#1}%
  \ifx\FamilyKeyState\FamilyKeyStateProcessed%
    \if@tempswa%
      \TUD@deprecated@key{sansmath}{cdmath=true}%
    \else%
      \TUD@deprecated@key{sansmath=#1}{cdmath=false}%
    \fi%
  \fi%
}
\TUD@key{serifmath}[true]{%
  \TUD@set@ifkey{serifmath}{@tempswa}{#1}%
  \ifx\FamilyKeyState\FamilyKeyStateProcessed%
    \if@tempswa%
      \TUD@deprecated@key{serifmath}{cdmath=false}%
    \else%
      \TUD@deprecated@key{serifmath=#1}{cdmath=true}%
    \fi%
  \fi%
}
%    \end{macrocode}
% \end{option}^^A serifmath
% \end{option}^^A sansmath
%
% \iffalse
%<*class>
% \fi
%
% \begin{option}{geometry}
% \changes{v2.02}{2014/07/08}{neu}^^A
% Umbennennung der zu allgemein benannten Option, um zukünftigen Konflikten mit
% \KOMAScript{} aus dem Weg zu gehen.
%    \begin{macrocode}
\TUD@key{geometry}[true]{\TUD@deprecated@key{geometry=#1}{cdgeometry=#1}}
%    \end{macrocode}
% \end{option}^^A geometry
% \begin{option}{barfont}
% \changes{v2.02}{2014/07/08}{neu}^^A
% Mit dieser Option kann die Schrift des \CDs und deren Schriftstärke in der
% TUD-Kopfzeile unabhängig von der gewählten Schriftart für den Fließtext 
% aktiviert werden.
%    \begin{macrocode}
\TUD@key{barfont}[true]{%
  \TUD@set@numkey{barfont}{@tempa}{%
    \TUD@numkey@bool,%
    {nocd}{0},{nocdfont}{0},{nocdfonts}{0},{notudfonts}{0},%
    {cd}{1},{cdfont}{1},{cdfonts}{1},{tudfonts}{1},%
    {light}{1},{lightfont}{1},{lite}{1},{litefont}{1},{noheavyfont}{1},%
    {heavy}{2},{heavyfont}{2},{bold}{2},{boldfont}{2}%
  }{#1}%
  \ifx\FamilyKeyState\FamilyKeyStateProcessed%
    \ifcase\@tempa\relax% false
      \TUD@deprecated@key{barfont=#1}{cdhead=false}%
    \or% true
      \TUD@deprecated@key{barfont=#1}{cdhead=true}%
    \or% heavy
      \TUD@deprecated@key{barfont=#1}{cdhead=heavy}%
    \fi%
  \fi%
}
%    \end{macrocode}
% \end{option}^^A barfont
% \begin{option}{widehead}
% Der Schalter dient zur Steuerung der Breite des Querbalkens im Kopf der
% \env{tudpage}-Seite. Entweder der Querbalken liegt im Satzspiegel oder
% aber über die komplette Papierbreite.
%    \begin{macrocode}
\TUD@key{widehead}[true]{%
  \TUD@set@ifkey{widehead}{@tempswa}{#1}%
  \ifx\FamilyKeyState\FamilyKeyStateProcessed%
    \if@tempswa%
      \TUD@deprecated@key{widehead}{cdhead=paperwidth}%
    \else%
      \TUD@deprecated@key{widehead=#1}{cdhead=textwidth}%
    \fi%
  \fi%
}
%    \end{macrocode}
% \end{option}^^A widehead
% \begin{option}{tudfoot}
% Diese Option wird nur aus Gründen der Kompatibilität zu v1.0 definiert.
%    \begin{macrocode}
\TUD@key{tudfoot}[true]{%
  \TUD@set@ifkey{tudfoot}{@tempswa}{#1}%
  \ifx\FamilyKeyState\FamilyKeyStateProcessed%
    \if@tempswa%
      \TUD@deprecated@key{tudfoot}{cdfoot=true}%
    \else%
      \TUD@deprecated@key{tudfoot=#1}{cdfoot=false}%
    \fi%
  \fi%
}
%    \end{macrocode}
% \end{option}^^A tudfoot
% \begin{option}{color}
% \begin{option}{colour}
% Die alte \cls{tudbook}-Klasse hat die Option \opt{color} definiert, mit
% welcher ein Umschalten auf farbige Titel- und Kapitelseiten möglich ist. Aus
% Kompatibilitätsgründen wird diese hier ebenfalls vorgehalten.
%    \begin{macrocode}
\TUD@key{color}[true]{%
  \TUD@set@numkey{color}{@tempa}{%
    \TUD@numkey@bool,%
    {nocolor}{0},{nocolour}{0},{monochrome}{0},{monochromatic}{0},%
    {color}{1},{colour}{1},%
    {lite}{2},{light}{2},{pale}{2},%
    {colorlite}{2},{litecolor}{2},{colourlite}{2},{litecolour}{2},%
    {colorlight}{2},{lightcolor}{2},{colourlight}{2},{lightcolour}{2},%
    {colorpale}{2},{palecolor}{2},{colourpale}{2},{palecolour}{2},%
    {bicolor}{3},{bicolour}{3},{twocolor}{3},{twocolour}{3},%
    {bichrome}{3},{bichromatic}{3},{dichrome}{3},{dichromatic}{3},%
    {full}{4},{colorfull}{4},{fullcolor}{4},{colourfull}{4},{fullcolour}{4}%
  }{#1}%
  \ifx\FamilyKeyState\FamilyKeyStateProcessed%
    \ifcase\@tempa\relax% false
      \TUD@deprecated@key{color=#1}{cd=true}%
    \or% true
      \TUD@deprecated@key{color=#1}{cd=color}%
    \or% litecolor
      \TUD@deprecated@key{color=#1}{cd=pale}%
    \or% bicolor
      \TUD@deprecated@key{color=#1}{cd=bicolor}%
    \or% full
      \TUD@deprecated@key{color=#1}{cd=fullcolor}%
    \fi%
  \fi%
}
\TUD@key{colour}[true]{\TUDoptions{color=#1}}
%    \end{macrocode}
% \end{option}^^A colour
% \end{option}^^A color
%
% \iffalse
%</class>
% \fi
%
% \begin{option}{tudscrver}
% \changes{v2.02}{2014/08/22}{neu}^^A
% \begin{macro}{\tud@v@comp}
% \changes{v2.02}{2014/08/22}{neu}^^A
% \begin{macro}{\tud@v@first}
% \changes{v2.02}{2014/08/22}{neu}^^A
% \begin{macro}{\tud@v@2.02}
% \changes{v2.02}{2014/08/22}{neu}^^A
% \begin{macro}{\tud@v@2.03}
% \changes{v2.03}{2015/01/13}{neu}^^A
% \begin{macro}{\tud@v@2.03a}
% \changes{v2.03a}{2015/06/12}{neu}^^A
% \begin{macro}{\tud@v@2.04}
% \changes{v2.04}{2015/02/18}{neu}^^A
% \begin{macro}{\tud@v@2.04a}
% \changes{v2.04a}{2015/07/24}{neu}^^A
% \begin{macro}{\tud@v@2.04b}
% \changes{v2.04b}{2015/10/06}{neu}^^A
% \begin{macro}{\tud@v@2.05}
% \changes{v2.05}{2015/07/03}{neu}^^A
% \begin{macro}{\tud@v@last}
% \changes{v2.02}{2014/08/22}{neu}^^A
% \changes{v2.03}{2015/01/13}{angepasst}^^A
% \changes{v2.04}{2015/02/18}{angepasst}^^A
% In einigen Fällen sind Änderungen mit früheren Versionen nicht kompatibel 
% oder unerwünscht, weil diese beispielsweise das Ausgabeergebnis verändern.
% Standardmäßig werden die Klassen in der aktuellen Version geladen. Mit 
% \opt{tudscrver=\meta{Version}} kann auf das Verhalten einer früheren
% Version geschaltet werden. Die eingestellte Kompatibilität wird als Zahl in 
% \cs{tud@v@comp} gespeichert. In den Makros \cs{tud@v@\meta{Version}} werden
% die zugehörigen Nummern gespeichert.
%    \begin{macrocode}
\newcommand*\tud@v@comp{\tud@v@last}
\TUD@key{tudscrver}[last]{%
  \def\@tempa{\@ifnextchar v{\@tempb}{\@tempb v}}%
  \def\@tempb v##1\@nil{\def\@tempc{##1}}%
  \@tempa #1\@nil%
  \ifcsvoid{tud@v@\@tempc}{%
    \def\tud@v@comp{0}%
%<*class>
    \ClassWarningNoLine{\TUDClassName}%
%</class>
%<*package&fonts>
    \PackageWarningNoLine{tudscrfonts}%
%</package&fonts>
    {%
      You have set option `tudscrver' to `\@tempc', but\MessageBreak%
      this value is not supported. Because of this, the\MessageBreak%
      value for `tudscrver' was set to `first'%
    }%
    \FamilyKeyStateProcessed%
  }{%
%<*class>
    \ClassInfoNoLine{\TUDClassName}%
%</class>
%<*package&fonts>
    \PackageInfoNoLine{tudscrfonts}%
%</package&fonts>
    {%
      Switching compatibility level to `\@tempc'%
    }%
    \edef\tud@v@comp{\@nameuse{tud@v@\@tempc}}%
    \FamilyKeyStateProcessed%
  }%
}
%    \end{macrocode}
% Eine zusätzliche Bedingung gibt es noch: Die Kompatibilität kann nur beim
% Laden der Klasse bzw. des Paketes gesetzt werden, danach nicht mehr.
%    \begin{macrocode}
%<*class>
\AtEndOfClass%
%</class>
%<*package>
\AtEndOfPackage%
%</package>
{%
  \TUD@key{tudscrver}[]{%
%<*class>
    \ClassError{\TUDClassName}%
%</class>
%<*package&fonts>
    \PackageError{tudscrfonts}%
%</package&fonts>
    {Option `tudscrver' too late}{%
      Option `tudscrver' can only be set while loading the\MessageBreak%
%<*class>
      class `\TUDClassName'
%</class>
%<*package&fonts>
      package `tudscrfonts'
%</package&fonts>
      but you have tried to set it up later.%
    }%
    \FamilyKeyStateProcessed%
  }%
%    \end{macrocode}
% Außerdem wird darauf geachtet, dass mindestens \KOMAScript-Version~v3.17
% vorhanden ist, andernfalls wird ein Fehler erzeugt.
%    \begin{macrocode}
%<*class>
  \ifcsdef{scr@v@3.17}{%
    \ifnum\scr@compatibility<\@nameuse{scr@v@3.12}\relax%
      \ClassError{\TUDClassName}{%
        KOMA-Script compatibility level too low%
      }{%
        \TUDVersion\space has to be used at least with\MessageBreak%
        KOMA-Script compatibility option `version=3.12'.%
      }%
    \fi%
  }{%
    \ClassError{\TUDClassName}{%
      KOMA-Script v3.17 is required%
    }{%
      \TUDVersion\space has to be used with KOMA-Script v3.17\MessageBreak%
      at least. You're using \KOMAScriptVersion%
    }%
  }%
%</class>
}
%    \end{macrocode}
% Die numerischen Werte zu den einzelnen Versionen.
% \ToDo{Für jede neue Version der entsprechende Befehl}[Release]
%    \begin{macrocode}
\@namedef{tud@v@first}{0}
\@namedef{tud@v@2.02}{0}
\@namedef{tud@v@2.03}{1}
\@namedef{tud@v@2.03a}{1}
\@namedef{tud@v@2.04}{2}
\@namedef{tud@v@2.04a}{2}
\@namedef{tud@v@2.04b}{2}
\@namedef{tud@v@2.04c}{2}
\@namedef{tud@v@2.04d}{2}
\@namedef{tud@v@2.05}{3}
\@namedef{tud@v@last}{3}
%    \end{macrocode}
% \end{macro}^^A \tud@v@last
% \end{macro}^^A \tud@v@2.05
% \end{macro}^^A \tud@v@2.04b
% \end{macro}^^A \tud@v@2.04a
% \end{macro}^^A \tud@v@2.04
% \end{macro}^^A \tud@v@2.03a
% \end{macro}^^A \tud@v@2.03
% \end{macro}^^A \tud@v@2.02
% \end{macro}^^A \tud@v@first
% \end{macro}^^A \tud@v@comp
% \end{option}^^A tudscrver
% \begin{macro}{\tud@ifverlower}
% \changes{v2.03}{2015/01/13}{neu}^^A
% Mit diesem Befehl können abhängig von der gegebenen Kompatibilitätsversion in 
% den Klassen unterschiedliche Einstellungen vorgenommen werden.
%    \begin{macrocode}
\newcommand*\tud@ifverlower[1]{%
  \ifnum\tud@v@comp<\@nameuse{tud@v@#1}\relax%
    \expandafter\@firstoftwo%
  \else%
    \expandafter\@secondoftwo%
  \fi%
}
%    \end{macrocode}
% \end{macro}^^A \tud@ifverlower
%
% \iffalse
%</option>
%</class|(package&fonts)>
%<*class&body>
% \fi
%
% \begin{length}{\chapterheadingvskip}
% Die Länge \cs{chapterheadingvskip} wird aus Gründen der Kompatibilität zu
% älteren Versionen definiert.
%    \begin{macrocode}
%<*book|report>
\newlength\chapterheadingvskip
\let\chapterheadingvskip\headingsvskip
%</book|report>
%    \end{macrocode}
% \end{length}^^A \chapterheadingvskip
% \begin{macro}{\professorship}
% Für die Angabe des Lehrstuhls bzw. der Professur mit kann anstelle von 
% \cs{chair} als Aliasbefehl auch \cs{professorship} genutzt werden.
%    \begin{macrocode}
\tud@deprecated@cmd{professorship}{chair}
%    \end{macrocode}
% \end{macro}^^A \professorship
%
% \iffalse
%<*book|report|article>
% \fi
%
% \begin{macro}{\studentid}
% \begin{macro}{\matriculationid}
% Zur Angabe von Matrikelnummer kann auch \cs{studentid} oder
% \cs{matriculationnumber} genutzt werden.
%    \begin{macrocode}
\tud@deprecated@cmd{studentid}{matriculationnumber}
\tud@deprecated@cmd{matriculationid}{matriculationnumber}
%    \end{macrocode}
% \end{macro}^^A \matriculationid
% \end{macro}^^A \studentid
% \begin{macro}{\enrolmentyear}
% Das Immatrikulationsjahr kann auch mit \cs{enrolmentyear} angegeben werden.
%    \begin{macrocode}
\tud@deprecated@cmd{enrolmentyear}{matriculationyear}
%    \end{macrocode}
% \end{macro}^^A \enrolmentyear
% \begin{macro}{\birthplace}
% Zur Angabe des Geburtsortes kann auch \cs{birthplace} verwendet werden.
%    \begin{macrocode}
\tud@deprecated@cmd{birthplace}{placeofbirth}
%    \end{macrocode}
% \end{macro}^^A \birthplace
% \begin{macro}{\submissiondate}
% Der Befehl \cs{submissiondate} kann als Aliasbefehl für den Standardbefehl 
% \cs{date} zur Datumsangabe genutzt werden.
%    \begin{macrocode}
\tud@deprecated@cmd{submissiondate}{date}
%    \end{macrocode}
% \end{macro}^^A \submissiondate
% \begin{macro}{\oralexaminationdate}
% Für \cs{defensedate} kann als Aliasbefehl auch \cs{oralexaminationdate}
% verwendet werden.
%    \begin{macrocode}
\tud@deprecated@cmd{oralexaminationdate}{defensedate}
%    \end{macrocode}
% \end{macro}^^A \oralexaminationdate
% \begin{macro}{\birthday}
% Der Geburtstag kann auch mit \cs{birthday} angegeben werden.
%    \begin{macrocode}
\tud@deprecated@cmd{birthday}{dateofbirth}
%    \end{macrocode}
% \end{macro}^^A \birthday
% \begin{macro}{\location}
% Für die Angabe des Ortes kann auch \cs{location} genutzt werden.
%    \begin{macrocode}
\tud@deprecated@cmd{location}{place}
%    \end{macrocode}
% \end{macro}^^A \location
%
% \iffalse
%</book|report|article>
% \fi
%
% \begin{macro}{\logofile}
% \begin{macro}{\logofilename}
% Diese beiden Befehle können anstelle von \cs{headlogo} eingesetzt werden.
%    \begin{macrocode}
\tud@deprecated@cmd{logofile}{headlogo}
\tud@deprecated@cmd{logofilename}{headlogo}
%    \end{macrocode}
% \end{macro}^^A \logofilename
% \end{macro}^^A \logofile
% \begin{macro}{\startdate}
% \begin{macro}{\finaldate}
% \begin{macro}{\maturitydate}
% Für das Paket \pkg{tudscrsupervisor} sind auch einige Befehle als veraltet 
% anzusehen.
%    \begin{macrocode}
\AfterPackage{tudscrsupervisor}{%
  \tud@deprecated@cmd{startdate}{issuedate}%
  \tud@deprecated@cmd{finaldate}{duedate}%
  \tud@deprecated@cmd{maturitydate}{duedate}%
}
%    \end{macrocode}
% \end{macro}^^A \maturitydate
% \end{macro}^^A \finaldate
% \end{macro}^^A \startdate
%
% \iffalse
%</class&body>
%<*class|(package&fonts)>
%<*body>
% \fi
%
% Ab der \TUDScript-Version~v2.04 werden einige vertikale Abstände abhängig von 
% der Schriftgröße definiert. Sollte eine frühere Kompatibilitätseinstellung 
% angegeben worden sein, wird dieses verhalten deaktiviert.
%    \begin{macrocode}
%<*class>
\AtEndOfClass%
%</class>
%<*package>
\AtEndOfPackage%
%</package>
{%
  \tud@ifverlower{2.04}{\TUDoptions{vspacing=false}}{}%
}
%    \end{macrocode}
% \begin{macro}{\dinBold}
% \changes{v2.05}{2015/07/13}{neu}^^A
% \begin{macro}{\dinfamily}
% \changes{v2.05}{2015/07/13}{neu}^^A
% Für die explizite Schriftauswahl gibt es ebenfalls zwei ältere Befehle.
%    \begin{macrocode}
%<*package&fonts>
\def\dinBold{\dinbn}
\def\dinfamily{\dinbn}
%</package&fonts>
%    \end{macrocode}
% \end{macro}^^A \dinfamily
% \end{macro}^^A \dinBold
%
% \iffalse
%</body>
%</class|(package&fonts)>
% \fi
%
% \subsection{Das Paket \pkg{tudscrcomp} -- Kompatibilität veralteter Klassen}
%
% Neben den \TUDScript-Klassen gibt es zahlreiche andere \LaTeX-Umsetzungen des 
% \CDs der \TnUD. Um eine Migration von diesen \enquote{veralteten} Klassen
% zu erleichtern, wird das Paket \pkg{tudscrcomp} bereitgestellt. Mit diesem 
% können bei der Verwendung von \TUDScript die meisten Optionen und Befehle 
% genutzt werden, welche durch die Klasse \cls{tudbook} und teilweise vormals 
% durch die Klassen die \TUDScript-Klassen in der Version~v1.0 sowie die 
% Posterklassen \cls{tudmathposter} bzw. \cls{tudposter} bereitgestellt wurden.
%
% \iffalse
%<*package&comp>
%<*body>
% \fi
%
% \ToDo{%
%   Optionen für \cls{tudbook} und \cls{tudmathposter}\val{=Mathematik(A0)}; 
%   überprüfen, ob passende Klasse geladen wurde.%
% }[v2.05]
%
% \begin{macro}{\einrichtung}
% \begin{macro}{\fachrichtung}
% \begin{macro}{\institut}
% \begin{macro}{\professur}
% Es werden Aliasbefehle für die Eingabefelder definiert.
%    \begin{macrocode}
\newcommand*\einrichtung{\faculty}
\newcommand*\fachrichtung{\department}
\newcommand*\institut{\institute}
\newcommand*\professur{\chair}
%    \end{macrocode}
% \end{macro}^^A \professur
% \end{macro}^^A \institut
% \end{macro}^^A \fachrichtung
% \end{macro}^^A \einrichtung
%
% \iffalse
%</body>
%<*option>
% \fi
%
% \subsubsection{Kompatibilität zu \cls{tudbook}}
%
% Die folgenden Optionen und Befehle werden durch die Klasse \cls{tudbook} und
% teilweise durch \TUDScript~v1.0 definiert.
%
% \begin{option}{colortitle}
% \begin{option}{nocolortitle}
% Für farbige Einstellungen wird von \cls{tudbook} die Option \opt{color} 
% definiert. Soll die Titelseite konträr dazu gesetzt werden, muss sich mit den
% Schlüsseln \opt{colortitle} und \opt{nocolortitle} beholfen werden.
%    \begin{macrocode}
\TUD@key{colortitle}[true]{%
  \TUD@set@ifkey{colortitle}{@tempswa}{#1}%
  \ifx\FamilyKeyState\FamilyKeyStateProcessed%
    \if@tempswa%
      \TUDoptions{cdtitle=color}%
    \else%
      \TUDoptions{cdtitle=true}%
    \fi%
  \fi%
}
\TUD@key{nocolortitle}[true]{%
  \TUD@set@ifkey{nocolortitle}{@tempswa}{#1}%
  \ifx\FamilyKeyState\FamilyKeyStateProcessed%
    \if@tempswa%
      \TUDoptions{cdtitle=true}%
    \else%
      \TUDoptions{cdtitle=color}%
    \fi%
  \fi%
}
%    \end{macrocode}
% \end{option}^^A nocolortitle
% \end{option}^^A colortitle
% \begin{option}{ddcfooter}
% Außer der Option \opt{ddc} gibt es bei der alten \cls{tudbook}-Klasse noch
% den Schlüssel \opt{ddcfooter}. Dieser wird auf die Option \opt{ddcfoot} 
% gelegt.
%    \begin{macrocode}
\TUD@key{ddcfooter}[true]{%
  \TUD@set@ifkey{ddcfooter}{@tempswa}{#1}%
  \ifx\FamilyKeyState\FamilyKeyStateProcessed%
    \if@tempswa%
      \TUDoptions{ddcfoot=true}%
    \else%
      \TUDoptions{ddcfoot=false}%
    \fi%
  \fi%
}
%    \end{macrocode}
% \end{option}^^A ddcfooter
%
% \iffalse
%</option>
%<*body>
% \fi
%
% \begin{macro}{\moreauthor}
% \begin{macro}{\submitdate}
% \begin{macro}{\supervisorII}
% \begin{macro}{\supervisedby}
% \begin{macro}{\supervisedIIby}
% \begin{macro}{\submittedon}
% Es werden weitere Aliasbefehle für die Eingabefelder der \cls{tudbook}-Klasse 
% definiert.
%    \begin{macrocode}
\newcommand*\moreauthor{\authormore}
\newcommand*\submitdate{\date}
\newcommand*\supervisorII[1]{%
  \expandafter\gdef\expandafter\@supervisor\expandafter{\@supervisor\and #1}%
}
\newcommand*\supervisedby[1]{\gdef\supervisorname{#1}}
\newcommand*\supervisedIIby[1]{\gdef\supervisorothername{#1}}
\newcommand*\submittedon[1]{\gdef\datetext{#1}}
%    \end{macrocode}
% \end{macro}^^A \submittedon
% \end{macro}^^A \supervisedIIby
% \end{macro}^^A \supervisedby
% \end{macro}^^A \supervisorII
% \end{macro}^^A \submitdate
% \end{macro}^^A \moreauthor
% \begin{macro}{\dissertation}
% Bei der Definition des Typs der Abschlussarbeit mit \cs{dissertation} wird
% die Lokalisierungsvariable \cs{dissertationname} verwendet und die Feldnamen
% angepasst.
%    \begin{macrocode}
\newcommand*\dissertation{%
  \thesis{\dissertationname}%
  \let\supervisorname\refereename%
  \let\supervisorothername\refereeothername%
}
%    \end{macrocode}
% \end{macro}^^A \dissertation
% \begin{environment}{theglossary}
% \begin{macro}{\glossaryname}
% \begin{macro}{\glossitem}
% Eine rudimentäre Umgebung für ein Glossar.
%    \begin{macrocode}
\AtBeginDocument{%
  \ifdef{\theglossary}{}{%
    \providecommand*\glossaryname{Glossar}%
    \newenvironment{theglossary}[1][]{%
      \PackageWarning{tudscrcomp}{%
        Using the environment `theglossary' is not\MessageBreak%
        recommended. You should rather use an appropriate\MessageBreak%
        package such as glossaries%
      }%
      \let\bibname\glossaryname%
      \bib@heading%
      #1%
      \list{}{%
        \setlength\labelsep{\z@}%
        \setlength\labelwidth{\z@}%
        \setlength\itemindent{-\leftmargin}%
      }%
    }{\endlist}%
    \newcommand\glossitem[1]{\item[] #1\par}%
  }%
}
%    \end{macrocode}
% \end{macro}^^A \glossitem
% \end{macro}^^A \glossaryname
% \end{environment}^^A theglossary
% \begin{macro}{\chapterpage}
% \begin{macro}{\if@tud@chapterpage@temp}
% \begin{macro}{\tud@chapterpage@set}
% \begin{macro}{\tud@chapterpage@unset}
% \begin{macro}{\tud@chapterpage@wrn}
% Die alte \cls{tudbook}-Klasse stellt den Befehl \cs{chapterpage} bereit.
% Mit diesem ist es möglich, das Verhalten der Kapitelseiten~-- welches durch
% die Option \opt{chapterpage} gesteuert wird~-- temporär umzuschalten, also
% statt Kapitelseiten lediglich Überschriften zu setzen und umgekehrt. Dies ist
% typographisch durchaus zu hinterfragen, allerdings sollen die neuen Klassen
% möglichst kompatibel zu der alten sein, weshalb diese Funktionalität trotzdem
% implementiert wird. Der Befehl \cs{chapterpage} setzt den globalen Schalter
% \cs{if@tud@chapterpage@temp}. Der Befehl \cs{tud@chapterpage@set} setzt für 
% Kapitel das komplementäre Verhalten zur eigentlich gewählten
% \opt{chapterpage}"=Option. Nach dem Setzen der entsprechenden Überschrift
% wird \cs{tud@chapterpage@set} nochmals aufgerufen, das Verhalten auf den
% ursprünglichen Zustand geschaltet und der globale Schalter
% \cs{if@tud@chapterpage@temp} zurückgesetzt.
%    \begin{macrocode}
\newif\if@tud@chapterpage@temp
\newcommand*\chapterpage{%
  \global\@tud@chapterpage@temptrue%
  \tud@chapterpage@wrn%
}
\newcommand*\tud@chapterpage@set[1][]{%
  \if@tud@chapterpage@temp%
    \if@tud@chapterpage%
      \TUDoptions{chapterpage=false}%
    \else%
      \TUDoptions{chapterpage=true}%
    \fi%
  \fi%
}
\newcommand*\tud@chapterpage@unset[1][]{%
  \tud@chapterpage@set%
  \global\@tud@chapterpage@tempfalse%
}
%    \end{macrocode}
% Da wie bereits beschrieben das Vorgehen äußerst fragwürdig ist, wird bei der
% Verwendung von \cs{chapterpage} zumindest einmalig eine Warnung ausgegeben.
%    \begin{macrocode}
\newcommand*\tud@chapterpage@wrn{%
  \PackageWarning{tudscrcomp}{%
    The command \string\chapterpage\space is not\MessageBreak%
    recommended. You should use the same style for\MessageBreak%
    chapters throughout the document%
  }%
  \global\let\tud@chapterpage@wrn\relax%
}
%    \end{macrocode}
% \end{macro}^^A \tud@chapterpage@wrn
% \end{macro}^^A \tud@chapterpage@unset
% \end{macro}^^A \tud@chapterpage@set
% \end{macro}^^A \if@tud@chapterpage@temp
% \end{macro}^^A \chapterpage
% \begin{macro}{\tud@chapter}
% \begin{macro}{\tud@schapter}
% \begin{macro}{\tud@addchap}
% \begin{macro}{\tud@saddchap}
% Hier erfolgt die notwendige Anpassungen der internen Gliederungsbefehle.
%    \begin{macrocode}
\AtEndPreamble{%
  \if@chapter%
    \pretocmd{\tud@chapter}{\tud@chapterpage@set}%
      {}{\tud@patch@wrn{tud@nchapter}}%
    \apptocmd{\tud@chapter}{\tud@chapterpage@unset}%
      {}{\tud@patch@wrn{tud@nchapter}}%
    \pretocmd{\tud@schapter}{\tud@chapterpage@set}%
      {}{\tud@patch@wrn{tud@schapter}}%
    \apptocmd{\tud@schapter}{\tud@chapterpage@unset}%
      {}{\tud@patch@wrn{tud@schapter}}%
    \pretocmd{\tud@addchap}{\tud@chapterpage@set}%
      {}{\tud@patch@wrn{tud@naddchap}}%
    \apptocmd{\tud@addchap}{\tud@chapterpage@unset}%
      {}{\tud@patch@wrn{tud@naddchap}}%
    \pretocmd{\tud@saddchap}{\tud@chapterpage@set}%
      {}{\tud@patch@wrn{tud@saddchap}}%
    \apptocmd{\tud@saddchap}{\tud@chapterpage@unset}%
      {}{\tud@patch@wrn{tud@saddchap}}%
  \fi%
}
%    \end{macrocode}
% \end{macro}^^A \tud@saddchap
% \end{macro}^^A \tud@addchap
% \end{macro}^^A \tud@schapter
% \end{macro}^^A \tud@chapter
%
% \subsubsection{Kompatibilität zu \cls{tudmathposter}}
%
% \ToDo{%
% Folgende Befehle sollten für maximale Kompatibilität unterstützt werden:
% \begin{itemize}
%   \item Umgebung \env{farbtabellen} mit \cs{grautabelle} und \cs{blautabelle}
%   \item \cs{telefon}, \cs{fax}, \cs{homepage}, \cs{email}
%   \item \cs{fusszeile} \cs{footcolumn} Die linke Spalte enthält Hochschule,
%      Einrichtung, Fachrichtung, Institut und Professur. Die rechte Spalte ist
%      frei wählbar, und kann mit den Variablen \cs{author}, \cs{telefon}, 
%      \cs{email} und \cs{homepage} oder mit einem frei gewählten Absatz 
%      (\cs{footcolumn2}) gefüllt werden.
%   \item \cs{zweitlogo(file)}, \cs{drittlogo(file)}, \cs{institutslogo(file)}
%   \item \cs{maketitle}
%   \item \cs{schnittrand} mit \pkg{crop}?
%   \item Abstände der Gliederungsebenen?
%   \item Umgebungen \env{tablehere} und \env{figurehere}
%   \item \cs{multicolsep}
%   \item Optionen \opt{Mathematik} und \opt{MathematikA0}
%   \item \cs{pagestyle}\val{\{tudposter\}}
%   \item Option \opt{bluebg} implementieren
%   \item Gliederungsbefehle \cs{topsection}, \cs{topsubsection}, 
%     \cs{topcentersection}, \cs{centersubsection}, 
%     \cs{topcentersubsection}, \cs{centersection}
% \end{itemize}%
% Testen mit \file{beispiel-utf8-serifmath.tex}
% }[v2.05]
%
% \iffalse
%</body>
%</package&comp>
% \fi
%
% \subsection{\pkg{fix-tudscrfonts} -- Schriften des \CDs für alte TUD-Klassen}
%
% Das Paket \pkg{fix-tudscrfonts} bietet die \TUDScript-Schriftfamilien für die
% veralteten Klassen von Klaus Bergmann sowie die Posterklassen an. 
%
% \iffalse
%<*package&fontsfix>
%<*option>
% \fi
%
% Sämtliche Optionen werden an das Paket \pkg{tudscrfonts} durchgereicht, 
% welches automatisch am Ende der Präambel geladen wird, falls dies zuvor durch 
% den Anwender geschieht. Für die Klassen von Klaus Bergmann, \TUDScript in der 
% Version~v1.0 sowie die alternativen Posterklassen bleiben die vertikalen 
% Abstände schriftgrößenunabhängig. Außerdem wird für die fetten Schriften ein
% stärkerer Schriftschnitt genutzt.
%    \begin{macrocode}
\PassOptionsToPackage{vspacing=false,cdfont=ultrabold}{tudscrfonts}
\DeclareOption*{\PassOptionsToPackage{\CurrentOption}{tudscrfonts}}
\ProcessOptions\relax
\AtEndPreamble{%
  \tud@fontsfix@maybe@err%
  \RequirePackage{tudscrfonts}%
}
%    \end{macrocode}
% \begin{macro}{\tud@fontsfix@maybe@err}
% \changes{v2.05}{2016/01/02}{neu}^^A
% Je nachdem, ob das Paket zu spät oder mit einer falschen Klasse geladen 
% wurde, wird eine Warnung ausgegeben.
%    \begin{macrocode}
\ifx\usepackage\RequirePackage
  \newcommand*\tud@fontsfix@maybe@err{%
    \PackageError{fix-tudscrfonts}{Package too late}{%
      You must load `fix-tudscrfonts' before\MessageBreak%
      any document class with `\string\RequirePackage'%
    }%
  }%
\else
  \newcommand*\tud@fontsfix@maybe@err{%
    \PackageError{fix-tudscrfonts}{Unsupported class found}{%
      You have to use `fix-tudscrfonts' only\MessageBreak%
      with supported classes. These are: `tudbook',\MessageBreak%
      `tudbeamer', `tudfax', `tudform', `tudhaus',\MessageBreak%
      `tudletter', `tudposter' and `tudmathposter',\MessageBreak%
      as well as `tudscrbookold', `tudscrreprtold'\MessageBreak%
      and `tudscrartclold'. Otherwise it's adequate\MessageBreak%
      to use package `tudscrfonts' as usual%
    }%
  }%
\fi
%    \end{macrocode}
% \end{macro}^^A \tud@fontsfix@maybe@err
%
% \iffalse
%</option>
%<*body>
% \fi
%
% \subsubsection{Patches für die unterstützten Klassen}
%
% Für alle unterstützten Klassen sind vor und nach dem Laden einige Anpassungen 
% notwendig.
%
% \begin{macro}{\tud@fontsfix@BeforeClass}
% \changes{v2.05}{2016/01/02}{neu}^^A
% \begin{macro}{\tud@fontsfix@AfterClass}
% \changes{v2.05}{2016/01/02}{neu}^^A
% \begin{macro}{\univBoldHead}
% \begin{macro}{\univLightHead}
% \begin{macro}{\univBoldXV}
% \begin{macro}{\univLightObliqueXV}
% \begin{macro}{\univLightXV}
% \begin{macro}{\univBoldXI}
% \begin{macro}{\univLightObliqueXI}
% \begin{macro}{\univLightXI}
% \begin{macro}{\univBoldIX}
% \begin{macro}{\univLightObliqueIX}
% \begin{macro}{\univLightIX}
% \begin{macro}{\univBoldVII}
% \begin{macro}{\univLightObliqueVII}
% \begin{macro}{\univLightVII}
% Mit diesen beiden Befehlen werden vor dem Laden der alten Klassen einige
% Befehle gesichert und die benötigten Optionen gesetzt. Nach dem Laden der 
% jeweiligen Klasse werden die gesicherten Makrodefinitionen wiederhergestellt 
% und einige Schriftdefinitionen angepasst. Bei den alten TUD-Klassen betrifft 
% dies in erster Linie die Mathematikschriften sowie die in den Klassen
% definierten, fixen Schriften.
%    \begin{macrocode}
\newcommand*\tud@fontsfix@BeforeClass[2]{%
  \BeforeClass{#1}{%
    \let\tud@fontsfix@maybe@err\relax%
    \tud@cmd@store{DeclareFixedFont}%
    \renewcommand*\DeclareFixedFont[6]{}%
    \tud@cmd@store{rmdefault}%
    \tud@cmd@store{sfdefault}%
    \tud@cmd@store{ttdefault}%
    \tud@cmd@store{bfdefault}%
    \tud@cmd@store{mddefault}%
    \tud@cmd@store{itdefault}%
    \tud@cmd@store{sldefault}%
    \tud@cmd@store{scdefault}%
    \tud@cmd@store{updefault}%
    \tud@cmd@store{rmfamily}%
    \tud@cmd@store{sffamily}%
    \tud@cmd@store{ttfamily}%
    \tud@cmd@store{familydefault}%
    \tud@cmd@store{seriesdefault}%
    \tud@cmd@store{shapedefault}%
    \tud@cmd@store{normalfont}%
    \let\normalfont\relax%
    #2%
    \PassOptionsToClass{serifmath}{#1}%
  }%
}
\newcommand*\tud@fontsfix@AfterClass[2]{%
  \AfterClass{#1}{%
    \tud@cmd@restore{DeclareFixedFont}%
    \tud@cmd@restore{rmdefault}%
    \tud@cmd@restore{sfdefault}%
    \tud@cmd@restore{ttdefault}%
    \tud@cmd@restore{bfdefault}%
    \tud@cmd@restore{mddefault}%
    \tud@cmd@restore{itdefault}%
    \tud@cmd@restore{sldefault}%
    \tud@cmd@restore{scdefault}%
    \tud@cmd@restore{updefault}%
    \tud@cmd@restore{rmfamily}%
    \tud@cmd@restore{sffamily}%
    \tud@cmd@restore{ttfamily}%
    \tud@cmd@restore{familydefault}%
    \tud@cmd@restore{seriesdefault}%
    \tud@cmd@restore{shapedefault}%
    \tud@cmd@restore{normalfont}%
    \AfterPackage*{tudscrfonts}{%
      #2%
      \AtBeginDocument{%
        \newcommand*\univLightVII{}%
        \newcommand*\univLightObliqueVII{}%
        \newcommand*\univBoldVII{}%
        \newcommand*\univLightIX{}%
        \newcommand*\univLightObliqueIX{}%
        \newcommand*\univBoldIX{}%
        \newcommand*\univLightXI{}%
        \newcommand*\univLightObliqueXI{}%
        \newcommand*\univBoldXI{}%
        \newcommand*\univLightXV{}%
        \newcommand*\univLightObliqueXV{}%
        \newcommand*\univBoldXV{}%
        \newcommand*\univLightHead{}%
        \newcommand*\univBoldHead{}%
        \providecommand*\tud@head@fntsz{9}%
        \if@tud@univers%
          \if@tud@fontspec%
            \DeclareFixedFont{\univLightVII}{\encodingdefault}%
              {UniversLight}{m}{n}{7}%
            \DeclareFixedFont{\univLightObliqueVII}{\encodingdefault}%
              {UniversLightOblique}{m}{n}{7}%
            \DeclareFixedFont{\univBoldVII}{\encodingdefault}%
              {UniversBold}{m}{n}{7}%
            \DeclareFixedFont{\univLightIX}{\encodingdefault}%
              {UniversLight}{m}{n}{9}%
            \DeclareFixedFont{\univLightObliqueIX}{\encodingdefault}%
              {UniversLightOblique}{m}{n}{9}%
            \DeclareFixedFont{\univBoldIX}{\encodingdefault}%
              {UniversBold}{m}{n}{9}%
            \DeclareFixedFont{\univLightXI}{\encodingdefault}%
              {UniversLight}{m}{n}{11}%
            \DeclareFixedFont{\univLightObliqueXI}{\encodingdefault}%
              {UniversLightOblique}{m}{n}{11}%
            \DeclareFixedFont{\univBoldXI}{\encodingdefault}%
              {UniversBold}{m}{n}{11}%
            \DeclareFixedFont{\univLightXV}{\encodingdefault}%
              {UniversLight}{m}{n}{15}%
            \DeclareFixedFont{\univLightObliqueXV}{\encodingdefault}%
              {UniversLightOblique}{m}{n}{15}%
            \DeclareFixedFont{\univBoldXV}{\encodingdefault}%
              {UniversBold}{m}{n}{15}%
            \DeclareFixedFont{\univLightHead}{\encodingdefault}%
              {UniversLight}{m}{n}{\tud@head@fntsz}%
            \DeclareFixedFont{\univBoldHead}{\encodingdefault}%
              {UniversBold}{m}{n}{\tud@head@fntsz}%
          \else%
            \DeclareFixedFont{\univLightVII}{\encodingdefault}%
              {\tud@univers}{l}{n}{7}%
            \DeclareFixedFont{\univLightObliqueVII}{\encodingdefault}%
              {\tud@univers}{l}{sl}{7}%
            \DeclareFixedFont{\univBoldVII}{\encodingdefault}%
              {\tud@univers}{b}{n}{7}%
            \DeclareFixedFont{\univLightIX}{\encodingdefault}%
              {\tud@univers}{l}{n}{9}%
            \DeclareFixedFont{\univLightObliqueIX}{\encodingdefault}%
              {\tud@univers}{l}{sl}{9}%
            \DeclareFixedFont{\univBoldIX}{\encodingdefault}%
              {\tud@univers}{b}{n}{9}%
            \DeclareFixedFont{\univLightXI}{\encodingdefault}%
              {\tud@univers}{l}{n}{11}%
            \DeclareFixedFont{\univLightObliqueXI}{\encodingdefault}%
              {\tud@univers}{l}{sl}{11}%
            \DeclareFixedFont{\univBoldXI}{\encodingdefault}%
              {\tud@univers}{b}{n}{11}%
            \DeclareFixedFont{\univLightXV}{\encodingdefault}%
              {\tud@univers}{l}{n}{15}%
            \DeclareFixedFont{\univLightObliqueXV}{\encodingdefault}%
              {\tud@univers}{l}{sl}{15}%
            \DeclareFixedFont{\univBoldXV}{\encodingdefault}%
              {\tud@univers}{b}{n}{15}%
            \DeclareFixedFont{\univLightHead}{\encodingdefault}%
              {\tud@univers}{l}{n}{\tud@head@fntsz}%
            \DeclareFixedFont{\univBoldHead}{\encodingdefault}%
              {\tud@univers}{b}{n}{\tud@head@fntsz}%
          \fi%
        \else%
          \DeclareFixedFont{\univLightVII}{\encodingdefault}%
            {\sfdefault}{\mddefault}{\updefault}{7}%
          \DeclareFixedFont{\univLightObliqueVII}{\encodingdefault}%
            {\sfdefault}{\mddefault}{\sldefault}{7}%
          \DeclareFixedFont{\univBoldVII}{\encodingdefault}%
            {\sfdefault}{\bfdefault}{\updefault}{7}%
          \DeclareFixedFont{\univLightIX}{\encodingdefault}%
            {\sfdefault}{\mddefault}{\updefault}{9}%
          \DeclareFixedFont{\univLightObliqueIX}{\encodingdefault}%
            {\sfdefault}{\mddefault}{\sldefault}{9}%
          \DeclareFixedFont{\univBoldIX}{\encodingdefault}%
            {\sfdefault}{\bfdefault}{\updefault}{9}%
          \DeclareFixedFont{\univLightXI}{\encodingdefault}%
            {\sfdefault}{\mddefault}{\updefault}{11}%
          \DeclareFixedFont{\univLightObliqueXI}{\encodingdefault}%
            {\sfdefault}{\mddefault}{\sldefault}{11}%
          \DeclareFixedFont{\univBoldXI}{\encodingdefault}%
            {\sfdefault}{\bfdefault}{\updefault}{11}%
          \DeclareFixedFont{\univLightXV}{\encodingdefault}%
            {\sfdefault}{\mddefault}{\updefault}{15}%
          \DeclareFixedFont{\univLightObliqueXV}{\encodingdefault}%
            {\sfdefault}{\mddefault}{\sldefault}{15}%
          \DeclareFixedFont{\univBoldXV}{\encodingdefault}%
            {\sfdefault}{\bfdefault}{\updefault}{15}%
          \DeclareFixedFont{\univLightHead}{\encodingdefault}%
            {\sfdefault}{\mddefault}{\updefault}{\tud@head@fntsz}%
          \DeclareFixedFont{\univBoldHead}{\encodingdefault}%
            {\sfdefault}{\bfdefault}{\updefault}{\tud@head@fntsz}%
        \fi%
      }%
    }%
  }%
}
%    \end{macrocode}
% \end{macro}^^A \univLightVII
% \end{macro}^^A \univLightObliqueVII
% \end{macro}^^A \univBoldVII
% \end{macro}^^A \univLightIX
% \end{macro}^^A \univLightObliqueIX
% \end{macro}^^A \univBoldIX
% \end{macro}^^A \univLightXI
% \end{macro}^^A \univLightObliqueXI
% \end{macro}^^A \univBoldXI
% \end{macro}^^A \univLightXV
% \end{macro}^^A \univLightObliqueXV
% \end{macro}^^A \univBoldXV
% \end{macro}^^A \univLightHead
% \end{macro}^^A \univBoldHead
% \end{macro}^^A \tud@fontsfix@AfterClass
% \end{macro}^^A \tud@fontsfix@BeforeClass
%
% Nach der Definition der notwendigen Befehle erfolgt nun die Umsetzung für 
% alle unterstützten Klassen.
%    \begin{macrocode}
\tud@fontsfix@BeforeClass{tudletter}{}
\tud@fontsfix@AfterClass{tudletter}{}
\tud@fontsfix@BeforeClass{tudfax}{}
\tud@fontsfix@AfterClass{tudfax}{}
\tud@fontsfix@BeforeClass{tudform}{}
\tud@fontsfix@AfterClass{tudform}{}
\tud@fontsfix@BeforeClass{tudhaus}{}
\tud@fontsfix@AfterClass{tudhaus}{}
\tud@fontsfix@BeforeClass{tudposter}{}
\tud@fontsfix@AfterClass{tudposter}{}
\tud@fontsfix@BeforeClass{tudbeamer}{}
\tud@fontsfix@AfterClass{tudbeamer}{}
%    \end{macrocode}
%
% Wird das Paket \pkg{fix-tudscrfonts} mit den alten TUD-Klassen \cls{tudbook}, 
% \cls{tudmathposter} etc. oder \TUDScript in der Version~v1.0 verwendet,
% müssen einige Kompatibilitätseinstellungen vorgenmommen werden. Hierfür sind
% wenige Patches notwendig. 
%
% \subsubsection{Spezielle Patches für die Klasse \cls{tudbook}}
%
% Die notwendigen Anpassungen der Klasse \cls{tudbook} betreffen die
% Schriftauswahl bei Überschriften.
%
% \begin{macro}{\@makechapterhead}
% \begin{macro}{\@makeschapterhead}
% \begin{macro}{\section}
% \begin{macro}{\showtitle}
% Die Überschriften sollen abhängig von der Option \opt{cdfont} und nicht immer 
% zwingend in \DIN gesetzt werden.
%    \begin{macrocode}
\tud@fontsfix@BeforeClass{tudbook}{}
\tud@fontsfix@AfterClass{tudbook}{%
  \patchcmd{\@makechapterhead}%
    {\dinBold\Huge\bfseries}%
    {\tud@disposition@font[\bfseries]\Huge}%
    {}{\tud@patch@wrn{@makechapterhead}}%
  \patchcmd{\@makechapterhead}%
    {\MakeUppercase}%
    {\tud@makeuppercase}%
    {}{\tud@patch@wrn{@makechapterhead}}%
  \patchcmd{\@makeschapterhead}%
    {\dinBold\Huge\bfseries}%
    {\tud@disposition@font[\bfseries]\Huge}%
    {}{\tud@patch@wrn{@makeschapterhead}}%
  \patchcmd{\@makeschapterhead}%
    {\MakeUppercase}%
    {\tud@makeuppercase}%
    {}{\tud@patch@wrn{@makeschapterhead}}%
  \patchcmd{\section}%
    {\dinBold\Large\bfseries\MakeUppercase}%
    {\tud@disposition@font[\bfseries]\Large\tud@makeuppercase}%
    {}{\tud@patch@wrn{section}}%
  \patchcmd{\showtitle}%
    {\dinBold\Huge\bfseries\MakeUppercase}%
    {\tud@disposition@font[\bfseries]\Huge\tud@makeuppercase}%
    {}{\tud@patch@wrn{showtitle}}%
}
%    \end{macrocode}
% \end{macro}^^A \showtitle
% \end{macro}^^A \section
% \end{macro}^^A \@makeschapterhead
% \end{macro}^^A \@makechapterhead
%
% \subsubsection{Spezielle Patches für die Klasse \cls{tudmathposter}}
%
% Auch für die Posterklasse \cls{tudmathposter} sowie das Paket \pkg{tudfonts} 
% sind kleine Anpassungen notwendig.
%
% \begin{KOMAfont}{title}
% \changes{v2.04}{2015/04/24}{\cls{tudmathposter} wird unterstützt}^^A
% \begin{macro}{\subtitlefont}
% \changes{v2.04}{2015/04/24}{\cls{tudmathposter} wird unterstützt}^^A
% \begin{macro}{\sectionfont}
% \changes{v2.04}{2015/04/24}{\cls{tudmathposter} wird unterstützt}^^A
% \begin{macro}{\subsectionfont}
% \changes{v2.04}{2015/04/24}{\cls{tudmathposter} wird unterstützt}^^A
% \begin{macro}{\if@sansmath}
% \changes{v2.04}{2015/04/24}{\cls{tudmathposter} wird unterstützt}^^A
% \begin{macro}{\tudfont}
% \changes{v2.04}{2015/04/24}{\cls{tudmathposter} wird unterstützt}^^A
% Es werden einige Einstellungen für die Überschriften angepasst.
%    \begin{macrocode}
\tud@fontsfix@BeforeClass{tudmathposter}{}
\tud@fontsfix@AfterClass{tudmathposter}{%
  \apptocmd{\maketitle}{\vspace{-2em}}{}{\tud@patch@wrn{maketitle}}%
  \setkomafont{title}{\tud@disposition@font[\bfseries]\Huge}%
  \renewcommand*\subtitlefont{%
    \ifstr{\mddefault}{l}{%
      \tud@disposition@font[\fontseries{m}]%
    }{%
      \tud@disposition@font[\fontseries{bx}]%
    }%
    \huge%
  }%
  \renewcommand\preprocesstitle[1]{\tud@makeuppercase{#1}}%
  \if@mathevorgabe%
    \undef\sectionfont%
    \undef\subsectionfont%
  \fi%
  \newcommand*\sectionfont{\tud@disposition@font[\bfseries]\LARGE}%
  \newcommand*\subsectionfont{\sectionfont\large}%
%    \end{macrocode}
% Für die Schriftstärke in der Kopfzeile muss der Seitenstil angepasst und  
% anschließend nochmals akiviert werden.
%    \begin{macrocode}
  \patchcmd{\ps@tudposter}%
    {\textbf{\@einrichtung}}%
    {\textbf{\fontseries{b}\selectfont\@einrichtung}}%
    {}{\tud@patch@wrn{ps@tudposter}}%
  \pagestyle{tudposter}%
}
%    \end{macrocode}
% Die Klasse \cls{tudmathposter} lädt für die Einstellungen der Schriften das 
% Paket \pkg{tudfonts}. Wenn \pkg{tudscrfonts} zum Einsatz kommen soll, ist das 
% unerwünscht, weshalb das Laden hiermit unterbunden wird.
%    \begin{macrocode}
\PreventPackageFromLoading[%
  \RequirePackage{amsmath}%
  \RequirePackage{amsfonts}%
  \let\DeclareTudMathSizes\@gobblefour%
  \newif\if@tudfonts@nodin%
]{tudfonts}
%    \end{macrocode}
% \end{macro}^^A \tudfont
% \end{macro}^^A \if@sansmath
% \end{macro}^^A \subsectionfont
% \end{macro}^^A \sectionfont
% \end{macro}^^A \subtitlefont
% \end{KOMAfont}^^A title
%
% \subsubsection{Patches für die \TUDScript-Klassen~v1.0}
%
% Für die Verwendung des Paketes \pkg{fix-tudscrfonts} mit \TUDScript in der
% Version~v1.0 sind ebenfalls einige Anpassungen notwendig. Es muss Sorge dafür
% getragen werden, dass einige veraltete Befehle aktualisiert werden. Hierfür
% müssen diese \enquote{unschädlich} gemacht werden, noch bevor das Paket
% selbst in Erscheinung tritt.
%    \begin{macrocode}
\tud@fontsfix@BeforeClass{tudscrbookold}{}
\tud@fontsfix@BeforeClass{tudscrreprtold}{}
\tud@fontsfix@BeforeClass{tudscrartclold}{}
\tud@fontsfix@AfterClass{tudscrbookold}{\let\@setfonts\relax}
\tud@fontsfix@AfterClass{tudscrreprtold}{\let\@setfonts\relax}
\tud@fontsfix@AfterClass{tudscrartclold}{\let\@setfonts\relax}
\BeforePackage{tudscrfonts}{%
  \undef\univln%
  \undef\univrn%
  \undef\univbn%
  \undef\univxn%
  \undef\univls%
  \undef\univrs%
  \undef\univbs%
  \undef\univxs%
  \undef\dinbn%
  \undef\textuln%
  \undef\texturn%
  \undef\textubn%
  \undef\textuxn%
  \undef\textuls%
  \undef\texturs%
  \undef\textubs%
  \undef\textuxs%
  \undef\textdbn%
}
%    \end{macrocode}
%
% \iffalse
%</body>
%</package&fontsfix>
% \fi
%
% \Finale
%
\endinput
