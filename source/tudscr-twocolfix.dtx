% \CheckSum{119}
% \iffalse meta-comment
% ======================================================================
%
% Das Corporate Design der TU Dresden auf Basis der KOMA-Script-Klassen
%
% ======================================================================
% This work may be distributed and/or modified under the conditions of
% the LaTeX Project Public License, version 1.3c of the license.
% The latest version of this license is in
%     http://www.latex-project.org/lppl.txt
% and version 1.3c or later is part of all distributions of LaTeX
% version 2005/12/01 or later and of this work.
% This work has the LPPL maintenance status "author-maintained".
% The current maintainer and author of this work is Falk Hanisch.
% ----------------------------------------------------------------------
% Dieses Werk darf nach den Bedingungen der LaTeX Project Public Lizenz,
% Version 1.3c, verteilt und/oder veraendert werden.
% Die neuste Version dieser Lizenz ist
%     http://www.latex-project.org/lppl.txt
% und Version 1.3c ist Teil aller Verteilungen von LaTeX
% Version 2005/12/01 oder spaeter und dieses Werks.
% Dieses Werk hat den LPPL-Verwaltungs-Status "author-maintained"
% (allein durch den Autor verwaltet).
% Der aktuelle Verwalter und Autor dieses Werkes ist Falk Hanisch.
% ======================================================================
% \fi
%
% \CharacterTable
%  {Upper-case    \A\B\C\D\E\F\G\H\I\J\K\L\M\N\O\P\Q\R\S\T\U\V\W\X\Y\Z
%   Lower-case    \a\b\c\d\e\f\g\h\i\j\k\l\m\n\o\p\q\r\s\t\u\v\w\x\y\z
%   Digits        \0\1\2\3\4\5\6\7\8\9
%   Exclamation   \!     Double quote  \"     Hash (number) \#
%   Dollar        \$     Percent       \%     Ampersand     \&
%   Acute accent  \'     Left paren    \(     Right paren   \)
%   Asterisk      \*     Plus          \+     Comma         \,
%   Minus         \-     Point         \.     Solidus       \/
%   Colon         \:     Semicolon     \;     Less than     \<
%   Equals        \=     Greater than  \>     Question mark \?
%   Commercial at \@     Left bracket  \[     Backslash     \\
%   Right bracket \]     Circumflex    \^     Underscore    \_
%   Grave accent  \`     Left brace    \{     Vertical bar  \|
%   Right brace   \}     Tilde         \~}
%
% \iffalse
%%% From File: tudscr-twocolfix.dtx
%<*driver>
\ifx\ProvidesFile\undefined\def\ProvidesFile#1[#2]{}\fi
\ProvidesFile{tudscr-twocolfix.dtx}[%
  2014/10/10 v2.02 TUD-KOMA-Script\space%
%</driver>
%<package>\NeedsTeXFormat{LaTeX2e}[2011/06/27]
%<package>\ProvidesPackage{twocolfix}[%
%<*driver|package>
%!TUDVersion
%<package>  package
  (twocolumn layout bugfix)%
]
%</driver|package>
%<*driver>
\RequirePackage[ngerman=ngerman-x-latest]{hyphsubst}
\documentclass[english,ngerman]{tudscrdoc}
\usepackage{selinput}\SelectInputMappings{adieresis={ä},germandbls={ß}}
\usepackage[T1]{fontenc}
\usepackage{babel}
\IfFileExists{tudscrfonts.sty}{\usepackage{tudscrfonts}}{}
\KOMAoptions{parskip=half-}
\CodelineIndex
\RecordChanges
\GetFileInfo{tudscr-twocolfix.dtx}
\begin{document}
  \maketitle
  \DocInput{\filename}
\end{document}
%</driver>
% \fi
%
% \selectlanguage{ngerman}
%
% \changes{v2.02}{2014/07/10}{Versionsnummer angepasst}%^^A
%
% \section{Bugfix für den zweispaltigen Satz}
%
% Der \LaTeXe-Kernel enthält einen Fehler, der Kapitelüberschriften im
% zweispaltigen Layout höher setzt, als im einspaltigen. Der Fehler ist zwar
% schon länger bekannt,\footnote{\url{%
% http://www.latex-project.org/cgi-bin/ltxbugs2html?pr=latex/3126}} allerdings
% noch nicht in \pkg{ltxfix2e} übernommen worden. Das Paket \pkg{twocolfix}
% soll das Problem beheben. Eine Integration dieses Bugfixes in \KOMAScript{}
% wurde bereits bei Markus Kohm angefragt,\footnote{\url{%
% http://www.komascript.de/node/1681}} jedoch von ihm bis jetzt nicht weiter
% verfolgt.
%
% \StopEventually{\PrintIndex\PrintChanges}
%
% \iffalse
%<*package>
% \fi
%
% \subsection{Das Paket \pkg{twocolfix}}
%
% Es wird der fehlerhafte Befehl aus dem \LaTeXe-Kernel neu definiert. Die
% \KOMAScript-Klassen selbst definieren \cs{@topnewpage} um und sichern vorher
% das Original in \cs{scr@topnewpage}. Daher wird der neue Befehl erst temporär
% definiert. 
%    \begin{macrocode}
\long\def \@tempa [#1]{%
  \@nodocument%
  \@next\@currbox\@freelist{}{}%
  \global \setbox\@currbox%
    \vbox {%
      \break%
      \prevdepth\z@%
      \begingroup%
      \normalcolor%
      \hsize\textwidth%
%    \end{macrocode}
% Damit der Inhalt des optionale Argumentes mit den gleichen Absatzeinstellungen
% gesetzt wird, darf der Befehl \cs{@parboxrestore} nicht aufgerufen werden.
% Alternativ müsste \KOMAScript{} eine Option bereitstellen, mit der sich die
% Einstellungen für Absätze wiederherstellen lassen.
%    \begin{macrocode}
%      \@parboxrestore
%      \KOMAoptions{parskip=last}
      \col@number \@ne%
      \ignorespaces #1\par%
      \ifdim\parskip>\z@\null\fi%
      \vskip -\dbltextfloatsep%
      \endgroup%
      \ifdim\parskip>\z@\vskip\parskip\else\null\fi%
      \vskip -\topskip%
  }%
  \begingroup%
    \splitmaxdepth\maxdepth \splittopskip\topskip%
    \setbox\@tempboxa \vsplit\@currbox to\z@%
  \endgroup%
  \ifdim \ht\@currbox>\textheight%
    \ht\@currbox \textheight%
  \fi%
  \global \count\@currbox \tw@%
  \@tempdima -\ht\@currbox%
  \advance \@tempdima -\dbltextfloatsep%
  \global \advance \@colht \@tempdima%
  \ifx \@dbltoplist \@empty%
  \else%
    \@latexerr{Float(s) lost}\@ehb%
    \let \@dbltoplist \@empty%
  \fi%
  \@cons \@dbltoplist \@currbox%
  \global \@dbltopnum \m@ne%
  \ifdim \@colht<2.5\baselineskip%
    \@latex@warning@no@line {Optional argument of \noexpand\twocolumn
                too tall on page \thepage}%
    \@emptycol%
    \if@firstcolumn%
    \else%
      \@emptycol%
    \fi%
  \else%
    \global \vsize \@colht%
    \global \@colroom \@colht%
    \@floatplacement%
  \fi%
}
%    \end{macrocode}
% Nachdem der Befehl zuerst temporär definert wurde, wird nun abhängig von der
% aktiven Klasse der notwendige Befehl mit der neuen Definition überschrieben.
%    \begin{macrocode}
\@ifundefined{scr@topnewpage}
  {\let\@topnewpage\@tempa}
  {\let\scr@topnewpage\@tempa}
%    \end{macrocode}
%
% \iffalse
%</package>
% \fi
%
% \Finale
%
\endinput
