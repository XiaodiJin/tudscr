% \CheckSum{130}
% \iffalse meta-comment
% ======================================================================
%
% Das Corporate Design der TU Dresden auf Basis der KOMA-Script-Klassen
%
% ======================================================================
% This work may be distributed and/or modified under the conditions of
% the LaTeX Project Public License, version 1.3c of the license.
% The latest version of this license is in
%     http://www.latex-project.org/lppl.txt
% and version 1.3c or later is part of all distributions of LaTeX
% version 2005/12/01 or later and of this work.
% This work has the LPPL maintenance status "author-maintained".
% The current maintainer and author of this work is Falk Hanisch.
% ----------------------------------------------------------------------
% Dieses Werk darf nach den Bedingungen der LaTeX Project Public Lizenz,
% Version 1.3c, verteilt und/oder veraendert werden.
% Die neuste Version dieser Lizenz ist
%     http://www.latex-project.org/lppl.txt
% und Version 1.3c ist Teil aller Verteilungen von LaTeX
% Version 2005/12/01 oder spaeter und dieses Werks.
% Dieses Werk hat den LPPL-Verwaltungs-Status "author-maintained"
% (allein durch den Autor verwaltet).
% Der aktuelle Verwalter und Autor dieses Werkes ist Falk Hanisch.
% ======================================================================
% \fi
%
% \CharacterTable
%  {Upper-case    \A\B\C\D\E\F\G\H\I\J\K\L\M\N\O\P\Q\R\S\T\U\V\W\X\Y\Z
%   Lower-case    \a\b\c\d\e\f\g\h\i\j\k\l\m\n\o\p\q\r\s\t\u\v\w\x\y\z
%   Digits        \0\1\2\3\4\5\6\7\8\9
%   Exclamation   \!     Double quote  \"     Hash (number) \#
%   Dollar        \$     Percent       \%     Ampersand     \&
%   Acute accent  \'     Left paren    \(     Right paren   \)
%   Asterisk      \*     Plus          \+     Comma         \,
%   Minus         \-     Point         \.     Solidus       \/
%   Colon         \:     Semicolon     \;     Less than     \<
%   Equals        \=     Greater than  \>     Question mark \?
%   Commercial at \@     Left bracket  \[     Backslash     \\
%   Right bracket \]     Circumflex    \^     Underscore    \_
%   Grave accent  \`     Left brace    \{     Vertical bar  \|
%   Right brace   \}     Tilde         \~}
%
% \iffalse
%%% From File: tudscr-mathswap.dtx
%<*driver>
% \fi
\ProvidesFile{tudscr-mathswap.dtx}%
  [2014/04/22 v2.00 TUD-KOMA-Script (mathswap)]
% \iffalse
\documentclass{tudscrdoc}
\KOMAoptions{parskip=half-}
\CodelineIndex
\RecordChanges
\GetFileInfo{tudscr-mathswap.dtx}
\begin{document}
  \maketitle
  \DocInput{\filename}
\end{document}
%</driver>
% \fi
%
% \selectlanguage{ngerman}
%
% \section{Aktive Trennzeichen im mathematischen Satz}
%
% Die Verwendung von Dezimal- und Tausendertrennzeichen im mathematischen Satz 
% sind regional sehr unterschiedlich. In den meisten englischsprachigen L�ndern 
% wird der Punkt als Dezimaltrennzeichen und das Komma zur Zifferngruppierung 
% verwendet, im restlichen Europa wird dies genau entgegengesetzt praktiziert.
% Dieses Paket soll dazu dienen, beliebige formatierte Zahlen in ihrer Ausgabe 
% anzupassen. Daf�r werden die Zeichen Punkt (\ .\ ) und Komma (\ ,\ ) als 
% aktive Zeichen im Mathematikmodus definiert.
% 
% �hnliche Funktionalit�ten werden bereits durch die Pakete \pkg{icomma} und 
% \pkg{ziffer} bereitgestellt. Bei \pkg{icomma} muss jedoch beim
% Verfassen des Dokumentes durch den Autor beachtet werden, ob das verwendete
% Komma einem Dezimaltrennzeichen entspricht ($t=1,\!2$) oder aber einem
% normalen Komma im Mathematiksatz ($z=f(x,y)$), wo ein gewisser Abstand nach
% dem Komma durchaus gew�nscht ist. Das Paket \pkg{ziffer} liefert daf�r die
% gew�nschte Funktionalit�t,%
% \footnote{kein Leerraum nach Komma, wenn direkt danach eine Ziffer folgt}
% ist allerdings etwas unflexibel, was den Umgang mit den Trennzeichen anbelangt.
% Als Alternative zu diesem Paket kann au�erdem \pkg{ionumbers} verwendet werden.
% 
% Das Paket \pkg{mathswap} sorgt daf�r, dass Trennzeichen direkt vor einer 
% Ziffer erkannt und nach bestimmten Vorgaben ersetzt werden. Sollte sich jedoch 
% zwischen Trennzeichen und Ziffer Leerraum befinden, wird dieser als solcher
% auch gesetzt.
%
% \StopEventually{\PrintIndex\PrintChanges}
%
% \iffalse
%<*package&header>
% \fi
%
% \subsection{Identifizierung des Pakets \pkg{mathswap}}
%
%    \begin{macrocode}
\NeedsTeXFormat{LaTeX2e}
\ProvidesPackage{mathswap}[%
  \TUDVersion\space package (swap mathematical separators)%
]
%    \end{macrocode}
%
% \iffalse
%</package&header>
%<*package&body>
% \fi
%
% \subsection{Das Paket \pkg{mathswap}}
%
% \begin{macro}{\commaswap}
% \begin{macro}{\dotswap}
% \begin{macro}{\@commaswap}
% \begin{macro}{\@dotswap}
% Die beiden Befehle \cs{commaswap} und \cs{dotswap} sind die zentrale 
% Benutzerschnittstelle des Paketes. Das Makro \cs{commaswap} definiert das 
% Trennzeichen oder den Inhalt, wodurch ein Komma ersetzt werden soll, auf 
% welches direkt danach eine Ziffer folgt. Normalerweise setzt \LaTeX
% nach einem Komma im mathematischen Satz zus�tzlich einen horizontalen Abstand.
% Bei der Ersetzung durch \cs{commaswap} entf�llt dieser. Die Voreinstellung
% f�r \cs{commaswap} ist deshalb auf ein Komma (,) gesetzt.
% 
% Mit dem Makro \cs{dotswap} kann definiert werden, wodurch der Punkt im 
% mathematischen Satz ersetzt werden soll, wenn auf diesen direkt anschlie�end 
% eine Ziffer folgt. Da der Punkt im deutschsprachigem Raum zur Gruppierung von
% Ziffern genutzt wird, ist hierf�r standardm��ig ein halbes gesch�tztes
% Leerzeichen definiert (\cs{,}).
%
% Die mit \cs{commaswap}\marg{Trennzeichen} und \cs{dotswap}\marg{Trennzeichen}
% definierten Trennzeichen werden in \cs{@commaswap} und \cs{@dotswap} gesichert.
%    \begin{macrocode}
\newcommand*\@commaswap{,}
\newcommand*\commaswap[1]{\gdef\@commaswap{#1}}
\newcommand*\@dotswap{\,}
\newcommand*\dotswap[1]{\gdef\@dotswap{#1}}
%    \end{macrocode}
% \end{macro}^^A \@dotswap
% \end{macro}^^A \@commaswap
% \end{macro}^^A \dotswap
% \end{macro}^^A \commaswap
% \begin{macro}{\ms@comma}
% \begin{macro}{\ms@commachar}
% \begin{macro}{\ms@dot}
% \begin{macro}{\ms@dotchar}
% In den Befehlen \cs{ms@commachar} und \cs{ms@dotchar} wird die originale 
% Definition von Komma und Punkt im Mathematikmodus gesichert, welche weiterhin 
% verwendet wird, falls nach diesen nicht direkt eine Ziffer folgt. Mit den 
% Makros \cs{ms@comma} und \cs{ms@dot} werden die Definitionen von Komma und 
% Punkt als normale Textzeichen f�r den mathematischen Satz bereitgehalten. 
% Anschlie�end werden beide als aktive Zeichen f�r den Mathematikmodus gesetzt.
%    \begin{macrocode}
\newcommand*\ms@commachar{}
\newcommand*\ms@dotchar{}
\AtBeginDocument{%
  \DeclareMathSymbol{\ms@comma}{\mathord}{letters}{"3B}%
  \mathchardef\ms@commachar\mathcode\expandafter`,%
  \mathcode\expandafter`,="8000%
  \DeclareMathSymbol{\ms@dot}{\mathord}{letters}{"3A}%
  \mathchardef\ms@dotchar\mathcode\expandafter`.%
  \mathcode\expandafter`.="8000%
}
%    \end{macrocode}
% \end{macro}^^A \ms@dotchar
% \end{macro}^^A \ms@dot
% \end{macro}^^A \ms@commachar
% \end{macro}^^A \ms@comma
% \begin{macro}{\ms@token}
% \begin{macro}{.}
% \begin{macro}{,}
% \begin{macro}{\ms@smartcomma}
% \begin{macro}{\ms@smartdot}
% Mit \cs{futurelet} wird dem Makro \cs{ms@token} das nachfolgende Zeichen von 
% Komma bzw. Punkt zugewiesen. Anschlie�end wird mittels \cs{ms@smartcomma} bzw.
% \cs{ms@smartdot} gepr�ft, ob es sich beim Inhalt von \cs{ms@token} um eine 
% Ziffer handelt. Dazu wird der Befehl \cs{ms@ifnum} verwendet. Sollte dies der 
% Fall sein, so wird ein Komma innerhalb einer Zahl durch \cs{@commaswap} und
% ein Punkt entsprechend mit \cs{@dotswap} ersetzt. Sollte direkt nach dem
% Trennzeichen keine Ziffer folgen, wird f�r Punkt und Komma die originale 
% Definition aus \cs{ms@commachar} bzw. \cs{ms@dotchar} verwendet.
%    \begin{macrocode}
\newcommand*\ms@token{}
\begingroup%
  \catcode`,=\active%
  \gdef,{\futurelet\ms@token\ms@smartcomma}%
  \catcode`.=\active%
  \gdef.{\futurelet\ms@token\ms@smartdot}%
\endgroup%
\def\ms@smartcomma{\ms@ifnum{\@commaswap}{\ms@commachar}}
\def\ms@smartdot{\ms@ifnum{\@dotswap}{\ms@dotchar}}
%    \end{macrocode}
% \end{macro}^^A \ms@smartdot
% \end{macro}^^A \ms@smartcomma
% \end{macro}^^A .
% \end{macro}^^A ,
% \end{macro}^^A \ms@token
% \begin{macro}{\ms@ifnum}
% \begin{macro}{\ms@@ifnum}
% Der Befehl \cs{ms@ifnum} reicht das erste Argument an \cs{ms@@ifnum} weiter,
% falls der Inhalt von \cs{ms@token} zu einer Ziffer expandiert. Andernfalls
% wird das zweite Argument ausgef�hrt.
%    \begin{macrocode}
\def\ms@ifnum#1#2{%
  \ifx\ms@token0\ms@@ifnum{#1}\else%
  \ifx\ms@token1\ms@@ifnum{#1}\else%
  \ifx\ms@token2\ms@@ifnum{#1}\else%
  \ifx\ms@token3\ms@@ifnum{#1}\else%
  \ifx\ms@token4\ms@@ifnum{#1}\else%
  \ifx\ms@token5\ms@@ifnum{#1}\else%
  \ifx\ms@token6\ms@@ifnum{#1}\else%
  \ifx\ms@token7\ms@@ifnum{#1}\else%
  \ifx\ms@token8\ms@@ifnum{#1}\else%
  \ifx\ms@token9\ms@@ifnum{#1}\else%
  #2%
  \fi\fi\fi\fi\fi\fi\fi\fi\fi\fi%
}
%    \end{macrocode}
% Wurde eine nachfolgende Ziffer erkannt, so wird die Definition von Punkt und 
% Komma auf die eines normalen Textzeichens gesetzt und anschlie�end die mit 
% \cs{commaswap} bzw. \cs{dotswap} angegebene Ersetzung ausgegeben.
%    \begin{macrocode}
\def\ms@@ifnum#1{%
  \begingroup%
  \mathcode\expandafter`,=\the\ms@comma%
  \mathcode\expandafter`.=\the\ms@dot%
  #1%
  \endgroup%
}
%    \end{macrocode}
% \end{macro}^^A \ms@@ifnum
% \end{macro}^^A \ms@ifnum
%
% \iffalse
%</package&body>
% \fi
%
% \Finale
%
\endinput