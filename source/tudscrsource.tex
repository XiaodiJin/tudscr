\begingroup
  \def\ProvidesFile#1[#2]{\csname endinput\endcsname}%
  \input tudscr-version.dtx
  \gdef\filename{\jobname}
\endgroup
\ProvidesFile{tudscrsource.tex}[\TUDVersion (sourcedoc)]
\RequirePackage[ngerman=ngerman-x-latest]{hyphsubst}
\documentclass[english,ngerman]{tudscrdoc}
\usepackage{selinput}\SelectInputMappings{adieresis={ä},germandbls={ß}}
\usepackage{babel}
\usepackage{bookmark}
\usepackage{microtype}
%\usepackage{tudscrfonts}
\makeatletter
\AtBeginDocument{%
  \expandafter\ifx\csname if@tud@cdfonts\endcsname\iftrue\relax
    \usepackage{mweights}
    \def\mddefault{m}
    \def\mdseries@tt{m}
    \renewcommand*\@pnumwidth{1.7em}
  \else%
    \usepackage{lmodern}%
  \fi
}
\makeatother
\EnableCrossrefs
\CodelineIndex
\RecordChanges
\GetFileInfo{tudscrsource.tex}
\KOMAoptions{parskip=half-}
\begin{document}
\addtokomafont{subject}{\sffamily}
\subject{\TUDScript{} \vTUDScript{} basierend auf \KOMAScript{} v3.12}
\title{Ein \LaTeXe-Bundle für Dokumente \mbox{im neuen \CD der} \mbox{\TnUD}}
\subtitle{Dokumentierter Quelltext}
\date{\filedate\\[1ex] Version \fileversion}
\author{Falk Hanisch\thanks{\tudscrmail}}
\makeatletter
\begingroup%
  \def\and{, }%
  \let\thanks\@gobble%
  \let\footnote\@gobble%
  \hypersetup{%
    pdfauthor = {\@author},%
    pdftitle = {\@title},%
    pdfsubject = {Quelltextdokumentation des \TUDScript-Bundles},%
    pdfkeywords = {LaTeX, \TUDScript, Quelltext},%
  }%
\endgroup%
\makeatother
\maketitle
Das \TUDScript-Bundle setzt das \CD der \TnUD für \LaTeXe{} um. Die darin 
enthaltenen Klassen und Paketen basieren auf dem \KOMAScript-Bundle und sind 
sehr eng mit diesen verwoben. Momentan ergänzen sie die alten, auf den 
Standard-\LaTeX-Klassen basierenden Vorlagen von Klaus Bergmann, sollen diese 
jedoch mittel- bis langfristig ersetzen.%
\footnote{%
  aktuell ist dies \cls{tudbook}, geplant \cls{tudfax}, \cls{tudletter}, 
  \cls{tudform}, \cls{tudhaus} und evtl. auch \cls{tudbeamer}
}
Es handelt sich bei diesem Dokument \emph{nicht} um das Anwenderhandbuch
sondern um den dokumentierten Quelltext der Implementierung von \TUDScript.
Das Handbuch für Anwender ist durch den Konsolenaufruf \texttt{texdoc tudscr} 
zu finden. 

\tableofcontents
\clearpage
\DocInclude{tudscr-version}
\DocInclude{tudscr-base}
\DocInclude{tudscr-fields}
\DocInclude{tudscr-locale}
\DocInclude{tudscr-fonts}
\DocInclude{tudscr-pagestyle}
\DocInclude{tudscr-layout}
\DocInclude{tudscr-title}
\DocInclude{tudscr-frontmatter}
\DocInclude{tudscr-misc}
\DocInclude{tudscr-color}
\DocInclude{tudscr-supervisor}
\DocInclude{tudscr-comp}
\DocInclude{tudscr-twocolfix}
\DocInclude{tudscr-mathswap}
%\DocInclude{tudscr-manual}
%\DocInclude{tudscr-doc}

\addpart{\appendixname}
\phantomsection\addxcontentsline{toc}{section}{\indexname}
\PrintIndex
\clearpage
\phantomsection\addxcontentsline{toc}{section}{Change History}
\PrintChanges
\end{document}
