\begingroup
  \def\ProvidesFile#1[#2]{\csname endinput\endcsname}%
  \input tudscr-version.dtx
  \gdef\filename{\jobname}
\endgroup
\ProvidesFile{tudscrsource.tex}[\TUDVersion (sourcedoc)]
\RequirePackage[ngerman=ngerman-x-latest]{hyphsubst}
\documentclass[final]{tudscrdoc}
\EnableCrossrefs
\CodelineIndex
\RecordChanges
\GetFileInfo{tudscrsource.tex}
\usepackage{bookmark}
\KOMAoptions{parskip=half-}
\begin{document}
\title{\LaTeX{}-Vorlagen im Corporate Design der Technischen Universit�t Dresden
  \\basierend auf \KOMAScript{} v3.11b}
\subtitle{Dokumentierter Quelltext}
\date{\filedate\\[1ex] Version \fileversion}
\author{Falk Hanisch\footnote{\filemail}}
\maketitle
\begin{abstract}
\KOMAoptions{parskip=half-}
\noindent Dies ist die Dokumentation der \LaTeX{}"=Implementierung des
Corporate Designs der Technischen Universit�t Dresden in unterschiedliche
Formatvorlagen. Die Vorlagen basieren auf den modernen \KOMAScript{}-Klassen
und sind sehr eng mit diesen verwoben. Sie sollen die alten, auf den
Standard-\LaTeX{}-Klassen basierenden Vorlagen\footnote{aktuell ist dies
\cls{tudbook}, geplant \cls{tudfax}, \cls{tudletter}, \cls{tudform},
\cls{tudhaus} und evtl. auch \cls{tudbeamer}} von Klaus Bergmann ersetzen.

Es handelt sich bei diesem Dokument \emph{nicht} um das Anwenderhandbuch
\texttt{tudscr.pdf}. Dieses ist -- nach einer erfolgreichen Installation des
Bundles -- beispielsweise durch den Konsolenaufruf \texttt{texdoc tudscr} zu
finden.
\end{abstract}
\tableofcontents
\clearpage
\DocInclude{tudscr-version}
\DocInclude{tudscr-base}
\DocInclude{tudscr-fields}
\DocInclude{tudscr-locale}
\DocInclude{tudscr-fonts}
\DocInclude{tudscr-pagestyle}
\DocInclude{tudscr-layout}
\DocInclude{tudscr-title}
\DocInclude{tudscr-frontmatter}
\DocInclude{tudscr-misc}
\DocInclude{tudscr-color}
\DocInclude{tudscr-supervisor}
\DocInclude{tudscr-comp}
\DocInclude{tudscr-twocolfix}
\DocInclude{tudscr-mathswap}
\PrintIndex
\PrintChanges
\end{document}