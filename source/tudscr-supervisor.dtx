% \CheckSum{586}
% \iffalse meta-comment
% ======================================================================
%
% Das Corporate Design der TU Dresden auf Basis der KOMA-Script-Klassen
%
% ======================================================================
% This work may be distributed and/or modified under the conditions of
% the LaTeX Project Public License, version 1.3c of the license.
% The latest version of this license is in
%     http://www.latex-project.org/lppl.txt
% and version 1.3c or later is part of all distributions of LaTeX
% version 2005/12/01 or later and of this work.
% This work has the LPPL maintenance status "author-maintained".
% The current maintainer and author of this work is Falk Hanisch.
% ----------------------------------------------------------------------
% Dieses Werk darf nach den Bedingungen der LaTeX Project Public Lizenz,
% Version 1.3c, verteilt und/oder veraendert werden.
% Die neuste Version dieser Lizenz ist
%     http://www.latex-project.org/lppl.txt
% und Version 1.3c ist Teil aller Verteilungen von LaTeX
% Version 2005/12/01 oder spaeter und dieses Werks.
% Dieses Werk hat den LPPL-Verwaltungs-Status "author-maintained"
% (allein durch den Autor verwaltet).
% Der aktuelle Verwalter und Autor dieses Werkes ist Falk Hanisch.
% ======================================================================
% \fi
%
% \CharacterTable
%  {Upper-case    \A\B\C\D\E\F\G\H\I\J\K\L\M\N\O\P\Q\R\S\T\U\V\W\X\Y\Z
%   Lower-case    \a\b\c\d\e\f\g\h\i\j\k\l\m\n\o\p\q\r\s\t\u\v\w\x\y\z
%   Digits        \0\1\2\3\4\5\6\7\8\9
%   Exclamation   \!     Double quote  \"     Hash (number) \#
%   Dollar        \$     Percent       \%     Ampersand     \&
%   Acute accent  \'     Left paren    \(     Right paren   \)
%   Asterisk      \*     Plus          \+     Comma         \,
%   Minus         \-     Point         \.     Solidus       \/
%   Colon         \:     Semicolon     \;     Less than     \<
%   Equals        \=     Greater than  \>     Question mark \?
%   Commercial at \@     Left bracket  \[     Backslash     \\
%   Right bracket \]     Circumflex    \^     Underscore    \_
%   Grave accent  \`     Left brace    \{     Vertical bar  \|
%   Right brace   \}     Tilde         \~}
%
% \iffalse
%%% From File: tudscr-supervisor.dtx
%<*driver>
\ifx\ProvidesFile\undefined\def\ProvidesFile#1[#2]{}\fi
\ProvidesFile{tudscr-supervisor.dtx}[%
  2014/09/12 v2.02 TUD-KOMA-Script\space%
%</driver>
%<package>\NeedsTeXFormat{LaTeX2e}[2011/06/27]
%<package>\ProvidesPackage{tudscrsupervisor}[%
%<*driver|package>
%!TUDVersion
%<package>  package
  (commands for supervisors)%
]
%</driver|package>
%<*driver>
\RequirePackage[ngerman=ngerman-x-latest]{hyphsubst}
\documentclass[english,ngerman]{tudscrdoc}
\usepackage{selinput}\SelectInputMappings{adieresis={ä},germandbls={ß}}
\usepackage{babel}
\KOMAoptions{parskip=half-}
\CodelineIndex
\RecordChanges
\GetFileInfo{tudscr-supervisor.dtx}
\begin{document}
  \maketitle
  \DocInput{\filename}
\end{document}
%</driver>
% \fi
%
% \selectlanguage{ngerman}
%
% \section{Aufgabenstellung}
%
% Diese Paket stellt für die \cls{tudscr}-Klassen eine Umgebung für die
% Aufgabenstellung einer Abschlussarbeit bereit.
%
% \StopEventually{\PrintIndex\PrintChanges}
%
% \iffalse
%<*package>
% \fi
%
% \subsection{Das Paket \pkg{tudscrsupervisor}}
%
% \begin{environment}{task}
% Die Umgebung für die Aufgabenstellung nutzt die \env{tudpage}-Umgebung. Sie
% wird auf einer neuen (rechten) Seite gesetzt. Es wird zu Beginn eine Tabelle 
% mit Informationen zum Autor gesetzt. Zum Abschluss werden Betreuer,
% Hochschullehrer und ggf. Vorsitzender des Prüfungsausschusses ausgegeben.
%    \begin{macrocode}
\newenvironment{task}[1][]{%
%    \end{macrocode}
% Die \env{tudpage}-Umgebung wird geöffnet. Mit dem Parameter \opt{headline} 
% kann die standardmäßige Überschrift überschrieben werden.
%    \begin{macrocode}
  \cleardoubleoddpage%
  \def\@headline{}%
  \TUD@parameter{tudpage}{%
    \TUD@parameter@define{headline}{\def\@headline{##1}}%
    \TUD@parameter@let{line}{headline}%
    \TUD@parameter@sethandler{\def\@headline{##1}}%
    \TUD@parameter@set{pagestyle=empty}%
  }%
  \tudpage[#1]%
%    \end{macrocode}
% Zu Beginn wird als erstes die Überschrift und~-- die entsprechende Option
% vorausgesetzt~-- im PDF einen Lesezeichen- oder auch Outline-Eintrag gesetzt.
%    \begin{macrocode}
  \tudbookmark{\taskname}{task}%
  \subsection*{%
    \ifx\@headline\@empty%
      \taskname\space%
      \ifx\tasktext\@empty\else\ifx\@thesis\@empty\else%
        \ignorespaces\tasktext\space\trim@spaces{\@thesis}%
      \fi\fi%
    \else\@headline\fi%
  }%
  \tud@author@table%
}{%
%    \end{macrocode}
% Da auch Betreuer durch den Befehl \cs{and} getrennt werden, wird dieser für
% die korrekte Ausgabe umdefiniert. Anschließend folgt die Ausgabe in einer
% Tabelle, die Spalte der Bezeichner entspricht der aus dem oberen Teil.
%    \begin{macrocode}
  \def\and{%
    \tabularnewline%
    \ifx\supervisorothername\@empty\else\supervisorothername\@titledelim\fi%
      & \def\and{\tabularnewline &}%
  }%
  \vskip-\lastskip%
  \ifdim\parskip>\z@\vskip\parskip\else\vskip\topsep\fi\medskip%
  \begingroup%
  \setparsizes{\z@}{\z@}{\z@\@plus 1fil}\par@updaterelative%
  \begin{tabular}{@{}p{\tud@dim@table}l@{}}%
    \supervisorname\@titledelim &
      \@supervisor\tabularnewline[\smallskipamount]%
    \issuedatetext\@titledelim & \@issuedate\tabularnewline%
    \duedatetext\@titledelim & \@duedate\tabularnewline%
  \end{tabular}%
%    \end{macrocode}
% Darunter wird etwas Platz für die Unterschriften von betreuendem Professor
% und ggf. Prüfungsausschussvorsitzenden gehalten. Auch diese beiden werden
% in einer Tabelle ausgegeben. Die \env{tudpage}-Umgebung wird beendet, und
% eine neue (rechte) Seite geöffnet.
%    \begin{macrocode}
  \vskip 15mm plus 10mm minus 10mm%
  \ifx\@chairman\@empty\else%
    \begin{tabular}{@{}l@{}}%
      \@chairman\tabularnewline%
      \chairmanname\tabularnewline%
    \end{tabular}%
    \hfill%
  \fi%
  \ifx\@professor\@empty\else%
    \begin{tabular}{@{}l@{}}%
      \@professor\tabularnewline%
      \professorname\tabularnewline%
    \end{tabular}%
  \fi%
  \par%
  \endgroup%
  \endtudpage%
  \cleardoublepage%
}
%    \end{macrocode}
% \end{environment}^^A task
% \begin{macro}{\taskform}
% Dies soll die Standardform einer Aufgabenstellung sein. Im ersten Argument
% werden kurz die Ziele motiviert und erläutert, im zweiten Argument werden im
% besten Fall die Schwerpunkte in einer \env{itemize}-Umgebung aufgeschlüsselt.
%    \begin{macrocode}
\newcommand\taskform[3][]{%
  \begin{task}[#1]
    \vskip-\lastskip%
    \ifxblank{#2}{}{\minisec{\objectivesname}#2}%
    \ifxblank{#3}{}{%
      \minisec{\focusname}%
      \begin{itemize}\tud@RaggedRight%
        #3
      \end{itemize}%
    }%
  \end{task}%
}
%    \end{macrocode}
% \end{macro}^^A \taskform
% \begin{environment}{evaluation}
% Die Umgebung für das Gutachten nutzt ebenfalls die \env{tudpage}-Umgebung. Sie
% wird auf einer neuen (rechten) Seite gesetzt. Es wird zu Beginn eine Tabelle 
% mit Informationen zum Autor gesetzt. Zum Abschluss werden Ort, Datum und 
% Gutachter ausgegeben.
%    \begin{macrocode}
\newenvironment{evaluation}[1][]{%
%    \end{macrocode}
% Die \env{tudpage}-Umgebung wird geöffnet.  Mit dem Parameter \opt{headline} 
% kann die standardmäßige Überschrift überschrieben werden. Zu Beginn wird als
% erstes die Überschrift und~-- die entsprechende Option vorausgesetzt~-- im
% PDF einen Lesezeichen- oder auch Outline-Eintrag gesetzt.
%    \begin{macrocode}
  \cleardoubleoddpage%
  \def\@headline{}%
  \TUD@parameter{tudpage}{%
    \TUD@parameter@define{headline}{\def\@headline{##1}}%
    \TUD@parameter@let{line}{headline}%
    \TUD@parameter@define{grade}{\def\@grade{##1}}%
    \TUD@parameter@sethandler{\def\@headline{##1}}%
    \TUD@parameter@set{pagestyle=empty}%
  }%
  \tudpage[#1]%
  \tudbookmark{\evaluationname}{evaluation}%
  \subsection*{%
    \ifx\@headline\@empty%
      \evaluationname\space%
      \ifx\evaluationtext\@empty\else\ifx\@thesis\@empty\else%
        \ignorespaces\evaluationtext\space\trim@spaces{\@thesis}%
      \fi\fi%
    \else\@headline\fi%
  }%
  \tud@author@table%
}{%
%    \end{macrocode}
% Die gegebenen Note sowie Ort und Datum werden am Ende ggf. ausgegeben.
%    \begin{macrocode}
  \vskip-\lastskip%
  \ifdim\parskip>\z@\vskip\parskip\else\vskip\topsep\fi%
  \medskip%
  \ifx\@grade\@empty\else%
    \noindent\gradetext%
    \vskip\bigskipamount%
  \fi%
  \ifx\@date\@empty\else%
    \noindent%
    \ifx\@place\@empty\else\@place,~\fi\@date%
    \vskip\bigskipamount%
  \fi%
  \bigskip\bigskip\noindent%
%    \end{macrocode}
% Der Befehl \cs{and} wird für einen möglichen Zweitgutachter angepasst. Das 
% Hilfsmakro \cs{@tempa} dient zur Übernahme des richtigen Bezeichners für
% Erst- bzw. Zweitgutachter. Sollten mit \cs{referee} keine Gutachter angegeben
% sein, so werden die angegeben Betreuer verwendet.
%    \begin{macrocode}
  \ifx\@referee\@empty\let\@referee\@supervisor\fi%
  \let\@tempa\refereename%
  \def\and{%
    \tabularnewline%
    \@tempa%
    \endtabular%
    \hfill%
    \tabular{@{}l@{}}%
    \global\let\@tempa\refereeothername%
  }%
  \begin{tabular}{@{}l@{}}%
  \@referee%
  \tabularnewline%
  \@tempa%
  \end{tabular}%
  \hfill\null%
  \endtudpage%
  \cleardoublepage%
}
%    \end{macrocode}
% \end{environment}^^A evaluation
% \begin{macro}{\evaluationform}
% Dies soll die Standardform eines Gutachtens sein. Im ersten Argument wird 
% kurz die Aufgabenstellung zusammengefasst, im zweiten Argument wird der
% Inhalt und die Struktur der Arbeit kurz beschrieben. Im dritten Argument
% erfolgt die Bewertung, das letzte Argument beinhaltet die Note.
%    \begin{macrocode}
\newcommand\evaluationform[5][]{%
  \begin{evaluation}[#1]
    \vskip-\lastskip%
    \ifxblank{#2}{}{\minisec{\taskname}#2}%
    \ifxblank{#3}{}{\minisec{\contentname}#3}%
    \ifxblank{#4}{}{\minisec{\assessmentname}#4}%
    \ifxblank{#5}{}{\def\@grade{#5}}%
  \end{evaluation}%
}
%    \end{macrocode}
% \end{macro}^^A \evaluationform
% \begin{macro}{\tud@author@table}
% \changes{v2.01b}{2014/06/04}{Probleme mit \pkg{calc} behoben}%^^A
% \begin{length}{\tud@dim@table}
% Der Befehl \cs{tud@author@table} dient zur Ausgabe einer Tabelle mit
% Informationen zum Autor/zu den Autoren\footnote{Matrikelnummer, Jahrgang,
% Studiengang etc.} für Aufgabenstellung und Gutachten.
%    \begin{macrocode}
\newlength{\tud@dim@table}%
\newcommand*\tud@author@table{%
  \tud@check@author%
  \begingroup%
  \let\thanks\@gobble%
  \let\footnote\@gobble%
%    \end{macrocode}
% Der Befehl \cs{tud@split@author} ist original aus den \cls{tudscr}-Klassen
% und dient zur formatierten Ausgabe von mehreren Autoren, welche durch
% \cs{author}\marg{Autor(en)} angegeben und mit \cs{and} getrennt wurden.
% Er wird hier auf die Ausgabe der Autoren mit den jeweils zusätzlich gegebenen
% Informationen in einer Tabelle angepasst.
%    \begin{macrocode}
  \renewcommand*\tud@split@author[2]{%
%    \end{macrocode}
% Weil alle Autoren in einer Tabelle gesetzt werden wird geprüft, welche Felder
% individuell via \cs{author} angegeben wurden. Damit die Tabellen die gleiche
% Höhe haben, auch wenn für einen Autor ein Feld ausgelassen wurde, werden alle
% insgesamt angegebenen Felder mit via \cs{tud@multiple@setfields} mit \cs{null}
% initialisiert. Anschließend werden für den aktuellen Autor angegebenen Felder
% gesetzt.
%    \begin{macrocode}
    \tud@multiple@setfields{\null}{##1}%
%    \end{macrocode}
% Nach viel Geplänkel kommt nun die eigentliche Tabelle mit ggf. zusätzlichen
% Informationen zum Autor.
%    \begin{macrocode}
    \begin{tabular}{l@{}}%
      \ifx\@course\@empty\else\@course\tabularnewline\fi%
      \ifx\@discipline\@empty\else\@discipline\tabularnewline\fi%
      \textsf{\textbf{\ignorespaces##1}}\tabularnewline%
      \ifx\@matriculationnumber\@empty\else%
        \@matriculationnumber\tabularnewline%
      \fi%
      \ifx\@matriculationyear\@empty\else%
        \@matriculationyear\tabularnewline%
      \fi%
    \end{tabular}%
%    \end{macrocode}
% Sollte ein weiterer Autor folgen, wird \cs{tabcolsep} zusätzlich eingefügt,
% um den Standardabstand bei Tabellen zu sichern, da die Tabelle vorher ohne
% rechten \glqq Rand\grqq{} gesetzt wurde, um die letzte Tabelle ggf. genau bis
% zum rechten Rand setzen zu können.
%    \begin{macrocode}
    \tud@multiple@@@split{##2}{~~\hspace{\tabcolsep}}
  }%
%    \end{macrocode}
% Zu Beginn wird eine Tabelle mit den Bezeichnern aller genutzten Feldern
% ausgegeben. Danach folgen alle Autoren. Damit ein einheitliches Layout
% entsteht und auch die Tabellen am Ende der Umgebung in der ersten Spalte die
% gleiche Breite haben wie im oberen Teil, ist die Bestimmung einer festen
% Spaltenbreite notwendig, die so breit wie der längste Bezeichner ist.
% Dafür muss festgestellt werden, welche optionalen Felder denn nun überhaupt
% genutzt werden. Dafür wird \cs{tud@multiple@setfields} mit \cs{null} als
% Argument aufgerufen, um alle potentiellen Felder erkennen zu können.
%    \begin{macrocode}
  \setlength{\tud@dim@table}{2em}%
  \gdef\tud@multiple@field{@author}%
  \tud@multiple@setfields{\null}{}%
%    \end{macrocode}
% Anschließend werden die Bezeichner sowohl der obligatorischen als auch der
% genutzten, optionalen Felder expandiert. Anschließend wir mit dieser Liste
% der längste bestimmt und in \cs{tud@dim@table} gespeichert.
%    \begin{macrocode}
  \edef\@tempa{%
    \authorname,\titlename,\supervisorname,\supervisorothername,%
    \issuedatetext,\duedatetext,%
    \expandafter\ifx\@matriculationnumber\@empty\else%
      \matriculationnumbername%
    \fi,%
    \expandafter\ifx\@matriculationyear\@empty\else%
      \matriculationyearname%
    \fi,%
    \expandafter\ifx\@course\@empty\else\coursename\fi,%
    \expandafter\ifx\@discipline\@empty\else\disciplinename\fi,%
  }%
  \@for\@tempb:=\@tempa\do{%
    \settowidth{\@tempdima}{\@tempb\@titledelim}%
    \ifdim\@tempdima>\tud@dim@table\relax%
      \setlength{\tud@dim@table}{\@tempdima}%
    \fi%
  }%
  \global\tud@dim@table=\tud@dim@table%
%    \end{macrocode}
% Die Tabelle mit den benötigten Bezeichnern. Damit diese bis an den Seiterand
% ohne Warnungen gesetzt werden können, wird die Auszeichnung von Absatzenden
% aufgehoben.
%    \begin{macrocode}
  \begingroup%
  \setparsizes{\z@}{\z@}{\z@\@plus 1fil}\par@updaterelative%
  \begin{tabular}{@{}p{\tud@dim@table}}%
    \ifx\@course\@empty\else%
      \coursename\@titledelim\tabularnewline%
    \fi%
    \ifx\@discipline\@empty\else%
      \disciplinename\@titledelim\tabularnewline%
    \fi%
    \authorname\@titledelim\tabularnewline%
    \ifx\@matriculationnumber\@empty\else%
      \matriculationnumbername\@titledelim\tabularnewline%
    \fi%
    \ifx\@matriculationyear\@empty\else%
      \matriculationyearname\@titledelim\tabularnewline%
    \fi%
  \end{tabular}%
%    \end{macrocode}
% Folgend werden die Autoren wie schon bei \cs{maketitle} mit den gleichen
% Makros gesplittet und separat ausgegeben.
%    \begin{macrocode}
  \tud@multiple@split{@author}%
%    \end{macrocode}
% Nach den Autoren wird der Titel über die komplette Textbreite ausgegeben.
% Danach wird der Inhalt der Aufgabenstellung ausgegeben.
%    \begin{macrocode}
  \vskip\smallskipamount%
  \begin{tabular}{@{}p{\tud@dim@table}%
    p{\dimexpr\textwidth-\tud@dim@table-2\tabcolsep\relax}@{}}%
    \titlename\@titledelim & \tud@RaggedRight\textsf{\textbf{\@@title}}%
  \end{tabular}%
  \par%
  \endgroup%
  \ifdim\parskip>\z@\else\vskip\topsep\fi%
  \endgroup%
  \noindent\ignorespaces%
}
%    \end{macrocode}
% \end{length}^^A \tud@dim@table
% \end{macro}^^A \tud@author@table
% \begin{macro}{\tud@author@checkfields}
% Der Befehl \cs{tud@author@checkfields} wird um die hier im Paket zusätzlich
% definierten Felder erweitert.
%    \begin{macrocode}
\patchcmd{\tud@split@author@list}{\authormore}%
  {\authormore,\course,\discipline}{}{\tud@patch@wrn{tud@split@author@list}}
%    \end{macrocode}
% \end{macro}^^A \tud@author@checkfields
% \begin{environment}{notice}
% Die Umgebung für Aushänge nutzt ebenfalls die \env{tudpage}-Umgebung. Sie wird
% auf einer neuen (rechten) Seite gesetzt. Die Überschrift wird in der 
% Voreinstellung auf den sprachabhängigen Bezeichner \cs{noticename} gesetzt, 
% welcher allerdings mit dem Parameter \opt{headline} überschrieben werden kann.
%    \begin{macrocode}
\newenvironment{notice}[1][]{%
  \cleardoubleoddpage%
  \def\@headline{\noticename}%
  \TUD@parameter{tudpage}{%
    \TUD@parameter@define{headline}{\def\@headline{##1}}%
    \TUD@parameter@let{line}{headline}%
    \TUD@parameter@sethandler{\def\@headline{##1}}%
    \TUD@parameter@set{pagestyle=empty}%
  }%
  \tudpage[#1]%
  \tudbookmark{\noticename}{notice}%
%    \end{macrocode}
% Es wird zu Beginn das angegebene Datum oben auf der rechten Seite ausgegeben. 
% Anschließend wird die Überschrift und der gegebene Titel gesetzt. 
%    \begin{macrocode}
  \ifx\@date\@empty\else%
    \begingroup%
      \vspace*{-\parskip}%
      \vspace*{-2\baselineskip}%
      \raggedleft
      \@date\par%
      \vspace*{-\parskip}%
    \endgroup%
  \fi%
  \ifx\@headline\@empty\else%
    \section*{\@headline}%
  \fi%
}{%
%    \end{macrocode}
% Wenn keine Kontaktperson direkt angegeben wurden, werden die Informationen der
% angegeben Betreuer verwendet. Wenn eine Personenangabe gefunden wurde, werden 
% die Kontaktdaten ausgegeben.
%    \begin{macrocode}
  \ifx\@contactperson\@empty\let\@contactperson\@supervisor\fi%
  \ifx\@contactperson\@empty\else%
    \vskip-\lastskip%
    \ifdim\parskip>\z@\vskip\parskip\else\vskip\topsep\fi%
    \subsection*{\contactpersonname}%
    \noindent\tud@multiple@split{@contactperson}\hfill\null%
  \fi%
  \endtudpage%
  \cleardoublepage%
}
%    \end{macrocode}
% \end{environment}^^A notice
% \begin{macro}{\noticeform}
% Dies soll die Standardform eines Aushangs für eine Abschlussarbeit sein. Im
% ersten Argument wird kurz der Inhalt zusammengefasst, im zweiten Argument 
% werden die Arbeitsschwerpunkte beschrieben.
%    \begin{macrocode}
\newcommand\noticeform[3][]{%
  \begin{notice}[#1]
    \vskip-\lastskip%
    \ifxblank{#2}{}{%
      \ifx\@@title\@empty\else%
        \minisec{\expandonce{\@@title}}%
      \fi%
      #2%
    }%
    \ifxblank{#3}{}{%
      \minisec{\focusname}%
      \begin{itemize}\tud@RaggedRight%
      #3
      \end{itemize}%
    }%
  \end{notice}%
}
%    \end{macrocode}
% \end{macro}^^A \noticeform
% \begin{macro}{\tud@split@contactperson}
% \begin{macro}{\tud@split@contactperson@list}
% Mit diesem Befehl werden für einen Aushang die Daten für einen oder mehrere 
% Kontaktpersonen ausgegeben.
%    \begin{macrocode}
\newcommand*\tud@split@contactperson[2]{%
  \tud@multiple@setfields{\null}{#1}%
  \begin{tabular}{@{}l@{}}%
    \ignorespaces#1\tabularnewline%
    \ifx\@office\@empty\else\@office\tabularnewline\fi%
    \ifx\@telephone\@empty\else\@telephone\tabularnewline\fi%
    \ifx\@emailaddress\@empty\else\@emailaddress\tabularnewline\fi%
  \end{tabular}%
  \tud@multiple@@@split{#2}{\hfill}%
}
\newcommand*\tud@split@contactperson@list{\office,\telephone,\emailaddress}
%    \end{macrocode}
% \end{macro}^^A \tud@split@contactperson@list
% \end{macro}^^A \tud@split@contactperson
%
% \iffalse
%</package>
% \fi
%
% \Finale
%
\endinput