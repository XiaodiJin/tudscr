% \CheckSum{1400}
% \iffalse meta-comment
% 
% ============================================================================
% 
%  TUD-KOMA-Script
%  Copyright (c) Falk Hanisch <tudscr@gmail.com>, 2012-2015
% 
% ============================================================================
% 
%  This work may be distributed and/or modified under the conditions of the
%  LaTeX Project Public License, version 1.3c of the license. The latest
%  version of this license is in http://www.latex-project.org/lppl.txt and 
%  version 1.3c or later is part of all distributions of LaTeX 2005/12/01
%  or later and of this work. This work has the LPPL maintenance status 
%  "author-maintained". The current maintainer and author of this work
%  is Falk Hanisch.
% 
% ----------------------------------------------------------------------------
% 
% Dieses Werk darf nach den Bedingungen der LaTeX Project Public Lizenz
% in der Version 1.3c, verteilt und/oder veraendert werden. Die aktuelle 
% Version dieser Lizenz ist http://www.latex-project.org/lppl.txt und 
% Version 1.3c oder spaeter ist Teil aller Verteilungen von LaTeX 2005/12/01 
% oder spaeter und dieses Werks. Dieses Werk hat den LPPL-Verwaltungs-Status 
% "author-maintained", wird somit allein durch den Autor verwaltet. Der 
% aktuelle Verwalter und Autor dieses Werkes ist Falk Hanisch.
% 
% ============================================================================
%
% \fi
%
% \CharacterTable
%  {Upper-case    \A\B\C\D\E\F\G\H\I\J\K\L\M\N\O\P\Q\R\S\T\U\V\W\X\Y\Z
%   Lower-case    \a\b\c\d\e\f\g\h\i\j\k\l\m\n\o\p\q\r\s\t\u\v\w\x\y\z
%   Digits        \0\1\2\3\4\5\6\7\8\9
%   Exclamation   \!     Double quote  \"     Hash (number) \#
%   Dollar        \$     Percent       \%     Ampersand     \&
%   Acute accent  \'     Left paren    \(     Right paren   \)
%   Asterisk      \*     Plus          \+     Comma         \,
%   Minus         \-     Point         \.     Solidus       \/
%   Colon         \:     Semicolon     \;     Less than     \<
%   Equals        \=     Greater than  \>     Question mark \?
%   Commercial at \@     Left bracket  \[     Backslash     \\
%   Right bracket \]     Circumflex    \^     Underscore    \_
%   Grave accent  \`     Left brace    \{     Vertical bar  \|
%   Right brace   \}     Tilde         \~}
%
% \iffalse
%%% From File: tudscr-title.dtx
%<*driver>
\ifx\ProvidesFile\undefined\def\ProvidesFile#1[#2]{}\fi
\ProvidesFile{tudscr-title.dtx}[%
  2015/02/15 v2.03 TUD-KOMA-Script (title)%
]
\RequirePackage[ngerman=ngerman-x-latest]{hyphsubst}
\documentclass[english,ngerman]{tudscrdoc}
\usepackage{selinput}\SelectInputMappings{adieresis={ä},germandbls={ß}}
\usepackage[T1]{fontenc}
\usepackage{babel}
\usepackage{tudscrfonts} % only load this package, if the fonts are installed
\KOMAoptions{parskip=half-}
\CodelineIndex
\RecordChanges
\GetFileInfo{tudscr-title.dtx}
\begin{document}
  \maketitle
  \DocInput{\filename}
\end{document}
%</driver>
% \fi
%
% \selectlanguage{ngerman}
%
% \changes{v2.02}{2014/06/24}{Unterstützung für \pkg{titlepage} entfernt}%^^A
%
% \section{Titelei}
%
% Die Titelseite wird für die \TUDScript-Klassen bereitgestellt.
%
% \StopEventually{\PrintIndex\PrintChanges}
%
% \iffalse
%<*class&body>
% \fi
%
% \subsection{Titelseite und Titelkopf}
%
% Für alle Klassen im \CD der Technischen Universität Dresden wird eine neue
% Titelseite und ein neuer Titelkopf bereitgestellt. Es werden eine monochrome
% Version, eine Version mit leichtem Farbeinsatz und eine durchweg farbige
% Version bereitgestellt Außerdem wird hier das Aussehen und die Position des
% Titels analog zu den Überschriften auf Teile- und Kapitelseiten festgelegt.
%
% \iffalse
%</class&body>
%<*class&option>
% \fi
%
% \begin{option}{subjectthesis}
% \begin{macro}{\if@tud@subjectthesis}
% \begin{macro}{\if@tud@subjectthesis@locked}
% Die alte \cls{tudbook}-Klasse hat den Typ der Arbeit in der Betreffzeile des
% Titels gesetzt. Um dieses Verhalten nachbilden zu können, wird diese Option
% zur Verfügung gestellt.
%    \begin{macrocode}
\newif\if@tud@subjectthesis
\TUD@key@lock{subjectthesis}[true]{%
  \TUD@set@ifkey@lock{subjectthesis}{@tud@subjectthesis}{#1}%
}
%    \end{macrocode}
% \end{macro}^^A \if@tud@subjectthesis@locked
% \end{macro}^^A \if@tud@subjectthesis
% \end{option}^^A subjectthesis
%
% \iffalse
%</class&option>
%<*class&body>
% \fi
%
% \begin{KOMAfont}{thesis}
% \begin{KOMAfont}{titlepage}
% \changes{v2.01}{2014/04/24}{Explizite Wahl der genutzten Schriftstärke}%^^A
% \changes{v2.02}{2014/08/29}{Anpassungen für \pkg{fontspec}}%^^A
% Diese Font-Elemente werden auf der Titelseite verwendet. \opt{thesis}, um den
% Typ der Arbeit und ggf. den zu erlangenden akademischen Grad zu setzen.
% Allerdings kommt letzteres nur zum Tragen, wenn die Option \opt{subthesis}
% deaktiviert ist. Es wird die gleiche Schrift wie für den Titel verwendet,
% allerdings etwas kleiner. \opt{titlepage} für alle Zusatzinformationen wie
% Autor, Datum etc.
%    \begin{macrocode}
\newkomafont{thesis}{\usekomafont{title}\LARGE}
\newkomafont{titlepage}{%
  \usekomafont{subject}%
  \fontseries{\mddefault}%
  \selectfont%
  \large%
}
%    \end{macrocode}
% \end{KOMAfont}^^A titlepage
% \end{KOMAfont}^^A thesis
% \begin{macro}{\tud@title@fontcolor}
% Das Makro wird für die Hintergrundfarbe der Titelseite verwendet.
%    \begin{macrocode}
\newcommand*\tud@title@fontcolor{black}
%    \end{macrocode}
% \end{macro}^^A \tud@title@fontcolor
% \begin{macro}{\tud@maketitle}
% \changes{v2.01a}{2014/06/03}{Fehler bei Nutzung von \cs{subtitle} behoben}%^^A
% \changes{v2.02}{2014/06/23}{\val{firstiscover} wird als Wert für die Option 
%   \opt{titlepage} unterstützt}%^^A
% \changes{v2.02}{2014/07/19}{Im zweispaltigen Satz wird für den Titelkopf
%   \cs{tud@title@twocolumn} verwendet}%^^A
% \changes{v2.02}{2014/09/05}{Neue Parameter}%^^A
% \changes{v2.02}{2014/09/05}{Titelseite mit \cs{tud@maketitlepage}}%^^A
% \begin{macro}{\tud@titlepagenumber}
% \changes{v2.02}{2014/06/23}{neu}%^^A
% Bei der entsprechend gewählten Layoutoption, wird eine neue Titelseite im \CD
% gesetzt. Damit auf das Standardverhalten von \KOMAScript{} zurückgestellt
% werden kann, wird der ursprüngliche \cs{maketitle}-Befehl bei der Abarbeitung 
% der Layoutoptionen gesichert. (siehe \cs{tud@layout@process}) und kann ggf. 
% wiederhergestellt bzw. mit \cs{tud@cmd@use} verwendet werden. Ist das Layout 
% des \CDs gefordert wird entweder mit \cs{tud@maketitlepage} eine Titelseite 
% gesetzt~-- wobei sich diese in ihrer Struktur am originalen \cs{maketitle}
% orientiert~-- oder das Titelkopflayout wird angepasst.
%    \begin{macrocode}
\newcommand*\tud@titlepagenumber{}%
\let\tud@titlepagenumber\relax%
\newcommand*\tud@maketitle[1][]{%
  \begingroup%
%    \end{macrocode}
% Das optionale Argument wird für das Setzen der Parameter verwendet.
%    \begin{macrocode}
    \TUD@parameter@set{title}{#1}%
    \ifcase\tud@layout@title\relax%
      \ifx\tud@titlepagenumber\relax\gdef\tud@titlepagenumber{1}\fi%
      \tud@cmd@use{maketitle}[\tud@titlepagenumber]%
    \else%
      \if@titlepage%
        \tud@maketitlepage%
%    \end{macrocode}
% Neben einer kompletten Titelseite gibt es auch noch einen Titelkopf. Auch 
% hierfür wird die Ausprägung festgelegt. 
%    \begin{macrocode}
      \else%
        \def\tud@title@fontcolor{black}%
        \ifcase\tud@layout@title\relax\or\else%
          \def\tud@title@fontcolor{HKS41}%
        \fi%
%    \end{macrocode}
% Um einen Titelkopf zu setzen, wird einfach der Originalbefehl verwendet und
% nur der innere Teil (\cs{@maketitle}) angepasst.
%    \begin{macrocode}
        \let\@maketitle\tud@@maketitle%
        \ifx\tud@titlepagenumber\relax\gdef\tud@titlepagenumber{1}\fi%
%    \end{macrocode}
% Um einen zweispaltigen Satz für den Titelkopf ermöglichen zu können und 
% gleichzeitig dabei den TUD-Kopf zu verwenden, wird der interne Befehl 
% \cs{tud@title@twocolumn} verwendet. Damit wird der Titel selber einspaltig 
% gesetzt und direkt danach der zweispaltige Satz begonnen.
%    \begin{macrocode}
        \if@twocolumn%
          \tud@title@twocolumn{\tud@cmd@use{maketitle}[\tud@titlepagenumber]}%
        \else%
          \tud@cmd@use{maketitle}[\tud@titlepagenumber]%
        \fi%
      \fi%
    \fi%
  \endgroup%
}
%    \end{macrocode}
% \end{macro}^^A \tud@titlepagenumber
% \end{macro}^^A \tud@maketitle
% \begin{parameter}{cdtitle}
% \changes{v2.03}{2015/01/12}{neu}^^A
% \begin{parameter}{cd}
% \changes{v2.03}{2015/01/12}{neu}^^A
% \begin{parameter}{pagenumber}
% \changes{v2.02}{2014/09/05}{neu}^^A
% \begin{parameter}{cdfont}
% \changes{v2.02}{2014/09/05}{neu}^^A
% \begin{parameter}{cdhead}
% \begin{parameter}{cdfoot}
% \begin{parameter}{headlogo}
% \begin{parameter}{footlogo}
% \begin{parameter}{ddc}
% \begin{parameter}{ddchead}
% \begin{parameter}{ddcfoot}
% Die einzelnen Parameter dienen zur Einstellung der Titelseite und entsprechen
% in ihrem Verhalten den gleichnamigen Klassenoptionen.
%    \begin{macrocode}
\TUD@parameter{title}{%
  \TUD@parameter@define{cdtitle}[true]{\TUDoption{cdtitle}{#1}}%
  \TUD@parameter@let{cd}{cdtitle}%
  \TUD@parameter@define{pagenumber}{\def\tud@titlepagenumber{#1}}%
%    \end{macrocode}
% Mit \cs{TUD@parameter@define@ps} werden alle verfügbaren Parameter für die 
% Kopf- und Fußzeilen definiert.
%    \begin{macrocode}
  \TUD@parameter@define@ps{#1}%
%    \end{macrocode}
% Für den Fall, dass ein Wert nicht in der Schlüssel"=Wert"=Notation angegeben
% wurde, wird versucht, das gegebene Argument als Seitenzahl zu verwenden.
%    \begin{macrocode}
  \TUD@parameter@sethandler{\TUD@parameter@set{title}{pagenumber=#1}}%
}%
%    \end{macrocode}
% \end{parameter}^^A ddcfoot
% \end{parameter}^^A ddchead
% \end{parameter}^^A ddc
% \end{parameter}^^A footlogo
% \end{parameter}^^A headlogo
% \end{parameter}^^A cdfoot
% \end{parameter}^^A cdhead
% \end{parameter}^^A cdfont
% \end{parameter}^^A pagenumber
% \end{parameter}^^A cd
% \end{parameter}^^A cdtitle
% Die Befehle \cs{thanks} und \cs{footnotemark} werden mit \cs{robustify} 
% behandelt, um die Makros vor dem Expandieren zu schützen. Dies ist z.\,B. für
% \cs{tud@multiple@split} aber auch für weitere Felder auf dem Titel nötig.
%    \begin{macrocode}
\robustify{\thanks}%
\robustify{\footnotemark}%
%    \end{macrocode}
% \begin{macro}{\tud@maketitlepage}
% \changes{v2.02}{2014/09/05}{neu}%^^A
% \begin{macro}{\tud@titleback}
% Bei der entsprechend gewählten Layoutoption, wird eine neue Titelseite im \CD
% gesetzt. Damit auf das Standardverhalten von \KOMAScript{} zurückgestellt
% werden kann, wird der standardmäßige \cs{maketitle}-Befehl nur wenn notwendig
% überschrieben (siehe \cs{tud@layout@process}) und kann ggf. wiederhergestellt
% werden. Der Befehl selbst orientiert sich dabei in seiner Struktur an dem
% originalen \cs{maketitle}. Für die ggf. farbige Ausgabe der Titelrückseite 
% wird das Hilfmakro \cs{tud@titleback} definiert. Mit \cs{tud@titlepagenumber} 
% kann eine Seitenzahl für den Titel~-- auch über den Befehl \cs{makecover}~--
% angegeben werden.
%    \begin{macrocode}
\newcommand*\tud@titleback{}%
\newcommand*\tud@maketitlepage{%
%    \end{macrocode}
% Es wird durch die \KOMAScript-Option \opt{titlepage} unterschieden, ob eine
% Titelseite oder aber ein Titelkopf gesetzt werden soll. In Abhängigkeit davon
% werden Farben für die Schriften und den Seitenhintergrund festgelegt. Hierfür
% wird die entsprechende Layoutoption ausgewertet und entsprechend dieser die
% Werte belegt.
%    \begin{macrocode}
  \def\tud@title@fontcolor{black}%
  \ifcase\tud@layout@title\relax\or\or%
    \def\tud@title@fontcolor{HKS41}%
  \or%
    \def\tud@title@fontcolor{HKS41}%
  \or%
    \def\tud@title@fontcolor{HKS41!30}%
  \or%
    \def\tud@title@fontcolor{HKS41!30}%
  \fi%
%    \end{macrocode}
% Die Titelseite. Es existieren die Optionen \opt{full} für eine vollständig
% farbige Seite sowie \opt{lite} für farbige Schriften und \opt{on} für den
% Standardtitel.
%    \begin{macrocode}
  \begin{titlepage}%
%    \end{macrocode}
% Dieser Teil entspricht der Standarddefinition. Für das Setzen des Covers ist
% ein Großteil der Einstellungen irrelevant und wird übersrpungen.
%    \begin{macrocode}
    \ifx\tud@titlepagenumber\relax\else%
      \setcounter{page}{\tud@titlepagenumber}%
    \fi%
%    \end{macrocode}
% Dieser Teil entspricht der Standarddefinition. Für das Setzen des Covers ist
% ein Großteil der Einstellungen irrelevant und wird übersrpungen.
%    \begin{macrocode}
    \if@tud@cover\else%
      \let\footnotesize\small%
      \let\footnoterule\relax%
      \let\footnote\thanks%
      \renewcommand*\thefootnote{\@fnsymbol\c@footnote}%
      \let\@oldmakefnmark\@makefnmark%
      \renewcommand*\@makefnmark{\rlap\@oldmakefnmark}%
%    \end{macrocode}
% Das Paket \pkg{hyperref} definert im Normalfall die Fußnoten so um, dass es
% anklickbare Links erzeugt. Damit dies auf der Titelseite zu ungewollten
% Effekten führt, werden die Hyperlinks hier deaktiviert. Leider macht dies
% \pkg{hyperref} direkt beim Laden. Da die \TUDScript-Klassen aber nun mal zwei 
% unterschiedliche Titel bereitstellen und die Möglichkeit bestehen soll, dass
% Layout per später Optionswahl noch zu ändern, muss der interne Code aus
% \pkg{hyperref} hier noch einmal für den TUD-Titel aufgerufen werden.
%    \begin{macrocode}
      \tud@footnote@unhyper%
%    \end{macrocode}
% Im Gegensatz zu den \KOMAScript-Klassen wird für eine Umschlagseite die 
% Option \opt{titlepage=firstiscover} nur für den Schmutztitel \cs{extratitle}
% und nicht für die eigentlcihe Titelseite unterstützt. Für ein Cover im \CD
% ist der Befehl \cs{makecover} mit den dazugehörigen Parametern zu nutzen.
%    \begin{macrocode}
      \ifx\@extratitle\@empty%
        \if@titlepageiscoverpage%
          \ClassWarning{\tudcls@name}{%
            Using option `titlepage=firstiscover' is only\MessageBreak%
            possible with an extra title. Alternatively you can\MessageBreak%
            create a cover with `\string\makecover'%
          }%
        \fi%
      \else%
%    \end{macrocode}
% Der Teil ist aus den \KOMAScript-Klassen kopiert\dots
%    \begin{macrocode}
        \if@titlepageiscoverpage%
          \edef\titlepage@restore{%
            \noexpand\endgroup%
            \noexpand\global\noexpand\@colht\the\@colht%
            \noexpand\global\noexpand\@colroom\the\@colroom%
            \noexpand\global\vsize\the\vsize%
            \noexpand\global\noexpand\@titlepageiscoverpagefalse%
            \noexpand\let\noexpand\titlepage@restore\noexpand\relax%
          }%
          \begingroup%
          \topmargin=\dimexpr\coverpagetopmargin-1in\relax%
          \oddsidemargin=\dimexpr\coverpageleftmargin-1in\relax%
          \evensidemargin=\dimexpr\coverpageleftmargin-1in\relax%
          \textwidth=\dimexpr\paperwidth-\coverpageleftmargin%
            -\coverpagerightmargin\relax%
          \textheight=\dimexpr\paperheight-\coverpagetopmargin%
            -\coverpagebottommargin\relax%
          \headheight=0pt%
          \headsep=0pt%
          \footskip=\baselineskip%
          \@colht=\textheight%
          \@colroom=\textheight%
          \vsize=\textheight%
          \columnwidth=\textwidth%
          \hsize=\columnwidth%
          \linewidth=\hsize%
        \else%
          \let\titlepage@restore\relax%
        \fi%
        \noindent\@extratitle\next@tdpage%
      \fi%
    \fi%
%    \end{macrocode}
% Ist die Option \opt{tudbookmarks} aktiviert, werden für Umschlag- und 
% Titelseite PDF"=Lesezeichen gesetzt.
%    \begin{macrocode}
    \if@tud@bookmarks%
      \if@tud@cover%
%<*book|report>
        \tudbookmark[0]{\coverpagename}{cover}%
%</book|report>
%<*article>
        \tudbookmark[1]{\coverpagename}{cover}%
%</article>
        \ifdef{\bookmarksetup}{\bookmarksetup{startatroot}}{}%
      \else%
%<*book|report>
        \tudbookmark[0]{\titlepagename}{title}%
%</book|report>
%<*article>
        \tudbookmark[1]{\titlepagename}{title}%
%</article>
        \ifdef{\bookmarksetup}{\bookmarksetup{startatroot}}{}%
      \fi%
    \fi%
%    \end{macrocode}
% Das ganze Layout wird vertikal so verschoben, dass der Titel auf der gleichen
% Höhe steht wie die Überschriften auf Teil- und Kapitelseiten.
%    \begin{macrocode}
    \vspace*{%
      \dimexpr\tud@dim@headdiff+\tud@dim@headingsvskip-\baselineskip%
        +\parskip\relax%
    }%
    \vspace*{\pageheadingsvskip}%
    \setparsizes{\z@}{\z@}{\z@\@plus1fil}\par@updaterelative%
%    \end{macrocode}
% Damit die Farbeinstellungen der Schriften lokal begrenzt bleiben, wird dieser
% Teil des Titels in einer Gruppe (\cs{begingroup}) gesetzt.
%    \begin{macrocode}
    \begingroup%
      \usekomafont{titlepage}%
%    \end{macrocode}
% Da die Überschrift auf einer festen Position liegen soll, muss von allem, was
% über der Überschrift positioniert werden soll, die Höhe bestimmt werden, um
% einen entsprechenden Offset einfügen zu können. Dafür werden zwei temporäre
% Standardboxen verwendet. In diesen wird der gewünschte Inhalt für die beiden
% Felder \cs{@titlehead} bzw. \cs{@subject} abgelegt und die sich daraus
% kumulierende Höhe in \cs{@tempdima} gespeichert. Vor dem Setzen der Felder
% kann nun der \textbf{insgesamt} notwendige vertikaler Freiraum vor dem
% eigentlichen Titel für \cs{@titlehead} \textbf{und} \cs{@subject} abgezogen
% werden. Anschließend werden die beiden Boxen ausgegeben und der Aufbau der
% Titelseite kann beginnen.
%    \begin{macrocode}
      \setlength\@tempdima{\z@}%
      \ifx\@titlehead\@empty\else%
        \setbox0\vbox{%
          \begin{minipage}[t]{\textwidth}%
            {\usekomafont{titlehead}{\@titlehead\par\null}}%
          \end{minipage}\newline\null%
        }%
        \addtolength\@tempdima{\dimexpr\ht0+\dp0\relax}%
      \fi%
%    \end{macrocode}
% Erkennen, ob in der Betreffzeile eine Abschlussarbeit ausgegeben werden soll
% und kann.
%    \begin{macrocode}
      \tud@letsubjectthesis%
      \ifx\@subject\@empty\else%
        \setbox1\vbox{{\usekomafont{subject}{\@subject\par\null}}}%
        \addtolength\@tempdima{\dimexpr\ht1+\dp1+1pt\relax}%
      \fi%
      \vspace*{-\@tempdima}%
      \ifx\@titlehead\@empty\else\usebox0\par\fi%
      \ifx\@subject\@empty\else\usebox1\par\fi%
%    \end{macrocode}
% Nachdem bereits \cs{@titlehead} und \cs{@subject} gesetzt worden, folgen nun
% die weiteren Felder wie vom \CD~-- mehr oder weniger~-- vorgegeben auf der
% Titelseite in unterschiedlichen Größen und Schriftstärken.
%    \begin{macrocode}
      {\usekomafont{title}{\tud@title@uppercase{\@title}\par}}%
      \vskip 2ex\@plus1ex\@minus1ex%
%    \end{macrocode}
% Der Untertitel wird ggf. in der gleichen Schrift gesetzt wie der Titel,
% allerdings etwas kleiner.
%    \begin{macrocode}
      \ifx\@subtitle\@empty\else%
        {\usekomafont{subtitle}{\tud@title@uppercase{\@subtitle}\par}}%
        \vskip 2ex\@plus1ex\@minus1ex%
      \fi%
      \vskip 2\bigskipamount%
%    \end{macrocode}
% Ausgabe eines Autors oder mehrerer Autoren ggf. mit Zusatzinformationen.
%    \begin{macrocode}
      \tud@check@author%
      \tud@multiple@split{@author}\par%
      \vskip 2ex\@plus1ex\@minus1ex%
%    \end{macrocode}
% Die Titelseite unterscheidet sich ein wenig, abhängig davon ob eine eine
% Abschlussarbeit gesetzt wird oder nicht. Falls dies nicht der Fall ist, dann
% wird das angegebene bzw. das heutige Datum und ggf. die Ergänzung ausgegeben.
%    \begin{macrocode}
      \ifx\@thesis\@empty%
        \if@tud@cover\else%
          \ifx\@date\@empty\else%
            \vskip 2ex\@plus.1fill%
            {\usekomafont{date}{\@date\space\ignorespaces\@datemore\par}}%
          \fi%
        \fi%
        \vskip 2ex\@plus.25fill%
%    \end{macrocode}
% Ist eine Abschlussarbeit angegeben, so wird zusätzlich unterschieden, ob 
% diese bereits~-- gesteuert durch die Option \opt{subjectthesis}~-- in der
% Betreffzeile angegeben wurde oder aber separat und in deutlicherer Form auf
% dem Titelblatt ausgegeben werden soll. Sowohl Arbeitstyp als auf akademischer
% Grad werden dabei im Verhältnis 1:2 zwischen allen weiteren Angaben auf dem
% Titel vertikal ausgerichtet.
%    \begin{macrocode}
      \else%
        \vskip 2ex\@plus.25fill%
        \if@tud@subjectthesis\else%
          {\usekomafont{thesis}{\tud@title@uppercase{\@thesis}\par}}%
          \vskip 2ex\@plus1ex\@minus1ex%
        \fi%
%    \end{macrocode}
% Ebenfalls wird in Abhängigkeit von der Option \opt{subjectthesis} der 
% angestrebte akademische Grad ausgegeben.
%    \begin{macrocode}
        \if@tud@cover\else%
          \ifx\@graduation\@empty\else%
            \def\@tempc{%
              \@graduation%
              \ifx\@graduationabbr\@empty\else\space\@graduationabbr\fi%
            }%
            {\usekomafont{titlepage}{\graduationtext\par}}%
            \vskip 2ex\@plus1ex\@minus1ex%
            \if@tud@subjectthesis%
              {\usekomafont{subject}{\@tempc\par}}%
            \else%
              {\usekomafont{thesis}{\tud@title@uppercase{\@tempc}\par}}%
            \fi%
            \vskip 3ex\@plus1ex\@minus1ex%
          \fi%
        \fi%
        \vskip 2ex\@plus.5fill%
      \fi%
%    \end{macrocode}
% Wenn Betreuer angegeben wurden, dann werden diese auf dem Titel ausgeben. Um
% die Vorlage auch für Dissertationen verwenden zu können, ist ein Feld für
% Gutachter vorgesehen. Außerdem wird für Promotionsschriften ein Feld für
% Fachreferenten bereitgehalten. Das Format der Ausgabe ist dabei identisch.
% Darauf folgend wird der betreuende Hochschullehrer ausgegeben.
%    \begin{macrocode}
      \if@tud@cover\else%
        \tud@title@names{supervisor}%
        \tud@title@names{referee}%
        \tud@title@names{advisor}%
        \tud@title@names{professor}%
%    \end{macrocode}
% Bei einer Abschlussarbeit gehört normalerweise auch ein Abgabedatum auf den
% Titel. Sollte keines angegeben sein, so erscheint eine Warnung. Es wird an
% Stelle dessen das angegebene Enddatum oder aber das heutige Datum verwendet.
%    \begin{macrocode}
        \ifx\@thesis\@empty\else%
          \ifdefstring{\@date}{\today}{%
%    \end{macrocode}
% Durch das Paket \pkg{tudscrtask} wird das zusätzliche Feld \cs{@duedate}
% definiert, welches hier ggf. für auf ein nutzbares Datum geprüft wird.
%    \begin{macrocode}
            \ifdefvoid{\@duedate}{%
              \ClassWarning{\tudcls@name}{%
                `\string\date' was not given.\MessageBreak%
                Since a thesis is a self-contained work, an end\MessageBreak%
                date should be specified by the author.\MessageBreak%
                Nevertheless, today's date is used%
              }%
            }{%
              \ClassWarning{\tudcls@name}{%
                `\string\date' was not given.\MessageBreak%
                It's substituted by the given due date%
              }%
              \let\@date\@duedate%
            }%
          }{}%
          \ifx\@date\@empty\else%
            \vskip 2ex\@plus1ex\@minus1ex%
            \ifstr{\datetext}{}{}{\datetext\@titledelim}%
            {\usekomafont{date}{\@date\space\ignorespaces\@datemore\par}}%
%    \end{macrocode}
% Zusätzlich kann noch ein Datum der mündlichen Verteidigung angegeben werden.
%    \begin{macrocode}
            \ifx\@defensedate\@empty\else%
              \ifstr{\defensedatetext}{}{}{\defensedatetext\@titledelim}%
              {\usekomafont{date}{\@defensedate\par}}%
            \fi%
          \fi%
        \fi%
      \fi%
%    \end{macrocode}
% Zu guter Letzt kann das Standard-Verlagsfeld befüllt werden.
%    \begin{macrocode}
      \ifx\@publishers\@empty\else%
        \vskip 2ex\@plus.25fill%
        {\usekomafont{publishers}{\@publishers\par}}%
        \vskip\z@\@plus.25fill%
      \fi%
    \endgroup%
%    \end{macrocode}
% Die Rückseite des Titels wird~-- wie bei \KOMAScript~-- bloß bei zweiseitigem
% Satz ausgegeben. Für diese Klassen wird aufgrund der möglicherweise farbigen
% Rückseite das Hilfsmakro \cs{tud@titleback} benötigt.
%    \begin{macrocode}
    \let\tud@titleback\relax%
    \if@twoside%
      \@tempswatrue%
      \ifx\@uppertitleback\@empty\ifx\@lowertitleback\@empty%
        \@tempswafalse%
      \fi\fi%
      \if@tempswa%
        \gdef\tud@titleback{%
          \next@tpage%
          \begin{minipage}[t]{\textwidth}%
            \@uppertitleback%
          \end{minipage}\par%
          \vfill%
          \begin{minipage}[b]{\textwidth}%
            \@lowertitleback%
          \end{minipage}\par%
          \@thanks\let\@thanks\@empty%
        }%
      \fi%
    \fi%
%    \end{macrocode}
% Es werden Schriftart, die Schriftstärke, sowie die Farbe der Seite und des
% Kopfes und die Linienstärke des Querbalkens und die \DDC-Logos aus den
% Einstellungen übernommen.
%    \begin{macrocode}
    \def\@tempa{%
      \def\tud@pagecolor{}%
      \ifcase\tud@layout@title\relax\or%
        \def\tud@head@bar{0}%
      \or%
        \def\tud@head@bar{1}%
      \or%
        \def\tud@head@bar{2}%
      \or%
        \def\tud@pagecolor{HKS41}%
        \def\tud@head@bar{-1}%
        \def\tud@head@logocolor{white}%
        \def\tud@foot@logocolor{white}%
        \bool@std@lock{@tud@head@widebar}{true}%
      \or%
        \def\tud@pagecolor{HKS41}%
        \def\tud@head@bar{2}%
      \fi%
      \color{\tud@title@fontcolor}%
      \addtokomafont{pagenumber}{\sffamily\color{\tud@title@fontcolor}}%
      \addtokomafont{footnote}{\sffamily\color{\tud@title@fontcolor}}%
      \thispagestyle{empty.tudheadings}%
%    \end{macrocode}
% Im Kompatibilitätsmodus für die Version~v2.02 wird im Zweifelsfall die Seite 
% für den Fuß mit \DDC-Logo vergrößert.
%    \begin{macrocode}
      \iftudscrverlower{2.03}{\tud@ddc@enlargepage}{}%
    }%
%    \end{macrocode}
% Bei der Ausgabe des Covers werden lediglich die Farb- und Schriftdefinitionen
% benötigt, die Seite kann danach beendet werden.
%    \begin{macrocode}
    \if@tud@cover%
      \begingroup%
        \@tempa%
        \cleardoubleemptypage%
      \endgroup%
    \else%
%    \end{macrocode}
% Beim Titel selbst muss noch ausgewertet werden, wie die Rückseite zu 
% gestalten ist. Sollte als Klassenoption das strikte Beginnen aller Teile
% oder Kapitel auf einer rechten Seite aktiviert sein (\opt{open=right}) und
% zusätzlich bei farbigem \CD auch die Rückseiten farbig geleert werden, so
% erfolgt dies auch für die Titelseite. Die Klasse \cls{tudscrartcl} setzt
% lediglich die Titelseite farbig, die Rückseite niemals.
%    \begin{macrocode}
%<*book|report>
      \tud@cleardoublespecialpage[\tud@titleback]{%
%</book|report>
%<*article>
      \begingroup%
%</article>
        \@tempa%
        \begingroup%
          \let\thispagestyle\@gobble%
          \next@tpage%
        \endgroup%
%<*article>
        \tud@restorepagecolor%
      \endgroup%
      \tud@titleback%
%</article>
%<*book|report>
      }%
%</book|report>
%    \end{macrocode}
% Die Danksagung ist identisch zu \KOMAScript.
%    \begin{macrocode}
      \ifx\@dedication\@empty\else%
        \next@tdpage\null\vfill%
        {\centering\usekomafont{dedication}{\@dedication\par}}%
        \vskip\z@\@plus3fill%
        \@thanks\let\@thanks\@empty%
        \cleardoubleemptypage%
      \fi%
    \fi%
  \end{titlepage}%
  \if@tud@cover\else%
    \setcounter{footnote}{0}%
    \global\let\and\relax%
  \fi%
}
%    \end{macrocode}
% \end{macro}^^A \tud@titleback
% \end{macro}^^A \tud@maketitlepage
% \begin{macro}{\tud@title@uppercase}
% Für das Setzen einzelner Felder in Majuskeln werden diese beiden Befehle
% definiert. Bei der Verwendung des Befehls \cs{thanks} auf der Titelseite soll
% sichergestellt werden, dass der Inhalt selbst nicht in Majuskeln gesetzt wird.
%    \begin{macrocode}
\newcommand*\tud@title@uppercase[1]{%
  \let\@tempa\relax%
  \begingroup%
%    \end{macrocode}
% Der Befehl \cs{thanks} wird so umdefiniert, dass der Inhalt in ein temporäres
% Makro gespeichert wird. Danach wird durch das Setzen einer temporären Box der
% Befehl \cs{thanks} ausgeführt und der Inhalt gesichert. Für die Ausführung von
% \cs{tud@makeuppercase} wird das Argument von \cs{thanks} ignoriert. Nachdem 
% das gewünschte Feld in Majuskeln gesetzt wurde, wird die gespeicherte Fußnote
% ausgegeben.
%    \begin{macrocode}
    \def\thanks##1{\gdef\@tempa{##1}}%
    \let\footnote\thanks%
    \setbox0\vbox{#1}%
    \let\thanks\@gobble%
    \let\footnote\@gobble%
    \tud@makeuppercase{#1}%
  \endgroup%
  \ifx\@tempa\relax\else\thanks\@tempa\fi%
}
%    \end{macrocode}
% \end{macro}^^A \tud@title@uppercase
% \begin{macro}{\tud@title@names}
% Dieser Befehl dient zur getrennten Ausgabe von mehreren, mit dem Makro 
% \cs{and} separierten Einträgen in den Feldern \cs{@supervisor}, \cs{@referee}
% und \cs{@advisor}.
%    \begin{macrocode}
\newcommand*\tud@title@names[1]{%
  \letcs\@tempa{@#1}%
  \ifx\@tempa\@empty\else%
    \def\and{%
      \ifcsvoid{#1othername}{\par}{%
        \vskip 1ex\@plus.5ex\@minus.5ex%
        {\usekomafont{titlepage}{\@nameuse{#1othername}\par}}%
      }%
      \let\and\par%
    }%
    \vskip 1ex\@plus.5ex\@minus.5ex%
    {\usekomafont{titlepage}{\@nameuse{#1name}\par}}%
    {\usekomafont{author}{\@nameuse{@#1}\par}}%
  \fi%
}
%    \end{macrocode}
% \end{macro}^^A \tud@title@names
% \begin{macro}{\titledelimiter}
% \begin{macro}{\@titledelim}
% Dieses Makro wird für das Trennzeichen auf der Titelseite genutzt.
%    \begin{macrocode}
\newcommand*\@titledelim{:\nobreakspace}
\newcommand*\titledelimiter[1]{\gdef\@titledelim{#1}}
%    \end{macrocode}
% \end{macro}^^A \@titledelim
% \end{macro}^^A \titledelimiter
% \begin{macro}{\tud@@maketitle}
% Der eigentliche Befehl für einen Titelkopf (\opt{titlepage=no}).
%    \begin{macrocode}
\newcommand*\tud@@maketitle{%
%    \end{macrocode}
% Ab der \KOMAScript-Version 3.11c gibt es mit \cs{@makeextratitle} einen
% neuen, separaten Befehl für das Feld \cs{@extratitle}. Falls eine frühere
% Version verwendet wird, muss der Extratitel hier mit \cs{tud@extratitle}
% gesetzt werden.
%    \begin{macrocode}
  \global\@topnum=\z@%
  \vspace*{%
    \dimexpr\tud@dim@headdiff+\tud@dim@headingsvskip-\baselineskip%
      +\parskip\relax%
  }%
  \vspace*{\headingsvskip}%
  \setparsizes{\z@}{\z@}{\z@\@plus1fil}\par@updaterelative%
%    \end{macrocode}
% Damit die Farbeinstellungen der Schriften lokal begrenzt bleiben, wird dieser
% Teil des Titels in einer Gruppe (\cs{begingroup}) gesetzt. Außerdem werden 
% wie auch für die Titelseite die Hyperlinks der Fußnoten deaktiviert.
%    \begin{macrocode}
  \begingroup%
    \usekomafont{titlepage}%
    \tud@footnote@unhyper%
%    \end{macrocode}
% Ähnlich wie bei der Titelseite wird auch beim Titelkopf die Inhalte von 
% \cs{@titlehead} und \cs{@subject} so nach oben verschoben, dass der Titel 
% selbst auf der gewünschten Höhe landet.
%    \begin{macrocode}
    \setlength\@tempdima{\z@}%
    \ifx\@titlehead\@empty\else%
      \setbox0\vbox{%
        \begin{minipage}[t]{\textwidth}%
          {\usekomafont{titlehead}{\@titlehead\par\null}}%
        \end{minipage}\newline\null%
      }%
      \addtolength\@tempdima{\dimexpr\ht0+\dp0\relax}%
    \fi%
%    \end{macrocode}
% Erkennen, ob in der Betreffzeile eine Abschlussarbeit ausgegeben werden soll
% und kann. Bei einem Titelkopf ist nur diese Variante möglich.
%    \begin{macrocode}
    \TUDoptions{subjectthesis=true}%
    \tud@letsubjectthesis%
    \ifx\@subject\@empty\else%
      \setbox1\vbox{{\usekomafont{subject}{\@subject\par\null}}}%
      \addtolength\@tempdima{\dimexpr\ht1+\dp1+1pt\relax}%
    \fi%
    \vspace*{-\@tempdima}%
    \ifx\@titlehead\@empty\else\usebox0\par\fi%
    \ifx\@subject\@empty\else\usebox1\par\fi%
    {\usekomafont{title}{\tud@title@uppercase{\@title}\par}}%
    \vskip 2ex\@plus1ex\@minus1ex%
    \ifx\@subtitle\@empty\else%
      {\usekomafont{subtitle}{\tud@title@uppercase{\@subtitle}\par}}%
      \vskip 2ex\@plus1ex\@minus1ex%
    \fi%
    \vskip 2\bigskipamount%
%    \end{macrocode}
% Ausgabe eines Autors oder mehrerer Autoren ggf. mit Zusatzinformationen.
%    \begin{macrocode}
    \tud@check@author%
    \tud@multiple@split{@author}\par%
    \vskip 2ex\@plus1ex\@minus1ex%
%    \end{macrocode}
% Ausgabe von Datum, Verlag und Danksagung wie im Original.
%    \begin{macrocode}
    {\usekomafont{date}{\@date\space\ignorespaces\@datemore\par}}%
    \ifx\@publishers\@empty\else%
      \vskip 2ex\@plus1ex\@minus1ex%
      {\usekomafont{publishers}{\@publishers\par}}%
    \fi%
    \ifx\@dedication\@empty\else%
      \vskip 2ex\@plus1ex\@minus1ex%
      {\centering\usekomafont{dedication}{\@dedication\par}}%
    \fi%
%    \end{macrocode}
% Falls Felder genutzt wurden, die nur auf der Titelseite, nicht aber im
% Titelkopf ausgegeben werden, dann wird eine Warnung ausgegeben.
%    \begin{macrocode}
    \ifx\@graduation\@empty\else\tud@title@wrn{graduation}\fi%
    \ifx\@supervisor\@empty\else\tud@title@wrn{supervisor}\fi%
    \ifx\@referee\@empty\else\tud@title@wrn{referee}\fi%
    \ifx\@professor\@empty\else\tud@title@wrn{professor}\fi%
%    \end{macrocode}
% Am Ende wird noch etwas vertikaler Abstand eingefügt.
%    \begin{macrocode}
    \vskip 2\bigskipamount%
%    \end{macrocode}
% Wird das \DDC-Logo im Kompatibilitätsmodus für die Version~v2.02 gesetzt, 
% muss dafür am unteren Rand entsprechend Platz gelassen werden.
%    \begin{macrocode}
    \iftudscrverlower{2.03}{\tud@ddc@enlargepage[\titlepagestyle]}{}%
  \endgroup%
}
%    \end{macrocode}
% \end{macro}^^A \tud@@maketitle
% Es wird die Angabe einer kollaborativen Autorenschaft ermöglicht, wo für
% jeden einzelnen Autor weitere Angaben (Matrikelnummer etc.) gemacht werden
% können. Hierfür ist eine spezielle Art der Eingabe im Dokument notwendig.
% \begin{macro}{\tud@multiple@split}
% \begin{macro}{\tud@multiple@@split}
% \begin{macro}{\tud@multiple@@@split}
% \begin{macro}{\tud@multiple@field}
% Mehrere Autoren werden durch \cs{and} voneinander getrennt. Damit alle
% zusätzlichen Felder für alle Autoren einzeln nutzbar sind, muss der Inhalt
% des Feldes \cs{@author} an jedem \cs{and}-Befehl getrennt und für jeden Teil
% separat ausgewertet. Dafür wird der Befehl \cs{tud@multiple@@split} definiert,
% welcher allgemein gehalten ist um diesen auch für andere Felder nutzen zu 
% können. Das obligatorische Argument ist dabei das Feld mit dem auszuwertenden 
% Inhalt, der jeweils an \cs{and} getrennt werden soll. Dieses wird in 
% \cs{tud@multiple@field} gespeichert.
%    \begin{macrocode}
\newcommand*\tud@multiple@field{}
\newcommand*\tud@multiple@split[1]{%
  \gdef\tud@multiple@field{#1}%
  \begingroup%
    \let\and\relax%
%    \end{macrocode}
% Das Feld wird mit \cs{and} terminiert, um der Definition von
% \cs{tud@multiple@@split} in jedem Fall zu entsprechen.
%    \begin{macrocode}
    \edef\@tempa{\csname#1\endcsname\and}%
  \expandafter\endgroup%
  \expandafter\tud@multiple@@split\@tempa\relax%
}
%    \end{macrocode}
% Das Makro \cs{tud@multiple@@split} wird so definiert, dass der Befehl 
% \cs{and} als Separator für die einzelnen Argumente dient. Mit den beiden
% freigestellten Argumenten kann das Makro zur eigentlichen Ausgabe aufgerufen
% werden, welches aus dem gegebenen Feld (\cs{tud@multiple@field}) konstruiert
% wird (\cs{tud@split\meta{Feld}}). An dieses wird der jeweils aktuelle Autor 
% im ersten Argument und die restlichen im zweiten Argument übergeben.
%    \begin{macrocode}
\newcommand*\tud@multiple@@split{}%
\def\tud@multiple@@split#1\and#2\relax{%
  \expandafter\csname tud@split\tud@multiple@field\endcsname{#1}{#2}%
}
%    \end{macrocode}
% Der Befehl \cs{tud@multiple@@@split} prüft zum Schluss, ob noch weitere
% Autoren angegeben sind. Sollte dies der Fall sein, so wird der Inhalt des
% zweiten Argumentes ausgeführt und \cs{tud@multiple@@split} ein weiteres Mal
% aufgerufen, um so sequentiell alle Autoren abzuarbeiten. Dafür muss 
% \cs{tud@multiple@@@split} innerhalb des verarbeitenden Makro
% (\cs{tud@split\meta{Feld}}) aufgerufen werden.
%    \begin{macrocode}
\newcommand*\tud@multiple@@@split[2]{%
  \ifx\relax#1\relax%
    \def\@tempb{}%
  \else%
    \def\@tempb{#2\tud@multiple@@split#1\relax}%
  \fi%
  \@tempb%
}
%    \end{macrocode}
% \end{macro}^^A \tud@multiple@field
% \end{macro}^^A \tud@multiple@@@split
% \end{macro}^^A \tud@multiple@@split
% \end{macro}^^A \tud@multiple@split
% \begin{macro}{\tud@multiple@setfields}
% Der Befehl \cs{tud@multiple@setfields}\marg{Initialisierung}\marg{Teilinhalt}
% wird verwendet, um innerhalb des Feldes \cs{@tud@multiple@field} zu prüfen, ob
% sich darin bestimmte Befehle aus \cs{tud@split\\meta{Feld}@list} zum Setzen
% zusätzlicher Informationen befinden. Sollte dies der Fall sein, so werden 
% diese mit \meta{Initialisierung} initialisiert. Anschließend wird der 
% übergebene Teilinhalt mit \cs{setbox0}\cs{vtop}\marg{\#1} in eine Box
% expandiert. Dieser Trick führt dazu, dass die Felder bereits gesetzt werden,
% ohne eine Ausgabe zu erzeugen.
%    \begin{macrocode}
\newcommand*\tud@multiple@setfields[2]{%
  \letcs\@tempa{tud@split\tud@multiple@field @list}%
  \let\and\relax%
  \@for\@tempb:=\@tempa\do{%
    \ifx\@tempb\@empty\else%
      \begingroup%
        \protected@expandtwoargs\in@{\@tempb}{\@nameuse{\tud@multiple@field}}%
        \ifin@ \@tempb{#1}\fi%
      \endgroup%
    \fi%
  }%
  \begingroup%
    \let\thanks\@gobble%
    \let\footnote\@gobble%
    \setbox0\vtop{#2}%
  \endgroup%
}
%    \end{macrocode}
% \end{macro}^^A \tud@multiple@setfields
% \begin{macro}{\tud@check@author}
% \changes{v2.02}{2014/08/17}{neu}%^^A
% \begin{macro}{\tud@split@author}
% \changes{v2.02}{2014/07/25}{neu, aus Umbenennung \cs{tud@split@@author}}%^^A
% \begin{macro}{\tud@split@author@list}
% \changes{v2.02}{2014/07/25}{neu, Umbenennung \cs{tud@split@@author@list}}%^^A
% \begin{macro}{\tud@newline}
% Das Makro \cs{tud@split@author} dient der formatierten Ausgabe aller 
% Informationen für mehrere Autoren auf der Titelseite. Falls das Autorenfeld 
% nicht gesetzt wurde, wird mit \cs{tud@check@author} die standardmäßig 
% vorhandene Warnung ausgegeben und danach das Feld als leer definiert.
%    \begin{macrocode}
\newcommand*\tud@newline{}
\newcommand*\tud@check@author{%
  \ifpatchable{\@author}{\@latex@warning@no@line}{\@author\gdef\@author{}}{}%
}
\newcommand*\tud@split@author[2]{%
%    \end{macrocode}
% Zu Beginn werden alle Felder, welche innerhalb des \meta{Feld}-Argumentes
% angegeben sind, initial gelöscht (erstes Argument \cs{tud@multiple@setfields})
% Anschließend werden die Felder des aktuellen Teils gefüllt (zweites Argument 
% \cs{tud@multiple@setfields}). 
%    \begin{macrocode}
  \tud@multiple@setfields{}{#1}%
%    \end{macrocode}
% Um die Abstände definiert zu setzen, wird \cs{tud@newline} verwendet. Sobald
% auch nur eine zusätzliche Information ausgegeben werden soll, wird ein etwas
% kleiner Abstand eingefügt. Anschließend wird jede Information einfach in
% einer neuen Zeile ausgegeben.
%    \begin{macrocode}
  \def\tud@newline{\vskip .5ex\let\tud@newline\newline}%
%    \end{macrocode}
% Die Felder \cs{@author} und ggf. \cs{@authormore} werden sowohl auf der
% Titelseite als auch im Titelkopf jedoch nicht auf der Umschlagseite 
% ausgegeben.
%    \begin{macrocode}
  {\usekomafont{author}{\ignorespaces#1}}%
  \if@tud@cover\else%
    \ifx\@authormore\@empty\else%
      \tud@newline%
      \@authormore%
    \fi%
%    \end{macrocode}
% Nur auf der Titelseite erfolgt die Ausgabe von Geburtsdatum, Geburtsort,
% Matrikelnummer und Immatrikulationsjahr. Sollte ein Titelkopf verwendet
% werden, so wird für jedes gesetzte Feld eine Warnung ausgegeben.
%    \begin{macrocode}
    \ifx\@dateofbirth\@empty\else%
      \if@titlepage%
        \tud@newline%
        \ifstr{\dateofbirthtext}{}%
          {}{\dateofbirthtext\@titledelim}\@dateofbirth%
        \ifx\@placeofbirth\@empty\else%
          \space\placeofbirthtext\unskip\space\@placeofbirth%
        \fi%
      \else%
        \tud@title@wrn{dateofbirth}%
      \fi%
    \fi%
    \ifx\@matriculationnumber\@empty\else%
      \if@titlepage%
        \tud@newline%
        \ifstr{\matriculationnumbername}{}%
          {}{\matriculationnumbername\@titledelim}%
        \@matriculationnumber%
      \else%
        \tud@title@wrn{matriculationnumber}%
      \fi%
    \fi%
    \ifx\@matriculationyear\@empty\else%
      \if@titlepage%
        \tud@newline%
        \ifstr{\matriculationyearname}{}%
          {}{\matriculationyearname\@titledelim}%
        \@matriculationyear%
      \else%
        \tud@title@wrn{matriculationyear}%
      \fi%
    \fi%
  \fi%
%    \end{macrocode}
% Durch den Aufruf von \cs{tud@multiple@@@split} wird das übergebene \meta{Feld}
% sukzessive abgearbeitet.
%    \begin{macrocode}
  \tud@multiple@@@split{#2}{\vskip 2ex}%
}
%    \end{macrocode}
% Innerhalb von \cs{tud@split@author@list} werden die Feldbefehle hinterlegt, 
% die durch \cs{tud@multiple@setfield} geprüft und ggf. initialisert werden 
% sollen.
%    \begin{macrocode}
\newcommand*\tud@split@author@list{%
  \authormore,\matriculationyear,\enrolmentyear,%
  \matriculationnumber,\studentid,\matriculationid,%
  \placeofbirth,\birthplace,\dateofbirth,\birthday,%
}
%    \end{macrocode}
% \end{macro}^^A \tud@newline
% \end{macro}^^A \tud@split@author@list
% \end{macro}^^A \tud@split@author
% \end{macro}^^A \tud@check@author
% \begin{macro}{\maketitleonecolumn}
% \changes{v2.02}{2014/08/16}{einspaltiger Text obligatorisch}%^^A
% \begin{macro}{\@maketitleonecolumn}
% \changes{v2.02}{2014/08/12}{entfernt}%^^A
% \begin{macro}{\tud@title@twocolumn}
% \changes{v2.03}{2015/01/13}{Bugfix für falsche Position des Titels}%^^A
% \begin{macro}{\tud@title@twocolumn@box}
% Zum Setzen eines einspaltigen Titels mit einspaltiger Zusammenfassung bei
% einem Titelkopf~-- dank der Hilfe von Markus Kohm.%
% \footnote{\url{http://www.komascript.de/node/1699}}
%
% Der Befehl \cs{maketitleonecolumn}\oarg{Parameter}\oarg{Inhalt} kann bei
% zweispaltigem Satz dazu genutzt werden, einen einspaltigen Titel zu bekommen.
% Beim der Verwendung einer Titelseite wird lediglich der Titel sowie
% anschließend ggf. \meta{Inhalt} einspaltig ausgegeben. Beim Satz eines
% Titelkopfes wird der Befehl \cs{tud@title@twocolumn} zur Ausgabe genutzt.
%    \begin{macrocode}
\newcommand\maketitleonecolumn[2][]{%
%    \end{macrocode}
% Hier erfolgt die Unterscheidung der Ausgabe, abhängig von der Option
% \opt{titlepage}.
%    \begin{macrocode}
  \if@twocolumn%
    \if@titlepage%
      \maketitle[#1]%
      \onecolumn{#2}\twocolumn%
    \else%
      \tud@title@twocolumn{\maketitle[#1]{#2}}%
    \fi%
  \else%
    \ClassError{\tudcls@name}{%
      `\string\maketitleonecolumn' can only be used in twocolumn mode%
    }{}%
  \fi%
}
%    \end{macrocode}
% Es folgt die Verarbeitung des Inhaltes des Titelkopfes, welcher einspaltig
% gesetzt werden soll. Damit sich dieser auch über mehrere Seiten erstrecken
% kann, wird mit einer Box gearbeitet.
%    \begin{macrocode}
\newsavebox\tud@title@twocolumn@box
\newcommand\tud@title@twocolumn[1]{%
%    \end{macrocode}
% Zunächst wird auf einseitigen Satz umgeschaltet. Der zu schreibende Inhalt
% wird in einer Box gespeichert. Während des Zusammenbaus der Box sind keine
% expliziten \cs{newpage}- oder \cs{clearpage}-Befehle erlaubt.
%    \begin{macrocode}
  \onecolumn%
  \begingroup%
    \let\clearpage\relax%
    \let\newpage\relax%
%    \end{macrocode}
% Sollte der Inhalt der Box größer als die verfügbare Texthöhe des Satzspiegels 
% sein, muss aufgrund der verwendeten Box ein zusätzlicher vertikaler Abstand 
% von \val{.9}\cs{baselineskip} zur korrekten Postionierung der Überschrift
% eingefügt werden.
%    \begin{macrocode}
    \global\setbox\tud@title@twocolumn@box\vbox{%
      \vspace*{.9\baselineskip}#1\bigskip%
    }%
  \endgroup%
%    \end{macrocode}
% Als erstes muss geprüft werden, ob der Inhalt größer als der Textbereich ist. 
% Vor der Version~v2.03 gab es für das \DDC-Logo einen abweichenden Satzspiegel 
% der ggf. beachtet werden muss.
%    \begin{macrocode}
  \setlength\@tempdima{\textheight}%
  \iftudscrverlower{2.03}{%
    \tud@ddc@check%
    \ifcase\@tempb\relax\else%
      \setlength\@tempdima{\dimexpr\textheight-\tud@dim@ddcdiff\relax}%
      \@tempswatrue%
    \fi%
  }{}%
%    \end{macrocode}
% Jetzt werden der Inhalt stückweise so lange auf einzelnen Seiten ausgeben,
% bis der immer kleiner werdende Rest in der Box auf eine Seite passt.
%    \begin{macrocode}
  \ifdim\ht\tud@title@twocolumn@box>\@tempdima\relax%
    \@whiledim\ht\tud@title@twocolumn@box>\@tempdima\do{%
      \splittopskip\topskip%
      \splitmaxdepth\maxdepth%
      \vsplit\tud@title@twocolumn@box to \@tempdima%
      \clearpage%
      \setlength\@tempdima{\textheight}%
      \iftudscrverlower{2.03}{\@tempswafalse}{}%
    }%
%    \end{macrocode}
% Passt der auszugebende einspaltige Teil auf eine Seite, ist eine vertikale 
% Korrektur von \val{.9}\cs{baselineskip} nicht notwendig. In diesem Fall wird
% die Box vor der Ausgabe einfach neu und ohne diese Länge gesetzt.
%    \begin{macrocode}
  \else%
    \setbox\tud@title@twocolumn@box\vbox{#1\bigskip}%
  \fi%
%    \end{macrocode}
% Der Rest wir einfach als optionales Argument von \cs{twocolumn} ausgegeben.
%    \begin{macrocode}
  \twocolumn[\unvbox\tud@title@twocolumn@box]%
%    \end{macrocode}
% In der Version~v2.02 wird eine Warnung ausgegeben, damit der Anwender den
% Satzspiegel manuell anpassen kann. Die Verwendung von \pkg{afterpage} liefert 
% leider nicht in allen Fällen das gewollte Ergebnis, weshalb darauf verzichtet 
% wird.
%    \begin{macrocode}
  \iftudscrverlower{2.03}{%
    \if@tempswa%
      \ClassWarning{\tudcls@name}{%
        You are using a two-column title along with\MessageBreak%
        the DDC-logo. Unfortunately in this case, the\MessageBreak%
        type area hast to be adjusted manually. Please\MessageBreak%
        use `\string\enlargethispage{-\the\tud@dim@ddcdiff}'\MessageBreak%
        in both columns after the title%
      }%
    \fi%
  }{}%
}
%    \end{macrocode}
% \end{macro}^^A \tud@title@twocolumn@box
% \end{macro}^^A \tud@title@twocolumn
% \end{macro}^^A \@maketitleonecolumn
% \end{macro}^^A \maketitleonecolumn
% \begin{macro}{\tud@letsubjectthesis}
% Bei der Verwendung der Option \opt{subjectthesis} wird zum einen geprüft, ob
% das Feld \cs{@thesis} gesetzt ist und zum anderen, ob das Feld \cs{@subject}
% auch leer ist.
%    \begin{macrocode}
\newcommand*\tud@letsubjectthesis{%
  \if@tud@subjectthesis\ifx\@thesis\@empty\else%
    \ifx\@subject\@empty\else%
      \ClassWarning{\tudcls@name}{%
        The option `subjectthesis' was activated.\MessageBreak%
        The field `subject' will be overwritten with the\MessageBreak%
        given thesis%
      }%
    \fi%
    \let\@subject\@thesis%
  \fi\fi%
}
%    \end{macrocode}
% \end{macro}^^A \tud@letsubjectthesis
% \begin{macro}{\tud@title@wrn}
% Bei der Nutzung eines Titelkopfes wird bloß eine reduzierte Menge an
% Informationen ausgegeben. Darauf wird der Anwender ggf. hingewiesen.
%    \begin{macrocode}
\newcommand*\tud@title@wrn[1]{%
  \ClassWarning{\tudcls@name}{%
    `\@backslashchar#1' has to be used with\MessageBreak%
    activated titlepage option%
  }%
}
%    \end{macrocode}
% \end{macro}^^A \tud@title@wrn
% \begin{macro}{\next@tdpage}
% \changes{v2.02}{2014/05/17}{entfernt}%^^A
% \begin{macro}{\tud@extratitle}
% \changes{v2.02}{2014/06/23}{entfernt}%^^A
% Mit \TUDScript~v2.02 wurde die Bereitstellung dieser beiden Befehle aufgrund
% der zwingenden Verwendung von \KOMAScript{}~v3.12 entfernt.
% \end{macro}^^A \tud@extratitle
% \end{macro}^^A \next@tdpage
%
% \subsection{Cover}
%
% Zusätzlich zur eigentlichen Titelseite kann für Druckerzeugnisse eine 
% separate Umschlagseite erzeugt werden. Um die Gestaltung dieser anzupassen, 
% wurden Einstellungsmöglichkeiten vorgesehen, die insbesondere den verwendeten
% Satzspiegel betreffen.
%
% \begin{macro}{\makecover}
% \changes{v2.02}{2014/06/23}{Warnung für die Option \opt{titlepage} mit dem
%   Wert \val{firstiscover}}%^^A
% \changes{v2.02}{2014/08/08}{\cs{makecover} jetzt auch ohne \CD verwendbar}^^A
% \changes{v2.02}{2014/08/12}{Satzspiegel mit \cs{coverpage\dots{}margin}
% änderbar, falls CD-Layout deaktiviert}^^A
% \changes{v2.03}{2015/01/09}{Anpassungen für neue Satzspiegelerzeugung}%^^A
% \begin{macro}{\if@tud@cover}
% Der Befehl \cs{makecover} ist für eine Umschlagseite bei einem gebundenen 
% Werk zu verwenden.
%    \begin{macrocode}
\newif\if@tud@cover
\newcommand*\makecover[1][]{%
%    \end{macrocode}
% Ist die Option \opt{titlepage=firstiscover} aktiv, wird eine Warnung erzeugt.
%    \begin{macrocode}
  \if@titlepageiscoverpage%
    \ClassWarning{\tudcls@name}{%
      Using both option `titlepage=firstiscover' and\MessageBreak%
      `\string\makecover' together isn't supported. The\MessageBreak%
      option `titlepage=firstiscover' will be ignored. If you\MessageBreak%
      want to create an individual cover, you have to use\MessageBreak%
      `\string\extratitle' instead of `\string\makecover'.\MessageBreak%
      See also the documentation of KOMA-Script%
    }%
    \@titlepageiscoverpagefalse%
  \fi%
%    \end{macrocode}
% Die Erscheinung des Covers lässt sich durch wenige Optionen etwas anpassen.
% Standardmäßig ist das Cover farblos, die farbige Darstellung wird durch die 
% Klassenoption \opt{cdcover} gewählt. Außerdem wird normalerwe ein Layout
% erzeugt, welches sich genau an die Vorgaben des \CDs hält und außerdem die
% Binderandkorrektur \emph{nicht} beachtet. Optional kann das Cover im
% Satzspiegel erstellt werden. Zuletzt ist die Verwendung der Schriften des
% \CDs deaktivierbar.
%    \begin{macrocode}
  \clearpage%
  \begingroup%
    \let\thanks\@gobble%
    \let\footnote\@gobble%
    \TUD@parameter@set{cover}{#1}%
    \let\tud@layout@title\tud@layout@cover%
    \tud@layout@process%
%    \end{macrocode}
% Wird der Parameter \opt{cdlayout} aktiviert, werden die Seitenränder wie 
% durch das \CD vorgegeben eingestellt. Andernfalls entsprechen diese den 
% Einstellungen des Satzspiegels und können mit \cs{coverpage\dots margin} 
% durch den Anwender angepasst werden.
%    \begin{macrocode}
    \if@tud@cover@cdlayout%
      \let\coverpagetopmargin\tud@dim@slimmargin%
      \let\coverpagebottommargin\tud@dim@widemargin%
      \let\coverpageleftmargin\tud@dim@widemargin%
      \let\coverpagerightmargin\tud@dim@slimmargin%
      \iftudscrverlower{2.03}{%
        \def\coverpagetopmargin{.5\tud@dim@both}%
        \def\coverpagebottommargin{.5\tud@dim@both}%
      }{}%
    \fi%
%    \end{macrocode}
% Für den Fall, dass eine Umschlagseite gesetzt werden soll, die nicht im \CD 
% erscheint, wird einfach das originale \cs{maketitle} aufgerufen und dabei
% die Option \opt{titlepage=firstiscover} verwendet, um die Seitenränder 
% anpassen zu können. Außerdem werden ein mit \cs{extatitle} gegebener
% Schmutztitel ebenso ignoriert, wie Danksagung oder Einträge für die Rückseite.
%    \begin{macrocode}
    \ifcase\tud@layout@cover\relax%
      \KOMAoptions{titlepage=firstiscover}%
      \let\@extratitle\@empty%
      \let\@uppertitleback\@empty%
      \let\@lowertitleback\@empty%
      \let\@dedication\@empty%
      \ifx\tud@titlepagenumber\relax\gdef\tud@titlepagenumber{1}\fi%
      \tud@cmd@use{maketitle}[\tud@titlepagenumber]%
    \else%
%    \end{macrocode}
% Der Befehl verwendet das Layout des Titels im \CD. Allerdings werden nicht
% alle sondern nur für ein Cover relavante Angaben ausgegeben. Der Schalter
% \cs{if@tud@cover} wird innerhalb des Befehles \cs{tud@maketitle} genutzt, um 
% die Unterscheidung zu machen, ob es sich bei der aktuellen Ausgabe um das
% Cover oder aber die Titelseite handelt.
%    \begin{macrocode}
      \@tud@covertrue%
      \@titlepagetrue%
%    \end{macrocode}
% Der Satzspiegel wird hierfür äquivalent zu den \KOMAScript-Klassen gesetzt,
% wenn die Option \opt{titlepage=firstiscover} aktiviert ist.
%    \begin{macrocode}
      \edef\titlepage@restore{%
        \noexpand\endgroup%
        \noexpand\global\noexpand\@colht\the\@colht%
        \noexpand\global\noexpand\@colroom\the\@colroom%
        \noexpand\global\vsize\the\vsize%
        \noexpand\let\noexpand\titlepage@restore\noexpand\relax%
      }%
%    \end{macrocode}
% Linker und rechter Rand sowie die Textbreite werden identisch erzeugt. Für 
% das Layout im \CD wurden die Seitenränder (\cs{coverpage\dots margin}) zuvor 
% angepasst. Die Höhe der Kopfzeile wird auf \val{0pt} gesetzt, die Fußzeile 
% entspricht dem normalen Satzspiegel im \CD.
%    \begin{macrocode}
      \begingroup%
        \oddsidemargin=\dimexpr\coverpageleftmargin-1in\relax%
        \evensidemargin=\dimexpr\coverpageleftmargin-1in\relax%
        \textwidth=\dimexpr\paperwidth-\coverpageleftmargin%
          -\coverpagerightmargin\relax%
        \headheight=0pt%
        \headsep=0pt%
        \footskip=\dimexpr\tud@dim@footsep+\tud@ta@footheight\relax%
        \iftudscrverlower{2.03}{%
          \footskip=\dimexpr\tud@dim@headsep+\tud@ta@footheight%
            -1.25\baselineskip\relax%
        }{}%
%    \end{macrocode}
% Bei der Höhe des oberen Seitenrandes muss unterschieden werden, ob der
% Satzspiegel im \CD oder aber im Buchblock erscheinen soll.
%    \begin{macrocode}
        \if@tud@cover@cdlayout%
          \def\coverpagetopmargin{%
            \dimexpr\tud@dim@topmargin+\tud@dim@barheight%
              +\tud@dim@headsep\relax%
          }%
        \else%
%    \end{macrocode}
% Bei der Verwendung eines eigenen Satzspiegels kann der obere Seitenrand über 
% das Makro \cs{coverpagetopmargin} geändert werden. Das Vermeiden einer
% unerwünschten Überdeckung des Kopfes oder ein zu kleiner Fußbereich muss
% durch den Anwender korrigiert werden. Standardmäßig wird die Höhe jedoch für
% das \CD angepasst. Auch die Fußhöhe wird im Zweifelsfall gesetzt.
%    \begin{macrocode}
          \topmargin=\dimexpr\tud@dim@topmargin+\tud@dim@barheight%
            +\tud@dim@headsep-1in\relax%
          \def\@tempa{2\dimexpr\coverpagetopmargin\relax}%
          \ifx\@tempa\coverpagebottommargin\relax%
            \let\coverpagebottommargin\tud@dim@widemargin%
            \iftudscrverlower{2.03}{%
              \def\coverpagebottommargin{.5\tud@dim@both}%
            }{}%
          \fi%
        \fi%
%    \end{macrocode}
% Damit alle Einstellungen korrekt und wie gewünscht übernommen werden, muss 
% die Höhe des Textbereiches \cs{textheight} \emph{vor} der Größe des oberen 
% Seitenrande gesetzt werden. Dies hat mit der originären Definition von 
% \cs{coverpagetopmargin} zu tun. 
%    \begin{macrocode}
        \textheight=\dimexpr\paperheight-\coverpagetopmargin%
          -\coverpagebottommargin-\tud@dim@extrabottommargin\relax%
        \topmargin=\dimexpr\coverpagetopmargin-1in\relax%
        \@colht=\textheight%
        \@colroom=\textheight%
        \vsize=\textheight%
        \columnwidth=\textwidth%
        \hsize=\columnwidth%
        \linewidth=\hsize%
        \tud@dim@headdiff=\z@%
%    \end{macrocode}
% Weil bereits die Titelseite normalerweise die Seitenzahl eins bekommt, kann es
% bei der Verwendung des Paketes \pkg{hyperref} zu einer Warnung aufgrund der
% doppelt verwendeten Seitenzahl kommen. Um dies zu verhindern, werden die
% Seitenanker für das Cover deaktiviert.
%    \begin{macrocode}
        \ifdef{\hypersetup}{\hypersetup{pageanchor=false}}{}%
%    \end{macrocode}
% Die Ausgabe des Covers und ggf. das Wiederherstellen des Satzspiegels.
%    \begin{macrocode}
        \tud@maketitle%
        \clearpage%
      \csname titlepage@restore\endcsname%
    \fi%
  \endgroup%
}
%    \end{macrocode}
% \end{macro}^^A \if@tud@cover
% \end{macro}^^A \makecover
% \begin{parameter}{cdcover}
% \changes{v2.03}{2015/01/12}{neu}^^A
% \begin{parameter}{cd}
% \changes{v2.03}{2015/01/12}{neu}^^A
% \begin{parameter}{cdlayout}
% \changes{v2.02}{2014/06/23}{neu}%^^A
% \begin{macro}{\if@tud@cover@cdlayout}
% \begin{macro}{\if@tud@cover@cdlayout@locked}
% \begin{parameter}{pagenumber}
% \begin{parameter}{cdfont}
% \begin{parameter}{cdhead}
% \begin{parameter}{cdfoot}
% \begin{parameter}{headlogo}
% \begin{parameter}{footlogo}
% \begin{parameter}{ddc}
% \begin{parameter}{ddchead}
% \begin{parameter}{ddcfoot}
% Die einzelnen Parameter dienen zur Einstellung der Umschlagseite und
% entsprechen in ihrem Verhalten den gleichnamigen Klassenoptionen~--
% \opt{cdlayout} ausgenommen.
%    \begin{macrocode}
\bool@lock{@tud@cover@cdlayout}
\TUD@parameter{cover}{%
  \TUD@parameter@define{cdcover}[true]{\TUDoption{cdcover}{#1}}%
  \TUD@parameter@let{cd}{cdcover}%
  \TUD@parameter@define{cdlayout}[true]{%
    \ifstrbool{#1}{%
      \bool@set@lock{@tud@cover@cdlayout}{true}%
    }{%
      \bool@set@lock{@tud@cover@cdlayout}{false}%
    }{%
      \TUD@parameter@error{cdlayout}{true,false}%
    }%
  }%
  \TUD@parameter@define{pagenumber}{\def\tud@titlepagenumber{#1}}%
%    \end{macrocode}
% Mit \cs{TUD@parameter@define@ps} werden alle verfügbaren Parameter für die 
% Kopf- und Fußzeilen definiert.
%    \begin{macrocode}
  \TUD@parameter@define@ps{#1}%
%    \end{macrocode}
% Für den Fall, dass ein Wert nicht in der Schlüssel"=Wert"=Notation angegeben
% wurde, wird versucht, das gegebene Argument entweder als booleschen Ausdruck 
% für das Seitenlayout oder als Seitenzahl zu interpretieren.
%    \begin{macrocode}
  \TUD@parameter@sethandler{%
    \ifxnumber{#1}{%
      \TUD@parameter@set{cover}{pagenumber=#1}%
    }{%
      \TUD@parameter@set{cover}{cdlayout=#1}%
    }%
  }%
}
%    \end{macrocode}
% \end{parameter}^^A ddcfoot
% \end{parameter}^^A ddchead
% \end{parameter}^^A ddc
% \end{parameter}^^A footlogo
% \end{parameter}^^A headlogo
% \end{parameter}^^A cdfoot
% \end{parameter}^^A cdhead
% \end{parameter}^^A cdfont
% \end{parameter}^^A pagenumber
% \end{macro}^^A \if@tud@cover@cdlayout@locked
% \end{macro}^^A \if@tud@cover@cdlayout
% \end{parameter}^^A cdlayout
% \end{parameter}^^A cd
% \end{parameter}^^A cdcover
%
% \iffalse
%</class&body>
% \fi
%
% \Finale
%
\endinput
