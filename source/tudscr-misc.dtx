% \CheckSum{106}
% \iffalse meta-comment
% 
% ============================================================================
% 
%  TUD-KOMA-Script
%  Copyright (c) Falk Hanisch <tudscr@gmail.com>, 2012-2015
% 
% ============================================================================
% 
%  This work may be distributed and/or modified under the conditions of the
%  LaTeX Project Public License, version 1.3c of the license. The latest
%  version of this license is in http://www.latex-project.org/lppl.txt and 
%  version 1.3c or later is part of all distributions of LaTeX 2005/12/01
%  or later and of this work. This work has the LPPL maintenance status 
%  "author-maintained". The current maintainer and author of this work
%  is Falk Hanisch.
% 
% ----------------------------------------------------------------------------
% 
% Dieses Werk darf nach den Bedingungen der LaTeX Project Public Lizenz
% in der Version 1.3c, verteilt und/oder veraendert werden. Die aktuelle 
% Version dieser Lizenz ist http://www.latex-project.org/lppl.txt und 
% Version 1.3c oder spaeter ist Teil aller Verteilungen von LaTeX 2005/12/01 
% oder spaeter und dieses Werks. Dieses Werk hat den LPPL-Verwaltungs-Status 
% "author-maintained", wird somit allein durch den Autor verwaltet. Der 
% aktuelle Verwalter und Autor dieses Werkes ist Falk Hanisch.
% 
% ============================================================================
%
% \fi
%
% \CharacterTable
%  {Upper-case    \A\B\C\D\E\F\G\H\I\J\K\L\M\N\O\P\Q\R\S\T\U\V\W\X\Y\Z
%   Lower-case    \a\b\c\d\e\f\g\h\i\j\k\l\m\n\o\p\q\r\s\t\u\v\w\x\y\z
%   Digits        \0\1\2\3\4\5\6\7\8\9
%   Exclamation   \!     Double quote  \"     Hash (number) \#
%   Dollar        \$     Percent       \%     Ampersand     \&
%   Acute accent  \'     Left paren    \(     Right paren   \)
%   Asterisk      \*     Plus          \+     Comma         \,
%   Minus         \-     Point         \.     Solidus       \/
%   Colon         \:     Semicolon     \;     Less than     \<
%   Equals        \=     Greater than  \>     Question mark \?
%   Commercial at \@     Left bracket  \[     Backslash     \\
%   Right bracket \]     Circumflex    \^     Underscore    \_
%   Grave accent  \`     Left brace    \{     Vertical bar  \|
%   Right brace   \}     Tilde         \~}
%
% \iffalse
%%% From File: tudscr-misc.dtx
%<*driver>
\ifx\ProvidesFile\undefined\def\ProvidesFile#1[#2]{}\fi
\ProvidesFile{tudscr-misc.dtx}[%
  2015/05/18 v2.04 TUD-KOMA-Script (miscellaneous)%
]
\RequirePackage[ngerman=ngerman-x-latest]{hyphsubst}
\documentclass[english,ngerman]{tudscrdoc}
\usepackage{selinput}\SelectInputMappings{adieresis={ä},germandbls={ß}}
\usepackage[T1]{fontenc}
\usepackage{babel}
\usepackage{tudscrfonts} % only load this package, if the fonts are installed
\KOMAoptions{parskip=half-}
\CodelineIndex
\RecordChanges
\GetFileInfo{tudscr-misc.dtx}
\begin{document}
  \maketitle
  \DocInput{\filename}
\end{document}
%</driver>
% \fi
%
% \selectlanguage{ngerman}
%
% \changes{v2.02}{2014/06/23}{Unterstützung für \pkg{titlepage} entfernt}^^A
% \changes{v2.02}{2014/07/08}{Optionen nutzen \cs{FamilyKeyState}}^^A
%
% \section{Verschiedenes für die Hauptklassen}
%
% Alles, wofür sich eine separate Datei nicht lohnt, landet hier.
%
% \StopEventually{\PrintIndex\PrintChanges}
%
% \iffalse
%<*class&option>
% \fi
%
% \subsection{Fußnoten in Überschriften}
%
% \begin{option}{footnotes}
% \changes{v2.02}{2014/06/27}{neu}^^A
% \begin{macro}{\if@tud@symbolheadings}
% \changes{v2.02}{2014/06/27}{neu}^^A
% Die \KOMAScript-Option \opt{footnotes} wird um den Schlüssel 
% \val{symbolheadings} erweitert, mit welchem eingestellt werden kann, ob in den
% Überschriften Symbole für die Fußnoten anstelle von Zahlen verwendet werden
% sollen.
%    \begin{macrocode}
\newif\if@tud@symbolheadings
\TUD@key{footnotes}{%
  \TUD@set@numkey{footnotes}{@tempa}{%
    {nosymbolheadings}{0},{numberheadings}{0},%
    {symbolheadings}{1}%
  }{#1}%
  \ifx\FamilyKeyState\FamilyKeyStateProcessed%
    \ifcase\@tempa\relax% nosymbolheadings
      \@tud@symbolheadingsfalse%
    \or% symbolheadings
      \@tud@symbolheadingstrue%
    \fi%
  \fi%
}
%    \end{macrocode}
% \end{macro}^^A \if@tud@bookmarks
% \end{option}^^A symbolheadings
%
% \iffalse
%</class&option>
%<*class&body>
% \fi
%
% \begin{counter}{symbolheadings}
% \changes{v2.02}{2014/06/27}{neu}^^A
% Dieser Zähler wird bei aktivierter Option \opt{footnotes=symbolheadings} für 
% die Symboleauswahl von Fußnoten in Überschirften inkrementiert.
%    \begin{macrocode}
%<book|report>\newcounter{symbolheadings}[chapter]%
%<article>\newcounter{symbolheadings}%
%    \end{macrocode}
% \end{counter}^^A symbolheadings
%
% \iffalse
%</class&body>
%<*class&option>
% \fi
%
% \subsection{Sprungmarken}
%
% \begin{option}{tudbookmarks}
% \begin{macro}{\if@tud@bookmarks}
% Wird das Paket \pkg{hyperref} geladen, so kann die Option \opt{tudbookmarks}
% genutzt werden, um zu steuern, ob für Titel und Inhaltsverzeichnis
% automatisch ein Eintrag für die Sprungmarken erzeugt werden soll.
%    \begin{macrocode}
\newif\if@tud@bookmarks
\TUD@ifkey{tudbookmarks}{@tud@bookmarks}
%    \end{macrocode}
% \end{macro}^^A \if@tud@bookmarks
% \end{option}^^A tudbookmarks
%
% \iffalse
%</class&option>
%<*class&body>
% \fi
%
% \begin{macro}{\tudbookmark}
% \begin{macro}{\tud@footnote@unhyper}
% \begin{macro}{\tud@footnote@rehyper}
% \changes{v2.02}{2014/11/13}{entfernt}^^A
% Die Befehle zum Eintragen der Sprungmarken. Im Dokument kann auch der Befehl
% \cs{tudbookmark} vom Anwender genutzt werden, um manuell weitere Lesezeichen
% zu erzeugen.
%
% Für die Titelseite werden~-- wie auch durch das Paket \pkg{hyperref}~-- die 
% Fußnoten mit dem Befehl \cs{tud@footnote@unhyper} auf den Originalzustand
% zurückgesetzt.
%    \begin{macrocode}
\newcommand*\tudbookmark[1][]{\@gobbletwo}
\newcommand*\tud@footnote@unhyper{}
\AfterPackage{hyperref}{%
  \renewcommand*\tudbookmark[3][]{%
    \relax%
    \if@tud@bookmarks%
      \phantomsection%
      \ifblank{#1}{\pdfbookmark{#2}{#3}}{\pdfbookmark[#1]{#2}{#3}}%
    \fi%
  }%
  \renewcommand*\tud@footnote@unhyper{%
    \let\Hy@saved@footnotemark\@footnotemark%
    \let\Hy@saved@footnotetext\@footnotetext%
    \let\@footnotemark\H@@footnotemark%
    \let\@footnotetext\H@@footnotetext%
  }%
  \pdfstringdefDisableCommands{\let\NoCaseChange\@firstofone}%
}
%    \end{macrocode}
% \end{macro}^^A \tud@footnote@rehyper
% \end{macro}^^A \tud@footnote@unhyper
% \end{macro}^^A \tudbookmark
% \begin{macro}{\maketitle}
% \begin{macro}{\tableofcontents}
% Für den Titel und das Inhaltsverzeichnis werden Outline-Einträge erzeugt.
%    \begin{macrocode}
\pretocmd{\maketitle}{%
  \if@tud@bookmarks%
    \ifdef{\bookmarksetup}{\bookmarksetup{startatroot}}{}%
  \fi%
%<*book|report>
  \tudbookmark[0]{\titlepagename}{title}%
%</book|report>
%<*article>
  \tudbookmark[1]{\titlepagename}{title}%
%</article>
}{}{\tud@patch@wrn{maketitle}}
\pretocmd{\tableofcontents}{%
  \if@tud@bookmarks%
    \ifdef{\bookmarksetup}{\bookmarksetup{startatroot}}{}%
  \fi%
%<*book|report>
  \tudbookmark[0]{\contentsname}{toc}%
%</book|report>
%<*article>
  \tudbookmark[1]{\contentsname}{toc}%
%</article>
}{}{\tud@patch@wrn{tableofcontents}}
%    \end{macrocode}
% \end{macro}^^A \tableofcontents
% \end{macro}^^A \maketitle
% Das Paket \pkg{tocloft} verursacht allerlei Probleme. Wird es geladen, so 
% erscheint eine entsprechende Warnung.
%    \begin{macrocode}
\BeforePackage{tocloft}{%
  \ClassWarning{\tudcls@name}{%
    It is not recommended to use package `tocloft'.\MessageBreak%
    Loading the package will certainly lead to problems\MessageBreak%
    with table of contents and any list of floats. You\MessageBreak%
    should use the appropriate options of KOMA-Script%
  }%
}
%    \end{macrocode}
%
% \iffalse
%</class&body>
%<*execute>
% \fi
%
% \subsection{Durchreichen von Optionen und Standardoptionen}
%
% Durchreichen aller Klassenoptionen an die \KOMAScript-Klasse bzw. an die
% genutzte \TUDScript-Elternklasse.
%    \begin{macrocode}
%<*class&!inherit>
\DeclareOption*{\PassOptionsToClass{\CurrentOption}{\scrcls@name}}
%</class&!inherit>
%<*class&inherit>
\DeclareOption*{\PassOptionsToClass{\CurrentOption}{\tudinh@name}}
%</class&inherit>
%<*package&fonts>
\DeclareOption*{\KOMAoptions{\CurrentOption}}
%</package&fonts>
%<*package&poster>
\DeclareOption*{\TUDoptions{\CurrentOption}}
%</package&poster>
%    \end{macrocode}
% Es werden die Standardoptionen ausgeführt. Die Ausführung selbst wird durch
% den Befehl \cs{TUD@noworlater} innerhalb von \cs{TUD@key} auf das Ende der
% \KOMAScript-Klasse verzögert.
%    \begin{macrocode}
%<*class&!inherit>
\TUDExecuteOptions{cd=true,cdfont=true,tudbookmarks=true}
%</class&!inherit>
%    \end{macrocode}
% Für das Paket \pkg{tudscrfonts} werden die Schriften aktiviert.
%    \begin{macrocode}
%<*package&fonts>
\TUDExecuteOptions{cdfont=true}
%</package&fonts>
%<*package&poster>
\TUDExecuteOptions{cdstyle=bicolor}
%</package&poster>
\TUDProcessOptions\relax
%    \end{macrocode}
% Die korrespindierende \KOMAScript-Klasse bzw. \TUDScript-Elternklasse wird 
% geladen.
%    \begin{macrocode}
%<*class&!inherit>
\LoadClass{\scrcls@name}[2015/04/23]
%</class&!inherit>
%<*class&inherit>
\LoadClass{\tudinh@name}
%</class&inherit>
%    \end{macrocode}
%
% \iffalse
%</execute>
% \fi
%
% \Finale
%
\endinput
