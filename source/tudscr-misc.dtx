% \CheckSum{289}
% \iffalse meta-comment
% ======================================================================
%
% Das Corporate Design der TU Dresden auf Basis der KOMA-Script-Klassen
%
% ======================================================================
% This work may be distributed and/or modified under the conditions of
% the LaTeX Project Public License, version 1.3c of the license.
% The latest version of this license is in
%     http://www.latex-project.org/lppl.txt
% and version 1.3c or later is part of all distributions of LaTeX
% version 2005/12/01 or later and of this work.
% This work has the LPPL maintenance status "author-maintained".
% The current maintainer and author of this work is Falk Hanisch.
% ----------------------------------------------------------------------
% Dieses Werk darf nach den Bedingungen der LaTeX Project Public Lizenz,
% Version 1.3c, verteilt und/oder veraendert werden.
% Die neuste Version dieser Lizenz ist
%     http://www.latex-project.org/lppl.txt
% und Version 1.3c ist Teil aller Verteilungen von LaTeX
% Version 2005/12/01 oder spaeter und dieses Werks.
% Dieses Werk hat den LPPL-Verwaltungs-Status "author-maintained"
% (allein durch den Autor verwaltet).
% Der aktuelle Verwalter und Autor dieses Werkes ist Falk Hanisch.
% ======================================================================
% \fi
%
% \CharacterTable
%  {Upper-case    \A\B\C\D\E\F\G\H\I\J\K\L\M\N\O\P\Q\R\S\T\U\V\W\X\Y\Z
%   Lower-case    \a\b\c\d\e\f\g\h\i\j\k\l\m\n\o\p\q\r\s\t\u\v\w\x\y\z
%   Digits        \0\1\2\3\4\5\6\7\8\9
%   Exclamation   \!     Double quote  \"     Hash (number) \#
%   Dollar        \$     Percent       \%     Ampersand     \&
%   Acute accent  \'     Left paren    \(     Right paren   \)
%   Asterisk      \*     Plus          \+     Comma         \,
%   Minus         \-     Point         \.     Solidus       \/
%   Colon         \:     Semicolon     \;     Less than     \<
%   Equals        \=     Greater than  \>     Question mark \?
%   Commercial at \@     Left bracket  \[     Backslash     \\
%   Right bracket \]     Circumflex    \^     Underscore    \_
%   Grave accent  \`     Left brace    \{     Vertical bar  \|
%   Right brace   \}     Tilde         \~}
%
% \iffalse
%%% From File: tudscr-misc.dtx
%<*driver>
\ifx\ProvidesFile\undefined\def\ProvidesFile#1[#2]{}\fi
\ProvidesFile{tudscr-misc.dtx}[%
  2014/11/07 v2.02 TUD-KOMA-Script (miscellaneous)%
]
\RequirePackage[ngerman=ngerman-x-latest]{hyphsubst}
\documentclass[english,ngerman]{tudscrdoc}
\usepackage{selinput}\SelectInputMappings{adieresis={ä},germandbls={ß}}
\usepackage[T1]{fontenc}
\usepackage{babel}
\usepackage{tudscrfonts} % only load this package, if the fonts are installed
\KOMAoptions{parskip=half-}
\CodelineIndex
\RecordChanges
\GetFileInfo{tudscr-misc.dtx}
\begin{document}
  \maketitle
  \DocInput{\filename}
\end{document}
%</driver>
% \fi
%
% \selectlanguage{ngerman}
%
% \changes{v2.02}{2014/06/23}{Unterstützung für \pkg{titlepage} entfernt}%^^A
% \changes{v2.02}{2014/07/08}{Verwendung \cs{FamilyKeyState}}%^^A
%
% \section{Verschiedenes für die Hauptklassen}
%
% Alles, wofür sich eine separate Datei nicht lohnt, landet hier.
%
% \StopEventually{\PrintIndex\PrintChanges}
%
% \iffalse
%<*class&option>
% \fi
%
% \subsection{Fußnoten in Überschriften}
%
% \begin{option}{footnotes}
% \changes{v2.02}{2014/06/27}{neu}%^^A
% \begin{macro}{\if@tud@symbolheadings}
% \changes{v2.02}{2014/06/27}{neu}%^^A
% Die \KOMAScript-Option \opt{footnotes} wird um den Schlüssel 
% \val{symbolheadings} erweitert, mit welchem eingestellt werden kann, ob in den
% Überschriften Symbole für die Fußnoten anstelle von Zahlen verwendet werden
% sollen.
%    \begin{macrocode}
\newif\if@tud@symbolheadings
\TUD@key{footnotes}{%
  \TUD@set@numkey{footnotes}{@tempa}{%
    {nosymbolheadings}{0},{numberheadings}{0},%
    {symbolheadings}{1}%
  }{#1}%
  \ifx\FamilyKeyState\FamilyKeyStateProcessed%
    \ifcase \@tempa\relax%
      \@tud@symbolheadingsfalse%
    \or%
      \@tud@symbolheadingstrue%
    \fi%
  \fi%
}
%    \end{macrocode}
% \end{macro}^^A \if@tud@bookmarks
% \end{option}^^A symbolheadings
%
% \iffalse
%</class&option>
%<*class&body>
% \fi
%
% \begin{counter}{symbolheadings}
% \changes{v2.02}{2014/06/27}{neu}%^^A
% Dieser Zähler wird bei aktivierter Option \opt{footnotes=symbolheadings} für 
% die Symboleauswahl von Fußnoten in Überschirften inkrementiert.
%    \begin{macrocode}
%<*book|report>
\newcounter{symbolheadings}[chapter]
%</book|report>
%<*article>
\newcounter{symbolheadings}
%</article>
%    \end{macrocode}
% \end{counter}^^A symbolheadings
%
% \iffalse
%</class&body>
%<*class&option>
% \fi
%
% \subsection{Sprungmarken}
%
% \begin{option}{tudbookmarks}
% \begin{macro}{\if@tud@bookmarks}
% Wird das Paket \pkg{hyperref} geladen, so kann die Option \opt{tudbookmarks}
% genutzt werden, um zu steuern, ob für Titel und Inhaltsverzeichnis
% automatisch ein Eintrag für die Sprungmarken erzeugt werden soll.
%    \begin{macrocode}
\newif\if@tud@bookmarks
\TUD@ifkey{tudbookmarks}{@tud@bookmarks}
%    \end{macrocode}
% \end{macro}^^A \if@tud@bookmarks
% \end{option}^^A tudbookmarks
%
% \iffalse
%</class&option>
%<*class&body>
% \fi
%
% \begin{macro}{\tudbookmark}
% \begin{macro}{\tud@footnote@unhyper}
% \begin{macro}{\tud@footnote@rehyper}
% Die Befehle zum Eintragen der Sprungmarken. Im Dokument kann auch der Befehl
% \cs{tudbookmark} vom Anwender genutzt werden, um manuell weitere Lesezeichen
% zu erzeugen.
%
% Außerdem müssen für die Titelseite die Fußnoten auf den Originalzustand
% zurückgesetzt werden. Dafür werden die Befehle \cs{tud@footnote@unhyper} und
% \cs{tud@footnote@rehyper} definert.
%    \begin{macrocode}
\newcommand*\tudbookmark[1][]{\@gobbletwo}
\newcommand*\tud@footnote@unhyper{}
\newcommand*\tud@footnote@rehyper{}
\AfterPackage{hyperref}{%
  \renewcommand*\tudbookmark[3][]{%
    \relax%
    \if@tud@bookmarks%
      \phantomsection%
      \ifblank{#1}{\pdfbookmark{#2}{#3}}{\pdfbookmark[#1]{#2}{#3}}%
    \fi%
  }%
  \renewcommand*\tud@footnote@unhyper{%
    \let\Hy@saved@footnotemark\@footnotemark%
    \let\Hy@saved@footnotetext\@footnotetext%
    \let\@footnotemark\H@@footnotemark%
    \let\@footnotetext\H@@footnotetext%
  }%
  \renewcommand*\tud@footnote@rehyper{%
    \ifx\@footnotemark\H@@footnotemark%
      \let\@footnotemark\Hy@saved@footnotemark%
    \fi%
    \ifx\@footnotetext\H@@footnotetext%
      \let\@footnotetext\Hy@saved@footnotetext%
    \fi%
  }%
  \pdfstringdefDisableCommands{\let\NoCaseChange\@firstofone}%
}
\pretocmd{\maketitle}{%
  \if@tud@bookmarks%
    \ifdef{\bookmarksetup}{\bookmarksetup{startatroot}}{}%
  \fi%
%<*book|report>
  \tudbookmark[0]{\titlepagename}{title}%
%</book|report>
%<*article>
  \tudbookmark[1]{\titlepagename}{title}%
%</article>
}{}{\tud@patch@wrn{maketitle}}
\pretocmd{\tableofcontents}{%
%<*book|report>
  \if@openright\cleardoublepage\else\clearpage\fi%
%</book|report>
  \if@tud@bookmarks%
    \ifdef{\bookmarksetup}{\bookmarksetup{startatroot}}{}%
  \fi%
%<*book|report>
  \tudbookmark[0]{\contentsname}{toc}%
%</book|report>
%<*article>
  \tudbookmark[1]{\contentsname}{toc}%
%</article>
}{}{\tud@patch@wrn{tableofcontents}}
%    \end{macrocode}
% \end{macro}^^A \tud@footnote@rehyper
% \end{macro}^^A \tud@footnote@unhyper
% \end{macro}^^A \tudbookmark
%
% \subsection{Parameter für Umgebungen und mehrspaltige Texte}
%
% Diese Befehle dienen dazu, bei Umgebungen die Sprache über einen Parameter
% anzugeben sowie das Paket \pkg{multicol} verwenden zu können.
% \begin{macro}{\tud@environmenthandler}
% Hiermit kann sowohl die zu verwendende Sprache als auch die Anzahl der
% gewünschten Spalten für bestimmte Umgebungen ohne die explizite Angabe eines
% Schlüssels festgelegt werden. Momentan betrifft das die beiden Umgebungen
% \env{abstract} und \env{tudpage}.
%    \begin{macrocode}
\newcommand*\tud@environmenthandler[1]{%
  \def\@tempa{#1}%
  \@for\@tempb:=\@tempa\do{%
    \ifx\@tempb\@empty\else%
      \ifxnumber{\@tempb}{\let\tud@multicols\@tempb}{%
      \ifstr{\@tempb}{twocolumn}{\def\tud@multicols{2}}{%
        \expandafter\selectlanguage\expandafter{\@tempb}%
      }}%
    \fi%
  }%
}
%    \end{macrocode}
% \end{macro}^^A \tud@environmenthandler
% \begin{macro}{\tud@multicols}
% \begin{macro}{\tud@multicols@check}
% Im Makro \cs{tud@multicols} wird die Anzahl der gewünschten Spalten in einer
% Umgebung für die Verwendung des \pkg{multicol}"=Paketes gespeichert.
%    \begin{macrocode}
\newcommand*\tud@multicols{1}
%    \end{macrocode}
% Der Befehl \cs{tud@multicols@check} prüft, ob das Paket \pkg{multicol} 
% geladen wurde. Falls dies nicht der Fall ist, wird eine Warnung ausgegeben.
%    \begin{macrocode}
\newcommand*\tud@multicols@check{%
  \ifdef{\multicols}{}{%
    \ifnum\tud@multicols>1\relax%
      \ClassWarning{\tudcls@name}{%
        The option `columns=\tud@multicols' is only supported,\MessageBreak%
        when package `multicol' is loaded%
      }%
      \def\tud@multicols{1}%
    \fi%
  }%
}
%    \end{macrocode}
% \end{macro}^^A \tud@multicols@check
% \end{macro}^^A \tud@multicols
%
% \subsection{Bedingte Majuskeln für Überschriften}
%
% Überschriften sollen bloß in Großbuchstaben gesetzt werden, wenn auch 
% tatsächlich die Schrift DIN~Bold verwendet wird.
% \begin{macro}{\tud@makeuppercase}
% Der Befehl führt \cs{MakeTextUppercase}\marg{Text} deshalb nur aus, wenn die 
% richtige Schriftfamilie verwendet wird. Außerdem wird gleich noch die zu hoch
% liegende Grundlinie bei Überschriften der beiden Ebenen \cs{section} und
% \cs{subsection} nach unten korrigiert (derber Hack).
%    \begin{macrocode}
\newcommand*\tud@makeuppercase[1]{%
  \ifdin{\begingroup\MakeTextUppercase{#1}\endgroup}{#1}%
  \protect\vphantom{\"A\"O\"U}%
%    \end{macrocode}
% Aufgrund eines Fehlers im \LaTeX-Kernels liegt die Grundlinie für die 
% Gliederungsebene \cs{section} zu hoch. Mit dem Einfügen des vertikalen 
% Freiraums für die Umlaute würde diese automatisch nach unten verschoben. 
% Allerdings wäre das ein ziemlich übler Hack.
%    \begin{macrocode}
%% \addtokomafont{section}{\strut\ignorespaces}%
}
%    \end{macrocode}
% \end{macro}^^A \tud@makeuppercase
%
% \subsection{Erzwungene Minuskeln für Strings}
%
% Um angegebene Werte bei Schlüssel-Wert-Paaren oder Schlüsselwörter in
% bestimmten Feldern sicher erkennen zu können, werden diese zwingend in
% Kleinbuchstaben geschieben.
% \begin{macro}{\tud@lowerstring}
% Das Makro wird mit \cs{tud@lowerstring}\marg{Zielmakro}\marg{String} bennutzt.
%    \begin{macrocode}
\newcommand*\tud@lowerstring[2]{%
  \protected@edef#1{#2}%
  \lowercase\expandafter{%
    \expandafter\gdef\expandafter #1\expandafter{#1}%
  }%
}
%    \end{macrocode}
% \end{macro}^^A \tud@lowerstring
%
% \iffalse
%</class&body>
%<*manual|doc>
% \fi
%
% \section{Verschiedenes für die Dokumentationklassen}
%
% \begin{macro}{\vTUDScript}
% \begin{macro}{\vKOMAScript}
% \begin{macro}{\Attention}
% \changes{v2.02}{2014/08/16}{neu}%^^A
% \begin{macro}{\Forum}
% \begin{macro}{\notudscrartcl}
% \begin{length}{\tempdim}
% \begin{macro}{\hrfn}
% \begin{macro}{\scrguide}
% \changes{v2.02}{2014/09/04}{neu}%^^A
% \begin{macro}{\CD}
% \begin{macro}{\CDs}
% \begin{macro}{\TUD}
% \begin{macro}{\TnUD}
% \begin{macro}{\DDC}
% \begin{macro}{\Univers}
% \begin{macro}{\DIN}
% Diese Befehle stellen regelmäßig in der Quelltextdokumentatuion und im 
% Handbuch genutzte Textbausteine bereit. Dazu wird der Befehl \cs{xspace} aus 
% dem \pkg{xspace}-Paket genutzt.
%    \begin{macrocode}
\RequirePackage{xspace}[2009/10/20]
\xspaceaddexceptions{"=}
%    \end{macrocode}
% Der Befehl \cs{vTUDScript} enthält die aktuelle Versionsnummer des 
% \TUDScript-Bundles.
%    \begin{macrocode}
\newcommand*\vTUDScript{}
\AtBeginDocument{%
  \def\@tempb#1 #2 #3\relax#4\relax{\def\vTUDScript{#2}}%
  \edef\@tempa{\TUDVersion}%
  \expandafter\@tempb\@tempa\relax? ? \relax\relax%
}
\newcommand*\vKOMAScript{v3.12\xspace}
%    \end{macrocode}
% Ganz zum Schluss noch Bugfixes für unterschiedliche Pakete.
%    \begin{macrocode}
\RequirePackage{scrhack}[2014/07/07]
%    \end{macrocode}
% \dots und der Rest.
%    \begin{macrocode}
%<*manual&class>
\newcommand*\Attention[1]{\marginnote{\fbox{Achtung!}}\emph{#1}}
\NewDocumentCommand\Forum{s}{%
  \IfBooleanTF{#1}{%
    \url{http://latex.wcms-file3.tu-dresden.de/phpBB3/}%
  }{%
    \hrfn{http://latex.wcms-file3.tu-dresden.de/phpBB3/}{Forum}%
  }%
  \xspace%
}
\newcommand*\notudscrartcl{%
  F\"ur die Klasse \Class{tudscrartcl} ist diese Einstellung nicht verf\"ugbar.
}
\newlength{\tempdim}
%</manual&class>
%<*manual>
\newcommand*\hrfn[2]{\href{#1}{#2}\footnote{\scriptsize\url{#1}}}
\NewDocumentCommand\scrguide{s}{%
  \def\@tempc{%
    http://mirrors.ctan.org/macros/latex/contrib/koma-script/doc/scrguide.pdf%
  }%
  \IfBooleanTF{#1}{%
    \href{\@tempc}{\File{scrguide.pdf}}%
  }{%
    \hrfn{\@tempc}{\File{scrguide.pdf}}%
  }%
  \xspace%
}
%</manual>
\newcommand*\CD{Corporate Design\xspace}
\newcommand*\CDs{Corporate Designs\xspace}
\newcommand*\TUD{Technische Universit\"at Dresden\xspace}
\newcommand*\TnUD{Technischen Universit\"at Dresden\xspace}
\newcommand*\DDC{\mbox{DRESDEN-concept}\xspace}
\AfterPackage*{babel}{%
  \@expandtwoargs\in@{,english,}{,\bbl@loaded,}%
  \ifin@%
    \renewcommand*\CD{\foreignlanguage{english}{Corporate Design}\xspace}
    \renewcommand*\CDs{\foreignlanguage{english}{Corporate Designs}\xspace}
  \fi
}
%<*manual>
\newcommand*\Univers{\textubn{Univers}\xspace}
\newcommand*\DIN{\textdbn{DIN~BOLD}\xspace}
\newcommand*\sbnfont{\sffamily\bfseries\upshape}
\newcommand*\sbsfont{\sffamily\bfseries\slshape}
\newcommand*\textsbn{}
\newcommand*\textsbs{}
\DeclareTextFontCommand\textsbn{\sffamily\bfseries\upshape}
\DeclareTextFontCommand\textsbs{\sffamily\bfseries\slshape}
%</manual>
%    \end{macrocode}
% \end{macro}^^A \DIN
% \end{macro}^^A \Univers
% \end{macro}^^A \DDC
% \end{macro}^^A \TnUD
% \end{macro}^^A \TUD
% \end{macro}^^A \CDs
% \end{macro}^^A \CD
% \end{macro}^^A \scrguide
% \end{macro}^^A \hrfn
% \end{length}^^A \tempdim
% \end{macro}^^A \notudscrartcl
% \end{macro}^^A \Forum
% \end{macro}^^A \Attention
% \end{macro}^^A \vKOMAScript
% \end{macro}^^A \vTUDScript
%
% \iffalse
%</manual|doc>
% \fi
%
% \Finale
%
\endinput
