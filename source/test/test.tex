\RequirePackage[ngerman=ngerman-x-latest]{hyphsubst}
\documentclass[english,ngerman,cdfont=no,cd=color]{tudscrartcl}
\usepackage{selinput}
\SelectInputMappings{adieresis={ä},germandbls={ß}}
\usepackage[T1]{fontenc}
\usepackage{babel}
\usepackage{enumitem}\setlist{noitemsep}
\usepackage{tabu,booktabs,units}
\usepackage{blindtext}
\usepackage[colorlinks]{hyperref}
\usepackage[%
  translate=babel,
  automake,%
%  xindy,%={language=german-din},%
  acronym,% Abkürzungen
  symbols,% Formelzeichen
  nomain,% kein Glossar
  nogroupskip,%
]{glossaries}
\setStyleFile{\jobname-temp}
\makeglossaries

\begin{document}
\newacronym{spsp}{SPSP}{Single-Pair Shortest Path}
\newacronym{sssp}{SSSP}{Single-Source Shortest Path}
\newacronym{apsp}{APSP}{All-Pairs Shortest Path}
In der Graphentheorie wird häufig die Lösung des Problems des kürzesten Pfades 
zwischen zwei Knoten gesucht, welches auch als \gls{spsp} bezeichnet wird. 
Dieses Problem lässt sich auf die Variationen \gls{sssp} und \gls{apsp} 
erweitern. Für die Lösung von \gls{spsp}, \gls{sssp} oder \gls{apsp} kommen 
unterschiedliche Algorithmen zum Einsatz.

\printacronyms

\newglossarystyle{acronymstabu}{%
  \renewenvironment{theglossary}{
    \begin{tabu}spread 0pt[l]{@{}lX<{\strut}l@{}}
  }{
    \end{tabu}
  }
  \renewcommand*{\glossaryheader}{}%
  \renewcommand*{\glsgroupheading}[1]{}%
  \renewcommand*{\glsgroupskip}{}%
  \renewcommand*{\glossentry}[2]{%
    \glsentryitem{##1}% Entry number if required
    \glstarget{##1}{\sffamily\bfseries\glossentryname{##1}} &
    \glsentrydesc{##1} &
    ##2\tabularnewline
  }
}
\newacronym{lwusab}{LWUBSA}{%
  Das ist ein ziemlich \textbf{l}anger und \textbf{w}ahnsinnig 
  \textbf{u}mständlicher \textbf{B}egriff der hier möglichst \textbf{s}innvoll 
  \textbf{a}bgekürzt werden soll
}
\gls{lwusab}
\printacronyms[style=acronymstabu]

\newcommand{\newsymbol}[5][]{%
  \newglossaryentry{#2}{%
    name={#3},%
    symbol={\ensuremath{#4}},%
    user1={\ensuremath{\mathrm{#5}}},%
    type=symbols,%
    description={},%
    sort=#2,%
    #1%
  }%
}
\newsymbol{l}{Länge}{l}{m}
\newsymbol{m}{Masse}{m}{kg}
\newsymbol{t}{Zeit}{t}{s}
\newsymbol{f}{Frequenz}{f}{s^{-1}}
\newsymbol{F}{Kraft}{F}{m \cdot kg \cdot s^{-2}=\frac{J}{m}}

Die Einheiten für die \gls{f} und die \gls{F} werden aus den SI"=Basiseinheiten 
der Basisgrößen \gls{l}, \gls{m} und \gls{t} abgeleitet.

\printsymbols

\newglossarystyle{symbolstabu}{%
  \renewenvironment{theglossary}{%
    \begin{longtabu}spread 0pt[l]{ccX<{\strut}l}%
  }{%
    \end{longtabu}%
  }%  
  \renewcommand*{\glossaryheader}{%
    \toprule
    \bfseries Symbol & \bfseries Einheit & \bfseries Name & \bfseries Seite(n)
    \\\midrule\endhead\bottomrule\endfoot%
  }%
  \renewcommand*{\glsgroupheading}[1]{}%
  \renewcommand*{\glsgroupskip}{}%
  \renewcommand*{\glossentry}[2]{%
    \glsentryitem{##1}% Entry number if required
    \glstarget{##1}{\glossentrysymbol{##1}} &
    \glsentryuseri{##1} &
    \glossentryname{##1} &
    ##2\tabularnewline%
  }% 
}
\defglsentryfmt[symbols]{\glsgenentryfmt~\glsentrysymbol{\glslabel}}
\newcommand{\sym}[1]{\glssymbol{#1}}

Die Einheiten für die \gls{f} und die \gls{F} werden aus den SI"=Basiseinheiten 
der Basisgrößen \gls{l}, \gls{m} und \gls{t} abgeleitet.

\printsymbols[style=symbolstabu]
\end{document}

\RequirePackage[ngerman=ngerman-x-latest]{hyphsubst}
\documentclass[english,ngerman]{tudscrreprt}% andere Klassen bedingt möglich
\usepackage{babel}
\usepackage{selinput}\SelectInputMappings{adieresis={ä},germandbls={ß}}
\usepackage[T1]{fontenc}
\usepackage{fixltx2e}
\usepackage{tudscrsupervisor}
\usepackage{hyperref}
\begin{document}
\faculty{Juristische Fakultät}
\department{Fachrichtung Strafrecht}
\institute{Institut für Kriminologie}
\chair{Lehrstuhl für Kriminalprognose}
\title{%
Entwicklung eines optimalen Verfahrens zur Eroberung des
Geldspeichers in Entenhausen
}
\thesis{master}
\graduation[M.Sc.]{Master of Science}
\author{%
Mickey Mouse
\matriculationnumber{12345678}
\dateofbirth{2.1.1990}
\placeofbirth{Dresden}
\and%
Donald Duck
\matriculationnumber{87654321}
\dateofbirth{1.2.1990}
\placeofbirth{Berlin}
}
\matriculationyear{2010}
\supervisor{Dagobert Duck \and Mac Moneysac}
\professor{Prof. Dr. Kater Karlo}
\date{01.10.2014}
\makecover
\maketitle
\faculty{Juristische Fakultät}
\department{Fachrichtung Strafrecht}
\institute{Institut für Kriminologie}
\chair{Lehrstuhl für Kriminalprognose}
\title{%
Entwicklung eines optimalen Verfahrens zur Eroberung des
Geldspeichers in Entenhausen
}
\thesis{master}
\graduation[M.Sc.]{Master of Science}
\author{%
Mickey Mouse
\matriculationnumber{12345678}
\dateofbirth{2.1.1990}
\placeofbirth{Dresden}
\course{Klinische Prognostik}
\discipline{Individualprognose}
\and%
Donald Duck\matriculationnumber{87654321}
\dateofbirth{1.2.1990}
\placeofbirth{Berlin}
\course{Statistische Prognostik}
\discipline{Makrosoziologische Prognosen}
}
\matriculationyear{2010}
\issuedate{1.2.2015}
\duedate{1.8.2015}
\supervisor{Dagobert Duck \and Mac Moneysac}
\professor{Prof. Dr. Kater Karlo}
\chairman{Prof. Dr. Primus von Quack}
\newcommand\taskcontent{%
Momentan ist das besagte Thema in aller Munde. Insbesondere wird es
gerade in vielen~-- wenn nicht sogar in allen~-- Medien diskutiert.
Es ist momentan noch nicht abzusehen, ob und wann sich diese Situation
ändert. Eine kurzfristige Verlagerung aus dem Fokus der Öffentlichkeit
wird nicht erwartet.
Als Ziel dieser Arbeit soll identifiziert werden, warum das Thema
gerade so omnipräsent ist und wie man diesen Effekt abschwächen
könnte. Zusätzlich sollen Methoden entwickelt werden, wie sich ein
ähnlicher Vorgang zukünftig vermeiden ließe.
}
\begin{task}
\bigskip\taskcontent
\end{task}
\taskform[pagestyle=empty]{\taskcontent}{%
\item Recherche
\item Analyse
\item Entwicklung eines Konzeptes
\item Anwendung der entwickelten Methodik
\item Dokumentation und grafische Aufbereitung der Ergebnisse
}
\title{%
Entwicklung eines optimalen Verfahrens zur Eroberung des
Geldspeichers in Entenhausen
}
\author{Mickey Mouse\and Donald Duck}
\declaration[company=FIRMA]
\TUDoption{abstract}{multiple,section}
\begin{abstract}
Dies ist der deutschsprachige Teil der Zusammenfassung, in dem die
Motivation sowie der Inhalt der nachfolgenden wissenschaftlichen
Abhandlung kurz dargestellt werden.
\nextabstract[english]
This is the english part of the summary, in which the motivation and
the content of the following academic treatise are briefly presented.
\end{abstract}
\tableofcontents
\listoffigures
\listoftables

\end{document}