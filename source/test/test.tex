\RequirePackage[ngerman=ngerman-x-latest]{hyphsubst}
\documentclass[english,ngerman,cdfont=no,cd=color]{tudscrartcl}
\usepackage{selinput}
\SelectInputMappings{adieresis={ä},germandbls={ß}}
\usepackage[T1]{fontenc}
\usepackage{babel}
\usepackage{enumitem}\setlist{noitemsep}
\usepackage{tabu,booktabs,units}
\usepackage{blindtext}
\usepackage[colorlinks]{hyperref}
\usepackage[%
  automake,%
  translate=babel,
%  xindy,%={language=german-din},%
  acronym,% Abkürzungen
  symbols,% Formelzeichen
  nomain,% kein Glossar
  nogroupskip,%
]{glossaries}
\setStyleFile{\jobname-temp}
\makeglossaries

\begin{document}

\newacronym{spsp}{SPSP}{Single-Pair Shortest Path}
\newacronym{sssp}{SSSP}{Single-Source Shortest Path}
\newacronym{apsp}{APSP}{All-Pairs Shortest Path}

In der Graphentheorie wird häufig die Lösung des Problems des kürzesten Pfades 
zwischen zwei Knoten gesucht, welches auch als \gls{spsp} bezeichnet wird. 
Dieses Problem lässt sich auf die Variationen \gls{sssp} und \gls{apsp} 
erweitern. Für die Lösung von \gls{spsp}, \gls{sssp} oder \gls{apsp} kommen 
unterschiedliche Algorithmen zum Einsatz.

\printacronyms

\newglossarystyle{acronymstabu}{%
  \renewenvironment{theglossary}{
    \begin{tabu}spread 0pt[l]{@{}lX<{\strut}l@{}}
  }{
    \end{tabu}
  }
  \renewcommand*{\glossaryheader}{}%
  \renewcommand*{\glsgroupheading}[1]{}%
  \renewcommand*{\glsgroupskip}{}%
  \renewcommand*{\glossentry}[2]{%
    \glsentryitem{##1}% Entry number if required
    \glstarget{##1}{\sffamily\bfseries\glossentryname{##1}} &
    \glsentrydesc{##1} &
    ##2\tabularnewline
  }
}
\newacronym{lwusab}{LWUBSA}{%
  Das ist ein ziemlich \textbf{l}anger und \textbf{w}ahnsinnig 
  \textbf{u}mständlicher \textbf{B}egriff der hier möglichst \textbf{s}innvoll 
  \textbf{a}bgekürzt werden soll
}
\gls{lwusab}

\printacronyms[style=acronymstabu]

\newcommand{\newsymbol}[5][]{%
  \newglossaryentry{#2}{%
    name={#3},%
    symbol={\ensuremath{#4}},%
    user1={\ensuremath{\mathrm{#5}}},%
    type=symbols,%
    description={},%
    #1%
  }%
}
\newsymbol{l}{Länge}{l}{m}
\newsymbol{m}{Masse}{m}{kg}
\newsymbol{t}{Zeit}{t}{s}
\newsymbol{f}{Frequenz}{f}{s^{-1}}
\newsymbol{F}{Kraft}{F}{m \cdot kg \cdot s^{-2}=\frac{J}{m}}

Die Einheiten für die \gls{f} und die \gls{F} werden aus den SI"=Basiseinheiten 
der Basisgrößen \gls{l}, \gls{m} und \gls{t} abgeleitet.

\printsymbols

\newglossarystyle{symbolstabu}{%
  \renewenvironment{theglossary}{%
    \begin{longtabu}spread 0pt[l]{ccX<{\strut}l}%
  }{%
    \end{longtabu}%
  }%  
  \renewcommand*{\glossaryheader}{%
    \toprule
    \bfseries Symbol & \bfseries Einheit & \bfseries Name & \bfseries Seite(n)
    \\\midrule\endhead\bottomrule\endfoot%
  }%
  \renewcommand*{\glsgroupheading}[1]{}%
  \renewcommand*{\glsgroupskip}{}%
  \renewcommand*{\glossentry}[2]{%
    \glsentryitem{##1}% Entry number if required
    \glstarget{##1}{\glossentrysymbol{##1}} &
    \glsentryuseri{##1} &
    \glossentryname{##1} &
    ##2\tabularnewline%
  }% 
}
\defglsentryfmt[symbols]{\glsgenentryfmt~\glsentrysymbol{\glslabel}}
\newcommand{\sym}[1]{\glssymbol{#1}}

Die Einheiten für die \gls{f} und die \gls{F} werden aus den SI"=Basiseinheiten 
der Basisgrößen \gls{l}, \gls{m} und \gls{t} abgeleitet.

\printsymbols[style=symbolstabu]
\end{document}