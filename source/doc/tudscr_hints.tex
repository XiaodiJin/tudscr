\chapter{Praktische Tipps \& Tricks}
\label{sec:tips}
\newcommand*\TaT{\hyperref[sec:tips]{Tipps \& Tricks}}
\section{\NoCaseChange{\hologo{LaTeX}}-Editoren}
\label{sec:tips:editor}
Hier werden die gängigsten Editoren zum Erzeugen von \hologo{LaTeX}"=Dateien 
genannt. Ich persönlich bin mittlerweile sehr überzeugter Nutzer von 
\Application{\hologo{TeX}studio}, da dieser viele Unterstützungs- und 
Assistenzfunktionen bietet. Neben diesen gibt es noch weitere, gut nutzbare 
\hologo{LaTeX}-Editoren. Egal, für welchen Editor man sich letztendlich 
entscheidet, sollte dieser auf jeden Fall eine Unicode"=Unterstützung~(UTF-8) 
enthalten:
%
\begin{itemize}
\item \Application{\hologo{TeX}maker}
\item \Application{Kile}
\item \Application{\hologo{TeX}works}
\item \Application{\hologo{TeX}lipse}~-- Plug-in für \Application{Eclipse}
\item \Application{\hologo{TeX}nicCenter}
\item \Application{WinEdt}
\item \Application{LEd}~-- früher \hologo{LaTeX}~Editor
\item \Application{\hologo{LyX}}~-- grafisches Front"~End für \hologo{LaTeX}
\end{itemize}
%
Für den Editor\Application{\hologo{TeX}studio} werden im GitHub-Repository
\hrfn{https://github.com/tud-cd/tudscr/tree/master/addon/texstudio}{tudscr/addon/texstudio}
Dateien zur Erweiterung der automatischen Befehlsvervollständigung für das 
\TUDScript-Bundle bereitgestellt. Diese müssen unter Windows in
\Path{\%APPDATA\%\textbackslash texstudio} beziehungsweise unter unixoiden 
Betriebssystemen in \Path{.config/texstudio} eingefügt werden.

Außerdem findet man für die Verwendung des \TUDScript-Bundles zusammen mit 
\Application{\hologo{LyX}} unter
\hrfn{https://github.com/tud-cd/tudscr/tree/master/addon/tudscr4lyx}{tudscr/addon/tudscr4lyx}
die notwendigen Dateien und ein Minimalbeispiel. Die Layout-Dateien müssen 
dafür im \Application{\hologo{LyX}}"=Installationspfad in den passenden 
Unterordner kopiert werden. Dieser ist bei Windows
\Path{\%PROGRAMFILES(X86)\%\textbackslash{}LyX~2.0\textbackslash{}Resources\textbackslash{}layouts}
beziehungsweise bei unixoiden Betriebssystemen \Path{/usr/share/lyx/layouts}.



\section{Literaturverwaltung in \NoCaseChange{\hologo{LaTeX}}}
\ChangedAt{v2.02!\TaT: Literaturverwaltung}
%
Die simpelste Variante, eine Literaturdatenbank in \hologo{LaTeX} zu verwalten, 
ist dies mit dem verwendeten \hologo{LaTeX}-Editors manuell zu erledigen. Dies 
ist allerdings nicht sonderlich komfortabel. Einfacher ist es, dies mit einer 
darauf spezialisierten Anwendung zu bewerkstelligen. Für die Referenzverwaltung 
in \hologo{LaTeX} gibt es dafür zwei sehr gute Programme
%
\begin{itemize}
\item \Application{Citavi}
\item \Application{JabRef}
\end{itemize}
%
Das Programm \Application{Citavi} ermöglicht den Import von bibliographischen 
Informationen aus dem Internet. Allerdings sind diese teilweise unvollständig 
oder mangelhaft. Mit \Application{JabRef} hingegen muss die Literaturdatenbank 
manuell erstellt werden. Allerdings lassen sich einzelne Einträge aus 
.bib-Dateien sehr importieren. Beide Anwendungen unterstützen den Export 
beziehungsweise die Erstellung von Datenbanken im Stil von \Package{biblatex}. 
Für \Application{JabRef} muss diese durch den Anwender explizit aktiviert 
werden.\footnote{Optionen/Einstellungen/Erweitert/BibLaTeX-Modus}

Zur Verwendung der beiden Programme in Verbindung mit \Package{biblatex} und 
\Application{biber} gibt es ein gutes Tutorial unter diesem
\href{http://www.suedraum.de/latex/stammtisch/degenkolb_latex_biblatex_folien-final.pdf}{Link}.



\section{Finden von unbekannten \NoCaseChange{\hologo{LaTeX}}-Symbolen}
\index{Symbole}
Für \hologo{LaTeX} stehen jede Menge Symbole zur Verfügung, die allerdings 
nicht immer einfach zu finden sind. In der Zusammenfassung
\hrfn{http://mirrors.ctan.org/info/symbols/comprehensive/symbols-a4.pdf}{\File{symbols-a4.pdf}}
werden viele Symbole aus mehreren Paketen aufgeführt. Allerdings ist das 
Auffinden eines speziellen Symbols nicht sehr komfortabel. Alternativ dazu kann 
dieser \hrfn{http://detexify.kirelabs.org/classify.html}{Link} verwendet 
werden. Auf dieser Seite wird das gesuchte Symbol einfach gezeichnet, die dazu 
ähnlichsten werden zurückgegeben.



\section{Zeilenabstände in Überschriften}
\label{sec:tips:headings}
Mit dem Paket \Package{setspace} kann der Zeilenabstand beziehungsweise der 
Durchschuss innerhalb des Dokumentes geändert werden. Sollte dieser erhöht 
worden sein, können die Abstände bei mehrzeiligen Überschriften als zu groß 
erscheinen. Um dies zu korrigieren kann mit dem Befehl \Macro{addtokomafont}%
\PParameter{disposition}\PParameter{\Macro*{setstretch}\PParameter{1}} der 
Zeilenabstand aller Überschriften auf einzeilig zurückgeschaltet werden. Soll 
dies nur für eine bestimmte Gliederungsebene erfolgen, so ist 
\PParameter{disposition} durch das entsprechende Schriftelement zu ersetzen.



\section{Unterdrückung des Einzuges eines Absatzes}
\index{Absatzauszeichnung}
Verwendet man~-- wie es aus typographischer Sicht zumeist sinnvoll ist~-- 
Einzüge und keine vertikalen Abstände zur Auszeichnung von Absätzen im Dokument
(\Option{parskip}[false]), kann es vorkommen, dass ein bestimmter Absatz~-- 
beispielsweise der nach einer gewissen Umgebung folgende~-- ungewollt 
eingerückt ist. Dies kann sehr einfach behoben werden, indem direkt zu Beginn 
des Absatzes das Makro \Macro{noindent} verwendet wird. Möchte man das für 
bestimmte Umgebungen oder Befehle automatisiert gestalten, ist das Paket
\Package{noindentafter} zu empfehlen.



\section{Unterbinden des Zurücksetzens von Fußnoten}%
\label{sec:tips:counter}
\index{Fußnoten}
Oft taucht die Frage auf, wie man auch über Kapitel fortlaufende Fußnoten 
erhalten kann. Dies ist sehr einfach mit dem Paket \Package{chngcntr} möglich. 
Nach dem Laden des Paketes, kann das Rücksetzen des Zählers nach einem Kapitel 
mit \Macro*{counterwithout*}\PParameter{footnote}\PParameter{chapter} 
deaktiviert werden. Auch andere \hologo{LaTeX}-Zähler~-- wie beispielsweise der 
bereits vorgestellte \Counter{symbolheadings}~-- lassen sich mit diesem 
Paket manipulieren.



\section{URL-Umbrüche im Literaturverzeichnis mit \Package{biblatex}}
\index{Literaturverzeichnis}
%
\ChangedAt{v2.02!\TaT: URL-Umbrüche im Literaturverzeichnis}
Wird das Paket \Package{biblatex} verwendet, kann es unter Umständen dazu 
kommen, das eine URL nicht vernünftig umbrochen werden. Ist dies der Fall, 
können die Zählern \Counter*{biburlnumpenalty}, \Counter*{biburlucpenalty} und 
\Counter*{biburllcpenalty} erhöht werden. Die möglichen Werte liegen zwischen 0 
und 10000, wobei es bei höheren Werte der Zähler zu mehr URL-Umbrüchen an 
Ziffern (\Counter*{biburlnumpenalty}), Groß- (\Counter*{biburlucpenalty}) und 
Kleinbuchstaben (\Counter*{biburllcpenalty}) kommt. Genaueres hierzu ist der 
Dokumentation des \Package{biblatex}"=Paketes zu entnehmen.



\section{Bezeichnungen der Gliederungsebenen durch \Package{hyperref}}
\index{Querverweise}
%
\ChangedAt{v2.02!\TaT: Bezeichnungen der Gliederungsebenen}
Das Paket \Package{hyperref} stellt für Querverweise unter anderem den Befehl 
\Macro*{autoref}\Parameter{label} zur Verfügung. Mit diesem wird~-- im 
Gegensatz zur Verwendung von \Macro*{ref}~-- bei einer Referenz nicht nur die 
Nummerierung selber sondern auch das entsprechende Element wie Kapitel oder 
Abbildung vorangestellt. Bei der Benennung des referenzierten Elementes wird 
sequentiell geprüft, ob das Makro \Macro*{}\PName{Element}\PValue{autorefname}
oder \Macro*{}\PName{Element}\PValue{name} existiert. Soll die Bezeichnung 
eines Elementes geändert werden, muss man den entsprechende Bezeichner anpassen.
%
\begin{Example}
Bezeichnungen von Gliederungsebenen können folgendermaßen verändert werden.
\begin{Code}
\renewcaptionname{ngerman}{\sectionautorefname}{Unterkapitel}
\renewcaptionname{ngerman}{\subsectionautorefname}{Abschnitt}
\renewcaptionname{ngerman}{\subsubsectionautorefname}{Unterabschnitt}
\end{Code}
\end{Example}



\section{Setzen von Einheiten mit \Package{siunitx}}
\label{sec:tips:siunitx}
\index{Einheiten}
Wenn \Package*{siunitx} in einem deutschsprachigen Dokument genutzt soll
werden, muss zumindest mit \Macro*{sisetup}\PParameter{locale = DE} die 
richtige Lokalisierung angegeben werden. Sollen auch die Zahlen richtig 
formatiert sein, müssen weitere Einstellungen vorgenommen werden. Die meiner 
Meinung nach besten sind die folgenden.
%
\begin{quoting}
\begin{Code}
\sisetup{%
  locale = DE,%
  input-decimal-markers={,},%
  input-ignore={.},%
  group-separator={\,},%
  group-minimum-digits=3%
}
\end{Code}
\end{quoting}
%
Das Komma kommt als Dezimaltrennzeichen zum Einsatz. Des Weiteren werden Punkte 
innerhalb der Zahlen ignoriert und eine Gruppierung von jeweils drei Ziffern 
vorgenommen. Alternativ zu diesem Paket kann übrigens auch \Package{units} 
verwendet werden.



\section{Leer- und Satzzeichen nach \NoCaseChange{\hologo{LaTeX}}-Befehlen}%
\label{sec:tips:xspace}
\index{Typographie}
Normalerweise \enquote{schluckt} \hologo{LaTeX} die Leerzeichen nach einem 
Makro ohne Argumente. Dies ist jedoch nicht immer~-- genau genommen in den 
seltensten Fällen~-- erwünscht. Für dieses Handbuch ist beispielsweise der 
Befehl \Macro*{TUD} definiert worden, um \enquote{\TUD{}} nicht ständig 
ausschreiben zu müssen. Um sich bei der Verwendung des Befehl innerhalb eines 
Satzes sich für den Erhalt eines folgenden Leerzeichens das Setzen der 
geschweiften Klammer nach dem Befehl zu sparen (\Macro*{TUD}\PParameter{}), 
kann \Macro*{xspace} aus dem Paket \Package{xspace} genutzt werden. Damit wird 
ein folgendes Leerzeichen erhalten. Der Befehl \Macro*{TUD} ist wie folgt 
definiert:
%
\begin{quoting}
\begin{Code}
\newcommand*\TUD{Technische Universit\"at Dresden\xspace}
\end{Code}
\end{quoting}
%
Das Paket \Package{xpunctuate} erweitert die Funktionalität nochmals. Damit 
können auch Abkürzungen so definiert werden, dass ein versehentlicher Punkt 
ignoriert wird:
%
\begin{quoting}
\begin{Code}
\newcommand*\zB{z.\,B\xperiod}
\end{Code}
\end{quoting}



\section{Automatisiertes Einbinden von \Application{Inkscape}-Grafiken }
\label{sec:tips:svg}
\index{Grafiken}
In \hrfn{http://www.ctan.org/pkg/svg-inkscape}{\Package{svg-inkscape}} wird das 
automatisierte Einbinden von \Application{Inkscape}-Grafiken in ein 
\hologo{LaTeX}"=Dokument erläutert. Hier wird ein daraus abgeleiteter und 
verbesserter Ansatz vorgestellt. Nutzer von unixartigen Systemen können 
alternativ auch das Paket \Package{svg} nutzen, welches den folgend erläuterten 
Befehl \Macro{includesvg} definiert.

Die mit \Application{Inkscape} erstellte Grafik soll automatisch kompiliert und 
eingebunden werden. Dies soll allerdings nicht bei jeder Kompilierung des 
Hauptdokumentes erfolgen, sondern lediglich, wenn die originale Bilddatei 
geändert beziehungsweise aktualisiert wurde. Hierfür wird \Package{filemod} 
verwendet. Die automatisierte Übersetzung einer Grafik im SVG"~Format in eine 
PDF"~Datei und die daran anschließende Einbindung dieser in das Dokument ist 
mit der Definition von \Macro{includesvg}\OParameter{Breite}\Parameter{Datei} 
in der Präambel des Dokumentes wie folgt möglich:
%
\makeatletter
\label{macros:includesvg}%
\Hy@raisedlink{\hyperdef{\jobname}{macros:includesvg}{}}%
\makeatother
\begin{quoting}
\begin{Code}[escapechar=§]
\usepackage{filemod}
\newcommand*{\includesvg}[2][\textwidth]{%
  \def\svgwidth{#1}
  \filemodCmp{#2.pdf}{#2.svg}{}{%
    \immediate\write18{%
      inkscape -z -D --file=#2.svg --export-pdf=#2.pdf --export-latex
    }%
  }%
  \input{#2.pdf_tex}%
}
\end{Code}
\end{quoting}
%
Mit \Macro*{immediate}\Macro*{write18}\Parameter{externer Aufruf} wird das 
zwischenzeitliche Ausführen eines externen Programms beim Durchlauf von 
\hologo{pdfLaTeX}~-- in diesem Fall von \File{inkscape.exe}~-- möglich. Damit 
der externe Aufruf auch tatsächlich durchgeführt wird, muss \hologo{pdfLaTeX} 
mit der Option \Path{-{}-shell-escape} ausgeführt werden. Außerdem muss der 
Pfad zur Datei \File{inkscape.exe} dem System bekannt sein.%
\footnote{%
  Genauer gesagt, muss der Pfad zu \File{inkscape.exe} in der 
  \texttt{PATH}-Variable des Betriebssystems enthalten sein.
}
Bei der Verwendung des Befehls \Macro{includesvg} \emph{muss} der Dateiname 
ohne Endung angegeben werden. Die einzubindende SVG"~Datei sollte sich hierbei 
im gleichen Pfad wie das Hauptdokument befinden. Ist die SVG"~Datei in einem 
Unterordner relativ zum Pfad des Hauptdokumentes, kann dieser einfach mit 
\Macro{includesvg}\PParameter{\PName{Ordner}/\PName{Datei}} im Argument 
angegeben werden.



\section{Warnung wegen zu geringer Höhe der Kopf-/Fußzeile}
\label{sec:tips:headline}
Wird das Paket \Package{setspace} verwendet, kann es passieren, dass nach der 
Änderung des Zeilenabstandes \emph{innerhalb} des Dokumentes eine oder beide 
der folgenden Warnungen erscheinen:
%
\begin{quoting}
\begin{Code}
scrlayer-scrpage Warning: \headheight to low.
scrlayer-scrpage Warning: \footheight to low.
\end{Code}
\end{quoting}
%
Dies liegt an dem durch den vergrößerten Zeilenabstand erhöhten Bedarf für die
Kopf- und Fußzeile, die Höhen können in diesem Fall direkt mit der Verwendung 
von \Macro{recalctypearea} angepasst werden. Allerdings ändert das den 
Satzspiegel im Dokument, was eine andere und durchaus berechtigte Warnung von 
\Package{typearea} zur Folge hat. Falls die Änderung des Durchschusses wirklich 
nötig ist, sollte dies in der Präambel des Dokumentes einmalig passieren. Dann 
entfallen auch die Warnungen.



\section{Warnung beim Erzeugen des Inhaltsverzeichnisses}
\index{Inhaltsverzeichnis}%
%
\ChangedAt{v2.02!\TaT: Warnung beim Inhaltsverzeichnis}
Erstellt man ein Inhaltsverzeichnis für ein Dokument mit einer dreistelligen 
Seitenanzahl, so erhält man bei der Verwendung von \Macro*{tableofcontents} 
viele Warnungen mit der Meldung:
%
\begin{quoting}
\begin{Code}
overfull \hbox
\end{Code}
\end{quoting}
%
Das liegt daran, dass die Seitenzahl in einer Box mit der Breite 
\Macro*{@pnumwidth} gesetzt wird. Der hierfür standardmäßig verwendete Wert von 
\PValue{1.55em} ist in diesem Fall zu klein. Dieser kann folgendermaßen 
geändert werden:
%
\begin{quoting}
\begin{Code}
\makeatletter
\renewcommand*\@pnumwidth{1.7em}
\makeatother
\end{Code}
\end{quoting}
%
Dabei sollte der eingesetzte Wert nicht zu groß ausfallen.



\section{Änderung des Papierformates}
\index{Papierformat}
Es kann vorkommen, dass man innerhalb eines Dokumentes kurzzeitig das 
Papierformat ändern möchte, um beispielsweise eine Konstruktionsskizze in der 
digitalen PDF"~Datei einzubinden. Dabei ist es sowohl möglich, lediglich die 
Ausrichtung mit \Option{paper}[landscape] in ein Querformat zu ändern, als 
auch die Größe des Papierformates selber.
%
\begin{Example}
Ein Dokument im A4"~Format soll kurzzeitig auf ein A3"=Querformat geändert 
werden. Das folgende Minimalbeispiel zeigt, wie das Papierformat mit den 
Mitteln von \KOMAScript{} geändert werden kann.
\begin{Code}
\documentclass[paper=a4,pagesize]{tudscrreprt}
\usepackage{selinput}
\SelectInputMappings{adieresis={ä},germandbls={ß}}
\usepackage[T1]{fontenc}
\usepackage[ngerman]{babel}
\usepackage{blindtext}

\begin{document}
\chapter{Überschrift Eins}
\Blindtext

\cleardoublepage
\storeareas\PotraitArea% speichert den aktuellen Satzspiegel
\KOMAoptions{paper=A3,paper=landscape,DIV=current}
\chapter{Überschrift Zwei}
\Blindtext

\cleardoublepage
\PotraitArea% lädt den gespeicherten Satzspiegel
\chapter{Überschrift Drei}
\Blindtext
\end{document}
\end{Code}
\end{Example}



\section{Einrückung von Tabellenspalten verhindern}%
\label{sec:tips:table}
\index{Tabellen}
Bei Tabellen wird vor und nach Spalte durch \hologo{LaTeX} ein horizontaler 
Abstand von \Length{tabcolsep} gesetzt. Dies geschieht auch \emph{vor} der 
ersten und \emph{nach} der letzten Spalte. Diese Einrückung an den äußeren 
Rändern kann insbesondere bei Tabellen, welche die komplette Seitenbreite 
überspannen, stören. Das Paket \Package{tabularborder} versucht, dieses Problem 
automatisiert zu beheben, ist jedoch nicht zu allen \hologo{LaTeX}-Paketen für 
den Tabellensatz kompatibel.

Dieses Problem lässt sich auch manuell durch den Anwender lösen. Bei der 
Deklaration einer Tabelle kann mit \PValue{@}\PParameter{\dots} vor und 
nach dem Spaltentyp angegeben werden, was anstelle von \Length{tabcolsep} vor 
beziehungsweise nach der eigentlichen Spalte eingeführt werden soll. Dies kann 
für das Entfernen der Einrückungen genutzt werden.
%
\begin{Example}
Eine Tabelle mit zwei Spalten, wobei bei einer die Breite automatisch berechnet 
wird, soll über die komplette Textbreite gesetzt werden. Dabei soll der Rand 
vor der ersten und nach der letzten entfernt werden.
\begin{Code}[escapechar=§]
\begin{tabularx}{\textwidth}{@{}lX@{}}
§\dots§ & §\dots§ \tabularnewline
§\dots§
\end{tabularx}
\end{Code}
\end{Example}






\section{Warnungen bei der Verwendung von \Package{multicol}}
%
\ChangedAt{v2.02!\TaT: Warnung mit \Package*{multicol}}
Das einzig momentan bekannte Problem der \TUDScript"=Klasse tritt in Verbindung 
mit dem Paket \Package[?]{multicol} auf. Dieses greift~-- ähnlich zu dem für 
die Seitenstile verwendeten Paket \Package{scrlayer-scrpage}~-- sehr stark in 
die Ausgaberoutine von \hologo{LaTeXe} ein. Bei Spaltenumbrüchen innerhalb der 
\Environment*{multicols}"=Umgebung kommt es sehr häufig zu folgender Warnung:
%
\begin{quoting}
\begin{Code}
Underfull \hbox (badness 10000) has occurred while \output is active
\end{Code}
\end{quoting}
%
Wird \Package{scrlayer-scrpage} nicht verwendet, unterbleibt die Warnung. Beide 
Pakete haben in der Kombination miteinander augenscheinlich ein Problem. Die 
Autoren der beiden Pakete wurden über dieses Problem informiert, signalisierten 
jedoch verständlicherweise wenig Bereitschaft, der Ursache auf den Grund zu 
gehen. Leider kann für die \TUDScript-Klassen auf \Package{scrlayer-scrpage} 
nicht verzichtet werden, es ist von essentieller Bedeutung für die Seitenstile 
im \CD der \TnUD. Am Ausgabeergebnis ändert sich nichts. Im Zweifelsfall werden 
jedoch eine Menge bedeutungsloser Warnungen generiert. 

Es gibt allerdings eine Möglichkeit, diese zu unterdrücken. Hierfür wird auf 
die Funktionalitäten von \Package{etoolbox} zurückgegriffen, was ohnehin von 
den \TUDScript-Klassen geladen wird. In der Präambel kann folgender Quelltext 
verwendet werden:
%
\begin{quoting}
\begin{Code}
\makeatletter
\apptocmd{\prepare@multicols}{\hbadness10000}{}{}
\makeatother
\end{Code}
\end{quoting}
%
Dadurch werden die durch \hologo{LaTeXe} generierten Warnungen innerhalb der 
\Environment*{multicols}"=Umgebung deaktiviert. Es muss dabei beachtet werden, 
dass dies für alle Boxen~-- und nicht nur jene beim Spaltenumbruch~-- gilt.



\section{Lokale Änderungen von Befehlen und Einstellungen}
\index{Gruppierungen}
%
\ChangedAt{v2.02!\TaT: Lokale Änderungen}
Ein zentraler Bestandteil von \hologo{LaTeX} ist die Verwendung von Gruppen 
oder Gruppierungen. Innerhalb dieser bleiben alle vorgenommenen Änderungen an 
Befehlen, Umgebungen oder Einstellungen lokal. Dies kann sehr nützlich sein, 
wenn beispielsweise das Verhalten eines bestimmten Makros einmalig oder 
innerhalb von selbst definierten Befehlen oder Umgebungen geändert werden, im 
Normalfall jedoch die ursprüngliche Funktionalität behalten soll.
\begin{Example}
\index{Schriftauszeichnung}
Der Befehl \Macro*{emph} wird von \hologo{LaTeX} für Hervorhebungen im Text 
bereitgestellt und führt normalerweise zu einer kursiven oder~-- falls kein 
Schriftschnitt mit echten Kursiven vorhanden ist~-- kursivierten Auszeichnung. 
Soll nun in einem bestimmten Abschnitt die Auszeichnung mit fetter Schrift 
erfolgen, kann der Befehl \Macro*{emph} innerhalb einer Gruppierung geändert 
und verändert werden. Wird diese beendet, verhält sich der Befehl wie gewohnt.
\begin{Code}
In diesem Text wird ein bestimmtes \emph{Wort} hervorgehoben.

\begingroup
\renewcommand*\emph[1]{\textbf{#1}}%
In diesem Text wird ein bestimmtes \emph{Wort} hervorgehoben.
\endgroup

In diesem Text wird ein bestimmtes \emph{Wort} hervorgehoben.
\end{Code}
\end{Example}
Eine Gruppierung kann entweder mit \Macro*{begingroup} und \Macro*{endgroup} 
oder einfach mit einem geschweiften Klammerpaar \texttt{\{\dots\}} definiert 
werden.



\section{Platzierung von Gleitobjekten}
\label{sec:tips:floats}
\index{Gleitobjekte|?}
Die standardmäßige Platzierung von Gleitobjekten wird durch die im Folgenden 
aufgezählten Befehle beeinflusst. Diese können mit 
\Macro*{renewcommand*}\Parameter{Befehl}\Parameter{Wert} geändert werden.

\begin{Declaration}{\Macro{floatpagefraction}}[0\floatpagefraction]
\begin{Declaration}{\Macro{dblfloatpagefraction}}[0\dblfloatpagefraction]
\printdeclarationlist*%
%
Der Wert gibt die relative Größe eines Gleitobjektes bezogen auf die Texthöhe 
(\Macro*{textheight}) an, die mindestens erreicht sein muss, damit für dieses 
gegebenenfalls vor dem Beginn eines neuen Kapitels eine separate Seite erzeugt 
wird. Dabei wird einspaltiges (\Macro*{floatpagefraction}) und zweispaltiges 
(\Macro*{dblfloatpagefraction}) Layout unterschieden. Der Wert für beide 
Befehle sollte im Bereich von \PValue{0.5\dots 0.8} liegen.
\end{Declaration}
\end{Declaration}

\begin{Declaration}{\Macro{topfraction}}[0\topfraction]
\begin{Declaration}{\Macro{dbltopfraction}}[0\dbltopfraction]
\printdeclarationlist*%
%
Diese Werte geben den maximalen Seitenanteil für Gleitobjekte, die am oberen 
Seitenrand platziert werden, für einspaltiges und zweispaltiges Layout an. Er 
sollte im Bereich von \PValue{0.5\dots 0.8} liegen und größer als 
\Macro*{floatpagefraction} beziehungsweise \Macro*{dblfloatpagefraction} sein.
\end{Declaration}
\end{Declaration}

\begin{Declaration}{\Macro{bottomfraction}}[0\bottomfraction]
\printdeclarationlist*%
%
Dies ist der maximale Seitenanteil für Gleitobjekte, die am unteren Seitenrand 
platziert werden. Er sollte zwischen \PValue{0.2} und \PValue{0.5} betragen.
\end{Declaration}

\begin{Declaration}{\Macro{textfraction}}[0\textfraction]
\printdeclarationlist*%
%
Dies ist der Mindestanteil an Text, der auf einer Seite mit Gleitobjekten 
vorhanden sein muss, wenn diese nicht auf einer eigenen Seite ausgegeben 
werden. Er sollte im Bereich von \PValue{0.1\dots 0.3} liegen.
\end{Declaration}

\begin{Declaration}{\Counter{totalnumber}}[\arabic{totalnumber}]
\begin{Declaration}{\Counter{topnumber}}[\arabic{topnumber}]
\begin{Declaration}{\Counter{dbltopnumber}}[\arabic{dbltopnumber}]
\begin{Declaration}{\Counter{bottomnumber}}[\arabic{bottomnumber}]
\printdeclarationlist*%
%
Außerdem gibt es noch Zähler, welche die maximale Anzahl an Gleitobjekten pro 
Seite insgesamt (\Counter*{totalnumber}), am oberen (\Counter*{topnumber}) und 
am unteren Rand der Seite (\Counter*{bottomnumber}) sowie im Zweispaltensatz 
beide Spalten überspannend (\Counter*{dbltopnumber}) festlegen. Die Werte 
können mit \Macro*{setcounter}\Parameter{Zähler}\Parameter{Wert} geändert 
werden.
\end{Declaration}
\end{Declaration}
\end{Declaration}
\end{Declaration}

\begin{Declaration}{\Length{@fptop}}
\begin{Declaration}{\Length{@fpsep}}
\begin{Declaration}{\Length{@fpbot}}
\begin{Declaration}{\Length{@dblfptop}}
\begin{Declaration}{\Length{@dblfpsep}}
\begin{Declaration}{\Length{@dblfpbot}}
\printdeclarationlist*%
%
Sind vor Beginn eines Kapitels noch Gleitobjekte verblieben, so werden diese 
durch \hologo{LaTeX} normalerweise auf einer separaten vertikal zentriert Seite 
ausgegeben. Dabei bestimmen diese Längen jeweils den Abstand vor dem ersten 
Gleitobjekt zum oberen Seitenrand (\Length*{@fptop}, \Length*{@dblfptop}), 
zwischen den einzelnen Objekten (\Length*{@fpsep}, \Length*{@dblfpsep}) sowie 
zum unteren Seitenrand (\Length*{@fpbot}, \Length*{@dblfpbot}). Soll dies nicht 
geschehen, können die Längen durch den Anwender geändert werden.
\end{Declaration}
\end{Declaration}
\end{Declaration}
\end{Declaration}
\end{Declaration}
\end{Declaration}
%
\begin{Example}
Alle Gleitobjekte auf einer dafür speziell gesetzten Seite sollen direkt zu 
Beginn dieser ausgegeben werden. In der Dokumentpräambel kann man dafür 
schreiben:
\begin{Code}
\makeatletter
\setlength{\@fptop}{0pt}
\setlength{\@dblfptop}{0pt} % twocolumn
\makeatother
\end{Code}
\end{Example}