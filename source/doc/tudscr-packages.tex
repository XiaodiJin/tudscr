\chapter{Benötigte, unterstützte und empfehlenswerte Pakete}
\tudhyperdef{sec:packages}
\index{Kompatibilität|?}%
\section{Von den neuen Hauptklassen benötigte Pakete}
\label{sec:packages:needed}
\subsection{Erforderliche Pakete bei der Schriftinstallation}
%
Für die Installation der Schriften sind die folgend genannten Pakete von
\emph{essentieller} Bedeutung und daher \emph{zwingend} notwendig. Das 
Vorhandensein dieser wird durch die jeweiligen Schriftinstallationsskripte
(\seeref{\autoref{sec:install}}) geprüft und die Installation beim Fehlen eines 
oder mehrerer Pakete mit einer entsprechenden Warnung abgebrochen.
%
\begin{packages}
\item[fontinst](fontware)
  Dieses Paket wird für die Installation der Schriften \Univers sowie \DIN 
  benötigt. Weiterhin ist \Package{fontware} der \hologo{LaTeX}-Distribution 
  für die Schriftkonvertierung notwendig.
\item[cmbright,hfbright,cm-super]
  Alle mathematischen Glyphen und Symbole, die nicht in den \Univers-Schriften 
  enthalten sind, werden diesem Paket entnommen. Außerdem werden für die 
  \PValue{T1}"~Schriftkodierung die beiden Pakete \Package{cm-super} und 
  \Package{hfbright} benötigt. Aus dem Paket \Package{lmodern} werden 
  zusätzlich die \texttt{Schreibmaschinenschriften} verwendet.
\item[iwona]
  Sowohl für \Univers als auch für \DIN werden fehlenden Glyphen entnommen.
\end{packages}



\subsection{Notwendige Pakete für die Verwendung der Hauptklassen}
Diese Pakete werden von den neuen Klassen zwingend benötigt und automatisch 
geladen.
%
\begin{packages}
\item[koma-script,typearea,scrlayer-scrpage,scrbase,scrwfile]
  \begin{Declaration*}{\Class{scrbook}}
  \begin{Declaration*}{\Class{scrreprt}}
  \begin{Declaration*}{\Class{scrartcl}}
  \begin{Declaration*}{\Class{scrlttr2}}
  \ChangedAt{%
    v2.02:Paket \Package{scrlayer-scrpage} ist für \TUDScript unabdingbar%
  }%
  Das \KOMAScript-Bundle ist die zentrale Grundlage für \TUDScript. Neben den 
  Klassen \Class{scrbook}, \Class{scrreprt} und \Class{scrartcl} wird das Paket 
  \Package{scrbase} benötigt. Dieses erlaubt das Definieren von Klassenoptionen 
  im Stil von \KOMAScript, welche auch noch nach dem Laden der Klasse mit den 
  Befehlen \Macro{TUDoption} und \Macro{TUDoptions} geändert werden können. Für 
  die Bereitstellung der \PageStyle{tudheadings}-Seitenstile ist das Paket 
  \Package{scrlayer-scrpage} notwendig. Wenn es nicht durch den Anwender~-- mit 
  beliebigen Optionen~-- geladen wird, erfolgt dies am Ende der Präambel 
  automatisch durch \TUDScript.
  \end{Declaration*}
  \end{Declaration*}
  \end{Declaration*}
  \end{Declaration*}
\item[kvsetkeys]
  Hiermit wird das von \Package{scrbase} geladene Paket \Package{keyval} 
  verbessert, welches das Definieren von Klassen"~ und Paketoptionen sowie 
  Parametern nach dem Schlüssel"=Wert"=Prinzip ermöglicht. Das Verhalten für 
  unbekannte Schlüssel  kann mit \Package{kvsetkeys} festgelegt werden.
\item[etoolbox]
  Es werden viele Funktionen zum Testen und zur Ablaufkontrolle bereitgestellt 
  und das einfache Manipulieren vorhandener Makros ermöglicht.
\item[geometry]\index{Satzspiegel}%
  Das Paket wird zum Festlegen der Seitenränder respektive des Satzspiegels 
  verwendet. Ein Weiterreichen zusätzlicher Optionen an das Paket wird 
  dringlich nicht empfohlen.
\item[textcase]\index{Schriftauszeichnung}%
  Mit \Macro{MakeTextUppercase} wird die Großschreibung der Überschriften in 
  \DIN erzwungen. Im \emph{Ausnahmefall} kann dies mit \Macro{NoCaseChange} 
  unterbunden werden.
\item[graphicx]\index{Grafiken}%
  Dies ist das De-facto-Standard-Paket zum Einbinden von Grafiken. Zum Setzen 
  des Logos der \TnUD im Kopf wird \Macro{includegraphics} genutzt. Es kann 
  auch durch den Anwender in der Präambel geladen werden.
\item[xcolor]\index{Farben}%
  Damit werden die Farben des \CDs zur Verwendung im Dokument definiert. 
  Genaueres ist bei der Beschreibung von \Package{tudscrcolor}'auto' zu finden. 
  Ein Laden beider Pakete in der  Präambel durch den Nutzer ist problemlos 
  möglich.
\item[environ]\index{Befehlsdeklaration}%
  Es wird eine verbesserte Deklaration von Umgebungen ermöglicht, bei der auch 
  beim Abschluss der Umgebung auf die übergebenen Parameter zugegriffen werden 
  kann. Dies wird die Neugestaltung der \Environment{abstract}"=Umgebung 
  benötigt.
\item[trimspaces]
  Bei mehreren Eingabefeldern für den Anwender werden die Argumente mithilfe 
  dieses Paketes um eventuell angegebene, unnötige Leerzeichen befreit.
\end{packages}
%
Möchten Sie eines der hier aufgezählten Pakete selber nutzen, es jedoch mit 
bestimmten Optionen laden, so sollten diese bereits \emph{vor} der Definition 
der Dokumentklasse an das Paket weitergereicht werden, falls bei der 
Beschreibung nichts anderweitiges angegeben worden ist.
%
\begin{Example}
Das Weiterreichen von Optionen an Pakete muss folgendermaßen erfolgen:
\begin{Code}[escapechar=§]
\PassOptionsToPackage§\Parameter{Optionenliste}\Parameter{Paket}§
\documentclass§\OParameter{Klassenoptionen}\PParameter{tudscr\dots}§
\end{Code}
\end{Example}



\section{Durch \TUDScript direkt unterstütze Pakete}
%
Wird eines der genannten Pakete geladen, so wird die Funktionalität von 
\TUDScript durch dieses verbessert beziehungsweise erweitert oder es wird ein 
Bugfix bereitgestellt.
%
\begin{packages}
\item[hyperref]\index{Lesezeichen}\index{Querverweise}%
  Hiermit können in einem PDF-Dokument Lesezeichen, Querverweise und 
  Hyperlinks erstellt werden. Wird es geladen, sind außerdem die Option 
  \Option{tudbookmarks} sowie der Befehl \Macro{tudbookmark} nutzbar. Das 
  Paket \Package{bookmark} erweitert die Unterstützung nochmals. Beide 
  genannten Pakete sollten~-- bis auf sehr wenige Ausnahmen wie beispielsweise 
  \Package{glossaries}~-- als letztes in der Präambel eingebunden werden.
\item[isodate]\index{Datum|?}%
  Dieses Paket formatiert mit \Macro{printdate}[\Parameter{Datum}] die Ausgabe 
  eines Datums automatisch in ein spezifiziertes Format. Wird es geladen, 
  werden alle Datumsfelder, welche durch die \TUDScript-Klassen definiert 
  wurden,%
  \footnote{%
    \Macro{date}, \Macro{dateofbirth} und \Macro{defensedate} sowie aus 
    \Package{tudscrsupervisor} \Macro{duedate}(\Package{tudscrsupervisor}) und 
    \Macro{issuedate}(\Package{tudscrsupervisor})
  }
  in diesem Format ausgegeben.
\item[multicol]\index{Satzspiegel!zweispaltig|?}\index{Mehrspaltensatz}%
  \index{Satzspiegel!mehrspaltig}%
  Hiermit kann jeglicher beliebiger Inhalt in zwei oder mehr Spalten ausgegeben 
  werden, wobei~-- im Gegensatz zur \KOMAScript-Option \Option{twocolumn}~-- 
  für einen Spaltenausgleich gesorgt wird. Unterstützt wird das Paket innerhalb 
  der Umgebungen \Environment{abstract} und \Environment{tudpage}.
\item[quoting]\index{Zitate}%
  \hologo{LaTeX} bietet von Haus aus \emph{zwei} verschiedene Umgebungen~-- 
  \Environment{quote} und \Environment{quotation}~-- für Zitate und ähnliches 
  an. Die \KOMAScript-Option \Option{parskip=\PName{Methode}}|declare| wird 
  allerdings durch beide ignoriert. Dieses Problem wird durch die Umgebung 
  \Environment{quoting} behoben. Wird das Paket geladen, kommt diese innerhalb 
  der \Environment{abstract}"=Umgebung zum Einsatz.
\item[ragged2e]\index{Worttrennung}%
  Das Paket verbessert den Flattersatz, indem für diesen die Worttrennung 
  aktiviert wird.
\end{packages}



\newcommand*\RecPack{%
  \hyperref[sec:packages:recommended]{Paketbeschreibung}:\xspace%
}
\section{Empfehlenswerte Pakete}
\label{sec:packages:recommended}
In diesem \autorefname wird eine Vielzahl an Paketen~-- zumeist kurz~-- 
vorgestellt, welche sich für mich persönlich bei der Arbeit mit \hologo{LaTeX} 
bewährt haben. Einige davon werden außerdem im Tutorial \Tutorial{treatise} in 
ihrer Anwendung beschrieben. Für detaillierte Informationen sowie bei Fragen zu 
den einzelnen Paketen sollte die jeweilige Dokumentation zu Rate gezogen
werden,\footnote{Kommandozeile/Terminal: \Path{texdoc\,\PName{Paketname}}}
das Lesen der hier gegebenen Kurzbeschreibung ersetzt dies in keinem Fall. 


\subsection{Pakete zur Verwendung in jedem Dokument}
Die hier vorgestellten Pakete gehören meiner Meinung nach in die Präambel eines 
jeden Dokumentes. Egal, in welcher Sprache das Dokument verfasst wird, sollte 
diese mit dem Paket \Package{babel} definiert werden~-- auch wenn dies 
Englisch ist. Für deutschsprachige Dokumente ist für eine annehmbare 
Worttrennung das Paket \Package{hyphsubst} unbedingt zu verwenden.

\begin{packages}
\item[fontenc]\index{Zeichensatzkodierung}%
  Das Paket erlaubt Festlegung der Zeichensatzkodierung des Ausgabefonts. Als 
  Voreinstellung ist die Ausgabe als 7"~bit kodierte Schrift gewählt, was unter 
  anderem dazu führt, dass keine echten Umlaute im erzeugten PDF-Dokument 
  verwendet werden. Um auf 8"~bit"~Schriften zu schalten, ist
  \Macro*{usepackage}[\POParameter{T1}\PParameter{fontenc}] zu nutzen.
\item[selinput,inputenc]\index{Eingabekodierung}\index{Minimalbeispiel|?}%
  Hiermit erfolgt die (automatische) Festlegung der Eingabekodierung. Diese ist 
  vom genutzten \hyperref[sec:tips:editor]{Editors (\autoref{sec:tips:editor})} 
  und den darin gewählten Einstellung abhängig. Mit:
  \begin{Code}
    \usepackage{selinput}
    \SelectInputMappings{adieresis={ä},germandbls={ß}}
  \end{Code}\vspace{-\baselineskip}%
  wird es verwendet. Dies macht den Quelltext portabel, womit beispielsweise 
  einfach via Copy~\&~Paste ein \hrfn{http://www.komascript.de/minimalbeispiel}%
  {Minimalbeispiel} bei Problemstellungen in einem Forum bereitgestellt werden 
  kann. Alternativ dazu lässt sich mit dem Paket \Package{inputenc} die zu 
  verwendende Eingabekodierung manuell einstellen
  (\Macro*{usepackage}[\OParameter{Eingabekodierung}\PParameter{inputenc}]).
\item[babel,polyglossia]\index{Sprachunterstützung}\index{Bezeichner}%
  Mit diesem Paket erfolgt die Einstellung der im Dokument verwendeten 
  Sprache(n). Bei mehreren angegebenen Sprachen ist die zuletzt geladene die 
  Hauptsprache des Dokumentes. Die gewünschten Sprachen sollten als nicht als 
  Paketoption sondern als Klassenoption und gesetzt werden, damit auch andere 
  Pakete auf die Spracheinstellungen zugreifen können. Für deutschsprachige 
  Dokumente ist die Option \Option{ngerman} für die neue oder \Option{german} 
  für die alte deutsche Rechtschreibung zu verwenden. 
  
  Mit dem Laden von \Package{babel} und der dazugehörigen Sprachen werden 
  sowohl die Trennmuster als auch die sprachabhängigen Bezeichner angepasst.
  Von einer Verwendung der obsoleten Pakete \Package{german} beziehungsweise 
  \Package{ngerman} anstelle von \Package{babel} wird abgeraten. Für 
  \hologo{LuaLaTeX} und \hologo{XeLaTeX} kann das Paket \Package{polyglossia} 
  genutzt werden.
\item[microtype]\index{Typografie}%
  Dieses Paket kümmert sich um den optischen Randausgleich%
  \footnote{englisch: protrusion, margin kerning}
  und das Nivellieren der Wortzwischenräume%
  \footnote{englisch: font expansion}
  im Dokument. Es funktioniert nicht mit der klassischen \hologo{TeX}-Engine, 
  wohl jedoch mit \hologo{pdfTeX} als auch \hologo{LuaTeX} sowie \hologo{XeTeX}.
\item[hyphsubst,dehyph-exptl]\index{Worttrennung|!}%
  Die möglichen Trennstellen von Wörtern wird von \hologo{LaTeX} mithilfe 
  eines Algorithmus berechnet. Dieser wird für deutschsprachige Texte mit dem 
  Paket \Package{hyphsubst} entscheidend verbessert. Es muss bereits \emph{vor} 
  der Dokumentklasse wie folgt geladen werden:
  \begin{Code}[escapechar=§]
    \RequirePackage[ngerman=ngerman-x-latest]{hyphsubst}
  \end{Code}\vspace{-\baselineskip}%
  In \autoref{sec:tips:hyphenation} wird genauer auf das Zusammenspiel von 
  \Package{hyphsubst} und \Package{babel} sowie \Package{fontenc} eingegangen,
  ein Blick dahin wird dringend empfohlen. Zusätzliche werden dort weitere 
  Hinweise für eine verbesserte Worttrennung gegeben. 
\end{packages}


\subsection{Pakete zur situativen Verwendung}
Die nachfolgenden Pakete sollten nicht zwangsweise in jedem Dokument geladen 
werden sondern nur, falls dies auch tatsächlich notwendig ist. Zur besseren 
Übersicht wurde versucht, diese thematisch passend zu gruppieren. Daraus lässt 
sich keinerlei Wertung bezüglich ihrer Nützlichkeit oder meiner persönlichen 
Wertschätzung ableiten.

\subsubsection{Typografie und Layout}
\begin{packages}
\index{Typografie|(}%
\item[setspace]\index{Zeilenabstand}%
  Die Vergrößerung des Zeilenabstandes wird:
  %
  \begin{enumerate}[itemindent=0pt,labelwidth=*,labelsep=1em,label=\Roman*.]
  \stditem viel zu häufig und völlig unnötig gefordert und
  \stditem schließlich auch noch zu groß gewählt.
  \end{enumerate}
  %
  Die Forderung nach Erhöhung des Zeilenabstandes~-- in der Typografie als 
  Durchschuss bezeichnet~-- kommt noch aus den Zeiten der Textverarbeitung mit 
  der Schreibmaschine. Ein einzeiliger Zeilenabstand bedeutete hier, dass die 
  Unterlängen der oberen Zeile genau auf der Höhe der Oberlängen der folgenden 
  Zeile lagen. Ein anderthalbzeiliger Zeilenabstand erzielte hier somit einen 
  akzeptablen Durchschuss. Eine Erhöhung des Durchschusses bei der Verwendung 
  von \hologo{LaTeX} ist an und für sich nicht notwendig. Sinnvoll ist dies 
  nur, wenn im Fließtext serifenlose Schriften zum Einsatz kommen, um die damit 
  verbundene schlechte Lesbarkeit etwas zu verbessern.
  
  Ist die Erhöhung des Durchschusses wirklich notwendig, sollte das Paket 
  \Package{setspace} genutzt werden. Dieses stellt den Befehl 
  \Macro{setstretch}[\Parameter{Faktor}] zur Verfügung, mit dem der Durchschuss 
  respektive Zeilenabstand angepasst werden kann. Der Wert des Faktors ist 
  standardmäßig auf~1 gestellt und sollte maximal bis~1.25 vergrößert werden. 
  Der Befehl \Macro*{onehalfspacing} aus diesem Paket setzt diesen Wert auf 
  eben genau~1.25. Allerdings ist hier anzumerken, dass die Vergrößerung des 
  Zeilenabstandes~-- so wie ich es mir angelesen habe~-- aus der Sicht eines 
  Typographen keine Spielerei ist sondern vielmehr allein der Lesbarkeit des 
  Textes dient und möglichst gering ausfallen sollte.
  
  Ziel ist es, beim Lesen nach dem Beenden der aktuellen Zeile das Auffinden 
  der neuen Zeile zu vereinfachen. Bei Serifen ist dies durch die Betonung der 
  Grundlinie sehr gut möglich. Bei serifenlosen Schriften~-- wie der im \TUDCD 
  verwendeten \Univers~-- ist dies schwieriger und ein erweiterter Abstand der 
  Zeilen kann dabei durchaus hilfreich sein. Jedoch sollte nicht nach dem Motto 
  \enquote{viel hilft viel} verfahren werden. In diesem Dokument wurde als 
  Faktor für den Zeilenabstand \Macro{setstretch}[\PParameter{1.1}] gewählt. 
  Nach einer Einstellung des Zeilenabstandes sollte der Satzspiegel unbedingt 
  mit \Macro{recalctypearea} neu berechnet werden. Weitere Tipps sind in 
  \autoref{sec:tips:headings} sowie \autoref{sec:tips:headline} zu finden.
\item[csquotes]\index{Zitate}%
  Das Paket stellt unter anderem den Befehl \Macro{enquote}[\Parameter{Zitat}] 
  zur Verfügung, welcher Anführungszeichen in Abhängigkeit der gewählten 
  Sprache setzt. Zusätzlich werden weitere Kommandos und Optionen für die 
  spezifischen Anforderungen des Zitierens bei wissenschaftlichen Arbeiten 
  angeboten. Außerdem wird es durch \Package{biblatex} unterstützt und sollte 
  zumindest bei dessen Verwendung geladen werden.
\item[noindentafter]
  \ChangedAt{v2.02:\RecPack \Package{noindentafter}}
  Mit diesem Paket lassen sich automatische Absatzeinzüge für selbst zu 
  bestimmende Befehle und Umgebungen unterdrücken.
\item[xspace]\index{Befehlsdeklaration}%
  Mit \Package{xspace} kann bei der Definition eigener Makros der Befehl 
  \Macro{xspace} genutzt werden. Dieser setzt ein gegebenenfalls notwendiges 
  Leerzeichen automatisch. In \autoref{sec:tips:xspace} ist die Definition 
  eines solchen Befehls exemplarisch ausgeführt.
\item[ellipsis]\index{Befehlsdeklaration}%
  In \hologo{LaTeX} folgen den Befehlen für Auslassungspunkte (\Macro{dots} und 
  \Macro{textellipsis}) \emph{immer} ein Leerzeichen. Dies kann unter Umständen 
  unerwünscht sein. Mit dem Paket \Package{ellipsis} wird das nachfolgende 
  Leerzeichen~-- im Gegensatz zum Standardverhalten~-- nur gesetzt, wenn ein 
  Satzzeichen und kein Buchstabe folgt, \seeref*{\autoref{sec:tips:dots}}.
\item[xpunctuate]\index{Befehlsdeklaration}%
  Die Funktionalität von \Package{xspace} wird um die Beachtung von 
  Interpunktionen erweitert.
\stditem[\href{http://www.ctan.org/pkg/delig}{\Application{DeLig}}]
  \index{Ligaturen}%
  Hierbei handelt es sich um ein Java-Script, welches anhand eines Wörterbuches 
  falsche Ligaturen innerhalb eines Dokumentes automatisiert entfernt. Wird 
  \Univers verwendet ist dies jedoch nicht notwendig, da diese keinerlei 
  Ligaturen enthält, die insbesondere in deutschen Texten für einen guten Satz 
  manuell aufgelöst werden müssten.%
  \footnote{%
    Das sind ff, fi, fl, ffi, und ffl bei den \hologo{LaTeX}"=Standardschriften.
  }
  Mit \hologo{LuaLaTeX} als Dokumentprozessor kann alternativ dazu auch 
  \Package{selnolig} verwendet werden.
\item[balance]\index{Satzspiegel!zweispaltig}%
  Dieses Paket ermöglicht einen Spaltenausgleich im zweispaltigen Satz auf der 
  letzten Dokumentseite. Alternativ dazu kann auch \Package{multicol} verwendet 
  werden.
\index{Typografie|)}%
\end{packages}

\subsubsection{Rechtschreibung}
\index{Rechtschreibung|(}%
%
Für die Rechtschreibkontrolle zeichnet im Normalfall der verwendete Editor 
verantwortlich. Dennoch gibt es einige wenige Pakete, welche sich diesem Thema 
widmen. Diese sind jedoch lediglich nutzbar, wenn \hologo{LuaLaTeX} als 
Dokumentprozessor genutzt wird.
\begin{packages}
\item[lua-check-hyphen]\index{Worttrennung}%
  Hiermit lassen sich mit \hologo{LuaLaTeX} Trennstellen am Zeilenende zur 
  Prüfung markieren. Zum Thema der \textit{korrekten Worttrennung} sei außerdem 
  auf \autoref{sec:tips:hyphenation} verwiesen.
\item[spelling]
  Wird \hologo{LuaLaTeX} als Prozessor verwendet, wird mit diesem Paket der 
  reine Textanteil aus dem \hologo{LaTeX}"~Dokument extrahiert~-- wobei Makros 
  und aktive Zeichen entfernt werden~-- und in eine separate Textdatei 
  geschrieben. Anschließend kann diese Datei mit einer externen Software zur 
  Rechtschreibprüfung wie \Application{GNU~Aspell} oder \Application{Hunspell} 
  analysiert werden. Wird durch dieses Programm eine Liste falsch geschriebener 
  Wörter ausgegeben, können diese mit \Package{spelling} im PDF"~Dokument 
  hervorgehoben werden.
\index{Rechtschreibung|)}%
\end{packages}

\subsubsection{Schriften und Sonderzeichen}
\begin{packages}
\item[lmodern]<lm>\index{Schriftart}%
  Soll mit den klassischen \hologo{LaTeX}"=Standardschriften gearbeitet werden, 
  empfiehlt sich die Verwendung des Paketes \Package{lmodern}. Dieses 
  verbessert die Darstellung der Computer~Modern sowohl am Bildschirm als auch 
  beim finalen Druck.
\item[cfr-lm]\index{Schriftart}%
  Dieses experimentelle Paket liefert weitere Schriftschnitte für das Paket 
  \Package{lmodern}.
\item[newtx,newtxmath]<newtx>\index{Schriftart}\index{Mathematiksatz}%
  Es werden einige alternative Schriften sowohl für den Fließtext 
  (\textit{Times} und \textit{Helvetica}) als auch den Mathematikmodus 
  bereitgestellt.
\item[libertine]\index{Schriftart}%
  Das Paket stellt die Schriften Linux~Libertine und Linux~Biolinum zur 
  Verfügung. Um diese Schriftart auch für den Mathematikmodus verwenden zu 
  können, sollte \Package{newtxmath} aus dem \Package{newtx}-Bundle mit 
  \Macro*{usepackage}[\POParameter{libertine}\PParameter{newtxmath}] in der 
  Präambel eingebunden werden. Das Paket \Package{libgreek} enthält griechische 
  Buchstaben für Linux~Libertine.
\item[mweights]\index{Schriftstärke}%
  In \hologo{LaTeXe} existieren die drei Schriftfamilien für Serifenschriften 
  (\Macro{rmfamily}), serifenlose Schriften (\Macro{sffamily}) sowie die 
  Schreibmaschinenschriften (\Macro{ttfamily}). Deren Schriftstärke wird für 
  gewöhnlich mit den beiden Befehlen \Macro{mddefault} und \Macro{bfdefault} 
  einheitlich festgelegt. Bei der Verwendung unterschiedlicher Schriftpakete 
  kann es unter Umständen zu Problemen bei den Schriftstärken kommen. Das Paket 
  \Package{mweights} erlaubt die individuelle Definition der Schriftstärke für 
  jede der drei Schriftfamilien.
\item[fontspec]\index{Zeichensatzkodierung}%
  \ChangedAt{v2.02:\RecPack \Package{fontspec}}%
  Wird als Dokumentprozessor nicht \hologo{pdfLaTeX} sondern \hologo{XeLaTeX} 
  oder \hologo{LuaLaTeX} verwendet, können mit diesem Paket Systemschriften im 
  OpenType-Format und einer beliebigen Zeichensatzkodierung eingebunden werden, 
  womit sich die Auswahl der verwendbaren Schriften stark erweitert. Das Paket 
  wird durch \TUDScript unterstützt.
\item[relsize]\index{Schriftgröße}%
  Die Größe einer Textauszeichnung kann relativ zur aktuellen Schriftgröße 
  gesetzt werden.
\item[textcomp]\index{Sonderzeichen}%
  Es werden zusätzliche Symbole und Sonderzeichen wie beispielsweise das 
  Promille- oder Eurozeichen sowie Pfeile für den Fließtext zur Verfügung 
  gestellt.
\end{packages}
%
Auch für (serifenlose) Mathematikschriften gibt es einige nützliche Pakete. 
Werden die Schriften des \CDs genutzt, sei auf die Option \Option{cdmath} 
verwiesen.
\index{Mathematiksatz|(}%
%
\begin{packages}
\item[sansmathfonts,sansmath]
  Sollten die normalen \hologo{LaTeX}-Schriften Computer~Modern verwendet 
  werden, lässt sich dieses Paket zum serifenlosen Setzen mathematischer 
  Ausdrücke nutzen. Ein alternatives Paket mit der gleichen Zielstellung ist 
  \Package{sansmath}
\item[sfmath]
  Diese Paket verfolgt ein ähnliches Ziel, kann jedoch im Gegensatz zu 
  \Package{sansmath} nicht nur für Computer~Modern sondern mit der 
  entsprechenden Option auch für Latin~Modern, Helvetica und 
  Computer~Modern~Bright verwendet werden.
\item[mathastext]
  Mit dem Paket wird das Ziel verfolgt, aus der genutzten Schrift für den 
  Fließtext alle notwendigen Zeichen für den Mathematiksatz zu extrahieren.
\end{packages}

\subsubsection{Mathematiksatz}
Dies sind Pakete, die Umgebungen und Befehle für den Mathematiksatz sowie das 
Setzen von Einheiten und Zahlen im Allgemeinen anbieten.

\begin{packages}
  \item[mathtools,amsmath]
    Dieses Paket stellt für das De-facto-Standard-Paket \Package{amsmath} für 
    Mathematikumgebungen Bugfixes zur Verfügung und erweitert dieses.
  \item[bm]
    Das Paket bietet mit \Macro{bm} eine Alternative zu \Macro{boldsymbol} im 
    \hrfn{http://tex.stackexchange.com/questions/3238}{Mathematiksatz}.
\end{packages}
%
Für das typografisch korrekte Setzen von Einheiten~-- ein halbes Leerzeichen 
zwischen Zahl und \emph{aufrecht} gesetzter Einheit~-- gibt es zwei gut 
nutzbare Pakete.
%
\begin{packages}
\index{Einheiten|(}%
\item[units]
  Dies ist ein einfaches und sehr zweckdienliches Paket zum Setzen von 
  Einheiten und für die meisten Anforderungen völlig ausreichend.
\item[siunitx]
  Dieses Paket ist in seinem Umfang im Vergleich deutlich erweitert. Neben 
  Einheiten können zusätzlich auch Zahlen typografisch korrekt gesetzt werden. 
  Die Ausgabe lässt sich in vielerlei Hinsicht an individuelle Bedürfnisse 
  anpassen. Für deutschsprachige Dokumenten sollte die Lokalisierung angegeben 
  werden. Mehr dazu in \autoref{sec:tips:siunitx}.
\index{Einheiten|)}%
\end{packages}
%
Die korrekte Formatierung von Zahlen ist häufig ein Problem bei der Verwendung 
von \hologo{LaTeX}. Insbesondere, wenn in einem deutschsprachigen Dokument 
Daten im englischsprachigen Format verwendet werden, kommt es zu Problemen. 
Dafür wird im \TUDScript-Bundle das Paket \Package{mathswap} bereitgestellt. 
Dennoch gibt es zu diesem auch Alternativen.
%
\begin{packages}
\index{Zifferngruppierung|(}%
  \item[icomma]
    Wird im Mathematikmodus nach dem Komma ein Leerzeichen gesetzt, wird dies 
    bei der Ausgabe beachtet. Der Verfasser muss sich demzufolge jederzeit 
    selbst um die typografisch korrekte Ausgabe kümmern.
  \item[ziffer]
    Für deutschsprachige Dokumente wird das Komma als Dezimaltrennzeichen 
    zwischen zwei Ziffern definiert. Folgt dem Komma keine Ziffer, wird 
    jederzeit der obligatorische Freiraum gesetzt, was meiner Meinung nach 
    besser als das Verhalten von \Package{icomma} ist.
  \item[ionumbers]
    Dieses Paket ist mir tatsächlich erst bei der Arbeit an \Package{mathswap} 
    bekannt geworden. Es bietet mehr Funktionalitäten und kann als Alternative 
    dazu betrachtet werden.
\index{Zifferngruppierung|)}%
\end{packages}
%
Weitere Hinweise und Anwendungsfälle zur mathematischen Typografie werden in 
\autoref{sec:exmpl:mathtype} sowie \autoref{sec:exmpl:mathswap} gegeben.
\index{Mathematiksatz|)}%

\subsubsection{Verzeichnisse aller Art}
\index{Verzeichnisse|?}%
Neben dem Erstellen des eigentlichen Dokumentes sind für eine wissenschaftliche 
Arbeit meist auch allerhand Verzeichnisse gefordert. Fester Bestandteil ist 
dabei das Literaturverzeichnis, auch ein Abkürzungs- und Formelzeichen- 
beziehungsweise Symbolverzeichnis werden häufig gefordert. Gegebenenfalls wird 
auch noch ein Glossar benötigt. Hier werden die passenden Pakete vorgestellt. 
Sollen im Dokument komplette Quelltexte oder auch nur Auszüge daraus erscheinen 
und für diese auch gleich ein entsprechendes Verzeichnis generiert werden, so 
sei auf das Paket \Package{listings}'full' verwiesen.

\begin{packages}
\index{Verzeichnisse|(?}%
\item[biblatex]\index{Literaturverzeichnis}%
  Das Paket kann als legitimer Nachfolger zu \hologo{BibTeX} gesehen werden. 
  Ähnlich dazu bietet \Package{biblatex} die Möglichkeit, Literaturdatenbanken 
  einzubinden und verschiedene Stile der Referenzierung und Darstellung des 
  Literaturverzeichnisses auszuwählen. 
  
  Mit \Package{biblatex} ist die Anpassung eines bestimmten Stiles wesentlich 
  besser umsetzbar als mit \hologo{BibTeX}. Wird \Application{biber} für die 
  Sortierung des Literaturverzeichnisses genutzt, ist die Verwendung einer 
  UTF"~8-kodierten Literaturdatenbank problemlos möglich. In Verbindung mit 
  \Package{biblatex} wird die zusätzliche Nutzung des Paketes 
  \Package{csquotes} sehr empfohlen.
\item[acro,acronym]\index{Abkürzungsverzeichnis}%
  Soll lediglich ein Abkürzungsverzeichnis erstellt werden, ist dieses Paket 
  die erste Wahl. Es stellt Befehle zur Definition von Abkürzungen sowie zu 
  deren Verwendung im Text und zur sortierten Ausgabe eines Verzeichnisses 
  bereit. Alternativ dazu kann das Paket \Package{acronym} verwendet werden. 
  Die Sortierung des Abkürzungsverzeichnisses muss hier allerdings manuell 
  durch den Anwender erfolgen.
\item[glossaries,nomencl]\index{Glossar}\index{Abkürzungsverzeichnis}%
  \index{Formelzeichenverzeichnis}\index{Symbolverzeichnis}%
  Dies ist ein sehr mächtiges Paket zum Erstellen eines Glossars sowie 
  Abkürzungs- und Symbolverzeichnisses. Die mannigfaltige Anzahl an Optionen 
  ist zu Beginn eventuell etwas abschreckend. Insbesondere wenn Verzeichnisse 
  für Abkürzungen \emph{und} Formelzeichen beziehungsweise Symbole notwendig 
  sind, sollte dieses Paket in Erwägung gezogen werden.
  
  Alternativ dazu kann für ein Symbolverzeichnis auch lediglich eine manuell 
  gesetzte Tabelle genutzt werden. Das hierfür sehr häufig empfohlene Paket 
  \Package{nomencl} bietet meiner Meinung nach demgegenüber keinerlei Vorteile.
\index{Verzeichnisse|?)}%
\end{packages}

\subsubsection{Listen}
\begin{packages}
\index{Listen|(?}%
\item[enumitem]
  Das Paket \Package{enumitem} erweitert die rudimentären Funktionalitäten der 
  \hologo{LaTeX}"=Standardlisten \Environment{itemize}, \Environment{enumerate}
  sowie \Environment{description} und ermöglicht die individuelle Anpassung 
  dieser durch die Bereitstellung vieler optionale Parameter nach dem
  Schlüssel"=Wert"=Prinzip. Eine von mir sehr häufig genutzte Funktion ist 
  beispielsweise die Entfernung des zusätzlichen Abstand zwischen den einzelnen 
  Einträgen einer Liste mit \Macro{setlist}[\PParameter{noitemsep}].
\index{Listen|?)}%
\end{packages}

\subsubsection{Tabellen}
\index{Tabellen|(?}%
Für den Tabellensatz in \hologo{LaTeX} werden von Haus aus die Umgebungen 
\Environment{tabbing} und \Environment{tabular} beziehungsweise 
\Environment{tabular*} bereitgestellt, welche in ihrer Funktionalität meist 
für einen qualitativ hochwertigen Tabellensatz nicht ausreichen. Es werden 
deshalb Pakete vorgestellt, die zusätzlich verwendet werden können. 
\begin{packages}
\item[array]
  Dieses Paket ermöglicht mit dem Befehl \Macro{newcolumntype} das Erstellen 
  eigener Spaltentypen sowie die erweiterte Definition von Tabellenspalten
  (\PValue{>\PParameter{\dots}}\PName{Spaltentyp}\PValue{<\PParameter{\dots}}), 
  wobei mithilfe sogenannter \enquote{Hooks} vor und nach Einträgen innerhalb 
  einer Spalte gezielt Anweisungen gesetzt werden können. Außerdem kann die 
  Höhe der Zeilen einer Tabelle mit \Macro{extrarowheight} angepasst werden. 
\item[multirow]
  Es wird der Befehl \Macro{multirow} definiert, der~-- ähnlich zum Makro 
  \Macro{multicolumn}~-- das Zusammenfassen von mehreren Zeilen in einer 
  Spalte ermöglicht.
\item[widetable]
  Mit der Standard"=\hologo{LaTeX}"=Umgebung \Environment{tabular*} kann eine 
  Tabelle mit einer definierten Breite gesetzt werden. Dieses Paket stellt die 
  Umgebung \Environment{widetable} zur Verfügung, die als Alternative genutzt 
  werden kann und eine symmetrische Tabelle erzeugt.
\item[booktabs]
  Für einen guten Tabellensatz mit \hologo{LaTeX} gibt es bereits zahlreiche 
  \hrfn{http://userpage.fu-berlin.de/latex/Materialien/tabsatz.pdf}{Tipps} im 
  Internet zu finden. Zwei Regeln sollten dabei definitiv beachtet werden:
  %
  \begin{enumerate}[itemindent=0pt,labelwidth=*,labelsep=1em,label=\Roman*.]
  \stditem keine vertikalen Linien
  \stditem keine doppelten Linien
  \end{enumerate}
  %
  Das Paket \Package{booktabs} (deutsche Dokumentation \Package*{booktabs-de}) 
  ist für den Satz von hochwertigen Tabellen eine große Hilfe und stellt die 
  Befehle \Macro{toprule}, \Macro{midrule} sowie \Macro{cmidrule} und 
  \Macro{bottomrule} für unterschiedliche horizontale Linien bereit.
\item[tabularborder]
  Bei Tabellen wird zwischen Spalten automatisch ein horizontaler Abstand 
  (\Length{tabcolsep}) gesetzt~-- besser gesagt jeweils vor und nach einer 
  Spalte. Dies geschieht auch \emph{vor} der ersten und \emph{nach} der letzten 
  Spalte. Dieser zusätzliche Platz an den äußeren Rändern kann störend wirken, 
  insbesondere wenn die Tabelle über die komplette Textbreite gesetzt wird. Mit 
  dem Paket \Package{tabularborder} kann dieser Platz automatisch entfernt 
  werden.
  
  Dies funktioniert allerdings nur mit der \Environment{tabular}"=Umgebung. 
  Die Tabellen aus den Paketen \Package{tabularx}, \Package{tabulary} und 
  \Package{tabu} werden nicht unterstützt. Wie dieser Abstand bei diesen 
  manuell entfernt werden kann, ist unter \autoref{sec:tips:table} zu finden.
\item[tabularx]
  Auch mit diesem Paket kann die Gesamtbreite einer Tabelle spezifiziert 
  werden. Dafür wird der Spaltentyp \PValue{X} definiert, welcher als Argument 
  der \Environment{tabularx}"=Umgebung beliebig häufig angegeben werden kann
  (\Macro*{begin}[\PParameter{tabularx}\Parameter{Breite}\Parameter{Spalten}]). 
  Die \PValue{X}"~Spalten ähneln denen vom Typ~\PValue{p}\Parameter{Breite}, 
  wobei die Breite dieser aus der gewünschten Tabellengesamtbreite und dem 
  benötigten Platz der gegebenenfalls vorhandenen Standardspalten automatisch 
  berechnet wird.
\item[tabulary]
  Dies ist ein weiteres Paket zur automatischen Berechnung von Spaltenbreiten. 
  Der zur Verfügung stehende Platz~-- gewünschte Gesamtbreite abzüglich der 
  notwendigen Breite für die Standardspalten~-- wird jedoch nicht wie bei der 
  Umgebung \Environment{tabularx} auf alle Spalten gleichmäßig verteilt sondern 
  in der \Environment{tabulary}"=Umgebung für die Spaltentypen~\PValue{LCRJ} 
  anhand ihres Zellinhaltes gewichtet vergeben. 
  (\Macro*{begin}[\PParameter{tabulary}\Parameter{Breite}\Parameter{Spalten}]). 
\item[longtable,ltablex]
  Sollen mehrseitige Tabellen mit Seitenumbruch erstellt werden, ist dieses 
  Paket das Mittel der ersten Wahl. Für die Kombination mehrseitiger Tabellen 
  mit einer \Environment{tabularx}"=Umgebung können die Pakete 
  \Package{ltablex} oder besser noch \Package{ltxtable} verwendet werden.
\item[ltxtable]
  Wie bereits erwähnt sollte dieses Paket für mehrseitige Tabellen, die mit der 
  Umgebung \Environment{tabularx} erstellt wurden, verwendet werden. 
  Alternativ dazu lässt sich auch \Package{tabu} nutzen.
\item[tabu]
  \ChangedAt{%
    v2.02:\RecPack \Package{tabu} ist nur bedingt empfehlenswert
  }
  Dies ist ein relativ neues Paket, welches versucht, viele der zuvor genannten 
  Funktionalitäten zu implementieren und weitere bereitzustellen. Dafür werden 
  die Umgebungen \Environment{tabu} und \Environment{longtabu} definiert. Es 
  kann alternativ zu \Package{tabularx} verwendet werden und ist insbesondere 
  als Ersatz für das Paket \Package{ltxtable} empfehlenswert.
  
  \Attention{%
    Leider wären für das Paket in der Version~v2.8 seit geraumer Zeit ein paar 
    Bugfixes notwendig. Außerdem wird sich die 
    \hrfn{https://groups.google.com/d/topic/comp.text.tex/xRGJTC74uCI}{%
      Benutzerschnittstelle in einer zukünftigen Version
    } sehr stark ändern. Der Anwender sollte sich zumindest bewusst sein, dass 
    er mit der Version~v2.8 gesetzte Dokumente gegebenenfalls später anpassen 
    muss.
  }%
\index{Tabellen|?)}%
\end{packages}



\subsubsection{Gleitobjekte}
\index{Gleitobjekte|(?}%
Es werden Pakete für die Beeinflussung von Aussehen, Beschriftung und 
Positionierung von Gleitobjekten vorgestellt. Unter \autoref{sec:tips:floats} 
sind außerdem Hinweise zur manuellen Manipulation der Gleitobjektplatzierung zu 
finden.

\begin{packages}
\item[placeins]
  Mit diesem Paket kann die Ausgabe von Gleitobjekten vor Kapiteln und wahlweise
  Unterkapiteln erzwungen werden.
\item[flafter]
  Dieses Paket erlaubt die frühestmögliche Platzierung von Gleitobjekten im 
  ausgegeben Dokument erst an der Stelle ihres Auftretens im Quelltext. Diese 
  werden dementsprechend nie vor ihrer Definition am Anfang der Seite 
  erscheinen.
\item[caption]
  Die \KOMAScript-Klassen bieten mit der Option \Option{captions=\PSet}|declare|
  bereits einige Möglichkeiten zum Formatieren der Beschriftungen für 
  Gleitobjekte. Dieses Paket ist daher meist nur in gewissen Ausnahmefällen für 
  spezielle Anweisungen notwendig, allerdings auch bei der Verwendung 
  unbedenklich.
\item[subcaption]
  Diese Paket kann zum einfachen Setzen von Unterabbildungen oder "~tabellen 
  mit den entsprechenden Beschriftungen genutzt werden. Das dazu alternative 
  Paket \Package{subfig} sollte vermieden werden, da es nicht mehr gepflegt 
  wird und es mit diesem im Zusammenspiel mit anderen Paketen des Öfteren zu 
  Problemen kommt. Sollte der Funktionsumfang von \Package{subcaption} nicht 
  ausreichen, kann anstelle dessen das Paket \Package{floatrow} verwendet 
  werden, welches ähnliche Funktionalitäten wie \Package{subfig} bereitstellt.
\item[floatrow]
  Mit diesem Paket können global wirksame Einstellungen und Formatierungen für 
  \emph{alle} Gleitobjekte eines Dokumentes vorgenommen werden. So kann unter 
  anderem die verwendete Schrift (\Macro{floatsetup}[\PParameter{font=\dots}]) 
  innerhalb der Umgebungen \Environment{float} und \Environment{table} 
  eingestellt werden. Das typografisch richtige Setzen der Beschriftungen von 
  Abbildungen als Unterschriften 
  (\Macro{floatsetup}[\POParameter{figore}\PParameter{capposition=bottom}])
  sowie Tabellen als Überschriften 
  (\Macro{floatsetup}[\POParameter{table}\PParameter{capposition=top}])
  kann automatisch erzwungen werden~-- unabhängig von der Position des Befehls 
  zur Beschriftung \Macro{caption} innerhalb der Gleitobjektumgebung. Wird das 
  Verhalten so wie empfohlen mit dem \Package{floatrow}-Paket eingestellt, 
  sollte für eine richtige Platzierung der Tabellenüberschriften außerdem die 
  \KOMAScript-Option \Option{captions=tableheading} genutzt werden.
\index{Gleitobjekte|?)}%
\end{packages}


\subsubsection{Grafiken und Abbildungen}
\index{Grafiken|(?}%
Grafiken für wissenschaftliche Arbeiten sollten als Vektorgrafiken erstellt 
werden, um die Skalierbarkeit und hohe Druckqualität zu gewährleisten. 
Bestenfalls folgen diese auch dem Stil der dazugehörigen Arbeit.%
\footnote{%
  Für qualitativ hochwertige Dokumente sollten übernommene Grafiken nicht 
  direkt kopiert oder gescannt sondern im gewünschten Zielformat neu erstellt 
  und mit der Referenz auf die Quelle ins Dokument eingebunden werden.
}
Für das Erstellen eigener Vektorgrafiken, welche die \hologo{LaTeX}"=Schriften 
und das Layout des Hauptdokumentes nutzen, gibt es zwei mögliche Ansätze. 
Entweder die Grafiken werden ähnlich wie das Dokument \enquote{programmiert} 
oder Zeichenprogramme, welche wiederum die Ausgabe oder das Weiterreichen von 
Text an \hologo{LaTeX} unterstützen, werden genutzt. Für das Programmieren 
von Grafiken sollen hier die wichtigsten Pakete vorgestellt werden. Wie diese 
zu verwenden sind, ist den dazugehörigen Paketdokumentationen zu entnehmen. 
Außerdem wird im Tutorial \Tutorial{treatise} für beide Pakete jeweils ein 
Beispiel gegeben.

\begin{packages}
\item[tikz]<pgf>
  Dies ist ein sehr mächtiges Paket für das Programmieren von Vektorgrafiken 
  und sehr häufig~-- insbesondere bei Einsteigern~-- die erste Wahl bei der 
  Verwendung von \hologo{pdfLaTeX}.
\item[pstricks]
  Das Paket \Package{pstricks} stellt die zweite Variante zum Programmieren 
  von Grafiken dar. Mit diesem Paket exisiteren \emph{noch} mehr Möglichkeiten 
  bei der Erstellung eigener Grafiken, da mit \Package{pstricks} auf PostScript 
  zugegriffen werden kann und einige der bereitgestellten Befehle davon rege 
  Gebrauch machen. Der daraus resultierende Nachteil ist, dass mit 
  \Package{pstricks} die direkte Verwendung von \hologo{pdfLaTeX} nicht 
  möglich ist.
  
  Die Grafiken aus den \Environment{pspicture}"=Umgebungen müssen deshalb erst 
  über den Pfad \Path{latex \textrightarrow{} dvips \textrightarrow{} ps2pdf}
  in PDF"~Dateien gewandelt werden. Diese lassen sich von \hologo{pdfLaTeX} 
  anschließend als Abbildungen einbinden. Um dieses Vorgehen zu ermöglichen, 
  können folgende Pakete genutzt werden:
  %
  \begin{packages}
  \item[pst-pdf]
    Dieses Paket stellt Methoden für den Export von PostSript-Grafiken in 
    PDF-Datien bereit. Die einzelnen Aufrufe zur Kompilierung von DVI über 
    PostScript zu PDF müssen durch den Anwender manuell beziehungsweise über 
    die Ausgaberoutinen des verwendeten Editors durchgeführt werden.
  \item[auto-pst-pdf,pdftricks2]
    Das Paket automatisiert die Erzeugung der \Package{pstricks}"=Grafiken mit 
    dem Paket \Package{pst-pdf}. Dafür muss \hologo{pdfLaTeX} per Option mit 
    Schreibrechten ausgeführt werden. Dazu ist der Aufruf von \Path{pdflatex} 
    mit der Option \Path{-{}-shell-escape} beziehungsweise für Nutzer von 
    \Distribution{\hologo{MiKTeX}} mit \Path{-{}-enable-write18} notwendig. 
    Bitte beachten Sie die Hinweise in \autoref{sec:tips:auto-pst-pdf}. Eine 
    Alternative dazu ist das Paket \Package{pdftricks2}.
  \end{packages}
\end{packages}
%
Um bei der Erstellung von Grafiken mit \Package{pstricks} oder \Package{tikz} 
nicht bei jeder Änderung das komplette Dokument kompilieren zu müssen, können 
diese in separate Dateien ausgelagert werden. Hierfür sind die beiden Pakete 
\Package{standalone} oder \Package{subfiles} sehr nützlich.

Für das Zeichnen einer Grafik mit einem Bildbearbeitungsprogramm, welches die 
Weiterverarbeitung durch \hologo{LaTeX} erlaubt, möchte ich auf die freien 
Programme \Application{LaTeXDraw} und \Application{Inkscape} verweisen. 
Insbesondere das zuletzt genannte Programm ist sehr empfehlenswert. Für die 
erstellten Grafiken kann der Anwender den Export für die Einbindung in 
\hologo{LaTeX} manuell durchführen. In \autoref{sec:tips:svg} wird vorgestellt, 
wie sich dies automatisieren lässt.
\index{Grafiken|?)}%


\subsubsection{Aufteilung des Hauptdokumentes in Unterdateien}
Um während des Entwurfes eines Dokumentes die Zeitdauer für das Kompilieren zu 
verkürzen, kann dieses in Unterdokumente gegliedert werden. Dadurch wird es 
möglich, nur den momentan bearbeiteten Dokumentteil~-- respektive die aktuelle 
\Package{tikz}- oder \Package{pstricks}-Grafik~-- zu kompilieren. Die meiner 
Meinung nach besten Pakete für dieses Unterfangen werden folgend vorgestellt.
%
\begin{packages}
\item[standalone]
  \ChangedAt{%
    v2.02:Bugfix für Verwendung der Klasse \Class{standalone} aus dem Paket 
    \Package{standalone}%
  }
  Dieses Paket ist für das Erstellen eigenständiger (Unter)"~Dokumente gedacht, 
  welche später in ein Hauptdokument eingebunden werden können. Jedes dieser 
  Teildokumente benötigt eine eigene Präambel. Optional lassen sich die 
  Präambeln der Unterdokumente automatisch in ein Hauptdokument einbinden.
\item[subfiles]
\ChangedAt{v2.02:\RecPack \Package{subfiles}}
  Dieses Paket wählt einen etwas anderen Ansatz als \Package{standalone}. Es 
  ist von Anfang an dafür gedacht, ein dediziertes Hauptdokument zu verwenden. 
  Die darin mit \Macro{subfiles} eingebundenen Unterdateien nutzen bei der 
  autarken Kompilierung dessen Präambel.
\end{packages}
%
Unabhängig davon, ob Sie eines der beiden Pakete nutzen oder alles in einem 
Dokument belassen, ist es ratsam, eigens definierte Befehle, Umgebungen und 
ähnliches in ein separates Paket auszulagern. Dafür müssen Sie lediglich ein 
leeres \hologo{LaTeX}-Dokument erzeugen und es unter \File*{mypreamble.sty} 
oder einem anderen Namen im gleichen Ordner wie das Hauptdokument speichern. 
Dann können Sie in dieser Datei ihre Deklarationen ausführen und diese mit 
\Macro*{usepackage}[\PParameter{mypreamble}] in das Dokument einbinden. Dies 
hat den Vorteil, dass das Hauptdokument zum einen übersichtlich bleibt und Sie 
zum anderen Ihre persönliche Präambel generisch wachsen lassen und für andere 
Dokumente wiederverwenden können.

\subsubsection{Die kleinen und großen Helfer\dots}
Hier taucht alles auf, was nicht in die vorherigen Kategorien eingeordnet 
werden konnte.
%
\begin{packages}
\item[bookmark]\index{Lesezeichen}\index{Querverweise}%
  Dieses Paket verbessert und erweitert die von \Package{hyperref} angebotenen 
  Möglichkeiten zur Erstellung von Lesezeichen~-- auch Outline"=Einträge~-- im 
  PDF-Dokument. Beispielsweise können Schriftfarbe- und "~stil geändert werden.
\item[calc]\index{Berechnungen}%
  Normalerweise können Berechnungen nur mit Low-Level-\hologo{TeX}-Primitiven 
  im Dokument durchgeführt werden. Dieses Paket stellt eine einfachere Syntax 
  für Rechenoperationen der vier Grundrechenarten zur Verfügung. Zusätzlich 
  werden neue Befehle zur Bestimmung der Höhe und Breite bestimmter Textauszüge 
  definiert.
\item[chngcntr]\index{Zähler}%
  Das Paket erlaubt die Manipulation aller möglichen, bereits definierten 
  \hologo{LaTeX}-Zähler. Es können Zähler so umdefiniert werden, dass sie bei 
  der Änderung eines anderen Zählers automatisch zurückgesetzt werden oder eben 
  nicht. Ein kleines Beispiel dazu ist in \autoref{sec:tips:counter} zu finden.
\item[varioref]\index{Querverweise}%
  Mit diesem Paket lassen sich sehr gute Verweise auf bestimmte Seiten 
  erzeugen. Insbesondere, wenn der Querverweis auf die aktuelle, die 
  vorhergehende oder nachfolgende sowie im doppelseitigen Satz auf die 
  gegenüberliegende Seite erfolgt, werden passende Textbausteine für diesen 
  verwendet.
\item[cleveref]\index{Querverweise}%
  Dieses Paket vereint die Vorzüge von \Package{varioref} mit der automatischen 
  Benennung der referenzierten Objekte mit dem Befehl \Macro{autoref} aus dem 
  Paket \Package{hyperref}.
\item[marginnote]\index{Randnotizen}%
  Randnotizen, welche mit \Macro{marginpar} erzeugt werden, sind spezielle 
  Gleitobjekte in \hologo{LaTeX}. Dies kann dazu führen, dass eine Notiz am 
  Blattrand nicht direkt da gesetzt wird, wo diese intendiert war. Dieses Paket 
  stellt den Befehl \Macro{marginnote} für nicht"~gleitende Randnotizen zur 
  Verfügung. Alternativ dazu lässt sich auch \Package{mparhack} verwenden.
\item[todonotes]\index{Randnotizen}%
  Mit \Package{todonotes} können noch offene Aufgaben in unterschiedlicher 
  Formatierung am Blattrand oder im direkt Fließtext ausgegeben werden. Aus 
  allen Anmerkungen lässt sich eine Liste aller offenen Punkte erzeugen.
\item[listings]\index{Quelltextdokumentation}%
  Dieses Paket eignet sich hervorragend zur Quelltextdokumentation in 
  \hologo{LaTeX}. Es bietet die Möglichkeit, externe Quelldateien einzulesen 
  und darzustellen sowie die Syntax in Abhängigkeit der verwendeten 
  Programmiersprache hervorzuheben. Zusätzlich lässt sich ein Verzeichnis mit 
  allen eingebundenen sowie direkt im Dokument angegebenen Quelltextauszügen 
  erstellen.
  \ChangedAt{v2.02:\RecPack \Package{listings}}
  Wird \Package{listings} in Dokumenten mit UTF"~8-Kodierung verwendet, sollte 
  direkt nach dem Laden des Paketes in der Präambel Folgendes hinzugefügt 
  werden:
  \begin{Code}
    \lstset{%
      inputencoding=utf8,extendedchars=true,
      literate=%
        {ä}{{\"a}}1 {ö}{{\"o}}1 {ü}{{\"u}}1
        {Ä}{{\"A}}1 {Ö}{{\"O}}1 {Ü}{{\"U}}1
        {~}{{\textasciitilde}}1 {ß}{{\ss}}1
    }
  \end{Code}\vspace{-\baselineskip}%
\item[xparse]\index{Befehlsdeklaration}%
  Dieses mächtige Paket entstammt dem \hologo{LaTeX3}-Projekt und bietet für 
  die Erstellung eigener Befehle und Umgebungen einen alternativen Ansatz zu 
  den bekannten \hologo{LaTeX}"=Deklarationsbefehlen \Macro*{newcommand} und 
  \Macro*{newenvironment} sowie deren Derivaten. Mit \Package{xparse} wird es 
  möglich, obligatorische und optionale Argumente an beliebigen Stellen 
  innerhalb des Befehlskonstruktes zu definieren. Auch die Verwendung anderer 
  Zeichen als eckige Klammern für die Spezifizierung eines optionalen 
  Argumentes ist möglich.
\item[xkeyval,keyval,kvoptions,pgfkeys]\index{Befehlsdeklaration}%
  \ChangedAt{v2.02:\RecPack \Package{xkeyval}}
  Das \KOMAScript"=Bundle lädt das Paket \Package{keyval}, um Optionen mit 
  einer Schlüssel"=Wert"=Syntax deklarieren zu können. Zusätzlich wird von 
  \TUDScript das Paket \Package{kvsetkeys} geladen, um auf nicht definierte 
  Schlüssel reagieren zu können. Die Schlüssel"=Wert"=Syntax kann auch für 
  eigens definierte Makros genutzt werden, um sich das exzessive Verwenden von 
  optionalen Argumenten zu ersparen. Damit wäre folgende Definition möglich:
  \Macro*{newcommand}\Macro*{Befehl}[%
    \OParameter{Schlüssel"=Wert"=Liste}\Parameter{Argument}%
  ].
  
  Das Paket \Package{xkeyval} erweitert insbesondere die Möglichkeiten zur 
  Deklaration unterschiedlicher Typen von Schlüsseln. Sollten die bereits durch 
  \TUDScript geladenen Pakete \Package{keyval} und \Package{kvsetkeys} in ihrer 
  Funktionalität nicht ausreichen, kann dieses Paket verwendet werden. Für die 
  Entwicklung eigener Pakete, deren Optionen das Schlüssel"=Wert"=Format 
  unterstützen, kann das Paket \Package{scrbase} genutzt werden. Soll aus einem 
  Grund auf \KOMAScript{} gänzlich verzichtet werden, sind die beiden Pakete 
  \Package{kvoptions} oder \Package{pgfkeys} eine Alternative.
\item[afterpage]
  Der Befehl \Macro{afterpage}[\Parameter{\dots}] kann genutzt werden, um den 
  Inhalt aus dessen Argument direkt nach der Ausgabe der aktuellen Seite 
  auszuführen.
\item[pagecolor]\index{Farben}%
  Mit dem Paket kann die Hintergrundfarbe der Seiten im Dokument geändert 
  werden.
\item[pdfpages]
  Das Paket ermöglicht die Einbindung von einzelnen oder mehreren PDF"~Dateien.
\item[mwe]\index{Minimalbeispiel|?}%
  \ChangedAt{v2.02:\RecPack \Package{mwe}}
  Mit diesem Paket lassen sich sehr einfach Minimalbeispiele erzeugen, die 
  sowohl Blindtexte respektive Abbildungen enthalten sollen.
\item[filemod]
  Wird entweder \hologo{pdfLaTeX} oder \hologo{LuaLaTeX} als Prozessor 
  eingesetzt, können mit diesem Paket das Änderungsdatum zweier Dateien 
  miteinander verglichen und in Abhängigkeit davon definierbare Aktionen 
  ausgeführt werden.
\item[coseoul]
  Mit diesem Paket kann die Struktur der Gliederung relativ angegeben werden. 
  Es wird keine absolute Gliederungsebene (\Macro*{chapter}, \Macro*{section}) 
  angegeben sondern die Relation zwischen vorheriger und aktueller Ebene 
  (\Macro*{levelup}, \Macro*{levelstay}, \Macro*{leveldown}).
\item[dprogress]\index{Debugging}%
  Das Paket schreibt bei der Kompilierung des Dokumentes die Gliederung in die 
  Logdatei. Dies kann im Fehlerfall beim Auffinden des Problems im Dokument 
  helfen. Allerdings werden dafür die Gliederungsebenen so umdefiniert, dass 
  diese keine optionalen Argumente mehr unterstützen,was jedoch für die 
  \TUDScript-Klassen von essentieller Bedeutung ist. Zum Debuggen kann es 
  trotzdem sporadisch eingesetzt werden.
\end{packages}

\subsubsection{Bugfixes}
\begin{packages}
\item[scrhack]<koma-script>
  Das Paket behebt Kompatibilitätsprobleme der \KOMAScript-Klassen mit den 
  Paketen \Package{hyperref}, \Package{float}, \Package{floatrow} und
  \Package{listings}. Es ist durchaus empfehlenswert, jedoch sollte unbedingt 
  die Dokumentation beachtet werden.
\item[mparhack]
  Zur Behebung falsch gesetzter Randnotizen wird ein Bugfix für 
  \Macro{marginpar} bereitgestellt. Alternativ dazu lässt sich auch 
  \Package{marginnote} verwenden.
\end{packages}

\ToDo[doc]{%
  \Package{blindtext},
  \Package{crop},
  \Package{filecontents},
  \Package{fix-cm},
  \Package{morewrites},
  \Package{scrwfile}
}[v2.05]
