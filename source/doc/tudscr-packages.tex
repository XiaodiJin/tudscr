\chapter{Unerlässliche und beachtenswerte Pakete}
\label{sec:packages}
\index{Kompatibilität!Pakete}
\section{Von den neuen Hauptklassen benötigte Pakete}
\label{sec:packages:needed}
\subsection{Erforderliche Pakete bei der Schriftinstallation}
%
Für die Installation der Schriften sind die folgend genannten Pakete von
\emph{essentieller} Bedeutung und daher \emph{zwingend} notwendig. Das 
Vorhandensein dieser wird durch die jeweiligen Schriftinstallationsskripte
(siehe~\autoref{sec:install}) geprüft und die Installation beim Fehlen eines 
oder mehrerer Pakete mit einer entsprechenden Warnung abgebrochen.
%
\begin{packages}
\item[fontinst]
  Dieses Paket wird für die Installation der Schriften \Univers sowie \DIN 
  benötigt.
\item[cmbright]
  Alle mathematischen Glyphen und Symbole, die nicht in den \Univers-Schriften 
  enthalten sind, werden diesem Paket entnommen.
\item[iwona]
  Sowohl für die \Univers-Schriftfamilie als auch für \DIN werden fehlenden 
  Glyphen und Symbole hieraus entnommen.
\end{packages}
%
Zusätzlich werden die \texttt{Schreibmaschinenschriften} aus dem Paket 
\Package{lmodern} verwendet.

\subsection{Notwendige Pakete für die Verwendung der Hauptklassen}
In diesem \autorefname werden alle Pakete genannt, die von den neuen Klassen 
zwingend benötigt und geladen werden, um den Anwender das mehrmalige Laden 
dieser Pakete oder mögliche Konflikte mit anderen Paketen zu ersparen.
%
\begin{packages}
\item[scrbase](koma-script)
  Dieses Paket gehört zum \KOMAScript-Bundle und erlaubt das Definieren von 
  Klassenoptionen im Stil von \KOMAScript, welche auch noch nach dem Laden der 
  Klasse mit den Befehlen \Macro{TUDoption} und \Macro{TUDoptions} geändert 
  werden können. Von diesem wird das Paket \Package{keyval} geladen, welches 
  das Definieren von Klassen"~ und Paketoptionen sowie Parametern nach dem 
  Schlüssel"=Wert"=Prinzip ermöglicht.
\item[kvsetkeys]
  Hiermit wird das von \Package{scrbase} geladene Paket \Package{keyval} 
  verbessert. Unter anderem kann das Verhalten für einen übergebenen, 
  unbekannten Schlüssel festgelegt werden.
\item[scrlayer-scrpage](koma-script)
  \ChangedAt{v2.02!\Package{scrlayer-scrpage}: Für Verwendung notwendig}
  Dieses \KOMAScript-Paket wird für die \PageStyle{tudheadings}-Seitenstile 
  benötigt.
\item[etoolbox]
  Es werden viele Funktionen zum Testen und zur Ablaufkontrolle bereitgestellt 
  und das einfache Manipulieren vorhandener Makros ermöglicht.
\item[geometry]\index{Satzspiegel}
  Das Paket ist essentiell für die \TUDScript-Klassen. Es wird zum Festlegen 
  der Seitenränder respektive des Satzspiegels verwendet.
\item[environ]\index{Befehle!Deklaration}
  Es wird eine verbesserte Deklaration von Umgebungen ermöglicht, bei der auch 
  beim Abschluss der Umgebung auf die übergebenen Parameter zugegriffen werden 
  kann. Dies wird die Neugestaltung der \Environment{abstract}"=Umgebung 
  benötigt. Das Paket lädt \Package{trimspaces}, womit das Entfernen von 
  überflüssigen Leerraum um einen Strings ermöglicht wird.
\item[textcase]\index{Schriftauszeichnung}
  Mit \Macro{MakeTextUppercase} wird die Großschreibung der Überschriften in 
  \DIN erzwungen. Im \emph{Ausnahmefall} kann dies mit \Macro{NoCaseChange} 
  unterbunden werden.
\item[graphicx]\index{Grafiken}
  Dies ist das De-facto-Standard-Paket zum Einbinden von Grafiken. Zum Setzen 
  des Logos der \TnUD im Kopf wird \Macro{includegraphics} genutzt.
\item[xcolor]\index{Farben}
  Damit werden die Farben des \CDs zur Verwendung im Dokument definiert. 
  Genaueres ist bei der Beschreibung von \Package{tudscrcolor}'auto' zu finden.
\item[afterpage]
  \ChangedAt{v2.02!\Package{afterpage}: Für Verwendung notwendig}
  Der Befehl \Macro*{afterpage}\Parameter{\dots} kann genutzt werden, um den 
  Inhalt aus dessen Argument direkt nach der Ausgabe der aktuellen Seite 
  auszuführen. Es wird für den Titelkopf im \CD für das zweispaltige Layout 
  benötigt (\autoref{sec:title}).
\end{packages}
%
Soll eines der hier aufgezählten Pakete mit bestimmten Optionen geladen werden, 
so müssen diese bereits \emph{vor} der Definition der Dokumentklasse an das 
entsprechende Paket werden.
%
\begin{Example}
Das Weiterreichen von Optionen an Pakete muss folgendermaßen erfolgen:
\begin{Code}[escapechar=§]
\PassOptionsToPackage§\Parameter{Optionenliste}\Parameter{Paket}§
\documentclass§\OParameter{Klassenoptionen}\PParameter{tudscr\dots}§
\end{Code}
\end{Example}\vspace{-\baselineskip}%



\section{Durch \TUDScript direkt unterstütze Pakete}
%
\begin{packages}
\item[hyperref]\index{Lesezeichen}\index{Querverweise}
  Hiermit können in einem PDF-Dokument Lesezeichen, Querverweise und   
  Hyperlinks erstellt werden. Wird es geladen, sind die Option 
  \Option{tudbookmarks} sowie \Macro{tudbookmark} nutzbar. Das Paket 
  \Package{bookmark} erweitert die Unterstützung nochmals. \Package{hyperref} 
  beziehungsweise \Package{bookmark} sollte zuletzt in der Präambel eingebunden 
  werden.%
  \footnote{%
    \Package{glossaries} ist eine von wenigen Ausnahmen und muss \textbf{nach} 
    \Package{hyperref} geladen werden.
  }
\item[isodate]\index{Datum|?}
  Dieses Paket formatiert mit \Macro{printdate}\Parameter{Datum} die Ausgabe 
  eines Datums automatisch in ein spezifiziertes Format. Wird es geladen, 
  werden alle Datumsfelder, welche durch die \TUDScript-Klassen definiert 
  wurden,%
  \footnote{%
    \Macro{date}, \Macro{dateofbirth}, \Macro{defensedate}, \Macro{duedate}, 
    \Macro{issuedate}
  }
  in diesem Format ausgegeben.
\item[multicol]\index{Zweispaltensatz|?}
  Hiermit kann jeglicher beliebiger Inhalt in zwei oder mehr Spalten ausgegeben 
  werden, wobei~-- im Gegensatz zur \hologo{LaTeX}-Option \Option{twocolumn}~-- 
  für einen Spaltenausgleich gesorgt wird. Unterstützt wird das Paket innerhalb 
  der Umgebungen \Environment{abstract} und \Environment{tudpage}.
\item[quoting]\index{Zitate}
  \hologo{LaTeX} bietet von Haus aus \emph{zwei} verschiedene Umgebungen für 
  Zitate und ähnliches. Beide sind in ihrer Ausprägung starr und ignorieren 
  beispielsweise die Einstellungen von \Option{parskip}. Dies wird durch die 
  Umgebung \Environment{quoting} verbessert. Wird das Paket geladen, kommt 
  diese gegebenenfalls innerhalb der \Environment{abstract}"=Umgebung zum 
  Einsatz.
\item[ragged2e]\index{Worttrennung}
  Das Paket verbessert den Flattersatz, indem für diesen die Worttrennung 
  aktiviert wird.
\item[pagecolor]\index{Farben}
  Mit dem Paket kann die Hintergrundfarbe der Seiten im Dokument geändert 
  werden. Nach der Ausgabe einer farbigen Titel"~, Teile"~, oder Kapitelseite 
  wird auf diese zurückgeschaltet.
\end{packages}



\section{Empfehlenswerte Pakete}
\label{sec:packages:recommended}
In diesem \autorefname wird eine Vielzahl an Paketen~-- zumeist kurz~-- 
vorgestellt, welche sich für mich persönlich bei der Arbeit mit \hologo{LaTeX} 
bewährt haben. Einige davon werden außerdem im Tutorial \Tutorial{treatise} in 
ihrer Anwendung beschrieben. Für detaillierte Informationen sowie bei Fragen zu 
den einzelnen Paketen sollte die jeweilige Dokumentation zu Rate gezogen
werden,\footnote{Kommandozeile/Terminal: \Path{texdoc\,\PName{Paketname}}}
das Lesen der hier gegebenen Kurzbeschreibung ersetzt dies in keinem Fall. 


\subsection{Pakete zur Verwendung in jedem Dokument}
Die hier vorgestellten Pakete gehören meiner Meinung nach in die Präambel eines 
jeden Dokumentes. Egal, in welcher Sprache das Dokument verfasst wird, sollte 
diese mit dem Paket \Package{babel} definiert werden~-- auch wenn dies 
Englisch ist. Für deutschsprachige Dokumente ist für eine annehmbare 
Worttrennung das Paket \Package{hyphsubst} unbedingt zu verwenden.

\begin{packages}
\item[fontenc]\index{Zeichensatzkodierung}
  Das Paket erlaubt Festlegung der Zeichensatzkodierung des Ausgabefonts. Als 
  Voreinstellung ist die Ausgabe als 7"~bit kodierte Schrift gewählt, was unter 
  anderem dazu führt, dass keine echten Umlaute im erzeugten PDF-Dokument 
  verwendet werden. Um auf 8"~bit"~Schriften zu schalten, sollte man
  \Macro*{usepackage}\POParameter{T1}\PParameter{fontenc} nutzen.
\item[selinput]\index{Eingabekodierung}
  Hiermit erfolgt die (automatische) Festlegung der Eingabekodierung. Diese ist 
  vom genutzten \hyperref[sec:tips:editor]{Editors (\autoref{sec:tips:editor})} 
  und den darin gewählten Einstellung abhängig. Mit:
  \begin{Code}
  \usepackage{selinput}
  \SelectInputMappings{adieresis={ä},germandbls={ß}}
  \end{Code}\vspace{-\baselineskip}%
  wird es verwendet. Dies macht den Quelltext portabel, womit beispielsweise 
  einfach via Copy~\&~Paste ein \hrfn{http://www.komascript.de/minimalbeispiel}%
  {Minimalbeispiel} bei Problemstellungen in einem Forum bereitgestellt werden 
  kann. Alternativ dazu lässt sich mit dem Paket \Package{inputenc} die zu 
  verwendende Eingabekodierung manuell einstellen
  (\Macro*{usepackage}\OParameter{Eingabekodierung})\PParameter{inputenc}).
\item[microtype]\index{Typographie}
  Dieses Paket kümmert sich um den optischen Randausgleich%
  \footnote{englisch: protrusion, margin kerning}
  und das Nivellieren der Wortzwischenräume%
  \footnote{englisch: font expansion}
  im Dokument. Es funktioniert nicht mit der klassischen \hologo{TeX}-Engine, 
  wohl jedoch mit \hologo{pdfTeX} als auch \hologo{LuaTeX} sowie \hologo{XeTeX}.
\item[babel]\index{Sprachunterstützung}\index{Bezeichner}
  Mit diesem Paket erfolgt die Einstellung der im Dokument verwendeten 
  Sprache(n). Bei mehreren angegebenen Sprachen ist die zuletzt geladene die 
  Hauptsprache des Dokumentes. Die gewünschten Sprachen sollten als nicht als 
  Paketoption sondern als Klassenoption und gesetzt werden, damit auch andere 
  Pakete auf die Spracheinstellungen zugreifen können. Für deutschsprachige 
  Dokumente ist die Option \Option*{ngerman} für die neue oder \Option*{german} 
  für die alte deutsche Rechtschreibung zu verwenden. 
  
  Mit dem Laden von \Package{babel} und der dazugehörigen Sprachen werden 
  sowohl die Trennmuster als auch die sprachabhängigen Bezeichner angepasst.
  Von einer Verwendung der obsoleten Pakete \Package*{german} beziehungsweise 
  \Package*{ngerman} anstelle von \Package{babel} wird abgeraten. Für 
  \hologo{LuaLaTeX} und \hologo{XeLaTeX} kann das Paket \Package{polyglossia} 
  genutzt werden.
\item[hyphsubst]\index{Worttrennung|!}
  Die möglichen Trennstellen von Wörtern wird von \hologo{LaTeX} mithilfe 
  eines Algorithmus berechnet. Dieser ist jedoch in seiner ursprünglichen Form 
  für die englische Sprache konzipiert worden. Für deutschsprachige Texte wird 
  die Worttrennung~-- insbesondere für zusammengeschriebenen Wörtern~-- mit dem 
  Paket \Package{hyphsubst} entscheidend verbessert. Dafür wird ein um 
  Trennstellen ergänztes Wörterbuch aus dem Paket \Package{dehyph-exptl} 
  genutzt. \Package{hyphsubst} muss bereits \emph{vor} den Dokumentklassen 
  selbst wie folgt geladen werden:
  \begin{Code}[escapechar=§]
  \RequirePackage§\POParameter{ngerman=ngerman-x-latest}\PParameter{hyphsubst}§
  \documentclass§\OParameter{Klassenoptionen}\PParameter{tudscr\dots}§
  \end{Code}\vspace{-\baselineskip}%
  Sollte trotzdem einmal ein bestimmtes Wort falsch getrennt werden, so kann 
  die Worttrennung dieses Wortes manuell und global geändert werden. Dies wird 
  mit dem Befehl \Macro{hyphenation}\PParameter{Sil-ben-tren-nung} gemacht. Es 
  ist zu beachten, dass dies für alle Flexionsformen des Wortes erfolgen 
  sollte. Für eine lokale/temporäre Worttrennung kann mit Befehlen aus dem 
  Paket \Package{babel} gearbeitet werden. Diese sind:
  %
  \vskip\topsep\noindent
  \begin{tabular}{@{}ll}
  \textbf{Beschreibung} & \textbf{Befehl}\tabularnewline
  ausschließliche Trennstellen & \textbackslash-\tabularnewline
  zusätzliche Trennstellen & "'-\tabularnewline
  Umbruch ohne Trennstrich & "'"'\tabularnewline
  Bindestrich, welcher weitere Trennstellen erlaubt & "'=\tabularnewline
  geschützter Bindestrich ohne Umbruch & "'\textasciitilde\tabularnewline
  \end{tabular}
\end{packages}

\subsection{Pakete zur situativen Verwendung}
\subsubsection{Verzeichnisse aller Art}
\index{Verzeichnisse|?}
Neben dem Erstellen des eigentlichen Dokumentes sind für eine wissenschaftliche 
Arbeit meist auch allerhand Verzeichnisse gefordert. Fester Bestandteil ist 
dabei das Literaturverzeichnis, auch ein Abkürzungs- und Formelzeichen- 
beziehungsweise Symbolverzeichnis werden häufig gefordert. Gegebenenfalls wird 
auch noch ein Glossar benötigt. Hier werden die passenden Pakete vorgestellt. 
Sollen im Dokument komplette Quelltexte oder auch nur Auszüge daraus erscheinen 
und für diese auch gleich ein entsprechendes Verzeichnis generiert werden, so 
sei auf das Paket \Package{listings}'full' verwiesen.

\begin{packages}
\item[biblatex]\index{Literaturverzeichnis}
  Das Paket gibt es seit geraumer Zeit und es kann als legitimer Nachfolger zu 
  \hologo{BibTeX} gesehen werden. Ähnlich dazu bietet \Package{biblatex} 
  die Möglichkeit, Literaturdatenbanken einzubinden und verschiedene Stile der 
  Referenzierung und Darstellung des Literaturverzeichnisses auszuwählen. 
  
  Mit \Package{biblatex} ist die Anpassung eines bestimmten Stiles wesentlich 
  besser umsetzbar als mit \hologo{BibTeX}. Wird \Application{biber} für die 
  Sortierung des Literaturverzeichnisses genutzt, ist die Verwendung einer 
  UTF"~8-kodierten Literaturdatenbank problemlos möglich. In Verbindung mit 
  \Package{biblatex} wird die zusätzliche Nutzung des Paketes 
  \Package{csquotes} sehr empfohlen.
\item[acro]\index{Abkürzungsverzeichnis}
  Soll lediglich ein Abkürzungsverzeichnis erstellt werden, ist dieses Paket 
  die erste Wahl. Es stellt Befehle zur Definition von Abkürzungen sowie zu 
  deren Verwendung im Text und zur sortierten Ausgabe eines Verzeichnisses 
  bereit. Alternativ dazu kann das Paket \Package{acronym} verwendet werden. 
  Die Sortierung des Abkürzungsverzeichnisses muss hier allerdings manuell 
  durch den Anwender erfolgen.
\item[glossaries]\index{Glossar}\index{Abkürzungsverzeichnis}%
  \index{Formelzeichenverzeichnis}\index{Symbolverzeichnis}%
  Dies ist ein sehr mächtiges Paket zum Erstellen eines Glossars sowie 
  Abkürzungs- und Symbolverzeichnisses. Die mannigfaltige Anzahl an Optionen 
  ist zu Beginn eventuell etwas abschreckend. Insbesondere wenn Verzeichnisse 
  für Abkürzungen \emph{und} Formelzeichen beziehungsweise Symbole benötigt 
  werden, sollte man dieses Paket in Erwägung ziehen.
  
  Alternativ dazu kann für ein Symbolverzeichnis auch lediglich eine manuell 
  gesetzte Tabelle genutzt werden. Das hierfür sehr häufig empfohlene Paket 
  \Package{nomencl} bietet meiner Meinung nach demgegenüber keinerlei Vorteile.
\end{packages}

\subsubsection{Grafiken und Abbildungen}
\index{Grafiken|?}
Grafiken für wissenschaftliche Arbeiten sollten als Vektorgrafiken erstellt 
werden, um Skalierbarkeit und hohe Druckqualität zu gewährleisten. Bestenfalls 
folgen diese auch dem Stil der dazugehörigen Arbeit.%
\footnote{%
  Aus anderen Arbeiten übernommene Grafiken sollten meiner Meinung nach für 
  qualitativ hochwertige Dokumente nicht direkt kopiert sondern nach der 
  Vorlage im entsprechenden Format neu erstellt und mit der Referenz auf die 
  Quelle ins Dokument eingebunden werden.
}
Für das Erstellen eigener Vektorgrafiken in \hologo{LaTeX}, die unter anderem 
die \hologo{LaTeX}"=Schriften und das Layout des Hauptdokumentes nutzen, gibt
es zwei mögliche Wege. Entweder, man \enquote{programmiert} die Grafiken, 
ähnlich wie auch das Dokument selber oder man nutzt Zeichenprogramme, die 
wiederum die Ausgabe oder das Weiterreichen von Text an \hologo{LaTeX} 
unterstützen. Für das Programmieren von Grafiken sollen hier die wichtigsten 
Pakete vorgestellt werden. Wie diese zu verwenden sind, ist den dazugehörigen 
Paketdokumentationen zu entnehmen.

\begin{packages}
\item[tikz](pgf)
  Dies ist ein sehr mächtiges Paket für das Programmieren von Vektorgrafiken 
  und höchstwahrscheinlich die erste Wahl bei der Verwendung von 
  \hologo{pdfLaTeX}.
\item[pstricks]
  Das Paket \Package{pstricks} stellt die zweite Variante zum Programmieren 
  von Grafiken dar. Mit diesem Paket hat man \emph{noch} mehr Möglichkeiten bei 
  der Erstellung eigener Grafiken, da man mit \Package{pstricks} auf 
  PostScript zugreifen kann und einige der bereitgestellten Befehle davon rege 
  Gebrauch machen. Der daraus resultierende Nachteil ist, dass mit 
  \Package{pstricks} die direkte Verwendung von \hologo{pdfLaTeX} nicht 
  möglich ist.
  
  Die Grafiken aus den \Environment{pspicture}"=Umgebungen müssen deshalb erst 
  über den Pfad \Path{latex \textrightarrow{} dvips \textrightarrow{} ps2pdf}
  in PDF"~Dateien gewandelt werden. Diese lassen sich von \hologo{pdfLaTeX} 
  anschließend als Abbildungen einbinden. Um dieses Vorgehen zu ermöglichen, 
  können folgende Pakete genutzt werden:
  %
  \begin{packages}
  \item[pst-pdf]
    Dieses Paket stellt die prinzipiellen Methoden für den Export bereit. Die 
    einzelnen Aufrufe zur Kompilierung von DVI über PostScript zu PDF müssen 
    manuell durchgeführt werden.
  \item[auto-pst-pdf]
    Das Paket automatisiert die Erzeugung der \Package{pstricks}"=Grafiken. 
    Dafür muss \hologo{pdfLaTeX} mit der Option \Path{-{}-shell-escape} 
    aufgerufen werden.
  \item[pdftricks2]
    Ein weiteres Paket mit der gleichen Intention wie \Package{auto-pst-pdf}, 
    allerdings anders implementiert. Auch hier ist für \hologo{pdfLaTeX} die 
    Option \Path{-{}-shell-escape} notwendig.
  \end{packages}
  %
\item[standalone]
  \ChangedAt{v2.02!\Class{standalone}: Probleme behoben}
  Sollte \Package{tikz} und/oder \Package{pstricks} eingesetzt werden, kann 
  das Paket \Package{standalone} genutzt werden, um die Grafiken einerseits 
  als eigenständiges Dokument übersetzen zu können und andererseits diese 
  Grafiken mit dem eigentlichen Dokument zu kompilieren. Ebenfalls kann damit 
  das Hauptdokument in mehrere Unterdokumente geteilt werden. Für diese Aufgabe 
  ist außerdem das Paket \Package{subfiles} eine Alternative. Mit einem solchen 
  Vorgehen kann im Entwurfsprozess die Zeitdauer verkürzt werden, um den gerade 
  bearbeiteten Teil des Dokumentes\footnote{Kapitel oder Grafik} zu prüfen, da 
  hierfür nicht immer das vollständige Hauptdokument kompiliert werden muss. 
\end{packages}
%
Als Möglichkeit des Zeichnens einer Grafik mit einem Bildbearbeitungsprogramm, 
welches die Weiterverarbeitung durch \hologo{LaTeX} erlaubt, möchte ich auf die 
freien Programme \Application{LaTeXDraw} und \Application{Inkscape} verweisen. 
Insbesondere das zuletzt genannte Programm ist sehr empfehlenswert. Für die 
erstellten Grafiken kann man den Export für die Einbindung in \hologo{LaTeX} 
manuell durchführen. In \autoref{sec:tips:svg} wird vorgestellt, wie sich dies 
automatisieren lässt.

\subsubsection{Gleitobjekte}
\index{Gleitobjekte|?}
\index{Tabellen}\index{Grafiken}
Es werden Pakete für die Beeinflussung von Aussehen, Beschriftung und 
Positionierung von Gleitobjekten vorgestellt. Unter \autoref{sec:tips:floats} 
sind außerdem Hinweise zur manuellen Manipulation der Gleitobjektplatzierung zu 
finden.

\begin{packages}
\item[caption]\index{Gleitobjekte!Beschriftung}
  Die \KOMAScript-Klassen bietet bereits einige Möglichkeiten zum Setzen der 
  Beschriftungen für Gleitobjekte. Dieses Paket ist daher meist nur in gewissen
  Ausnahmefällen für spezielle Anweisungen notwendig, allerdings auch bei der 
  Verwendung unbedenklich.
\item[subcaption]\index{Gleitobjekte!Beschriftung}
  Diese Paket kann zum einfachen Setzen von Unterabbildungen oder "~tabellen 
  mit den entsprechenden Beschriftungen genutzt werden. Das dazu alternative 
  Paket \Package{subfig} sollte vermieden werden, da es nicht mehr gepflegt 
  wird und es mit diesem im Zusammenspiel mit anderen Paketen des Öfteren zu 
  Problemen kommt. Sollte der Funktionsumfang von \Package{subcaption} nicht 
  ausreichen, kann anstelle dessen das Paket \Package{floatrow} verwendet 
  werden, welches ähnliche Funktionalitäten wie \Package{subfig} bereitstellt.
\item[floatrow]\index{Gleitobjekte!Beschriftung}
  Mit diesem Paket können global wirksame Einstellungen und Formatierungen für 
  \emph{alle} Gleitobjekte eines Dokumentes vorgenommen werden. So kann unter 
  anderem die verwendete Schrift (\Macro*{floatsetup}\PParameter{font=\dots}) 
  innerhalb der Umgebungen \Environment*{float} und \Environment*{table} 
  eingestellt werden. Das typographisch richtige Setzen der Beschriftungen von 
  Abbildungen als Unterschriften 
  (\Macro*{floatsetup}\POParameter{figore}\PParameter{capposition=bottom})
  sowie Tabellen als Überschriften 
  (\Macro*{floatsetup}\POParameter{table}\PParameter{capposition=top})
  kann automatisch erzwungen werden~-- unabhängig von der Position des Befehls 
  zur Beschriftung \Macro*{caption} innerhalb der Gleitobjektumgebung. Wird das 
  Verhalten so wie empfohlen mit dem \Package{floatrow}-Paket eingestellt, 
  sollte für eine richtige Platzierung der Tabellenüberschriften außerdem die 
  \KOMAScript-Option \Option{captions}[tableheading] genutzt werden.
\item[placeins]\index{Gleitobjekte!Platzierung}
  Mit diesem Paket kann die Ausgabe von Gleitobjekten vor Kapiteln und wahlweise
  Unterkapiteln erzwungen werden.
\item[flafter]\index{Gleitobjekte!Platzierung}
  Dieses Paket erlaubt die frühestmögliche Platzierung von Gleitobjekten im 
  ausgegeben Dokument erst an der Stelle ihres Auftretens im Quelltext. Sie 
  werden dementsprechend nie vor ihrer Definition am Anfang der Seite 
  erscheinen.
\end{packages}

\subsubsection{Listen und Tabellen}
\index{Listen|?}\index{Tabellen|?}
Für den Tabellensatz in \hologo{LaTeX} werden von Haus aus die Umgebungen 
\Environment*{tabbing} und \Environment*{tabular} beziehungsweise 
\Environment*{tabular*} bereitgestellt, welche in ihrer Funktionalität meist 
für einen qualitativ hochwertigen Tabellensatz nicht ausreichen. Es werden 
deshalb Pakete vorgestellt, die zusätzlich verwendet werden können. Ebenfalls 
können die Umgebungen für Auflistungen in \hologo{LaTeX} verbessert werden.

\begin{packages}
\item[enumitem]
  Das Paket \Package{enumitem} erweitert die rudimentären Funktionalitäten der 
  \hologo{LaTeX}"=Standardlisten \Environment{itemize}, \Environment{enumerate}
  sowie \Environment{description} und ermöglicht die individuelle Anpassung 
  dieser durch die Bereitstellung vieler optionale Parameter nach dem
  Schlüssel"=Wert"=Prinzip. 
  
  Eine von mir sehr häufig genutzte Funktion ist beispielsweise die Entfernung 
  des zusätzlichen Abstand zwischen den einzelnen Einträgen einer Liste mit 
  \Macro*{setlist}\PParameter{noitemsep}.
\item[array]
  Dieses Paket ermöglicht die erweiterte Definition von Tabellenspalten sowie 
  das Erstellen neuer Spaltentypen mit \Macro*{newcolumntype}. Außerdem kann 
  mit \Macro*{extrarowheight} die Höhe der Zeilen einer Tabelle angepasst 
  werden.
\item[multirow]
  Es wird der Befehl \Macro*{multirow} definiert, der~-- ähnlich zum Makro 
  \Macro*{multicolumn}~-- das Zusammenfassen von mehreren Zeilen in einer 
  Spalte ermöglicht.
\item[booktabs]
  Für einen guten Tabellensatz mit \hologo{LaTeX} gibt es bereits zahlreiche 
  \hrfn{http://userpage.fu-berlin.de/latex/Materialien/tabsatz.pdf}{Tipps} im 
  Internet zu finden. Zwei Regeln sollten dabei definitiv beachtet werden:
  %
  \begin{enumerate}[itemindent=0pt,labelwidth=*,labelsep=1em,label=\Roman*.]
  \itempackages keine vertikalen Linien
  \itempackages keine doppelten Linien
  \end{enumerate}
  %
  Das Paket \Package{booktabs} ist für den Satz von hochwertigen Tabellen~-- 
  zusammen mit der deutschsprachigen Dokumentation \Package*{booktabs-de}~-- 
  eine große Hilfe und stellt neue Befehle für horizontale Linien bereit.
\item[widetable]
  Mit der Standard"=\hologo{LaTeX}"=Umgebung \Environment*{tabular*} kann eine 
  Tabelle mit einer definierten Breite gesetzt werden. Dieses Paket stellt die 
  Umgebung \Environment*{widetable} zur Verfügung, die als Alternative genutzt 
  werden kann und eine symmetrische Tabelle erzeugt.
\item[tabularx]
  Auch mit diesem Paket kann die Breite eine Tabelle spezifiziert werden. Dafür 
  wird der neue Spaltentyp \PValue{X} definiert, welcher als Argument der 
  \Environment*{tabularx}"=Umgebung beliebig häufig angegeben werden kann
  (\Macro*{begin}\PParameter{tabularx}\Parameter{Breite}\Parameter{Spalten}). 
  \PValue{X}"~Spalten entsprechen denen vom Typ~\PValue{p}\OParameter{Breite}, 
  die Breite wird allerdings aus der gegebenen Tabellenbreite und dem 
  benötigten Platz der Standardspalten automatisch berechnet.
\item[longtable]
  Sollen mehrseitige Tabellen mit Seitenumbruch erstellt werden, ist dieses 
  Paket das Mittel der ersten Wahl. Für die Kombination mehrseitiger Tabellen 
  mit einer \Environment*{tabularx}"=Umgebung können die Pakete 
  \Package{ltablex} oder besser noch \Package{ltxtable} verwendet werden.
\item[ltxtable]
  Wie bereits erwähnt sollte dieses Paket für mehrseitige Tabellen, die mit der 
  Umgebung \Environment*{tabularx} erstellt wurden, verwendet werden. 
  Alternativ dazu kann man auch \Package{tabu} nutzen.
\item[tabu]
  Dies ist ein relativ neues Paket, welches versucht, viele der zuvor genannten 
  Funktionalitäten zu implementieren und weitere bereitzustellen. Dafür werden 
  die Umgebungen \Environment*{tabu} und \Environment*{longtabu} definiert. Es 
  kann als Alternative für \Package{tabularx} verwendet werden und ist 
  insbesondere als Ersatz für das Paket \Package{ltxtable} empfehlenswert.
\item[tabularborder]
  Bei Tabellen wird zwischen Spalten automatisch ein horizontaler Abstand 
  (\Length{tabcolsep}) gesetzt~-- besser gesagt jeweils vor und nach einer 
  Spalte. Dies geschieht auch \emph{vor} der ersten und \emph{nach} der letzten 
  Spalte. Dieser zusätzliche Platz an den äußeren Rändern kann störend wirken, 
  insbesondere wenn die Tabelle über die komplette Textbreite gesetzt wird. Mit 
  dem Paket \Package{tabularborder} kann dieser Platz automatisch entfernt 
  werden.
  
  Dies funktioniert allerdings nur mit der \Environment*{tabular}"=Umgebung. 
  Die Tabellen aus den Paketen \Package{tabularx} und \Package{tabu} werden 
  nicht unterstützt. Wie dieser Abstand bei diesen manuell entfernt werden 
  kann, ist in einem Beispiel unter \autoref{sec:tips:table} zu finden.
\end{packages}

\subsubsection{Typographie und Layout}
\index{Typographie}
%
\begin{packages}
\item[setspace]\index{Zeilenabstand}
  Die Vergrößerung des Zeilenabstandes wird:
  %
  \begin{enumerate}[itemindent=0pt,labelwidth=*,labelsep=1em,label=\Roman*.]
  \itempackages viel zu häufig und völlig unnötig gefordert und
  \itempackages schließlich auch noch zu groß gewählt.
  \end{enumerate}
  %
  Die Forderung nach Erhöhung des Zeilenabstandes~-- in der Typographie als 
  Durchschuss bezeichnet~-- kommt noch aus den Zeiten der Textverarbeitung mit 
  der Schreibmaschine. Ein einzeiliger Zeilenabstand bedeutete hier, dass die 
  Unterlängen der oberen Zeile genau auf der Höhe der Oberlängen der folgenden 
  Zeile lagen. Ein anderthalbzeiliger Zeilenabstand erzielte hier somit einen 
  akzeptablen Durchschuss. Eine Erhöhung des Durchschusses bei der Verwendung 
  von \hologo{LaTeX} ist an und für sich nicht notwendig. Sinnvoll ist dies 
  nur, wenn im Fließtext serifenlose Schriften zum Einsatz kommen, um die damit 
  verbundene schlechte Lesbarkeit etwas zu verbessern.
  
  Ist die Erhöhung des Durchschusses wirklich notwendig, sollte das Paket 
  \Package{setspace} verwendet werden. Dieses stellt den Befehl 
  \Macro*{setstretch}\Parameter{Faktor} zur Verfügung, mit dem der Durchschuss 
  beziehungsweise Zeilenabstand angepasst werden kann. Der Wert des Faktors 
  ist standardmäßig auf~1 gestellt und sollte maximal bis~1.25 vergrößert 
  werden. Der Befehl \Macro*{onehalfspacing} aus diesem Paket setzt diesen Wert 
  auf eben genau~1.25. Allerdings ist hier anzumerken, dass die Vergrößerung 
  des Zeilenabstandes~-- so wie ich es mir angelesen habe~-- aus der Sicht 
  eines Typographen keine Spielerei ist sondern vielmehr allein der Lesbarkeit 
  des Textes dient und möglichst gering ausfallen sollte.
  
  Ziel ist es, beim Lesen nach dem Beenden der aktuellen Zeile das Auffinden 
  der neuen Zeile zu vereinfachen. Bei Serifen ist dies durch die Betonung der 
  Grundlinie sehr gut möglich. Bei serifenlosen Schriften~-- wie der im \CD der 
  \TnUD verwendeten \Univers~-- ist dies schwieriger und ein erweiterter 
  Abstand der   Zeilen kann dabei durchaus hilfreich sein. Jedoch sollte nicht 
  nach dem Motto \enquote{viel hilft viel} verfahren werden. In diesem Dokument 
  wurde als Faktor für den Zeilenabstand \Macro*{setstretch}\PParameter{1.1} 
  gewählt. Nach einer Einstellung des Zeilenabstandes sollte der Satzspiegel 
  unbedingt mit \Macro{recalctypearea} neu berechnet werden. Siehe dazu auch 
  \autoref{sec:tips:headings} sowie \autoref{sec:tips:headline}.
\item[csquotes]\index{Zitate}
  Das Paket stellt unter anderem den Befehl \Macro{enquote}\Parameter{Zitat} 
  zur Verfügung, welcher Anführungszeichen in Abhängigkeit der gewählten 
  Sprache setzt. Außerdem werden weitere Kommandos und Optionen für die 
  spezifischen Anforderungen des Zitierens bei wissenschaftlichen Arbeiten 
  angeboten. Außerdem wird es durch \Package{biblatex} unterstützt und sollte 
  zumindest bei dessen Verwendung geladen werden.
\item[xspace]\index{Befehle!Deklaration}
  Mit \Package{xspace} kann bei der Definition eigener Makros der Befehl 
  \Macro*{xspace} genutzt werden. Dieser setzt ein gegebenenfalls notwendiges 
  Leerzeichen automatisch. In \autoref{sec:tips:xspace} ist die Definition 
  eines solchen Befehls exemplarisch ausgeführt.
\item[xpunctuate]\index{Befehle!Deklaration}
  Die Funktionalität von \Package{xspace} wird um die Beachtung von 
  Interpunktionen erweitert.
\item[ellipsis]\index{Befehle!Deklaration}
  In \hologo{LaTeX} folgten den Befehlen für Auslassungspunkte (\Macro*{dots} 
  und \Macro*{textellipsis}) \emph{immer} ein Leerzeichen. Dies kann unter 
  Umständen unerwünscht sein. Mit dem Paket \Package{ellipsis} wird das 
  folgende Leerzeichen~-- im Gegensatz zum Standardverhalten~-- nur gesetzt, 
  wenn ein Satzzeichen und kein Buchstabe folgt.
\itempackages[\href{http://www.ctan.org/pkg/delig}{\Application{DeLig}}]
  \index{Typographie}\index{Ligaturen}
  Hierbei handelt es sich um ein Java-Script, welches anhand eines Wörterbuches 
  falsche Ligaturen innerhalb eines Dokumentes automatisiert entfernt. Wird 
  \Univers verwendet ist dies jedoch nicht notwendig, da diese keinerlei 
  Ligaturen enthält, die insbesondere in deutschen Texten für einen guten Satz 
  manuell aufgelöst werden müssten.%
  \footnote{%
    Das sind ff, fi, fl, ffi, und ffl bei den \hologo{LaTeX}"=Standardschriften.
  }
  Mit \hologo{LuaLaTeX} als Dokumentprozessor kann alternativ dazu auch 
  \Package{selnolig} verwendet werden.
\item[noindentafter]
  \ChangedAt{v2.02!\Package{noindentafter}: Beschreibung hinzugefügt}
  Mit diesem Paket lassen sich automatische Absatzeinzüge für selbst zu 
  bestimmende Befehle und Umgebungen unterdrücken.
\item[balance]\index{Zweispaltensatz}
  Dieses Paket ermöglicht einen Spaltenausgleich im Zweispaltensatz auf der 
  letzten Dokumentseite. Alternativ dazu kann auch \Package{multicol} verwendet 
  werden.
\end{packages}

\subsubsection{Schriften, Sonderzeichen und Rechtschreibung}
%
\begin{packages}
\item[lmodern](lm)\index{Schriftart}
  Soll mit den klassischen \hologo{LaTeX}"=Standardschriften gearbeitet werden, 
  empfiehlt sich die Verwendung des Paketes \Package{lmodern}. Dieses 
  verbessert die Darstellung der Computer~Modern sowohl am Bildschirm als auch 
  beim finalen Druck.
\item[cfr-lm]\index{Schriftart}
  Dieses experimentelle Paket liefert weitere Schriftschnitte für das Paket 
  \Package{lmodern}.
\item[libertine]\index{Schriftart}
  Das Paket stellt die Schriften Linux~Libertine und Linux~Biolinum zur 
  Verfügung.
  %
  \begin{packages}
    \item[libgreek]
      Es werden griechische Buchstaben für Linux~Libertine bereitgestellt.
    \item[newtxmath](newtx)
      Das Paket aus dem \Package*{newtx}-Bundle erlaubt die Verwendung der 
      Linux~Libertine im Mathematikmodus. Es wird mit
      \Macro*{usepackage}\POParameter{libertine}\PParameter{newtxmath} geladen.
  \end{packages}
  %
\item[mweights]\index{Schrift!Stärke}
  Werden Schriften aus unterschiedlichen Paketen verwendet, kann es unter 
  Umständen zu Problemen bei den Schriftstärken der Schriften kommen. 
  Normalerweise gibt es bei \hologo{LaTeXe} die Schriftfamilien für 
  Serifenschriften (\Macro*{rmfamily}), serifenlose Schrift (\Macro*{sffamily}) 
  sowie die Schreibmaschinenschriften (\Macro*{ttfamily}). Die Schriftstärke 
  dieser drei Schriftfamilien wird für gewöhnlich einheitlich über die Beiden 
  Befehle \Macro*{mddefault} und \Macro*{bfdefault} festgelegt. Mit dem Paket 
  \Package{mweights} kann die Schriftstärke für jede der drei Schriftfamilien 
  individuell festgelegt werden.
\item[relsize]\index{Schrift!Größe}
  Mithilfe dieses Paketes kann die Größe einer Textauszeichnung relativ zur 
  aktuell gewählten Schriftgröße gesetzt werden.
\item[textcomp]\index{Sonderzeichen}
  Es werden verschiedene zusätzliche Symbole wie beispielsweise das Promille- 
  oder Eurozeichen sowie Pfeile im Text zur Verfügung gestellt.
\item[fontspec]
  \ChangedAt{v2.02!\Package{fontspec}: %
    Beschreibung hinzugefügt{,} mit \Option{fontspec} nutzbar%
  }%
  Wird als Dokumentprozessor nicht \hologo{pdfLaTeX} sondern \hologo{XeLaTeX} 
  oder \hologo{LuaLaTeX} verwendet, können mit dem Paket \Package{fontspec} 
  auch Schriften im OpenType-Format eingebunden werden, womit sich die Auswahl 
  der verwendbaren Schriften in einem \hologo{LaTeX}"=Dokument stark erweitert. 
  Für die Verwendung von OpenType"=Schriften müssen diese lediglich für das 
  Betriebssystem jedoch nicht speziell für \hologo{LaTeX} installiert sein.
\item[spelling]\index{Rechtschreibung}
  Wird \hologo{LuaLaTeX} als Prozessor verwendet, wird mit diesem Paket der 
  reine Textanteil aus dem \hologo{LaTeX}"~Dokument extrahiert~-- wobei Makros 
  und aktive Zeichen entfernt werden~-- und in eine separate Textdatei 
  geschrieben. Anschließend kann diese Datei mit einer externen Software zur 
  Rechtschreibprüfung wie \Application{GNU Aspell} oder \Application{Hunspell} 
  analysiert werden. Wird durch dieses Programm eine Liste falsch geschriebener 
  Wörter ausgegeben, können diese mit \Package{spelling} im PDF"~Dokument 
  hervorgehoben werden.
\item[lua-check-hyphen]\index{Worttrennung}
  Mit diesem Paket lassen sich bei der Verwendung \hologo{LuaLaTeX} 
  Trennstellen am Zeilenende zur Prüfung markieren.
\end{packages}

\subsubsection{Mathematiksatz}
\index{Mathematiksatz}
Dies sind Pakete, die Umgebungen und Befehle für den Mathematiksatz sowie das 
Setzen von Einheiten und Zahlen im Allgemeinen anbieten.

\begin{packages}
  \item[mathtools]
    Dieses Paket stellt für das De-facto-Standard-Paket \Package{amsmath} für 
    Mathematikumgebungen Bugfixes zur Verfügung und erweitert dieses.
  \item[sansmath]
    Sollten die normalen \hologo{LaTeX}-Schriften Computer~Modern verwendet 
    werden, kann man dieses Paket zum serifenlosen Setzen mathematischer 
    Ausdrücke nutzen. Für die \TUDScript-Hauptklassen sei hierzu auf die Option
    \Option{sansmath} verwiesen.
  \item[sfmath]
    Diese Paket verfolgt ein ähnliches Ziel, kann jedoch im Gegensatz zu 
    \Package{sansmath} nicht nur für Computer~Modern sondern mit der 
    entsprechenden Option auch für Latin~Modern, Helvetica und 
    Computer~Modern~Bright verwendet werden.
  \item[mathastext]
     Mit dem Paket wird das Ziel verfolgt, aus der genutzten Schrift für den 
     Fließtext alle notwendigen Zeichen für den Mathematiksatz zu extrahieren.
  \item[bm]
    Das Paket bietet mit \Macro*{bm} eine Alternative zu \Macro*{boldsymbol} im 
    \hrfn{http://tex.stackexchange.com/questions/3238}{Mathematiksatz}.
\end{packages}
%
Die korrekte Formatierung von Zahlen ist häufig ein Problem bei der Verwendung 
von \hologo{LaTeX}. Insbesondere, wenn in einem deutschsprachigen Dokument 
Daten im typischen englischsprachigen Format verwendet werden, kommt es zu 
Problemen. Dafür wird im \TUDScript-Bundle das Paket \Package{mathswap} 
bereitgestellt. Dennoch gibt es zu diesem auch Alternativen.
%
\begin{packages}\index{Trennzeichen}
  \item[icomma]
    Wird im Mathematikmodus nach dem Komma ein Leerzeichen gesetzt, wird dies 
    bei der Ausgabe beachtet. Der Verfasser muss sich demzufolge jederzeit 
    selbst um die typographisch korrekte Ausgabe kümmern.
  \item[ziffer]
    Für deutschsprachige Dokumente wird das Komma als Dezimaltrennzeichen 
    zwischen zwei Ziffern definiert. Folgt dem Komma keine Ziffer, wird 
    jederzeit der obligatorische Freiraum gesetzt, was meiner Meinung nach 
    besser als das Verhalten von \Package{icomma} ist.
  \item[ionumbers]
    Dieses Paket ist mir tatsächlich erst bei der Arbeit an \Package{mathswap} 
    bekannt geworden. Es bietet mehr Funktionalitäten und kann als Alternative 
    dazu betrachtet werden.
\end{packages}
%
Für das typographisch korrekte Setzen von Einheiten~-- ein halbes Leerzeichen 
zwischen Zahl und \emph{aufrecht} gesetzter Einheit~-- gibt es zwei gut 
nutzbare Pakete.
%
\begin{packages}\index{Einheiten}
\item[units]
  Dies ist ein einfaches und sehr zweckdienliches Paket zum Setzen von 
  Einheiten und für die meisten Anforderungen völlig ausreichend.
\item[siunitx]
  Dieses Paket ist in seinem Umfang im Vergleich deutlich erweitert. Mir hat 
  sich persönlich noch nicht erschlossen, was genau die daraus resultierenden 
  Vorteile sind. Damit das Paket in deutschsprachigen Dokumenten alle Ziffern 
  richtig setzt, muss zumindest die Lokalisierung angegeben werden. Mehr dazu 
  in \autoref{sec:tips:siunitx}.
\end{packages}
%
Weitere Hinweise und Anwendungsfälle zur mathematischen Typographie werden in 
\autoref{sec:exmpl:mathtype} sowie \autoref{sec:exmpl:mathswap} gegeben.

\subsubsection{Die kleinen und großen Helfer\dots}
Hier taucht alles auf, was sich nicht eignete, in die vorherigen Kategorien 
eingeordnet zu werden.
%
\begin{packages}
\item[xparse]\index{Befehle!Deklaration}
  Dieses mächtige Paket entstammt dem \hologo{LaTeX3}-Projekt und bietet für 
  die Erstellung eigener Befehle und Umgebungen einen alternativen Ansatz zu 
  den bekannten \hologo{LaTeX}"=Deklarationsbefehlen \Macro*{newcommand} und 
  \Macro*{newenvironment} sowie deren Derivaten. Mit \Package{xparse} wird es 
  möglich, obligatorische und optionale Argumente an beliebigen Stellen 
  innerhalb des Befehlskonstruktes zu definieren. Auch die Verwendung anderer 
  Zeichen als eckige Klammern für die Spezifizierung eines optionalen 
  Argumentes ist möglich.
\item[xkeyval]\index{Befehle!Deklaration}
  \ChangedAt{v2.02!\Package{xkeyval}: Beschreibung hinzugefügt}
  Das \KOMAScript{}"=Bundle lädt das Paket \Package{keyval}, um Optionen mit 
  einer Schlüssel"=Wert"=Syntax deklarieren zu können. Zusätzlich wird von 
  \TUDScript das Paket \Package{kvsetkeys} geladen, um auf nicht definierte 
  Schlüssel reagieren zu können. Die Schlüssel"=Wert"=Syntax kann auch für 
  eigens definierte Makros genutzt werden, um sich das exzessive Verwenden von 
  optionalen Argumenten zu ersparen. Damit wäre folgende Definition möglich:
  \Macro*{newcommand}\Macro*{Befehl}\OParameter{Schlüssel"=Wert"=Liste}%
  \Parameter{Argument}.
  
  Das Paket \Package{xkeyval} erweitert insbesondere die Möglichkeiten zur 
  Deklaration unterschiedlicher Typen von Schlüsseln. Sollten die bereits durch 
  \TUDScript geladenen Pakete \Package{keyval} und \Package{kvsetkeys} in ihrer 
  Funktionalität nicht ausreichen, kann dieses Paket verwendet werden. Für die 
  Entwicklung eigener Pakete, deren Optionen das Schlüssel"=Wert"=Format 
  unterstützen, kann das Paket \Package{scrbase} genutzt werden. Soll aus einem 
  Grund auf \KOMAScript{} gänzlich verzichtet werden, sind die beiden Pakete 
  \Package{kvoptions} oder \Package{pgfkeys} eine Alternative.
\item[calc]\index{Berechnungen}
  Normalerweise können Berechnungen nur mit Low-Level-\hologo{TeX}-Primitiven 
  im Dokument durchgeführt werden. Dieses Paket stellt eine einfachere Syntax 
  für Rechenoperationen der vier Grundrechenarten zur Verfügung. Zusätzlich 
  werden neue Befehle zur Bestimmung der Höhe und Breite bestimmter Textauszüge 
  definiert.
\item[bookmark]\index{Lesezeichen}\index{Querverweise}
  Dieses Paket verbessert und erweitert die von \Package{hyperref} angebotenen 
  Möglichkeiten zur Erstellung von Lesezeichen~-- auch Outline"=Einträge~-- im 
  PDF-Dokument. Beispielsweise können Schriftfarbe- und "~stil geändert werden.
\item[varioref]\index{Querverweise}
  Mit diesem Paket lassen sich sehr gute Verweise auf bestimmte Seiten 
  erzeugen. Insbesondere, wenn der Querverweis auf die aktuelle, die 
  vorhergehende oder nachfolgende sowie im zweiseitigen Satz auf die 
  gegenüberliegende Seite erfolgt, werden passende Textbausteine für diesen 
  verwendet.
\item[listings]\index{Quelltexte einbinden}\index{Quelltextverzeichnis}%
  Dieses Paket eignet sich hervorragend zur Quelltextdokumentation in 
  \hologo{LaTeX}. Es bietet die Möglichkeit, externe Quelldateien einzulesen 
  und darzustellen sowie die Syntax in Abhängigkeit der verwendeten 
  Programmiersprache hervorzuheben. Zusätzlich lässt sich ein Verzeichnis mit 
  allen eingebundenen sowie direkt im Dokument angegebenen Quelltextauszügen 
  erstellen.
  
  \ChangedAt{v2.02!\Package{listings}: Beschreibung ergänzt}
  Wird \Package{listings} in Dokumenten mit UTF"~8-Kodierung verwendet, sollte 
  direkt nach dem Laden des Paketes in der Präambel Folgendes hinzugefügt 
  werden:
  \begin{Code}
  \lstset{%
    inputencoding=utf8,extendedchars=true,
    literate=%
      {ä}{{\"a}}1 {ö}{{\"o}}1 {ü}{{\"u}}1
      {Ä}{{\"A}}1 {Ö}{{\"O}}1 {Ü}{{\"U}}1
      {~}{{\textasciitilde}}1 {ß}{{\ss}}1
  }
  \end{Code}\vspace{-\baselineskip}%
\item[chngcntr]\index{Zählermanipulation}
  Das Paket erlaubt die Manipulation aller möglichen, bereits definierten 
  \hologo{LaTeX}-Zähler. Es können Zähler so umdefiniert werden, dass sie bei 
  der Änderung eines anderen Zählers automatisch zurückgesetzt werden oder eben 
  nicht. Ein kleines Beispiel dazu ist in \autoref{sec:tips:counter} zu finden.
\item[marginnote]\index{Randnotizen}
  Randnotizen, welche mit \Macro*{marginpar} erzeugt werden, sind spezielle 
  Gleitobjekte in \hologo{LaTeX}. Dies kann dazu führen, dass eine Notiz am 
  Blattrand nicht direkt da gesetzt wird, wo diese intendiert war. Dieses Paket 
  stellt den Befehl \Macro*{marginnote} für nicht"~gleitende Randnotizen zur 
  Verfügung. Alternativ dazu kann man auch \Package{mparhack} verwenden.
\item[todonotes]\index{Randnotizen}
  Mit \Package{todonotes} können noch offene Aufgaben in unterschiedlicher 
  Formatierung am Blattrand oder im direkt Fließtext ausgegeben werden. Aus 
  allen Anmerkungen lässt sich eine Liste aller offenen Punkte erzeugen.
\item[filemod]
  Wird entweder \hologo{pdfLaTeX} oder \hologo{LuaLaTeX} als Prozessor 
  eingesetzt, können mit diesem Paket das Änderungsdatum zweier Dateien 
  miteinander verglichen und in Abhängigkeit davon definierbare Aktionen 
  ausgeführt werden.
\item[mwe]\index{Minimalbeispiel|!}
  \ChangedAt{v2.02!\Package{mwe}: Beschreibung hinzugefügt}
  Hiermit lassen sich sehr einfach Minimalbeispiele mit Abbildungen erzeugen.
\item[coseoul]
  Mit diesem Paket kann man die Struktur der Gliederung relativ angeben. Es 
  wird keine absolute Gliederungsebene (\Macro*{chapter}, \Macro*{section}) 
  angegeben sondern die Relation zwischen vorheriger und aktueller Ebene 
  (\Macro*{levelup}, \Macro*{levelstay}, \Macro*{leveldown}).
\item[dprogress]\index{Debugging}
  Das Paket schreibt bei der Kompilierung des Dokumentes die Gliederung in die 
  Logdatei. Dies kann im Fehlerfall beim Auffinden des Problems im Dokument 
  helfen. Allerdings werden dafür die Gliederungsebenen so umdefiniert, dass 
  diese keine optionalen Argumente mehr unterstützen,was jedoch für die 
  \TUDScript-Klassen von essentieller Bedeutung ist. Zum Debuggen kann es 
  trotzdem sporadisch eingesetzt werden.
\end{packages}

\subsubsection{Bugfixes}
%
\begin{packages}
\item[scrhack](koma-script)
  Das Paket behebt Kompatibilitätsprobleme der \KOMAScript-Klassen mit den 
  Paketen \Package{hyperref}, \Package{float}, \Package{floatrow} und
  \Package{listings}. Es ist durchaus empfehlenswert, jedoch sollte man 
  unbedingt die Dokumentation beachten.
\item[fixltx2e]
  Dieses Paket enthält Bugfixes für \hologo{LaTeXe}. Da diese eventuell zu 
  Inkompatibilitäten mit früheren Versionen führen könnten, wurden diese nicht 
  in den \hologo{LaTeXe}-Kernel eingepflegt.
\item[mparhack]
  Zur Behebung falsch gesetzter Randnotizen wird ein Bugfix für 
  \Macro*{marginpar} bereitgestellt. Alternativ dazu kann man auch 
  \Package{marginnote} verwenden.
\item[etex]
  Das Paket kann genutzt werden, falls die standardmäßig maximale Anzahl der 
  \hologo{LaTeX}-Register für Längen, Zähler etc. überschritten wurde.
\end{packages}
