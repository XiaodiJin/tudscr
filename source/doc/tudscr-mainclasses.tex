\chapter[Die Klassen tudscrbook, tudscrreprt und tudscrartcl]{Die Hauptklassen}
\label{sec:mainclasses}
\ChangedAt*{%
  v2.00!Robustheit vieler Befehle und Optionen erhöht,%
  v2.02!Umbenennung einiger Befehle für Kompatibilität mit anderen Paketen,%
  v2.03!Anpassungen interner Befehle an \KOMAScript-Version~v3.15%
}
\begin{Declaration*}{\Class{tudscrbook}}
\begin{Declaration*}{\Class{tudscrreprt}}
\begin{Declaration*}{\Class{tudscrartcl}}
\index{Hauptklassen|!}
Es werden die drei neuen Hauptklassen
%
\begin{description}
\item \Class{tudscrbook}
\item \Class{tudscrreprt}
\item \Class{tudscrartcl}
\end{description}
%
eingeführt, welche auf den \KOMAScript-Klassen basieren und grundsätzlich alle
deren Optionen, Umgebungen und Befehle~-- beispielsweise \Option{parskip} für 
die Absatzeinstellungen oder \Option{BCOR} zur Festlegung der Bindekorrektur~-- 
unterstützen. Zusätzlich zu den \KOMAScript"=Klassen werden weitere Pakete 
zwingend benötigt, welche unter \autoref{sec:packages:needed} aufgeführt sind 
und auf jeden Fall durch \TUDScript geladen werden.

Es sei hier abermals auf die Anwenderdokumentation \scrguide von \KOMAScript{} 
hingewiesen, viele der folgend beschriebenen Befehle und Optionen beziehen sich 
auf die darin vorgestellten Einstellungsmöglichkeiten. Die Anpassungen und 
Erweiterungen der \KOMAScript"=Klassen an das \CD und die neu definierten 
beziehungsweise geänderten Befehle und Optionen werden im Folgenden erläutert.
\end{Declaration*}
\end{Declaration*}
\end{Declaration*}

\begin{Declaration}{\Macro{TUDoptions}\Parameter{Optionenliste}}
\begin{Declaration}{\Macro{TUDoption}\Parameter{Option}\Parameter{Werteliste}}
\printdeclarationlist%
\index{Optionen|!}%
%
Mit diesen Befehlen hat man bei den meisten der neuen Klassenoptionen die 
Möglichkeit, den Wert der Optionen noch nach dem Laden der Klasse zu ändern.
Man kann wahlweise mit der Anweisung \Macro{TUDoptions} die Werte einer Reihe 
von Optionen ändern. Jede Option der Optionenliste hat dabei die Form
\PName{Option}\PValue{=}\PName{Wert}. Die meisten Optionen besitzen auch einen 
Säumniswert\footnote{engl.: default value}. Versäumt man die Angabe eines 
Wertes~-- verwendet demzufolge einfach die Form \PName{Option}~-- so wird 
automatisch dieser Säumniswert angenommen.

Manche Optionen können gleichzeitig mehrere Werte besitzen. Für diese besteht 
die Möglichkeit, mit \Macro{TUDoption} der einen Option nacheinander eine 
Reihe von Werten zuzuweisen. Die einzelnen Werte sind dabei in der Werteliste 
durch Komma voneinander getrennt.

Mit diesen beiden Befehlen kann im Bedarfsfall das Verhalten von einer Option 
oder mehreren Optionen im Dokument geändert werden. Werden diese Befehle in 
einer Umgebung oder einer Gruppe verwendet, bleiben die gemachten Einstellungen 
innerhalb dieser lokal begrenzt.
\end{Declaration}
\end{Declaration}



\section{Die Schriften des \CDs}
\label{sec:fonts}
\index{Schrift|?(}
%
\ChangedAt*{%
  v2.00!Schriften~-- insbesondere für den mathematischen Satz~-- verbessert,%
  v2.01!Unterschneidung (Kerning) der Ziffern für \Univers verbessert%
}
Das \CD der \TnUD gibt die Verwendung der Schriften \Univers für den Fließtext 
sowie \DIN für das Setzen von Überschriften vor, was durch \TUDScript in der 
Standardkonfiguration auch so umgesetzt wird. Da jedoch in längeren Texten die 
Verwendung von Serifenschriften zu empfehlen ist, gibt es die Möglichkeit, die 
ursprünglich vorgesehenen Schriften des \CDs nicht zu laden und stattdessen die 
\hologo{LaTeX}-Standardschriften beziehungsweise ein anderes Schriftpaket zu 
verwenden. Die Erläuterungen dazu sind in \autoref{sec:text} zu finden.

Durch das \CD werden keine Schriften für den Mathematiksatz festgelegt. Das ist 
insbesondere für sowohl mathematische Abhandlungen als auch ingenieur- und 
naturwissenschaftliche Dokumente nicht tragbar. Dieser Mangel wird behoben, 
indem im Mathematikmodus die lateinischen Buchstaben der Hausschriften mit 
griechischen Lettern und mathematischen Symbolen aus anderen Paketen ergänzt 
werden.%
\footnote{%
  \Package{iwona} für die Schrift \DIN und zusätzlich \Package{cmbright} für 
  die \Univers"=Schriftfamilie%
}
Diese Grundeinstellung lässt sich ebenfalls deaktivieren, wodurch die 
Standardschriften oder gegebenenfalls die eines zusätzlich geladenen Paketes 
für den mathematischen Satz genutzt werden. Die dafür relevanten Einstellungen 
werden in \autoref{sec:math} erläutert. In \autoref{sec:exmpl:mathtype} sowie 
\autoref{sec:exmpl:mathswap} sind zusätzliche Hinweise zum typografisch guten 
Mathematiksatz zu finden.


\subsection{Schriften für den Textsatz}
\begin{Declaration}[%
  v2.02!Werte für Option \Option{cdhead} ergänzt%
]{\Option{cdfont}[\PSet]}[true]%
\printdeclarationlist%
\label{sec:text}%
\index{Schrift!Fließtext}\index{Schriftstärke}%
%
Mit der Option \Option{cdfont} können durch den Anwender alle zentralen 
Schrifteinstellungen für die \TUDScript-Klassen vorgenommen werden. Dies 
betrifft sowohl die Schriften für Überschriften als auch den Fließtext sowie 
die Mathematikschriften. Die verwendete Schriftstärke im charakteristischen 
Querbalken der Kopfzeile lässt sich hiermit ebenfalls einstellen.
%
\begin{values}
\itemfalse
  Es werden die \hologo{LaTeX}"=Standardschriften und nicht die Hausschriften 
  des \CDs verwendet. Der Anwender kann beliebige Schriftpakete nutzen.%
  \footnote{%
    Für die Verwendung der klassischen \hologo{LaTeX}"=Schriften, ist das Paket 
    \Package{lmodern} sehr empfehlenswert.%
  }
  Sollte das Layout des \CDs aktiviert sein (\see*{\Option{cd}}), werden die 
  Überschriften in Großbuchstaben und \DIN gesetzt und nur die Schriftart des 
  Fließtextes kann angepasst werden.
\itemtrue*[light/lightfont/noheavyfont]
  Es werden die Hausschriften im Stil des \CDs der \TnUD genutzt. Überschriften 
  der obersten Gliederungsebenen bis einschließlich \Macro{subsubsection} 
  verwenden \DIN, darunter liegende%
  \footnote{\Macro{paragraph} und \Macro{subparagraph}} 
  \textubn{Univers~65~Bold}. Für den Fließtext im Dokument kommt 
  \textuln{Univers~45~Light} zum Einsatz. Aus \Package{lmodern} wird die
  \texttt{Schreibmaschinenschrift} verwendet.
\item[heavy/heavyfont]
  Die Schriftstärke der Hausschriften wird erhöht. Die beiden untersten 
  Gliederungsebenen werden in \textuxn{Univers~75~Black} gesetzt, der Fließtext 
  in \texturn{Univers~55~Regular}. Ansonsten entspricht alles der Option 
  \Option{cdfont}[true]. Die Mathematikschriften werden nicht beeinflusst, 
  diese lassen sich gegebenenfalls mit \Macro{boldmath} auf den fetten Schnitt 
  umschalten.
\item[nodin]
  Für die Überschriften wird \DIN nicht verwendet. Für \Option{cdfont}[true] 
  wird \Univers genutzt. Die Schriftstärke ist dabei abhängig von der   
  Einstellung \Option{cdfont}[light/heavy]. Ist die Verwendung der Schriften 
  des \CDs deaktiviert (\Option{cdfont}[false]), kommt die fette Schriftstärke 
  der eingestellten serifenlosen Schriftfamilie zum Einsatz.
\item[din]
  Mit dieser Einstellung wird die Schrift \DIN in den Überschriften verwendet. 
  Diese ist standardmäßig aktiviert.
\end{values}
%
\ChangedAt{v2.02}
Für den Text im Querbalken gibt es folgende Einstellmöglichkeiten:
%
\begin{values}
\item[head/lighthead/lightfonthead/noheavyfonthead]
  Für den Querbalken der Kopfzeile wird unabhängig von der Verwendung der 
  Hausschriften die Schrift \Univers in normaler Schriftstärke verwendet,
  \see*{\Option{cdhead}[true]}.
\item[heavyhead/heavyfonthead]
  Die im Querbalken verwendete Schrift ist \Univers in erhöhter Stärke, 
  \see*{\Option{cdhead}[heavy]}.
\end{values}
%
Die verwendeten Mathematikschriften lassen sich mit folgenden Werte 
beeinflussen:
%
\begin{values}
\item[nomath/nocdmath]  
  Diese Einstellung deaktiviert die Verwendung von serifenlosen Schriften für 
  den mathematischen Satz. Es werden die \hologo{LaTeX}"=Standardschriften 
  verwendet und der Benutzer kann beliebige Schriftpakete für den 
  Mathematikmodus nutzen, \see*{\Option{cdmath}[false]}.
\item[math/cdmath]
  Es werden serifenlose Mathematikschriften für lateinische und griechische 
  Lettern genutzt, \see*{\Option{cdmath}[true]}.
\item[upgreek/uprightgreek]
  Die großen griechischen Buchstaben werden im Mathematikmodus aufrecht gesetzt,
  \see*{\Option{slantedgreek}[false]}.
\item[slgreek/slantedgreek]
  In mathematischen Umgebungen erfolgt die Ausgabe der griechischen Majuskeln 
  kursiv, \see*{\Option{slantedgreek}[true]}.
\end{values}
%
\ChangedAt{v2.02,v2.04!Einfachere Verwendung von \Package{fontspec}}
Normalerweise kommen die Schriften des \CDs im PostScript"=Format zum Einsatz, 
wenn diese wie unter \autoref{sec:install} beschrieben installiert wurden.
Wird entweder \hologo{LuaLaTeX} oder \hologo{XeLaTeX} als Dokumentprozessor 
verwendet und das Paket \Package{fontspec} geladen, so werden die Schriften des 
\CDs im OpenType"=Format verwendet. Hierfür sind die Hinweise in 
\fullref{sec:fonts:fontspec} unbedingt zu beachten.
\end{Declaration}


\subsubsection{Auszeichnungen in Überschriften}
\index{Schriftelemente}
Für die Schriftauswahl der Überschriften aller Gliederungsebenen sind die durch 
\KOMAScript{} bereitgestellten Schriftelemente verantwortlich. Mehr dazu ist in 
\autoref{sec:fonts:elements} zu finden. Da die Überschriften der obersten 
Gliederungsebenen bis einschließlich \Macro{subsubsection} normalerweise in 
Majuskeln gesetzt werden, bestehen für den Anwender mit den folgenden Befehlen 
gewisse Einflussmöglichkeiten, deren Ausprägung anzupassen.

\begin{Declaration}{\Macro{ifdin}\Parameter{Dann-Teil}\Parameter{Sonst-Teil}}%
\printdeclarationlist%
\index{Überschriften}\index{Schrift!Überschriften}\index{Schriftauszeichnung}%
\index{Kolumnentitel}\index{Layout!Kolumnentitel}
%
Der Befehl \Macro{ifdin} prüft, ob die Schriftfamilie \DIN aktiv ist und führt 
in diesem Fall \Parameter{Dann-Teil} aus, andernfalls \Parameter{Sonst-Teil}. 
Dies ist beispielsweise bei Überschriften sinnvoll, wenn innerhalb des 
obligatorischen Argumentes zwischen der Ausgabe im Dokument selber und dem 
Eintrag für das Inhaltsverzeichnis sowie der Ausprägung der automatischen 
Kolumnentitel unterschieden werden soll.
\end{Declaration}

\begin{Declaration}{\Macro{MakeTextUppercase}\Parameter{Text}}%
\begin{Declaration}{\Macro{NoCaseChange}\Parameter{Text}}%
\printdeclarationlist%
\index{Überschriften}\index{Schrift!Überschriften}\index{Schriftauszeichnung}%
%
Der Befehl \Macro{MakeTextUppercase} stammt aus dem Paket \Package{textcase} 
und setzt den Text seines Argumentes in Majuskeln. Die Überschriften der 
Gliederungsebenen bis einschließlich \Macro{subsubsection} werden damit in 
Großbuchstaben der Schrift \DIN gesetzt, wenn das Layout des \CDs nicht 
deaktiviert wurde (\Option{cd}[false]). Sollen in einer Überschrift bestimmte 
Kleinbuchstaben erhalten bleiben, ist der Befehl \Macro{NoCaseChange} zu 
nutzen, welcher ebenfalls von besagtem Paket bereitgestellt wird.
\end{Declaration}
\end{Declaration}
%
\begin{Example}
In einer Kapitelüberschrift wird ein einzelnes Wort in Kleinbuchstaben 
geschrieben:
\begin{Code}[escapechar=§]
\chapter{§Ü§berschrift mit \NoCaseChange{kleinem} Wort}
\end{Code}
\end{Example}

\subsubsection{Auszeichnungen im Text}
\index{Schrift!Fließtext}\index{Schriftstärke}%
\index{Schrift!Befehle}\index{Schrift!Schalter}%
%
Unabhängig davon, welche Schriftfamilie verwendet wird, können die Schriften 
des \CDs jederzeit entweder mit einem der hier aufgeführten Textschalter oder 
Textkommandos innerhalb des Dokumentes genutzt werden. Ein Textschalter wirkt 
sich~-- wenn er nicht in einer Gruppe oder einer Umgebung verwendet und damit 
lokal begrenzt wird~-- global auf das Dokument aus, wie etwa beispielsweise 
\Macro*{bfseries}. Bei einem Textkommando hingegen erfolgt die Änderung der 
Schriftart nur für das nachfolgend angegebene Argument, wie zum Beispiel bei
\Macro*{textbf}\Parameter{Text}. Darauf ist bei der Nutzung zu achten. 
%
\begin{Declaration}{\Macro{univln}}
\begin{Declaration}{\Macro{textuln}\Parameter{Text}}
\begin{Declaration}{\Macro{univrn}}
\begin{Declaration}{\Macro{texturn}\Parameter{Text}}
\begin{Declaration}{\Macro{univbn}}
\begin{Declaration}{\Macro{textubn}\Parameter{Text}}
\begin{Declaration}{\Macro{univxn}}
\begin{Declaration}{\Macro{textuxn}\Parameter{Text}}
\begin{Declaration}{\Macro{univls}}
\begin{Declaration}{\Macro{textuls}\Parameter{Text}}
\begin{Declaration}{\Macro{univrs}}
\begin{Declaration}{\Macro{texturs}\Parameter{Text}}
\begin{Declaration}{\Macro{univbs}}
\begin{Declaration}{\Macro{textubs}\Parameter{Text}}
\begin{Declaration}{\Macro{univxs}}
\begin{Declaration}{\Macro{textuxs}\Parameter{Text}}
\begin{Declaration}{\Macro{dinbn}}
\begin{Declaration}{\Macro{textdbn}\Parameter{Text}}
\settowidth\tempdim{\Macro{textuln}\Parameter{Text}}%
\addtolength\tempdim{\dimexpr 2\tabcolsep+2\arrayrulewidth-\textwidth}%
\printdeclarationlist(%
  \begin{minipage}{-\tempdim}%
  \centering%
  \begin{tabularm}{3}%
    \toprule%
    \textbf{Schriftart}                  & \textbf{Schalter}
      & \textbf{Textkommando}\tabularnewline
    \midrule
    \textuln{Univers 45 Light}           & \Macro{univln}{}
      & \Macro{textuln}\Parameter{Text}\tabularnewline
    \texturn{Univers 55 Regular}         & \Macro{univrn}{}
      & \Macro{texturn}\Parameter{Text}\tabularnewline
    \textubn{Univers 65 Bold}            & \Macro{univbn}{}
      & \Macro{textubn}\Parameter{Text}\tabularnewline
    \textuxn{Univers 75 Black}           & \Macro{univxn}{}
      & \Macro{textuxn}\Parameter{Text}\tabularnewline
    \textuls{Univers 45 Light Oblique}   & \Macro{univls}{}
      & \Macro{textuls}\Parameter{Text}\tabularnewline
    \texturs{Univers 55 Regular Oblique} & \Macro{univrs}{}
      & \Macro{texturs}\Parameter{Text}\tabularnewline
    \textubs{Univers 65 Bold Oblique}    & \Macro{univbs}{}
      & \Macro{textubs}\Parameter{Text}\tabularnewline
    \textuxs{Univers 75 Black Oblique}   & \Macro{univxs}{}
      & \Macro{textuxs}\Parameter{Text}\tabularnewline
    \DIN & \Macro{dinbn}{}
      & \Macro{textdbn}\Parameter{Text}\tabularnewline
    \bottomrule%
    \allcolumnpar{\footnotesize\vskip0pt%
       Die Schrift \DIN darf laut \CD nur mit Majuskeln (Großbuchstaben) 
       verwendet werden. Wird diese Schrift manuell verwendet, sollte dies mit 
       \Macro{MakeTextUppercase}\PParameter{\Macro{textdbn}\Parameter{Text}}  
       geschehen. Sollen dabei im Argument einzelne Teile zwingend klein 
       geschrieben werden, wird der Befehl \Macro{NoCaseChange} benötigt.
    }
  \end{tabularm}%
  \end{minipage}%
)%
Alternativ zu den beschriebenen Textschaltern und -kommandos können seit der 
Version~v2.04 auch die beiden Befehle \Macro{cdfont} und \Macro{textcdfont} 
verwendet werden, welche die gleiche Funktionalität wesentlich komfortabler 
bereitstellen.
\end{Declaration}
\end{Declaration}
\end{Declaration}
\end{Declaration}
\end{Declaration}
\end{Declaration}
\end{Declaration}
\end{Declaration}
\end{Declaration}
\end{Declaration}
\end{Declaration}
\end{Declaration}
\end{Declaration}
\end{Declaration}
\end{Declaration}
\end{Declaration}
\end{Declaration}
\end{Declaration}

\begin{Declaration}[v2.04]{\Macro{cdfont}\Parameter{Schriftart}}
\begin{Declaration}[v2.04]{%
  \Macro{textcdfont}\Parameter{Schriftart}\Parameter{Text}%
}
\printdeclarationlist
Diese beiden Befehle dienen ebenfalls zu gezielten Aktivierung einer Schriftart 
des \CDs in Stärke und Schnitt. Hierbei entspricht \Macro{cdfont} einem 
Textschalter und ändert die aktuell verwendete Schriftart im aktuellen 
Geltungsbereich auf \PName{Schriftart}, wohingegen \Macro{textcdfont} als 
Textkommando fungiert und den im zweiten Argument gegebenen \PName{Text} in 
\PName{Schriftart} setzt ohne dabei die Dokumentschriftart selbst zu ändern.

Für die Schriftauswahl muss im ersten Argument der Name der zu verwendenden 
Schriftart angegeben werden. Dieser ist der obigen Tabelle zu entnehmen. Für 
die Auswahl der Schriftfamilie \Univers kann der Vorsatz \PValue{Univers} im 
Argument \PName{Schriftart} entfallen. Ebenso sind weder Leerzeichen noch die 
passende Groß- und Kleinschreibung notwendig. Für die Wahl der Schriftstärke 
ist die entsprechende Zahl \emph{oder} die Bezeichnung allein ausreichend.%
\footnote{\PValue{45/55/65/75} oder \PValue{Light/Regular/Bold/Black}}
Anstelle des Suffix' \PValue{Oblique} ist auch die Nutzung von \PValue{Italic} 
oder \PValue{Slanted} als Alias für die geneigten Schriftschnitte möglich. Zur 
Auswahl von \DIN ist \PValue{din} als Argument hinreichend.

\end{Declaration}
\end{Declaration}


\subsection{Schriften für den Mathematiksatz}
\begin{Declaration}[v2.03]{\Option{cdmath}[\PBoolean]}%
  [true][\Option{cdfont}[false]:false]
\printdeclarationlist%
\label{sec:math}
\index{Schrift!Mathematiksatz}\index{Mathematiksatz|!}
\index{Schrift!Griechische Buchstaben}\index{Griechische Buchstaben}
%
Diese Option dient zur Anpassung der Mathematikschriften. Wird diese aktiviert, 
so werden zu den Hausschriften passende im Mathematikmodus genutzt, mit 
\Option{cdmath}[false] wird auf die Standardschriften zurückgeschaltet. Ein 
Umschalten innerhalb des Dokumentes ist~-- beispielsweise für Abbildungen oder 
Tabellen~-- durch \Macro{TUDoptions}\PParameter{\Option{cdmath}[true/false]} 
möglich. Mit \Macro{boldmath} kann auf fette Mathematikschriften umgeschaltet 
werden. Gültige Werte für die Option \Option{cdmath} sind:
%
\begin{values}
\itemfalse
  Es werden die normalen \hologo{LaTeX}"=Serifenschriften beziehungsweise die 
  Schriften beliebig nutzbarer Pakete für den Mathematiksatz verwendet.
\itemtrue*
  Im Mathematikmodus wird \Univers genutzt. Außerdem kommen die griechischen 
  Buchstaben aus dem Paket \Package{cmbright} sowie Symbole aus dem Paket 
  \Package{iwona} zum Einsatz.
\item[upgreek/uprightgreek]
  Die griechischen Majuskeln werden aufrecht gesetzt, 
  \see*{\Option{slantedgreek}[false]}.
\item[slgreek/slantedgreek]
  Die Ausgabe der griechischen Großbuchstaben erfolgt kursiv, 
  \see*{\Option{slantedgreek}[true]}.
\end{values}
\end{Declaration}

\subsubsection{Griechischen Buchstaben}
\vskip-\lastskip%
\label{sec:greek}%
\index{Schrift!Griechische Buchstaben}\index{Griechische Buchstaben}%
%
\begin{Declaration}{\Macro{varDelta}}
\begin{Declaration}{\Macro{varTheta}}
\begin{Declaration}{\Macro{varLambda}}
\begin{Declaration}{\Macro{varXi}}
\begin{Declaration}{\Macro{varPi}}
\begin{Declaration}{\Macro{varSigma}}
\begin{Declaration}{\Macro{varUpsilon}}
\begin{Declaration}{\Macro{varPhi}}
\begin{Declaration}{\Macro{varPsi}}
\begin{Declaration}{\Macro{varOmega}}
\begin{Declaration}{\Macro{upDelta}}
\begin{Declaration}{\Macro{upTheta}}
\begin{Declaration}{\Macro{upLambda}}
\begin{Declaration}{\Macro{upXi}}
\begin{Declaration}{\Macro{upPi}}
\begin{Declaration}{\Macro{upSigma}}
\begin{Declaration}{\Macro{upUpsilon}}
\begin{Declaration}{\Macro{upPhi}}
\begin{Declaration}{\Macro{upPsi}}
\begin{Declaration}{\Macro{upOmega}}
\index{Schrift!Griechische Buchstaben}\index{Griechische Buchstaben}%
\settowidth\tempdim{\Macro{varUpsilon}}%
\addtolength\tempdim{\dimexpr 2\tabcolsep+2\arrayrulewidth-\textwidth\relax}%
\printdeclarationlist(%
  \begin{minipage}{-\tempdim}%
    \newcommand\tablecontent{}%
    \newcommand*\greekLetters{%
      Delta,Theta,Lambda,Xi,Pi,Sigma,Upsilon,Phi,Psi,Omega%
    }%
    \def\do#1{\appto\tablecontent{%
      \Macro*{var#1} & $\csuse{var#1}$ & & 
      \Macro*{up#1} & $\csuse{up#1}$\tabularnewline
    }}%
    \expandafter\docsvlist\expandafter{\greekLetters}%
    \centering%
    \vspace{\intextsep}\noindent
    \begin{tabularm}{5}
      \toprule%
      \textbf{Befehl (kursiv)} & \textbf{Symbol} & &
      \textbf{Befehl (aufrecht)} & \textbf{Symbol}
      \tabularnewline\midrule\tablecontent\bottomrule%
      \allcolumnpar{\footnotesize\vskip0pt%
        Die Befehle \Macro*{up}\PName{Name} und \Macro*{var}\PName{Name}
        werden normalerweise durch einige Pakete, unter anderem auch von 
        \Package{cmbright} oder \Package{amsmath}, bereitgestellt.
      }
    \end{tabularm}
  \end{minipage}%
)%
%
Unabhängig von den beiden Optionen \Option{cdmath} und \Option{slantedgreek} 
können sowohl kursive als auch aufrechte griechischen Großbuchstaben im 
Mathematikmodus mit diesen Befehlen direkt verwendet werden. Dies ist nützlich, 
um zwischen kursiven Variablen und aufrechten Konstanten zu unterscheiden. Die 
griechischen Minuskeln sind leider nur in der kursiven Variante verfügbar.
\end{Declaration}
\end{Declaration}
\end{Declaration}
\end{Declaration}
\end{Declaration}
\end{Declaration}
\end{Declaration}
\end{Declaration}
\end{Declaration}
\end{Declaration}
\end{Declaration}
\end{Declaration}
\end{Declaration}
\end{Declaration}
\end{Declaration}
\end{Declaration}
\end{Declaration}
\end{Declaration}
\end{Declaration}
\end{Declaration}

\begin{Declaration}{\Option{slantedgreek}[\PBoolean]}[false]
\printdeclarationlist%
%
Die Option ändert die standardmäßige Neigung der griechischen Großbuchstaben im 
Mathematikmodus bei der Verwendung der Befehle \Macro*{Delta}, \Macro*{Theta}, 
\Macro*{Lambda}, \Macro*{Xi}, \Macro*{Pi}, \Macro*{Sigma}, \Macro*{Upsilon}, 
\Macro*{Phi}, \Macro*{Psi} und \Macro*{Omega}. Wie unabhängig von der Option 
\Option{slantedgreek} gezielt kursive und aufrechte Buchstaben gesetzt werden 
können, wird \vpageref{sec:greek} beschrieben.
%
\begin{values}
\itemfalse
  Die griechischen Majuskeln werden wie bei den Standardklassen aufrecht 
  gesetzt.
\itemtrue*
  Die Ausgabe der griechischen Großbuchstaben erfolgt kursiv.
\end{values}
\end{Declaration}


\subsubsection{Zusätzliche Hinweise zum Mathematiksatz}
Weitere Hinweise zum typografisch guten Mathematiksatz sind außerdem in 
\autoref{sec:exmpl:mathswap} sowie \autoref{sec:exmpl:mathtype} zu finden.


\subsection{Die Schriften des \CDs im OpenType-Format}
\label{sec:fonts:fontspec}
\index{OpenType-Schriften}
%
\ChangedAt{v2.02!OpenType-Schriften mit \Package{fontspec} verwendbar}
Das \TUDScript-Bundle unterstützt die Nutzung der Schriften des \CDs sowohl 
im PostScript- als auch im OpenType"=Format. Letztere müssen ebenfalls über das 
\hrfn{http://tu-dresden.de/service/publizieren/cd/1_basiselemente/03_hausschrift/schriftbestellung.html}%
{Universitätsmarketing auf Anfrage} bestellt werden. Die in den beiden Archiven 
\File{Univers\_8\_TTF.zip} und \File{DIN\_TTF.zip} enthaltenen Schriften müssen 
für das Betriebssystem installiert werden und lassen sich anschließend mit dem 
Paket \Package{fontspec} verwenden.

Auf die Installation der PostScript"=Schriften kann dennoch nicht ohne Weiteres 
verzichtet werden. Denn einerseits sind diese für die Kompilierung eines 
\hologo{LaTeX}"~Dokumentes über den klassischen Prozesspfad via
\Path{latex \textrightarrow{} dvips \textrightarrow{} ps2pdf}~-- wie es 
beispielsweise für die Erstellung von Grafiken mit \Package{pstricks} notwendig 
ist~-- nötig. Andererseits liefern die Schriftfamilien des \CDs keinerlei 
mathematische Glyphen, sodass diese bei der PostScript"=Schriftinstallation aus 
den Schriftpaketen \Package{cmbright} und \Package{iwona} entnommen werden. Bei 
der Nutzung der Schriften im OpenType"=Format ist dies leider nicht so einfach 
möglich, da es schlichtweg an passenden Schriftpaketen für den Mathematiksatz 
im OpenType"=Format mangelt. Weiterhin kommt es auch beim Kerning der Schriften 
zu Problemen.

Die Verwendung der Schriften des \CDs im OpenType"=Format sollte folglich nur 
erfolgen, wenn eine Installation der PostScript"=Schriften \emph{absolut} nicht 
möglich ist beziehungsweise \hologo{LuaLaTeX} oder \hologo{XeLaTeX} zwingend 
genutzt werden müssen. Hierfür ist es~-- nach der systemweiten Installation der 
OpenType"=Schriften~-- ausreichend, das Paket \Package{fontspec} zu laden.
\index{Schrift|?)}%



\section{Das Layout des \CDs}
Das Hauptaugenmerk der neuen Klassen liegt auf der Umsetzung des \CDs der
\TnUD für \hologo{LaTeX}. Ein großer Teil der definierten Optionen und Befehle
dient genau dazu und wird nachfolgend beschrieben.

Einige spezielle Seiten werden im prägnanten Stil mit dem Logo der \TnUD und 
der dazugehörigen Kopfzeile mit Querbalken gesetzt. Dies betrifft insbesondere 
\hyperref[sec:title]{die Umschlagseite und den Titel aus \autoref{sec:title}}, 
die \hyperref[sec:part]{Teileseiten in \autoref{sec:part}} sowie die
\hyperref[sec:chapter]{Kapitelseiten in \autoref{sec:chapter}}. Mit den 
\PageStyle{tudheadings}"=Seitenstilen oder der \Environment{tudpage}-Umgebung  
können weitere Seiten in diesem Stil erzeugt werden. Wird das Paket 
\Package{tudscrsupervisor} verwendet und mit den bereitgestellten Befehlen oder 
Umgebungen eine Aufgabenstellung, ein Gutachten oder ein Aushang erstellt, so 
erscheinen auch diese in besagtem Seitenstil.


\subsection{Die Gestalt von Titel, Umschlagseite, Teilen und Kapiteln}
\begin{Declaration}[%
  v2.03!neue Farbvarianten
    (\protect\PValue{bicolor} und \protect\PValue{fullcolor}),%
  v2.04!farbiger Querbalken möglich (\protect\PValue{barcolor})%
]{\Option{cd}[\PSet]}[true]
\printdeclarationlist%
\index{Layout}%
%
Mit dieser Option wird festgelegt, ob und wie das \CD der \TnUD verwendet wird. 
Sie hat Einfluss auf die Ausprägung von Titel"~, Teil"~, und Kapitelseiten.
%
\begin{values}
\itemfalse
  Diese Einstellung erzeugt das Standard"=Verhalten der \KOMAScript"=Klassen, 
  es wird kein \CD genutzt.
\itemtrue*[nocolor/monochrome]
  Das Layout für Titel"~, Teil"~ und Kapitelseiten ist im \CD, es wird 
  schwarze Schrift für Titel, Teil"~ und Kapitelüberschriften verwendet. Die 
  Ausprägung des Seitenkopfes ist abhängig von der Option \Option{cdhead}.
\item[lightcolor/pale]
  Die Einstellung entspricht weitestgehend der Option \Option{cd}[true], 
  allerdings wird die primäre Hausfarbe \Color{HKS41} für den Kopf des 
  \PageStyle{tudheadings}"=Seitenstils und die Überschriften genutzt.
\item[barcolor]
  \ChangedAt{v2.04} Zusätzlich zur vorherigen Einstellung wird außerdem der 
  Querbalken mit einem farbigen Hintergrund gesetzt.
\item[bicolor/bichrome]
  \ChangedAt{v2.03} Der Kopf wird mit einem farbigen Hintergrund in der 
  Hausfarbe gesetzt, auch der Querbalken wird farbig hinterlegt. Die Schrift 
  wird in der primären Hausfarbe gesetzt.
\item[color]
  Der Titel sowie Teil"~ und Kapitelseiten werden allesamt farbig und im \CD 
  gestaltet, der Seitenkopf wird in der primären Hausfarbe \Color{HKS41} 
  gesetzt, der Querbalken erhält Linien als Begrenzung.
\item[full/fullcolor]
  \ChangedAt{v2.03} Entspricht der vorherigen Einstellung, allerdings wird der 
  Querbalken nicht durch Linien begrenzt sondern farbig hinterlegt.
\end{values}
\end{Declaration}


\subsubsection{Einstellungen für Titel, Umschlagseite, Teile und Kapitel}
Das Verhalten aller für das Layout relevanten Elemente wird von der eben zuvor 
erläuterten Option \Option{cd}[\PSet] bestimmt. Dies betrifft zum einen sowohl 
den Titel~(\Macro{maketitle}) als auch die Umschlagseite~(\Macro{makecover}) 
und zum anderen alle Teileseiten~(\Macro{part}, \Macro{addpart}) und 
Kapitelseiten~(\Macro{chapter}, \Macro{addchap}).

Soll ein bestimmtes Element des Layouts abweichend erscheinen, so kann eine der 
folgenden Optionen genutzt werden, um dieses individuell anzupassen. Die 
gültigen Wertzuweisungen für die einzelnen Elemente entsprechend dabei den 
möglichen Werten für die Option \Option{cd}.

\begin{Declaration}[%
  v2.03!neue Farbvarianten
    (\protect\PValue{bicolor} und \protect\PValue{fullcolor}),%
  v2.04!farbiger Querbalken möglich (\protect\PValue{barcolor})%
]{\Option{cdtitle}[\PSet]}
\printdeclarationlist%
\index{Titel}\index{Layout!Titel}%
%
Mit der Option \Option{cdtitle} kann die allgemeine Einstellung für den Titel 
überschrieben werden. Es kann zwischen dem normalen (\Option{cdtitle}[false]) 
und dem im \CD umgeschaltet werden. Die neue Titelseite unterstützt alle durch 
\KOMAScript{} definierten Befehle für den Titel.%
\footnote{\raggedright%
  \Macro{extratitle}\Parameter{Schmutztitel},\Macro{titlehead}\Parameter{Kopf},
  \Macro{subject}\Parameter{Typisierung},\Macro{title}\Parameter{Titel},
  \Macro{subtitle}\Parameter{Untertitel},\Macro{author}\Parameter{Autor},
  \Macro{date}\Parameter{Datum},\Macro{publishers}\Parameter{Verlag},
  \Macro{and} und \Macro{thanks}\Parameter{Fußnote} sowie
  \Macro{uppertitleback}\Parameter{Titelrückseitenkopf},
  \Macro{lowertitleback}\Parameter{Titelrückseitenfuß}
  und \Macro{dedication}\Parameter{Widmung}
}
Zusätzlich werden viele neue Felder definiert, welche vor allem für eine 
wissenschaftliche Arbeit von Relevanz sind. Genaueres dazu 
ist in \autoref{sec:title} nachzulesen. Unabhängig von der gewählten Variante 
der Titelseite wird diese immer mit \Macro{maketitle} erzeugt.
\end{Declaration}

\begin{Declaration}[%
  v2.02,%
  v2.03!neue Farbvarianten
    (\protect\PValue{bicolor} und \protect\PValue{fullcolor}),%
  v2.04!farbiger Querbalken möglich (\protect\PValue{barcolor})%
]{\Option{cdcover}[\PSet]}
\printdeclarationlist%
\index{Umschlagseite|!}\index{Titel!Umschlagseite}\index{Layout!Umschlagseite}%
%
Die \TUDScript-Klassen führen zusätzlich den Befehl \Macro{makecover} ein, mit 
dem sich neben dem Titel eine separate Umschlagseite erzeugen lässt. Diese ist 
in ihrer Gestalt der Titelseite sehr ähnlich, wird normalerweise jedoch in 
einem anderen Satzspiegel als dem des Buchblocks gesetzt. Mit der Option 
\Option{cdcover} kann~-- unabhängig von \Option{cd}~-- das Erscheinungsbild 
der Umschlagseite geändert werden. Wird \Option{cdcover}[false] gewählt, 
entspricht die Umschlagseite dem originalen \KOMAScript-Titel. Die Verwendung 
des Befehls \Macro{makecover} sowie die dazugehörigen Parameter werden 
detailliert in \autoref{sec:title} erläutert.
\end{Declaration}

\begin{Declaration}[%
  v2.03!neue Farbvarianten
    (\protect\PValue{bicolor} und \protect\PValue{fullcolor}),%
  v2.04!farbiger Querbalken möglich (\protect\PValue{barcolor})%
]{\Option{cdpart}[\PSet]}
\printdeclarationlist%
\index{Teileseiten}\index{Layout!Teileseiten}%
%
Für die Teileseiten kann der Wert des Schlüssels \Option{cd} separat 
überschrieben und somit deren Layout respektive Erscheinungsbild beeinflusst 
werden, welches bei der Benutzung der Befehle \Macro{part} beziehungsweise 
\Macro{addpart} und deren Sternversionen genutzt wird.
\end{Declaration}

\begin{Declaration}[%
  v2.03!neue Farbvarianten
    (\protect\PValue{bicolor} und \protect\PValue{fullcolor}),%
  v2.04!farbiger Querbalken möglich (\protect\PValue{barcolor})%
]{\Option{cdchapter}[\PSet]}
\printdeclarationlist%
\index{Kapitelseiten}\index{Layout!Kapitelseiten}%
%
Für Kapitelseiten kann der Schlüsselwert \Option{cd} ebenfalls angepasst und 
damit das Layout respektive Erscheinungsbild geändert werden, das bei der 
Verwendung von \Macro{chapter} beziehungsweise \Macro{addchap} und den 
dazugehörigen Sternversionen genutzt wird.
\end{Declaration}
%
\begin{Example}
Soll die Titelseite in Farbe, der Rest des Dokumentes allerdings in schwarzer 
Schrift gesetzt werden, so kann dies folgendermaßen erreicht werden:
\begin{Code}[escapechar=§]
\documentclass[cd=true,cdtitle=color]{§\PName{Dokumentklasse}§}
\end{Code}
\end{Example}


\subsubsection{Position von Überschriften}
\begin{Declaration}[v2.02]{\Length{pageheadingsvskip}}
\begin{Declaration}[v2.02]{\Length{headingsvskip}}
\printdeclarationlist%
\index{Überschriften!Position}\index{Titel!Position}%
\index{Teileüberschriften}\index{Kapitelüberschriften}%
\index{Kapitelseiten}\index{Layout!Kapitelseiten}\index{Layout!Überschriften}%
%
Diese beiden Längen haben Auswirkung auf die vertikale Position bestimmter
Überschriften. Mit \Length{pageheadingsvskip} lassen sich sowohl der Titel auf
der Titelseite (\Option{titlepage}[true]) als auch die Überschriften von Teilen 
und Kapiteln, welche als einzelne Kapitelseite (\Option{chapterpage}[true]) 
gesetzt werden, verschieben. Demgegenüber erlaubt es \Length{headingsvskip}, 
sowohl den Titel innerhalb eines Titelkopfes (\Option{titlepage}[false]) als 
auch die Überschrift eines Kapitels bei deaktivierter Kapitelseite 
(\Option{chapterpage}[false]) in ihrer vertikalen Position anzupassen.

Die zuvor genannten Überschriften werden normalerweise im Layout relativ tief 
im Textbereich gesetzt. Mit negativen Werten werden die Überschriften nach oben 
verschoben, wobei darauf geachtet werden sollte, dass diese sich danach noch 
innerhalb des Satzspiegels befinden. Positive Werte setzen die Überschriften 
dementsprechend tiefer.
\end{Declaration}
\end{Declaration}



\subsection{Seiten im Stil des \CDs}
\begin{Declaration}[v2.02]{\PageStyle{tudheadings}}
\begin{Declaration}[v2.02]{\PageStyle{plain.tudheadings}}
\begin{Declaration}[v2.02]{\PageStyle{empty.tudheadings}}
\printdeclarationlist%
\label{sec:tudheadings}
%
\ChangedAt*{%
  v2.03!Seitenstile um zweifarbigen Kopf und farbigen Fuß erweitert%
}%
Ein zentrales Element des \CDs der \TnUD ist der prägnante Seitenkopf mit der 
Angabe von Fakultät~(\Macro{faculty}), Einrichtung~(\Macro{department}), 
Institut~(\Macro{institute}) und Lehrstuhl~(\Macro{chair}) im dazugehörigen 
Querbalken. Durch die Nutzung des Paketes \Package{scrlayer-scrpage} lassen 
sich entweder einzelne Seiten oder auch ganze Dokumente sehr einfach in diesem 
Stil setzen. Hierzu muss lediglich mit \Macro{pagestyle}\Parameter{Seitenstil} 
einer der Seitenstile geladen werden. 

Allen Seitenstilen gemein ist der typische Kopf mit dem charakteristischen 
Querbalken, dessen Gestalt für \emph{alle} Seitenstile gleichermaßen über die 
Option \Option{cdhead} angepasst werden kann. Mit dem Befehl \Macro{headlogo} 
lässt sich ein zusätzliches Zweitlogo im Kopfbereich ausgegeben.
\Attention{%
  Für die speziellen Layout-Elemente Titel und Umschlagseite sowie Teile- und 
  Kapitelseiten wird die Einstellung von \Option{cdhead} durch die Nutzung der 
  Option~\Option{cd} überschrieben.
}

Die Ausprägung des Fußes unterscheidet sich bei den einzelnen Seitenstilen. 
Dieser ist beim Seitenstil \PageStyle{empty.tudheadings} immer leer. Die beiden 
Stile~-- oder vielmehr das Seitenstil-Paar~-- \PageStyle{tudheadings} und 
\PageStyle{plain.tudheadings} übernehmen die Einstellungen für die Fußzeile aus 
der Anwenderschnittstelle von \Package{scrlayer-scrpage}.%
\footnote{%
  Es können die Befehle \Macro{lefoot}, \Macro{cefoot} und \Macro{refoot} sowie 
  \Macro{lofoot}, \Macro{cofoot} und \Macro{rofoot} respektive \Macro{ofoot}, 
  \Macro{cfoot} und \Macro{ifoot} genutzt werden.
}
Wie diese zu verwenden ist, kann der \KOMAScript"=Anleitung entnommen werden. 
Alternativ zu einer eigenen Definition der Fußzeile lässt sich außerdem die 
Option \Option{cdfoot} verwenden. Zusätzlich kann über \Macro{footcontent} ein 
freier Inhalt für den Fußbereich definiert werden, mit \Macro{footlogo} ist die 
Ausgabe von einem oder mehreren Logos in diesem möglich. Die verwendete Schrift 
im Fußbereich wird durch das Schriftelement~\Font{tudheadings} festgelegt.

Sobald einer der definierten Stile mit \Macro{pagestyle}\Parameter{Seitentil} 
aktiviert wurde, sind die beiden Seitenstile \PageStyle{tudheadings} sowie 
\PageStyle{plain.tudheadings} zusätzlich unter den Namen \PageStyle{headings} 
respektive \PageStyle{plain} verwendbar. Dies hat den Vorteil, dass bei 
Optionen oder Befehlen, welche automatisch zwischen den beiden Seitenstilen 
\PageStyle{headings} und \PageStyle{plain} umschalten, durch die einmalige 
Auswahl von einem der \PageStyle{tudheadings}-Stilen nun zwischen diesen  
umgeschaltet wird.

Der Seitenstil \PageStyle{empty} erzeugt allerdings weiterhin eine komplett 
leere Seite. Soll eine Seite mit der prägnanten Kopfzeile der \TnUD und leerem 
Seitenfuß erschienen, so muss \Macro{pagestyle}\PParameter{empty.tudheadings} 
manuell aufgerufen werden. Um auf das normale Verhalten von \KOMAScript{} 
zurückzuschalten, muss einer der beiden Stile \PageStyle{scrheadings} 
beziehungsweise \PageStyle{plain.scrheadings} aktiviert werden.

\Attention{%
  Die beschriebenen Seitenstile werden erst \emph{nach} dem Laden des Paketes 
  \Package{scrlayer-scrpage} definiert. Wird dieses nicht durch den Anwender 
  geladen, sollte \Macro{pagestyle}\Parameter{Seitenstil} erst nach 
  \Macro*{begin}\PParameter{document} verwendet werden.
}%
\end{Declaration}
\end{Declaration}
\end{Declaration}

\begin{Declaration}[v2.04]{\Font{tudheadings}}
\printdeclarationlist%
\index{Schriftelemente}
%
Im Fußbereich der Seiten im \PageStyle{tudheadings}-Seitenstil wird das  
Schriftelement~\Font{tudheadings} verwendet. Dieses wirkt sich auf die 
Seitenzahlen, den Kolumnentitel und die mit \Macro{footcontent} angegebenen 
Inhalte aus. Hierüber wird die Wahl der richtigen Schriftfarbe in Abhängigkeit 
vom Seitenhintergrund und den Einstellungen für die Optionen \Option{cdhead} 
sowie \Option{cdfoot} realisiert. Wie \Font{tudheadings} angepasst werden kann, 
ist in \autoref{sec:fonts:elements} zu finden.
\end{Declaration}

\begin{Declaration}{\Macro{faculty}\Parameter{Fakultät}}
\begin{Declaration}{\Macro{department}\Parameter{Einrichtung}}
\begin{Declaration}{\Macro{institute}\Parameter{Institut}}
\begin{Declaration}{\Macro{chair}\Parameter{Lehrstuhl}}
\begin{Declaration}{\Macro{extraheadline}\Parameter{Textzeile}}
\printdeclarationlist%
\index{Kopfzeile}\index{Layout!Kopfzeile}%
\index{Querbalken}\index{Layout!Querbalken}%
%
Für den Seitenstil des \CDs der \TnUD typisch ist die Kopfzeile mit dem 
charakteristischen Querbalken. In dieser wird~-- falls angegeben~-- in fetter 
Schrift die Fakultät ausgegeben, danach folgen durch Kommas getrennt die 
Einrichtung, das Institut und der Lehrstuhl beziehungsweise die Professur. 
Sollte der Platz in der ersten Zeile nicht ausreichen, erfolgt ein 
automatischer Zeilenumbruch.

In besonderen Ausnahmefällen erlaubt das \CD die Angabe einer zusätzlichen
zweiten beziehungsweise dritten Zeile unterhalb der Angaben des Bereichs an der 
\TnUD, welche weitere, frei wählbare Angaben enthält. Diese kann mit dem Befehl 
\Macro{extraheadline}\Parameter{Textzeile} definiert werden.
\end{Declaration}
\end{Declaration}
\end{Declaration}
\end{Declaration}
\end{Declaration}

\begin{Declaration}[%
  v2.03,%
  v2.04!farbiger Querbalken möglich (\PValue{barcolor})%
]{\Option{cdhead}[\PSet]}[true][%
  \Option{cdfont}[false]:false%
]
\printdeclarationlist%
\index{Kopfzeile}\index{Layout!Kopfzeile}%
\index{Querbalken}\index{Layout!Querbalken}%
%
\ToDo[imp]{%
  \Option{cdfont}[false]: Standard \Option{cdhead}[true] für \Option{cd}[true]
}[v2.05]
Mit dieser Option lassen sich für die \PageStyle{tudheadings}"=Seitenstile 
sowohl die Gestalt des Logos sowie des Querbalkens als auch die darin 
verwendete Schrift beeinflussen. Die folgenden Werte können für eine Anpassung 
der Schriftart im Balken verwendet werden:
%
\begin{values}
\itemfalse
  Sollte mit \Option{cdfont}[false] die Verwendung der Hausschrift im Stil des 
  \CDs der \TnUD deaktiviert worden sein, wird die Kopfzeile im Querbalken in
  den Serifenlosen der genutzten Schrift gesetzt. Sind die Hausschriften 
  aktiviert, hat diese Einstellung keinen Einfluss.
\itemtrue*[light/lightfont/noheavyfont]
  Im Querbalken wird für \Macro{faculty} \textubn{Univers~65~Bold} verwendet, 
  die Felder \Macro{department}, \Macro{institute}, \Macro{chair} und 
  \Macro{extraheadline} kommt \textuln{Univers~45~Light} zum Einsatz.
\item[heavy/heavyfont]
  Der Inhalt von \Macro{faculty} wird weiterhin in \textubn{Univers~65~Bold} 
  gesetzt, für die restlichen Felder wird \texturn{Univers~55~Regular} genutzt.
\end{values}
%
Bei der Ausprägung des Kopfes und des Querbalkens gibt es mehrere Varianten. 
Einerseits kann der Querbalken mit zwei Außenlinien dargestellt werden:
%
\begin{values}
\item[nocolor/monochrome]
  Der Kopf und die Linien des Querbalkens erscheinen in schwarzer Farbe.
\item[lightcolor/pale]
  Sowohl Kopf als auch Querbalken werden in der primären Hausfarbe gesetzt.
\end{values}
%
Andererseits ist auch eine Darstellung mit mehr Farbeinsatz möglich, bei 
welcher der Querbalken und gegebenenfalls der ganze Seitenkopf farblich 
abgesetzt wird.
%
\begin{values}
\item[barcolor]
  \ChangedAt{v2.04} Im Gegensatz zur vorherigen Einstellung wird der 
  Querbalken mit farbigem Hintergrund verwendet.
\item[bicolor/bichrome]
  \ChangedAt{v2.03} Die Kopfzeile wird farblich abgesetzt, wobei der 
  Hintergrund des Logos und der Querbalken unterschiedlich ausfallen. Die 
  Außenlinien der Querbalkens entfallen.
\end{values}
%
Für den Fall, dass der Querbalken lediglich mit zwei Außenlinien dargestellt 
wird, kann zusätzlich dessen Laufweite festgelegt werden:
%
\begin{values}
\item[textwidth/slim]
  Der Querbalken im Kopf erstreckt sich nur über den Textbereich. Diese 
  Einstellung ist insbesondere sinnvoll, wenn ein randloser Ausdruck technisch 
  nicht möglich ist.   
\item[paperwidth/wide]
  Die horizontale Ausdehnung des Querbalkens erstreckt sich über die komplette 
  Seitenbreite bis an den Blattrand. Dieses Verhalten ist standardmäßig im 
  farbigen Layout aktiviert.
\end{values}
\end{Declaration}

\begin{Declaration}[%
  v2.03!farbiger Hintergrund der Fußzeile möglich%
]{\Option{cdfoot}[\PSet]}[false]%
\printdeclarationlist%
\index{Kolumnentitel}\index{Layout!Kolumnentitel}
\index{Satzspiegel!doppelseitig}%

Die \TUDScript-Klassen sind~-- insbesondere aufgrund der Möglichkeit zur 
Verwendung des Paketes \Package{scrlayer-scrpage}~-- bei der Gestaltung der 
Kopf"~ und Fußzeilen sehr flexibel und individuell anpassbar. Die Ausprägung 
und der Inhalt dieser ist nicht explizit durch das \CD vorgegeben und können 
durch den Anwender beliebig gewählt und geändert werden. Wird die Klassenoption 
\Option{automark} angegeben, werden für das automatische Setzen der Marken die 
Titel der Gliederungsebenen verwendet. Genaueres hierzu sowie der Möglichkeit, 
die Kolumnentitel manuell festzulegen, ist dem Handbuch von \KOMAScript{} zu 
entnehmen.

Eine Möglichkeit für deren Gestaltung zeigt das Handbuch für das \CD der \TnUD. 
Dieses wird ohne Kopf"~ und mit einer einfachen Fußzeile gesetzt, welche den 
aktuellen Kolumnentitel sowie die Paginierung enthält. Mit \Option{cdfoot} kann 
diese Ausprägung aktiviert werden, was auch für dieses Anwenderhandbuch 
geschehen ist.
%
\begin{values}
\itemfalse
  Die Kopf"~ und Fußzeilen zeigen Standardverhalten, zur manuellen Änderung 
  dieser sollte unbedingt das \KOMAScript"=Paket \Package{scrlayer-scrpage} 
  verwendet werden.
\itemtrue*
  Die Fußzeilen des Dokumentes werden wie im Handbuch des \CDs der \TnUD 
  mit Kolumnentitel und Seitenzahl gesetzt. Im doppelseitigen Satz 
  (\Option{twoside}[true]) wird die Paginierung außen platziert.
\end{values}
%
\ChangedAt{v2.03} 
Sollte einer der \PageStyle{tudheadings}-Seitenstil aktiviert sein und es wird 
auf der erzeugten Seite ein farbiges Layout~--  beispielsweise der zweifarbige 
Kopf (\Option{cdhead}[bicolor]) oder ein farbiger Seitenhintergrund~-- genutzt, 
so kann auch die Fußzeile einen farbigen Hintergrund erhalten.
%
\begin{values}
\item[nocolor/monochrome]
  Der Fuß wird immer ohne farbigen Hintergrund gesetzt.
\item[color/bicolor/bichrome]
  Die Fußzeile wird farblich abgesetzt, falls entweder der zweifarbige Kopf
  (\Option{cdhead}[bicolor]) oder zumindest eine Seite mit einem farbigen 
  Hintergrund in der Hausfarbe (Titel oder Kapitelseite) verwendet wurde.
\end{values}
%
Wird der Option \Option{cdfoot} eine Längenmaß übergeben, entspricht dies der 
Verwendung von \Option{extrabottommargin}.
\end{Declaration}

\begin{Declaration}{\Macro{headlogo}\LParameter\Parameter{Dateiname}}
\printdeclarationlist%
\index{Kopfzeile}\index{Layout!Kopfzeile}%
\index{Zweitlogo|?}\index{Layout!Zweitlogo}\index{\DDC-Logo}%
%
Neben dem Logo der \TnUD darf zusätzlich ein Zweitlogo im Kopf verwendet 
werden. Dieses lässt sich mit diesem Befehl einbinden. Normalerweise wird es 
auf die Höhe der Erstlogos skaliert. Über das optionale Argument können weitere 
Formatierungsbefehle an den verwendeten Befehl \Macro{includegraphics} 
durchgereicht werden, um beispielsweise die Größe des Zweitlogos anzupassen.
Welche Parameter angepasst werden können, ist der Dokumentation des
\Package{graphicx}-Paketes zu entnehmen.

Sollte die Option \Option{ddc} aktiviert sein, wird das \DDC-Logo nicht im Kopf 
sondern automatisch im Fuß gesetzt. Die Option \Option{ddchead} setzt dieses 
auf jeden Fall im Kopf und überschreibt damit das mit \Macro{headlogo} 
angegebene Zweitlogo.
\end{Declaration}

\begin{Declaration}[v2.03]{%
  \Macro{footlogo}\LParameter\Parameter{Dateinamenliste}%
}
\begin{Declaration}[v2.03]{\Macro{footlogosep}}%
\begin{Declaration}[v2.03]{\Length{footlogoheight}}%
\printdeclarationlist%
\index{Drittlogo|?}\index{Layout!Drittlogo}\index{\DDC-Logo}%
\index{Fußzeile}\index{Layout!Fußzeile}%

Laut den Richtlinien des \CDs dürfen im Fußbereich weitere Logos erscheinen, 
beispielsweise von kooperierenden Unternehmen oder Sponsoren. Die Dateinamen 
der gewünschten Logos können als kommaseparierte Liste im obligatorischen 
Argument des Befehls \Macro{footlogo} angegeben werden. Sollte tatsächlich 
nicht nur ein Dateiname sondern eine Liste übergeben worden sein, so wird bei 
der Ausgabe der Logos zwischen diesen jeweils der in \Macro{footlogosep} 
gespeicherte Separator~-- standardmäßig \Macro*{hfill}~-- gesetzt. Dieser kann 
mit \Macro*{renewcommand*}\Macro{footlogosep}\PParameter{\dots} beliebig durch 
den Anwender angepasst werden. Der Separator wird auch gesetzt, wenn in der 
\Parameter{Dateinamenliste} lediglich ein Komma verwendet wurde. So kann man 
beispielsweise ein Logo mit \Macro{footlogo}\PParameter{,\PName{Dateiname},} 
zentriert im Fuß setzen.

Das optionale Argument von \Macro{footlogo} wird an \Macro{includegraphics} 
weitergereicht. Dies geschieht für alle angegeben Dateien aus der Liste 
gleichermaßen. Sollen für einzelne Logos individuelle Einstellungen vorgenommen 
werden, so sind die entsprechenden Parameter im obligatorischen Argument nach 
dem jeweiligen Dateinamen mit einem Doppelpunkt~\enquote{\PValue{:}} als 
Separator (\Macro{footlogo}\PParameter{\PName{Dateiname}:\PName{Parameter}}) zu 
übergeben, wobei diese \emph{nach} den allgemeinen Einstellungen für alle Logos 
angewendet werden. Die möglichen Parameter und Werte für die optionalen 
Argumente sind der Dokumentation des \Package{graphicx}-Paketes zu entnehmen.

Ohne die Angabe eines optionalen Argumentes für die Größe werden alle Logos im 
Fuß auf die auf die Höhe des Logos der \TnUD skaliert. Der Anwender kann dies 
ändern, indem der Wert der Länge \Length{footlogoheight} mit \Macro*{setlength} 
auf einen von \PValue{0pt} verschiedenen gesetzt wird. Sollte die Höhe des 
Fußbereiches nicht ausreichen, um alle Logos in der gewünschten Größe 
darstellen zu können, kann diese über die Option \Option{extrabottommargin} 
beziehungsweise \Option{cdfoot} angepasst werden.
\end{Declaration}
\end{Declaration}
\end{Declaration}

\begin{Declaration}[v2.04]{\Macro{footcontent}%
  \OParameter{Anweisungen}\Parameter{Inhalt}\OParameter{Inhalt}%
}
\begin{Declaration}[v2.04]{\Macro*{footcontent*}
    \OParameter{Anweisungen}\Parameter{Inhalt}\OParameter{Inhalt}%
}
\printdeclarationlist%
\index{Fußzeile}\index{Layout!Fußzeile}%
%
Mit diesem Befehl kann beliebiger Inhalt entweder einspaltig oder zweispaltig 
im Fußbereich der \PageStyle{tudheadings}"=Seitenstile gesetzt werden. In der 
Form \Macro{footcontent}\Parameter{Inhalt} wird der Inhalt über die komplette 
Textbreite im Fuß ausgegeben. Wird der Befehl jedoch in der zweiten Variante 
\Macro{footcontent}\Parameter{linker Inhalt}\OParameter{rechter Inhalt} mit 
einem optionalen \emph{nach} dem obligatorischen Argument verwendet, so 
erscheint der Fußbereich zweispaltig, wobei der Inhalt aus dem ersten, 
obligatorischen Argument in der linken und der Inhalt aus dem zweiten, 
optionalen Argument entsprechend in der rechten Fußspalte gesetzt wird.

Im Normalfall wird das Schriftelement \Font{tudheadings} für die Schrift im 
Fußbereich verwendet. Mit dem ersten optionalen Argument können weitere 
Schrifteinstellungen respektive Anweisungen vor der eigentlichen Ausgabe des 
Inhaltes erfolgen. Soll die Definition des Inhalts für den Fußbereich gänzlich 
ohne die automatische Schriftformatierungen erfolgen, so kann die Sternversion 
\Macro{footcontent*} genutzt und gegebenenfalls die Schriftformatierung über 
das optionale Argument vorgenommen werden.
\end{Declaration}
\end{Declaration}

\begin{Declaration}[%
  v2.02!\protect\DDC-Logo automatisch in Kopf oder Fuß%
]{\Option{ddc}[\PSet]}[false]
\begin{Declaration}[v2.02]{\Option{ddchead}[\PSet]}[false]
\begin{Declaration}[%
  v2.02!neue Werte für die Farbwahl des Logos von \protect\DDC%
]{\Option{ddcfoot}[\PSet]}[false]
\printdeclarationlist%
\index{Zweitlogo}\index{Layout!Zweitlogo}\index{\DDC-Logo}%
%
Diese Optionen fügen das Logo von \DDC entweder im Kopf oder im Fuß der Seiten
mit dem Stil \PageStyle{tudheadings} ein. Mit \Option{ddc} wird dieses 
automatisch entweder im Kopf oder~-- falls ein Zweitlogo mit \Macro{headlogo} 
angegeben wurde~-- im Fuß gesetzt. Die anderen beiden Optionen setzen das Logo 
zwingend entweder im Kopf (\Option{ddchead}) oder im Fuß (\Option{ddcfoot}), 
wobei erstgenannte ein optionales Zweitlogo dabei unterdrückt. Die Verwendung 
einer der drei Optionen führt zur Deaktivierung der anderen beiden, sie 
schließen sich folglich gegenseitig aus. Die möglichen Werte für diese Optionen 
sind:
%
\begin{values}
\itemfalse
  Bei den \PageStyle{tudheadings}-Seitenstile erscheint kein Logo von \DDC.
\itemtrue*
  Das Logo von \DDC wird im Kopf beziehungsweise im Fuß verwendet. Die Wahl der 
  Farbe des Logos geschieht passend zur farblichen Ausprägung der Seite selbst.
\end{values}
%
Soll die Farbe des \DDC-Logos manuell erfolgen, können folgende Werte verwendet 
werden:
%
\begin{values}
\item[color]
  Im Kopf oder Fuß wird die achtfarbige 4C"~Variante des \DDC-Logos genutzt.
\item[colorblack]
  Es wird das achtfarbige Logo mit schwarzem \DDC-Schriftzug anstelle des 
  grauen verwendet. Für den Fuß wird der grüne Claim ebenfalls durch einen 
  schwarzen ersetzt. Dies ist insbesondere für kleine Darstellungen des Logos 
  im Fuß sinnvoll.
\item[gray/grey]
  Dies Ausgabe des \DDC-Logos erfolgt in Graustufen.
\item[black]
  Verwendung des Logos in Graustufen mit schwarzem Schriftzug.
\item[blue]
  Der Schriftzug und das Logo werden in der primären Hausfarbe \Color{HKS41} 
  und den entsprechenden Abstufungen gesetzt
\item[white]
  Das \DDC-Logo sowie der dazugehörige Schriftzug sind vollständig weiß.
\end{values}
%
Die Größe des \DDC-Logos ist vorgegeben. Es wird sowohl im Kopf als auch im Fuß 
in der gleichen Höhe gesetzt, wie das Logo der \TnUD. Wird es im Fuß gesetzt, 
lässt sich die Größe allerdings über die Länge \Length{footlogoheight} ändern. 
Sollte nach einer Vergrößerung der Darstellung die Höhe des Fußbereiches nicht 
ausreichen, so kann diese über die Option \Option{cdfoot} beziehungsweise 
\Option{extrabottommargin} angepasst werden.
\end{Declaration}
\end{Declaration}
\end{Declaration}

\ToDo[imp,nxt]{%
  Spezialseite zur freien Gestaltung mit Hintergrundebene für Bilder und Texte  
  (CD-Handbuch S. 80 ff.); Seitenstil: \PageStyle{special.tudheadings}
}[v2.06]

\begin{Declaration}[%
  v2.02!\Key{\Environment{tudpage}}{head} \emph{entfällt},
  v2.02!\Key{\Environment{tudpage}}{foot} \emph{entfällt},
  v2.03!\Key{\Environment{tudpage}}{color} \emph{entfällt}
]{\Environment{tudpage}[\OLParameter{Sprache}]}
\begin{Declaration}{\Key{\Environment{tudpage}}{language}[\PName{Sprache}]}
\begin{Declaration}{\Key{\Environment{tudpage}}{columns}[\PName{Anzahl}]}
\begin{Declaration}[v2.02]{\Key{\Environment{tudpage}}{pagestyle}[\PSet]}
\begin{Declaration}{\Key{\Environment{tudpage}}{cdfont}[\PSet]}{%
  \see*{\Option{cdfont}'ppage'}%
}
\begin{Declaration}[v2.03]{\Key{\Environment{tudpage}}{cdhead}[\PSet]}{%
  \see*{\Option{cdhead}'ppage'}%
}
\begin{Declaration}[v2.03]{\Key{\Environment{tudpage}}{cdfoot}[\PSet]}{%
  \see*{\Option{cdfoot}'ppage'}%
}
\begin{Declaration}{\Key{\Environment{tudpage}}{headlogo}[\PName{Dateiname}]}{%
  \see*{\Macro{headlogo}'ppage'}%
}
\begin{Declaration}[v2.03]{%
  \Key{\Environment{tudpage}}{footlogo}[\PName{Dateinamenliste}]
}{\see*{\Macro{footlogo}'ppage'}}
\begin{Declaration}[v2.02]{\Key{\Environment{tudpage}}{ddc}[\PSet]}{%
  \see*{\Option{ddc}'ppage'}%
}
\begin{Declaration}[v2.02]{\Key{\Environment{tudpage}}{ddchead}[\PSet]}{%
  \see*{\Option{ddchead}'ppage'}%
}
\begin{Declaration}[v2.02]{\Key{\Environment{tudpage}}{ddcfoot}[\PSet]}{%
  \see*{\Option{ddcfoot}'ppage'}%
}
\printdeclarationlist%
\index{Layout}\index{Layout!Seitenstil}%
\index{Kopfzeile}\index{Layout!Kopfzeile}%
\index{Fußzeile}\index{Layout!Fußzeile}%
%
Die \Environment{tudpage}"=Umgebung hat ihren Ursprung in einer früheren 
Version, als die \PageStyle{tudheadings}"=Seitenstile noch nicht verfügbar 
waren, welche mittlerweile anstelle dieser Umgebung verwendet werden können.
Für die \Environment{tudpage}"=Umgebung lassen sich verschiedene Parameter als 
optionales Argument angegeben. Wird das Paket \Package{babel} genutzt, kann die 
verwendete Sprache mit \Key{\Environment{tudpage}}{language}[\PName{Sprache}] 
geändert werden, was zur Anpassung der sprachspezifischen Trennungsmuster und 
Bezeichner führt. Wurde das Paket \Package{multicol} geladen, wird mit dem 
Parameter \Key{\Environment{tudpage}}{columns}[\PName{Anzahl}] der Inhalt der 
Umgebung mehrspaltig gesetzt. Mit \Key{\Environment{tudpage}}{pagestyle} kann 
der Seitenstil angepasst werden, wobei \PValue{headings}, \PValue{plain} und 
\PValue{empty} gültige Werte sind. 

Die weiteren Parameter entsprechen in ihrem Verhalten prinzipiell den 
gleichnamigen Klassenoptionen respektive Befehlen, wirken sich jedoch nur lokal 
innerhalb der \Environment{tudpage}"=Umgebung aus. Das Verhalten sowie die 
jeweils gültigen Wertzuweisungen können in den entsprechenden Abschnitten des 
Handbuchs nachgelesen werden.
\end{Declaration}
\end{Declaration}
\end{Declaration}
\end{Declaration}
\end{Declaration}
\end{Declaration}
\end{Declaration}
\end{Declaration}
\end{Declaration}
\end{Declaration}
\end{Declaration}
\end{Declaration}



\subsection{Der Titel und die Umschlagseite}
\label{sec:title}%\label{sec:cover}%
\index{Titel|!(}\index{Umschlagseite|!}\index{Layout!Umschlagseite}%
%
\ChangedAt*{%
  v2.03!Bugfix für Umschlagseite und Titel beim Satzspiegel%
}
Für den Titel werden alle Felder unterstützt, die bereits durch \KOMAScript{} 
bereitgestellt werden. Darüber hinaus werden für die \TUDScript-Klassen weitere 
Felder definiert, die Auswirkungen auf die Gestalt des Titels haben. Diese 
werden nachfolgend in diesem \autorefname erläutert. Der Titel~-- bestehend aus 
möglichem Schmutztitel, der eigentlichen Titelseite und der nachgelagerten 
Elementen~-- kann mit dem Befehl \Macro{maketitle} ausgegeben werden. Außerdem 
kann im zweispaltigen Satz der \Macro{maketitleonecolumn} verwendet werden, 
welcher einen einspaltigen Einfügung nach dem Titel selbst ermöglicht.

Zusätzlich zum Titel lässt sich mit \Macro{makecover} eine Umschlagseite 
erzeugen. Diese kann insbesondere für gebundene Arbeiten verwendet werden. Es 
wird~-- im Vergleich zum Titel~-- lediglich einer reduzierte Anzahl an Feldern 
auf dieser ausgegeben.

\ChangedAt{v2.02}
Für alle Felder des Titels und der Umschlagseite lässt sich die verwendete 
Schrift anpassen. Dabei werden sowohl die bereits durch \KOMAScript{} 
bereitgestellten Schriftelemente \Font{titlehead}, \Font{subject}, 
\Font{title}, \Font{subtitle}, \Font{author}, \Font{date}, \Font{publishers} 
und \Font{dedication} als auch die neuen \Font{titlepage} und \Font{thesis} 
unterstützt.
%
\begin{Example}
In diesem Dokument wurde der Untertitel derart geändert, dass dieser nicht 
standardmäßig in \DIN sondern in \textubn{Univers~65~Bold} ausgegeben wird.
\begin{Code}[escapechar=§]
\addtokomafont{subtitle}{\univbn}
\end{Code}
\end{Example}

\begin{Declaration}[%
  v2.01!Bugfix für Schriftstärke auf Titelseite,%
  v2.02!Unterstützung der Schriftelemente \Font*{titlehead}{,} 
    \Font*{subject}{,} \Font*{title}{,} \Font*{subtitle}{,} \Font*{author}{,} 
    \Font*{date}{,} \Font*{publishers}{,} \Font*{dedication}{,} 
    \Font*{titlepage} und \Font*{thesis}%
]{\Macro{maketitle}\OLParameter{Seitenzahl}}
\begin{Declaration}[v2.02]{%
  \Key{\Macro{maketitle}}{pagenumber}[\PName{Seitenzahl}]%
}
\begin{Declaration}[v2.02]{\Key{\Macro{maketitle}}{cdfont}[\PSet]}{%
  \see*{\Option{cdfont}'ppage'}%
}
\begin{Declaration}[v2.03]{\Key{\Macro{maketitle}}{cdhead}[\PSet]}{%
  \see*{\Option{cdhead}'ppage'}%
}
\begin{Declaration}[v2.03]{\Key{\Macro{maketitle}}{cdfoot}[\PSet]}{%
  \see*{\Option{cdfoot}'ppage'}%
}
\begin{Declaration}[v2.03]{%
  \Key{\Macro{maketitle}}{headlogo}[\PName{Dateiname}]%
}{\see*{\Macro{headlogo}'ppage'}}
\begin{Declaration}[v2.03]{%
  \Key{\Macro{maketitle}}{footlogo}[\PName{Dateinamenliste}]
}{\see*{\Macro{footlogo}'ppage'}}
\begin{Declaration}[v2.03]{\Key{\Macro{maketitle}}{ddc}[\PSet]}{%
  \see*{\Option{ddc}'ppage'}%
}
\begin{Declaration}[v2.03]{\Key{\Macro{maketitle}}{ddchead}[\PSet]}{%
  \see*{\Option{ddchead}'ppage'}%
}
\begin{Declaration}[v2.03]{\Key{\Macro{maketitle}}{ddcfoot}[\PSet]}{%
  \see*{\Option{ddcfoot}'ppage'}%
}
\printdeclarationlist%
\index{Layout!Titel}%
\index{Satzspiegel!doppelseitig}%
%
Der Befehl \Macro{maketitle} setzt für \Option{cdtitle}[false] den normalen 
\KOMAScript"=Titel{}, ansonsten wird die Titelseite im \CD der \TnUD erzeugt. 
Die letztere Variante ist im Vergleich zum Standardtitel um eine Vielzahl von 
Feldern erweitert worden und erlaubt insbesondere die Angabe von Daten für das 
Deckblatt einer akademischen Abschlussarbeit. Die einzelnen Felder werden 
später in diesem \autorefname erläutert. Wird das Dokument doppelseitig und mit 
rechts öffnenden Kapiteln gesetzt,%
\footnote{%
  \Option{twoside} und \Option{open}[right], Standard für \Class{tudscrbook}
}
so wird zusätzlich die Option \Option{clearcolor} beachtet. Dies gilt es 
insbesondere bei der Verwendung der Befehle \Macro{uppertitleback} respektive 
\Macro{lowertitleback} für die Titelrückseite zu beachten.

Das optionale Argument erlaubt~-- ebenso wie bei den \KOMAScript"=Klassen~-- 
die Änderung der Seitenzahl der Titelseite. Diese wird jedoch nicht ausgegeben, 
sondern beeinflusst lediglich die Zählung. Sie sollten hier unbedingt eine 
ungerade Zahl wählen, da sonst die gesamte Zählung durcheinander gerät. 
Wird eine Titelseite (\Option{titlepage}[true]) im \CD der \TnUD gesetzt 
(\Option{cdtitle}[true]), können auch die weiterhin aufgeführten Parameter im 
optionalen Argument verwendet werden. Diese entsprechen in ihrem Verhalten den 
gleichnamigen Optionen respektive Befehlen, wirken sich jedoch nur lokal und 
einzig auf die Titelseite aus. So kann beispielsweise die Nutzung eines 
\DDC-Logos auf den Titel beschränkt bleiben.
\end{Declaration}
\end{Declaration}
\end{Declaration}
\end{Declaration}
\end{Declaration}
\end{Declaration}
\end{Declaration}
\end{Declaration}
\end{Declaration}
\end{Declaration}

\begin{Declaration}{%
  \Macro{maketitleonecolumn}\OLParameter{Seitenzahl}\Parameter{Einspaltentext}%
}
\printdeclarationlist%
\index{Layout!Titel}%
\index{Satzspiegel!doppelseitig}%
\index{Zweispaltensatz}%
%
Im zweispaltigen Satz (\Option{twocolumn}) wird mit \Macro{maketitle} die 
Titelseite selbst immer einspaltig gesetzt. Direkt nach dem Titel folgt 
normalerweise der zweispaltige Fließtext. Mit dem \TUDScript-Befehl 
\Macro{maketitleonecolumn} kann nach dem Titel zusätzlich noch weiterer 
Inhalt~-- beispielsweise eine Zusammenfassung beziehungsweise eine 
Kurfassung~-- einspaltig gesetzt werden.

Wird der Befehl bei einer Titelseite (\Option{titlepage}[true]) verwendet, wird 
der Inhalt des Argumentes (\PName{Einspaltentext}) direkt nach dieser auf einer 
oder gegebenenfalls mehreren neuen Seiten ebenfalls einspaltig ausgegeben. 
Kommt jedoch ein Titelkopf (\Option{titlepage}[false]) zum Einsatz, so folgt 
diesem die einspaltige Textpassage aus dem obligatorischen Argument direkt. 
Dabei erfolgt gegebenenfalls ein automatischer Seitenumbruch, falls der Inhalt 
nicht auf eine einzelne Seite passt. Nach dem \PName{Einspaltentext} des 
obligatorischen Argumentes wird direkt danach und ohne zusätzlichen Umbruch auf 
das zweispaltige Layout umgeschaltet.

Der optionale Parameter von \Macro{maketitleonecolumn} kann äquivalent zu 
\Macro{maketitle} für die Änderung der Seitenzahl, der verwendeten Schrift 
sowie zur Anpassung von Kopf und Fuß verwendet werden. Dabei ist zu beachten, 
dass ein Großteil der Parameter nur Auswirkungen haben, falls eine Titelseite
(\Option{titlepage}[true]) verwendet wird.
\end{Declaration}

\begin{Declaration}[%
  v2.02!Umschlagseite für Layout ohne \noexpand\CD hinzugefügt,%
  v2.02!Unterstützung der Schriftelemente \Font*{titlehead}{,} 
    \Font*{subject}{,} \Font*{title}{,} \Font*{subtitle}{,} \Font*{author}{,} 
    \Font*{publishers}{,} \Font*{titlepage} und \Font*{thesis}%
]{\Macro{makecover}\OLParameter{Seitenzahl}}
\begin{Declaration}[v2.02]{%
  \Key{\Macro{makecover}}{pagenumber}[\PName{Seitenzahl}]%
}
\begin{Declaration}{\Key{\Macro{makecover}}{cdgeometry}[\PBoolean]}
\begin{Declaration}[v2.02]{\Key{\Macro{makecover}}{cdfont}[\PSet]}{%
  \see*{\Option{cdfont}'ppage'}%
}
\begin{Declaration}[v2.03]{\Key{\Macro{makecover}}{cdhead}[\PSet]}{%
  \see*{\Option{cdhead}'ppage'}%
}
\begin{Declaration}[v2.03]{\Key{\Macro{makecover}}{cdfoot}[\PSet]}{%
  \see*{\Option{cdfoot}'ppage'}%
}
\begin{Declaration}[v2.03]{%
  \Key{\Macro{makecover}}{headlogo}[\PName{Dateiname}]%
}{\see*{\Macro{headlogo}'ppage'}}
\begin{Declaration}[v2.03]{%
  \Key{\Macro{makecover}}{footlogo}[\PName{Dateinamenliste}]
}{\see*{\Macro{footlogo}'ppage'}}
\begin{Declaration}[v2.03]{\Key{\Macro{makecover}}{ddc}[\PSet]}{%
  \see*{\Option{ddc}'ppage'}%
}
\begin{Declaration}[v2.03]{\Key{\Macro{makecover}}{ddchead}[\PSet]}{%
  \see*{\Option{ddchead}'ppage'}%
}
\begin{Declaration}[v2.03]{\Key{\Macro{makecover}}{ddcfoot}[\PSet]}{%
  \see*{\Option{ddcfoot}'ppage'}%
}
\printdeclarationlist%
%
Eine Umschlagseite wird zumeist für gebundene Abschlussarbeiten verlangt, um 
diese beispielsweise für einen Prägedruck auf dem Buchdeckel zu verwenden. 
Deshalb ist die farbige Ausprägung der Umschlagseite auch deaktiviert, wenn 
diese für das restliche Dokument aktiv ist (\Option{cd}[color]). Dies kann 
jedoch jederzeit mit \Option{cdcover}[\PSet] überschrieben werden.

Wird \Option{cdcover}[true] gewählt, so wird die Umschlagseite im \CD der 
\TnUD gesetzt. Auf dieser werden der Titel des Dokumentes, die Typisierung 
durch \Macro{thesis} und/oder \Macro{subject} sowie der Autor oder respektive 
die Autoren und gegebenenfalls der mit \Macro{publishers} angegebene Verlag 
ausgegeben.
\ChangedAt{v2.02}
Für die Einstellung \Option{cdcover}[false] wird lediglich der normale 
\KOMAScript"=Titel als separate Umschlagseite ausgegeben. 

Die Titelseite selbst gehört immer zum Buchblock und wird daher im gleichen 
Satzspiegel gesetzt. Dem entgegen steht die Umschlagseite, welche zumeist in 
einem anderen Layout erscheint. Normalerweise wird das Cover~-- unabhängig von 
der Option \Option{cdgeometry}~-- im asymmetrischen Satzspiegel des \CDs 
gesetzt. Mit \Key{\Macro{makecover}}{cdgeometry}[false] im optionalen Argument 
kann das Verhalten geändert werden. In diesem Fall erscheint auch die 
Umschlagseite im Buchblock des restlichen Dokumentes. Allerdings können für 
diese Einstellung die Seitenränder mit den Befehlen \Macro{coverpagetopmargin}, 
\Macro{coverpageleftmargin}, \Macro{coverpagerightmargin} sowie 
\Macro{coverpagebottommargin} durch den Nutzer frei angepasst werden. Mehr dazu 
ist im \KOMAScript"=Handbuch \scrguide zu finden.

Außerdem kann mit dem optionalen Argument die Seitenzahl der Umschlagseite 
geändert werden. Diese wird jedoch nicht ausgegeben, sondern beeinflusst 
lediglich die Zählung. Sie sollten hier unbedingt eine ungerade Zahl wählen, da 
sonst die gesamte Zählung durcheinander gerät. Die weiterhin aufgeführten 
Parameter entsprechen in ihrem Verhalten beziehungsweise ihrer Funktion den 
gleichnamigen Optionen respektive Befehlen, wirken sich jedoch nur lokal und 
einzig auf die Umschlagseite aus.

\end{Declaration}
\end{Declaration}
\end{Declaration}
\end{Declaration}
\end{Declaration}
\end{Declaration}
\end{Declaration}
\end{Declaration}
\end{Declaration}
\end{Declaration}
\end{Declaration}

\begin{Declaration}{\Macro{titledelimiter}\Parameter{Trennzeichen}}
\printdeclarationlist%
\index{Titel!Felder}\index{Titel!Trennzeichen}%%
%
Für den Titel und die Umschlagseite werden durch die \TUDScript-Klassen
eine Reihe von zusätzlichen Feldern bereitgestellt. Einigen dieser Felder wird 
eine Beschreibung (\see*{\autoref{sec:localization}}) vorangestellt. Dazwischen 
wird bei der Ausgabe ein Trennzeichen eingefügt. Ein Doppelpunkt gefolgt von 
einem Leerzeichen (:\Macro*{nobreakspace}) ist hierfür die Voreinstellung. Mit 
dem Befehl \Macro{titledelimiter} lässt sich dieses Trennzeichen beliebig an 
die individuellen Wünsche des Anwenders anpassen.
\end{Declaration}

\begin{Declaration}{\Macro{author}\Parameter{Autor(en)}}
\begin{Declaration}{\Macro{authormore}\Parameter{Autorenzusatz}}
\begin{Declaration}{\Macro{dateofbirth}\Parameter{Geburtsdatum}}
\begin{Declaration}{\Macro{placeofbirth}\Parameter{Geburtsort}}
\begin{Declaration}{\Macro{matriculationnumber}\Parameter{Matrikelnummer}}
\begin{Declaration}{\Macro{matriculationyear}\Parameter{Immatrikulationsjahr}}
\printdeclarationlist%
\index{Titel!Felder}\index{Autorenangaben|?}%
\index{Datum!Geburtsdatum|?}%
%
Mit dem Befehl \Macro{author} wird der Autor angegeben. Innerhalb des 
Argumentes können auch mehrere Autoren aufgeführt werden, wobei diese in diesem 
Fall jeweils mit \Macro{and} zu trennen sind. Alle weiteren hier vorgestellten 
Befehle können selbst im Argument von \Macro{author} verwendet werden, wodurch 
für jeden Autor individuelle Angaben möglich sind.

Mit \Macro{authormore} wird unter dem Autor eine Zeile ausgegeben, welche 
durch den Anwender frei belegt werden kann. Sollte das Paket \Package{isodate} 
geladen sein, so wird die damit eingestellte Formatierung des Datums durch 
\Macro{dateofbirth}~-- wie übrigens bei jedem anderem Datumsfeld der 
\TUDScript-Klassen auch~-- verwendet. Dafür der Befehl \Macro{printdate} aus 
diesem Paket verwendet. Die weiteren Befehle als zusätzliche Angabe erklären 
sich von selbst.
\end{Declaration}
\end{Declaration}
\end{Declaration}
\end{Declaration}
\end{Declaration}
\end{Declaration}

\begin{Declaration}{\Macro{and}}
\printdeclarationlist%
\index{Kollaboratives Schreiben|?}\index{Titel!Kollaboratives Schreiben}%
\index{Autorenangaben!kollaborativ}
%
Dieser Befehl wird sowohl bei den \hologo{LaTeX}"=Standardklassen als auch bei 
den \KOMAScript"=Klassen lediglich auf der Titelseite dazu verwendet, mehrere 
Autoren im Argument von \Macro{author} voneinander zu trennen.

Bei den \TUDScript-Klassen hingegen ist dieser Befehl derart in seiner Funktion 
erweitert worden, dass damit die Angabe einer kollaborativen Autorenschaft für 
Abschlussarbeiten innerhalb des Befehls \Macro{author} möglich ist. Außerdem 
kann er noch im Argument von \Macro{supervisor}, \Macro{referee} sowie 
\Macro{advisor} verwendet werden, um mehrere Betreuer beziehungsweise Gutachter 
und Fachreferenten anzugeben. Er ist dabei nicht auf die Verwendung für den 
Titel allein beschränkt. Auch bei den Umgebungen \Environment{task}, 
\Environment{evaluation} und \Environment{notice} kann er eingesetzt werden.
\end{Declaration}
%
\begin{Example}
Angenommen, es soll eine Abschussarbeit von zwei unterschiedlichen Autoren in 
kollaborativer Gemeinschaft erstellt werden, so könnte man die Autorenangaben 
folgendermaßen gestalten:
\begin{Code}
\author{%
  Mickey Mouse
  \matriculationnumber{12345678}
  \dateofbirth{2.1.1990}
  \placeofbirth{Dresden}
\and%
  Donald Duck
  \matriculationnumber{87654321}
  \dateofbirth{1.2.1990}
  \placeofbirth{Berlin}
}
\matriculationyear{2010}
\end{Code}
Alle zusätzlichen Angaben außerhalb des Argumentes von \Macro{author} werden 
für beide Autoren gleichermaßen übernommen. Angaben innerhalb des Argumentes 
von \Macro{author} werden den jeweiligen, mit \Macro{and} getrennten Autoren 
zugeordnet. Mehr dazu ist im Minimalbeispiel in \autoref{sec:exmpl:thesis}.
\end{Example}

\begin{Declaration}{\Macro{thesis}\Parameter{Typisierung}}
\begin{Declaration}{\Macro{subject}\Parameter{Typisierung}}
\printdeclarationlist%
\index{Titel!Felder}%
\index{Abschlussarbeit|!}\index{Typisierung}%
%
Mit diesen beiden Befehlen kann der Typ der Dokumentes beziehungsweise der 
Abschlussarbeit angegeben werden. Während der Befehl \Macro{thesis} den Inhalt 
des Feldes unter dem Titel vertikal zentriert und in \DIN auf der Titelseite 
ausgibt, erscheint der Inhalt des Befehls \Macro{subject} in \Univers oberhalb 
des Titels. Es können auch beide Befehle parallel mit unterschiedlichen 
Inhalten verwendet werden. Der Befehl \Macro{thesis} dient den 
\TUDScript"=Dokumentklassen außerdem zur Erkennung von Abschlussarbeiten 
gedacht, da für diese spezielle Felder bereitgehalten werden und auch die 
Titelseite leicht geändert gesetzt wird.

Des Weiteren ist es bei beiden Befehlen möglich, spezielle Werte als Argument 
zur Typisierung des Dokumentes zu verwenden. Diese werden entsprechend der 
gewählten Dokumentensprache~-- entweder Deutsch oder Englisch~-- entschlüsselt 
und gesetzt. Die möglichen Werte sind \autoref{tab:thesis} zu entnehmen. Dabei 
ist zu beachten, dass das Setzen eines speziellen Wertes für \emph{entweder} 
\Macro{thesis} \emph{oder} \Macro{subject} möglich ist. Die Verwendung eines 
der genannten Werte führt immer dazu, dass das Dokument als Abschlussarbeiten 
erkannt und die erweiterte Titelseite aktiviert wird. Gleichzeitig wird damit 
die Option \Option{subjectthesis} beeinflusst. Sollte vom Anwender kein 
explizites Verhalten für \Option{subjectthesis} definiert sein, so führt die 
Verwendung von \Macro{thesis}\Parameter{Wert} zu \Option{subjectthesis}[false] 
und \Macro{subject}\Parameter{Wert} zu \Option{subjectthesis}[true].
%
\begin{table}
\index{Bezeichner}\index{Bezeichner!Typisierung}%\\
\index{Abschlussarbeit!Typisierung}%
\caption{%
  Spezielle Werte zur Typisierung des Dokumentes für
  \Macro{thesis} und \Macro{subject}%
}
\label{tab:thesis}%
\centering%
\makeatletter%
\def\@tempa#1{%
  \Term{#1} & \@nameuse{#1} & \selectlanguage{english}\@nameuse{#1}%
  \tabularnewline%
}%
\begin{tabular}{llll}
  \toprule
  \textbf{Wert} & \textbf{Bezeichner}
    & \textbf{Deutsch} & \textbf{Englisch} \tabularnewline
  \midrule
  diss & \@tempa{dissertationname}
  doctoral & \@tempa{dissertationname}
  phd & \@tempa{dissertationname}
  diploma & \@tempa{diplomathesisname}
  master & \@tempa{masterthesisname}
  bachelor & \@tempa{bachelorthesisname}
  student & \@tempa{studentresearchname}
  project & \@tempa{projectpapername}
  seminar & \@tempa{seminarpapername}
  research & \@tempa{researchname}
  log & \@tempa{logname}
  report & \@tempa{reportname}
  internship & \@tempa{internshipname}
  \bottomrule
\end{tabular}
\makeatother%
\end{table}
\end{Declaration}
\end{Declaration}

\begin{Declaration}{\Option{subjectthesis}[\PBoolean]}%
  [false][\Macro{subject}\Parameter{\autoref{tab:thesis}}:true]
\printdeclarationlist%
%
Der Befehl \Macro{thesis} dient den \TUDScript"=Hauptklassen zur Unterscheidung 
zwei unterschiedlicher Ausprägungen der Titelseite und ist speziell für 
Abschlussarbeiten gedacht. Außerdem kann bei der Nutzung spezieller Werte 
aus \autoref{tab:thesis} innerhalb des Argumentes von \Macro{subject} ebenfalls 
das Verhalten für Abschlussarbeiten aktiviert werden, wobei hierdurch die 
Einstellung \Option{subjectthesis}[true] automatisch vorgenommen wird.

Für den Standardfall~-- bekanntlich \Option{subjectthesis}[false]~-- wird der 
durch \Macro{thesis} gegebene Typ der Abschlussarbeit sowie der gegebenenfalls 
durch \Macro{graduation} gesetzte angestrebte Abschluss in großen Lettern und 
sehr zentral auf der Titelseite gesetzt. Die Verwendung von \Macro{subject} ist 
hierbei weiterhin möglich.
%
Wird die Option mit \Option{subjectthesis}[true] aktiviert, so wird die mit 
\Macro{thesis} gesetzte Bezeichnung nicht unterhalb sondern oberhalb des Titels 
an der Stelle von \Macro{subject} ausgegeben. Der mit \Macro{graduation} 
angegebene Abschluss wird weiterhin unter dem Titel, allerdings in schlankerer 
Schrift gesetzt. Eine etwaige Verwendung des Befehls \Macro{subject} wird in 
diesem Fall ignoriert.
%
\begin{values}
\itemfalse
  Die Ausgabe des Typs der Abschlussarbeit (\Macro{thesis}) selbst sowie des 
  angestrebten Abschlusses (\Macro{graduation}) erfolgt in großen Lettern in 
  \DIN zentral auf der Titelseite.
\itemtrue*
  Der Typ der Abschlussarbeit (\Macro{thesis}) wird oberhalb des Titels in der 
  Betreffzeile gesetzt. Der angestrebte Abschluss (\Macro{graduation}) wird 
  zentral in der schlankeren \Univers ausgegeben.
\end{values}
\end{Declaration}

\begin{Declaration}[v2.02]{\Macro{graduation}\OParameter{Kurzform}\Parameter{Grad}}
\printdeclarationlist%
\index{Titel!Felder}%
%
Mit diesem Befehl wird der angestrebte akademische Grad auf der Titelseite 
ausgegeben. Da dies nur mit einer Abschlussarbeit erreicht werden kann erfolgt 
die Ausgabe nur, wenn entweder \Macro{thesis} oder \Macro{subject} verwendet 
wurde, wobei bei letzterem Befehl im Argument zwingend ein Wert aus 
\autoref{tab:thesis} verwendet werden muss.

Die Option \Option{subjectthesis} hat Einfluss auf die Ausgabe auf der 
Titelseite. Für die Einstellung \Option{subjectthesis}[false] wird der 
Abschuss~-- ähnlich wie 
der Typ der Abschlussarbeit~-- zentral und in relativ großen Lettern gesetzt. 
Für \Option{subjectthesis}[true] erfolgt die Ausgabe kleiner und in weniger 
starken Buchstaben.
\end{Declaration}

\begin{Declaration}{\Macro{supervisor}\Parameter{Name(n)}}
\begin{Declaration}{\Macro{referee}\Parameter{Name(n)}}
\begin{Declaration}{\Macro{advisor}\Parameter{Name(n)}}
\begin{Declaration}{\Macro{professor}\Parameter{Name}}
\printdeclarationlist%
\index{Titel!Felder}%
\index{Betreuer|?}\index{Gutachter|?}\index{Referent|?}%
%
Mit \Macro{supervisor}, \Macro{referee} und \Macro{advisor} werden die Betreuer 
einer Abschlussarbeit beziehungsweise die Gutachter und Fachreferenten einer 
Dissertation angegeben. Zusätzlich kann mit \Macro{professor} der betreuende 
Hochschullehrer beziehungsweise die betreuenden Professoren für studentische 
Arbeiten angegeben werden. Die Angabe mehrerer Person erfolgt wie beim Befehl 
\Macro{author} durch die Trennung mittels \Macro{and}.
\end{Declaration}
\end{Declaration}
\end{Declaration}
\end{Declaration}

\begin{Declaration}{\Macro{date}\OParameter{Ergänzung}\Parameter{Datum}}
\begin{Declaration}{\Macro{defensedate}\Parameter{Verteidigungsdatum}}
\printdeclarationlist%
\index{Titel!Felder}
\index{Datum|?}\index{Datum!Verteidigungsdatum|?}%
%
Mit \Macro{date} kann das Datum angegeben werden. Das optionale Argument 
erlaubt eine zusätzliche Anmerkung, welche nach dem Datum ausgegeben wird. Das 
Datum wird bei normalen Dokumenten direkt nach dem Autor beziehungsweise den 
Autoren ausgegeben. Bei Abschlussarbeiten~-- aktiviert durch die Verwendung von 
\Macro{thesis} oder \Option{subjectthesis}~-- erscheint dieses am Ende der 
Titelseite als Abgabedatum. Außerdem kann in diesem Fall mit  dem Befehl
\Macro{defensedate} das Datum der Verteidigung angegeben werden, wie es 
beispielsweise bei dem Druck von Dissertationen üblich ist.

Sollte das Paket \Package{isodate} geladen sein, so wird die damit eingestellte 
Formatierung des Datums durch den Befehl \Macro{printdate} aus diesem Paket für 
alle Datumsfelder des Dokumentes und folglich auch für die beiden Felder 
\Macro{date} und \Macro{defensedate} verwendet.
\end{Declaration}
\end{Declaration}

\begin{Declaration}{\Macro{extratitle}\Parameter{Schmutztitel}}
\begin{Declaration}{\Macro{titlehead}\Parameter{Kopf}}
\begin{Declaration}{\Macro{title}\Parameter{Titel}}
\begin{Declaration}[%
  v2.01!Bugfix für Schriftstärke bei Verwendung des Untertitels%
]{\Macro{subtitle}\Parameter{Untertitel}}
\begin{Declaration}{\Macro{publishers}\Parameter{Verlag}}
\begin{Declaration}{\Macro{thanks}\Parameter{Fußnote}}
\begin{Declaration}{\Macro{uppertitleback}\Parameter{Titelrückseitenkopf}}
\begin{Declaration}{\Macro{lowertitleback}\Parameter{Titelrückseitenfuß}}
\begin{Declaration}{\Macro{dedication}\Parameter{Widmung}}
\printdeclarationlist%
\index{Titel!Felder}%
%
Diese Befehle entsprechen den in ihrem Verhalten den originalen Pendants der 
\KOMAScript"=Klassen{} und sollen hier der Vollständigkeit halber erwähnt 
werden.

Die Ausgabe des mit \Macro{extratitle} definierten Schmutztitels~-- welcher 
beliebig gestaltet und formatiert werden kann~-- erfolgt als Bestandteil der 
Titelei mit \Macro{maketitle} vor der eigentlichen Titelseite. Mit dem Befehl 
\Macro{titlehead} kann ein zusätzlicher, beliebig formatierbarer Text oberhalb 
der Typisierung und des Titels ausgegeben werden. Da die vertikale Position des 
Dokumenttitels durch das \CD fest vorgegeben ist, kann es~-- im Gegensatz zu 
den \KOMAScript"=Klassen~-- passieren, dass der Kopf des Haupttitels selbst in 
die Kopfzeile ragt. Dies wird durch die \TUDScript-Klassen nicht geprüft und 
muss gegebenenfalls vom Anwender kontrolliert werden.

Die Befehle \Macro{title} und \Macro{subtitle} bedürfen keiner weiteren 
Erklärung. Anzumerken ist, dass sowohl Titel als auch Untertitel normalerweise 
in Majuskeln und \DIN gesetzt werden. Der mit dem Befehl \Macro{publishers} 
definierte Inhalt muss nicht zwingende einen Verlag bezeichnen sondern kann 
auch andere Informationen beinhalten, welche am Ende der Titelseite ausgegeben 
werden sollen.

Fußnoten werden auf dem Titel nicht mit \Macro{footnote}, sondern mit der 
Anweisung \Macro{thanks} erzeugt. Diese dienen in der Regel für Anmerkungen bei 
Titel oder den Autoren. Als Fußnotenzeichen werden dabei Symbole statt Zahlen 
verwendet. Der Befehl \Macro{thanks} kann nur innerhalb des Arguments einer 
der Anweisungen für die Titelseite wie beispielsweise \Macro{author} oder 
\Macro{title} verwendet werden.

\index{Satzspiegel!doppelseitig}%
Im doppelseitigen Druck lässt sich die Rückseite der Haupttitelseite für 
weitere Angaben nutzen. Sowohl den Titelrückseitenkopf als auch den
Titelrückseitenfuß kann der Anwender mit \Macro{uppertitleback} und 
\Macro{lowertitleback} frei gestalten.

Mit \Macro{dedication} lässt eine separate Widmungsseite zentriert und in etwas 
größerer Schrift setzen. Die Rückseite ist~-- wie auch die des Schmutztitels~-- 
grundsätzlich leer. Die Widmung wird mit der restlichen Titelei ausgegeben und 
muss daher vor der Nutzung von \Macro{maketitle} angegeben werden.
\end{Declaration}
\end{Declaration}
\end{Declaration}
\end{Declaration}
\end{Declaration}
\end{Declaration}
\end{Declaration}
\end{Declaration}
\end{Declaration}

\begin{Declaration}[v2.02]{\Font{titlepage}}
\begin{Declaration}[v2.02]{\Font{thesis}}
\printdeclarationlist%
\index{Schriftelemente}
%
Die \TUDScript-Klassen definieren diese neuen Schriftelemente. Dabei wird 
\Font{titlepage} auf der Titelseite für alle Felder verwendet, welche kein 
spezielles Schriftelement verwenden, welches ohnehin durch \KOMAScript{} 
bereitgestellt wird. Das mit \Macro{thesis} angegebene Feld, in welchem der Typ 
einer Abschlussarbeit angegeben wird, nutzt das Schriftelement~\Font{thesis}. 
Wie diese Elemente angepasst werden können, ist in \autoref{sec:fonts:elements} 
zu finden. 
\end{Declaration}
\end{Declaration}
\index{Titel|!)}


\subsection{Die Teileseite}
\label{sec:part}
%
\ChangedAt{%
  v2.02!\PageStyle{plain.tudheadings} wird genutzt!\Macro{partpagestyle}%
}[Implementierung]
Wird für die Teileseiten das Layout des \CDs verwendet, so wird der Seitenstil 
dieser (\Macro{partpagestyle}) auf \PageStyle{plain.tudheadings} gesetzt. 
Möchten Sie stattdessen einen anderen Seitenstil nutzen, so kann dieser mit 
\Macro*{renewcommand*}\PParameter{\Macro{partpagestyle}}\Parameter{Seitenstil} 
angepasst werden.

\begin{Declaration}{\Option{parttitle}[\PBoolean]}[false]%
\printdeclarationlist%
\index{Teileseiten|?}\index{Layout!Teileseiten}%
%
Diese Option ermöglicht es, den mit \Macro{title} gegebenen Titel des 
Dokumentes selbst in großer Schrift auf einer Teileseite auszugeben, die 
Bezeichnung des mit \Macro{part}\Parameter{Bezeichnung} erzeugten Teils wird 
in diesem Fall in kleiner Schrift direkt darunter gesetzt. Diese 
Layout"=Variante findet sich im Handbuch für das \CD der \TnUD. \notudscrartcl
%
\begin{values}
\itemfalse
  Die Bezeichnung des Teils erscheint in großer Schrift auf der Seite, der 
  Titel des Dokumentes gar nicht.
\itemtrue*
  Der Titel wird in großer Auszeichnung auf der Teileseite gesetzt, die 
  Bezeichnung des Teils selber in kleinerer.
\end{values}
\end{Declaration}

\begin{Declaration}[v2.02]{\Font{parttitle}}
\printdeclarationlist%
\index{Schriftelemente}
%
Mit dem Schriftelement~\Font{parttitle} lässt sich~-- bei aktivierter 
\Option{parttitle}-Option~-- die Schrift für die Bezeichnung des Teils 
beeinflussen. In \autoref{sec:fonts:elements} ist zu finden, wie dieses 
angepasst werden kann.
\end{Declaration}


\subsection{Die Kapitelseite}
\begin{Declaration}{\Option{chapterpage}[\PBoolean]}%
  [false][\Option{cd}[color]:true]%
\printdeclarationlist%
\label{sec:chapter}%
\index{Kapitelseiten|?}\index{Layout!Kapitelseiten|?}%
\index{Satzspiegel!doppelseitig}\index{Vakatseiten}%
%
Mit dieser Einstellung kann die Überschrift eines Kapitels separat auf einer 
Seite ausgegeben werden. Der nachfolgende Text wird auf der nächsten 
beziehungsweise bei doppelseitigem Satz und rechts öffnenden Kapiteln%
\footnote{%
  \Option{twoside} und \Option{open}[right], Standard für \Class{tudscrbook}
}
auf der übernächsten Seite ausgegeben. Die in diesem Fall erzeugte Rückseite 
wird in ihrer Ausprägung~-- wie auch Teileseiten~-- durch die Einstellung von 
\Option{cleardoublespecialpage} bestimmt. Beim farbigen Layout ist diese Option 
standardmäßig aktiviert. \notudscrartcl
%
\begin{values}
\itemfalse
  Es gibt keine Sonderstellung von Kapiteln, der nachfolgende Text wird direkt 
  unter der Überschrift respektive nach der mit \Macro{setchapterpreamble} 
  erzeugten Kapitelpräambel auf der gleichen Seite ausgegeben.
\itemtrue*
  Die Kapitelüberschrift und gegebenenfalls die Kapitelpräambel werden auf 
  einer separaten Seite gesetzt. Der folgende Text erscheint auf der nächsten   
  respektive übernächsten Seite, \seealso*{\Option{cleardoublespecialpage}}.
\end{values}
%
\ChangedAt{%
  v2.02!nicht mehr abhängig von \Macro{partpagestyle}!\Macro{chapterpagestyle}
}[Implementierung]
Mit \Macro*{renewcommand*}\PParameter{\Macro{chapterpagestyle}}%
\Parameter{Seitenstil} lässt sich übrigens~-- unabhängig von der Option 
\Option{chapterpage}~-- der Seitenstil von Kapiteln anpassen. Bei der 
Verwendung von separaten Kapitelseiten ist außerdem das Aktivieren der 
\KOMAScript-Option \Option{chapterprefix} empfehlenswert. Damit werden die
Kapitelüberschriften mit einer Vorsatzzeile gesetzt. Falls ein nummeriertes 
Kapitel erzeugt wird, so wird zunächst in einer Zeile \enquote{Kapitel} gefolgt 
von der aktuellen Kapitelnummer ausgegeben, in der nächsten Zeile wird 
anschließend die eigentliche Überschrift in linksbündigem Flattersatz 
ausgegeben. Genaueres hierzu ist in der \KOMAScript"=Dokumentation 
nachzulesen.
\end{Declaration}



\subsection{Vakatseiten}
\index{Vakatseiten}%
Automatisch erzeugte Vakatseiten~-- auch absichtliche Leerseiten genannt~-- 
findet man in Dokumenten mit den aktivierten Optionen \Option{twoside} und 
\Option{open}[right]\footnote{Standard bei \Class{tudscrbook}} beziehungsweise 
\Option{open}[left] beim Beginn von Teilen und Kapiteln. Für diese kann der 
Seitenstil mit der \KOMAScript"=Option \Option{cleardoublepage} eingestellt 
werden.

\ToDo[doc,imp,nxt]{%
  Rückseite bei Kapitelseiten (auch bei Teilen?) im zweiseitigen Satz per 
  Option nicht als Vakatseite setzen. Implementierung von farbigen Seiten davor 
  (\Option*{open}[right]) sowie danach (\Option*{open}[left]). Setzen von 
  speziellen Inhalten auf diesen Seiten äquivalent zu \Macro*{setpartpreamble} 
  bzw. \Macro*{setchapterpreamble}; ggf. temporär umschalten bzw. Warnung bei 
  Konflikt.
  \url{http://latex.wcms-file3.tu-dresden.de/phpBB3/viewtopic.php?f=11&t=396}
}[v2.06]
\begin{Declaration}{\Option{cleardoublespecialpage}[\PSet]}[true]%
\printdeclarationlist%
\index{Teileseiten}\index{Layout!Teileseiten}%
\index{Kapitelseiten}\index{Layout!Kapitelseiten}%
\index{Satzspiegel!doppelseitig}\index{Layout!Rückseiten}%
%
Diese Option wirkt sich lediglich bei aktiviertem doppelseitigem Satz und 
ausschließlich rechts eröffnenden Seiten für Teile beziehungsweise Kapitel
aus.%
\footnote{\Option{twoside} und \Option{open}[right]}
In diesem Fall kann der Stil der darauffolgenden, linken Seite~-- sprich der 
Rückseite~-- beeinflusst werden. Das Normalverhalten sieht vor, dass nach einem 
Teil die Rückseite unabhängig von der Einstellung für \Option{cleardoublepage} 
immer als vollständig leere Seite ohne Kopf"~ oder Fußzeilen gesetzt wird.

Diese Einstellung erlaubt es, dieses Normalverhalten zu deaktivieren und für 
die Seite nach der Teileseite~-- und abhängig von \Option{chapterpage} 
auch nach einem Kapitelanfang auf einer separaten Seite~-- den Seitenstil der 
Option \Option{cleardoublepage} zu übernehmen. Des Weiteren kann auch ein 
anderer, beliebiger, bereits definierter Seitenstil gewählt werden. Außerdem
kann im farbigen Layout die Rückseite in der gleichen Farbe wie die 
Vorderseite von Teil oder Kapitel gesetzt werden. \notudscrartcl
%
\begin{values}
\itemfalse
  Die Rückseiten sind vollständig leere Seiten, unabhängig von Option
  \Option{cleardoublepage}.
\itemtrue*
  Der Seitenstil der Rückseite von Teilen und gegebenenfalls Kapiteln 
  entspricht der Einstellung von \Option{cleardoublepage} für Vakatseiten.
\item[current]
  Es wird der aktuell definierte Seitenstil (\Macro{pagestyle}) für die 
  erzeugte Rückseite verwendet.
\itemvalues[\PName{Seitenstil}:]
  Mit der Angabe von \Option{cleardoublespecialpage}[\PName{Seitenstil}] 
  kann ein beliebiger, bereits definierter Seitenstil für die Rückseite nach 
  Teilen und Kapiteln verwendet werden.
\item[color]
  Im farbigen Layout ist auch die Rückseite von Teilen und Kapiteln farbig, 
  \see*{\Option{clearcolor}}.
\ToDo[imp,nxt]{\PValue{nocolor}}[v2.06]
\end{values}
\end{Declaration}

\ToDo[doc,imp,nxt]{%
  \Option{clearcolor} in \Option{cleardoublespecialpage}%
}[v2.06]
\begin{Declaration}{\Option{clearcolor}[\PBoolean]}[false]%
\printdeclarationlist%
\index{Titel}\index{Layout!Titel}%
\index{Teileseiten}\index{Layout!Teileseiten}%
\index{Kapitelseiten}\index{Layout!Kapitelseiten}%
\index{Satzspiegel!doppelseitig}\index{Vakatseiten}%
%
Sollten beim farbigen Layout die Optionen \Option{twoside} sowie auch
\Option{open}[right] gesetzt sein, so werden beim Aktivieren dieser Option die 
Rückseiten von Teilen~-- und je nach Einstellung von \Option{chapterpage} 
gegebenenfalls auch von Kapiteln~-- farbig gesetzt.%
\footnote{%
  Dies führt bei der Ausgabe zu farbigen Blättern (Vorder- und Rückseite) der 
  entsprechenden Elemente des Layouts.
}
Die Option wirkt sich ebenfalls auf die Rückseite des Titels aus.%
\footnote{%
  \see*{\Macro{uppertitleback} und \Macro{lowertitleback}} der 
  \KOMAScript"=Dokumentation (\scrguide*)
}
Der Stil dieser zusätzlich eingefügten Rückseiten ist abhängig von der Option
\Option{cleardoublespecialpage}.
%
\begin{values}
\itemfalse
  Es werden weiße Rückseiten bei Titel, Teilen und gegebenenfalls Kapiteln 
  erzeugt.
\itemtrue*
  Die rückwärtigen Seiten der genannten Elemente des Layouts sind farbig.
\end{values}
\end{Declaration}




\subsection{Der Satzspiegel}
\begin{Declaration}[v2.03]{\Option{cdgeometry}[\PSet]}[true]%
\printdeclarationlist%
\index{Satzspiegel|(}\index{Layout!Satzspiegel|(}%
\index{Seitenstil}\index{Layout!Seitenstil}%
\index{Satzspiegel!doppelseitig}%
\index{Layout!Seitenränder}%
%
\ToDo[imp]{automatisch auf zweiseitigen Satz umschalten?}[v2.05]
\ToDo[doc]{Hinweis auf die Verwendung von open=left}[v2.05]
Diese Option ist für die Aufteilung beziehungsweise die Berechnung des 
Satzspiegels verantwortlich. Das Maß der Seitenränder ist im \CD fest 
vorgegeben und wird standardmäßig von den \TUDScript-Klassen eingehalten. 
Allerdings lassen sich die Seitenränder anpassen, um beispielsweise einen 
vernünftigen doppelseitigen Satz zu ermöglichen.%
\footnote{Hierbei sollte der innere Rand schmaler als der äußere sein}
Des Weiteren besteht die Möglichkeit, auf das Standardverhalten von 
\KOMAScript{} zurückzufallen und die Satzspiegelberechnung durch das Paket
\Package{typearea} vornehmen zu lassen. Hier hat insbesondere die Klassenoption 
\Option{DIV}[\PSet] maßgeblichen Einfluss auf den Satzspiegel. Mehr dazu ist in 
der Dokumentation von \KOMAScript{} zu finden.
%
\begin{values}
\itemfalse
  Die Satzspiegelberechnung erfolgt via \Package{typearea}, die Vorgaben des 
  \CDs bezüglich der Seitenränder werden ignoriert.
\itemtrue*[asymmetric/cd]
  Die Seitenränder werden im asymmetrischen Stil des \CDs fest definiert und 
  auch für den doppelseitigen Satz (\Option{twoside}[true]) genutzt.%
  \footnote{links: 30\,mm, rechts: 20\,mm, oben: 25\,mm, unten: 30\,mm}
\item[symmetric/centred/centered]
  Der Satzspiegel wird im einseitigen sowie doppelseitigen Satz auf der Seite 
  zentriert.%
  \footnote{links: 25\,mm, rechts: 25\,mm, oben: 25\,mm, unten: 30\,mm}
\item[twoside/balanced]
  Im einseitigen Layout ist diese Einstellung zu \Option{cdgeometry}[symmetric] 
  identisch. Beim doppelseitigen Satz wird der Satzspiegel derart verändert, 
  dass die Ränder der inneren Seiten schmaler sind als die der äußeren.%
  \footnote{innen: 20\,mm, außen: 30\,mm, oben: 25\,mm, unten: 30\,mm}
  \Attention{%
    Der so erzeugte Satzspiegel ist jedoch nicht sehr vorteilhaft. Es ist zu 
    beachten, dass dabei das Logo der \TnUD sehr nah am inneren Seitenrand 
    des Dokumentes gesetzt wird, folglich insbesondere auf rechten respektive 
    ungeraden Seiten sehr weit an den Blattrand rückt.
  }%
\end{values}
%
Für die Festlegung der Seitenränder wird das Paket \Package{geometry} genutzt. 
Ist \Option{cdgeometry}[false] gewählt, erfolgt die Berechnung des Satzspiegels 
durch \Package{typearea}. Die damit berechneten Werte werden anschließend an 
\Package{geometry} weitergereicht und durch dieses umgesetzt.
\end{Declaration}

\begin{Declaration}[v2.03]{\Option{extrabottommargin}[\PName{Höhe}]}[0pt]%
\printdeclarationlist%
\index{Fußzeile|?}
%
Mit dieser Option kann die Größe des unteren Seitenrandes angepasst werden, 
wenn der Satzspiegel des \CDs (\Option{cdgeometry}[true/symmetric/balanced]) 
verwendet wird. Insbesondere für den Fall, dass im Fußbereich der Seiten im 
Stil \PageStyle{tudheadings} entweder mit \Macro{footlogo} Drittlogos verwendet 
werden und diese über das optionale Argument oder via \Length{footlogoheight} 
über die Standardhöhe hinaus vergrößert wurden oder mit \Macro{footcontent} ein 
übergroßer Inhalt angegeben wurde, kann dieser unter Umständen etwas zu klein 
sein. Mit der Option \Option{extrabottommargin} wird der Fußbereich durch 
positive Werte vergrößert, negative Werte verkleinern diesen entsprechend. 

Alternativ zu \Option{extrabottommargin} kann auch die Option \Option{cdfoot} 
mit einer Längenangabe verwendet werden. Dabei spielt es für beide Optionen 
keine Rolle, ob eine \hologo{LaTeX}-Länge, ein \hologo{TeX}-Abstand oder eine 
\hologo{TeX}-Ausdehnung als Länge bei der Wertzuweisung verwendet wird.
\end{Declaration}

\minisec{Kopf"~ und Fußzeile im Zusammenspiel mit dem Satzspiegel}
\index{Kopfzeile|?}\index{Layout!Kopfzeile}%
\index{Fußzeile|?}\index{Layout!Fußzeile}%
Da im \CD nicht festgelegt ist, wie die Gestaltung der Kopf"~ und Fußzeilen in 
einer wissenschaftlichen Arbeit auszuführen ist, bleibt dem Nutzer dafür eine 
gewisse Freiheit. Dafür sollte idealerweise das zu \KOMAScript{} gehörige Paket 
\Package{scrlayer-scrpage} genutzt werden. 

In der Dokumentation zu \Package{typearea} wird auch darauf eingegangen, wann 
Kopf"~ und Fußzeile bei der Satzspiegelkonstruktion entweder dem Rand oder dem 
Textkörper zugeschlagen werden sollten. Dies sollte bei der Erstellung eigener 
Kopf"~ und Fußzeilen beachtet werden. Die Einstellung dafür erfolgt mit den 
beiden \KOMAScript"=Optionen \Option{headinclude}[\PBoolean] sowie 
\Option{footinclude}[\PBoolean]. Diese können~-- unabhängig von der gewählten 
Einstellung zur Satzspiegelgestaltung für die Option \Option{cdgeometry}~-- 
verwendet werden.

\minisec{Bindekorrektur}
\index{Bindekorrektur|!}\index{Layout!Bindekorrektur}%
%
Zu erwähnen im Zusammenhang mit Seitenrändern und Satzspiegel ist die durch 
\Package{typearea} angebotene Option \Option{BCOR}[\PName{Länge}], mit der bei 
der Satzspiegelberechnung ein Heftrand beziehungsweise eine Bindekorrektur 
berücksichtigt wird. Die \TUDScript-Klassen reichen diesen Wert auch an 
\Package{geometry} weiter, so dass der Benutzer unabhängig von der Auswahl zur 
Satzspiegelgestaltung diese Option nutzen kann. So kann beispielsweise eine 
Bindekorrektur von \unit[5]{mm} mit der Klassenoption \Option{BCOR}[5mm] 
gesetzt werden.

Eine Anpassung der Bindekorrektur hat natürlich \emph{immer} eine Änderung der 
verfügbaren Breite des Textbereichs zur Folge hat und führt somit zwingend zu 
einer Anpassung des Satzspiegels. Da die Bindekorrektur jedoch abhängig von der 
Höhe des Buchblocks gewählt werden sollte, welche letztendlich erst mit dem 
Druck des fertiggestellten Dokumentes bestimmt werden kann, muss diese zu 
Beginn abgeschätzt werden.
%
\begin{Example}
Als Faustregel gilt, dass die erforderliche Bindekorrektur in etwa der halben 
Höhe des Buchblocks entsprechen sollte. Dessen Höhe wiederum ist abhängig von 
der Anzahl der Seiten sowie der Dichte des verwendeten Papiers. Wird normales 
Papier mit einer Dichte von \unit[80]{g/m²} verwendet, so entsprechen 100~Blatt 
in etwa einer Höhe von \unit[10]{mm}, bei \unit[100]{g/m²} ca. \unit[12]{mm}. 
Dementsprechend wäre eine Bindekorrektur von \Option{BCOR}[5mm] beziehungsweise 
\Option{BCOR}[6mm] bei diesem Beispiel zu wählen.
\end{Example}
\index{Satzspiegel|)}\index{Layout!Satzspiegel|)}%


\subsection{Verwendung von Schriftelementen}
\label{sec:fonts:elements}%
\index{Schriftelemente|!}
Von \TUDScript werden weitere Schriftelemente~-- in Ergänzung zu den bereits
durch \KOMAScript{} bereitgestellten~-- definiert. Dies sind \Font{titlepage}, 
\Font{thesis}, \Font{tudheadings} sowie \Font{parttitle}. Sowohl die bereits 
durch \KOMAScript{} definierten als auch alle hier genannten Schriftelemente 
und später erläuterten sollten im Bedarfsfall durch den Anwender über den 
Befehl \Macro{addtokomafont}\Parameter{Schriftelement}\Parameter{Einstellungen}
angepasst werden. Mehr dazu ist im \KOMAScript"=Handbuch \scrguide im 
Abschnitt \emph{Textauszeichnungen} zu finden.


\subsection{Die Farben des \CDs}
\index{Farben}%
% 
Zur Verwendung der Farben des \CDs wird das Paket \Package{tudscrcolor} 
genutzt. Falls dieses nicht in der Präambel geladen wird~-- um beispielsweise 
zusätzliche Optionen aufzurufen~-- binden die \TUDScript"=Klassen dieses 
automatisch ein. Detaillierte Informationen sind in der Dokumentation von 
\Package{tudscrcolor}'full' zu finden.



\section{Zusätzliche Optionen und Erweiterungen}
\ChangedAt*{%
  v2.03!Bugfix für \abstractname{,} \confirmationname{} und \blockingname{} 
    bei Seitenstil und Kolumnentitel%
}%
Neben den Befehlen für die Anpassung des Layouts an das \CD der \TnUD stellen 
die \TUDScript-Klassen weitere Befehle und Umgebungen zur Verfügung, um die 
Anwendung insbesondere für wissenschaftliche Arbeiten zu erleichtern.


\subsection{Zusammenfassung/Kurzfassung}
\ToDo[imp]{vertikale Zentrierung funktioniert nicht richtig}[v2.05]
\begin{Declaration}[%
  v2.02!Wert \PValue{double} mit \PValue{multi} ersetzt,%
  v2.02!Wert \PValue{tocleveldown} neu,%
  v2.02!Wert \PValue{markboth} neu,%
  v2.04!Wert \PValue{tocmultiple} neu%
]{\Option{abstract}[\PSet]}%
\printdeclarationlist%
\index{Zusammenfassung|!(}%
\index{Zweispaltensatz}%
%
Diese Option wird bereits durch \KOMAScript{} für die Klassen \Class{scrartcl} 
und \Class{scrreprt} standardmäßig bereitgestellt. Für die Klasse 
\Class{scrbook} geschieht dies nicht. Dazu heißt es im Handbuch:
%
\begin{quoting}
Bei Büchern wird in der Regel eine andere Art der Zusammenfassung verwendet. 
Dort setzt man ein entsprechendes Kapitel an den Anfang oder Ende des Werks. 
Oft wird diese Zusammenfassung entweder mit der Einleitung oder einem weiteren 
Ausblick verknüpft. Daher gibt es bei \Class{scrbook} generell keine 
\Environment{abstract}"=Umgebung. Bei Berichten im weiteren Sinne, etwa einer 
Studien- oder Diplomarbeit, ist ebenfalls eine Zusammenfassung in dieser Form 
zu empfehlen.
\end{quoting}
%
Durch die \TUDScript-Klassen wird die \Option{abstract}"=Option erweitert. 
Neben den Auswahlmöglichkeit, welche bereits \KOMAScript{} für die Klassen 
\Class{tudscrartcl} und \Class{tudscrreprt} anbietet, kann die Überschrift für 
die Zusammenfassung außerdem in Gestalt eines \sectionautorefname{}s oder für 
\Class{tudscrreprt} und \Class{tudscrbook} in der Form eines 
\chapterautorefname{}s ausgegeben werden.
%
\begin{values}
\itemfalse[][nur für \Class{tudscrartcl} und \Class{tudscrreprt} verfügbar]
  Es wird keine Überschrift für die \Environment{abstract}"=Umgebung ausgegeben.
\itemtrue*[][nur für \Class{tudscrartcl} und \Class{tudscrreprt} verfügbar]
  Wie bei den \KOMAScript"=Klassen wird eine zentrierte Überschrift mit dem 
  Bezeichner \Term{abstractname} vor der eigentlichen Zusammenfassung gesetzt.
\item[section/addsec]
  Die Überschrift (\Term{abstractname}) verwendet den Gliederungsbefehl 
  \Macro{section}.
\item[chapter/addchap][%
    (Säumniswert für \Class{tudscrbook})
    nur für \Class{tudscrreprt} und \Class{tudscrbook} verfügbar%
  ]
  Es wird der Befehl \Macro{chapter} für das Setzen der Überschrift 
  (\Term{abstractname}) genutzt. 
\item[heading]
  Es wird die höchstmögliche Gliederungsebene verwendet. Für 
  \Class{tudscrartcl} entspricht dies \Option{abstract}[section], bei 
  \Class{tudscrreprt} und \Class{tudscrbook} \Option{abstract}[chapter].
\end{values}
%
Abhängig von der gewählten Gliederungsebene der Überschrift wird das Verhalten 
für das Setzen eines Eintrages ins Inhaltsverzeichnis festgelegt. Ohne oder mit 
zentrierter Überschrift wird per Voreinstellung kein Eintrag erzeugt. Wird die 
Überschrift jedoch in Form einer Gliederungsebene gewählt, so erscheint die 
Zusammenfassung für gewöhnlich im Inhaltsverzeichnis auf der obersten Ebene. 
Das voreingestellte Verhalten für die Einträge ins Inhaltsverzeichnis kann 
jederzeit mit folgenden Werten durch den Anwender überschrieben werden.
%
\begin{values}
\item[notoc/nottotoc]
  Die Zusammenfassung wird definitiv nicht ins Inhaltsverzeichnis eingetragen.
\item[toc/totoc]
  Es wird ein nicht nummerierten Eintrag im Inhaltsverzeichnis auf der obersten 
  Gliederungsebene der verwendeten Dokumentklasse für die Zusammenfassung 
  gesetzt.
\item[leveldown/tocleveldown/totocleveldown]
  \ChangedAt{v2.02}
  Der Inhaltsverzeichniseintrag wird eine Gliederungsebene unterhalb der 
  obersten erzeugt.
\item[tocmultiple/totocmultiple/tocaggregate/totocaggregate]
  \ChangedAt{v2.04}
  Es wird ein \emph{einziger} Inhaltsverzeichniseintrag für \emph{alle} 
  Zusammenfassungen erstellt.
\end{values}
%
\ChangedAt{v2.02}
Außerdem kann das Verhalten für die Kolumnentitel durch den Nutzer beeinflusst 
werden. Normalerweise werden diese nur gesetzt, wenn automatische Kolumnentitel 
aktiviert sind (\Option{automark}) und sind von der Gliederungsebene der 
Überschrift abhängig. Werden manuelle Kolumnentitel genutzt, müssen diese auch 
für die Zusammenfassung manuell gesetzt werden. Mit \Option{abstract}[markboth] 
lässt sich das Setzen der Kolumnentitel jedoch forcieren.
%
\begin{values}
\item[markboth]
  Unabhängig von der Verwendung manueller oder automatischer Kolumnentitel 
  werden diese auf rechten sowie linken Seiten mit \Term{abstractname} gesetzt.
\item[nomarkboth]
  Die Einstellung für manuelle oder automatische Kolumnentitel werden beachtet 
  und abhängig von der verwendeten Gliederungsebene der Überschrift gesetzt.
\end{values}
%
Mit dem optionalen Parameter \Key{\Environment{abstract}}{markboth} der 
\Environment{abstract}"=Umgebung kann der Kolumnentitel mit einem beliebigen 
Inhalt gesetzt werden.

Häufig wird für Abschlussarbeiten verlangt, neben der deutschsprachigen auch 
noch eine englischsprachige Zusammenfassung zu verfassen. Mit der Einstellung 
\Option{abstract}[multiple] lassen sich mehrere Zusammenfassungen auf einer 
Seite ausgeben~-- sofern genügend Platz vorhanden ist. Außerdem kann die 
standardmäßige vertikale Zentrierung der \Environment{abstract}"=Umgebung 
auf einer Seite unterdrückt werden. Diese Einstellungen zur Positionierung der 
Zusammenfassungen innerhalb der \Environment{abstract}"=Umgebung werden nur 
wirksam, wenn eine Titelseite (\Option{titlepage}[true]) und \emph{keine} 
Überschriften in Form von Kapiteln (\Option{abstract}[chapter]) verwendet 
werden.
%
\begin{values}
\item[single/one/simple]
  Jede Zusammenfassung wird auf einer eigenen Seite
  beziehungsweise im zweispaltigen Satz in einer neuen Spalte ausgegeben.
\item[multiple/multi/all/aggregate]
  \ChangedAt{v2.02}
  Zusammenfassungen, welche mit \Macro{nextabstract} getrennt wurden, werden 
  direkt nacheinander auf der gleichen Seite ausgegeben, wenn ausreichend Platz 
  auf dieser vorhanden sein sollte. Ist die Option \Option{twocolumn} aktiviert,
  erfolgt die Ausgabe aller Zusammenfassungen ohne Spaltenumbruch.
\item[fil/fill/vfil/vfill]
  Alle Zusammenfassungen auf einer Ausgabeseite werden vertikal zentriert. Für 
  den zweispaltigen Satz mit \Option{twocolumn} steht diese Einstellung nicht 
  zur Verfügung.
\item[nofil/nofill/novfil/novfill]
  Die Ausgabe erfolgt wie im normalen Fließtext auch.
\end{values}
\end{Declaration}

\begin{Declaration}[%
  v2.02!\Macro{nextabstract} zur Trennung der einzelnen Teile%
]{\Environment{abstract}[\OLParameter{Sprache}]}
\begin{Declaration}[v2.02]{\Macro{nextabstract}\OLParameter{Sprache}}
\begin{Declaration}{\Key{\Environment{abstract}}{language}[\PName{Sprache}]}
\begin{Declaration}[v2.02]{%
  \Key{\Environment{abstract}}{markboth}[\PBName{Kolumnentitel}]%
}
\begin{Declaration}[v2.02]{%
  \Key{\Environment{abstract}}{pagestyle}[\PName{Seitenstil}]%
}
\begin{Declaration}{\Key{\Environment{abstract}}{columns}[\PName{Anzahl}]}
\begin{Declaration}{\Key{\Environment{abstract}}{option}[\PSet]}{%
  \see*{\Option{abstract}'ppage'}%
}
\printdeclarationlist%
\index{Zweispaltensatz}%
%
Die \Environment{abstract}-Umgebung dient speziell für die Ausgabe einer 
Zusammenfassung, entweder zu Beginn eines Dokumentes oder beispielsweise vor 
einem Teil oder Kapitel. Wird ein Titelkopf (\Option{titlepage}[false]) und 
keine Titelseite verwendet, so wird für den Fall, dass die Zusammenfassung 
\emph{nicht} mit der Überschrift einer Gliederungsebene gesetzt wird, diese wie 
bei den \KOMAScript"=Klassen in einer \Environment{quotation}"=Umgebung 
gesetzt, um diese vom restlichen Fließtext abzuheben. Diese hat jedoch den 
Nachteil, dass in besagter Umgebung die Option \Option{parskip} nicht beachtet 
wird. Um dieses Problem zu beheben, kann das Paket \Package{quoting} geladen 
werden, wodurch stattdessen die Umgebung \Environment{quoting} verwendet wird.

Mit der zuvor erläuterten Option \Option{abstract} kann eingestellt werden, in 
welcher Gestalt die Zusammenfassung ausgegeben werden soll. Des Weiteren lässt 
sich jede \Environment{abstract}"=Umgebung individuell über weitere Parameter 
als optionales Argument anpassen. Damit lassen sich gegebenenfalls für eine 
bestimmte \Environment{abstract}"=Umgebung die globalen Einstellungen 
der Option \Option{abstract} lokal ändern und gezielt anpassen. 

Wird das Paket \Package{babel} durch den Anwender geladen, kann mit dem 
optionalen Parameter \Key{\Environment{abstract}}{language}[\PName{Sprache}] 
die Sprache innerhalb der \Environment{abstract}"=Umgebung geändert werden. 
Dafür muss die gewünschte Sprache bereits mit dem Laden von \Package{babel} 
entweder als Paketoption oder besser noch als Klassenoption angegeben worden 
sein. Dadurch werden innerhalb der Umgebung die Bezeichnung \Term{abstractname} 
und die Trennungsmuster sprachspezifisch angepasst. Die gewünschte Sprache kann 
auch ohne die Verwendung des Parameters \Key{\Environment{abstract}}{language} 
direkt als optionales Argument übergeben werden.

\ChangedAt{v2.02}
Mit \Key{\Environment{abstract}}{markboth} können die gesetzten Kolumnentitel 
beeinflusst werden. Wird \Key{\Environment{abstract}}{markboth}[false] 
angegeben, werden automatische respektive manuelle Kolumnentitel verwendet. Die 
Einstellung \Key{\Environment{abstract}}{markboth}[true] wiederum setzt diese 
für linke und rechte Seiten auf \Term{abstractname}. Außerdem lässt sich der 
Kolumnentitel mit \Key{\Environment{abstract}}{markboth}[\PName{Kolumnentitel}] 
auch direkt festlegen. So können die Kolumnen beispielsweise mit der Verwendung 
von \Key{\Environment{abstract}}{markboth}[\PParameter{}] auch gelöscht werden. 
Sollte \Key{\Environment{abstract}}{markboth} aktiviert werden, so wird in der
Umgebung automatisch der Seitenstil \PageStyle{headings} genutzt~-- falls eine 
Titelseite und kein Titelkopf (\Option{titlepage}[true]) verwendet wird. Mit 
dem Parameter \Key{\Environment{abstract}}{pagestyle} kann dieser auch manuell 
angegeben werden, wobei die \PageStyle{tudheadings}"=Seitenstile ebenfalls 
unterstützt werden.

Wurde das Paket \Package{multicol} geladen, kann mit dem Parameter 
\Key{\Environment{abstract}}{columns}[\PName{Anzahl}] die Zusammenfassung 
mehrspaltig gesetzt werden. Dem Parameter \Key{\Environment{abstract}}{option} 
können alle gültigen, bereits erläuterten Werte der Option \Option{abstract} 
übergeben werden. Die damit gemachten Einstellungen wirken sich~-- im Gegensatz 
zur Angabe als Klassenoption oder über die Variante der späten Optionenwahl%
\footnote{%
  \Macro{TUDoption}\PParameter{abstract}\Parameter{Einstellung} oder
  \Macro{TUDoptions}\PParameter{abstract=\PName{Einstellung}}
}~-- lediglich lokal auf die verwendete \Environment{abstract}"=Umgebung aus.

\ChangedAt{v2.02}
Sollen mehrere Zusammenfassungen im gleichen Stil erzeugt und die Einstellungen 
der Option \Option{abstract}[simple/multiple/fill/nofill] beachtet werden, so 
ist die \Environment{abstract}"=Umgebung nur einmal zu verwenden. Innerhalb 
dieser müssen die einzelnen Zusammenfassungen mit \Macro{nextabstract} 
voneinander getrennt werden. Der Befehl akzeptiert dabei im optionalen Argument 
alle Parameter, die auch von der \Environment{abstract}"=Umgebung selbst 
unterstützt werden. Das Minimalbeispiel in \fullref{sec:exmpl:dissertation} 
zeigt hierfür das notwendige Vorgehen.

Wird die \Environment{abstract}"=Umgebung innerhalb des Argumentes der Befehle 
\Macro{setpartpreamble} beziehungsweise \Macro{setchapterpreamble} verwendet, 
so wird die Überschrift~-- im Fall, dass nicht \Option{abstract}[false] gewählt 
ist~-- \emph{immer} in Textgröße und zentriert gesetzt.
\end{Declaration}
\end{Declaration}
\end{Declaration}
\end{Declaration}
\end{Declaration}
\end{Declaration}
\end{Declaration}

\minisec{Umbenennung der \abstractname}
Mit dem \KOMAScript-Befehl \Macro{renewcaptionname} kann der Bezeichner~-- 
sprich der Wortlaut~-- der für die \Environment{abstract}-Umgebung verwendeten 
Überschrift verändert werden. Mehr dazu ist in \autoref{sec:localization} zu 
finden.
%
\begin{Example}
Die Überschrift der \Environment{abstract}-Umgebung soll für die Sprache 
\PValue{ngerman} von \enquote{\abstractname} in \enquote{Kurzfassung} umbenannt 
werden. Der Befehl \Macro{renewcaptionname} erwartet die drei obligatorischen 
Argumente \Parameter{Sprache}\Parameter{Makro}\Parameter{Inhalt}:
\begin{Code}[escapechar=§]
\renewcaptionname{ngerman}{\abstractname}{Kurzfassung}
\end{Code}
\end{Example}
%
\index{Zusammenfassung|!)}%


\subsection{Selbstständigkeitserklärung und Sperrvermerk}
\begin{Declaration}[%
  v2.02!Wert \PValue{double} mit \PValue{multi} ersetzt,%
  v2.02!Wert \PValue{tocleveldown} neu,%
  v2.02!Wert \PValue{markboth} neu,%
  v2.04!Wert \PValue{tocmultiple} neu%
]{\Option{declaration}[\PSet]}[true]%
\printdeclarationlist%
\index{Selbstständigkeitserklärung|!}\index{Sperrvermerk|!}%
%
Mit \Option{declaration} kann äquivalent zur Option \Option{abstract} die 
Gestaltung von Selbstständigkeitserklärung und Sperrvermerk angepasst werden.
Zur Ausgabe der Erklärungen werden die Umgebung \Environment{declarations} 
sowie die Befehle \Macro{declaration} beziehungsweise \Macro{confirmation} und 
\Macro{blocking} bereitgestellt. 

Die beiden Optionen \Option{abstract} und \Option{declaration} ähneln sich sehr 
stark. Alle möglichen Wertzuweisungen für \Option{declaration} wurden bereits 
bei der Beschreibung von \Option{abstract} ausführlich erläutert. Deshalb 
geschieht dies hier in einer etwas kürzeren Ausführung. Sollte Ihnen eine 
Erläuterung etwas dürftig erscheinen, so hilft mit Sicherheit ein Blick zur 
Erklärung der Option \Option{abstract}'full'.

Die möglichen Werte für die Gestaltung der Überschrift werden nachfolgend 
genannt. Im Gegensatz zur Option \Option{abstract} stehen die Einstellungen 
\Option{declaration}[true/false] auch für die Klasse \Class{tudscrbook} zur 
Verfügung.
%
\begin{values}
\itemfalse
  Es wird keine Überschrift über den Erklärungen selbst ausgegeben.
\itemtrue*
  Eine zentrierte Überschrift mit dem Bezeichner \Term{confirmationname} vor 
  der Selbstständigkeitserklärung beziehungsweise \Term{blockingname} vor dem 
  Sperrvermerk wird gesetzt. 
\item[section/addsec]
  Die Überschrift verwendet den Gliederungsbefehl \Macro{section}.
\item[chapter/addchap][%
    (Säumniswert für \Class{tudscrbook})
    nur für \Class{tudscrreprt} und \Class{tudscrbook} verfügbar%
  ]
  Es wird der Befehl \Macro{chapter} für das Setzen der Überschrift genutzt. 
\item[heading]
  Es wird die höchstmögliche Gliederungsebene verwendet. Für 
  \Class{tudscrartcl} entspricht dies \Option{declaration}[section], bei 
  \Class{tudscrreprt} und \Class{tudscrbook} \Option{declaration}[chapter].
\end{values}
%
Abhängig von der gewählten Gliederungsebene der Überschrift wird das Verhalten 
für das Setzen eines Eintrages ins Inhaltsverzeichnis festgelegt. Normalerweise 
wird nur für Überschriften in Form einer Gliederungsebene ein Eintrag der 
Erklärung ins Inhaltsverzeichnis erstellt, für \Option{declaration}[true/false] 
geschieht dies standardmäßig nicht. Das voreingestellte Verhalten kann mit 
folgenden Werten überschrieben werden.
%
\begin{values}
\item[notoc/nottotoc]
  Die Erklärung wird definitiv nicht ins Inhaltsverzeichnis eingetragen.
\item[toc/totoc]
  Unabhängig von der Wahl der Überschrift erhält jede Erklärung einen nicht
  nummerierten Eintrag im Inhaltsverzeichnis auf der obersten Gliederungsebene  
  der aktuell gerade verwendeten Dokumentklasse. 
\item[leveldown/tocleveldown/totocleveldown]
  \ChangedAt{v2.02}
  Der Inhaltsverzeichniseintrag wird eine Gliederungsebene unterhalb der 
  obersten erzeugt.
\item[tocmultiple/totocmultiple/tocaggregate/totocaggregate]
  \ChangedAt{v2.04}
  Es wird ein \emph{einziger} Inhaltsverzeichniseintrag für \emph{alle} 
  Erklärungen erstellt.
\end{values}
%
\ChangedAt{v2.02}
Normalerweise werden die automatischen Kolumnentitel in Abhängigkeit von der 
Gliederungsebene der Überschrift gesetzt, falls diese denn aktiviert sind 
(\Option{automark}). Werden manuelle Kolumnentitel genutzt, müssen diese auch 
für die Erklärungen manuell gesetzt werden. Mit \Option{declaration}[markboth] 
lässt sich außerdem das Setzen der Kolumnentitel auf linken und rechten Seiten 
forcieren, wobei hierfür der Titel der Überschrift genutzt wird.
%
\begin{values}
\item[markboth]
  Unabhängig von der Verwendung manueller oder automatischer Kolumnentitel 
  werden diese auf rechten sowie linken Seiten mit den Bezeichnern 
  \Term{confirmationname} beziehungsweise \Term{blockingname} gesetzt.
\item[nomarkboth]
  Die Einstellung für manuelle oder automatische Kolumnentitel werden beachtet.
\end{values}
%
Für \Macro{declaration} respektive \Macro{confirmation} und \Macro{blocking} 
sowie die \Environment{declaration}"=Umgebung lässt sich mit dem Parameter 
\Key{\Environment{declaration}}{markboth} ein beliebiger Kolumnentitel setzen. 

Die folgenden Einstellungen zur Positionierung der Erklärungen haben lediglich 
Auswirkungen, wenn die Überschrift der Erklärung \emph{nicht} im Form eines 
Kapitels ausgegeben und eine Titelseite (\Option{titlepage}[true]) verwendet 
wird.
%
\begin{values}
\item[single/one/simple]
  Jede Erklärung wird auf einer separaten Seite
  beziehungsweise im zweispaltigen Satz in einer neuen Spalte ausgegeben.
\item[multiple/multi/all/aggregate]
  \ChangedAt{v2.02}
  Erklärungen, welche in der \Environment{declarations}"=Umgebung mit den 
  Befehlen \Macro{confirmation}, \Macro{blocking} und \Macro{declaration} oder 
  außerhalb dieser mit \Macro{declaration} gesetzt wurden, werden direkt 
  nacheinander auf der gleichen Seite ausgegeben, wenn ausreichend Platz auf 
  dieser vorhanden sein sollte. Ist die Option \Option{twocolumn} aktiviert, 
  erfolgt die Ausgabe aller Erklärungen ohne Spaltenumbruch.
\item[fil/fill/vfil/vfill]
  Alle Erklärungen auf einer Ausgabeseite werden vertikal zentriert. Für 
  den zweispaltigen Satz mit \Option{twocolumn} steht diese Einstellung nicht 
  zur Verfügung.
\item[nofil/nofill/novfil/novfill]
  Die Ausgabe erfolgt wie im normalen Fließtext auch.
\end{values}
\end{Declaration}

\begin{Declaration}[v2.02]{\Environment{declarations}[\OLParameter{Sprache}]}
\begin{Declaration}[v2.04]{%
  \Macro{nextdeclaration}%
  \OLParameter{Sprache}\Parameter{Überschrift}\Parameter{Erklärung}
}
\begin{Declaration}{\Key{\Environment{declarations}}{language}[\PName{Sprache}]}
\begin{Declaration}[v2.02]{%
  \Key{\Environment{declarations}}{markboth}[\PBName{Kolumnentitel}]%
}
\begin{Declaration}[v2.02]{%
  \Key{\Environment{declarations}}{pagestyle}[\PName{Seitenstil}]%
}
\begin{Declaration}[v2.02]{%
  \Key{\Environment{declarations}}{columns}[\PName{Anzahl}]%
}
\begin{Declaration}{\Key{\Environment{declarations}}{option}[\PSet]}
\begin{Declaration}{%
  \Key{\Environment{declarations}}{supporter}[\PName{Unterstützer}]
}
\begin{Declaration}{\Key{\Environment{declarations}}{place}[\PName{Ort}]}
\begin{Declaration}{\Key{\Environment{declarations}}{closing}[\PName{Ende}]}
\begin{Declaration}{\Key{\Environment{declarations}}{company}[\PName{Firma}]}
\printdeclarationlist%
\index{Selbstständigkeitserklärung}\index{Sperrvermerk}%
%
Für Selbstständigkeitserklärung und Sperrvermerk sollten im einfachsten Fall 
die Befehle \Macro{declaration} beziehungsweise \Macro{confirmation} und 
\Macro{blocking} verwendet werden. Sobald diese jedoch in anderer Reihenfolge,  
mehrfacher Ausführung, unterschiedlichen Sprachen oder durch zusätzliche  
Erklärungen ergänzt werden, so bietet die \Environment{declarations}-Umgebung 
die notwendigen Freiheiten.

Innerhalb dieser Umgebung können Selbstständigkeitserklärung und Sperrvermerk 
mit dem Befehl \Macro{declaration} direkt nacheinander folgend beziehungsweise 
mit \Macro{confirmation} und \Macro{blocking} auch separat ausgegeben werden. 
Dies kann in beliebiger Reihenfolge und auch mehrmals geschehen, um diese 
beispielsweise mehrsprachig zu setzen.
\ChangedAt{v2.04} Des Weiteren gibt es mit \Macro{nextdeclaration} die 
Möglichkeit, eine Erklärung völlig frei zu verfassen. Dieser Befehl kann 
\emph{ausschließlich} innerhalb der \Environment{declarations}"=Umgebung 
genutzt werden, wobei im ersten Argument die gewünschte Überschrift und im 
zweiten der Inhalt respektive Text der Erklärung selbst angegeben werden muss.

Die im Folgenden beschriebenen Parameter können sowohl für die Umgebung 
\Environment{declarations} selbst als auch für die zuvor genannten Befehle als 
optionales Argument verwendet werden. Ähnlich wie die gleichnamigen Optionen 
sind auch die Umgebungen \Environment{abstract} und \Environment{declaration} 
sehr ähnlich zueinander. Deshalb werden die Erläuterungen relativ kurz 
gehalten. Ist ein Erklärung für einen Parameter etwas unverständlich, kann 
diese bei der Umgebung \Environment{abstract}'full' nachgelesen werden.

Wurde das Paket \Package{babel} geladen, kann die Sprache~-- sofern diese als 
Paketoption oder besser noch als Klassenoption angegeben wurde~-- mit dem 
Parameter \Key{\Environment{declarations}}{language}[\PName{Sprache}] für die 
\Environment{declarations}"=Umgebung geändert werden. Dadurch werden die 
Bezeichner~-- unter anderem \Term{confirmationname} und \Term{blockingname}~-- 
sowie die Trennungsmuster innerhalb der Umgebung sprachspezifisch angepasst. 

\ChangedAt{v2.02}
Die Kolumnentitel können mit \Key{\Environment{declarations}}{markboth} 
beeinflusst werden. Mit \Key{\Environment{abstract}}{markboth}[true] werden 
für diese auf linker und rechter Seite \Term{confirmationname} respektive 
\Term{blockingname} verwendet. Außerdem kann der Anwender selbige mit 
\Key{\Environment{declarations}}{markboth}[\PName{Kolumnentitel}] auch direkt 
festlegen. Sollte \Key{\Environment{declarations}}{markboth} verwendet werden, 
wird der Seitenstil automatisch auf \PageStyle{headings} gesetzt. Mit dem 
Parameter \Key{\Environment{declarations}}{pagestyle} lässt sich dieser für die 
Umgebung auch manuell angegeben. Wurde das Paket \Package{multicol} geladen, 
wird mit \Key{\Environment{declarations}}{columns}[\PName{Anzahl}] der Inhalt 
der Umgebung mehrspaltig gesetzt. Für \Key{\Environment{declarations}}{option} 
können alle gültigen Werte der Option \Option{declaration} angegeben werden. 
Die Verwendung der Parameter \Key{\Macro{confirmation}}{supporter} sowie
\Key{\Macro{confirmation}}{place} und \Key{\Macro{confirmation}}{closing} ist 
in der Dokumentation des Befehls \Macro{confirmation} zu finden, der Parameter 
\Key{\Macro{blocking}}{company} ist für \Macro{blocking} erläutert. 
\end{Declaration}
\end{Declaration}
\end{Declaration}
\end{Declaration}
\end{Declaration}
\end{Declaration}
\end{Declaration}
\end{Declaration}
\end{Declaration}
\end{Declaration}
\end{Declaration}

\begin{Declaration}{\Macro{confirmation}\OLParameter{Unterstützer}}
\begin{Declaration}{\Key{\Macro{confirmation}}{supporter}[\PName{Unterstützer}]}
\begin{Declaration}{\Key{\Macro{confirmation}}{place}[\PName{Ort}]}
\begin{Declaration}{\Key{\Macro{confirmation}}{closing}[\PName{Ende}]}
\begin{Declaration}{\Key{\Macro{confirmation}}{language}[\PName{Sprache}]}
\begin{Declaration}[v2.02]{%
  \Key{\Macro{confirmation}}{markboth}[\PBName{Kolumnentitel}]%
}
\begin{Declaration}[v2.02]{%
  \Key{\Macro{confirmation}}{pagestyle}[\PName{Seitenstil}]%
}
\begin{Declaration}[v2.02]{%
  \Key{\Macro{confirmation}}{columns}[\PName{Anzahl}]%
}
\begin{Declaration}{\Key{\Macro{confirmation}}{option}[\PSet]}
\printdeclarationlist%
\index{Selbstständigkeitserklärung}\index{Datum}%
%
Mit diesem Befehl wird ein sprachspezifischer Standardtext für eine 
Selbstständigkeitserklärung ausgegeben, welcher in \Term{confirmationtext} 
gespeichert ist. Wie dieser angepasst beziehungsweise geändert werden kann, ist 
unter \autoref{sec:localization} zu finden. Er kann sowohl innerhalb der 
\Environment{declarations}"=Umgebung als auch außerhalb dieser direkt im 
Dokument verwendet werden. 

Wird \Term{confirmationtext} nicht geändert, kann dieser über das optionale 
Argument von \Macro{confirmation} und die deklarierten Parameter angepasst 
werden. Im Standardtext der Selbstständigkeitserklärung werden sowohl der Titel 
als auch der Typ der Abschlussarbeit~-- falls dieser mit \Macro{thesis}, 
\Macro{subject}\Parameter{\autoref{tab:thesis}} beziehungsweise mit der Option 
\Option{subjectthesis} angegeben wurde~-- aufgeführt. Über den Parameter 
\Key{\Macro{confirmation}}{supporter} oder \emph{zuvor} mit dem Befehl 
\Macro{supporter} können weitere an der Arbeit beteiligte Personen angegeben 
werden. Mehrere zu nennende Personen sind auch hier durch \Macro{and} zu 
trennen. Das Feld der Unterstützer kann auch mit dem bloßen optionalen Argument 
ohne die Angabe eines Parameters angepasst werden.

Nach dem eigentlichen Text der Selbstständigkeitserklärung wird der mit 
\Key{\Macro{confirmation}}{place} beziehungsweise \Macro{place} angegebene Ort 
sowie das mit \Macro{date} eingestellte Datum ausgegeben. Als Voreinstellung 
ist für den Ort \enquote{Dresden} gewählt. Danach folgen~-- mit etwas 
vertikalem Freiraum für die notwendige Unterschrift~-- der Autor oder die 
Autoren, angegeben durch den Befehl \Macro{author}. Soll anstelle dessen etwas 
anderes nach dem Text der Selbstständigkeitserklärung gesetzt werden, kann dies 
mit dem Parameter \Key{\Macro{confirmation}}{closing} oder zuvor mit dem 
Befehl \Macro{confirmationclosing} angepasst werden. Die Parameter 
\Key{\Environment{declarations}}{language}, 
\Key{\Environment{declarations}}{markboth}, 
\Key{\Environment{declarations}}{pagestyle}, 
\Key{\Environment{declarations}}{columns} und 
\Key{\Environment{declarations}}{option} entsprechen in ihrem Verhalten denen 
der \Environment{declarations}"=Umgebung.
\end{Declaration}
\end{Declaration}
\end{Declaration}
\end{Declaration}
\end{Declaration}
\end{Declaration}
\end{Declaration}
\end{Declaration}
\end{Declaration}

\begin{Declaration}[v2.02]{\Macro{blocking}\OLParameter{Firma}}
\begin{Declaration}{\Key{\Macro{blocking}}{company}[\PName{Firma}]}
\begin{Declaration}{\Key{\Macro{blocking}}{language}[\PName{Sprache}]}
\begin{Declaration}[v2.02]{%
  \Key{\Macro{blocking}}{markboth}[\PBName{Kolumnentitel}]%
}
\begin{Declaration}[v2.02]{\Key{\Macro{blocking}}{pagestyle}[\PName{Seitenstil}]}
\begin{Declaration}[v2.02]{\Key{\Macro{blocking}}{columns}[\PName{Anzahl}]}
\begin{Declaration}{\Key{\Macro{blocking}}{option}[\PSet]}
\printdeclarationlist%
\index{Sperrvermerk}%
%
Beim Sperrvermerk verhält es sich äquivalent zur Selbstständigkeitserklärung.
Es wird der in \Term{blockingtext} hinterlegte Standardtext in der gewählten 
Sprache ausgegeben. Dieser kann durch den Anwender geändert werden. Wie genau 
ist in \autoref{sec:localization} beschrieben. Der Befehl \Macro{blocking} 
kann sowohl innerhalb der Umgebung \Environment{declarations} als auch 
außerhalb direkt im Dokument verwendet werden. 

In seiner ursprünglichen Definition, kann er im optionalen Argument über die 
deklarierten Parameter angepasst werden. Im Standardtext des Sperrvermerks 
werden sowohl der Titel als auch der Typ der Abschlussarbeit~-- falls dieser 
mit \Macro{thesis}, \Macro{subject}\Parameter{\autoref{tab:thesis}} respektive  
mit der Option \Option{subjectthesis} angegeben wurde~-- aufgeführt. Mit 
\Key{\Macro{blocking}}{company} oder \emph{vorher} mit \Macro{company} kann 
zusätzlich eine im Sperrvermerk zu nennende Firma oder ähnliches angegeben 
werden. Dieses Feld kann auch direkt im optionalen Argument ohne die Verwendung 
eines Parameters gesetzt werden. Die weiteren Parameter 
\Key{\Environment{declarations}}{language}, 
\Key{\Environment{declarations}}{markboth}, 
\Key{\Environment{declarations}}{pagestyle}, 
\Key{\Environment{declarations}}{columns} und 
\Key{\Environment{declarations}}{option} entsprechen in ihrem Verhalten denen 
der \Environment{declarations}"=Umgebung.
\end{Declaration}
\end{Declaration}
\end{Declaration}
\end{Declaration}
\end{Declaration}
\end{Declaration}
\end{Declaration}

\begin{Declaration}{\Macro{declaration}\LParameter}
\begin{Declaration}{\Key{\Macro{declaration}}{language}[\PName{Sprache}]}
\begin{Declaration}[v2.02]{%
  \Key{\Macro{declaration}}{markboth}[\PBName{Kolumnentitel}]%
}
\begin{Declaration}[v2.02]{%
  \Key{\Macro{declaration}}{pagestyle}[\PName{Seitenstil}]%
}
\begin{Declaration}[v2.02]{\Key{\Macro{declaration}}{columns}[\PName{Anzahl}]}
\begin{Declaration}{\Key{\Macro{declaration}}{option}[\PSet]}
\begin{Declaration}{\Key{\Macro{declaration}}{supporter}[\PName{Unterstützer}]}
\begin{Declaration}{\Key{\Macro{declaration}}{place}[\PName{Ort}]}
\begin{Declaration}{\Key{\Macro{declaration}}{closing}[\PName{Ende}]}
\begin{Declaration}{\Key{\Macro{declaration}}{company}[\PName{Firma}]}
\printdeclarationlist%
\index{Selbstständigkeitserklärung}\index{Sperrvermerk}%
%
Dieser Befehl gibt die Selbstständigkeitserklärung und den Sperrvermerk direkt 
aufeinanderfolgend aus. Dabei werden die Einstellungen zur Positionierung der 
einzelnen Erklärungen, welche über die Wertzuweisungen an die Option 
\Option{declaration}[simple/multiple/fill/nofill] erfolgen, beachtet. Er kann 
sowohl innerhalb der \Environment{declarations}"=Umgebung als auch außerhalb 
direkt im Dokument verwendet werden und akzeptiert im optionalen Argument dabei 
alle für die \Environment{declarations}"=Umgebung beschriebenen Parameter.
\end{Declaration}
\end{Declaration}
\end{Declaration}
\end{Declaration}
\end{Declaration}
\end{Declaration}
\end{Declaration}
\end{Declaration}
\end{Declaration}
\end{Declaration}

\begin{Declaration}{\Macro{supporter}\Parameter{Unterstützer}}
\begin{Declaration}{\Macro{place}\Parameter{Ort}}
\begin{Declaration}{\Macro{confirmationclosing}\Parameter{Ende}}
\begin{Declaration}{\Macro{company}\Parameter{Firma}}
\printdeclarationlist%
\index{Selbstständigkeitserklärung}\index{Sperrvermerk}%
%
Diese Makros ändern~-- im Gegensatz zu den Parametern der bereits vorgestellten 
Befehle \Macro{confirmation} und \Macro{blocking}~-- die entsprechenden 
Feldwerte für das gesamte Dokument. Genutzt werden kann dies beispielsweise 
wenn ein Erklärungstyp in unterschiedlichen Sprachen ausgegeben wird. Hiermit 
kann man sich die mehrfache Angabe eines Parameters sparen.
\end{Declaration}
\end{Declaration}
\end{Declaration}
\end{Declaration}


\subsection{Fußnoten in Überschriften}
\begin{Declaration}[%
  v2.02!Fußnoten in Überschriften können mit Symbolen gesetzt werden%
]{\Option{footnotes}[\PSet]}[nosymbolheadings]%
\begin{Declaration}[v2.02]{\Counter{symbolheadings}}%
\printdeclarationlist%
\index{Überschriften}\index{Überschriften!Fußnoten}\index{Fußnoten}%
%
\ToDo[imp]{Fehler mit \Macro{addchap} beheben, Paket \Package{footmisc}}[v2.06]
\ToDo[imp]{Zähler auch bei Sternversionen von Kapiteln zurücksetzen}[v2.06]
\ToDo[imp]{Fußnoten nicht ins Inhaltsverzeichnis?}[v2.06]
\ToDo[imp]{Problem mit \Package{hyperref} lösbar?}[v2.06]
Für die Überschriften wird die \KOMAScript-Option \Option{footnotes} erweitert.
Normalerweise kann diese die Werte \PValue{multiple} und \PValue{nomultiple} 
annehmen, wobei Letzteres der Standardfall ist. Die \TUDScript-Hauptklassen 
erweitern die Option dahingehend, dass auf die Verwendung von Symbolen anstelle 
von Zahlen innerhalb der Überschriften umgeschaltet werden kann. Hierfür wird 
der Zähler \Counter{symbolheadings} definiert, der mit dem Beginn eines neuen 
Kapitels zurückgesetzt wird.
%
\begin{values}
\item[nosymbolheadings/numberheadings]
  Die Fußnoten der Überschriften werden fortlaufend mit denen des Fließtextes 
  gesetzt.
\item[symbolheadings]
  Für die Überschriften werden symbolische Fußnoten mit einem eigenen Zähler 
  verwendet.
\end{values}
\end{Declaration}
\end{Declaration}


\subsection{Lesezeichen}
\begin{Declaration}{\Option{tudbookmarks}[\PBoolean]}[true]%
\printdeclarationlist%
\index{Lesezeichen}%
\index{Titel}\index{Umschlagseite}\index{Inhaltsverzeichnis}%
\index{Aufgabenstellung}\index{Gutachten}\index{Aushang}%
%
Diese Option wird wirksam, wenn \Package{hyperref} geladen wurde. Es werden für 
die Umschlag- und Titelseite, das Inhaltsverzeichnis sowie~-- bei der 
Verwendung des Paketes \Package{tudscrsupervisor}~-- die Aufgabenstellung 
Lesezeichen oder auch Outline"=Einträge im PDF-Dokument erzeugt.
%
\begin{values}
\itemfalse
  Es erfolgt kein Eintrag von ergänzenden Lesezeichen.
\itemtrue*
  Es werden automatisch zusätzliche Lesezeichen eingetragen.
\end{values}
\end{Declaration}

\begin{Declaration}{%
  \Macro{tudbookmark}\OParameter{Ebene}\Parameter{Text}\Parameter{Ankername}%
}%
\printdeclarationlist%
%
Der Befehl \Macro{tudbookmark} arbeitet wie \Macro{pdfbookmark} aus 
\Package{hyperref} mit dem Unterschied, dass die Lesezeichen nur generiert 
werden, wenn die Option \Option{tudbookmarks} aktiviert ist.
\end{Declaration}



\section{Sprachabhängige Bezeichner}
\label{sec:localization}
\index{Bezeichner|!(}%
%
Durch \KOMAScript{} werden Befehle, mit denen sprachabhängige Bezeichner 
erzeugt oder geändert werden können, zur Verfügung gestellt. Diese werden durch 
\TUDScript genutzt, um lokalisierte Begriffe für die Sprachen Englisch und 
Deutsch bereitzustellen. Ein Großteil davon betrifft Bezeichnungen für Felder 
auf der Titelseite (\autoref{sec:title}). Hierfür wird
\Macro{providecaptionname}\Parameter{Sprache}\Parameter{Makro}\Parameter{Inhalt}
verwendet, wobei \PName{Sprache} dem geladenen Sprachpaket~-- normalerweise das 
Paket \Package{babel}~-- bekannt sein muss.

Sollte der Anwender die im Folgenden erläuterten oder auch andere Bezeichner, 
welche von einem beliebigen (Sprach"~)Paket bereitgestellt werden, ändern 
wollen, ist hierfür der Befehl
\Macro{renewcaptionname}\Parameter{Sprache}\Parameter{Makro}\Parameter{Inhalt} 
zu verwenden. Es sollte natürlich dabei eine \PName{Sprache} angegeben werden, 
welche im Dokument durch \Package{babel} oder ein anderes Sprachpaket verwendet 
wird, beispielsweise \PValue{ngerman} oder \PValue{english}. 

Die Makros der Bezeichner und deren Verwendung werden folgend kurz beschrieben 
und tabellarisch aufgeführt. Dabei wurde versucht, alle Befehle der Bezeichner 
für bestimmte Begriffe auf \PValue{\dots{}name} und beschreibende Texte auf 
\PValue{\dots{}text} enden zu lassen.

\begin{Declaration}{\Term{supervisorname}}
\begin{Declaration}{\Term{supervisorothername}}
\begin{Declaration}[%
  v2.02!Unterscheidung von einem und mehreren Gutachtern%
]{\Term{refereename}}
\begin{Declaration}{\Term{refereeothername}}
\begin{Declaration}{\Term{advisorname}}
\begin{Declaration}{\Term{advisorothername}}
\begin{Declaration}[%
  v2.02!Unterscheidung von einem und mehreren Professoren%
]{\Term{professorname}}
\begin{Declaration}[v2.02]{\Term{professorothername}}
\printdeclarationlist%
\index{Titel}%
\index{Betreuer}\index{Gutachter}\index{Hochschullehrer}%
\index{Referent}%
%
Diese sprachabhängigen Begriffe sind die Bezeichner für die Titelseitenfelder 
von Betreuer (\Macro{supervisor}), Gutachter (\Macro{referee}) und Fachreferent 
(\Macro{advisor}). Soll innerhalb eines dieser Felder mehr als eine Person 
angegeben werden, so sind die Einzelpersonen jeweils mit dem Befehl \Macro{and} 
voneinander zu trennen. In diesem Fall werden alle nach der erstgenannten 
folgenden Personen durch den Bezeichner \PValue{\textbackslash\dots{}othername} 
ergänzt.

\ChangedAt{v2.02}
Bei der Bezeichnung des Gutachters wird unterschieden, ob einer oder mehrere 
angegeben wurden. Wird lediglich einer genannt, so ist eine Unterscheidung 
nicht notwendig. Werden jedoch zwei Gutachter angegeben, so werden diese auch 
mit Erst- und Zweitgutachter betitelt. Für den betreuenden Hochschullehrer 
(\Macro{professor}) wird ähnlich verfahren. Hier wird allerdings lediglich 
die Bezeichnung vom Singular in den Plural gegebenenfalls automatisch geändert.


\renewcaptionname{ngerman}{\refereename}{Gutachter/Erstgutachter}
\renewcaptionname{english}{\refereename}{Referee/First referee}
\TermTable{%
  supervisorname,supervisorothername,refereename,refereeothername,%
  advisorname,advisorothername,professorname,professorothername%
}
\end{Declaration}
\end{Declaration}
\end{Declaration}
\end{Declaration}
\end{Declaration}
\end{Declaration}
\end{Declaration}
\end{Declaration}

\begin{Declaration}{\Term{dissertationname}}
\begin{Declaration}{\Term{diplomathesisname}}
\begin{Declaration}{\Term{masterthesisname}}
\begin{Declaration}{\Term{bachelorthesisname}}
\begin{Declaration}{\Term{studentresearchname}}
\begin{Declaration}{\Term{projectpapername}}
\begin{Declaration}{\Term{seminarpapername}}
\begin{Declaration}{\Term{researchname}}
\begin{Declaration}{\Term{logname}}
\begin{Declaration}{\Term{internshipname}}
\begin{Declaration}{\Term{reportname}}
\printdeclarationlist%
\index{Titel}\index{Abschlussarbeit}\index{Typisierung}%
%
Diese Bezeichner dienen zur Typisierung speziell für eine Abschlussarbeit. Wie 
diese genutzt werden können, ist bei der Erläuterung von \Macro{thesis} und 
\Macro{subject}'full' beziehungsweise in \autoref{tab:thesis} zu finden.
\TermTable{%
  dissertationname,diplomathesisname,masterthesisname,bachelorthesisname,%
  studentresearchname,projectpapername,seminarpapername,researchname,%
  logname,internshipname,reportname%
}
\end{Declaration}
\end{Declaration}
\end{Declaration}
\end{Declaration}
\end{Declaration}
\end{Declaration}
\end{Declaration}
\end{Declaration}
\end{Declaration}
\end{Declaration}
\end{Declaration}

\begin{Declaration}{\Term{dateofbirthtext}}
\begin{Declaration}{\Term{placeofbirthtext}}
\begin{Declaration}{\Term{matriculationnumbername}}
\begin{Declaration}{\Term{matriculationyearname}}
\printdeclarationlist%
\index{Titel}\index{Autorenangaben}\index{Datum!Geburtsdatum}%
%
Werden für den Autor oder die Autoren das Geburtsdatum (\Macro{dateofbirth}), 
der Geburtsort (\Macro{placeofbirth}) sowie die
Matrikelnummer (\Macro{matriculationnumber}) und/oder das Immatrikulationsjahr 
(\Macro{matriculationyear}) angegeben, werden sowohl auf der Titelseite als 
auch auf der gegebenenfalls mit \Package{tudscrsupervisor} erstellten 
Aufgabenstellung die dazugehörigen Bezeichner vorangestellt. Auf dem Titel 
werden diese dabei mit dem durch \Macro{titledelimiter} gegebenen Trennzeichen 
vom eigentlichen Feld abgegrenzt.
\TermTable{%
  dateofbirthtext,placeofbirthtext,matriculationnumbername,%
  matriculationyearname%
}
\end{Declaration}
\end{Declaration}
\end{Declaration}
\end{Declaration}

\begin{Declaration}[v2.02]{\Term{graduationtext}}
\printdeclarationlist%
\index{Titel}\index{Abschlussarbeit}\index{Typisierung}%
%
Wurde erkannt, dass das Dokument eine Abschlussarbeit ist,%
\footnote{%
  Entweder wurde \Macro{thesis} oder \Macro{subject} mit einem speziellen Wert 
  oder der Option \Option{subjectthesis} verwendet.
}
so kann der zu erlangende akademische Grad mit dem Befehl \Macro{graduation} 
angegeben werden. Bei dessen Ausgabe auf dem Titel wird dabei der entsprechende 
Text dazu angegeben.
\TermTable*{graduationtext}{.78\textwidth}
\end{Declaration}

\begin{Declaration}{\Term{datetext}}
\begin{Declaration}{\Term{defensedatetext}}
\printdeclarationlist%
\index{Titel}\index{Abschlussarbeit}%
\index{Datum}\index{Datum!Verteidigungsdatum}%
%
Wird mit \Macro{date} das Datum und mit \Macro{defensedate} ein Datum der 
Verteidigung für eine Abschlussarbeit angegeben, so werden auch diese Felder 
durch einen einleitenden Text beschrieben.
\TermTable{datetext,defensedatetext}
\end{Declaration}
\end{Declaration}

\begin{Declaration}{\Term{abstractname}}
\printdeclarationlist%
%
Dieser Bezeichner wird lediglich für \Class{tudscrbook} definiert, da dieser 
von \KOMAScript{} für die Buchklasse nicht vorgesehen wird.
\TermTable{abstractname}
\end{Declaration}

\begin{Declaration}{\Term{confirmationname}}
\begin{Declaration}[v2.02]{\Term{blockingname}}
\printdeclarationlist%
\index{Selbstständigkeitserklärung}\index{Sperrvermerk}%
%
Es werden die Bezeichnungen für Selbstständigkeitserklärung und Sperrvermerk 
für die dazugehörigen Überschriften definiert.
\TermTable{confirmationname,blockingname}
\end{Declaration}
\end{Declaration}

\begin{Declaration}{\Term{confirmationtext}}
\begin{Declaration}[v2.02]{\Term{blockingtext}}
\printdeclarationlist%
%
Die Texte der Erklärungen selbst sind derart aufgebaut, dass sie in 
Abhängigkeit von den angegebenen Informationen unterschiedlich ausgeführt 
werden. Innerhalb der Selbstständigkeitserklärung (\Macro{confirmation}) werden 
gegebenenfalls die Felder für den Titel (\Macro{title}) und die Typisierung der 
Abschlussarbeit%
\footnote{%
  entweder \Macro{thesis} oder \Macro{subject}\Parameter{\autoref{tab:thesis}}
  beziehungsweise Option \Option{subjectthesis}[true]
}
sowie die angegebenen Unterstützer%
\footnote{%
  \Macro{confirmation}\POParameter{\Key{\Macro{confirmation}}{supporter}=\dots}
  oder \Macro{supporter}\PParameter{\dots}%
}
beachtet. Für den Sperrvermerk (\Macro{blocking}) wird neben dem Titel 
(\Macro{title}) optional außerdem noch das Feld der externen Firma%
\footnote{%
  \Macro{blocking}\POParameter{\Key{\Macro{blocking}}{company}=\dots} oder 
  \Macro{company}\PParameter{\dots}%
}
verwendet. Der Vollständigkeit halber werden im Folgenden noch die Texte für 
die Selbstständigkeitserklärung und den Sperrvermerk aufgeführt~-- allerdings 
lediglich die deutschsprachige Version. Dabei werden alle möglichen Felder 
angezeigt.

\begingroup
  \makeatletter
  \def\@@title{\PName{Titel}}
  \def\@@thesis{\PName{Abschlussarbeit}}
  \def\@supporter{\PName{Vorname Nachname} \and \PName{Vorname Nachname}}
  \def\@company{\PName{Firma}}
  \makeatother
  \vskip\baselineskipglue\noindent
  \textbf{Bezeichner}\quad\Term*{confirmationtext}%
  \begin{quoting}
  \confirmationtext
  \end{quoting}
  \textbf{Bezeichner}\quad\Term*{blockingtext}%
  \begin{quoting}
  \blockingtext
  \end{quoting}
\endgroup
\end{Declaration}
\end{Declaration}

\begin{Declaration}{\Term{coverpagename}}
\begin{Declaration}{\Term{titlepagename}}
\printdeclarationlist%
\index{Lesezeichen}\index{Titel}\index{Umschlagseite}%
%
Diese beiden Bezeichner werden bei aktivierter \Option{tudbookmarks} für das 
Eintragen von Lesezeichen in ein PDF"=Dokument genutzt.
\TermTable{coverpagename,titlepagename}
\end{Declaration}
\end{Declaration}

\begin{Declaration}{\Term{listingname}}
\begin{Declaration}{\Term{listlistingname}}
\printdeclarationlist%
%
Sollte ein Paket zur Einbindung von externem Quelltext~-- beispielsweise 
das Paket \Package{listings}~-- verwendet werden, so werden diese Bezeichnungen 
für Quelltextausschnitte und das Quelltextverzeichnis verwendet.
\TermTable{listingname,listlistingname}
\end{Declaration}
\end{Declaration}
\index{Bezeichner|!)}

\section{Kompatibilitätseinstellungen zu früheren Versionen}
Bei der Entwicklung von \TUDScript lässt es sich nicht immer vermeiden, dass 
Verbesserungen sowie Korrekturen an den Klassen und Paketen zu Änderungen am 
Ergebnis der Ausgabe führen, insbesondere bei Umbruch und Layout. Für bereits
archivierte Dokumente, welche mit einer früheren Version erstellt wurden ist 
dies jedoch bei einer erneuten Kompilierung unter Umständen eher unerwünscht.

\begin{Declaration}[v2.03]{\Option{tudscrver}[%
  \PName{Version}\textOR\PValue{first}\textOR\PValue{last}%
]}[last]
\printdeclarationlist%
\index{Kompatibilität|!}%
%
Mit dieser Option wird es möglich, auf das (Umbruch-)Verhalten einer älteren 
respektive früheren Version von \TUDScript umzuschalten, um nach der 
Kompilierung das erwartete Ergebnis zu erhalten. Neue Möglichkeiten, die sich 
nicht auf den Umbruch oder das Layout auswirken, sind auch für den Fall 
verfügbar, dass per Option die Kompatibilität zu einer älteren Version 
ausgewählt wurde. 

Bei der Angabe einer unbekannten Version als Wert wird eine Warnung ausgegeben 
und \Option{tudscrver}[first] angenommen. Mit \Option{tudscrver}[last] wird die 
jeweils aktuell verfügbare Version ausgewählt und folglich auf die zukünftige 
Kompatibilität des Dokumentes zu der aktuell genutzten Version verzichtet. 
Dieses Verhalten entspricht der Voreinstellung. Es ist zu beachten, dass die 
Nutzung von \Option{tudscrver} nur als Klassenoption möglich ist.
%
\begin{values}
\item[\PValue{2.02}\textOR\PValue{first}]
  \ChangedAt{%
    v2.03!Satzspiegel im \CD geändert{,} das \protect\DDC-Logo im Fußbereich 
    wird ohne vergrößerten Seitenrand verwendet
  }
  Der Satzspiegel im Layout des \CDs (\see*{\Option{cdgeometry}}) wurde in der 
  Version~v2.03 leicht geändert. Der obere Seitenrand wurde verkleinert, der 
  untere im gleichen Maße vergrößert. Der verfügbare Textbereich ist folglich 
  identisch. Bei der Aktivierung des \DDC-Logos im Fußbereich der Seite
  (\see*{\Option{ddcfoot}}) wird im Gegensatz zur Version~v2.02 der gleiche 
  Satzspiegel genutzt. Mit \Option{tudscrver}[2.02] kann dieses Verhalten 
  deaktiviert werden.
\item[\PValue{2.03}]
  \ChangedAt{%
    v2.04!Werte bestimmter Längen abhängig von der verwendeten Schriftgröße%
  }\index{Schriftgröße}%
  Seit der Version~v2.04 werden mehrere Längen in Abhängigkeit der gewählten 
  Schriftgröße (Option \Option{fontsize}) definiert. Dies betrifft sowohl die 
  dehnbaren Längen \Length{smallskipamount}, \Length{medskipamount} und 
  \Length{bigskipamount}, die von den Befehlen \Macro{smallskip},   
  \Macro{medskip} sowie \Macro{bigskip} für das Einfügen vertikaler Abstände 
  genutzt werden, als auch die beiden Längen \Length{abovecaptionskip} und 
  \Length{belowcaptionskip} für den Abstand zwischen einem Gleitobjekt und 
  dessen mit \Macro{caption} gesetzten Beschreibung sowie \Length{columnsep} 
  als Maß für den Abstand der einzelnen Textspalten im zwei- oder mehrspaltigen 
  Layout. Mit der Wahl \Option{tudscrver}[2.03] lässt sich diese Funktionalität 
  deaktivieren.
\item[\PValue{2.04}]
  Dies ist Kompatibilitätseinstellung für \TUDScript~\vTUDScript{} und wird für 
  zukünftige Änderungen bereits vorgehalten. Soll ein mit der momentan 
  aktuellen Version erzeugtes Dokument auch mit einer späteren Version von 
  \TUDScript nach einem \hologo{LaTeX}-Lauf das gleiche Ausgabeergebnis 
  liefern, muss dies mit \Option{tudscrver}[2.04] angegeben werden.
\item[\PValue{last}]
  Es werden keine Kompatibilitätseinstellungen für das Dokument vorgenommen. 
  Mit einer späteren Version von \TUDScript kann ein anderes Umbruchverhalten 
  innerhalb des Dokumentes auftreten. Dies ist die Standardeinstellung.
\end{values}
\end{Declaration}
