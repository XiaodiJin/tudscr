\chapter[Die Klassen tudscrbook, tudscrreprt und tudscrartcl]{Die Hauptklassen}
\begin{Declaration*}{\Class{tudscrbook}}
\begin{Declaration*}{\Class{tudscrreprt}}
\begin{Declaration*}{\Class{tudscrartcl}}
\index{Hauptklassen|!}
Es werden die drei neuen Hauptklassen
%
\begin{description}
\item \Class{tudscrbook}
\item \Class{tudscrreprt}
\item \Class{tudscrartcl}
\end{description}
%
eingeführt, welche auf den \KOMAScript"=Klassen basieren und grundsätzlich alle
deren bekannten Optionen, Umgebungen und Befehle~-- beispielsweise
\Option{BCOR} zur Festlegung der Bindekorrektur oder \Option{parskip} zur 
Regelung der Absatzeinstellungen~-- unterstützen. Zusätzlich zu den 
\KOMAScript"=Klassen werden weitere Pakete zwingend benötigt, welche unter 
\autoref{sec:packages:needed} aufgeführt sind und durch den Anwender nicht noch 
zusätzlich geladen werden müssen.

Es sei hier abermals auf die Anwenderdokumentation \scrguide von \KOMAScript{} 
hingewiesen, viele der folgend beschriebenen Befehle und Optionen beziehen sich 
auf die darin vorgestellten Einstellungsmöglichkeiten. Die Anpassungen und 
Erweiterungen der \KOMAScript"=Klassen an das \CD und die neu definierten 
beziehungsweise geänderten Befehle und Optionen werden im Folgenden erläutert.
\end{Declaration*}
\end{Declaration*}
\end{Declaration*}

\begin{Declaration}{\Macro{TUDoptions}\Parameter{Optionenliste}}
\begin{Declaration}{\Macro{TUDoption}\Parameter{Option}\Parameter{Werteliste}}
\printdeclarationlist%
\index{Optionen|!}%
\index{Optionenwahl|!}%
%
Mit diesen Befehlen hat man bei den meisten der neuen Klassenoptionen die 
Möglichkeit, den Wert der Optionen noch nach dem Laden der Klasse zu ändern.
Man kann wahlweise mit der Anweisung \Macro{TUDoptions} die Werte einer Reihe 
von Optionen ändern. Jede Option der Optionenliste hat dabei die Form
\PName{Option}\PValue{=}\PName{Wert}. Die meisten Optionen besitzen auch einen 
Säumniswert\footnote{engl.: default value}. Versäumt man die Angabe eines 
Wertes~-- verwendet demzufolge einfach die Form \PName{Option}~-- so wird 
automatisch dieser Säumniswert angenommen.

Manche Optionen können gleichzeitig mehrere Werte besitzen. Für diese besteht 
die Möglichkeit, mit \Macro{TUDoption} der einen Option nacheinander eine 
Reihe von Werten zuzuweisen. Die einzelnen Werte sind dabei in der Werteliste 
durch Komma voneinander getrennt.

Mit diesen beiden Befehlen kann im Bedarfsfall das Verhalten von einer Option 
oder mehreren Optionen im Dokument geändert werden. Werden diese Befehle in 
einer Umgebung oder einer Gruppe verwendet, bleiben die gemachten Einstellungen 
innerhalb dieser lokal begrenzt.
\end{Declaration}
\end{Declaration}


\section{Die Schriften des \CDs}
\label{sec:fonts}
\index{Schrift|?}
%
Das \CD der \TnUD gibt die Verwendung der Schriften \Univers für den Fließtext 
sowie \DIN für das Setzen von Überschriften vor. Im Standardfall wird dies so 
unterstützt. Da jedoch in längeren Texten die Verwendung von Serifenschriften 
zu empfehlen ist, gibt es die Möglichkeit, die ursprünglich vorgesehenen 
Schriften nicht zu laden und die Standardschriften beziehungsweise ein anderes 
Schriftpaket zu verwenden. Die Einstellungen und Befehle für den Fließtext sind 
in \autoref{sec:text} zu finden.

Durch das \CD werden keine Schriften für den Mathematiksatz bereitgestellt. 
Dies ist insbesondere für sowohl mathematische als auch ingenieur- und 
naturwissenschaftliche Dokumente nicht tragbar. Dieser Mangel wird behoben, 
indem im Mathematikmodus die lateinischen Buchstaben der Hausschriften mit 
griechischen Lettern und mathematischen Symbolen aus anderen Paketen ergänzt 
werden.%
\footnote{%
  \Package{iwona} für die Schrift \DIN und zusätzlich \Package{cmbright} für 
  die \Univers"=Schriftfamilie%
}
Diese Einstellungen kann natürlich ebenfalls mit der entsprechenden Option 
deaktiviert werden. Dann werden die Standardschriften oder gegebenenfalls die 
eines zusätzlichen Paketes für den mathematischen Satz genutzt. Alle Befehle 
und Optionen für den Mathematiksatz sind in \autoref{sec:math} erläutert. 
Weitere Hinweise zum typographisch guten Mathematiksatz sind außerdem in 
\autoref{sec:exmpl:mathswap} sowie \autoref{sec:exmpl:mathtype} zu finden.


\subsection{Schriften für den Textsatz}
\begin{Declaration}[%
  v2.02!Werte für die Optionen \Option{barfont} und \Option{fontspec} ergänzt%
]{\Option{cdfont}[\PSet]}[true]%
\printdeclarationlist%
\label{sec:text}%
\index{Schrift}\index{Schrift!Fließtext}
\index{Schrift!Corporate Design}\index{Schrift!Stärke}%
%
Mit dieser Option können durch den Benutzer alle zentralen Schrifteinstellungen 
für die \TUDScript-Klassen vorgenommen werden. Dies betrifft die Schriften für 
Überschriften, den Text im Querbalken der Kopfzeile sowie den Fließtext und die 
Mathematikschriften.
%
\begin{values}
\itemfalse
  Es werden keine Hausschriften sondern die \hologo{LaTeX}"=Standardschriften 
  verwendet und der Benutzer kann beliebige Schriftpakete nutzen.%
  \footnote{%
    Für die Verwendung der klassischen \hologo{LaTeX}"=Schriften, ist das Paket 
    \Package{lmodern} sehr empfehlenswert.%
  }
  Sollte das Layout des \CDs aktiviert sein (siehe \Option{cd}), werden die 
  Überschriften in serifenlosen Großbuchstaben gesetzt.
\itemtrue*[light/lightfont/noheavyfont]
  Es werden die Hausschriften im Stil des \CDs der \TnUD genutzt. Überschriften 
  der obersten Gliederungsebenen bis einschließlich \Macro*{subsubsection} 
  verwenden \DIN, darunter liegende%
  \footnote{\Macro*{paragraph} und \Macro*{subparagraph}} 
  \textubn{Univers~65~Bold}. Für den Fließtext im Dokument kommt 
  \textuln{Univers~45~Light} zum Einsatz. Aus \Package{lmodern} wird die
  \texttt{Schreibmaschinenschrift} verwendet.
\item[heavy/heavyfont]
  Die Schriftstärke der Hausschriften wird erhöht. Die beiden untersten 
  Gliederungsebenen werden in \textuxn{Univers~75~Black} gesetzt, der Fließtext 
  in \texturn{Univers~55~Regular}. Ansonsten entspricht alles der Option 
  \Option{cdfont}[true]. Die Mathematikschriften werden durch diese 
  Einstellung nicht beeinflusst. Gegebenenfalls sollte mit \Macro{boldmath} auf 
  den fetten Schnitt umgeschaltet werden.
\item[nodin]
  Für die Überschriften wird nicht \DIN verwendet. Ist \Option{cdfont}[true] 
  gewählt, wird \Univers genutzt. Die Schriftstärke ist dabei abhängig von der 
  Einstellung \Option{cdfont}[light/heavy]. Ist die Verwendung der Schriften 
  des \CDs deaktiviert (\Option{cdfont}[false]), kommt die fette Schriftstärke 
  der eingestellten serifenlosen Schriftfamilie zum Einsatz.
\item[din]
  Mit dieser Einstellung wird die Schrift \DIN in den Überschriften verwendet. 
  Sie ist standardmäßig aktiviert.
\end{values}
%
\ChangedAt{v2.02}
Für den Text im Querbalken gibt es folgende Einstellmöglichkeiten:
%
\begin{values}
\item[barfont]
  Für den Querbalken der Kopfzeile wird unabhängig von der Verwendung der 
  Hausschriften die Schrift \Univers in normaler Schriftstärke verwendet,
  siehe \Option{barfont}[true].
\item[heavybarfont]
  Die im Querbalken der Kopfzeile verwendete Stärke der Schrift \Univers wird 
  erhöht, siehe \Option{barfont}[heavy].
\end{values}
%
Die verwendeten Mathematikschriften lassen sich mit folgenden Werte 
beeinflussen:
%
\begin{values}
\item[serifmath/serif/nosansmath/nosans]  
  Diese Einstellung deaktiviert die Verwendung von serifenlosen Schriften für 
  den mathematischen Satz. Es werden die \hologo{LaTeX}"=Standardschriften 
  verwendet und der Benutzer kann beliebige Schriftpakete für den 
  Mathematikmodus nutzen, siehe \Option{sansmath}[false].
\item[sansmath/sans]
  Es werden serifenlose Mathematikschriften für lateinische und griechische 
  Lettern genutzt, siehe \Option{sansmath}[true].
\item[upgreek/uprightgreek/uprightGreek]
  Die großen griechischen Buchstaben werden im Mathematikmodus aufrecht gesetzt,
  siehe \Option{slantedgreek}[false].
\item[slgreek/slantedgreek/slantedGreek]
  In mathematischen Umgebungen erfolgt die Ausgabe der griechischen Majuskeln 
  kursiv, siehe \Option{slantedgreek}[true].
\end{values}
%
\ChangedAt{v2.02}
Außerdem kann mit folgenden Werten das verwendete Schriftformat eingestellt 
werden: 
%
\begin{values}
\item[fontspec/lualatex/xelatex]
  Es wird das Paket \Package{fontspec} geladen und die Schriften des \CDs im 
  OpenType"=Format verwendet. Hierfür muss entweder \hologo{LuaLaTeX} oder 
  \hologo{XeLaTeX} als Dokumentprozessor genutzt werden. Wird diese Einstellung 
  aktiviert, sind die Hinweise zur Option \Option{fontspec}'full' unbedingt 
  zu beachten.
\item[nofontspec/pdflatex]
  Es werden die Schriften des \CDs im PostScript"=Format verwendet, wenn diese 
  wie unter \autoref{sec:install} beschrieben installiert wurden. Diese 
  Einstellung ist standardmäßig aktiviert und sollte nur in Ausnahmefällen 
  geändert werden.
\end{values}
\end{Declaration}

\subsubsection{Auszeichnungen in Überschriften}
\begin{Declaration}[v2.02]{\Option{footnotes}[\PSet]}[nosymbolheadings]%
\begin{Declaration}[v2.02]{\Counter{symbolheadings}}%
\printdeclarationlist%
\index{Überschriften}\index{Überschriften!Fußnoten}\index{Fußnoten}%
%
Für die Überschriften wird die \KOMAScript-Option \Option{footnotes} erweitert.
Normalerweise kann diese die Werte \PValue{multiple} und \PValue{nomultiple} 
annehmen, wobei Letzteres der Standardfall ist. Die \TUDScript-Hauptklassen 
erweitern die Option dahingehend, dass auf die Verwendung von Symbolen anstelle 
von Zahlen innerhalb der Überschriften umgeschaltet werden kann. Hierfür wird 
der Zähler \Counter{symbolheadings} definiert, der mit dem Beginn eines neuen 
Kapitels zurückgesetzt wird.
%
\begin{values}
\item[nosymbolheadings/numberheadings]
  Die Fußnoten der Überschriften werden fortlaufend mit denen des Fließtextes 
  gesetzt.
\item[symbolheadings]
  Für die Überschriften werden symbolische Fußnoten mit einem eigenen Zähler 
  verwendet.
\end{values}
\end{Declaration}
\end{Declaration}

\begin{Declaration}{\Macro{ifdin}\Parameter{Dann-Teil}\Parameter{Sonst-Teil}}%
\printdeclarationlist%
\index{Überschriften}\index{Schrift!Überschriften}\index{Schriftauszeichnung}%
\index{Kolumnentitel}\index{Layout!Kolumnentitel}
%
Der Befehl \Macro{ifdin} prüft, ob die Schriftfamilie \DIN aktiv ist und führt 
in diesem Fall \Parameter{Dann-Teil} aus, andernfalls \Parameter{Sonst-Teil}. 
Dies ist beispielsweise bei Überschriften sinnvoll, wenn zwischen der Ausgabe 
im Fließtext und dem Eintrag für das Inhaltsverzeichnis sowie der Ausprägung 
der automatischen Kolumnentitel unterschieden werden soll.
\end{Declaration}

\begin{Declaration}{\Macro{MakeTextUppercase}\Parameter{Text}}%
\begin{Declaration}{\Macro{NoCaseChange}\Parameter{Text}}%
\printdeclarationlist%
\index{Überschriften}\index{Schrift!Überschriften}\index{Schriftauszeichnung}%
%
Diese beiden Befehle stammen aus dem Paket \Package{textcase}. Der Befehl 
\Macro{MakeTextUppercase} setzt den Text seines Argumentes in Majuskeln. Die 
Überschriften der Gliederungsebenen bis einschließlich \Macro*{subsubsection} 
werden damit in Großbuchstaben der Schrift \DIN gesetzt. Sollen bestimmte 
Kleinbuchstaben erhalten bleiben, ist der Befehl \Macro{NoCaseChange} zu nutzen.
\end{Declaration}
\end{Declaration}
%
\begin{Example}
In einer Kapitelüberschrift wird ein einzelnes Wort in Kleinbuchstaben 
geschrieben:
\begin{Code}[escapechar=§]
\chapter{§Ü§berschrift mit \NoCaseChange{kleinem} Wort}
\end{Code}
\end{Example}

\subsubsection{Auszeichnungen im Text}
\begin{Declaration}[v2.02]{\Font{titlepage}}
\begin{Declaration}[v2.02]{\Font{thesis}}
\begin{Declaration}[v2.02]{\Font{parttitle}}
\printdeclarationlist%
\index{Schriftelemente}
%
Die \TUDScript-Klassen definieren mit \Macro{newkomafont} diese neuen 
Schriftelemente. Dabei wird \Font{titlepage} auf der Titelseite für alle 
Felder verwendet, welche kein spezielles Schriftelement verwenden und 
\Font{thesis} für das mit \Macro{thesis} angegebene Feld, in welchem der Typ 
einer Abschlussarbeit angegeben wird (siehe \autoref{sec:title}). Mit 
\Font{parttitle} kann die Schrift für die Bezeichnung des Teils bei 
aktivierter \Option{parttitle}-Option beeinflusst werden. Alle Schriften lassen 
sich mit \Macro{addtokomafont}\Parameter{Schriftelement} anpassen.
\end{Declaration}
\end{Declaration}
\end{Declaration}

\begin{Declaration}{\Macro{univln}}
\begin{Declaration}{\Macro{textuln}\Parameter{Text}}
\begin{Declaration}{\Macro{univrn}}
\begin{Declaration}{\Macro{texturn}\Parameter{Text}}
\begin{Declaration}{\Macro{univbn}}
\begin{Declaration}{\Macro{textubn}\Parameter{Text}}
\begin{Declaration}{\Macro{univxn}}
\begin{Declaration}{\Macro{textuxn}\Parameter{Text}}
\begin{Declaration}{\Macro{univls}}
\begin{Declaration}{\Macro{textuls}\Parameter{Text}}
\begin{Declaration}{\Macro{univrs}}
\begin{Declaration}{\Macro{texturs}\Parameter{Text}}
\begin{Declaration}{\Macro{univbs}}
\begin{Declaration}{\Macro{textubs}\Parameter{Text}}
\begin{Declaration}{\Macro{univxs}}
\begin{Declaration}{\Macro{textuxs}\Parameter{Text}}
\begin{Declaration}{\Macro{dinbn}}
\begin{Declaration}{\Macro{textdbn}\Parameter{Text}}
\settowidth{\tempdim}{\Macro{textuln}\Parameter{Text}}%
\addtolength{\tempdim}{\dimexpr 2\tabcolsep+2\arrayrulewidth-\textwidth}%
\printdeclarationlist(%
  \begin{minipage}{-\tempdim}%
  \centering%
  \begin{tabularm}{3}%
    \toprule%
    \textbf{Schriftart}                  & \textbf{Schalter}
      & \textbf{Textkommando}\tabularnewline
    \midrule
    \textuln{Univers 45 Light}           & \Macro{univln}{}
      & \Macro{textuln}\Parameter{Text}\tabularnewline
    \texturn{Univers 55 Regular}         & \Macro{univrn}{}
      & \Macro{texturn}\Parameter{Text}\tabularnewline
    \textubn{Univers 65 Bold}            & \Macro{univbn}{}
      & \Macro{textubn}\Parameter{Text}\tabularnewline
    \textuxn{Univers 75 Black}           & \Macro{univxn}{}
      & \Macro{textuxn}\Parameter{Text}\tabularnewline
    \textuls{Univers 45 Light Oblique}   & \Macro{univls}{}
      & \Macro{textuls}\Parameter{Text}\tabularnewline
    \texturs{Univers 55 Regular Oblique} & \Macro{univrs}{}
      & \Macro{texturs}\Parameter{Text}\tabularnewline
    \textubs{Univers 65 Bold Oblique}    & \Macro{univbs}{}
      & \Macro{textubs}\Parameter{Text}\tabularnewline
    \textuxs{Univers 75 Black Oblique}   & \Macro{univxs}{}
      & \Macro{textuxs}\Parameter{Text}\tabularnewline
    \DIN & \Macro{dinbn}{}
      & \Macro{textdbn}\Parameter{Text}\tabularnewline
    \bottomrule%
    \allcolumnpar{\footnotesize\vskip0pt%
       Die Schrift \DIN darf laut \CD nur mit Majuskeln (Großbuchstaben) 
       verwendet werden. Wird diese Schrift manuell verwendet, sollte dies mit 
       \Macro{MakeTextUppercase}\PParameter{\Macro{textdbn}\Parameter{Text}}  
       geschehen. Sollen dabei im Argument einzelne Teile zwingend klein 
       geschrieben werden, wird der Befehl \Macro{NoCaseChange} benötigt.
    }
  \end{tabularm}%
  \end{minipage}%
)%
\index{Schrift!Befehle}\index{Schrift!Schalter}%
%
Unabhängig davon, welche Schriftfamilie verwendet wird, können die Schriften 
des \CDs jederzeit entweder mit einem Textschalter oder mit einem Textkommando
innerhalb des Dokumentes genutzt werden. Ein Textschalter wirkt sich~-- wenn er 
nicht in einer Gruppe oder einer Umgebung verwendet und damit lokal begrenzt 
wird~-- global auf das Dokument aus. Bei einem Textkommando hingegen erfolgt 
die Änderung der Schriftart nur für das angegebene Argument. Deshalb ist die 
Verwendung der letzteren Variante vorzuziehen.
\end{Declaration}
\end{Declaration}
\end{Declaration}
\end{Declaration}
\end{Declaration}
\end{Declaration}
\end{Declaration}
\end{Declaration}
\end{Declaration}
\end{Declaration}
\end{Declaration}
\end{Declaration}
\end{Declaration}
\end{Declaration}
\end{Declaration}
\end{Declaration}
\end{Declaration}
\end{Declaration}

\subsection{Schriften für den Mathematiksatz}
\begin{Declaration}{\Option{sansmath}[\PBoolean]}%
  [true][\Option{cdfont}[false]:false]
\printdeclarationlist%
\label{sec:math}
\index{Schrift!Mathematiksatz}\index{Mathematiksatz|!}
\index{Schrift!Griechische Buchstaben}\index{Griechische Buchstaben}
%
Diese Option dient zur Verwendung serifenloser Mathematikschriften. Dafür 
werden zum einen die griechischen Buchstaben aus \Package{cmbright} und zum 
anderen die Symbole aus \Package{iwona} verwendet. Für die lateinischen 
Buchstaben wird \Univers genutzt. Ein Umschalten auf Serifenlose und zurück 
innerhalb des Dokumentes ist~-- beispielsweise in einer Abbildung oder in einer 
Tabelle~-- durch \Macro{TUDoptions}\PParameter{\Option{sansmath}[true]} und 
\Macro{TUDoptions}\PParameter{\Option{sansmath}[false]} möglich. Mit
\Macro{boldmath} kann auf fette Mathematikschriften umgeschaltet werden.

Mit der Einstellung \Option{sansmath}[false] wird auf die Standardschriften
für den Mathematikmodus zurückgeschaltet. Sollen stattdessen andere serifenlose 
Mathematikschriften genutzt werden, so sei auf \Package{sansmath}, 
\Package{sansmathfonts}, \Package{mathastext}, \Package{sfmath} sowie 
\Package{sansmathaccent} verwiesen.
%
\begin{values}
\itemfalse
  Es werden die normalen \hologo{LaTeX}"=Serifenschriften beziehungsweise die 
  Schriften beliebig nutzbarer Pakete für den Mathematiksatz verwendet.
\itemtrue*
  Die serifenlose Mathematikschriften werden aktiviert.
\end{values}
\end{Declaration}

\subsubsection{Griechischen Buchstaben}
\label{sec:greek}
\index{Griechische Buchstaben}%\index{Griechische Buchstaben!Neigung}
%
\begin{Declaration}{\Macro{varDelta}}
\begin{Declaration}{\Macro{varTheta}}
\begin{Declaration}{\Macro{varLambda}}
\begin{Declaration}{\Macro{varXi}}
\begin{Declaration}{\Macro{varPi}}
\begin{Declaration}{\Macro{varSigma}}
\begin{Declaration}{\Macro{varUpsilon}}
\begin{Declaration}{\Macro{varPhi}}
\begin{Declaration}{\Macro{varPsi}}
\begin{Declaration}{\Macro{varOmega}}
\begin{Declaration}{\Macro{upDelta}}
\begin{Declaration}{\Macro{upTheta}}
\begin{Declaration}{\Macro{upLambda}}
\begin{Declaration}{\Macro{upXi}}
\begin{Declaration}{\Macro{upPi}}
\begin{Declaration}{\Macro{upSigma}}
\begin{Declaration}{\Macro{upUpsilon}}
\begin{Declaration}{\Macro{upPhi}}
\begin{Declaration}{\Macro{upPsi}}
\begin{Declaration}{\Macro{upOmega}}
\index{Schrift!Griechische Buchstaben}\index{Griechische Buchstaben}%
\settowidth{\tempdim}{\Macro{varUpsilon}}%
\addtolength{\tempdim}{\dimexpr 2\tabcolsep+2\arrayrulewidth-\textwidth}%
\printdeclarationlist(%
  \begin{minipage}{-\tempdim}%
    \newcommand\tablecontent{}%
    \newcommand*\greekLetters{%
      Delta,Theta,Lambda,Xi,Pi,Sigma,Upsilon,Phi,Psi,Omega%
    }%
    \def\do#1{\appto\tablecontent{%
      \Macro*{var#1} & $\csuse{var#1}$ & & 
      \Macro*{up#1} & $\csuse{up#1}$\tabularnewline
    }}%
    \expandafter\docsvlist\expandafter{\greekLetters}%
    \centering%
    \vspace{\intextsep}\noindent
    \begin{tabularm}{5}
      \toprule%
      \textbf{Befehl (kursiv)} & \textbf{Symbol} & &
      \textbf{Befehl (aufrecht)} & \textbf{Symbol}
      \tabularnewline\midrule\tablecontent\bottomrule%
      \allcolumnpar{\footnotesize\vskip0pt%
        Die Befehle \Macro*{up}\PName{Name} und \Macro*{var}\PName{Name}
        werden normalerweise durch einige Pakete, unter anderem auch von 
        \Package{cmbright} oder \Package{amsmath}, bereitgestellt.
      }
    \end{tabularm}
  \end{minipage}%
)%
%
Griechische Majuskeln werden sowohl in aufrechter als auch in geneigter Form 
bereitgestellt. Unabhängig von den beiden Optionen \Option{sansmath} und 
\Option{slantedgreek} können sowohl kursive als auch aufrechte griechischen 
Großbuchstaben im Mathematikmodus direkt verwendet werden. Dies ist nützlich, 
um zwischen kursiven Variablen und aufrechten Konstanten zu unterscheiden. Die 
griechischen Minuskeln sind leider nur in der kursiven Variante verfügbar.
\end{Declaration}
\end{Declaration}
\end{Declaration}
\end{Declaration}
\end{Declaration}
\end{Declaration}
\end{Declaration}
\end{Declaration}
\end{Declaration}
\end{Declaration}
\end{Declaration}
\end{Declaration}
\end{Declaration}
\end{Declaration}
\end{Declaration}
\end{Declaration}
\end{Declaration}
\end{Declaration}
\end{Declaration}
\end{Declaration}

\begin{Declaration}{\Option{slantedgreek}[\PBoolean]}%
  [true][\Option{cdfont}[false]:false]
\printdeclarationlist%
\index{Schrift!Griechische Buchstaben}\index{Griechische Buchstaben}%
\index{Griechische Buchstaben!Neigung}%
%
Die Option ändert die standardmäßige Neigung der griechischen Großbuchstaben im 
Mathematikmodus bei der Verwendung der Befehle \Macro*{Delta}, \Macro*{Theta}, 
\Macro*{Lambda}, \Macro*{Xi}, \Macro*{Pi}, \Macro*{Sigma}, \Macro*{Upsilon}, 
\Macro*{Phi}, \Macro*{Psi} und \Macro*{Omega}. Wie unabhängig von der Option 
\Option{slantedgreek} gezielt kursive und aufrechte Buchstaben gesetzt werden 
können, ist \vpageref{sec:greek} beschrieben.
%
\begin{values}
\itemfalse
  Die griechischen Majuskeln werden wie bei den Standardklassen aufrecht 
  gesetzt.
\itemtrue*
  Die Ausgabe der griechischen Großbuchstaben erfolgt kursiv.
\end{values}
\end{Declaration}

\subsubsection{Zusätzliche Hinweise zum Mathematiksatz}
Weitere Hinweise zum typographisch guten Mathematiksatz sind außerdem in 
\autoref{sec:exmpl:mathswap} sowie \autoref{sec:exmpl:mathtype} zu finden.


\subsection{Die Schriften des \CDs im OpenType-Format}
\label{sec:fonts:fontspec}
\index{OpenType-Schriften}
%
\ChangedAt{v2.02!OpenType-Schriften mit \Option{fontspec} verwendbar}
Das \TUDScript-Bundle unterstützt die Verwendung der Schriften des \CDs sowohl 
im PostScript- als auch im OpenType"=Format. Sind die OpenType"=Schriften über 
das Betriebssystem installiert, können diese mit dem Paket \Package{fontspec} 
eingebunden werden. Wäre dies ohne Probleme möglich, wäre die Installation der 
PostScript"=Schriften mithilfe eines Skriptes damit obsolet. Allerdings sind 
einerseits für die Kompilierung eines Dokumentes über den klassischen Prozess 
via \Path{latex \textrightarrow{} dvips \textrightarrow{} ps2pdf}~-- wie es 
beispielsweise für die Erstellung von Grafiken mit \Package{pstricks} notwendig 
ist~-- die Schriften im PostScript"=Format nötig. Andererseits liefern die 
Schriftfamilien des \CDs keinerlei mathematische Glyphen, sodass diese bei der 
PostScript"=Schriftinstallation aus den Schriftpaketen \Package{cmbright} und 
\Package{iwona} entnommen werden müssen.

Die Verwendung der Schriften des \CDs im OpenType"=Format sollte folglich nur 
genutzt werden, wenn eine Installation der PostScript"=Schriften \emph{absolut} 
nicht möglich ist. Sollten die PostScript"=Schriften installiert sein, gibt es 
auch beim Einsatz von \hologo{LuaLaTeX} oder \hologo{XeLaTeX} keinen triftigen 
Grund, die Option \Option{fontspec} zu verwenden.

\begin{Declaration}[v2.02]{\Option{fontspec}[\PBoolean]}[false]%
\printdeclarationlist%
%
Wird die Option aktiviert, werden die OpenType"=Varianten von \Univers und \DIN 
anstelle der PostScript"=Schriften verwendet. Diese sollte nur in absoluten
Ausnahmefällen genutzt werden. Hierfür müssen die OpenType"=Schriften auf dem 
Betriebssystem installiert sein.
\begin{values}
\itemfalse*
  Die Hausschriften im Stil des \CDs der \TnUD werden im PostScript"=Format 
  eingebunden. Sowohl Kerning als auch der mathematische Satz funktionieren 
  problemlos.
\itemtrue*
   Es werden die OpenType"=Varianten der Hausschriften verwendet. Dazu wird das 
   Paket \Package{fontspec} geladen, welches lediglich mit \hologo{LuaLaTeX} 
   oder \hologo{XeLaTeX} jedoch nicht mit \hologo{pdfLaTeX} als genutzt werden 
   kann. Sowohl beim mathematischen Satz als auch beim Kerning der Schriften 
   kann es zu Problemen kommen. Die Verwendung dieser Einstellung sollte nur 
   erfolgen, wenn eine Installation der PostScript"=Schriften nicht möglich ist.
\end{values}
\end{Declaration}


\section{Das Layout des \CDs}
Das Hauptaugenmerk der neuen Klassen liegt auf der Umsetzung des \CDs der
\TnUD für \hologo{LaTeX}. Ein großer Teil der definierten Optionen und Befehle
dient genau dazu und wird folgend beschrieben.

Einige spezielle Seiten werden im prägnanten Stil mit dem Logo der \TnUD und 
der dazugehörigen Kopfzeile mit Querbalken gesetzt. Dies betrifft insbesondere 
\hyperref[sec:title]{die Umschlagseite und den Titel in \autoref{sec:title}}, 
die \hyperref[sec:part]{Teileseiten in \autoref{sec:part}} sowie die
\hyperref[sec:chapter]{Kapitelseiten in \autoref{sec:chapter}}. 
Außerdem können entweder mit den \PageStyle{tudheadings}"=Seitenstilen oder 
mit der \Environment{tudpage}-Umgebung aus \autoref{sec:tudheadings} eigene 
Seiten im selben Stil erzeugt werden. Wird das Paket \Package{tudscrsupervisor} 
verwendet und mit den entsprechenden Befehlen oder Umgebungen aus diesem eine 
Aufgabenstellung, ein Gutachten oder ein Aushang erstellt, so erscheinen auch 
diese in besagtem Seitenstil.


\subsection{Das Erscheinungsbild von Titel, Teilen und Kapiteln}
\begin{Declaration}{\Option{cd}[\PSet]}[true]
\printdeclarationlist%
\index{Layout}%
%
Diese Option bestimmt, ob und wie das \CD der \TnUD verwendet wird. Sie hat
Einfluss auf die Ausprägung für Titel"~, Teil"~, und Kapitelseiten.
%
\begin{values}
\itemfalse
  Diese Einstellung erzeugt das Standard"=Verhalten der \KOMAScript"=Klassen, 
  es wird kein \CD genutzt.
\itemtrue*[standard/simple/monochrom]
  Das Layout für Titel"~, Teil"~ und Kapitelseiten ist im \CD, es wird 
  schwarze Schrift für Titel, Teil"~ und Kapitelüberschriften sowie im 
  Seitenkopf verwendet.
\item[lite/light/pale]
  Die Einstellung entspricht weitestgehend der Option \Option{cd}[true], 
  allerdings wird die primäre Hausfarbe \Color{HKS41} anstelle schwarzer 
  Schrift genutzt.
\item[color/colour/full]
  Der Titel sowie Teil"~ und Kapitelseiten werden allesamt farbig und im \CD 
  gestaltet, der Seitenkopf wird in der primären Hausfarbe \Color{HKS41} 
  gesetzt.
\end{values}
\end{Declaration}

\begin{Declaration}[v2.02]{\Length{pageheadingsvskip}}
\begin{Declaration}[v2.02]{\Length{headingsvskip}}
\printdeclarationlist%
\index{Kapitelseiten}\index{Layout!Kapitelseiten}\index{Überschriften!Position}%
%
Diese beiden Längen haben Auswirkung auf die vertikale Position verschiedener
Überschriften. Mit \Length{pageheadingsvskip} können sowohl der Titel auf der 
Titelseite (\Option{titlepage}[true]) als auch die Überschriften von Teilen und 
Kapiteln, welche als einzelne Kapitelseite (\Option{chapterpage}[true]) gesetzt 
werden, verschoben werden. Demgegenüber erlaubt es \Length{headingsvskip}, 
sowohl den Titel innerhalb eines Titelkopfes (\Option{titlepage}[false]) als 
auch die Überschrift eines Kapitels bei deaktivierter Kapitelseite 
(\Option{chapterpage}[false]) in ihrer vertikalen Position anzupassen.

Normalerweise werden alle der zuvor genannten Überschriften im Layout relativ 
tief im Textbereich gesetzt. Mit negativen Werten wird diese nach oben 
verschoben, positive Werte setzen diese dementsprechend tiefer. Beim 
Verschieben nach oben, sollte darauf geachtet werden, dass diese sich danach 
noch innerhalb des Satzspiegels befinden, da dies \emph{nicht} automatisch 
durch die Hauptklassen überprüft wird.
\end{Declaration}
\end{Declaration}

\subsubsection{Einstellungen für Titel, Umschlagseite, Teile und Kapitel}
Das Verhalten aller Elemente%
\footnote{%
  Titel (\Macro{maketitle}), Umschlagseite (\Macro{makecover}),
  Teileseite (\Macro{part}, \Macro{addpart}),
  Kapitelseite (\Macro{chapter}, \Macro{addchap})%
}
wird normalerweise von der Option \Option{cd}[\PSet] bestimmt. Bedarfsweise 
können einzelne Elemente andererseits auch individuell mit abweichenden 
Wertzuweisungen angepasst werden. Soll ein bestimmtes Elemente des Layouts 
anders erscheinen als der Rest des Dokumentes, so kann der entsprechende Wert 
mithilfe der folgenden Optionen überschrieben werden. Die gültigen 
Wertzuweisungen für die einzelnen Elemente entsprechend dabei den möglichen 
Werten für die Option \Option{cd}.

\begin{Declaration}{\Option{cdtitle}[\PSet]}
\printdeclarationlist%
\index{Titel}\index{Layout!Titel}%
%
Mit \Option{cdtitle} kann der Wert des Schlüssels \Option{cd} für die 
Titelseite überschrieben werden. Es kann zwischen dem Standardtitel~-- welcher 
durch \KOMAScript{} bereitgestellt wird~-- und dem Titel im \CD umgeschaltet 
werden. Die neue Titelseite unterstützt alle durch \KOMAScript{} definierten 
Befehle für den Titel.%
\footnote{\raggedright%
  \Macro{extratitle}\Parameter{Schmutztitel},\Macro{titlehead}\Parameter{Kopf},
  \Macro{subject}\Parameter{Typisierung},\Macro{title}\Parameter{Titel},
  \Macro{subtitle}\Parameter{Untertitel},\Macro{author}\Parameter{Autor},
  \Macro{date}\Parameter{Datum},\Macro{publishers}\Parameter{Verlag},
  \Macro{and} und \Macro{thanks}\Parameter{Fußnote} sowie
  \Macro{uppertitleback}\Parameter{Titelrückseitenkopf},
  \Macro{lowertitleback}\Parameter{Titelrückseitenfuß}
  und \Macro{dedication}\Parameter{Widmung}
}
Zusätzlich werden viele neue Felder für den Titel definiert, welche vor allem 
für den Titel einer wissenschaftlichen Arbeit von Relevanz sind. Genaueres dazu 
ist in \autoref{sec:title} nachzulesen. Unabhängig von der gewählten Variante 
der Titelseite wird diese immer mit \Macro{maketitle} erzeugt.
\end{Declaration}

\begin{Declaration}[v2.02]{\Option{cdcover}[\PSet]}
\printdeclarationlist%
\index{Umschlagseite|!}%
\index{Titel!Umschlagseite}\index{Layout!Umschlagseite}%
%
Die \TUDScript-Klassen führen zusätzlich den Befehl \Macro{makecover} ein, mit 
dem sich neben dem Titel eine separate Umschlagseite erzeugen lässt. Diese ist 
in ihrer Gestalt der Titelseite sehr ähnlich, wird normalerweise jedoch in 
einem anderen Satzspiegel als dem des Buchblocks gesetzt. Mit der Option 
\Option{cdcover} kann~-- unabhängig von \Option{cd}~-- das Erscheinungsbild 
der Umschlagseite geändert werden. Wird \Option{cdcover}[false] gewählt, 
entspricht die Umschlagseite dem originalen \KOMAScript-Titel. Die Verwendung 
des Befehls \Macro{makecover} sowie die dazugehörigen Parameter werden 
detailliert in \autoref{sec:title} erläutert.
\end{Declaration}

\begin{Declaration}{\Option{cdpart}[\PSet]}
\printdeclarationlist%
\index{Teileseiten}\index{Layout!Teileseiten}%
%
Für die Teileseiten kann der Wert des Schlüssels \Option{cd} separat 
überschrieben und somit deren Layout%
\footnote{\label{fn:layout}%
  \KOMAScript"=Layout beziehungsweise monochromes oder farbiges 
  Erscheinungsbild im \CD%
}
beeinflusst werden, welches bei der Benutzung der Befehle \Macro{part} 
beziehungsweise \Macro{addpart} und deren Sternversionen genutzt wird.
\end{Declaration}

\begin{Declaration}{\Option{cdchapter}[\PSet]}
\printdeclarationlist%
\index{Kapitelseiten}\index{Layout!Kapitelseiten}%
%
Für Kapitelseiten kann der Schlüsselwert \Option{cd} ebenfalls angepasst und 
damit das Erscheinungsbild\footref{fn:layout} geändert werden, das bei der 
Verwendung von \Macro{chapter} beziehungsweise \Macro{addchap} und den 
dazugehörigen Sternversionen genutzt wird.
\end{Declaration}
%
\begin{Example}
Soll die Titelseite in Farbe, der Rest des Dokumentes allerdings in schwarzer 
Schrift gesetzt werden, so kann dies folgendermaßen erreicht werden:
\begin{Code}[escapechar=§]
\documentclass[cd=true,cdtitle=color]{§\PName{Dokumentklasse}§}
\end{Code}
\end{Example}

\subsubsection{Vakatseiten/Leerseiten}
\index{Leerseiten}%
Automatisch erzeugte Vakatseiten~-- auch absichtliche Leerseiten genannt~-- 
findet man in Dokumenten mit den aktivierten Optionen \Option{twoside} und 
\Option{open}[right]\footnote{Standard bei \Class{tudscrbook}} beziehungsweise 
\Option{open}[left] beim Beginn von Teilen und Kapiteln. Für diese kann der 
Seitenstil mit der \KOMAScript"=Option \Option{cleardoublepage} eingestellt 
werden.

\begin{Declaration}{\Option{cleardoublespecialpage}[\PSet]}[true]%
\printdeclarationlist%
\index{Teileseiten}\index{Layout!Teileseiten}%
\index{Kapitelseiten}\index{Layout!Kapitelseiten}%
\index{Satzspiegel!doppelseitig}\index{Layout!Rückseiten}%
%
Diese Option wirkt sich lediglich bei aktiviertem doppelseitigem Satz und 
ausschließlich rechts eröffnenden Seiten für Teile beziehungsweise Kapitel
aus.%
\footnote{\Option{twoside} und \Option{open}[right]}
In diesem Fall kann der Stil der darauffolgenden, linken Seite~-- sprich der 
Rückseite~-- beeinflusst werden. Das Normalverhalten sieht vor, dass nach einem 
Teil die Rückseite unabhängig von der Einstellung für \Option{cleardoublepage} 
immer als vollständig leere Seite ohne Kopf"~ oder Fußzeilen gesetzt wird.

Diese Einstellung erlaubt es, dieses Normalverhalten zu deaktivieren und für 
die Seite nach der Teileseite~-- und abhängig von \Option{chapterpage} 
auch nach einem Kapitelanfang auf einer separaten Seite~-- den Seitenstil der 
Option \Option{cleardoublepage} zu übernehmen. Des Weiteren kann auch ein 
anderer, beliebiger, bereits definierter Seitenstil gewählt werden. Außerdem
kann im farbigen Layout die Rückseite in der gleichen Farbe wie die 
Vorderseite von Teil oder Kapitel gesetzt werden. \notudscrartcl
%
\begin{values}
\itemfalse
  Die Rückseiten sind vollständig leere Seiten, unabhängig von Option
  \Option{cleardoublepage}.
\itemtrue*
  Der Seitenstil der Rückseite von Teilen und gegebenenfalls Kapiteln 
  entspricht der Einstellung von \Option{cleardoublepage} für Vakatseiten.
\item[current]
  Es wird der aktuell definierte Seitenstil (\Macro{pagestyle}) für die 
  erzeugte Rückseite verwendet.
\item[color/colour]
  Im farbigen Layout ist auch die Rückseite von Teilen und Kapiteln farbig, 
  siehe \Option{clearcolor}.
\makeatletter\item@values[\PName{Seitenstil}\textsl{:}]\makeatother
  Mit der Angabe von \Option{cleardoublespecialpage}[\PName{Seitenstil}] 
  kann ein beliebiger, bereits definierter Seitenstil für die Rückseite nach 
  Teilen und Kapiteln verwendet werden.
\end{values}
\end{Declaration}

\begin{Declaration}{\Option{clearcolor}[\PBoolean]}[false]%
\printdeclarationlist%
\index{Titel}\index{Layout!Titel}%
\index{Teileseiten}\index{Layout!Teileseiten}%
\index{Kapitelseiten}\index{Layout!Kapitelseiten}%
\index{Satzspiegel!doppelseitig}\index{Leerseiten}%
%
Sollten beim farbigen Layout die Optionen \Option{twoside} sowie auch
\Option{open}[right] gesetzt sein, so werden beim Aktivieren dieser Option die 
Rückseiten von Teilen~-- und je nach Einstellung von \Option{chapterpage} 
gegebenenfalls auch von Kapiteln~-- farbig gesetzt.%
\footnote{%
  Dies führt bei der Ausgabe zu farbigen Blättern (Vorder- und Rückseite) der 
  entsprechenden Elemente des Layouts.
}
Die Option wirkt sich ebenfalls auf die Rückseite des Titels aus.%
\footnote{%
  siehe \Macro{uppertitleback} und \Macro{lowertitleback} der 
  \KOMAScript"=Dokumentation (\scrguide*)
}
Der Stil dieser zusätzlich eingefügten Rückseiten ist abhängig von der Option
\Option{cleardoublespecialpage}.
%
\begin{values}
\itemfalse
  Es werden weiße Rückseiten bei Titel, Teilen und gegebenenfalls Kapiteln 
  erzeugt.
\itemtrue*
  Die rückwärtigen Seiten der genannten Elemente des Layouts sind farbig.
\end{values}
\end{Declaration}

\subsection{Seiten im Stil des \CDs}
\begin{Declaration}[v2.02]{\PageStyle{tudheadings}}
\begin{Declaration}[v2.02]{\PageStyle{plain.tudheadings}}
\begin{Declaration}[v2.02]{\PageStyle{empty.tudheadings}}
\printdeclarationlist%
\label{sec:tudheadings}
%
Ein zentrales Element des \CDs der \TnUD ist der eingeführte prägnante 
Seitenkopf mit der Angabe von Fakultät (\Macro{faculty}), Einrichtung 
(\Macro{department}), Institut (\Macro{institute}) und Lehrstuhl 
(\Macro{chair}) in der dazugehörigen Kopfzeile mit Querbalken. Durch die 
Verwendung von \Package{scrlayer-scrpage} lassen sich einzelne Seiten oder auch 
ganze Dokumente sehr einfach in diesem Stil setzen. Hierzu muss lediglich einer 
der Seitenstile mit \Macro{pagestyle}\Parameter{Seitenstil} geladen werden. 

Allen Seitenstilen ist der typische Kopf gemein. Der Fuß des Seitenstils 
\PageStyle{empty.tudheadings} ist immer leer, \PageStyle{tudheadings} und 
\PageStyle{plain.tudheadings} übernehmen die Einstellungen für den Fuß aus der 
Anwenderschnittstelle von \Package{scrlayer-scrpage}.%
\footnote{%
  Es können die Befehle \Macro{lefoot}, \Macro{cefoot} und \Macro{refoot} sowie 
  \Macro{lofoot}, \Macro{cofoot} und \Macro{rofoot} verwendet werden.
}
Wie diese zu verwenden ist, ist der \KOMAScript"=Anleitung zu entnehmen. 
Alternativ zu einer eigenen Definition der Fußzeile kann außerdem die Option 
\Option{cdfoot} verwendet werden.

Sobald einer der neu definierten Seitenstile aktiviert wurde, sind diese 
zusätzlich unter den Namen \PageStyle*{headings}, \PageStyle*{plain} und 
\PageStyle*{empty} verwendbar. Das hat den Vorteil, dass bei Optionen oder 
Befehlen, die automatisch zwischen \PageStyle*{headings}, \PageStyle*{plain} 
und \PageStyle*{empty} umschalten, durch die einmalige Auswahl von einem der 
Stile \PageStyle{tudheadings}, \PageStyle{plain.tudheadings} oder 
\PageStyle{empty.tudheadings} nun zwischen diesen Stilen umgeschaltet wird. Um 
auf das normale Verhalten zurückzuschalten, muss einer der beiden Seitenstile 
\PageStyle*{scrheadings} oder \PageStyle*{plain.scrheadings} aktiviert werden.
\end{Declaration}
\end{Declaration}
\end{Declaration}

\begin{Declaration}{\Macro{faculty}\Parameter{Fakultät}}
\begin{Declaration}{\Macro{department}\Parameter{Einrichtung}}
\begin{Declaration}{\Macro{institute}\Parameter{Institut}}
\begin{Declaration}{\Macro{chair}\Parameter{Lehrstuhl}}
\begin{Declaration}{\Macro{extraheadline}\Parameter{Textzeile}}
\printdeclarationlist%
\index{Kopfzeile}\index{Layout!Kopfzeile}\index{Kopfzeile!Felder}%
\index{Querbalken}\index{Layout!Querbalken}\index{Querbalken!Felder}%
%
Für den Seitenstil des \CDs der \TnUD typisch ist die Kopfzeile mit dem 
charakteristischen Querbalken. In dieser wird~-- falls angegeben~-- in fetter 
Schrift die Fakultät ausgegeben, danach folgen durch Kommas getrennt die 
Einrichtung, das Institut und der Lehrstuhl beziehungsweise die Professur. 
Sollte der Platz in der ersten Zeile nicht ausreichen, erfolgt ein 
automatischer Zeilenumbruch.

In besonderen Ausnahmefällen erlaubt das \CD die Angabe einer zusätzlichen
zweiten beziehungsweise dritten Zeile, welche weitere, frei wählbare Angaben 
enthält. Diese kann mit dem Befehl \Macro{extraheadline}\Parameter{Textzeile} 
definiert werden.
\end{Declaration}
\end{Declaration}
\end{Declaration}
\end{Declaration}
\end{Declaration}

\begin{Declaration}[%
  v2.02!\protect\DDC-Logo automatisch in Kopf oder Fuß%
]{\Option{ddc}[\PSet]}[false]
\begin{Declaration}[v2.02]{\Option{ddchead}[\PSet]}[false]
\begin{Declaration}[%
  v2.02!neue Werte für die Farbwahl des Logos von \protect\DDC%
]{\Option{ddcfoot}[\PSet]}[false]
\printdeclarationlist%
\index{Zweitlogo}\index{Layout!Zweitlogo}\index{\DDC-Logo}%
%
Diese Optionen fügen das Logo von \DDC entweder im Kopf oder im Fuß der Seiten
mit dem Stil \PageStyle{tudheadings} ein. Mit \Option{ddc} wird dieses 
automatisch entweder im Kopf oder~-- falls ein Zweitlogo mit \Macro{headlogo} 
angegeben wurde~-- im Fuß gesetzt. Die anderen beiden Optionen setzen das Logo 
zwingend entweder im Kopf (\Option{ddchead}) oder im Fuß (\Option{ddcfoot}), 
wobei erstgenannte ein optionales Zweitlogo dabei unterdrückt. Die Verwendung 
einer der drei Optionen führt zur Deaktivierung der anderen beiden, sie 
schließen sich folglich gegenseitig aus. Die möglichen Werte für diese Optionen 
sind:
%
\begin{values}
\itemfalse
  Bei den \PageStyle{tudheadings}-Seitenstile erscheint kein Logo von \DDC.
\itemtrue*
  Das Logo von \DDC wird im Kopf beziehungsweise im Fuß verwendet. Die Wahl der 
  Farbe des Logos geschieht passend zur farblichen Ausprägung der Seite selbst.
\end{values}
%
Soll die Farbe des \DDC-Logos manuell erfolgen, können folgende Werte verwendet 
werden:
%
\begin{values}
\item[color/colour]
  Im Kopf oder Fuß wird die achtfarbige 4C"~Variante des \DDC-Logos genutzt.
\item[colorblack/colourblack]
  Das Logo wird in der achtfarbige 4C"~Variante mit schwarzem \DDC-Schriftzug 
  anstelle des grauen. Für den Fuß wird der grüne Claim ebenfalls durch einen 
  schwarzen ersetzt. Dies ist insbesondere für kleine Darstellungen des Logos 
  im Fuß sinnvoll.
\item[gray/grey/cdgray/cdgrey]
  Dies Ausgabe des \DDC-Logos erfolgt in Graustufen.
\item[black]
  Verwendung des Logos in Graustufen mit schwarzem Schriftzug.
\item[blue/cddarkblue]
  Der Schriftzug und das Logo werden in der primären Hausfarbe \Color{HKS41} 
  und den entsprechenden Abstufungen gesetzt
\item[white]
  Das \DDC-Logo sowie der dazugehörige Schriftzug sind vollständig weiß.
\end{values}
%
\end{Declaration}
\end{Declaration}
\end{Declaration}

\begin{Declaration}{%
  \Macro{headlogo}\LParameter\Parameter{Dateiname}%
}
\printdeclarationlist%
\index{Zweitlogo|?}\index{Layout!Zweitlogo}\index{\DDC-Logo}%
%
Neben dem Logo der \TnUD darf zusätzlich ein Zweitlogo im Kopf verwendet werden.
Dieses lässt sich mit diesem Befehl einbinden. Normalerweise wird es auf die 
Höhe der Erstlogos skaliert. Über das optionale Argument können weitere 
Formatierungsbefehle an den verwendeten Befehl \Macro{includegraphics} 
durchgereicht werden, um beispielsweise die Größe des Zweitlogos anzupassen.
Welche Parameter angepasst werden können, ist der Dokumentation des
\Package{graphicx}-Paketes zu entnehmen.

Sollte die Option \Option{ddc} aktiviert sein, wird das \DDC-Logo nicht im Kopf 
sondern automatisch im Fuß gesetzt. Die Option \Option{ddchead} setzt dieses 
auf jeden Fall im Kopf und überschreibt damit das mit \Macro{headlogo} 
angegebene Zweitlogo.
\end{Declaration}

\begin{Declaration}{\Option{widehead}[\PBoolean]}%
  [false][\Option{cd}[color]:true]%
\printdeclarationlist%
\index{Querbalken}\index{Layout!Querbalken}%
%
Für die \TUDScript-Klassen ist ein Seitenlayout entstanden, welche den Kopf des
\CDs umsetzt. Dieser besteht aus dem Logo der \TnUD sowie einem darunter 
befindlichen Querbalken, in welchem Fakultät, Einrichtung, Institut und 
Lehrstuhl%
\footnote{%
  \Macro{faculty}, \Macro{department}, \Macro{institute} sowie \Macro{chair}%
}
aufgeführt werden können. Bei der Ausprägung dieses Balkens gibt es zwei 
Varianten. Die Außenlinien laufen entweder bis zum Text"~ oder bis zum 
Blattrand.

Für den Fall, dass ein randloser Ausdruck technisch nicht möglich ist, 
kann die letztere der beiden Variante Probleme bereiten. Deshalb kann mit der 
Option \Option{widehead} die Breite des Querbalkens angepasst werden. 
Normalerweise ist der Balken auf die Textbreite begrenzt, lediglich im farbigen 
Layout wird dieser standardmäßig bis zum Blattrand verlängert.
%
\begin{values}
\itemfalse
  Der Querbalken im Kopf erstreckt sich nur über den Textbereich.
\itemtrue*
  Die horizontale Ausdehnung des Querbalkens erstreckt sich bis an den 
  Blattrand.\footnote{Voreinstellung bei \Option{cd}[color]} 
\end{values}
\end{Declaration}

\begin{Declaration}[v2.02]{\Option{barfont}[\PSet]}%
  [true][\Option{cdfont}[false]:false]
\printdeclarationlist%
\index{Kopfzeile!Schrift}%
\index{Layout!Kopfzeile}%
%
Mit dieser Option kann die Schrift im Querbalken der Kopfzeile für Seiten, 
welche in einem der \PageStyle{tudheadings}"=Seitenstilen gesetzt wird, 
beeinflusst werden.
%
\begin{values}
\itemfalse
  Sollte mit \Option{cdfont}[false] die Verwendung der Hausschrift im Stil des 
  \CDs der \TnUD deaktiviert worden sein, wird die Kopfzeile im Querbalken in
  den Serifenlosen der genutzten Schrift gesetzt. Sind die Hausschriften 
  aktiviert, hat diese Einstellung keinen Einfluss.
\itemtrue*[cdfont/light/lightfont/noheavyfont]
  Im Querbalken wird für \Macro{faculty} \textubn{Univers~65~Bold} verwendet, 
  für die Felder \Macro{department}, \Macro{institute}, \Macro{chair} und 
  \Macro{extraheadline} kommt \textuln{Univers~45~Light} zum Einsatz.
\item[heavy/heavyfont]
  Der Inhalt von \Macro{faculty} wird weiterhin in \textubn{Univers~65~Bold} 
  gesetzt, für die restlichen Felder wird \texturn{Univers~55~Regular} genutzt.
\end{values}
\end{Declaration}

\begin{Declaration}[%
  v2.02!Parameter \Key{\Environment{tudpage}}{head} und 
    \Key{\Environment{tudpage}}{foot} entfernt
]{\Environment{tudpage}[\OLParameter{Sprache}]}
\begin{Declaration}{\Key{\Environment{tudpage}}{language}[\PName{Sprache}]}
\begin{Declaration}{\Key{\Environment{tudpage}}{columns}[\PName{Anzahl}]}
\begin{Declaration}{\Key{\Environment{tudpage}}{color}[\PName{Farbe}]}
\begin{Declaration}[v2.02]{\Key{\Environment{tudpage}}{pagestyle}[\PSet]}
\begin{Declaration}{\Key{\Environment{tudpage}}{headlogo}[\PName{Dateiname}]}
\begin{Declaration}[v2.02]{\Key{\Environment{tudpage}}{ddc}[\PSet]}
\begin{Declaration}[v2.02]{\Key{\Environment{tudpage}}{ddchead}[\PSet]}
\begin{Declaration}[v2.02]{\Key{\Environment{tudpage}}{ddcfoot}[\PSet]}
\begin{Declaration}{\Key{\Environment{tudpage}}{cdfont}[\PSet]}
\begin{Declaration}[v2.02]{\Key{\Environment{tudpage}}{barfont}[\PSet]}
\begin{Declaration}{\Key{\Environment{tudpage}}{widehead}[\PBoolean]}
\printdeclarationlist%
\index{Layout}\index{Layout!Seitenstil}%
\index{Kopfzeile}\index{Layout!Kopfzeile}%
\index{Fußzeile}\index{Layout!Fußzeile}%
\index{Schrift}\index{Kopfzeile!Schrift}
%
Diese Umgebung hat ihren Ursprung, als die \PageStyle{tudheadings}"=Seitenstile 
noch nicht verfügbar waren. Mit dieser lassen sich ebenfalls eine oder mehrere 
Seiten innerhalb des Dokumentes mit dem Kopf im \CD setzen. 

Dabei lassen sich verschiedene Parameter als optionales Argument angegeben. 
Wird das Paket \Package{babel} verwendet, kann die Sprache innerhalb der 
Umgebung mit \Key{\Environment{tudpage}}{language}[\PName{Sprache}] geändert 
werden. Dafür muss die gewünschte Sprache entweder als Paketoption oder besser 
noch als Klassenoption angegeben worden sein. Dadurch werden lokal innerhalb 
der Umgebung die Bezeichner und Trennungsmuster sprachspezifisch angepasst. Des 
Weiteren wird das Paket \Package{multicol} unterstützt. Wird dieses geladen, 
wird mit dem Parameter \Key{\Environment{tudpage}}{columns}[\PName{Anzahl}] 
der Inhalt der Umgebung mehrspaltig gesetzt.

Mit dem Parameter \Key{\Environment{tudpage}}{color} kann die Farbe des Kopfes 
auf eine beliebige geändert werden. Diese ist für den Fall eines farbigen 
Layouts (\Option{cd}[pale|color]) auf die primäre Hausfarbe \Color{HKS41} 
gesetzt, sonst ist der Kopf standardmäßig schwarz. Außerdem kann mit
\Key{\Environment{tudpage}}{pagestyle} der Seitenstil angepasst werden. 
Gültige Werte sind \PValue{headings}, \PValue{plain}, \PValue{empty} oder einer 
der \PageStyle{tudheadings}"=Seitenstile. Soll lokal ein anderes Zweitlogo als 
das mit \Macro{headlogo} gegebene erscheinen, so kann der Parameter 
\Key{\Environment{tudpage}}{headlogo}[\PName{Dateiname}] verwendet werden.

Die anderen Parameter entsprechen in ihrem Verhalten prinzipiell den 
gleichnamigen Klassenoptionen, wirken sich jedoch nur lokal innerhalb der 
\Environment{tudpage}"=Umgebung aus. Dies sind namentlich sowohl die Optionen 
\Option{cdfont}'full' sowie \Option{ddc}, \Option{ddchead} und 
\Option{ddcfoot}'full' als auch \Option{barfont} und \Option{widehead}'page'. 
Das Verhalten sowie die jeweils gültigen Wertzuweisungen können in den 
entsprechenden Abschnitten der Dokumentation nachgelesen werden.
\end{Declaration}
\end{Declaration}
\end{Declaration}
\end{Declaration}
\end{Declaration}
\end{Declaration}
\end{Declaration}
\end{Declaration}
\end{Declaration}
\end{Declaration}
\end{Declaration}
\end{Declaration}

\ToDo[imp,nxt]{%
  Titelseite und Cover auf Drittlogos und dergleichen mit scrlayer erweitern.
}[v2.x]


\subsection{Der Titel und die Umschlagseite}
\label{sec:title}%\label{sec:cover}%
\index{Titel|!(}
\index{Umschlagseite|!}%
\index{Titel!Umschlagseite}\index{Layout!Umschlagseite}%
%
Für den Titel werden alle Felder unterstützt, die bereits durch \KOMAScript{} 
bereitgestellt werden. Darüber hinaus werden für die \TUDScript-Klassen weitere 
Felder definiert, die Auswirkungen auf die Gestalt des Titels haben. Diese 
werden nachfolgend in diesem \autorefname erläutert. Der Titel~-- bestehend aus 
möglichem Schmutztitel, der eigentlichen Titelseite und der nachgelagerten 
Elementen~-- kann mit dem Befehl \Macro{maketitle} ausgegeben werden. Außerdem 
kann im zweispaltigen Satz der \Macro{maketitleonecolumn} verwendet werden, 
welcher einen einspaltigen Einfügung nach dem Titel selbst ermöglicht.

Zusätzlich zum Titel lässt sich mit \Macro{makecover} eine Umschlagseite 
erzeugen. Diese kann insbesondere für gebundene Arbeiten verwendet werden. Es 
wird~-- im Vergleich zum Titel~-- lediglich einer reduzierte Anzahl an Feldern 
auf dieser ausgegeben.

\ChangedAt{v2.02}
Für alle Felder des Titels und der Umschlagseite lässt sich die verwendete 
Schrift anpassen. Dabei werden sowohl die bereits durch \KOMAScript{} 
bereitgestellten Schriftelemente \Font{titlehead}, \Font{subject}, 
\Font{title}, \Font{subtitle}, \Font{author}, \Font{date}, \Font{publishers} 
und \Font{dedication} als auch die neuen \Font{titlepage} und \Font{thesis} 
unterstützt.
%
\begin{Example}
In diesem Dokument wurde der Untertitel derart geändert, dass dieser nicht 
standardmäßig in \DIN sondern in \textubn{Univers~65~Bold} ausgegeben wird.
\begin{Code}[escapechar=§]
\addtokomafont{subtitle}{\univbn}
\end{Code}
\end{Example}

\begin{Declaration}[%
  v2.02!Unterstützung der Schriftelemente \Font*{titlehead}{,} 
    \Font*{subject}{,} \Font*{title}{,} \Font*{subtitle}{,} \Font*{author}{,} 
    \Font*{date}{,} \Font*{publishers}{,} \Font*{dedication}{,} 
    \Font*{titlepage} und \Font*{thesis}%
]{\Macro{maketitle}\OLParameter{Seitenzahl}}
\begin{Declaration}[v2.02]{%
  \Key{\Macro{maketitle}}{pagenumber}[\PName{Seitenzahl}]%
}
\begin{Declaration}[v2.02]{\Key{\Macro{maketitle}}{cdfont}[\PSet]}
\printdeclarationlist%
\index{Layout!Titel}%
\index{Satzspiegel!doppelseitig}%
%
Der Befehl \Macro{maketitle} setzt für \Option{cdtitle}[false] den normalen 
\KOMAScript"=Titel{}, ansonsten wird die Titelseite im \CD der \TnUD erzeugt. 
Letztere Variante ist im Vergleich zum Standardtitel um eine Vielzahl von 
Feldern erweitert worden und erlaubt insbesondere die Angabe von Daten für das 
Deckblatt einer akademischen Abschlussarbeit. Die einzelnen Felder werden in 
diesem \autorefname erläutert. Wird das Dokument doppelseitig und mit rechts 
öffnenden Kapiteln gesetzt,%
\footnote{%
  \Option{twoside} und \Option{open}[right], Standard für \Class{tudscrbook}
}
so wird zusätzlich die Option \Option{clearcolor} beachtet.

Das optionale Argument erlaubt~-- ebenso wie bei den \KOMAScript"=Klassen~-- 
die Änderung der Seitenzahl der Titelseite. Diese wird jedoch nicht ausgegeben, 
sondern beeinflusst lediglich die Zählung. Sie sollten hier unbedingt eine 
ungerade Zahl wählen, da sonst die gesamte Zählung durcheinander gerät. 
Zusätzlich kann der Parameter \Key{\Macro{maketitle}}{cdfont} im optionalen 
Argument verwendet werden, um die Nutzung der Schriften des \CDs zu regulieren. 
Er entspricht in seinem Verhalten der gleichnamigen Klassenoption. Die gültigen 
Wertzuweisungen können der Beschreibung der Option \Option{cdfont}'full' 
entnommen werden. Die Einstellungen dieses Parameters wirkt sich nur lokal und 
einzig auf die Umschlagseite aus.
\end{Declaration}
\end{Declaration}
\end{Declaration}

\begin{Declaration}{%
  \Macro{maketitleonecolumn}\OParameter{Seitenzahl}\Parameter{Einspaltentext}%
}
\begin{Declaration}[v2.02]{%
  \Key{\Macro{maketitleonecolumn}}{pagenumber}[\PName{Seitenzahl}]%
}
\begin{Declaration}[v2.02]{\Key{\Macro{maketitleonecolumn}}{cdfont}[\PSet]}
\printdeclarationlist%
\index{Layout!Titel}%
\index{Satzspiegel!doppelseitig}%
\index{Zweispaltensatz}%
%
Im zweispaltigen Satz (\Option{twocolumn}) wird mit \Macro{maketitle} die 
Titelseite selbst immer einspaltig gesetzt. Direkt nach dem Titel folgt 
normalerweise der zweispaltige Fließtext. Die \TUDScript-Klassen ermöglichen 
mit \Macro{maketitleonecolumn}, nach dem Titel zusätzlich auch noch weitere 
Textpassagen~-- beispielsweise eine Zusammenfassung~-- einspaltig zu setzen.

Wird der Befehl bei einer Titelseite (\Option{titlepage}[true]) verwendet, wird 
der Inhalt des Argumentes direkt nach dieser auf einer neuen Seite ebenfalls 
einspaltig ausgegeben. Kommt jedoch ein Titelkopf (\Option{titlepage}[false]) 
zum Einsatz, so folgt nach diesem die einspaltige Textpassage aus dem Argument. 
Danach wird auf das zweispaltige Layout umgeschaltet, 

Der optionale Parameter von \Macro{maketitleonecolumn} kann äquivalent zum 
Befehl \Macro{maketitle} für die Anpassung der Seitenzahl und der verwendeten 
Schrift verwendet werden.
\end{Declaration}
\end{Declaration}
\end{Declaration}

\begin{Declaration}[%
  v2.02!Umschlagseite für Layout ohne \CD hinzugefügt,%
  v2.02!Unterstützung der Schriftelemente \Font*{titlehead}{,} 
    \Font*{subject}{,} \Font*{title}{,} \Font*{subtitle}{,} \Font*{author}{,} 
    \Font*{publishers}{,} \Font*{titlepage} und \Font*{thesis}%
]{\Macro{makecover}\OLParameter{Seitenzahl}}
\begin{Declaration}{\Key{\Macro{makecover}}{cdlayout}[\PBoolean]}
\begin{Declaration}[v2.02]{%
  \Key{\Macro{makecover}}{pagenumber}[\PName{Seitenzahl}]%
}
\begin{Declaration}{\Key{\Macro{makecover}}{cdfont}[\PSet]}
\printdeclarationlist%
%
Eine Umschlagseite wird zumeist für gebundene Abschlussarbeiten verlangt, um 
diese beispielsweise für einen Prägedruck auf dem Buchdeckel zu verwenden. 
Hierfür ist es sinnvoll, mit \Option{cdcover}[true] die farbige Ausprägung der 
Umschlagseite zu deaktivieren, falls diese für das restliche Dokument aktiv ist 
(\Option{cd}[color]).

Wird \Option{cdcover}[true] gewählt, so wird die Umschlagseite im \CD der 
\TnUD gesetzt. Auf dieser werden der Titel des Dokumentes, die Typisierung 
durch \Macro{thesis} und/oder \Macro{subject} sowie der Autor oder respektive 
die Autoren und gegebenenfalls der mit \Macro{publishers} angegebene Verlag 
ausgegeben.
\ChangedAt{v2.02}
Für die Einstellung \Option{cdcover}[false] wird lediglich der normale 
\KOMAScript"=Titel als separate Umschlagseite ausgegeben. 

Die Titelseite selbst gehört immer zum Buchblock und wird daher im gleichen 
Satzspiegel gesetzt. Dem entgegen steht die Umschlagseite, welche zumeist in 
einem anderen Layout erscheint. Normalerweise wird das Cover~-- unabhängig von 
der Option \Option{geometry}~-- im asymmetrischen Satzspiegel des \CDs gesetzt. 
Mit \Key{\Macro{makecover}}{cdlayout}[false] im optionalen Argument kann das 
Verhalten geändert werden. In diesem Fall erscheint auch die Umschlagseite im 
Buchblock des restlichen Dokumentes. Allerdings können für diese Einstellung 
die Seitenränder durch den Nutzer mit den Befehlen \Macro{coverpagetopmargin}, 
\Macro{coverpageleftmargin}, \Macro{coverpagerightmargin} sowie 
\Macro{coverpagebottommargin} frei angepasst werden. Mehr dazu ist im 
\KOMAScript"=Handbuch \scrguide zu finden.

Die beiden anderen optionalen Parameter \Key{\Macro{makecover}}{pagenumber} 
sowie \Key{\Macro{makecover}}{cdfont} dienen~-- äquivalent zum Befehl 
\Macro{maketitle}~-- zur Anpassung der Seitenzahl und der verwendeten Schrift.
\end{Declaration}
\end{Declaration}
\end{Declaration}
\end{Declaration}

\begin{Declaration}{\Macro{extratitle}\Parameter{Schmutztitel}}
\begin{Declaration}{\Macro{titlehead}\Parameter{Kopf}}
\begin{Declaration}{\Macro{title}\Parameter{Titel}}
\begin{Declaration}{\Macro{subtitle}\Parameter{Untertitel}}
\begin{Declaration}{\Macro{publishers}\Parameter{Verlag}}
\begin{Declaration}{\Macro{thanks}\Parameter{Fußnote}}
\begin{Declaration}{\Macro{uppertitleback}\Parameter{Titelrückseitenkopf}}
\begin{Declaration}{\Macro{lowertitleback}\Parameter{Titelrückseitenfuß}}
\begin{Declaration}{\Macro{dedication}\Parameter{Widmung}}
\printdeclarationlist%
\index{Titel!Felder}%
%
Diese Befehle entsprechen den in ihrem Verhalten den originalen Pendants der 
\KOMAScript"=Klassen{} und sollen hier der Vollständigkeit halber erwähnt 
werden.

Die Ausgabe des mit \Macro{extratitle} definierten Schmutztitels~-- welcher 
beliebig gestaltet und formatiert werden kann~-- erfolgt als Bestandteil der 
Titelei mit \Macro{maketitle} vor der eigentlichen Titelseite. Mit dem Befehl 
\Macro{titlehead} kann ein zusätzlicher, beliebig formatierbarer Text oberhalb 
der Typisierung und des Titels ausgegeben werden. Da die vertikale Position des 
Dokumenttitels durch das \CD fest vorgegeben ist, kann es~-- im Gegensatz zu 
den \KOMAScript"=Klassen~-- passieren, dass der Kopf des Haupttitels selbst in 
die Kopfzeile ragt. Dies wird durch die \TUDScript-Klassen nicht geprüft und 
muss gegebenenfalls vom Anwender kontrolliert werden.

Die Befehle \Macro{title} und \Macro{subtitle} bedürfen keiner weiteren 
Erklärung. Anzumerken ist, dass sowohl Titel als auch Untertitel normalerweise 
in Majuskeln und \DIN gesetzt werden. Der mit dem Befehl \Macro{publishers} 
definierte Inhalt muss nicht zwingende einen Verlag bezeichnen sondern kann 
auch andere Informationen beinhalten, welche am Ende der Titelseite ausgegeben 
werden sollen.

Fußnoten werden auf dem Titel nicht mit \Macro{footnote}, sondern mit der 
Anweisung \Macro{thanks} erzeugt. Sie dienen in der Regel für Anmerkungen bei 
Titel oder den Autoren. Als Fußnotenzeichen werden dabei Symbole statt Zahlen 
verwendet. Der Befehl \Macro{thanks} kann nur innerhalb des Arguments einer 
der Anweisungen für die Titelseite wie beispielsweise \Macro{author} oder 
\Macro{title} verwendet werden.

\index{Satzspiegel!doppelseitig}%
Im doppelseitigen Druck lässt sich die Rückseite der Haupttitelseite für 
weitere Angaben nutzen. Sowohl den Titelrückseitenkopf als auch den
Titelrückseitenfuß kann der Anwender mit \Macro{uppertitleback} und 
\Macro{lowertitleback} frei gestalten.

Mit \Macro{dedication} kann eine eigene Widmungsseite zentriert und in etwas 
größerer Schrift gesetzt werden. Die Rückseite ist wie die des Schmutztitels 
grundsätzlich leer. Die Widmung wird zusammen mit der restlichen Titelei durch 
\Macro{maketitle} ausgegeben und muss daher vor dieser Anweisung definiert sein.
\end{Declaration}
\end{Declaration}
\end{Declaration}
\end{Declaration}
\end{Declaration}
\end{Declaration}
\end{Declaration}
\end{Declaration}
\end{Declaration}

\begin{Declaration}{\Macro{titledelimiter}\Parameter{Trennzeichen}}
\printdeclarationlist%
\index{Titel!Felder}\index{Titel!Trennzeichen}%%
%
Für den Titel und die Umschlagseite werden durch die \TUDScript-Klassen
eine Reihe von zusätzlichen Feldern bereitgestellt. Einigen dieser Felder wird 
eine Beschreibung (siehe dazu \autoref{sec:localization}) vorangestellt. 
Dazwischen wird bei der Ausgabe ein Trennzeichen eingefügt. Ein Doppelpunkt 
gefolgt von einem Leerzeichen (:\Macro*{nobreakspace}) ist hierfür die 
Voreinstellung. Mit dem Befehl \Macro{titledelimiter} lässt sich dieses 
Trennzeichen beliebig anpassen.
\end{Declaration}


\begin{Declaration}{\Macro{author}\Parameter{Autor(en)}}
\begin{Declaration}{\Macro{authormore}\Parameter{Autorenzusatz}}
\begin{Declaration}{\Macro{dateofbirth}\Parameter{Geburtsdatum}}
\begin{Declaration}{\Macro{placeofbirth}\Parameter{Geburtsort}}
\begin{Declaration}{\Macro{matriculationnumber}\Parameter{Matrikelnummer}}
\begin{Declaration}{\Macro{matriculationyear}\Parameter{Immatrikulationsjahr}}
\printdeclarationlist%
\index{Titel!Felder}\index{Autorenangaben|?}%
\index{Datum!Geburtsdatum|?}%
%
Mit dem Befehl \Macro{author} wird der Autor angegeben. Innerhalb des 
Argumentes können auch mehrere Autoren aufgeführt werden, wobei diese in diesem 
Fall jeweils mit \Macro{and} zu trennen sind. Zu erwähnen ist, dass alle 
weiteren hier vorgestellten Befehle selbst im Argument von \Macro{author} 
stehen können. Damit wird es möglich, jedem Autor unterschiedliche Angaben 
mitzugeben.

Mit \Macro{authormore} wird unter dem Autor eine Zeile ausgegeben, welche 
durch den Anwender frei belegt werden kann. Sollte das Paket \Package{isodate} 
geladen sein, so wird die damit eingestellte Formatierung des Datums durch 
\Macro{dateofbirth}~-- wie übrigens bei jedem anderem Datumsfeld der 
\TUDScript-Klassen auch~-- verwendet. Dafür der Befehl \Macro{printdate} aus 
diesem Paket verwendet. Die weiteren Befehle als zusätzliche Angabe erklären 
sich von selbst.
\end{Declaration}
\end{Declaration}
\end{Declaration}
\end{Declaration}
\end{Declaration}
\end{Declaration}

\begin{Declaration}{\Macro{and}}
\printdeclarationlist%
\index{Kollaboratives Schreiben|?}\index{Titel!Kollaboratives Schreiben}%
%
Dieser Befehl wird sowohl bei den \hologo{LaTeX}"=Standardklassen als auch bei 
den \KOMAScript"=Klassen lediglich auf der Titelseite dazu verwendet, mehrere 
Autoren im Argument von \Macro{author} voneinander zu trennen.

Bei den \TUDScript-Klassen hingegen ist dieser Befehl derart in seiner Funktion 
erweitert worden, dass damit die Angabe einer kollaborativen Autorenschaft für 
Abschlussarbeiten innerhalb des Befehls \Macro{author} möglich ist. Außerdem 
kann er noch im Argument von \Macro{supervisor}, \Macro{referee} sowie 
\Macro{advisor} verwendet werden, um mehrere Betreuer beziehungsweise Gutachter 
und Fachreferenten anzugeben. Er ist dabei nicht auf die Verwendung für den 
Titel allein beschränkt. Auch bei den Umgebungen \Environment{task}, 
\Environment{evaluation} und \Environment{notice} kann er eingesetzt werden.
\end{Declaration}
%
\begin{Example}
Angenommen, es soll eine Abschussarbeit von zwei unterschiedlichen Autoren in 
kollaborativer Gemeinschaft erstellt werden, so könnte man die Autorenangaben 
folgendermaßen gestalten:
\begin{Code}
\author{%
  Mickey Mouse
  \matriculationnumber{12345678}
  \dateofbirth{2.1.1990}
  \placeofbirth{Dresden}
\and%
  Donald Duck
  \matriculationnumber{87654321}
  \dateofbirth{1.2.1990}
  \placeofbirth{Berlin}
}
\matriculationyear{2010}
\end{Code}
Alle zusätzlichen Angaben außerhalb des Argumentes von \Macro{author} werden 
für beide Autoren gleichermaßen übernommen. Angaben innerhalb des Argumentes 
von \Macro{author} werden den jeweiligen, mit \Macro{and} getrennten Autoren 
zugeordnet. Mehr dazu ist im Minimalbeispiel in \autoref{sec:exmpl:thesis}.
\end{Example}

\begin{Declaration}{\Macro{thesis}\Parameter{Typisierung}}
\begin{Declaration}{\Macro{subject}\Parameter{Typisierung}}
\printdeclarationlist%
\index{Titel!Felder}%
\index{Abschlussarbeit|!}\index{Typisierung}%
%
Mit diesen beiden Befehlen kann der Typ der Dokumentes beziehungsweise der 
Abschlussarbeit angegeben werden. Während der Befehl \Macro{thesis} den Inhalt 
des Feldes unter dem Titel vertikal zentriert und in \DIN auf der Titelseite 
ausgibt, erscheint der Inhalt des Befehls \Macro{subject} in \Univers oberhalb 
des Titels. Es können auch beide Befehle parallel mit unterschiedlichen 
Inhalten verwendet werden. Der Befehl \Macro{thesis} dient den 
\TUDScript"=Dokumentklassen außerdem zur Erkennung von Abschlussarbeiten 
gedacht, da für diese spezielle Felder bereitgehalten werden und auch die 
Titelseite leicht geändert gesetzt wird.

Des Weiteren ist es bei beiden Befehlen möglich, spezielle Werte als Argument 
zur Typisierung des Dokumentes zu verwenden. Diese werden entsprechend der 
gewählten Dokumentensprache~-- entweder Deutsch oder Englisch~-- entschlüsselt 
und gesetzt. Die möglichen Werte sind \autoref{tab:thesis} zu entnehmen. Dabei 
ist zu beachten, dass das Setzen eines speziellen Wertes für \emph{entweder} 
\Macro{thesis} \emph{oder} \Macro{subject} möglich ist. Die Verwendung eines 
der genannten Werte führt immer dazu, dass das Dokument als Abschlussarbeiten 
erkannt und die erweiterte Titelseite aktiviert wird. Gleichzeitig wird damit 
die Option \Option{subjectthesis} beeinflusst. Sollte vom Anwender kein 
explizites Verhalten für \Option{subjectthesis} definiert sein, so führt die 
Verwendung von \Macro{thesis}\Parameter{Wert} zu \Option{subjectthesis}[false] 
und \Macro{subject}\Parameter{Wert} zu \Option{subjectthesis}[true].
%
\begin{table}
\index{Bezeichner}\index{Bezeichner!Typisierung}%\\
\index{Abschlussarbeit!Typisierung}%
\caption{%
  Spezielle Werte zur Typisierung des Dokumentes für
  \Macro{thesis} und \Macro{subject}%
}
\label{tab:thesis}%
\centering%
\makeatletter%
\def\@tempa#1{%
  \Term{#1} & \@nameuse{#1} & \selectlanguage{english}\@nameuse{#1}%
  \tabularnewline%
}%
\begin{tabular}{llll}
  \toprule
  \textbf{Wert} & \textbf{Bezeichner}
    & \textbf{Deutsch} & \textbf{Englisch} \tabularnewline
  \midrule
  diss & \@tempa{dissertationname}
  doctoral & \@tempa{dissertationname}
  phd & \@tempa{dissertationname}
  diploma & \@tempa{diplomathesisname}
  master & \@tempa{masterthesisname}
  bachelor & \@tempa{bachelorthesisname}
  student & \@tempa{studentresearchname}
  project & \@tempa{projectpapername}
  seminar & \@tempa{seminarpapername}
  research & \@tempa{researchname}
  log & \@tempa{logname}
  report & \@tempa{reportname}
  internship & \@tempa{internshipname}
  \bottomrule
\end{tabular}
\makeatother%
\end{table}
\end{Declaration}
\end{Declaration}

\begin{Declaration}{\Option{subjectthesis}[\PBoolean]}%
  [false][\Macro{subject}\Parameter{\autoref{tab:thesis}}:true]
\printdeclarationlist%
%
Der Befehl \Macro{thesis} dient den \TUDScript"=Hauptklassen zur Unterscheidung 
zweier unterschiedlichen Ausprägungen der Titelseite und ist im speziellen für 
Abschlussarbeiten gedacht. Außerdem kann bei der Verwendung spezieller Werte 
aus \autoref{tab:thesis} innerhalb des Argumentes von \Macro{subject} ebenfalls 
das Verhalten für Abschlussarbeiten aktiviert werden, wobei hierdurch die 
Einstellung \Option{subjectthesis}[true] automatisch vorgenommen wird.

Für den Standardfall~-- bekanntlich \Option{subjectthesis}[false]~-- wird der 
durch \Macro{thesis} gegebene Typ der Abschlussarbeit sowie der gegebenenfalls 
durch \Macro{graduation} gesetzte angestrebte Abschluss in großen Lettern und 
sehr zentral auf der Titelseite gesetzt. Die Verwendung von \Macro{subject} ist 
hierbei weiterhin möglich.
%
Wird die Option mit \Option{subjectthesis}[true] aktiviert, so wird die mit 
\Macro{thesis} gesetzte Bezeichnung nicht unterhalb sondern oberhalb des Titels 
an der Stelle von \Macro{subject} ausgegeben. Der mit \Macro{graduation} 
angegebene Abschluss wird weiterhin unter dem Titel, allerdings in schlankerer 
Schrift gesetzt. Eine etwaige Verwendung des Befehls \Macro{subject} wird in 
diesem Fall ignoriert.
%
\begin{values}
\itemfalse
  Die Ausgabe des Typs der Abschlussarbeit (\Macro{thesis}) selbst sowie des 
  angestrebten Abschlusses (\Macro{graduation}) erfolgt in großen Lettern in 
  \DIN zentral auf der Titelseite.
\itemtrue*
  Der Typ der Abschlussarbeit (\Macro{thesis}) wird oberhalb des Titels in der 
  Betreffzeile gesetzt. Der angestrebte Abschluss (\Macro{graduation}) wird 
  zentral in der schlankeren \Univers ausgegeben.
\end{values}
\end{Declaration}

\Rename[macros]{v2.02}{\Macro{degree}}{\Macro{graduation}}
\begin{Declaration}{\Macro{graduation}\OParameter{Kurzform}\Parameter{Grad}}
\printdeclarationlist%
\index{Titel!Felder}%
%
Mit diesem Befehl wird der angestrebte akademische Grad auf der Titelseite 
ausgegeben. Da dies nur mit einer Abschlussarbeit erreicht werden kann erfolgt 
die Ausgabe nur, wenn entweder \Macro{thesis} oder \Macro{subject} verwendet 
wurde, wobei bei letzterem Befehl im Argument zwingend ein Wert aus 
\autoref{tab:thesis} verwendet werden muss.

Die Option \Option{subjectthesis} hat Einfluss auf die Ausgabe auf der 
Titelseite. Für die Einstellung \Option{subjectthesis}[false] wird der 
Abschuss~-- ähnlich wie 
der Typ der Abschlussarbeit~-- zentral und in relativ großen Lettern gesetzt. 
Für \Option{subjectthesis}[true] erfolgt die Ausgabe kleiner und in weniger 
starken Buchstaben.
\end{Declaration}

\begin{Declaration}{\Macro{supervisor}\Parameter{Name(n)}}
\begin{Declaration}{\Macro{referee}\Parameter{Name(n)}}
\begin{Declaration}{\Macro{advisor}\Parameter{Name(n)}}
\begin{Declaration}{\Macro{professor}\Parameter{Name}}
\printdeclarationlist%
\index{Titel!Felder}%
\index{Betreuer|?}\index{Gutachter|?}\index{Referent|?}%
%
Mit \Macro{supervisor}, \Macro{referee} und \Macro{advisor} werden die Betreuer 
einer Abschlussarbeit beziehungsweise die Gutachter und Fachreferenten einer 
Dissertation angegeben. Zusätzlich kann mit \Macro{professor} der betreuende 
Hochschullehrer beziehungsweise die betreuenden Professoren für studentische 
Arbeiten angegeben werden. Die Angabe mehrerer Person erfolgt wie beim Befehl 
\Macro{author} durch die Trennung mittels \Macro{and}.
\end{Declaration}
\end{Declaration}
\end{Declaration}
\end{Declaration}

\begin{Declaration}{\Macro{date}\OParameter{Ergänzung}\Parameter{Datum}}
\begin{Declaration}{\Macro{defensedate}\Parameter{Verteidigungsdatum}}
\printdeclarationlist%
\index{Titel!Felder}
\index{Datum|?}\index{Datum!Verteidigungsdatum|?}%
%
Mit \Macro{date} kann das Datum angegeben werden. Das optionale Argument 
erlaubt eine zusätzliche Anmerkung, welche nach dem Datum ausgegeben wird. Das 
Datum wird bei normalen Dokumenten direkt nach dem Autor beziehungsweise den 
Autoren ausgegeben. Bei Abschlussarbeiten~-- aktiviert durch die Verwendung von 
\Macro{thesis} uder \Option{subjectthesis}~-- erscheint dieses am Ende der 
Titelseite als Abgabedatum. Außerdem kann in diesem Fall mit  dem Befehl
\Macro{defensedate} das Datum der Verteidigung angegeben werden, wie es 
beispielsweise bei dem Druck von Dissertationen üblich ist.

Sollte das Paket \Package{isodate} geladen sein, so wird die damit eingestellte 
Formatierung des Datums durch den Befehl \Macro{printdate} aus diesem Paket für 
alle Datumsfelder des Dokumentes und folglich auch für die beiden Felder 
\Macro{date} und \Macro{defensedate} verwendet.
\end{Declaration}
\end{Declaration}
\index{Titel|!)}


\subsection{Die Teileseite}
\label{sec:part}
%
\ChangedAt{%
  v2.02!\Macro{partpagestyle}: \PageStyle{plain.tudheadings} wird genutzt%
}
Wird für die Teileseiten das Layout des \CDs verwendet, so wird der Seitenstil 
dieser (\Macro{partpagestyle}) auf \PageStyle{plain.tudheadings} gesetzt.

\begin{Declaration}{\Option{parttitle}[\PBoolean]}[false]%
\printdeclarationlist%
\index{Teileseiten|?}\index{Layout!Teileseiten}%
%
Diese Option ermöglicht es, den mit \Macro{title} gegebenen Titel des 
Dokumentes selbst in großer Schrift auf einer Teileseite auszugeben, die 
Bezeichnung des mit \Macro{part}\Parameter{Bezeichnung} erzeugten Teils wird 
in diesem Fall in kleiner Schrift direkt darunter gesetzt. Diese 
Layout"=Variante findet sich im Handbuch für das \CD der \TnUD. \notudscrartcl
%
\begin{values}
\itemfalse
  Die Bezeichnung des Teils erscheint in großer Schrift auf der Seite, der 
  Titel des Dokumentes gar nicht.
\itemtrue*
  Der Titel wird in großer Auszeichnung auf der Teileseite gesetzt, die 
  Bezeichnung des Teils selber in kleinerer.
\end{values}
\end{Declaration}


\subsection{Die Kapitelseite}
\begin{Declaration}{\Option{chapterpage}[\PBoolean]}%
  [false][\Option{cd}[color]:true]%
\printdeclarationlist%
\label{sec:chapter}%
\index{Kapitelseiten|?}\index{Layout!Kapitelseiten|?}%
\index{Satzspiegel!doppelseitig}\index{Leerseiten}%
%
Mit dieser Einstellung kann die Überschrift eines Kapitels separat auf einer 
Seite ausgegeben werden. Der nachfolgende Text wird auf der nächsten 
beziehungsweise bei doppelseitigem Satz und rechts öffnenden Kapiteln%
\footnote{%
  \Option{twoside} und \Option{open}[right], Standard für \Class{tudscrbook}
}
auf der übernächsten Seite ausgegeben. Die in diesem Fall erzeugte Rückseite 
wird in ihrer Ausprägung~-- wie auch Teileseiten~-- durch die Einstellung von 
\Option{cleardoublespecialpage} bestimmt. Beim farbigen Layout ist diese Option 
standardmäßig aktiviert. \notudscrartcl
%
\begin{values}
\itemfalse
  Es gibt keine Sonderstellung von Kapiteln, der nachfolgende Text wird direkt 
  unter der Überschrift auf der gleichen Seite ausgegeben.
\itemtrue*
  Die Kapitelüberschrift wird auf einer separaten Seite gesetzt, der folgende
  Text wird erst auf der nächsten beziehungsweise übernächsten Seite 
  ausgegeben. Siehe dazu auch die Option \Option{cleardoublespecialpage}.
\end{values}
%
\ChangedAt{v2.02}
Der Seitenstil von Kapiteln lässt sich übrigens~-- unabhängig von der Option 
\Option{chapterpage}~-- ändern, indem der Befehl \Macro{chapterpagestyle} 
umdefiniert wird. Außerdem können mit der Option \Option{chapterprefix} 
Kapitelüberschriften mit einer Vorsatzzeile aktiviert werden. Dabei wird 
zunächst in einer Zeile \enquote{Kapitel} gefolgt von der Kapitelnummer 
ausgegeben, in der nächsten Zeile wird anschließend die Überschrift in 
linksbündigem Flattersatz ausgegeben. Insbesondere bei der Verwendung von 
separaten Kapitelseiten ist die Nutzung dieser Option empfehlenswert. Genaueres 
hierzu ist in der \KOMAScript"=Dokumentation nachzulesen.
\end{Declaration}


\subsection{Satzspiegel und Kolumnentitel}
\begin{Declaration}{\Option{geometry}[\PSet]}[true]%
\printdeclarationlist%
\index{Seitenstil}\index{Layout!Seitenstil}%
\index{Satzspiegel}\index{Satzspiegel!doppelseitig}\index{Layout!Satzspiegel}%
\index{Layout!Seitenränder}%
%
Diese Option ist für die Aufteilung beziehungsweise die Berechnung des 
Satzspiegels verantwortlich. Das Maß der Seitenränder ist im \CD fest 
vorgegeben und wird standardmäßig von den \TUDScript-Klassen eingehalten. 
Allerdings lassen sich die Seitenränder anpassen, um beispielsweise einen 
vernünftigen doppelseitigen Satz zu ermöglichen.%
\footnote{Hierbei sollte der innere Rand schmaler als der äußere sein}
Des Weiteren besteht die Möglichkeit, auf das Standardverhalten von 
\KOMAScript{} zurückzufallen und die Satzspiegelberechnung durch das Paket
\Package{typearea} vornehmen zu lassen. Hier hat insbesondere die Klassenoption 
\Option{DIV}[\PSet] maßgeblichen Einfluss auf den Satzspiegel. Siehe dazu die 
Dokumentation von \KOMAScript{}.
%
\begin{values}
\itemfalse
  Die Satzspiegelberechnung erfolgt via \Package{typearea}, die Vorgaben des 
  \CDs bezüglich der Seitenränder werden ignoriert.
\itemtrue*[tud/cd/asymmetric]
  Die Seitenränder werden im asymmetrischen Stil des \CDs fest definiert und 
  auch für den doppelseitigen Satz (\Option{twoside}[true]) genutzt.%
  \footnote{links: 30\,mm, rechts: 20\,mm, oben: 25\,mm, unten: 30\,mm}
\item[symmetric/centred/centered]
  Der Satzspiegel wird im einseitigen sowie doppelseitigen Satz auf der Seite 
  zentriert.%
  \footnote{links: 25\,mm, rechts: 25\,mm, oben: 25\,mm, unten: 30\,mm}
\item[balanced/twoside]
  Im einseitigen Layout ist das Verhalten der Einstellung identisch zu
  \Option{geometry}[symmetric]. Beim doppelseitigen Satz wird der Satzspiegel 
  derart verändert, dass die Ränder der inneren Seiten schmaler sind als die 
  der äußeren.%
  \footnote{innen: 20\,mm, außen: 30\,mm, oben: 25\,mm, unten: 30\,mm}
  \Attention{%
    Der so erzeugte Satzspiegel ist allerdings nicht sehr vorteilhaft. Es ist 
    zu beachten, dass dabei das Logo der \TnUD sehr nah am inneren Seitenrand 
    des Dokumentes gesetzt wird, folglich insbesondere auf rechten respektive 
    ungeraden Seiten sehr weit an den Blattrand rückt.
  }
\end{values}
%
Für die Festlegung der Seitenränder wird das Paket \Package{geometry} 
verwendet. Ist \Option{geometry}[false] gewählt, erfolgt die Berechnung des 
Satzspiegels durch \Package{typearea}. Die damit berechneten Werte werden 
anschließend an \Package{geometry} weitergereicht und durch dieses umgesetzt.
\end{Declaration}

\subsubsection{Bindekorrektur}
\index{Bindekorrektur|!}\index{Layout!Bindekorrektur}%
%
Zu erwähnen im Zusammenhang mit Seitenrändern und Satzspiegel ist die durch 
\Package{typearea} angebotene Option \Option{BCOR}[\PName{Länge}], mit der bei 
der Satzspiegelberechnung ein Heftrand beziehungsweise eine Bindekorrektur 
berücksichtigt wird. Die \TUDScript-Klassen reichen diesen Wert auch an 
\Package{geometry} weiter, so dass der Benutzer unabhängig von der Auswahl zur 
Satzspiegelgestaltung diese Option nutzen kann. So kann beispielsweise eine 
Bindekorrektur von \unit[5]{mm} mit der Klassenoption \Option{BCOR}[5mm] 
gesetzt werden.

Eine Anpassung der Bindekorrektur hat natürlich \emph{immer} eine Änderung der 
verfügbaren Breite des Textbereichs zur Folge hat und führt somit zwingend zu 
einer Anpassung des Satzspiegels. Da die Bindekorrektur jedoch abhängig von der 
Höhe des Buchblocks gewählt werden sollte, welche letztendlich erst mit dem 
Druck des fertiggestellten Dokumentes bestimmt werden kann, muss diese zu 
Beginn abgeschätzt werden.
%
\begin{Example}
Als Faustregel gilt, dass die erforderliche Bindekorrektur in etwa der halben 
Höhe des Buchblocks entsprechen sollte. Dessen Höhe wiederum ist abhängig von 
der Anzahl der Seiten sowie der Dichte des verwendeten Papiers. Wird normales 
Papier mit einer Dichte von \unit[80]{g/m²} verwendet, so entsprechen 100~Blatt 
in etwa einer Höhe von \unit[10]{mm}, bei \unit[100]{g/m²} ca. \unit[12]{mm}. 
Dementsprechend wäre eine Bindekorrektur von \Option{BCOR}[5mm] beziehungsweise 
\Option{BCOR}[6mm] bei diesem Beispiel zu wählen.
\end{Example}

\subsubsection{Kopf"~ und Fußzeile im Zusammenspiel mit dem Satzspiegel}
\index{Kopfzeile|!}\index{Layout!Kopfzeile}%
\index{Fußzeile|!}\index{Layout!Fußzeile}%
Da im \CD nicht festgelegt ist, wie die Gestaltung der Kopf"~ und Fußzeilen in 
einer wissenschaftlichen Arbeit auszuführen ist, bleibt dem Nutzer dafür eine 
gewisse Freiheit. Dafür sollte idealerweise das zu \KOMAScript{} gehörige Paket 
\Package{scrlayer-scrpage} genutzt werden. 

In der Dokumentation zu \Package{typearea} wird auch darauf eingegangen, wann 
Kopf"~ und Fußzeile bei der Satzspiegelkonstruktion entweder dem Rand oder dem 
Textkörper zugeschlagen werden sollten. Dies sollte bei der Erstellung eigener 
Kopf"~ und Fußzeilen beachtet werden. Die Einstellung dafür erfolgt mit den 
beiden \KOMAScript"=Optionen \Option{headinclude}[\PBoolean] sowie 
\Option{footinclude}[\PBoolean]. Diese können~-- unabhängig von der gewählten 
Einstellung zur Satzspiegelgestaltung über \Option{geometry}~-- verwendet 
werden.

\begin{Declaration}{\Option{cdfoot}[\PBoolean]}[false]%
\printdeclarationlist%
\index{Kolumnentitel}\index{Layout!Kolumnentitel}
\index{Satzspiegel!doppelseitig}%

Eine Möglichkeit zur Gestaltung der Kolumnentitel zeigt das Handbuch für das 
\CD der \TnUD. Dieses wird ohne Kopf"~ und mit einer einfachen Fußzeile 
gesetzt. Diese enthält dabei den aktuellen Kolumnentitel sowie die Paginierung. 
Eine derartige Ausprägung ist nicht explizit durch das \CD vorgegeben, wurde 
jedoch innerhalb der alten \Class{tudbook}"=Klasse exakt so umgesetzt.

Die neuen \TUDScript-Klassen sind~-- insbesondere aufgrund der Möglichkeit zur 
Verwendung des Paketes \Package{scrlayer-scrpage}~-- bei der Gestaltung der 
Kopf"~ und Fußzeilen wesentlich flexibler. Dennoch kann mit dieser Option das 
beschriebene Verhalten aktiviert werden. Hierbei wird beim doppelseitigen Satz 
(\Option{twoside}[true]) die Seitenzahl außen gesetzt.
%
\begin{values}
\itemfalse
  Die Kopf"~ und Fußzeilen zeigen Standardverhalten, zur manuellen Änderung 
  dieser sollte unbedingt das \KOMAScript"=Paket \Package{scrlayer-scrpage} 
  verwendet werden.
\itemtrue*
  Die Kopf"~ und Fußzeilen des Dokumentes werden wie im Handbuch des \CDs der 
  \TnUD beziehungsweise der \Class{tudbook}"=Klasse gesetzt.
\end{values}
%
Der Inhalt der Kolumnentitel kann durch den Anwender frei gewählt werden. Wird 
die Klassenoption \Option{automark} angegeben, werden für das automatische 
Setzen der Marken die Titel der Gliederungsebenen verwendet. Genaueres hierzu 
sowie der Möglichkeit, die Kolumnentitel manuell festzulegen, ist dem Handbuch 
von \KOMAScript{} zu entnehmen.
\end{Declaration}


\subsection{Die Farben des \CDs}
\index{Farben}%
% 
Zur Verwendung der Farben des \CDs wird das Paket \Package{tudscrcolor} 
genutzt. Falls dieses nicht in der Präambel geladen wird~-- um beispielsweise 
zusätzliche Optionen aufzurufen~-- binden die \TUDScript"=Klassen dieses 
automatisch ein. Detaillierte Informationen sind in der Dokumentation von 
\Package{tudscrcolor}'full' zu finden.



\section{Zusätzliche Optionen und Erweiterungen}
Neben den Befehlen für die Anpassung des Layouts an das \CD der \TnUD stellen 
die \TUDScript-Klassen weitere Befehle und Umgebungen zur Verfügung, um die 
Anwendung insbesondere für wissenschaftliche Arbeiten zu erleichtern.


\subsection{Zusammenfassung}
\begin{Declaration}[%
  v2.02!Wert \PValue{double} mit \PValue{multi} ersetzt%
]{\Option{abstract}[\PSet]}%
\printdeclarationlist%
\index{Zusammenfassung|!(}%
\index{Zweispaltensatz}%
%
Diese Option wird bereits durch \KOMAScript{} für die Klassen \Class{scrartcl} 
und \Class{scrreprt} standardmäßig bereitgestellt. Für die Klasse 
\Class{scrbook} geschieht dies nicht. Dazu heißt es im Handbuch:
%
\begin{quoting}
Bei Büchern wird in der Regel eine andere Art der Zusammenfassung verwendet. 
Dort setzt man ein entsprechendes Kapitel an den Anfang oder Ende des Werks. 
Oft wird diese Zusammenfassung entweder mit der Einleitung oder einem weiteren 
Ausblick verknüpft. Daher gibt es bei \Class{scrbook} generell keine 
\Environment{abstract}"=Umgebung. Bei Berichten im weiteren Sinne, etwa einer 
Studien- oder Diplomarbeit, ist ebenfalls eine Zusammenfassung in dieser Form 
zu empfehlen.
\end{quoting}
%
Durch die \TUDScript-Klassen wird die \Option{abstract}"=Option erweitert. 
Neben der standardmäßigen Auswahlmöglichkeit innerhalb der Klassen 
\Class{tudscrartcl} und \Class{tudscrreprt}, ob keine oder eine kleine und 
zentrierte Überschrift innerhalb der \Environment{abstract}"=Umgebung gesetzt 
werden soll, kann die Überschrift für die Zusammenfassung außerdem in Gestalt 
eines Unterkapitels oder für die Klassen \Class{tudscrreprt} und 
\Class{tudscrbook} in der Form eines Kapitels ausgegeben werden.

Abhängig von der gewählten Gliederungsebene der Überschrift wird das 
Standardverhalten für das Setzen eines Eintrages ins Inhaltsverzeichnis 
festgelegt. Ohne oder mit zentrierter Überschrift (\Option{abstract}[true]) 
wird per Voreinstellung kein Eintrag im Inhaltsverzeichnis erzeugt. Wird die 
Überschrift jedoch in Form einer Gliederungsebene gewählt, so erscheint die 
Zusammenfassung für gewöhnlich im Inhaltsverzeichnis auf der obersten Ebene. 
Dieses Verhalten kann jederzeit mit der Option \Option{abstract}[toc/notoc] 
durch den Anwender überschrieben werden.
%
\begin{values}
\itemfalse[][nur für \Class{tudscrartcl} und \Class{tudscrreprt} verfügbar]
  Es wird keine Überschrift für die \Environment{abstract}"=Umgebung ausgegeben.
\itemtrue*[][nur für \Class{tudscrartcl} und \Class{tudscrreprt} verfügbar]
  Wie bei den \KOMAScript"=Klassen wird eine zentrierte Überschrift mit dem 
  Bezeichner \Term{abstractname} vor der eigentlichen Zusammenfassung gesetzt.
\item[section]
  Die Überschrift verwendet den Gliederungsbefehl \Macro{addsec}.
\item[chapter][%
    (Säumniswert für \Class{tudscrbook})
    nur für \Class{tudscrreprt} und \Class{tudscrbook} verfügbar%
  ]
  Es wird der Befehl \Macro{addchap} für das Setzen der Überschrift genutzt. 
\item[heading]
  Es wird die höchstmögliche Gliederungsebene verwendet. Für 
  \Class{tudscrartcl} entspricht dies \Option{abstract}[section], bei 
  \Class{tudscrreprt} und \Class{tudscrbook} \Option{abstract}[chapter].
\item[toc/totoc]
  Unabhängig von der Wahl der Überschrift erhält die Zusammenfassung einen nicht
  nummerierten Eintrag im Inhaltsverzeichnis auf der obersten Gliederungsebene. 
\item[notoc/nottotoc]
  Die Zusammenfassung wird definitiv nicht ins Inhaltsverzeichnis eingetragen.
\end{values}
%
Häufig wird für Abschlussarbeiten verlangt, neben der deutschsprachigen 
auch noch eine englischsprachige Zusammenfassung zu verfassen. Es kann 
eingestellt werden, beide zusammen auf einer Seite auszugeben~-- sofern 
genügend Platz vorhanden ist. Außerdem kann die standardmäßige vertikale 
Zentrierung der \Environment{abstract}"=Umgebung auf einer Seite unterdrückt 
werden. Damit kann der Anwender gegebenenfalls die Positionierung selbstständig 
vornehmen. 
%
\begin{values}
\item[one/simple/single]Jede Zusammenfassung wird auf einer eigenen Seite
  beziehungsweise im zweispaltigen Satz in einer neuen Spalte ausgegeben.
\item[multi/multiple/all]
  \ChangedAt{v2.02}
  Zusammenfassungen, welche mit \Macro{nextabstract} getrennt wurden, werden 
  direkt nacheinander auf der gleichen Seite ausgegeben, wenn ausreichend Platz 
  auf dieser vorhanden sein sollte. Ist die Option \Option{twocolumn} aktiviert,
  erfolgt die Ausgabe der aller Zusammenfassungen ohne Spaltenumbruch.
\item[nofil/nofill/novfil/novfill]
  Die Ausgabe erfolgt wie im normalen Textsatz auch.
\item[fil/fill/vfil/vfill]
  Alle Zusammenfassungen auf einer Ausgabeseite werden vertikal zentriert. Für 
  den zweispaltigen Satz steht diese Option nicht zur Verfügung.
\end{values}
Es ist zu beachten, dass die zuvor genannten Einstellungen zur Positionierung 
der Zusammenfassungen mit \Option{abstract}[simple/multiple/fill/nofill] 
innerhalb der \Environment{abstract}"=Umgebung nur wirksam sind, wenn eine 
Titelseite (\Option{titlepage}[true]) und \emph{keine} Überschriften in Form 
von Kapiteln (\Option{abstract}[chapter]) verwendet werden.
\end{Declaration}

\begin{Declaration}[%
  v2.02!\Macro{nextabstract} zur Trennung der einzelnen Teile%
]{\Environment{abstract}[\OLParameter{Sprache}]}
\begin{Declaration}[v2.02]{\Macro{nextabstract}\OLParameter{Sprache}}
\begin{Declaration}{\Key{\Environment{abstract}}{language}[\PName{Sprache}]}
\begin{Declaration}[v2.02]{%
  \Key{\Environment{abstract}}{pagestyle}[\PName{Seitenstil}]%
}
\begin{Declaration}{\Key{\Environment{abstract}}{columns}[\PName{Anzahl}]}
\begin{Declaration}{\Key{\Environment{abstract}}{option}[\PSet]}
\printdeclarationlist%
\index{Zweispaltensatz}%
%
Diese Umgebung dient speziell für die Ausgabe einer Zusammenfassung. Mit der 
Option \Option{abstract} kann eingestellt werden, in welcher Gestalt diese 
ausgegeben werden soll. Wird keine Titelseite sondern ein Titelkopf verwendet 
(\Option{titlepage}[false]), so wird für den Fall, dass die Zusammenfassung 
\emph{nicht} mit einer Überschrift einer Gliederungsebene gesetzt wird, diese 
wie bei den \KOMAScript"=Klassen in einer \Environment{quotation}"=Umgebung 
gesetzt, um die Zusammenfassung abzuheben. Diese hat jedoch den Nachteil, dass 
in dieser die Option \Option{parskip} nicht beachtet wird. Um dieses Problem zu 
beheben, kann das Paket \Package{quoting} geladen werden, wodurch stattdessen 
die \Environment{quoting}"=Umgebung verwendet wird.

Zusätzlich können weitere Parameter als optionales Argument angegeben werden. 
Wird das Paket \Package{babel} verwendet, kann mit dem Parameter 
\Key{\Environment{abstract}}{language}[\PName{Sprache}] die Sprache innerhalb 
der \Environment{abstract}"=Umgebung geändert werden. Dafür muss die gewünschte 
Sprache bereits mit dem Laden von \Package{babel} entweder als Paketoption oder 
besser noch als Klassenoption angegeben worden sein. Dadurch werden lokal 
innerhalb der Umgebung die Bezeichnung \Term{abstractname} und die 
Trennungsmuster sprachspezifisch angepasst. Die gewünschte Sprache kann auch 
direkt und ohne den Parameter \Key{\Environment{abstract}}{language} als 
optionales Argument übergeben werden.

Wurde das Paket \Package{multicol} geladen, kann mit dem Parameter 
\Key{\Environment{abstract}}{columns}[\PName{Anzahl}] die Zusammenfassung 
mehrspaltig gesetzt werden.
\ChangedAt{v2.02}
Der für die Umgebung zu verwendende Seitenstil kann mit dem Parameter 
\Key{\Environment{abstract}}{pagestyle} angegeben werden. Dieser unterstützt 
auch die \PageStyle{tudheadings}"=Seitenstile.

Dem Parameter \Key{\Environment{abstract}}{option} können alle gültigen, 
bereits erläuterten Werte der Option \Option{abstract} übergeben werden. 
Die damit gemachten Einstellungen wirken sich~-- im Gegensatz zur Angabe als 
Klassenoption oder über die Variante der späten Optionenwahl%
\footnote{%
  \Macro{TUDoption}\PParameter{abstract}\Parameter{Einstellung} oder
  \Macro{TUDoptions}\PParameter{abstract=\PName{Einstellung}}
}~-- lediglich lokal auf die verwendete \Environment{abstract}"=Umgebung aus.

\ChangedAt{v2.02}
Sollen mehrere Zusammenfassungen im gleichen Stil erzeugt und die Einstellungen 
der Option \Option{abstract}[simple/multiple/fill/nofill] beachtet werden, so 
ist die \Environment{abstract}"=Umgebung nur einmal zu verwenden. Innerhalb 
dieser müssen die einzelnen Zusammenfassungen mit \Macro{nextabstract} 
voneinander getrennt werden. Der Befehl akzeptiert dabei im optionalen Argument 
alle Parameter, die auch von der \Environment{abstract}"=Umgebung selbst 
unterstützt werden. Das Minimalbeispiel in \autoref{sec:exmpl:dissertation} 
zeigt hierfür das notwendige Vorgehen.
%
%Sollte die \Environment{abstract}"=Umgebung innerhalb des Argumentes der 
%Befehle \Macro{partpreamble} beziehungsweise \Macro{chapterpreamble} verwendet 
%werden, wird die Überschrift~-- im Fall, dass nicht \Option{abstract}[false] 
%gewählt ist~-- \emph{immer} in Textgröße und zentriert gesetzt.
\end{Declaration}
\end{Declaration}
\end{Declaration}
\end{Declaration}
\end{Declaration}
\end{Declaration}
\index{Zusammenfassung|!)}%


\subsection{Selbstständigkeitserklärung und Sperrvermerk}
\begin{Declaration}[%
  v2.02!Wert \PValue{double} mit \PValue{multi} ersetzt%
]{\Option{declaration}[\PSet]}[true]%
\printdeclarationlist%
\index{Selbstständigkeitserklärung|!}\index{Sperrvermerk|!}%
%
Mit dieser Einstellung kann äquivalent zur Option \Option{abstract} die 
Gestaltung von Selbstständigkeitserklärung und Sperrvermerk angepasst werden.
Zur Ausgabe der Erklärungen werden die Umgebung \Environment{declarations} 
sowie die Befehle \Macro{declaration} beziehungsweise \Macro{confirmation} und 
\Macro{blocking} bereitgestellt. 

Abhängig von der gewählten Gliederungsebene der Überschrift wird das 
Standardverhalten für das Setzen eines Eintrages ins Inhaltsverzeichnis 
festgelegt. Ohne oder mit zentrierter Überschrift (\Option{declaration}[true]) 
wird per Voreinstellung kein Eintrag im Inhaltsverzeichnis erzeugt. Wird die 
Überschrift jedoch in Form einer Gliederungsebene gewählt, so erscheint die 
Erklärung für gewöhnlich im Inhaltsverzeichnis auf der obersten Ebene. Dieses 
Verhalten kann jederzeit mit der Option \Option{declaration}[toc/notoc] durch 
den Anwender überschrieben werden.
%
\begin{values}
\itemfalse
  Es wird keine Überschrift über den Erklärungen selbst ausgegeben.
\itemtrue*
  Eine zentrierte Überschrift mit dem Bezeichner \Term{confirmationname} vor 
  der Selbstständigkeitserklärung beziehungsweise \Term{blockingname} vor dem 
  Sperrvermerk wird gesetzt. 
\item[section]
  Die Überschrift verwendet den Gliederungsbefehl \Macro{addsec}.
\item[chapter]
  Es wird der Befehl \Macro{addchap} für das Setzen der Überschrift genutzt. 
\item[heading]
  Es wird die höchstmögliche Gliederungsebene verwendet. Für 
  \Class{tudscrartcl} entspricht dies \Option{declaration}[section], bei 
  \Class{tudscrreprt} und \Class{tudscrbook} \Option{declaration}[chapter].
\item[toc/totoc]
  Unabhängig von der Wahl der Überschrift erhält die Erklärung einen nicht
  nummerierten Eintrag im Inhaltsverzeichnis auf der obersten Gliederungsebene. 
\item[notoc/nottotoc]
  Die Erklärung wird definitiv nicht ins Inhaltsverzeichnis eingetragen.
\end{values}
%
Die folgenden Einstellungen haben lediglich Auswirkungen, wenn die Überschrift 
der Erklärung \emph{nicht} im Form eines Kapitels ausgegeben und eine 
Titelseite (\Option{titlepage}[true]) verwendet wird.
%
\begin{values}
\item[one/simple/single]Jede Erklärung wird auf einer separaten Seite
  beziehungsweise im zweispaltigen Satz in einer neuen Spalte ausgegeben.
\item[multi/multiple/all]
  \ChangedAt{v2.02}
  Erklärungen, welche in der \Environment{declarations}"=Umgebung mit den 
  Befehlen \Macro{confirmation}, \Macro{blocking} und \Macro{declaration} oder 
  außerhalb dieser mit \Macro{declaration} gesetzt wurden, werden direkt 
  nacheinander auf der gleichen Seite ausgegeben, wenn ausreichend Platz auf 
  dieser vorhanden sein sollte. Ist die Option \Option{twocolumn} aktiviert, 
  erfolgt die Ausgabe ohne Spaltenumbruch.
\item[nofil/nofill/novfil/novfill]
  Die Ausgabe erfolgt wie im normalen Textsatz auch.
\item[fil/fill/vfil/vfill]
  Alle Erklärungen auf einer Ausgabeseite werden vertikal zentriert. Für 
  den zweispaltigen Satz steht diese Option nicht zur Verfügung.
\end{values}
%
Es ist zu beachten, dass die Einstellungen zur Positionierung der Erklärungen 
mit der Option \Option{declaration}[simple/multiple/fill/nofill] innerhalb der 
Umgebung \Environment{declarations} nur wirksam sind, wenn eine Titelseite 
(\Option{titlepage}[true]) und \emph{keine} Überschriften in Form von Kapiteln 
(\Option{declaration}[chapter]) verwendet werden.
\end{Declaration}

\begin{Declaration}[v2.02]{\Environment{declarations}[\LParameter]}
\begin{Declaration}{\Key{\Environment{declarations}}{language}[\PName{Sprache}]}
\begin{Declaration}[v2.02]{%
  \Key{\Environment{declarations}}{pagestyle}[\PName{Seitenstil}]%
}
\begin{Declaration}[v2.02]{%
  \Key{\Environment{declarations}}{columns}[\PName{Anzahl}]%
}
\begin{Declaration}{\Key{\Environment{declarations}}{option}[\PSet]}
\begin{Declaration}{%
  \Key{\Environment{declarations}}{supporter}[\PName{Unterstützer}]
}
\begin{Declaration}{\Key{\Environment{declarations}}{place}[\PName{Ort}]}
\begin{Declaration}{\Key{\Environment{declarations}}{closing}[\PName{Ende}]}
\begin{Declaration}{\Key{\Environment{declarations}}{company}[\PName{Firma}]}

\printdeclarationlist%
\index{Selbstständigkeitserklärung}\index{Sperrvermerk}%
%
Innerhalb dieser Umgebung können Selbstständigkeitserklärung und Sperrvermerk 
mit dem Befehl \Macro{declaration} direkt nacheinander folgend beziehungsweise 
mit \Macro{confirmation} und \Macro{blocking} auch separat ausgegeben werden. 
Dies kann in beliebiger Reihenfolge und auch mehrmals geschehen, um diese 
beispielsweise mehrsprachig zu setzen. Die im Folgenden beschriebenen Parameter 
können dabei sowohl für die \Environment{declarations}"=Umgebung selbst als 
auch für die zuvor genannten Befehle als optionales Argument verwendet werden.

Mit \Key{\Environment{declarations}}{language} kann die Sprache der Erklärung 
geändert werden. Der verwendete Seitenstil lässt sich mit dem Parameter 
\Key{\Environment{declarations}}{pagestyle}[\PName{Seitenstil}] angeben. 
Dieser unterstützt auch die \PageStyle{tudheadings}"=Seitenstile. Mit dem 
Parameter \Key{\Environment{declarations}}{columns}[\PName{Anzahl}] kann die 
Erklärung mehrspaltig gesetzt werden, wenn das Paket \Package{multicol} geladen 
wurde.

Die Verwendung der Parameter \Key{\Macro{confirmation}}{supporter} sowie
\Key{\Macro{confirmation}}{place} und \Key{\Macro{confirmation}}{closing} ist 
in der Dokumentation des Befehls \Macro{confirmation} zu finden, der Parameter 
\Key{\Macro{blocking}}{company} ist für \Macro{blocking} erläutert. Für den 
Parameter \Key{\Environment{declarations}}{option} können alle gültigen Werte 
der Option \Option{declaration} angegeben werden. Diese wirken sich nur lokal 
innerhalb der \Environment{declarations}"=Umgebung aus.
\end{Declaration}
\end{Declaration}
\end{Declaration}
\end{Declaration}
\end{Declaration}
\end{Declaration}
\end{Declaration}
\end{Declaration}
\end{Declaration}

\begin{Declaration}{\Macro{declaration}\LParameter}
\begin{Declaration}{\Key{\Macro{declaration}}{language}[\PName{Sprache}]}
\begin{Declaration}[v2.02]{%
  \Key{\Macro{declaration}}{pagestyle}[\PName{Seitenstil}]%
}
\begin{Declaration}[v2.02]{\Key{\Macro{declaration}}{columns}[\PName{Anzahl}]}
\begin{Declaration}{\Key{\Macro{declaration}}{option}[\PSet]}
\begin{Declaration}{\Key{\Macro{declaration}}{supporter}[\PName{Unterstützer}]}
\begin{Declaration}{\Key{\Macro{declaration}}{place}[\PName{Ort}]}
\begin{Declaration}{\Key{\Macro{declaration}}{closing}[\PName{Ende}]}
\begin{Declaration}{\Key{\Macro{declaration}}{company}[\PName{Firma}]}
\printdeclarationlist%
\index{Selbstständigkeitserklärung}\index{Sperrvermerk}%
%
Dieser Befehl gibt die Selbstständigkeitserklärung und den Sperrvermerk direkt 
aufeinanderfolgend aus. Dabei werden die Einstellungen zur Positionierung der 
einzelnen Erklärungen, welche über die Wertzuweisungen an die Option 
\Option{declaration}[simple/multiple/fill/nofill] erfolgen, beachtet. Er kann 
sowohl innerhalb der \Environment{declarations}"=Umgebung als auch außerhalb 
direkt im Dokument verwendet werden und akzeptiert im optionalen Argument dabei 
alle für die genannte Umgebung beschriebenen Parameter.
\end{Declaration}
\end{Declaration}
\end{Declaration}
\end{Declaration}
\end{Declaration}
\end{Declaration}
\end{Declaration}
\end{Declaration}
\end{Declaration}

\begin{Declaration}{\Macro{confirmation}\OLParameter{supporter}}
\begin{Declaration}{\Key{\Macro{confirmation}}{supporter}[\PName{Unterstützer}]}
\begin{Declaration}{\Key{\Macro{confirmation}}{place}[\PName{Ort}]}
\begin{Declaration}{\Key{\Macro{confirmation}}{closing}[\PName{Ende}]}
\begin{Declaration}{\Key{\Macro{confirmation}}{language}[\PName{Sprache}]}
\begin{Declaration}[v2.02]{%
  \Key{\Environment{confirmation}}{pagestyle}[\PName{Seitenstil}]%
}
\begin{Declaration}[v2.02]{%
  \Key{\Environment{confirmation}}{columns}[\PName{Anzahl}]%
}
\begin{Declaration}{\Key{\Macro{confirmation}}{option}[\PSet]}
\printdeclarationlist%
\index{Selbstständigkeitserklärung}\index{Datum}%
%
Mit diesem Befehl wird ein sprachspezifischer Standardtext für eine 
Selbstständigkeitserklärung ausgegeben, welcher in \Term{confirmationtext} 
gespeichert ist. Wie dieser angepasst beziehungsweise geändert werden kann, ist 
unter \autoref{sec:localization} zu finden. Er kann sowohl innerhalb der 
\Environment{declarations}"=Umgebung als auch außerhalb direkt im Dokument 
verwendet werden. 

Wird er in seiner ursprünglichen Form belassen, kann er im optionalen Argument 
über die deklarierten Parameter angepasst werden. Im Standardtext der 
Selbstständigkeitserklärung werden sowohl der Titel als auch der Typ der 
Abschlussarbeit~-- falls dieser mit \Macro{thesis} oder \Macro{subject} und 
einem speziellen Wert aus \autoref{tab:thesis} beziehungsweise mit der Option 
\Option{subjectthesis} angegeben wurde~-- aufgeführt. Über den Parameter 
\Key{\Macro{confirmation}}{supporter} können weitere an der Arbeit beteiligte 
Personen angegeben werden. Dies ist auch mit dem Befehl \Macro{supporter} 
möglich, wenn dieser \emph{vor} \Macro{confirmation} verwendet wird. Mehrere 
zu nennende Personen sind auch hier durch \Macro{and} zu trennen. Das Feld der 
Unterstützer kann auch mit dem bloßen optionalen Argument ohne die Angabe eines 
Parameters angepasst werden.

Nach dem eigentlichen Text der Selbstständigkeitserklärung wird der mit 
\Key{\Macro{confirmation}}{place} beziehungsweise \Macro{place} angegebene Ort 
sowie das mit \Macro{date} eingestellte Datum ausgegeben. Als Voreinstellung 
ist für den Ort \enquote{Dresden} gewählt. Danach folgen~-- mit etwas 
vertikalem Freiraum für die notwendige Unterschrift~-- der Autor oder die 
Autoren, angegeben durch den Befehl \Macro{author}. Soll anstelle dessen etwas 
anderes nach dem Text der Selbstständigkeitserklärung gesetzt werden, kann dies 
mit dem Parameter \Key{\Macro{confirmation}}{closing} oder zuvor mit dem 
Befehl \Macro{confirmationclosing} angepasst werden. Die Parameter 
\Key{\Environment{declarations}}{language}, 
\Key{\Environment{declarations}}{pagestyle}, 
\Key{\Environment{declarations}}{columns} und 
\Key{\Environment{declarations}}{option} entsprechen in ihrem Verhalten denen 
der \Environment{declarations}"=Umgebung.
\end{Declaration}
\end{Declaration}
\end{Declaration}
\end{Declaration}
\end{Declaration}
\end{Declaration}
\end{Declaration}
\end{Declaration}

\Rename[macros]{v2.02}{\Macro{restriction}}{\Macro{blocking}}
\begin{Declaration}{\Macro{blocking}\OLParameter{company}}
\begin{Declaration}{\Key{\Macro{blocking}}{company}[\PName{Firma}]}
\begin{Declaration}{\Key{\Macro{blocking}}{language}[\PName{Sprache}]}
\begin{Declaration}[v2.02]{%
  \Key{\Environment{blocking}}{pagestyle}[\PName{Seitenstil}]%
}
\begin{Declaration}[v2.02]{%
  \Key{\Environment{blocking}}{columns}[\PName{Anzahl}]%
}
\begin{Declaration}{\Key{\Macro{blocking}}{option}[\PSet]}
\printdeclarationlist%
\index{Sperrvermerk}%
%
Beim Sperrvermerk verhält es sich äquivalent zur Selbstständigkeitserklärung.
Es wird der in \Term{blockingtext} hinterlegte Standardtext in der gewählten 
Sprache ausgegeben. Dieser kann durch den Anwender geändert werden. Wie genau 
ist in \autoref{sec:localization} beschrieben. Der Befehl \Macro{blocking} 
kann sowohl innerhalb der Umgebung \Environment{declarations} als auch 
außerhalb direkt im Dokument verwendet werden. 

In seiner ursprünglichen Definition, kann er im optionalen Argument über die 
deklarierten Parameter angepasst werden. Im Standardtext des Sperrvermerks 
werden sowohl der Titel als auch der Typ der Abschlussarbeit~-- falls dieser 
mit \Macro{thesis} oder \Macro{subject} und einem speziellen Wert aus 
\autoref{tab:thesis} beziehungsweise mit der Option \Option{subjectthesis} 
angegeben wurde~-- aufgeführt. Mit \Key{\Macro{blocking}}{company}~-- oder 
\emph{vorher} mit dem Befehl \Macro{company}~-- kann zusätzlich eine im 
Sperrvermerk zu nennende Firma oder ähnliches angegeben werden. Dieses Feld 
kann auch direkt im optionalen Argument ohne die Verwendung eines Parameters 
gesetzt werden. Die weiteren Parameter 
\Key{\Environment{declarations}}{language}, 
\Key{\Environment{declarations}}{pagestyle}, 
\Key{\Environment{declarations}}{columns} und 
\Key{\Environment{declarations}}{option} entsprechen in ihrem Verhalten denen 
der \Environment{declarations}"=Umgebung.
\end{Declaration}
\end{Declaration}
\end{Declaration}
\end{Declaration}
\end{Declaration}
\end{Declaration}

\begin{Declaration}{\Macro{supporter}\Parameter{Unterstützer}}
\begin{Declaration}{\Macro{place}\Parameter{Ort}}
\begin{Declaration}{\Macro{confirmationclosing}\Parameter{Ende}}
\begin{Declaration}{\Macro{company}\Parameter{Firma}}
\printdeclarationlist%
\index{Selbstständigkeitserklärung}\index{Sperrvermerk}%
%
Diese Makros ändern~-- im Gegensatz zu den Parametern der bereits vorgestellten 
Befehle \Macro{confirmation} und \Macro{blocking}~-- die entsprechenden 
Feldwerte global für das gesamte Dokument. Genutzt werden kann dies 
beispielsweise wenn ein Erklärungstyp in unterschiedlichen Sprachen ausgegeben 
wird. Dann kann man sich mit diesen Makros die mehrfache Angabe eines 
Parameters sparen.
\end{Declaration}
\end{Declaration}
\end{Declaration}
\end{Declaration}


\subsection{Lesezeichen}
\begin{Declaration}{\Option{tudbookmarks}[\PBoolean]}[true]%
\begin{Declaration}{%
  \Macro{tudbookmark}\OParameter{Ebene}\Parameter{Text}\Parameter{Ankername}%
}%
\printdeclarationlist%
\index{Lesezeichen}%
\index{Umschlagseite}\index{Titel}\index{Inhaltsverzeichnis}%
\index{Aufgabenstellung}\index{Gutachten}\index{Aushang}%
%
Diese Option wird wirksam, wenn \Package{hyperref} geladen wurde. Es werden für 
die Umschlag- und Titelseite, das Inhaltsverzeichnis sowie~-- bei der 
Verwendung des Paketes \Package{tudscrsupervisor}~-- die Aufgabenstellung 
jeweils Lesezeichen oder auch Outline"=Einträge im PDF-Dokument erzeugt.
%
\begin{values}
\itemfalse
  Es erfolgt kein Eintrag von ergänzenden Lesezeichen.
\itemtrue*
  Es werden automatisch zusätzliche Lesezeichen eingetragen.
\end{values}
%
Der Befehl \Macro{tudbookmark} arbeitet wie \Macro{pdfbookmark} aus 
\Package{hyperref} mit dem Unterschied, dass die Lesezeichen nur generiert 
werden, wenn die Option \Option{tudbookmarks} aktiviert ist.
\end{Declaration}
\end{Declaration}



\section{Sprachabhängige Bezeichner}
\label{sec:localization}
\index{Bezeichner|!(}%
%
Durch \KOMAScript{} werden Befehle, mit denen sprachabhängige Bezeichner 
erzeugt oder geändert werden können, zur Verfügung gestellt. Diese werden durch 
das \TUDScript-Bundle genutzt, um lokalisierte Begriffe für die Sprachen 
Englisch und Deutsch bereitzustellen. Ein Großteil davon betrifft Bezeichnungen 
für Felder auf der Titelseite (\autoref{sec:title}). Hierfür wird
\Macro{providecaptionname}\Parameter{Sprache}\Parameter{Makro}\Parameter{Inhalt}
verwendet, wobei \PName{Sprache} dem geladenen Sprachpaket~-- normalerweise das 
Paket \Package{babel}~-- bekannt sein muss.

Sollte der Anwender die im Folgenden erläuterten oder auch andere Bezeichner, 
welche von einem beliebigen (Sprach"~)Paket bereitgestellt werden, ändern 
wollen, ist hierfür der Befehl
\Macro{renewcaptionname}\Parameter{Sprache}\Parameter{Makro}\Parameter{Inhalt} 
zu verwenden. Seit der Version~v3.12 stellt \KOMAScript sicher, dass die 
Bezeichner erst \textbf{nach} \Macro*{begin}\PParameter{document} angepasst 
werden und somit nicht durch ein später geladenes Paket abermals geändert 
werden können. Es sollte natürlich dabei eine \PName{Sprache} angegeben werden, 
welche im Dokument durch \Package{babel} oder ein anderes Sprachpaket verwendet 
wird, beispielsweise \PValue{ngerman} oder \PValue{english}. 

Die Makros der Bezeichner und deren Verwendung werden folgend kurz beschrieben 
und tabellarisch aufgeführt. Dabei wurde versucht, alle Befehle der Bezeichner 
für bestimmte Begriffe auf \PValue{\dots{}name} und beschreibende Texte auf 
\PValue{\dots{}text} enden zu lassen.

\begin{Declaration}{\Term{dissertationname}}
\begin{Declaration}{\Term{diplomathesisname}}
\begin{Declaration}{\Term{masterthesisname}}
\begin{Declaration}{\Term{bachelorthesisname}}
\begin{Declaration}{\Term{studentresearchname}}
\begin{Declaration}{\Term{projectpapername}}
\begin{Declaration}{\Term{seminarpapername}}
\begin{Declaration}{\Term{researchname}}
\begin{Declaration}{\Term{logname}}
\begin{Declaration}{\Term{internshipname}}
\begin{Declaration}{\Term{reportname}}
\printdeclarationlist%
\index{Titel}\index{Abschlussarbeit}\index{Typisierung}%
%
Diese Bezeichner dienen zur Typisierung speziell für eine Abschlussarbeit. Wie 
diese genutzt werden können, ist bei der Erläuterung von \Macro{thesis} und 
\Macro{subject}'full' beziehungsweise in \autoref{tab:thesis} zu finden.
\TermTable{%
  dissertationname,diplomathesisname,masterthesisname,bachelorthesisname,%
  studentresearchname,projectpapername,seminarpapername,researchname,%
  logname,internshipname,reportname%
}
\end{Declaration}
\end{Declaration}
\end{Declaration}
\end{Declaration}
\end{Declaration}
\end{Declaration}
\end{Declaration}
\end{Declaration}
\end{Declaration}
\end{Declaration}
\end{Declaration}

\begin{Declaration}{\Term{supervisorname}}
\begin{Declaration}{\Term{supervisorothername}}
\begin{Declaration}[%
  v2.02!Unterscheidung zwischen einem und mehreren Gutachtern%
]{\Term{refereename}}
\begin{Declaration}{\Term{refereeothername}}
\begin{Declaration}{\Term{advisorname}}
\begin{Declaration}{\Term{advisorothername}}
\begin{Declaration}[%
  v2.02!Unterscheidung von einem und mehreren Professoren%
]{\Term{professorname}}
\begin{Declaration}[v2.02]{\Term{professorothername}}
\printdeclarationlist%
\index{Titel}%
\index{Betreuer}\index{Gutachter}\index{Hochschullehrer}%
\index{Referent}%
%
Diese sprachabhängigen Begriffe sind die Bezeichner für die Titelseitenfelder 
von Betreuer (\Macro{supervisor}), Gutachter (\Macro{referee}) und Fachreferent 
(\Macro{advisor}). Soll innerhalb eines dieser Felder mehr als eine Person 
angegeben werden, so sind die Einzelpersonen jeweils mit dem Befehl \Macro{and} 
voneinander zu trennen. In diesem Fall werden alle nach der erstgenannten 
folgenden Personen durch den Bezeichner \PValue{\bsc\dots{}othername} ergänzt.

\ChangedAt{v2.02}
Bei der Bezeichnung des Gutachters wird unterschieden, ob einer oder mehrere 
angegeben wurden. Wird lediglich einer genannt, so ist eine Unterscheidung 
nicht notwendig. Werden jedoch zwei Gutachter angegeben, so werden diese auch 
mit Erst- und Zweitgutachter betitelt. Für den betreuenden Hochschullehrer 
(\Macro{professor}) wird ähnlich verfahren. Hier wird allerdings lediglich 
die Bezeichnung vom Singular in den Plural gegebenenfalls automatisch geändert.


\renewcaptionname{ngerman}{\refereename}{Gutachter/Erstgutachter}
\renewcaptionname{english}{\refereename}{Referee/First referee}
\TermTable{%
  supervisorname,supervisorothername,refereename,refereeothername,%
  advisorname,advisorothername,professorname,professorothername%
}
\end{Declaration}
\end{Declaration}
\end{Declaration}
\end{Declaration}
\end{Declaration}
\end{Declaration}
\end{Declaration}
\end{Declaration}

\begin{Declaration}{\Term{dateofbirthtext}}
\begin{Declaration}{\Term{placeofbirthtext}}
\begin{Declaration}{\Term{matriculationnumbername}}
\begin{Declaration}{\Term{matriculationyearname}}
\printdeclarationlist%
\index{Titel}\index{Autorenangaben}\index{Datum!Geburtsdatum}%
%
Werden für den Autor oder die Autoren das Geburtsdatum (\Macro{dateofbirth}), 
der Geburtsort (\Macro{placeofbirth}) sowie die
Matrikelnummer (\Macro{matriculationnumber}) und/oder das Immatrikulationsjahr 
(\Macro{matriculationyear}) angegeben, werden sowohl auf der Titelseite als 
auch auf der gegebenenfalls mit \Package{tudscrsupervisor} erstellten 
Aufgabenstellung die dazugehörigen Bezeichner vorangestellt. Auf dem Titel 
werden diese dabei mit dem durch \Macro{titledelimiter} gegebenen Trennzeichen 
vom eigentlichen Feld abgegrenzt.
\TermTable{%
  dateofbirthtext,placeofbirthtext,matriculationnumbername,%
  matriculationyearname%
}
\end{Declaration}
\end{Declaration}
\end{Declaration}
\end{Declaration}

\Rename[terms]{v2.02}{\Term{degreetext}}{\Term{graduationtext}}
\begin{Declaration}{\Term{graduationtext}}
\printdeclarationlist%
\index{Titel}\index{Abschlussarbeit}\index{Typisierung}%
%
Wurde erkannt, dass das Dokument eine Abschlussarbeit ist,%
\footnote{%
  Entweder wurde \Macro{thesis} oder \Macro{subject} mit einem speziellen Wert 
  oder der Option \Option{subjectthesis} verwendet.
}
so kann der zu erlangende akademische Grad mit dem Befehl \Macro{graduation} 
angegeben werden. Bei dessen Ausgabe auf dem Titel wird dabei der entsprechende 
Text dazu angegeben.
\TermTable*{graduationtext}{.78\textwidth}
\end{Declaration}

\begin{Declaration}{\Term{datetext}}
\begin{Declaration}{\Term{defensedatetext}}
\printdeclarationlist%
\index{Titel}\index{Abschlussarbeit}%
\index{Datum}\index{Datum!Verteidigungsdatum}%
%
Wird mit \Macro{date} das Datum und mit \Macro{defensedate} ein Datum der 
Verteidigung für eine Abschlussarbeit angegeben, so werden auch diese Felder 
durch einen einleitenden Text beschrieben.
\TermTable{datetext,defensedatetext}
\end{Declaration}
\end{Declaration}

\begin{Declaration}{\Term{coverpagename}}
\begin{Declaration}{\Term{titlepagename}}
\printdeclarationlist%
\index{Lesezeichen}\index{Umschlagseite}\index{Titel!Umschlagseite}\index{Titel}
%
Diese beiden Bezeichner werden bei aktivierter \Option{tudbookmarks} für das 
Eintragen von Lesezeichen in ein PDF"=Dokument genutzt.
\TermTable{coverpagename,titlepagename}
\end{Declaration}
\end{Declaration}

\begin{Declaration}{\Term{listingname}}
\begin{Declaration}{\Term{listlistingname}}
\printdeclarationlist%
%
Sollte ein Paket zur Einbindung von externem Quelltext~-- beispielsweise 
das Paket \Package{listings}~-- verwendet werden, so werden diese Bezeichnungen 
für Quelltextausschnitte und das Quelltextverzeichnis verwendet.
\TermTable{listingname,listlistingname}
\end{Declaration}
\end{Declaration}

\begin{Declaration}{\Term{abstractname}}
\printdeclarationlist%
%
Dieser Bezeichner wird lediglich für \Class{tudscrbook} definiert, da dieser 
von \KOMAScript{} für die Buchklasse nicht vorgesehen wird.
\TermTable{abstractname}
\end{Declaration}

\Rename[terms]{v2.02}{\Term{restrictionname}}{\Term{blockingname}}
\begin{Declaration}{\Term{confirmationname}}
\begin{Declaration}{\Term{blockingname}}
\printdeclarationlist%
\index{Selbstständigkeitserklärung}\index{Sperrvermerk}%
%
Es werden die Bezeichnungen für Selbstständigkeitserklärung und Sperrvermerk 
für die dazugehörigen Überschriften definiert.
\TermTable{confirmationname,blockingname}
\end{Declaration}
\end{Declaration}

\Rename[terms]{v2.02}{\Term{restrictiontext}}{\Term{blockingtext}}
\begin{Declaration}{\Term{confirmationtext}}
\begin{Declaration}{\Term{blockingtext}}
\printdeclarationlist%
%
Die Texte der Erklärungen selbst sind derart aufgebaut, dass sie in 
Abhängigkeit von den angegebenen Informationen unterschiedlich ausgeführt 
werden. Innerhalb der Selbstständigkeitserklärung (\Macro{confirmation}) werden 
gegebenenfalls die Felder für den Titel (\Macro{title}) und die Typisierung der 
Abschlussarbeit%
\footnote{%
  entweder \Macro{thesis} oder \Macro{subject}\Parameter{\autoref{tab:thesis}}
  beziehungsweise Option \Option{subjectthesis}[true]
}
sowie die angegebenen Unterstützer%
\footnote{%
  \Macro{confirmation}\POParameter{\Key{\Macro{confirmation}}{supporter}=\dots}
  oder \Macro{supporter}\PParameter{\dots}%
}
beachtet. Für den Sperrvermerk (\Macro{blocking}) wird neben dem Titel 
(\Macro{title}) optional außerdem noch das Feld der externen Firma%
\footnote{%
  \Macro{blocking}\POParameter{\Key{\Macro{blocking}}{company}=\dots} oder 
  \Macro{company}\PParameter{\dots}%
}
verwendet. Der Vollständigkeit halber werden im Folgenden noch die Texte für 
die Selbstständigkeitserklärung und den Sperrvermerk aufgeführt~-- allerdings 
lediglich die deutschsprachige Version. Dabei werden alle möglichen Felder 
angezeigt.

\begingroup
  \makeatletter
  \def\@@title{\PName{Titel}}
  \def\@@thesis{\PName{Abschlussarbeit}}
  \def\@supporter{\PName{Vorname Nachname} \and \PName{Vorname Nachname}}
  \def\@company{\PName{Firma}}
  \makeatother
  
  \vskip\baselineskipglue\noindent
  \textbf{Bezeichner}\quad\Term*{confirmationtext}%
  \par\vskip\baselineskipglue%
  \begin{center}\begin{minipage}{.8\textwidth}
  \confirmationtext
  \end{minipage}\end{center}
  
  \vskip\baselineskipglue\noindent
  \textbf{Bezeichner}\quad\Term*{blockingtext}%
  \par\vskip\baselineskipglue%
  \begin{center}\begin{minipage}{.8\textwidth}
  \blockingtext
  \end{minipage}\end{center}
\endgroup
\end{Declaration}
\end{Declaration}
\index{Bezeichner|!)}
