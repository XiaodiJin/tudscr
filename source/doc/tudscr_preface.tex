\addchap{\prefacename}
Die im Folgenden beschriebenen Klassen und Pakete wurden für das Erstellen von 
\hologo{LaTeX}"=Dokumenten im \CD der \TnUD entwickelt.%
\footnote{%
  \url{http://tu-dresden.de/cd}\hfill
  \url{http://tu-dresden.de/service/publizieren/cd/6_handbuch/index.html}%
}
Sie basieren auf den gerade im deutschsprachigen Raum häufig verwendeten 
\KOMAScript"=Klassen, welche eine Vielzahl von Einstellmöglichkeiten bieten, 
die weit über die Möglichkeiten der \hologo{LaTeX}"=Standardklassen 
hinausgehen. Zusätzlich bietet das hier dokumentierten \TUDScript-Bundle 
weitere, insbesondere das Dokumentlayout betreffende Auswahlmöglichkeiten.

Es sei angemerkt, dass die hier beschriebenen Klassen eine Abweichung vom \CD 
der \TnUD zulassen, da dieses gerade unter typographischen Gesichtspunkten 
durchaus als diskussionswürdig zu erachten ist. Mit den entsprechenden 
Einstellungen kann bis auf das Standardlayout der \KOMAScript"=Klassen 
zurückgestellt werden. Inwieweit der Nutzer der \TUDScript"=Klassen von diesen 
Möglichkeiten Gebrauch macht, bleibt ihm selbst überlassen. Ohne die gezielte 
Verwendung der entsprechenden Optionen werden standardmäßig alle Vorgaben des 
\CDs umgesetzt.

Dieses Handbuch soll dazu dienen, eine schnelle Einführung in die neuen Klassen
und Pakete zu ermöglichen. Es werden Hinweise für eine einfache Installation 
und einen Überblick über die zusätzlich zu den \KOMAScript"=Klassen nutzbaren 
Optionen sowie die neu eingeführten Befehle gegeben. Dies bedeutet, dass 
Grundkenntnisse in der Verwendung von \hologo{LaTeX} vorausgesetzt werden. 
Sollten diese nicht vorhanden sein, wird dem Nutzer zumindest das Lesen der 
Kurzbeschreibung von \hologo{LaTeXe}
\hrfn{http://mirrors.ctan.org/info/lshort/german/l2kurz.pdf}{\File{l2kurz.pdf}}
dringend empfohlen. Des Weiteren sollte sowohl der Einsteiger als auch der 
erfahrene Nutzer mindestens einmal das \hologo{LaTeXe}"=Sündenregister
\hrfn{http://mirrors.ctan.org/info/l2tabu/german/l2tabu.pdf}{\File{l2tabu.pdf}}
überblickt haben, um sehr typische Fehler beim Umgang mit \hologo{LaTeX} zu 
vermeiden. Ein umfangreiches Tutorial für \hologo{LaTeX}-Einsteiger ist unter 
diesem \hrfn{http://www.fadi-semmo.de/latex/workshop/}{Link} zu finden. 
Antworten auf häufige Fragen liefert
\hrfn{http://projekte.dante.de/DanteFAQ/WebHome}{DANTE-FAQ}. Sollte der Nutzer 
unerfahren bei der Verwendung der \KOMAScript"=Klassen sein, so ist ein Blick 
in das dazugehörige Anwenderhandbuch
\hrfn{http://mirrors.ctan.org/macros/latex/contrib/koma-script/doc/scrguide.pdf}
{\File{scrguide.pdf}} sehr zu empfehlen, wenn nicht sogar unumgänglich.
Nichtsdestotrotz werden in \autoref{part:additional} Minimalbeispiele sowie 
etwas ausführlichere Tutorials für \hologo{LaTeX}-Neulinge angeboten. 

Der aktuelle Stand der Klassen und Pakete aus dem \TUDScript-Bundle wurde nach 
bestem Wissen und Gewissen auf Herz und Nieren getestet. Dennoch kann nicht für 
das Ausbleiben von Fehlern garantiert werden. Beim Auftreten eines Problems 
sollte dieses bitte genauso wie Inkompatibilitäten mit anderen Paketen im Forum 
unter
\begin{quote}
\Forum*%
\end{quote}
gemeldet beziehungsweise geäußert werden. Für eine schnelle und erfolgreiche 
Fehlersuche sollte ein \hrfn{http://www.komascript.de/minimalbeispiel} 
{\textbf{lauffähiges~Minimalbeispiel}} bereitgestellt werden. Auf Anfragen ohne 
dieses werde ich gegebenenfalls verspätet oder gar nicht reagieren. Ebenso sind 
dort auch \emph{Fragen}, \emph{Kritik} und \emph{Verbesserungsvorschläge}~-- 
sowohl das Bundle selbst als auch die Dokumentation betreffend~-- gerne 
gesehen. Da dieses Bundle in meiner Freizeit entstanden ist und auch gepflegt 
wird, bitte ich um Nachsicht, falls ich nicht sofort antworte und/oder eine 
Fehlerkorrektur vornehmen kann.

\makeatletter
\bigskip
\noindent Falk Hanisch\newline
Dresden, \@date
\makeatother