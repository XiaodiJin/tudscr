\RequirePackage[ngerman=ngerman-x-latest]{hyphsubst}
\documentclass[english,ngerman]{tudscrman}
\usepackage{selinput}\SelectInputMappings{adieresis={ä},germandbls={ß}}
\usepackage[T1]{fontenc}
\lstset{%
  inputencoding=utf8,extendedchars=true,
  literate=%
    {ä}{{\"a}}1 {ö}{{\"o}}1 {ü}{{\"u}}1
    {Ä}{{\"A}}1 {Ö}{{\"O}}1 {Ü}{{\"U}}1
    {~}{{\textasciitilde}}1 {ß}{{\ss}}1
}
\begin{document}
\addtokomafont{subtitle}{\univbn}
\subject{\TUDScript{} \vTUDScript{} basierend auf \KOMAScript{} \vKOMAScript}
\title{%
  Ein \NoCaseChange{\hologo{LaTeXe}}-Bundle für Dokumente im~neuen \CD der \TnUD
}
\ifdef{\tudprintflag}{%
  \subtitle{Benutzerhandbuch\thanks{\href{tudscr}{Online-Version}}}%
}{%
  \subtitle{Benutzerhandbuch\thanks{\href{tudscr_print}{Druckversion}}}%
}
\author{Falk Hanisch\thanks{\noexpand\href{mailto:\tudscrmail}{\tudscrmail}}}
\faculty{http://tu-dresden.de/cd}
\date{09.09.2014}
\maketitle
\addchap{\prefacename}
Die im Folgenden beschriebenen Klassen und Pakete wurden für das Erstellen von 
\hologo{LaTeX}"=Dokumenten im \CD der \TnUD entwickelt.%
\footnote{%
  \url{http://tu-dresden.de/cd}\hfill
  \url{http://tu-dresden.de/service/publizieren/cd/6_handbuch/index.html}%
}
Sie basieren auf den gerade im deutschsprachigen Raum häufig verwendeten 
\KOMAScript"=Klassen, welche eine Vielzahl von Einstellmöglichkeiten bieten, 
die weit über die Möglichkeiten der \hologo{LaTeX}"=Standardklassen 
hinausgehen. Zusätzlich bietet das hier dokumentierten \TUDScript-Bundle 
weitere, insbesondere das Dokumentlayout betreffende Auswahlmöglichkeiten.

Es sei angemerkt, dass die hier beschriebenen Klassen eine Abweichung vom \CD 
der \TnUD zulassen, da dieses gerade unter typographischen Gesichtspunkten 
durchaus als diskussionswürdig zu erachten ist. Mit den entsprechenden 
Einstellungen kann bis auf das Standardlayout der \KOMAScript"=Klassen 
zurückgestellt werden. Inwieweit der Nutzer der \TUDScript"=Klassen von diesen 
Möglichkeiten Gebrauch macht, bleibt ihm selbst überlassen. Ohne die gezielte 
Verwendung der entsprechenden Optionen werden standardmäßig alle Vorgaben des 
\CDs umgesetzt.

Dieses Handbuch soll dazu dienen, eine schnelle Einführung in die neuen Klassen
und Pakete zu ermöglichen. Es werden Hinweise für eine einfache Installation 
und einen Überblick über die zusätzlich zu den \KOMAScript"=Klassen nutzbaren 
Optionen sowie die neu eingeführten Befehle gegeben. Dies bedeutet, dass 
Grundkenntnisse in der Verwendung von \hologo{LaTeX} vorausgesetzt werden. 
Sollten diese nicht vorhanden sein, wird dem Nutzer zumindest das Lesen der 
Kurzbeschreibung von \hologo{LaTeXe}
\hrfn{http://mirrors.ctan.org/info/lshort/german/l2kurz.pdf}{\File{l2kurz.pdf}}
dringend empfohlen. Des Weiteren sollte sowohl der Einsteiger als auch der 
erfahrene Nutzer mindestens einmal das \hologo{LaTeXe}"=Sündenregister
\hrfn{http://mirrors.ctan.org/info/l2tabu/german/l2tabu.pdf}{\File{l2tabu.pdf}}
überblickt haben, um sehr typische Fehler beim Umgang mit \hologo{LaTeX} zu 
vermeiden. Ein umfangreiches Tutorial für \hologo{LaTeX}-Einsteiger ist unter 
diesem \hrfn{http://www.fadi-semmo.de/latex/workshop/}{Link} zu finden. 
Antworten auf häufige Fragen liefert
\hrfn{http://projekte.dante.de/DanteFAQ/WebHome}{DANTE-FAQ}. Sollte der Nutzer 
unerfahren bei der Verwendung der \KOMAScript"=Klassen sein, so ist ein Blick 
in das dazugehörige Anwenderhandbuch \scrguide sehr zu empfehlen, wenn nicht 
sogar unumgänglich. Nichtsdestotrotz werden in \autoref{part:additional} 
Minimalbeispiele sowie etwas ausführlichere Tutorials für angeboten. 

Der aktuelle Stand der Klassen und Pakete aus dem \TUDScript-Bundle wurde nach 
bestem Wissen und Gewissen auf Herz und Nieren getestet. Dennoch kann nicht für 
das Ausbleiben von Fehlern garantiert werden. Beim Auftreten eines Problems 
sollte dieses bitte genauso wie Inkompatibilitäten mit anderen Paketen im Forum 
unter
\begin{quote}
\Forum*%
\end{quote}
gemeldet beziehungsweise geäußert werden. Für eine schnelle und erfolgreiche 
Fehlersuche sollte ein \hrfn{http://www.komascript.de/minimalbeispiel} 
{\textbf{lauffähiges~Minimalbeispiel}} bereitgestellt werden. Auf Anfragen ohne 
dieses werde ich gegebenenfalls verspätet oder gar nicht reagieren. Ebenso sind 
dort auch \emph{Fragen}, \emph{Kritik} und \emph{Verbesserungsvorschläge}~-- 
sowohl das Bundle selbst als auch die Dokumentation betreffend~-- gerne 
gesehen. Da dieses Bundle in meiner Freizeit entstanden ist und auch gepflegt 
wird, bitte ich um Nachsicht, falls ich nicht sofort antworte und/oder eine 
Fehlerkorrektur vornehmen kann.

\makeatletter
\bigskip
\noindent Falk Hanisch\newline
Dresden, \@date
\makeatother
\makeatletter
\renewcommand*\@pnumwidth{1.7em}
\makeatother
\tableofcontents
\chapter{Einleitung}
\index{Distribution}
Für die Verwendung der \TUDScript-Klassen der Version~\vTUDScript{} werden 
zwingend die \KOMAScript"=Klassen~\vKOMAScript{} sowie die Schriften des \CDs 
\Univers und \DIN benötigt. Außerdem müssen durch die verwendete 
\hologo{LaTeX}"=Distribution weitere Pakete bereitgestellt werden. 
Bei den aktuellen Distributionen \Distribution{\hologo{TeX}~Live}[2014], 
\Distribution{Mac\hologo{TeX}}[2014] und \Distribution{\hologo{MiKTeX}}[2.9] 
ist das mit großer Sicherheit kein Problem. Ist jedoch eine ältere Distribution
installiert, könnte dies zu Problemen führen. Dann sollte bestenfalls
eine der aktuellen Distributionen installiert werden. Zumindest müssen jedoch 
die unter \autoref{sec:packages:needed} aufgeführten Pakete sowie \TUDScript 
(\autoref{sec:local:install}) in ihrer neuesten Version lokal installiert sein.

Das Vorlagenpaket von Klaus Bergmann ist für die Verwendung nicht notwendig. 
Allerdings beinhaltet dieses weitere Klassen zum Erstellen von Folien 
und Briefen.%
\footnote{%
  \Class{tudbook}, \Class{tudbeamer}, \Class{tudletter}, \Class{tudfax}, 
  \Class{tudhaus}, \Class{tudform}
}
Das \TUDScript-Bundle ist hauptsächlich für das Erstellen wissenschaftlicher 
Texte und Arbeiten gedacht und soll die ursprünglichen Vorlagen \emph{momentan} 
nicht ersetzen sondern vielmehr ergänzen. 

Eine Umsetzung des \CDs für die \Class{beamer}"=Klasse sowie für Briefe und 
Geschäftsschreiben auf Basis der \KOMAScript"=Brief"=Klasse \Class{scrlttr2} 
ist bis jetzt leider noch nicht entstanden, soll jedoch langfristig 
bereitgestellt werden. Allerdings existieren bereits im Bundle 
\Class{tudmathposter} für die \Class{beamer}"=Klasse mehrere Stile. Dieses 
Bundle ist sowohl im \hrfn{https://github.com/tud-cd/tud-cd}{GitHub} als 
auch auf der \hrfn{http://tu-dresden.de/service/publizieren/cd/4_latex} 
{\hologo{LaTeX}-Seite der \TnUD} zu finden.



\section{Zur Verwendung dieses Handbuchs}
Sämtliche neu definierten Optionen, Umgebungen und Befehle der 
\TUDScript-Klassen und \TUDScript-Pakete werden im Handbuch aufgeführt und 
beschrieben. Am Ende des Dokumentes befinden sich mehrere Indizes, die das 
Nachschlagen oder Auffinden von bisher unbekannten Befehlen oder Optionen 
erleichtern sollen.

Die folgend beschriebenen Optionen können~-- wie ein Großteil der Einstellungen 
der \KOMAScript"=Klassen~-- in der Syntax des \Package{keyval}"=Paketes als 
Schlüssel"=Wert"=Paare bei der Wahl der Dokumentklasse angegeben werden:
\Macro*{documentclass}\POParameter{\PName{Schlüssel}\PValue{=}\PName{Wert}}%
\Parameter{Klasse}.

Des Weiteren eröffnen die \KOMAScript"=Klassen die Möglichkeit der späten 
Optionenwahl. Damit können Optionen nicht nur direkt beim Laden als sogenannte 
Klassenoptionen angegeben werden, sondern lassen sich auch noch innerhalb des 
Dokumentes nach dem Laden der Klasse ändern. Die \KOMAScript"=Klassen sehen 
hierfür zwei Befehle vor. Mit \Macro{KOMAoptions}\Parameter{Optionenliste}
kann man beliebig vielen Schlüsseln jeweils genau einen Wert zuweisen, 
\Macro{KOMAoption}\Parameter{Option}\Parameter{Werteliste} erlaubt das 
gleichzeitige Setzen mehrere Werte für genau einen Schlüssel. Äquivalent 
dazu werden für die \emph{zusätzlichen} Optionen der \TUDScript-Klassen mit 
die Befehle \Macro{TUDoptions}\Parameter{Optionenliste} und 
\Macro{TUDoption}\Parameter{Option}\Parameter{Werteliste} definiert. Damit kann
das Verhalten von Optionen im Dokument~-- innerhalb einer Gruppe auch lokal~-- 
geändert werden.

Die Voreinstellung einer jeden Option ist durch \PValue{preset:\,}\PName{Wert}
bei deren Beschreibung angegeben. Einige dieser Standardwerte sind nicht 
immer gleich sondern werden zusätzlich in Abhängigkeit der genutzten Optionen 
und Benutzereinstellungen gesetzt. Diese bedingten Voreinstellungen werden durch
\PValue{preset:\,}\PName{Wert}\PValue{\,|\,}\PName{Bedingung}\PValue{:\,}%
\PName{bedingter~Wert} angegeben.

Jedem Schlüssel wird normalerweise durch den Benutzer ein gewünschter, gültiger 
Wert zugewiesen. Wird ein Schlüssel jedoch ohne Wertzuweisung genutzt, so 
wird~-- falls vorhanden~-- ein vordefinierter Säumniswert gesetzt, welcher in 
der Beschreibung der einzelnen Optionen durch die \PValue{\emph{kursive}} 
Schreibweise gekennzeichnet ist. In den allermeisten Fällen ist der Säumniswert
eines Schlüssels \PValue{true}, er entspricht folglich der Angabe 
\PName{Schlüssel}\PValue{=true}. Mit der expliziten Wertzuweisung 
eines Schlüssels durch den Benutzer werden immer sowohl normale als auch 
bedingte Voreinstellungen überschrieben. Die neben den Optionen neu 
eingeführten Befehle und Umgebungen der Klassen werden im gleichen Stil 
erläutert.



\section{Installation des \TUDScript-Bundles}
\label{sec:install}%
\index{Installation}\index{Update}%
%
\ChangedAt{v2.01!\TUDScript-Bundle auf CTAN veröffentlicht}%
%
Das \TUDScript-Bundle ist seit der Version~v2.01~-- aufgrund lizenzrechtlicher 
Bedingungen \emph{ohne} die geschützten Schriften \Univers und \DIN~-- im 
\foreignlanguage{english}{%
  \hrfn{http://www.ctan.org/}{Comprehensive TeX Archive Network (CTAN)}
} zu finden und dadurch in aktuellen \hologo{LaTeX}"=Distributionen wie 
\Distribution{\hologo{TeX}~Live}[2014], \Distribution{Mac\hologo{TeX}}[2014] 
oder auch \Distribution{\hologo{MiKTeX}}[2.9] enthalten. Aus diesem Grund ist 
eine komplett lokale Nutzerinstallation wie in den vorherigen Versionen nicht 
mehr notwendig. Auch manuelle Updates des \TUDScript-Bundles sind~-- für den 
Fall, dass die \hologo{LaTeX}"=Distribution durch den Nutzer aktuell gehalten 
wird~-- prinzipiell nicht mehr nötig. Zusätzlich werden im GitHub-Repository 
\hrfn{https://github.com/tud-cd/tudscr}{\Package*{tudscr}} die Quelldateien 
sowie Installationsskripte bereitgestellt.

Zur problemlosen Verwendung des \TUDScript-Bundles ist~-- neben \KOMAScript{} in
der Version~\vKOMAScript und den in \autoref{sec:packages:needed} aufgeführten 
\hologo{LaTeX}-Paketen~-- lediglich eine Installation der PostScript"=Schriften 
des \CDs der \TnUD notwendig. Diese müssen über das Universitätsmarketing auf 
\hrfn{http://tu-dresden.de/service/publizieren/cd/1_basiselemente/03_hausschrift/schriftbestellung.html}%
{Anfrage} mit dem Hinweis auf die Verwendung von \hologo{LaTeX} bestellt 
werden. Sobald Sie die notwendigen Archive \File{Univers\_PS.zip} und 
\File{DIN\_Bd\_PS.zip} erhalten haben, können die Schriften für Windows 
(\autoref{sec:install:fonts:win}) beziehungsweise unixoide Betriebssysteme 
(\autoref{sec:install:fonts:unix}) installiert werden. Die benötigten Skripte 
werden als \hrfn{https://github.com/tud-cd/tudscr/releases/tag/fonts}{Release} 
bereitgestellt.%
\footnote{%
  Das Einbinden von installierten Systemschriften im Open-Type-Format mit dem 
  Paket \Package{fontspec} für \hologo{LuaLaTeX} oder \hologo{XeLaTeX} wird 
  mittlerweile zwar unterstützt, empfehlenswert ist diese Variante allerdings 
  nur bedingt. Genaueres dazu in \autoref{sec:fonts:fontspec}.%
}

\Attention{%
  Wurde \TUDScript in einer früheren Ausführung als lokale Nutzerinstallation 
  eingerichtet oder soll es in der Version \vTUDScript{} lokal installiert 
  werden, so sei auf \autoref{sec:install:ext} verwiesen.%
}

\minisec{Anmerkung zu Windows}
Sollte Windows genutzt werden und noch keine \hologo{LaTeX}"=Distribution auf 
ihrem System installiert sein, so rate ich persönlich zur Verwendung von 
\Distribution{\hologo{TeX}~Live} anstelle von \Distribution{\hologo{MiKTeX}}. 
Der Vorteil ist, dass diese Distribution von mehreren Autoren gewartet wird, 
Updates von Paketen und Klassen auf CTAN meist schneller verfügbar sind und 
zusätzlich ein \textsc{Perl}"=Interpreter sowie \textsc{Ghostscript} 
mitgeliefert werden, welche die Ad"=hoc"=Verwendung einiger 
\hologo{LaTeX}"=Pakete wie beispielsweise \Package{glossaries} vereinfacht 
beziehungsweise verbessert.

\minisec{Anmerkung zu Linux und OS~X}
Für die Installationsskripte unter unixartigen Betriebssystemen müssen vor 
deren Ausführung erst die Dateirechte entsprechend gesetzt werden. Dies geht 
beispielsweise über das Terminal. Hat man diese aufgerufen, muss zum Pfad des 
Installationsordners navigiert und der Befehl \Path{chmod +x \PName{Skript}.sh}
ausgeführt werden. Anschließend lässt sich das Installationsskript im Terminal
über \Path{./\PName{Skript}.sh} starten.


\subsection{Installation der Schriften unter Windows}
\label{sec:install:fonts:win}
\index{Installation!Schriften}
Zur Installation der Schriften des \CDs für das \TUDScript-Bundle ist das Archiv
\hrfn{https://github.com/tud-cd/tudscr/releases/download/fonts/TUD-KOMA-Script_fonts_Windows.zip}%
{\File*{TUD-KOMA-Script\_fonts\_Windows.zip}} vorgesehen. Dieses ist sowohl für 
\Distribution{\hologo{TeX}~Live} als auch \Distribution{\hologo{MiKTeX}} 
nutzbar und enthält~-- bis auf die Schriftdateien selbst~-- alle benötigten 
Dateien. Diese sollten nach dem Entpacken des Archivs in das gleiche 
Verzeichnis kopiert werden. Vor der Verwendung des Skripts 
\File*{tudscrfonts\_install.bat} sollte sichergestellt werden, dass sich 
\emph{alle} der folgenden Dateien im selben Verzeichnis befinden:
%
\settowidth{\tempdim}{\File{tudscrfonts\_install.bat}}%
\begin{description}[labelwidth=\tempdim,labelsep=1em]
  \item[\File{tudscrfonts\_install.bat}]Installationsskript
  \item[\File{Univers\_PS.zip}]Archiv mit Schriftdateien für \Univers
  \item[\File{DIN\_Bd\_PS.zip}]Archiv mit Schriftdateien für \DIN
  \item[\File{tudscrfonts.zip}]Archiv mit Metriken für die
    Schriftinstallation via \Package{fontinst}
  \item[\File{7za.exe}]Stand-Alone-Version von 7-zip zum Entpacken der Archive%
    \footnote{%
      Windows stellt keine Bordmittel zum Extrahieren von Archiven auf 
      Kommandozeilen-/Skript-Ebene zur Verfügung.%
    }%
\end{description}
%
Beim Ausführen des Installationsskripts werden alle Schriften in das lokale 
Nutzerverzeichnis der jeweiligen Distribution installiert, falls kein anderes 
Verzeichnis explizit angegeben wird.

\minisec{Anmerkung zu \ifdin{TeX~Live}{\hologo{TeX}~Live}}
Sollte das Installationsskript fehlerfrei durchlaufen, die Schriften dennoch 
nicht verfügbar sein, so kann man mit \Path{updmap-sys -{}-syncwithtrees} 
die Synchronisierung aller Schriftdateien anstoßen. Anschließend sollte man 
noch \Path{updmap-sys -{}-force} aufrufen. Sind die Schriften danach immer noch 
nicht verfügbar, so wurden auf dem System bestimmt schon andere Schriften lokal 
installiert. In diesem Fall sollte man den Vorgang noch einmal für lokale 
Schriften mit \Path{updmap -{}-syncwithtrees} und \Path{updmap -{}-force} 
ausführen. Dies hat jedoch zur Folge, dass der Befehl \Path{updmap-sys} von nun 
an wirkungslos bleibt und auch in Zukunft bei der globalen Installation neuer 
Schriften diese mit \Path{updmap} lokal registriert werden müssen.

\minisec{Anmerkung zu \ifdin{MiKTeX}{\hologo{MiKTeX}}}
Für den Fall, dass \Distribution{\hologo{MiKTeX}} verwendet wird, ist sehr 
ratsam, das \TUDScript-Bundle~-- wie sämtliche manuell installierte Pakete und 
Schriftarten auch~-- nicht in das Standardnutzerverzeichnis oder gar in den
Pfad der \Distribution{\hologo{MiKTeX}}"=Distribution selbst sondern in einen 
separaten Ordner zu installieren und diesen bei \Distribution{\hologo{MiKTeX}} 
als lokalen \Path{texmf}-Baum anzumelden. Das mitgelieferte Installationsskript 
für diesen Schritt automatisiert aus. Am besten ist, den voreingestellten 
Installationspfad zu verwenden.

Es kann vorkommen, dass die für den Schriftinstallationsprozess notwendigen 
Pakete \Package{fontinst}, \Package{cmbright} sowie \Package{iwona} noch nicht 
installiert sind. Ist die automatische Nachinstallation fehlender Pakete 
aktiviert, so reicht es im Normalfall, nach einem ersten vollständigen 
Durchlauf das Skript abermals zu starten. Andernfalls müssen diese Pakete 
manuell über den Paketmanager installiert werden.

Das Installationsskript scheitert außerdem bei einigen Anwendern~-- aufgrund 
eingeschränkter Zugriffsrechte~-- beim Eintragen der Schriften in die Map-Datei.
Dies muss gegebenenfalls durch den Anwender über die Kommandozeile 
\Path{initexmf -{}-edit-config-file updmap} erfolgen. In der sich öffnenden 
Datei sollte sich der Eintrag \Path{Map tudscr.map} befinden. Ist dies nicht 
der Fall, ist diese Zeile manuell einzutragen und die Datei zu speichern. 
Danach muss in der Kommandozeile noch \Path{initexmf -{}-mkmaps} ausgeführt 
werden.


\subsection{Installation der Schriften unter Linux und OS~X}
\label{sec:install:fonts:unix}
\index{Installation!Schriften}
Für die Erstellung des Installationsskripts für Linux und OS~X geht mein Dank 
an Jons-Tobias Wamhoff, welcher sich für die erstmalige Portierung des Skripts 
von Windows zu unixartigen Systemen freiwillig zur Verfügung stellte.
Zur Installation der Schriften des \CDs für das \TUDScript-Bundle ist das Archiv
\hrfn{https://github.com/tud-cd/tudscr/releases/download/fonts/TUD-KOMA-Script_fonts_Unix.zip}%
{\File*{TUD-KOMA-Script\_fonts\_Unix.zip}} vorgesehen. Dieses ist sowohl für 
\Distribution{\hologo{TeX}~Live} als auch \Distribution{Mac\hologo{TeX}} 
nutzbar und enthält~-- bis auf die Schriftdateien selbst~-- alle benötigten 
Dateien. Diese sollten nach dem Entpacken des Archivs in das gleiche 
Verzeichnis kopiert werden. Vor der Verwendung des Skripts 
\File*{tudscrfonts\_install.sh} sollte sichergestellt werden, dass sich 
\emph{alle} der folgenden Dateien im selben Verzeichnis befinden:
%
\settowidth{\tempdim}{\File{tudscrfonts\_install.sh}}%
\begin{description}[labelwidth=\tempdim,labelsep=1em]
  \item[\File{tudscrfonts\_install.sh}]Installationsskript
  \item[\File{Univers\_PS.zip}]Archiv mit Schriftdateien für \Univers
  \item[\File{DIN\_Bd\_PS.zip}]Archiv mit Schriftdateien für \DIN
  \item[\File{tudscrfonts.zip}]Archiv mit Metriken für die
    Schriftinstallation via \Package{fontinst}
\end{description}
%
Beim Ausführen des Installationsskripts werden alle Schriften in das lokale 
Nutzerverzeichnis der jeweiligen Distribution installiert.


\minisec{Probleme bei der Installation der Schriften}
Treten bei der Installation wider Erwarten Probleme auf, so sollte eine 
Logdatei erstellt werden. Unter Windows muss das Skript, welches Probleme 
verursacht, aus der Kommandozeile mit
\Path{\PName{Skript}.bat > \PName{Skript}.log} aufgerufen werden, wobei hierbei 
das doppelte Betätigen der Eingabetaste nötig ist. Für unixartige Systeme 
ist der Aufruf \Path{./\PName{Skript}.sh > \PName{Skript}.log} aus dem Terminal 
heraus zu verwenden. Die erstellte Logdatei kann mit einer kurzen 
Fehlerbeschreibung direkt an \Email{\tudscrmail} gesendet werden.



\section{Schnelleinstieg}
Das Handbuch gliedert sich in drei Teile. In \autoref{part:main} ist für den 
Anwender der Kern des \TUDScript-Bundles zu finden. Hier werden alle neuen 
Optionen, Umgebungen und Befehle, die über die Funktionalität von \KOMAScript{} 
hinausgehen, erläutert. \autoref{part:additional} enthält zum einen einfache 
Minimalbeispiele, um den prinzipiellen Umgang und die Funktionalitäten von 
\TUDScript zu demonstrieren. Zum anderen werden hier auch ausführliche und 
dokumentierte Tutorials vor allem für \hologo{LaTeX}-Neulinge angeboten. 
Insbesondere das Tutorial \Tutorial{treatise} ist mehr als einen Blick wert, 
wenn eine wissenschaftliche Arbeit mit \hologo{LaTeXe} verfasst werden soll.
Abschließend werden verschiedenste Pakete vorgestellt, die nicht speziell für 
das \TUDScript-Bundle selber sondern auch für andere \hologo{LaTeX}-Klassen
verwendet werden können und demzufolge für alle \hologo{LaTeX}-Anwender 
interessant sein könnten. Außerdem werden hier einige Tipps \& Tricks beim 
Umgang mit \hologo{LaTeX} beschrieben, um kleinere oder größere Probleme zu 
lösen.

Die Klassen \Class{tudscrbook}, \Class{tudscrreprt} und \Class{tudscrartcl} 
sind Wrapper"=Klassen der bekannten \KOMAScript-Klassen \Class{scrbook}, 
\Class{scrreprt} sowie \Class{scrartcl} und können einfach anstelle deren 
verwendet werden. Auf diesen basierende Dokumente können durch das Umstellen 
der Dokumentklasse einfach in das \CD der \TnUD überführt werden. Bei 
Fragestellungen bezüglich Layout, Schriften oder ähnlichem ist in jedem Fall 
ein weiterer Blick in das hier vorliegende Handbuch empfehlenswert.
\setpartpreamble{%
  \begin{abstract}
    \hypersetup{linkcolor=red}
    \noindent Dies ist der Hauptteil des \TUDScript-Bundles. Hier findet der 
    Anwender alle verfügbaren Optionen, Umgebungen und Befehle, die über 
    die Funktionalität von \KOMAScript{} hinausgehen.
  \end{abstract}
}
\part{Das \TUDScript-Bundle}\label{part:main}
\chapter[Die Klassen tudscrbook, tudscrreprt und tudscrartcl]{Die Hauptklassen}
\DeclareClass{tudscrbook}\DeclareClass{tudscrreprt}\DeclareClass{tudscrartcl}
\index{Hauptklassen|!}
Es werden die drei neuen Hauptklassen
%
\begin{description}
\item \Class*{tudscrbook}
\item \Class*{tudscrreprt}
\item \Class*{tudscrartcl}
\end{description}
%
eingeführt, welche auf den \KOMAScript"=Klassen basieren und grundsätzlich alle
deren bekannten Optionen, Umgebungen und Befehle~-- beispielsweise
\Option{BCOR} zur Festlegung der Bindekorrektur oder \Option{parskip} zur 
Regelung der Absatzeinstellungen~-- unterstützen. Zusätzlich zu den 
\KOMAScript"=Klassen werden weitere Pakete zwingend benötigt, welche unter 
\autoref{sec:packages:needed} aufgeführt sind und durch den Anwender nicht noch 
zusätzlich geladen werden müssen.

Es sei hier abermals auf die Anwenderdokumentation \scrguide von \KOMAScript{} 
hingewiesen, viele der folgend beschriebenen Befehle und Optionen beziehen sich 
auf die darin vorgestellten Einstellungsmöglichkeiten. Die Anpassungen und 
Erweiterungen der \KOMAScript"=Klassen an das \CD und die neu definierten 
beziehungsweise geänderten Befehle und Optionen werden im Folgenden erläutert.

\begin{Declaration}{\Macro{TUDoptions}\Parameter{Optionenliste}}
\begin{Declaration}{\Macro{TUDoption}\Parameter{Option}\Parameter{Werteliste}}
\printdeclarationlist%
\index{Optionenwahl|!}%
%
Mit diesen Befehlen hat man bei den meisten der neuen Klassenoptionen die 
Möglichkeit, den Wert der Optionen noch nach dem Laden der Klasse zu ändern.
Man kann wahlweise mit der Anweisung \Macro*{TUDoptions} die Werte einer Reihe 
von Optionen ändern. Jede Option der Optionenliste hat dabei die Form
\PName{Option}\PValue{=}\PName{Wert}. Die meisten Optionen besitzen auch einen 
Säumniswert\footnote{engl.: default value}. Versäumt man die Angabe eines 
Wertes~-- verwendet demzufolge einfach die Form \PName{Option}~-- so wird 
automatisch dieser Säumniswert angenommen.

Manche Optionen können gleichzeitig mehrere Werte besitzen. Für diese besteht 
die Möglichkeit, mit \Macro*{TUDoption} der einen Option nacheinander eine 
Reihe von Werten zuzuweisen. Die einzelnen Werte sind dabei in der Werteliste 
durch Komma voneinander getrennt.

Mit diesen beiden Befehlen kann im Bedarfsfall das Verhalten von einer Option 
oder mehreren Optionen im Dokument geändert werden. Werden diese Befehle in 
einer Umgebung oder einer Gruppe verwendet, bleiben die gemachten Einstellungen 
innerhalb dieser lokal begrenzt.
\end{Declaration}
\end{Declaration}


\section{Die Schriften des \CDs}
\label{sec:fonts}
\index{Schrift|?}
%
Das \CD der \TnUD gibt die Verwendung der Schriften \Univers für den Fließtext 
sowie \DIN für das Setzen von Überschriften vor. Im Standardfall wird dies so 
unterstützt. Da jedoch in längeren Texten die Verwendung von Serifenschriften 
zu empfehlen ist, gibt es die Möglichkeit, die ursprünglich vorgesehenen 
Schriften nicht zu laden und die Standardschriften beziehungsweise ein anderes 
Schriftpaket zu verwenden. Die Einstellungen und Befehle für den Fließtext sind 
in \autoref{sec:text} zu finden.

Durch das \CD werden keine Schriften für den Mathematiksatz bereitgestellt. 
Dies ist insbesondere für sowohl mathematische als auch ingenieur- und 
naturwissenschaftliche Dokumente nicht tragbar. Dieser Mangel wird behoben, 
indem im Mathematikmodus die lateinischen Buchstaben der Hausschriften mit 
griechischen Lettern und mathematischen Symbolen aus anderen Paketen ergänzt 
werden.%
\footnote{%
  \Package{iwona} für die Schrift \DIN und zusätzlich \Package{cmbright} für 
  die \Univers"=Schriftfamilie%
}
Diese Einstellungen kann natürlich ebenfalls mit der entsprechenden Option 
deaktiviert werden. Dann werden die Standardschriften oder gegebenenfalls die 
eines zusätzlichen Paketes für den mathematischen Satz genutzt. Alle Befehle 
und Optionen für den Mathematiksatz sind in \autoref{sec:math} erläutert. 
Weitere Hinweise zum typographisch guten Mathematiksatz sind außerdem in 
\autoref{sec:exmpl:mathswap} sowie \autoref{sec:exmpl:mathtype} zu finden.


\subsection{Schriften für den Textsatz}
\begin{Declaration}[%
  v2.02!Werte für die Optionen \Option*{barfont} und \Option*{fontspec} ergänzt%
]{\Option{cdfont}[\PSet]}[true]%
\printdeclarationlist%
\label{sec:text}%
\index{Schrift}\index{Schrift!Fließtext}
\index{Schrift!Corporate Design}\index{Schrift!Stärke}%
%
Mit dieser Option können durch den Benutzer alle zentralen Schrifteinstellungen 
für die \TUDScript-Klassen vorgenommen werden. Dies betrifft die Schriften für 
Überschriften, den Text im Querbalken der Kopfzeile sowie den Fließtext und die 
Mathematikschriften.
%
\begin{values}
\itemfalse
  Es werden keine Hausschriften sondern die \hologo{LaTeX}"=Standardschriften 
  verwendet und der Benutzer kann beliebige Schriftpakete nutzen.%
  \footnote{%
    Für die Verwendung der klassischen \hologo{LaTeX}"=Schriften, ist das Paket 
    \Package{lmodern} sehr empfehlenswert.%
  }
  Sollte das Layout des \CDs aktiviert sein (siehe \Option{cd}), werden die 
  Überschriften in serifenlosen Großbuchstaben gesetzt.
\itemtrue*[light/lightfont/noheavyfont]
  Es werden die Hausschriften im Stil des \CDs der \TnUD genutzt. Überschriften 
  der obersten Gliederungsebenen bis einschließlich \Macro*{subsubsection} 
  verwenden \DIN, darunter liegende%
  \footnote{\Macro*{paragraph} und \Macro*{subparagraph}} 
  \textubn{Univers~65~Bold}. Für den Fließtext im Dokument kommt 
  \textuln{Univers~45~Light} zum Einsatz. Aus \Package{lmodern} wird die
  \texttt{Schreibmaschinenschrift} verwendet.
\item[heavy/heavyfont]
  Die Schriftstärke der Hausschriften wird erhöht. Die beiden untersten 
  Gliederungsebenen werden in \textuxn{Univers~75~Black} gesetzt, der Fließtext 
  in \texturn{Univers~55~Regular}. Ansonsten entspricht alles der Option 
  \Option*{cdfont}[true]. Die Mathematikschriften werden durch diese 
  Einstellung nicht beeinflusst. Gegebenenfalls sollte mit \Macro{boldmath} auf 
  den fetten Schnitt umgeschaltet werden.
\item[nodin]
  Für die Überschriften wird nicht \DIN verwendet. Ist \Option*{cdfont}[true] 
  gewählt, wird \Univers genutzt. Die Schriftstärke ist dabei abhängig von der 
  Einstellung \Option*{cdfont}[light/heavy]. Ist die Verwendung der Schriften 
  des \CDs deaktiviert (\Option*{cdfont}[false]), kommt die fette Schriftstärke 
  der eingestellten serifenlosen Schriftfamilie zum Einsatz.
\item[din]
  Mit dieser Einstellung wird die Schrift \DIN in den Überschriften verwendet. 
  Sie ist standardmäßig aktiviert.
\end{values}
%
\ChangedAt{v2.02}
Für den Text im Querbalken gibt es folgende Einstellmöglichkeiten:
%
\begin{values}
\item[barfont]
  Für den Querbalken der Kopfzeile wird unabhängig von der Verwendung der 
  Hausschriften die Schrift \Univers in normaler Schriftstärke verwendet,
  siehe \Option{barfont}[true].
\item[heavybarfont]
  Die im Querbalken der Kopfzeile verwendete Stärke der Schrift \Univers wird 
  erhöht, siehe \Option{barfont}[heavy].
\end{values}
%
Die verwendeten Mathematikschriften lassen sich mit folgenden Werte 
beeinflussen:
%
\begin{values}
\item[serifmath/serif/nosansmath/nosans]  
  Diese Einstellung deaktiviert die Verwendung von serifenlosen Schriften für 
  den mathematischen Satz. Es werden die \hologo{LaTeX}"=Standardschriften 
  verwendet und der Benutzer kann beliebige Schriftpakete für den 
  Mathematikmodus nutzen, siehe \Option{sansmath}[false].
\item[sansmath/sans]
  Es werden serifenlose Mathematikschriften für lateinische und griechische 
  Lettern genutzt, siehe \Option{sansmath}[true].
\item[upgreek/uprightgreek/uprightGreek]
  Die großen griechischen Buchstaben werden im Mathematikmodus aufrecht gesetzt,
  siehe \Option{slantedgreek}[false].
\item[slgreek/slantedgreek/slantedGreek]
  In mathematischen Umgebungen erfolgt die Ausgabe der griechischen Majuskeln 
  kursiv, siehe \Option{slantedgreek}[true].
\end{values}
%
\ChangedAt{v2.02}
Außerdem kann mit folgenden Werten das verwendete Schriftformat eingestellt 
werden: 
%
\begin{values}
\item[fontspec/lualatex/xelatex]
  Es wird das Paket \Package{fontspec} geladen und die Schriften des \CDs im 
  OpenType"=Format verwendet. Hierfür muss entweder \hologo{LuaLaTeX} oder 
  \hologo{XeLaTeX} als Dokumentprozessor genutzt werden. Wird diese Einstellung 
  aktiviert, sind die Hinweise zur Option \Option'{fontspec} unbedingt zu 
  beachten.
\item[nofontspec/pdflatex]
  Es werden die Schriften des \CDs im PostScript"=Format verwendet, wenn diese 
  wie unter \autoref{sec:install} beschrieben installiert wurden. Diese 
  Einstellung ist standardmäßig aktiviert und sollte nur in Ausnahmefällen 
  geändert werden.
\end{values}
\end{Declaration}

\subsubsection{Auszeichnungen in Überschriften}
\begin{Declaration}[v2.02]{\Option{footnotes}[\PSet]}[nosymbolheadings]%
\begin{Declaration}[v2.02]{\Counter{symbolheadings}}%
\printdeclarationlist%
\index{Überschriften}\index{Überschriften!Fußnoten}\index{Fußnoten}%
%
Für die Überschriften wird die \KOMAScript-Option \Option*{footnotes} erweitert.
Normalerweise kann diese die Werte \PValue{multiple} und \PValue{nomultiple} 
annehmen, wobei Letzteres der Standardfall ist. Die \TUDScript-Hauptklassen 
erweitern die Option dahingehend, dass auf die Verwendung von Symbolen anstelle 
von Zahlen innerhalb der Überschriften umgeschaltet werden kann. Hierfür wird 
der Zähler \Counter*{symbolheadings} definiert, der mit dem Beginn eines neuen 
Kapitels zurückgesetzt wird.
%
\begin{values}
\item[nosymbolheadings/numberheadings]
  Die Fußnoten der Überschriften werden fortlaufend mit denen des Fließtextes 
  gesetzt.
\item[symbolheadings]
  Für die Überschriften werden symbolische Fußnoten mit einem eigenen Zähler 
  verwendet.
\end{values}
\end{Declaration}
\end{Declaration}

\begin{Declaration}{\Macro{ifdin}\Parameter{Dann-Teil}\Parameter{Sonst-Teil}}%
\printdeclarationlist%
\index{Überschriften}\index{Schrift!Überschriften}\index{Schriftauszeichnung}%
\index{Kolumnentitel}\index{Layout!Kolumnentitel}
%
Der Befehl \Macro*{ifdin} prüft, ob die Schriftfamilie \DIN aktiv ist und führt 
in diesem Fall \Parameter{Dann-Teil} aus, andernfalls \Parameter{Sonst-Teil}. 
Dies ist beispielsweise bei Überschriften sinnvoll, wenn zwischen der Ausgabe 
im Fließtext und dem Eintrag für das Inhaltsverzeichnis sowie der Ausprägung 
der automatischen Kolumnentitel unterschieden werden soll.
\end{Declaration}

\begin{Declaration}{\Macro{MakeTextUppercase}\Parameter{Text}}%
\begin{Declaration}{\Macro{NoCaseChange}\Parameter{Text}}%
\printdeclarationlist%
\index{Überschriften}\index{Schrift!Überschriften}\index{Schriftauszeichnung}%
%
Diese beiden Befehle stammen aus dem Paket \Package{textcase}. Der Befehl 
\Macro*{MakeTextUppercase} setzt den Text seines Argumentes in Majuskeln. Die 
Überschriften der Gliederungsebenen bis einschließlich \Macro*{subsubsection} 
werden damit in Großbuchstaben der Schrift \DIN gesetzt. Sollen bestimmte 
Kleinbuchstaben erhalten bleiben, ist der Befehl \Macro{NoCaseChange} zu nutzen.
\end{Declaration}
\end{Declaration}
%
\begin{Example}
In einer Kapitelüberschrift wird ein einzelnes Wort in Kleinbuchstaben 
geschrieben:
\begin{Code}[escapechar=§]
\chapter{§Ü§berschrift mit \NoCaseChange{kleinem} Wort}
\end{Code}
\end{Example}

\subsubsection{Auszeichnungen im Text}
\begin{Declaration}[v2.02]{\Font{titlepage}}
\begin{Declaration}[v2.02]{\Font{thesis}}
\begin{Declaration}[v2.02]{\Font{parttitle}}
\printdeclarationlist%
\index{Schriftelemente}
%
Die \TUDScript-Klassen definieren mit \Macro{newkomafont} diese neuen 
Schriftelemente. Dabei wird \Font*{titlepage} auf der Titelseite für alle 
Felder verwendet, welche kein spezielles Schriftelement verwenden und 
\Font*{thesis} für das mit \Macro{thesis} angegebene Feld, in welchem der Typ 
einer Abschlussarbeit angegeben wird (siehe \autoref{sec:title}). Mit 
\Font*{parttitle} kann die Schrift für die Bezeichnung des Teils bei 
aktivierter \Option{parttitle}-Option beeinflusst werden. Alle Schriften lassen 
sich mit \Macro{addtokomafont}\Parameter{Schriftelement} anpassen.
\end{Declaration}
\end{Declaration}
\end{Declaration}

\begin{Declaration}{\Macro{univln}}
\begin{Declaration}{\Macro{textuln}\Parameter{Text}}
\begin{Declaration}{\Macro{univrn}}
\begin{Declaration}{\Macro{texturn}\Parameter{Text}}
\begin{Declaration}{\Macro{univbn}}
\begin{Declaration}{\Macro{textubn}\Parameter{Text}}
\begin{Declaration}{\Macro{univxn}}
\begin{Declaration}{\Macro{textuxn}\Parameter{Text}}
\begin{Declaration}{\Macro{univls}}
\begin{Declaration}{\Macro{textuls}\Parameter{Text}}
\begin{Declaration}{\Macro{univrs}}
\begin{Declaration}{\Macro{texturs}\Parameter{Text}}
\begin{Declaration}{\Macro{univbs}}
\begin{Declaration}{\Macro{textubs}\Parameter{Text}}
\begin{Declaration}{\Macro{univxs}}
\begin{Declaration}{\Macro{textuxs}\Parameter{Text}}
\begin{Declaration}{\Macro{dinbn}}
\begin{Declaration}{\Macro{textdbn}\Parameter{Text}}
\settowidth{\tempdim}{\Macro{textuln}\Parameter{Text}}%
\addtolength{\tempdim}{\dimexpr 2\tabcolsep+2\arrayrulewidth-\textwidth}%
\printdeclarationlist(%
  \begin{minipage}{-\tempdim}%
  \centering%
  \begin{tabularm}{3}%
    \toprule%
    \textbf{Schriftart}                  & \textbf{Schalter}
      & \textbf{Textkommando}\tabularnewline
    \midrule
    \textuln{Univers 45 Light}           & \Macro*{univln}{}
      & \Macro*{textuln}\Parameter{Text}\tabularnewline
    \texturn{Univers 55 Regular}         & \Macro*{univrn}{}
      & \Macro*{texturn}\Parameter{Text}\tabularnewline
    \textubn{Univers 65 Bold}            & \Macro*{univbn}{}
      & \Macro*{textubn}\Parameter{Text}\tabularnewline
    \textuxn{Univers 75 Black}           & \Macro*{univxn}{}
      & \Macro*{textuxn}\Parameter{Text}\tabularnewline
    \textuls{Univers 45 Light Oblique}   & \Macro*{univls}{}
      & \Macro*{textuls}\Parameter{Text}\tabularnewline
    \texturs{Univers 55 Regular Oblique} & \Macro*{univrs}{}
      & \Macro*{texturs}\Parameter{Text}\tabularnewline
    \textubs{Univers 65 Bold Oblique}    & \Macro*{univbs}{}
      & \Macro*{textubs}\Parameter{Text}\tabularnewline
    \textuxs{Univers 75 Black Oblique}   & \Macro*{univxs}{}
      & \Macro*{textuxs}\Parameter{Text}\tabularnewline
    \DIN & \Macro*{dinbn}{}
      & \Macro*{textdbn}\Parameter{Text}\tabularnewline
    \bottomrule%
    \allcolumnpar{\footnotesize\vskip0pt%
       Die Schrift \DIN darf laut \CD nur mit Majuskeln (Großbuchstaben) 
       verwendet werden. Wird diese Schrift manuell verwendet, sollte dies mit 
       \Macro{MakeTextUppercase}\PParameter{\Macro{textdbn}\Parameter{Text}}  
       geschehen. Sollen dabei im Argument einzelne Teile zwingend klein 
       geschrieben werden, wird der Befehl \Macro{NoCaseChange} benötigt.
    }
  \end{tabularm}%
  \end{minipage}%
)%
\index{Schrift!Befehle}\index{Schrift!Schalter}%
%
Unabhängig davon, welche Schriftfamilie verwendet wird, können die Schriften 
des \CDs jederzeit entweder mit einem Textschalter oder mit einem Textkommando
innerhalb des Dokumentes genutzt werden. Ein Textschalter wirkt sich~-- wenn er 
nicht in einer Gruppe oder einer Umgebung verwendet und damit lokal begrenzt 
wird~-- global auf das Dokument aus. Bei einem Textkommando hingegen erfolgt 
die Änderung der Schriftart nur für das angegebene Argument. Deshalb ist die 
Verwendung der letzteren Variante vorzuziehen.
\end{Declaration}
\end{Declaration}
\end{Declaration}
\end{Declaration}
\end{Declaration}
\end{Declaration}
\end{Declaration}
\end{Declaration}
\end{Declaration}
\end{Declaration}
\end{Declaration}
\end{Declaration}
\end{Declaration}
\end{Declaration}
\end{Declaration}
\end{Declaration}
\end{Declaration}
\end{Declaration}

\subsection{Schriften für den Mathematiksatz}
\begin{Declaration}{\Option{sansmath}[\PBoolean]}%
  [true][\Option{cdfont}[false]:false]
\printdeclarationlist%
\label{sec:math}
\index{Schrift!Mathematiksatz}\index{Mathematiksatz|!}
\index{Schrift!Griechische Buchstaben}\index{Griechische Buchstaben}
%
Diese Option dient zur Verwendung serifenloser Mathematikschriften. Dafür 
werden zum einen die griechischen Buchstaben aus \Package{cmbright} und zum 
anderen die Symbole aus \Package{iwona} verwendet. Für die lateinischen 
Buchstaben wird \Univers genutzt. Ein Umschalten auf Serifenlose und zurück 
innerhalb des Dokumentes ist~-- beispielsweise in einer Abbildung oder in einer 
Tabelle~-- durch \Macro{TUDoptions}\PParameter{\Option*{sansmath}[true]} und 
\Macro{TUDoptions}\PParameter{\Option*{sansmath}[false]} möglich. Mit
\Macro{boldmath} kann auf fette Mathematikschriften umgeschaltet werden.

Mit der Einstellung \Option*{sansmath}[false] wird auf die Standardschriften
für den Mathematikmodus zurückgeschaltet. Sollen stattdessen andere serifenlose 
Mathematikschriften genutzt werden, so sei auf \Package{sansmath}, 
\Package{sansmathfonts}, \Package{mathastext}, \Package{sfmath} sowie 
\Package{sansmathaccent} verwiesen.
%
\begin{values}
\itemfalse
  Es werden die normalen \hologo{LaTeX}"=Serifenschriften beziehungsweise die 
  Schriften beliebig nutzbarer Pakete für den Mathematiksatz verwendet.
\itemtrue*
  Die serifenlose Mathematikschriften werden aktiviert.
\end{values}
\end{Declaration}

\subsubsection{Griechischen Buchstaben}
\label{sec:greek}
\index{Griechische Buchstaben}%\index{Griechische Buchstaben!Neigung}
%
\begin{Declaration}{\Macro{varDelta}}
\begin{Declaration}{\Macro{varTheta}}
\begin{Declaration}{\Macro{varLambda}}
\begin{Declaration}{\Macro{varXi}}
\begin{Declaration}{\Macro{varPi}}
\begin{Declaration}{\Macro{varSigma}}
\begin{Declaration}{\Macro{varUpsilon}}
\begin{Declaration}{\Macro{varPhi}}
\begin{Declaration}{\Macro{varPsi}}
\begin{Declaration}{\Macro{varOmega}}
\begin{Declaration}{\Macro{upDelta}}
\begin{Declaration}{\Macro{upTheta}}
\begin{Declaration}{\Macro{upLambda}}
\begin{Declaration}{\Macro{upXi}}
\begin{Declaration}{\Macro{upPi}}
\begin{Declaration}{\Macro{upSigma}}
\begin{Declaration}{\Macro{upUpsilon}}
\begin{Declaration}{\Macro{upPhi}}
\begin{Declaration}{\Macro{upPsi}}
\begin{Declaration}{\Macro{upOmega}}
\index{Schrift!Griechische Buchstaben}\index{Griechische Buchstaben}%
\settowidth{\tempdim}{\Macro{varUpsilon}}%
\addtolength{\tempdim}{\dimexpr 2\tabcolsep+2\arrayrulewidth-\textwidth}%
\printdeclarationlist(%
  \begin{minipage}{-\tempdim}%
    \newcommand\tablecontent{}%
    \newcommand*\greekLetters{%
      Delta,Theta,Lambda,Xi,Pi,Sigma,Upsilon,Phi,Psi,Omega%
    }%
    \def\do#1{\appto\tablecontent{%
      \Macro*{var#1} & $\csuse{var#1}$ & & 
      \Macro*{up#1} & $\csuse{up#1}$\tabularnewline
    }}%
    \expandafter\docsvlist\expandafter{\greekLetters}%
    \centering%
    \vspace{\intextsep}\noindent
    \begin{tabularm}{5}
      \toprule%
      \textbf{Befehl (kursiv)} & \textbf{Symbol} & &
      \textbf{Befehl (aufrecht)} & \textbf{Symbol}
      \tabularnewline\midrule\tablecontent\bottomrule%
      \allcolumnpar{\footnotesize\vskip0pt%
        Die Befehle \Macro*{up}\PName{Name} und \Macro*{var}\PName{Name}
        werden normalerweise durch einige Pakete, unter anderem auch von 
        \Package{cmbright} oder \Package{amsmath}, bereitgestellt.
      }
    \end{tabularm}
  \end{minipage}%
)%
%
Griechische Majuskeln werden sowohl in aufrechter als auch in geneigter Form 
bereitgestellt. Unabhängig von den beiden Optionen \Option{sansmath} und 
\Option{slantedgreek} können sowohl kursive als auch aufrechte griechischen 
Großbuchstaben im Mathematikmodus direkt verwendet werden. Dies ist nützlich, 
um zwischen kursiven Variablen und aufrechten Konstanten zu unterscheiden. Die 
griechischen Minuskeln sind leider nur in der kursiven Variante verfügbar.
\end{Declaration}
\end{Declaration}
\end{Declaration}
\end{Declaration}
\end{Declaration}
\end{Declaration}
\end{Declaration}
\end{Declaration}
\end{Declaration}
\end{Declaration}
\end{Declaration}
\end{Declaration}
\end{Declaration}
\end{Declaration}
\end{Declaration}
\end{Declaration}
\end{Declaration}
\end{Declaration}
\end{Declaration}
\end{Declaration}

\begin{Declaration}{\Option{slantedgreek}[\PBoolean]}%
  [true][\Option{cdfont}[false]:false]
\printdeclarationlist%
\index{Schrift!Griechische Buchstaben}\index{Griechische Buchstaben}%
\index{Griechische Buchstaben!Neigung}%
%
Die Option ändert die standardmäßige Neigung der griechischen Großbuchstaben im 
Mathematikmodus bei der Verwendung der Befehle \Macro*{Delta}, \Macro*{Theta}, 
\Macro*{Lambda}, \Macro*{Xi}, \Macro*{Pi}, \Macro*{Sigma}, \Macro*{Upsilon}, 
\Macro*{Phi}, \Macro*{Psi} und \Macro*{Omega}. Wie unabhängig von der Option 
\Option*{slantedgreek} gezielt kursive und aufrechte Buchstaben gesetzt werden 
können, ist \vpageref{sec:greek} beschrieben.
%
\begin{values}
\itemfalse
  Die griechischen Majuskeln werden wie bei den Standardklassen aufrecht 
  gesetzt.
\itemtrue*
  Die Ausgabe der griechischen Großbuchstaben erfolgt kursiv.
\end{values}
\end{Declaration}

\subsubsection{Zusätzliche Hinweise zum Mathematiksatz}
Weitere Hinweise zum typographisch guten Mathematiksatz sind außerdem in 
\autoref{sec:exmpl:mathswap} sowie \autoref{sec:exmpl:mathtype} zu finden.


\subsection{Die Schriften des \CDs im OpenType-Format}
\label{sec:fonts:fontspec}
\index{OpenType-Schriften}
%
\ChangedAt{v2.02!OpenType-Schriften mit \Option*{fontspec} verwendbar}
Das \TUDScript-Bundle unterstützt die Verwendung der Schriften des \CDs sowohl 
im PostScript- als auch im OpenType"=Format. Sind die OpenType"=Schriften über 
das Betriebssystem installiert, können diese mit dem Paket \Package{fontspec} 
eingebunden werden. Wäre dies ohne Probleme möglich, wäre die Installation der 
PostScript"=Schriften mithilfe eines Skriptes damit obsolet. Allerdings sind 
einerseits für die Kompilierung eines Dokumentes über den klassischen Prozess 
via \Path{latex \textrightarrow{} dvips \textrightarrow{} ps2pdf}~-- wie es 
beispielsweise für die Erstellung von Grafiken mit \Package{pstricks} notwendig 
ist~-- die Schriften im PostScript"=Format nötig. Andererseits liefern die 
Schriftfamilien des \CDs keinerlei mathematische Glyphen, sodass diese bei der 
PostScript"=Schriftinstallation aus den Schriftpaketen \Package{cmbright} und 
\Package{iwona} entnommen werden müssen.

Die Verwendung der Schriften des \CDs im OpenType"=Format sollte folglich nur 
genutzt werden, wenn eine Installation der PostScript"=Schriften \emph{absolut} 
nicht möglich ist. Sollten die PostScript"=Schriften installiert sein, gibt es 
auch beim Einsatz von \hologo{LuaLaTeX} oder \hologo{XeLaTeX} keinen triftigen 
Grund, die Option \Option{fontspec} zu verwenden.

\begin{Declaration}[v2.02]{\Option{fontspec}[\PBoolean]}[false]%
\printdeclarationlist%
%
Wird die Option aktiviert, werden die OpenType"=Varianten von \Univers und \DIN 
anstelle der PostScript"=Schriften verwendet. Diese sollte nur in absoluten
Ausnahmefällen genutzt werden. Hierfür müssen die OpenType"=Schriften auf dem 
Betriebssystem installiert sein.
\begin{values}
\itemfalse*
  Die Hausschriften im Stil des \CDs der \TnUD werden im PostScript"=Format 
  eingebunden. Sowohl Kerning als auch der mathematische Satz funktionieren 
  problemlos.
\itemtrue*
   Es werden die OpenType"=Varianten der Hausschriften verwendet. Dazu wird das 
   Paket \Package{fontspec} geladen, welches lediglich mit \hologo{LuaLaTeX} 
   oder \hologo{XeLaTeX} jedoch nicht mit \hologo{pdfLaTeX} als genutzt werden 
   kann. Sowohl beim mathematischen Satz als auch beim Kerning der Schriften 
   kann es zu Problemen kommen. Die Verwendung dieser Einstellung sollte nur 
   erfolgen, wenn eine Installation der PostScript"=Schriften nicht möglich ist.
\end{values}
\end{Declaration}


\section{Das Layout des \CDs}
Das Hauptaugenmerk der neuen Klassen liegt auf der Umsetzung des \CDs der
\TnUD für \hologo{LaTeX}. Ein großer Teil der definierten Optionen und Befehle
dient genau dazu und wird folgend beschrieben.

Einige spezielle Seiten werden im prägnanten Stil mit dem Logo der \TnUD und 
der dazugehörigen Kopfzeile mit Querbalken gesetzt. Dies betrifft insbesondere 
\hyperref[sec:title]{die Umschlagseite und den Titel in \autoref{sec:title}}, 
die \hyperref[sec:part]{Teileseiten in \autoref{sec:part}} sowie die
\hyperref[sec:chapter]{Kapitelseiten in \autoref{sec:chapter}}. 
Außerdem können entweder mit den \PageStyle*{tudheadings}"=Seitenstilen oder 
mit der \Environment*{tudpage}-Umgebung aus \autoref{sec:tudheadings} eigene 
Seiten im selben Stil erzeugt werden. Wird das Paket \Package{tudscrsupervisor} 
verwendet und mit den entsprechenden Befehlen oder Umgebungen aus diesem eine 
Aufgabenstellung, ein Gutachten oder ein Aushang erstellt, so erscheinen auch 
diese in besagtem Seitenstil.


\subsection{Das Erscheinungsbild von Titel, Teilen und Kapiteln}
\begin{Declaration}{\Option{cd}[\PSet]}[true]
\printdeclarationlist%
\index{Layout}%
%
Diese Option bestimmt, ob und wie das \CD der \TnUD verwendet wird. Sie hat
Einfluss auf die Ausprägung für Titel"~, Teil"~, und Kapitelseiten.
%
\begin{values}
\itemfalse
  Diese Einstellung erzeugt das Standard"=Verhalten der \KOMAScript"=Klassen, 
  es wird kein \CD genutzt.
\itemtrue*[standard/simple/monochrom]
  Das Layout für Titel"~, Teil"~ und Kapitelseiten ist im \CD, es wird 
  schwarze Schrift für Titel, Teil"~ und Kapitelüberschriften sowie im 
  Seitenkopf verwendet.
\item[lite/light/pale]
  Die Einstellung entspricht weitestgehend der Option \Option*{cd}[true], 
  allerdings wird die primäre Hausfarbe \Color{HKS41} anstelle schwarzer 
  Schrift genutzt.
\item[color/colour/full]
  Der Titel sowie Teil"~ und Kapitelseiten werden allesamt farbig und im \CD 
  gestaltet, der Seitenkopf wird in der primären Hausfarbe \Color{HKS41} 
  gesetzt.
\end{values}
\end{Declaration}

\begin{Declaration}[v2.02]{\Length{pageheadingsvskip}}
\begin{Declaration}[v2.02]{\Length{headingsvskip}}
\printdeclarationlist%
\index{Kapitelseiten}\index{Layout!Kapitelseiten}\index{Überschriften!Position}%
%
Diese beiden Längen haben Auswirkung auf die vertikale Position verschiedener
Überschriften. Mit \Length*{pageheadingsvskip} können sowohl der Titel auf der 
Titelseite (\Option{titlepage}[true]) als auch die Überschriften von Teilen und 
Kapiteln, welche als einzelne Kapitelseite (\Option{chapterpage}[true]) gesetzt 
werden, verschoben werden. Demgegenüber erlaubt es \Length*{headingsvskip}, 
sowohl den Titel innerhalb eines Titelkopfes (\Option{titlepage}[false]) als 
auch die Überschrift eines Kapitels bei deaktivierter Kapitelseite 
(\Option{chapterpage}[false]) in ihrer vertikalen Position anzupassen.

Normalerweise werden alle der zuvor genannten Überschriften im Layout relativ 
tief im Textbereich gesetzt. Mit negativen Werten wird diese nach oben 
verschoben, positive Werte setzen diese dementsprechend tiefer. Beim 
Verschieben nach oben, sollte darauf geachtet werden, dass diese sich danach 
noch innerhalb des Satzspiegels befinden, da dies \emph{nicht} automatisch 
durch die Hauptklassen überprüft wird.
\end{Declaration}
\end{Declaration}

\subsubsection{Einstellungen für Titel, Umschlagseite, Teile und Kapitel}
Das Verhalten aller Elemente%
\footnote{%
  Titel (\Macro{maketitle}), Umschlagseite (\Macro{makecover}),
  Teileseite (\Macro{part}, \Macro{addpart}),
  Kapitelseite (\Macro{chapter}, \Macro{addchap})%
}
wird normalerweise von der Option \Option{cd}[\PSet] bestimmt. Bedarfsweise 
können einzelne Elemente andererseits auch individuell mit abweichenden 
Wertzuweisungen angepasst werden. Soll ein bestimmtes Elemente des Layouts 
anders erscheinen als der Rest des Dokumentes, so kann der entsprechende Wert 
mithilfe der folgenden Optionen überschrieben werden. Die gültigen 
Wertzuweisungen für die einzelnen Elemente entsprechend dabei den möglichen 
Werten für die Option \Option{cd}.

\begin{Declaration}{\Option{cdtitle}[\PSet]}
\printdeclarationlist%
\index{Titel}\index{Layout!Titel}%
%
Mit \Option*{cdtitle} kann der Wert des Schlüssels \Option{cd} für die 
Titelseite überschrieben werden. Es kann zwischen dem Standardtitel~-- welcher 
durch \KOMAScript{} bereitgestellt wird~-- und dem Titel im \CD umgeschaltet 
werden. Die neue Titelseite unterstützt alle durch \KOMAScript{} definierten 
Befehle für den Titel.%
\footnote{\raggedright%
  \Macro{extratitle}\Parameter{Schmutztitel},\Macro{titlehead}\Parameter{Kopf},
  \Macro{subject}\Parameter{Typisierung},\Macro{title}\Parameter{Titel},
  \Macro{subtitle}\Parameter{Untertitel},\Macro{author}\Parameter{Autor},
  \Macro{date}\Parameter{Datum},\Macro{publishers}\Parameter{Verlag},
  \Macro{and} und \Macro{thanks}\Parameter{Fußnote} sowie
  \Macro{uppertitleback}\Parameter{Titelrückseitenkopf},
  \Macro{lowertitleback}\Parameter{Titelrückseitenfuß}
  und \Macro{dedication}\Parameter{Widmung}
}
Zusätzlich werden viele neue Felder für den Titel definiert, welche vor allem 
für den Titel einer wissenschaftlichen Arbeit von Relevanz sind. Genaueres dazu 
ist in \autoref{sec:title} nachzulesen. Unabhängig von der gewählten Variante 
der Titelseite wird diese immer mit \Macro{maketitle} erzeugt.
\end{Declaration}

\begin{Declaration}[v2.02]{\Option{cdcover}[\PSet]}
\printdeclarationlist%
\index{Umschlagseite|!}%
\index{Titel!Umschlagseite}\index{Layout!Umschlagseite}%
%
Die \TUDScript-Klassen führen zusätzlich den Befehl \Macro{makecover} ein, mit 
dem sich neben dem Titel eine separate Umschlagseite erzeugen lässt. Diese ist 
in ihrer Gestalt der Titelseite sehr ähnlich, wird normalerweise jedoch in 
einem anderen Satzspiegel als dem des Buchblocks gesetzt. Mit der Option 
\Option*{cdcover} kann~-- unabhängig von \Option{cd}~-- das Erscheinungsbild 
der Umschlagseite geändert werden. Wird \Option*{cdcover}[false] gewählt, 
entspricht die Umschlagseite dem originalen \KOMAScript-Titel. Die Verwendung 
des Befehls \Macro{makecover} sowie die dazugehörigen Parameter werden 
detailliert in \autoref{sec:title} erläutert.
\end{Declaration}

\begin{Declaration}{\Option{cdpart}[\PSet]}
\printdeclarationlist%
\index{Teileseiten}\index{Layout!Teileseiten}%
%
Für die Teileseiten kann der Wert des Schlüssels \Option{cd} separat 
überschrieben und somit deren Layout%
\footnote{\label{fn:layout}%
  \KOMAScript"=Layout beziehungsweise monochromes oder farbiges 
  Erscheinungsbild im \CD%
}
beeinflusst werden, welches bei der Benutzung der Befehle \Macro{part} 
beziehungsweise \Macro{addpart} und deren Sternversionen genutzt wird.
\end{Declaration}

\begin{Declaration}{\Option{cdchapter}[\PSet]}
\printdeclarationlist%
\index{Kapitelseiten}\index{Layout!Kapitelseiten}%
%
Für Kapitelseiten kann der Schlüsselwert \Option{cd} ebenfalls angepasst und 
damit das Erscheinungsbild\footref{fn:layout} geändert werden, das bei der 
Verwendung von \Macro{chapter} beziehungsweise \Macro{addchap} und den 
dazugehörigen Sternversionen genutzt wird.
\end{Declaration}
%
\begin{Example}
Soll die Titelseite in Farbe, der Rest des Dokumentes allerdings in schwarzer 
Schrift gesetzt werden, so kann dies folgendermaßen erreicht werden:
\begin{Code}[escapechar=§]
\documentclass[cd=true,cdtitle=color]{§\PName{Dokumentklasse}§}
\end{Code}
\end{Example}

\subsubsection{Vakatseiten/Leerseiten}
\index{Leerseiten}%
Automatisch erzeugte Vakatseiten~-- auch absichtliche Leerseiten genannt~-- 
findet man in Dokumenten mit den aktivierten Optionen \Option{twoside} und 
\Option{open}[right]\footnote{Standard bei \Class{tudscrbook}} beziehungsweise 
\Option{open}[left] beim Beginn von Teilen und Kapiteln. Für diese kann der 
Seitenstil mit der \KOMAScript"=Option \Option{cleardoublepage} eingestellt 
werden.

\begin{Declaration}{\Option{cleardoublespecialpage}[\PSet]}[true]%
\printdeclarationlist%
\index{Teileseiten}\index{Layout!Teileseiten}%
\index{Kapitelseiten}\index{Layout!Kapitelseiten}%
\index{Satzspiegel!doppelseitig}\index{Layout!Rückseiten}%
%
Diese Option wirkt sich lediglich bei aktiviertem doppelseitigem Satz und 
ausschließlich rechts eröffnenden Seiten für Teile beziehungsweise Kapitel
aus.%
\footnote{\Option{twoside} und \Option{open}[right]}
In diesem Fall kann der Stil der darauffolgenden, linken Seite~-- sprich der 
Rückseite~-- beeinflusst werden. Das Normalverhalten sieht vor, dass nach einem 
Teil die Rückseite unabhängig von der Einstellung für \Option{cleardoublepage} 
immer als vollständig leere Seite ohne Kopf"~ oder Fußzeilen gesetzt wird.

Diese Einstellung erlaubt es, dieses Normalverhalten zu deaktivieren und für 
die Seite nach der Teileseite~-- und abhängig von \Option{chapterpage} 
auch nach einem Kapitelanfang auf einer separaten Seite~-- den Seitenstil der 
Option \Option{cleardoublepage} zu übernehmen. Des Weiteren kann auch ein 
anderer, beliebiger, bereits definierter Seitenstil gewählt werden. Außerdem
kann im farbigen Layout die Rückseite in der gleichen Farbe wie die 
Vorderseite von Teil oder Kapitel gesetzt werden. \notudscrartcl
%
\begin{values}
\itemfalse
  Die Rückseiten sind vollständig leere Seiten, unabhängig von Option
  \Option{cleardoublepage}.
\itemtrue*
  Der Seitenstil der Rückseite von Teilen und gegebenenfalls Kapiteln 
  entspricht der Einstellung von \Option{cleardoublepage} für Vakatseiten.
\item[current]
  Es wird der aktuell definierte Seitenstil (\Macro{pagestyle}) für die 
  erzeugte Rückseite verwendet.
\item[color/colour]
  Im farbigen Layout ist auch die Rückseite von Teilen und Kapiteln farbig, 
  siehe \Option{clearcolor}.
\makeatletter\item@values[\PName{Seitenstil}\textsl{:}]\makeatother
  Mit der Angabe von \Option*{cleardoublespecialpage}[\PName{Seitenstil}] 
  kann ein beliebiger, bereits definierter Seitenstil für die Rückseite nach 
  Teilen und Kapiteln verwendet werden.
\end{values}
\end{Declaration}

\begin{Declaration}{\Option{clearcolor}[\PBoolean]}[false]%
\printdeclarationlist%
\index{Titel}\index{Layout!Titel}%
\index{Teileseiten}\index{Layout!Teileseiten}%
\index{Kapitelseiten}\index{Layout!Kapitelseiten}%
\index{Satzspiegel!doppelseitig}\index{Leerseiten}%
%
Sollten beim farbigen Layout die Optionen \Option{twoside} sowie auch
\Option{open}[right] gesetzt sein, so werden beim Aktivieren dieser Option die 
Rückseiten von Teilen~-- und je nach Einstellung von \Option{chapterpage} 
gegebenenfalls auch von Kapiteln~-- farbig gesetzt.%
\footnote{%
  Dies führt bei der Ausgabe zu farbigen Blättern (Vorder- und Rückseite) der 
  entsprechenden Elemente des Layouts.
}
Die Option wirkt sich ebenfalls auf die Rückseite des Titels aus.%
\footnote{%
  siehe \Macro{uppertitleback} und \Macro{lowertitleback} der 
  \KOMAScript"=Dokumentation (\scrguide*)
}
Der Stil dieser zusätzlich eingefügten Rückseiten ist abhängig von der Option
\Option{cleardoublespecialpage}.
%
\begin{values}
\itemfalse
  Es werden weiße Rückseiten bei Titel, Teilen und gegebenenfalls Kapiteln 
  erzeugt.
\itemtrue*
  Die rückwärtigen Seiten der genannten Elemente des Layouts sind farbig.
\end{values}
\end{Declaration}

\subsection{Seiten im Stil des \CDs}
\begin{Declaration}[v2.02]{\PageStyle{tudheadings}}
\begin{Declaration}[v2.02]{\PageStyle{plain.tudheadings}}
\begin{Declaration}[v2.02]{\PageStyle{empty.tudheadings}}
\printdeclarationlist%
\label{sec:tudheadings}
%
Ein zentrales Element des \CDs der \TnUD ist der eingeführte prägnante 
Seitenkopf mit der Angabe von Fakultät (\Macro{faculty}), Einrichtung 
(\Macro{department}), Institut (\Macro{institute}) und Lehrstuhl 
(\Macro{chair}) in der dazugehörigen Kopfzeile mit Querbalken. Durch die 
Verwendung von \Package{scrlayer-scrpage} lassen sich einzelne Seiten oder auch 
ganze Dokumente sehr einfach in diesem Stil setzen. Hierzu muss lediglich einer 
der Seitenstile mit \Macro*{pagestyle}\Parameter{Seitenstil} geladen werden. 

Allen Seitenstilen ist der typische Kopf gemein. Der Fuß des Seitenstils 
\PageStyle*{empty.tudheadings} ist immer leer, \PageStyle*{tudheadings} und 
\PageStyle*{plain.tudheadings} übernehmen die Einstellungen für den Fuß aus der 
Anwenderschnittstelle von \Package{scrlayer-scrpage}.%
\footnote{%
  Es können die Befehle \Macro{lefoot}, \Macro{cefoot} und \Macro{refoot} sowie 
  \Macro{lofoot}, \Macro{cofoot} und \Macro{rofoot} verwendet werden.
}
Wie diese zu verwenden ist, ist der \KOMAScript"=Anleitung zu entnehmen. 
Alternativ zu einer eigenen Definition der Fußzeile kann außerdem die Option 
\Option{cdfoot} verwendet werden.

Sobald einer der neu definierten Seitenstile aktiviert wurde, sind diese 
zusätzlich unter den Namen \PageStyle*{headings}, \PageStyle*{plain} und 
\PageStyle*{empty} verwendbar. Das hat den Vorteil, dass bei Optionen oder 
Befehlen, die automatisch zwischen \PageStyle*{headings}, \PageStyle*{plain} 
und \PageStyle*{empty} umschalten, durch die einmalige Auswahl von einem der 
Stile \PageStyle*{tudheadings}, \PageStyle*{plain.tudheadings} oder 
\PageStyle*{empty.tudheadings} nun zwischen diesen Stilen umgeschaltet wird. Um 
auf das normale Verhalten zurückzuschalten, muss einer der beiden Seitenstile 
\PageStyle*{scrheadings} oder \PageStyle*{plain.scrheadings} aktiviert werden.
\end{Declaration}
\end{Declaration}
\end{Declaration}

\begin{Declaration}{\Macro{faculty}\Parameter{Fakultät}}
\begin{Declaration}{\Macro{department}\Parameter{Einrichtung}}
\begin{Declaration}{\Macro{institute}\Parameter{Institut}}
\begin{Declaration}{\Macro{chair}\Parameter{Lehrstuhl}}
\begin{Declaration}{\Macro{extraheadline}\Parameter{Textzeile}}
\printdeclarationlist%
\index{Kopfzeile}\index{Layout!Kopfzeile}\index{Kopfzeile!Felder}%
\index{Querbalken}\index{Layout!Querbalken}\index{Querbalken!Felder}%
%
Für den Seitenstil des \CDs der \TnUD typisch ist die Kopfzeile mit dem 
charakteristischen Querbalken. In dieser wird~-- falls angegeben~-- in fetter 
Schrift die Fakultät ausgegeben, danach folgen durch Kommas getrennt die 
Einrichtung, das Institut und der Lehrstuhl beziehungsweise die Professur. 
Sollte der Platz in der ersten Zeile nicht ausreichen, erfolgt ein 
automatischer Zeilenumbruch.

In besonderen Ausnahmefällen erlaubt das \CD die Angabe einer zusätzlichen
zweiten beziehungsweise dritten Zeile, welche weitere, frei wählbare Angaben 
enthält. Diese kann mit dem Befehl \Macro*{extraheadline}\Parameter{Textzeile} 
definiert werden.
\end{Declaration}
\end{Declaration}
\end{Declaration}
\end{Declaration}
\end{Declaration}

\begin{Declaration}[%
  v2.02!\protect\DDC-Logo automatisch in Kopf oder Fuß%
]{\Option{ddc}[\PSet]}[false]
\begin{Declaration}[v2.02]{\Option{ddchead}[\PSet]}[false]
\begin{Declaration}[%
  v2.02!neue Werte für die Farbwahl des Logos von \protect\DDC%
]{\Option{ddcfoot}[\PSet]}[false]
\printdeclarationlist%
\index{Zweitlogo}\index{Layout!Zweitlogo}\index{\DDC-Logo}%
%
Diese Optionen fügen das Logo von \DDC entweder im Kopf oder im Fuß der Seiten
mit dem Stil \PageStyle{tudheadings} ein. Mit \Option*{ddc} wird dieses 
automatisch entweder im Kopf oder~-- falls ein Zweitlogo mit \Macro{headlogo} 
angegeben wurde~-- im Fuß gesetzt. Die anderen beiden Optionen setzen das Logo 
zwingend entweder im Kopf (\Option*{ddchead}) oder im Fuß (\Option*{ddcfoot}), 
wobei erstgenannte ein optionales Zweitlogo dabei unterdrückt. Die Verwendung 
einer der drei Optionen führt zur Deaktivierung der anderen beiden, sie 
schließen sich folglich gegenseitig aus. Die möglichen Werte für diese Optionen 
sind:
%
\begin{values}
\itemfalse
  Bei den \PageStyle{tudheadings}-Seitenstile erscheint kein Logo von \DDC.
\itemtrue*
  Das Logo von \DDC wird im Kopf beziehungsweise im Fuß verwendet. Die Wahl der 
  Farbe des Logos geschieht passend zur farblichen Ausprägung der Seite selbst.
\end{values}
%
Soll die Farbe des \DDC-Logos manuell erfolgen, können folgende Werte verwendet 
werden:
%
\begin{values}
\item[color/colour]
  Im Kopf oder Fuß wird die achtfarbige 4C"~Variante des \DDC-Logos genutzt.
\item[colorblack/colourblack]
  Das Logo wird in der achtfarbige 4C"~Variante mit schwarzem \DDC-Schriftzug 
  anstelle des grauen. Für den Fuß wird der grüne Claim ebenfalls durch einen 
  schwarzen ersetzt. Dies ist insbesondere für kleine Darstellungen des Logos 
  im Fuß sinnvoll.
\item[gray/grey/cdgray/cdgrey]
  Dies Ausgabe des \DDC-Logos erfolgt in Graustufen.
\item[black]
  Verwendung des Logos in Graustufen mit schwarzem Schriftzug.
\item[blue/cddarkblue]
  Der Schriftzug und das Logo werden in der primären Hausfarbe \Color{HKS41} 
  und den entsprechenden Abstufungen gesetzt
\item[white]
  Das \DDC-Logo sowie der dazugehörige Schriftzug sind vollständig weiß.
\end{values}
%
\end{Declaration}
\end{Declaration}
\end{Declaration}

\begin{Declaration}{%
  \Macro{headlogo}\LParameter\Parameter{Dateiname}%
}
\printdeclarationlist%
\index{Zweitlogo|?}\index{Layout!Zweitlogo}\index{\DDC-Logo}%
%
Neben dem Logo der \TnUD darf zusätzlich ein Zweitlogo im Kopf verwendet werden.
Dieses lässt sich mit diesem Befehl einbinden. Normalerweise wird es auf die 
Höhe der Erstlogos skaliert. Über das optionale Argument können weitere 
Formatierungsbefehle an den verwendeten Befehl \Macro{includegraphics} 
durchgereicht werden, um beispielsweise die Größe des Zweitlogos anzupassen.
Welche Parameter angepasst werden können, ist der Dokumentation des
\Package{graphicx}-Paketes zu entnehmen.

Sollte die Option \Option{ddc} aktiviert sein, wird das \DDC-Logo nicht im Kopf 
sondern automatisch im Fuß gesetzt. Die Option \Option{ddchead} setzt dieses 
auf jeden Fall im Kopf und überschreibt damit das mit \Macro*{headlogo} 
angegebene Zweitlogo.
\end{Declaration}

\begin{Declaration}{\Option{widehead}[\PBoolean]}%
  [false][\Option{cd}[color]:true]%
\printdeclarationlist%
\index{Querbalken}\index{Layout!Querbalken}%
%
Für die \TUDScript-Klassen ist ein Seitenlayout entstanden, welche den Kopf des
\CDs umsetzt. Dieser besteht aus dem Logo der \TnUD sowie einem darunter 
befindlichen Querbalken, in welchem Fakultät, Einrichtung, Institut und 
Lehrstuhl%
\footnote{%
  \Macro{faculty}, \Macro{department}, \Macro{institute} sowie \Macro{chair}%
}
aufgeführt werden können. Bei der Ausprägung dieses Balkens gibt es zwei 
Varianten. Die Außenlinien laufen entweder bis zum Text"~ oder bis zum 
Blattrand.

Für den Fall, dass ein randloser Ausdruck technisch nicht möglich ist, 
kann die letztere der beiden Variante Probleme bereiten. Deshalb kann mit der 
Option \Option*{widehead} die Breite des Querbalkens angepasst werden. 
Normalerweise ist der Balken auf die Textbreite begrenzt, lediglich im farbigen 
Layout wird dieser standardmäßig bis zum Blattrand verlängert.
%
\begin{values}
\itemfalse
  Der Querbalken im Kopf erstreckt sich nur über den Textbereich.
\itemtrue*
  Die horizontale Ausdehnung des Querbalkens erstreckt sich bis an den 
  Blattrand.\footnote{Voreinstellung bei \Option{cd}[color]} 
\end{values}
\end{Declaration}

\begin{Declaration}[v2.02]{\Option{barfont}[\PSet]}%
  [true][\Option{cdfont}[false]:false]
\printdeclarationlist%
\index{Kopfzeile!Schrift}%
\index{Layout!Kopfzeile}%
%
Mit dieser Option kann die Schrift im Querbalken der Kopfzeile für Seiten, 
welche in einem der \PageStyle{tudheadings}"=Seitenstilen gesetzt wird, 
beeinflusst werden.
%
\begin{values}
\itemfalse
  Sollte mit \Option{cdfont}[false] die Verwendung der Hausschrift im Stil des 
  \CDs der \TnUD deaktiviert worden sein, wird die Kopfzeile im Querbalken in
  den Serifenlosen der genutzten Schrift gesetzt. Sind die Hausschriften 
  aktiviert, hat diese Einstellung keinen Einfluss.
\itemtrue*[cdfont/light/lightfont/noheavyfont]
  Im Querbalken wird für \Macro{faculty} \textubn{Univers~65~Bold} verwendet, 
  für die Felder \Macro{department}, \Macro{institute}, \Macro{chair} und 
  \Macro{extraheadline} kommt \textuln{Univers~45~Light} zum Einsatz.
\item[heavy/heavyfont]
  Der Inhalt von \Macro{faculty} wird weiterhin in \textubn{Univers~65~Bold} 
  gesetzt, für die restlichen Felder wird \texturn{Univers~55~Regular} genutzt.
\end{values}
\end{Declaration}

\begin{Declaration}[%
  v2.02!Parameter \Key*{\Environment{tudpage}}{head} und 
    \Key*{\Environment{tudpage}}{foot} entfernt
]{\Environment{tudpage}[\OLParameter{Sprache}]}
\begin{Declaration}{\Key{\Environment{tudpage}}{language}[\PName{Sprache}]}
\begin{Declaration}{\Key{\Environment{tudpage}}{columns}[\PName{Anzahl}]}
\begin{Declaration}{\Key{\Environment{tudpage}}{color}[\PName{Farbe}]}
\begin{Declaration}[v2.02]{\Key{\Environment{tudpage}}{pagestyle}[\PSet]}
\begin{Declaration}{\Key{\Environment{tudpage}}{headlogo}[\PName{Dateiname}]}
\begin{Declaration}[v2.02]{\Key{\Environment{tudpage}}{ddc}[\PSet]}
\begin{Declaration}[v2.02]{\Key{\Environment{tudpage}}{ddchead}[\PSet]}
\begin{Declaration}[v2.02]{\Key{\Environment{tudpage}}{ddcfoot}[\PSet]}
\begin{Declaration}{\Key{\Environment{tudpage}}{cdfont}[\PSet]}
\begin{Declaration}[v2.02]{\Key{\Environment{tudpage}}{barfont}[\PSet]}
\begin{Declaration}{\Key{\Environment{tudpage}}{widehead}[\PBoolean]}
\printdeclarationlist%
\index{Layout}\index{Layout!Seitenstil}%
\index{Kopfzeile}\index{Layout!Kopfzeile}%
\index{Fußzeile}\index{Layout!Fußzeile}%
\index{Schrift}\index{Kopfzeile!Schrift}
%
Diese Umgebung hat ihren Ursprung, als die \PageStyle{tudheadings}"=Seitenstile 
noch nicht verfügbar waren. Mit dieser lassen sich ebenfalls eine oder mehrere 
Seiten innerhalb des Dokumentes mit dem Kopf im \CD setzen. 

Dabei lassen sich verschiedene Parameter als optionales Argument angegeben. 
Wird das Paket \Package{babel} verwendet, kann die Sprache innerhalb der 
Umgebung mit \Key*{\Environment{tudpage}}{language}[\PName{Sprache}] geändert 
werden. Dafür muss die gewünschte Sprache entweder als Paketoption oder besser 
noch als Klassenoption angegeben worden sein. Dadurch werden lokal innerhalb 
der Umgebung die Bezeichner und Trennungsmuster sprachspezifisch angepasst. Des 
Weiteren wird das Paket \Package{multicol} unterstützt. Wird dieses geladen, 
wird mit dem Parameter \Key*{\Environment{tudpage}}{columns}[\PName{Anzahl}] 
der Inhalt der Umgebung mehrspaltig gesetzt.

Mit dem Parameter \Key*{\Environment{tudpage}}{color} kann die Farbe des Kopfes 
auf eine beliebige geändert werden. Diese ist für den Fall eines farbigen 
Layouts (\Option{cd}[pale|color]) auf die primäre Hausfarbe \Color{HKS41} 
gesetzt, sonst ist der Kopf standardmäßig schwarz. Außerdem kann mit
\Key*{\Environment{tudpage}}{pagestyle} der Seitenstil angepasst werden. 
Gültige Werte sind \PValue{headings}, \PValue{plain}, \PValue{empty} oder einer 
der \PageStyle{tudheadings}"=Seitenstile. Soll lokal ein anderes Zweitlogo als 
das mit \Macro{headlogo} gegebene erscheinen, so kann der Parameter 
\Key*{\Environment{tudpage}}{headlogo}[\PName{Dateiname}] verwendet werden.

Die anderen Parameter entsprechen in ihrem Verhalten prinzipiell den 
gleichnamigen Klassenoptionen, wirken sich jedoch nur lokal innerhalb der 
\Environment*{tudpage}"=Umgebung aus. Namentlich sind dies die Optionen 
\Option'{cdfont}, \Option{ddc}, \Option{ddchead} und \Option'{ddcfoot} sowie 
\Option{barfont} und \Option[pageref]{widehead}. Das Verhalten sowie die 
jeweils gültigen Wertzuweisungen können in den entsprechenden Abschnitten 
der Dokumentation nachgelesen werden.
\end{Declaration}
\end{Declaration}
\end{Declaration}
\end{Declaration}
\end{Declaration}
\end{Declaration}
\end{Declaration}
\end{Declaration}
\end{Declaration}
\end{Declaration}
\end{Declaration}
\end{Declaration}

\ToDo[imp,nxt]{%
  Titelseite und Cover auf Drittlogos und dergleichen mit scrlayer erweitern.
}[v2.x]


\subsection{Der Titel und die Umschlagseite}
\label{sec:title}%\label{sec:cover}%
\index{Titel|!(}
\index{Umschlagseite|!}%
\index{Titel!Umschlagseite}\index{Layout!Umschlagseite}%
%
Für den Titel werden alle Felder unterstützt, die bereits durch \KOMAScript{} 
bereitgestellt werden. Darüber hinaus werden für die \TUDScript-Klassen weitere 
Felder definiert, die Auswirkungen auf die Gestalt des Titels haben. Diese 
werden nachfolgend in diesem \autorefname erläutert. Der Titel~-- bestehend aus 
möglichem Schmutztitel, der eigentlichen Titelseite und der nachgelagerten 
Elementen~-- kann mit dem Befehl \Macro{maketitle} ausgegeben werden. Außerdem 
kann im zweispaltigen Satz der \Macro{maketitleonecolumn} verwendet werden, 
welcher einen einspaltigen Einfügung nach dem Titel selbst ermöglicht.

Zusätzlich zum Titel lässt sich mit \Macro{makecover} eine Umschlagseite 
erzeugen. Diese kann insbesondere für gebundene Arbeiten verwendet werden. Es 
wird~-- im Vergleich zum Titel~-- lediglich einer reduzierte Anzahl an Feldern 
auf dieser ausgegeben.

\ChangedAt{v2.02}
Für alle Felder des Titels und der Umschlagseite lässt sich die verwendete 
Schrift anpassen. Dabei werden sowohl die bereits durch \KOMAScript{} 
bereitgestellten Schriftelemente \Font{titlehead}, \Font{subject}, 
\Font{title}, \Font{subtitle}, \Font{author}, \Font{date}, \Font{publishers} 
und \Font{dedication} als auch die neuen \Font{titlepage} und \Font{thesis} 
unterstützt.
%
\begin{Example}
In diesem Dokument wurde der Untertitel derart geändert, dass dieser nicht 
standardmäßig in \DIN sondern in \textubn{Univers~65~Bold} ausgegeben wird.
\begin{Code}[escapechar=§]
\addtokomafont{subtitle}{\univbn}
\end{Code}
\end{Example}

\begin{Declaration}[%
  v2.02!Unterstützung der Schriftelemente \Font*{titlehead}{,} 
    \Font*{subject}{,} \Font*{title}{,} \Font*{subtitle}{,} \Font*{author}{,} 
    \Font*{date}{,} \Font*{publishers}{,} \Font*{dedication}{,} 
    \Font*{titlepage} und \Font*{thesis}%
]{\Macro{maketitle}\OLParameter{Seitenzahl}}
\begin{Declaration}[v2.02]{%
  \Key{\Macro{maketitle}}{pagenumber}[\PName{Seitenzahl}]%
}
\begin{Declaration}[v2.02]{\Key{\Macro{maketitle}}{cdfont}[\PSet]}
\printdeclarationlist%
\index{Layout!Titel}%
\index{Satzspiegel!doppelseitig}%
%
Der Befehl \Macro*{maketitle} setzt für \Option{cdtitle}[false] den normalen 
\KOMAScript"=Titel{}, ansonsten wird die Titelseite im \CD der \TnUD erzeugt. 
Letztere Variante ist im Vergleich zum Standardtitel um eine Vielzahl von 
Feldern erweitert worden und erlaubt insbesondere die Angabe von Daten für das 
Deckblatt einer akademischen Abschlussarbeit. Die einzelnen Felder werden in 
diesem \autorefname erläutert. Wird das Dokument doppelseitig und mit rechts 
öffnenden Kapiteln gesetzt,%
\footnote{%
  \Option{twoside} und \Option{open}[right], Standard für \Class{tudscrbook}
}
so wird zusätzlich die Option \Option{clearcolor} beachtet.

Das optionale Argument erlaubt~-- ebenso wie bei den \KOMAScript"=Klassen~-- 
die Änderung der Seitenzahl der Titelseite. Diese wird jedoch nicht ausgegeben, 
sondern beeinflusst lediglich die Zählung. Sie sollten hier unbedingt eine 
ungerade Zahl wählen, da sonst die gesamte Zählung durcheinander gerät. 
Zusätzlich kann der Parameter \Key*{\Macro{maketitle}}{cdfont} im optionalen 
Argument verwendet werden, um die Nutzung der Schriften des \CDs zu regulieren. 
Er entspricht in seinem Verhalten der gleichnamigen Klassenoption. Die gültigen 
Wertzuweisungen können der Beschreibung der Option \Option'{cdfont} entnommen 
werden. Die Einstellungen dieses Parameters wirkt sich nur lokal und einzig auf 
die Umschlagseite aus.
\end{Declaration}
\end{Declaration}
\end{Declaration}

\begin{Declaration}{%
  \Macro{maketitleonecolumn}\OParameter{Seitenzahl}\Parameter{Einspaltentext}%
}
\begin{Declaration}[v2.02]{%
  \Key{\Macro{maketitleonecolumn}}{pagenumber}[\PName{Seitenzahl}]%
}
\begin{Declaration}[v2.02]{\Key{\Macro{maketitleonecolumn}}{cdfont}[\PSet]}
\printdeclarationlist%
\index{Layout!Titel}%
\index{Satzspiegel!doppelseitig}%
\index{Zweispaltensatz}%
%
Im zweispaltigen Satz (\Option{twocolumn}) wird mit \Macro*{maketitle} die 
Titelseite selbst immer einspaltig gesetzt. Direkt nach dem Titel folgt 
normalerweise der zweispaltige Fließtext. Die \TUDScript-Klassen ermöglichen 
mit \Macro*{maketitleonecolumn}, nach dem Titel zusätzlich auch noch weitere 
Textpassagen~-- beispielsweise eine Zusammenfassung~-- einspaltig zu setzen.

Wird der Befehl bei einer Titelseite (\Option{titlepage}[true]) verwendet, wird 
der Inhalt des Argumentes direkt nach dieser auf einer neuen Seite ebenfalls 
einspaltig ausgegeben. Kommt jedoch ein Titelkopf (\Option{titlepage}[false]) 
zum Einsatz, so folgt nach diesem die einspaltige Textpassage aus dem Argument. 
Danach wird auf das zweispaltige Layout umgeschaltet, 

Der optionale Parameter von \Macro*{maketitleonecolumn} kann äquivalent zum 
Befehl \Macro{maketitle} für die Anpassung der Seitenzahl und der verwendeten 
Schrift verwendet werden.
\end{Declaration}
\end{Declaration}
\end{Declaration}

\begin{Declaration}[%
  v2.02!Umschlagseite für Layout ohne \CD hinzugefügt,%
  v2.02!Unterstützung der Schriftelemente \Font*{titlehead}{,} 
    \Font*{subject}{,} \Font*{title}{,} \Font*{subtitle}{,} \Font*{author}{,} 
    \Font*{publishers}{,} \Font*{titlepage} und \Font*{thesis}%
]{\Macro{makecover}\OLParameter{Seitenzahl}}
\begin{Declaration}{\Key{\Macro{makecover}}{cdlayout}[\PBoolean]}
\begin{Declaration}[v2.02]{%
  \Key{\Macro{makecover}}{pagenumber}[\PName{Seitenzahl}]%
}
\begin{Declaration}{\Key{\Macro{makecover}}{cdfont}[\PSet]}
\printdeclarationlist%
%
Eine Umschlagseite wird zumeist für gebundene Abschlussarbeiten verlangt, um 
diese beispielsweise für einen Prägedruck auf dem Buchdeckel zu verwenden. 
Hierfür ist es sinnvoll, mit \Option{cdcover}[true] die farbige Ausprägung der 
Umschlagseite zu deaktivieren, falls diese für das restliche Dokument aktiv ist 
(\Option{cd}[color]).

Wird \Option{cdcover}[true] gewählt, so wird die Umschlagseite im \CD der 
\TnUD gesetzt. Auf dieser werden der Titel des Dokumentes, die Typisierung 
durch \Macro{thesis} und/oder \Macro{subject} sowie der Autor oder respektive 
die Autoren und gegebenenfalls der mit \Macro{publishers} angegebene Verlag 
ausgegeben.
\ChangedAt{v2.02}
Für die Einstellung \Option{cdcover}[false] wird lediglich der normale 
\KOMAScript"=Titel als separate Umschlagseite ausgegeben. 

Die Titelseite selbst gehört immer zum Buchblock und wird daher im gleichen 
Satzspiegel gesetzt. Dem entgegen steht die Umschlagseite, welche zumeist in 
einem anderen Layout erscheint. Normalerweise wird das Cover~-- unabhängig von 
der Option \Option{geometry}~-- im asymmetrischen Satzspiegel des \CDs gesetzt. 
Mit \Key*{\Macro{makecover}}{cdlayout}[false] im optionalen Argument kann das 
Verhalten geändert werden. In diesem Fall erscheint auch die Umschlagseite im 
Buchblock des restlichen Dokumentes. Allerdings können für diese Einstellung 
die Seitenränder durch den Nutzer mit den Befehlen \Macro{coverpagetopmargin}, 
\Macro{coverpageleftmargin}, \Macro{coverpagerightmargin} sowie 
\Macro{coverpagebottommargin} frei angepasst werden. Mehr dazu ist im 
\KOMAScript"=Handbuch \scrguide zu finden.

Die beiden anderen optionalen Parameter \Key*{\Macro{makecover}}{pagenumber} 
sowie \Key*{\Macro{makecover}}{cdfont} dienen~-- äquivalent zum Befehl 
\Macro{maketitle}~-- zur Anpassung der Seitenzahl und der verwendeten Schrift.
\end{Declaration}
\end{Declaration}
\end{Declaration}
\end{Declaration}

\begin{Declaration}{\Macro{extratitle}\Parameter{Schmutztitel}}
\begin{Declaration}{\Macro{titlehead}\Parameter{Kopf}}
\begin{Declaration}{\Macro{title}\Parameter{Titel}}
\begin{Declaration}{\Macro{subtitle}\Parameter{Untertitel}}
\begin{Declaration}{\Macro{publishers}\Parameter{Verlag}}
\begin{Declaration}{\Macro{thanks}\Parameter{Fußnote}}
\begin{Declaration}{\Macro{uppertitleback}\Parameter{Titelrückseitenkopf}}
\begin{Declaration}{\Macro{lowertitleback}\Parameter{Titelrückseitenfuß}}
\begin{Declaration}{\Macro{dedication}\Parameter{Widmung}}
\printdeclarationlist%
\index{Titel!Felder}%
%
Diese Befehle entsprechen den in ihrem Verhalten den originalen Pendants der 
\KOMAScript"=Klassen{} und sollen hier der Vollständigkeit halber erwähnt 
werden.

Die Ausgabe des mit \Macro*{extratitle} definierten Schmutztitels~-- welcher 
beliebig gestaltet und formatiert werden kann~-- erfolgt als Bestandteil der 
Titelei mit \Macro{maketitle} vor der eigentlichen Titelseite. Mit dem Befehl 
\Macro*{titlehead} kann ein zusätzlicher, beliebig formatierbarer Text oberhalb 
der Typisierung und des Titels ausgegeben werden. Da die vertikale Position des 
Dokumenttitels durch das \CD fest vorgegeben ist, kann es~-- im Gegensatz zu 
den \KOMAScript"=Klassen~-- passieren, dass der Kopf des Haupttitels selbst in 
die Kopfzeile ragt. Dies wird durch die \TUDScript-Klassen nicht geprüft und 
muss gegebenenfalls vom Anwender kontrolliert werden.

Die Befehle \Macro*{title} und \Macro*{subtitle} bedürfen keiner weiteren 
Erklärung. Anzumerken ist, dass sowohl Titel als auch Untertitel normalerweise 
in Majuskeln und \DIN gesetzt werden. Der mit dem Befehl \Macro*{publishers} 
definierte Inhalt muss nicht zwingende einen Verlag bezeichnen sondern kann 
auch andere Informationen beinhalten, welche am Ende der Titelseite ausgegeben 
werden sollen.

Fußnoten werden auf dem Titel nicht mit \Macro*{footnote}, sondern mit der 
Anweisung \Macro*{thanks} erzeugt. Sie dienen in der Regel für Anmerkungen bei 
Titel oder den Autoren. Als Fußnotenzeichen werden dabei Symbole statt Zahlen 
verwendet. Der Befehl \Macro*{thanks} kann nur innerhalb des Arguments einer 
der Anweisungen für die Titelseite wie beispielsweise \Macro{author} oder 
\Macro{title} verwendet werden.

\index{Satzspiegel!doppelseitig}%
Im doppelseitigen Druck lässt sich die Rückseite der Haupttitelseite für 
weitere Angaben nutzen. Sowohl den Titelrückseitenkopf als auch den
Titelrückseitenfuß kann der Anwender mit \Macro*{uppertitleback} und 
\Macro*{lowertitleback} frei gestalten.

Mit \Macro*{dedication} kann eine eigene Widmungsseite zentriert und in etwas 
größerer Schrift gesetzt werden. Die Rückseite ist wie die des Schmutztitels 
grundsätzlich leer. Die Widmung wird zusammen mit der restlichen Titelei durch 
\Macro{maketitle} ausgegeben und muss daher vor dieser Anweisung definiert sein.
\end{Declaration}
\end{Declaration}
\end{Declaration}
\end{Declaration}
\end{Declaration}
\end{Declaration}
\end{Declaration}
\end{Declaration}
\end{Declaration}

\begin{Declaration}{\Macro{titledelimiter}\Parameter{Trennzeichen}}
\printdeclarationlist%
\index{Titel!Felder}\index{Titel!Trennzeichen}%%
%
Für den Titel und die Umschlagseite werden durch die \TUDScript-Klassen
eine Reihe von zusätzlichen Feldern bereitgestellt. Einigen dieser Felder wird 
eine Beschreibung (siehe dazu \autoref{sec:localization}) vorangestellt. 
Dazwischen wird bei der Ausgabe ein Trennzeichen eingefügt. Ein Doppelpunkt 
gefolgt von einem Leerzeichen (:\Macro*{nobreakspace}) ist hierfür die 
Voreinstellung. Mit dem Befehl \Macro*{titledelimiter} lässt sich dieses 
Trennzeichen beliebig anpassen.
\end{Declaration}


\begin{Declaration}{\Macro{author}\Parameter{Autor(en)}}
\begin{Declaration}{\Macro{authormore}\Parameter{Autorenzusatz}}
\begin{Declaration}{\Macro{dateofbirth}\Parameter{Geburtsdatum}}
\begin{Declaration}{\Macro{placeofbirth}\Parameter{Geburtsort}}
\begin{Declaration}{\Macro{matriculationnumber}\Parameter{Matrikelnummer}}
\begin{Declaration}{\Macro{matriculationyear}\Parameter{Immatrikulationsjahr}}
\printdeclarationlist%
\index{Titel!Felder}\index{Autorenangaben|?}%
\index{Datum!Geburtsdatum|?}%
%
Mit dem Befehl \Macro*{author} wird der Autor angegeben. Innerhalb des 
Argumentes können auch mehrere Autoren aufgeführt werden, wobei diese in diesem 
Fall jeweils mit \Macro{and} zu trennen sind. Zu erwähnen ist, dass alle 
weiteren hier vorgestellten Befehle selbst im Argument von \Macro*{author} 
stehen können. Damit wird es möglich, jedem Autor unterschiedliche Angaben 
mitzugeben.

Mit \Macro*{authormore} wird unter dem Autor eine Zeile ausgegeben, welche 
durch den Anwender frei belegt werden kann. Sollte das Paket \Package{isodate} 
geladen sein, so wird die damit eingestellte Formatierung des Datums durch 
\Macro*{dateofbirth}~-- wie übrigens bei jedem anderem Datumsfeld der 
\TUDScript-Klassen auch~-- verwendet. Dafür der Befehl \Macro{printdate} aus 
diesem Paket verwendet. Die weiteren Befehle als zusätzliche Angabe erklären 
sich von selbst.
\end{Declaration}
\end{Declaration}
\end{Declaration}
\end{Declaration}
\end{Declaration}
\end{Declaration}

\begin{Declaration}{\Macro{and}}
\printdeclarationlist%
\index{Kollaboratives Schreiben|?}\index{Titel!Kollaboratives Schreiben}%
%
Dieser Befehl wird sowohl bei den \hologo{LaTeX}"=Standardklassen als auch bei 
den \KOMAScript"=Klassen lediglich auf der Titelseite dazu verwendet, mehrere 
Autoren im Argument von \Macro{author} voneinander zu trennen.

Bei den \TUDScript-Klassen hingegen ist dieser Befehl derart in seiner Funktion 
erweitert worden, dass damit die Angabe einer kollaborativen Autorenschaft für 
Abschlussarbeiten innerhalb des Befehls \Macro{author} möglich ist. Außerdem 
kann er noch im Argument von \Macro{supervisor}, \Macro{referee} sowie 
\Macro{advisor} verwendet werden, um mehrere Betreuer beziehungsweise Gutachter 
und Fachreferenten anzugeben. Er ist dabei nicht auf die Verwendung für den 
Titel allein beschränkt. Auch bei den Umgebungen \Environment{task}, 
\Environment{evaluation} und \Environment{notice} kann er eingesetzt werden.
\end{Declaration}
%
\begin{Example}
Angenommen, es soll eine Abschussarbeit von zwei unterschiedlichen Autoren in 
kollaborativer Gemeinschaft erstellt werden, so könnte man die Autorenangaben 
folgendermaßen gestalten:
\begin{Code}
\author{%
  Mickey Mouse
  \matriculationnumber{12345678}
  \dateofbirth{2.1.1990}
  \placeofbirth{Dresden}
\and%
  Donald Duck
  \matriculationnumber{87654321}
  \dateofbirth{1.2.1990}
  \placeofbirth{Berlin}
}
\matriculationyear{2010}
\end{Code}
Alle zusätzlichen Angaben außerhalb des Argumentes von \Macro{author} werden 
für beide Autoren gleichermaßen übernommen. Angaben innerhalb des Argumentes 
von \Macro{author} werden den jeweiligen, mit \Macro{and} getrennten Autoren 
zugeordnet. Mehr dazu ist im Minimalbeispiel in \autoref{sec:exmpl:thesis}.
\end{Example}

\begin{Declaration}{\Macro{thesis}\Parameter{Typisierung}}
\begin{Declaration}{\Macro{subject}\Parameter{Typisierung}}
\printdeclarationlist%
\index{Titel!Felder}%
\index{Abschlussarbeit|!}\index{Typisierung}%
%
Mit diesen beiden Befehlen kann der Typ der Dokumentes beziehungsweise der 
Abschlussarbeit angegeben werden. Während der Befehl \Macro*{thesis} den Inhalt 
des Feldes unter dem Titel vertikal zentriert und in \DIN auf der Titelseite 
ausgibt, erscheint der Inhalt des Befehls \Macro*{subject} in \Univers oberhalb 
des Titels. Es können auch beide Befehle parallel mit unterschiedlichen 
Inhalten verwendet werden. Der Befehl \Macro*{thesis} dient den 
\TUDScript"=Dokumentklassen außerdem zur Erkennung von Abschlussarbeiten 
gedacht, da für diese spezielle Felder bereitgehalten werden und auch die 
Titelseite leicht geändert gesetzt wird.

Des Weiteren ist es bei beiden Befehlen möglich, spezielle Werte als Argument 
zur Typisierung des Dokumentes zu verwenden. Diese werden entsprechend der 
gewählten Dokumentensprache~-- entweder Deutsch oder Englisch~-- entschlüsselt 
und gesetzt. Die möglichen Werte sind \autoref{tab:thesis} zu entnehmen. Dabei 
ist zu beachten, dass das Setzen eines speziellen Wertes für \emph{entweder} 
\Macro*{thesis} \emph{oder} \Macro*{subject} möglich ist. Die Verwendung eines 
der genannten Werte führt immer dazu, dass das Dokument als Abschlussarbeiten 
erkannt und die erweiterte Titelseite aktiviert wird. Gleichzeitig wird damit 
die Option \Option{subjectthesis} beeinflusst. Sollte vom Anwender kein 
explizites Verhalten für \Option{subjectthesis} definiert sein, so führt die 
Verwendung von \Macro*{thesis}\Parameter{Wert} zu \Option{subjectthesis}[false] 
und \Macro*{subject}\Parameter{Wert} zu \Option{subjectthesis}[true].
%
\begin{table}
\index{Bezeichner}\index{Bezeichner!Typisierung}%\\
\index{Abschlussarbeit!Typisierung}%
\caption{%
  Spezielle Werte zur Typisierung des Dokumentes für
  \Macro*{thesis} und \Macro*{subject}%
}
\label{tab:thesis}%
\centering%
\makeatletter%
\def\@tempa#1{%
  \Term{#1} & \@nameuse{#1} & \selectlanguage{english}\@nameuse{#1}%
  \tabularnewline%
}%
\begin{tabular}{llll}
  \toprule
  \textbf{Wert} & \textbf{Bezeichner}
    & \textbf{Deutsch} & \textbf{Englisch} \tabularnewline
  \midrule
  diss & \@tempa{dissertationname}
  doctoral & \@tempa{dissertationname}
  phd & \@tempa{dissertationname}
  diploma & \@tempa{diplomathesisname}
  master & \@tempa{masterthesisname}
  bachelor & \@tempa{bachelorthesisname}
  student & \@tempa{studentresearchname}
  project & \@tempa{projectpapername}
  seminar & \@tempa{seminarpapername}
  research & \@tempa{researchname}
  log & \@tempa{logname}
  report & \@tempa{reportname}
  internship & \@tempa{internshipname}
  \bottomrule
\end{tabular}
\makeatother%
\end{table}
\end{Declaration}
\end{Declaration}

\begin{Declaration}{\Option{subjectthesis}[\PBoolean]}%
  [false][\Macro{subject}\Parameter{\autoref{tab:thesis}}:true]
\printdeclarationlist%
%
Der Befehl \Macro{thesis} dient den \TUDScript"=Hauptklassen zur Unterscheidung 
zweier unterschiedlichen Ausprägungen der Titelseite und ist im speziellen für 
Abschlussarbeiten gedacht. Außerdem kann bei der Verwendung spezieller Werte 
aus \autoref{tab:thesis} innerhalb des Argumentes von \Macro{subject} ebenfalls 
das Verhalten für Abschlussarbeiten aktiviert werden, wobei hierdurch die 
Einstellung \Option*{subjectthesis}[true] automatisch vorgenommen wird.

Für den Standardfall~-- bekanntlich \Option*{subjectthesis}[false]~-- wird der 
durch \Macro{thesis} gegebene Typ der Abschlussarbeit sowie der gegebenenfalls 
durch \Macro{graduation} gesetzte angestrebte Abschluss in großen Lettern und 
sehr zentral auf der Titelseite gesetzt. Die Verwendung von \Macro{subject} ist 
hierbei weiterhin möglich.
%
Wird die Option mit \Option*{subjectthesis}[true] aktiviert, so wird die mit 
\Macro{thesis} gesetzte Bezeichnung nicht unterhalb sondern oberhalb des Titels 
an der Stelle von \Macro{subject} ausgegeben. Der mit \Macro{graduation} 
angegebene Abschluss wird weiterhin unter dem Titel, allerdings in schlankerer 
Schrift gesetzt. Eine etwaige Verwendung des Befehls \Macro{subject} wird in 
diesem Fall ignoriert.
%
\begin{values}
\itemfalse
  Die Ausgabe des Typs der Abschlussarbeit (\Macro{thesis}) selbst sowie des 
  angestrebten Abschlusses (\Macro{graduation}) erfolgt in großen Lettern in 
  \DIN zentral auf der Titelseite.
\itemtrue*
  Der Typ der Abschlussarbeit (\Macro{thesis}) wird oberhalb des Titels in der 
  Betreffzeile gesetzt. Der angestrebte Abschluss (\Macro{graduation}) wird 
  zentral in der schlankeren \Univers ausgegeben.
\end{values}
\end{Declaration}

\Rename[macros]{v2.02}{\Macro{degree}}{\Macro{graduation}}
\begin{Declaration}{\Macro{graduation}\OParameter{Kurzform}\Parameter{Grad}}
\printdeclarationlist%
\index{Titel!Felder}%
%
Mit diesem Befehl wird der angestrebte akademische Grad auf der Titelseite 
ausgegeben. Da dies nur mit einer Abschlussarbeit erreicht werden kann erfolgt 
die Ausgabe nur, wenn entweder \Macro{thesis} oder \Macro{subject} verwendet 
wurde, wobei bei letzterem Befehl im Argument zwingend ein Wert aus 
\autoref{tab:thesis} verwendet werden muss.

Die Option \Option{subjectthesis} hat Einfluss auf die Ausgabe auf der 
Titelseite. Für die Einstellung \Option{subjectthesis}[false] wird der 
Abschuss~-- ähnlich wie 
der Typ der Abschlussarbeit~-- zentral und in relativ großen Lettern gesetzt. 
Für \Option{subjectthesis}[true] erfolgt die Ausgabe kleiner und in weniger 
starken Buchstaben.
\end{Declaration}

\begin{Declaration}{\Macro{supervisor}\Parameter{Name(n)}}
\begin{Declaration}{\Macro{referee}\Parameter{Name(n)}}
\begin{Declaration}{\Macro{advisor}\Parameter{Name(n)}}
\begin{Declaration}{\Macro{professor}\Parameter{Name}}
\printdeclarationlist%
\index{Titel!Felder}%
\index{Betreuer|?}\index{Gutachter|?}\index{Referent|?}%
%
Mit \Macro*{supervisor}, \Macro*{referee} und \Macro*{advisor} werden die 
Betreuer einer Abschlussarbeit beziehungsweise die Gutachter und Fachreferenten 
einer Dissertation angegeben. Zusätzlich kann mit \Macro*{professor} der 
betreuende Hochschullehrer beziehungsweise die betreuenden Professoren für 
studentische Arbeiten angegeben werden. Die Angabe mehrerer Person erfolgt wie 
beim Befehl \Macro{author} durch die Trennung mittels \Macro{and}.
\end{Declaration}
\end{Declaration}
\end{Declaration}
\end{Declaration}

\begin{Declaration}{\Macro{date}\OParameter{Ergänzung}\Parameter{Datum}}
\begin{Declaration}{\Macro{defensedate}\Parameter{Verteidigungsdatum}}
\printdeclarationlist%
\index{Titel!Felder}
\index{Datum|?}\index{Datum!Verteidigungsdatum|?}%
%
Mit \Macro*{date} kann das Datum angegeben werden. Das optionale Argument 
erlaubt eine zusätzliche Anmerkung, welche nach dem Datum ausgegeben wird. Das 
Datum wird bei normalen Dokumenten direkt nach dem Autor beziehungsweise den 
Autoren ausgegeben. Bei Abschlussarbeiten~-- aktiviert durch die Verwendung von 
\Macro{thesis} uder \Option{subjectthesis}~-- erscheint dieses am Ende der 
Titelseite als Abgabedatum. Außerdem kann in diesem Fall mit  dem Befehl
\Macro*{defensedate} das Datum der Verteidigung angegeben werden, wie es 
beispielsweise bei dem Druck von Dissertationen üblich ist.

Sollte das Paket \Package{isodate} geladen sein, so wird die damit eingestellte 
Formatierung des Datums durch den Befehl \Macro{printdate} aus diesem Paket für 
alle Datumsfelder des Dokumentes und folglich auch für die beiden Felder 
\Macro*{date} und \Macro*{defensedate} verwendet.
\end{Declaration}
\end{Declaration}
\index{Titel|!)}


\subsection{Die Teileseite}
\label{sec:part}
%
\ChangedAt{%
  v2.02!\Macro{partpagestyle}: \PageStyle{plain.tudheadings} wird verwendet%
}
Wird für die Teileseiten das Layout des \CDs verwendet, so wird der Seitenstil 
dieser (\Macro*{partpagestyle}) auf \PageStyle{plain.tudheadings} gesetzt.

\begin{Declaration}{\Option{parttitle}[\PBoolean]}[false]%
\printdeclarationlist%
\index{Teileseiten|?}\index{Layout!Teileseiten}%
%
Diese Option ermöglicht es, den mit \Macro{title} gegebenen Titel des 
Dokumentes selbst in großer Schrift auf einer Teileseite auszugeben, die 
Bezeichnung des mit \Macro{part}\Parameter{Bezeichnung} erzeugten Teils wird 
in diesem Fall in kleiner Schrift direkt darunter gesetzt. Diese 
Layout"=Variante findet sich im Handbuch für das \CD der \TnUD. \notudscrartcl
%
\begin{values}
\itemfalse
  Die Bezeichnung des Teils erscheint in großer Schrift auf der Seite, der 
  Titel des Dokumentes gar nicht.
\itemtrue*
  Der Titel wird in großer Auszeichnung auf der Teileseite gesetzt, die 
  Bezeichnung des Teils selber in kleinerer.
\end{values}
\end{Declaration}


\subsection{Die Kapitelseite}
\begin{Declaration}{\Option{chapterpage}[\PBoolean]}%
  [false][\Option{cd}[color]:true]%
\printdeclarationlist%
\label{sec:chapter}%
\index{Kapitelseiten|?}\index{Layout!Kapitelseiten|?}%
\index{Satzspiegel!doppelseitig}\index{Leerseiten}%
%
Mit dieser Einstellung kann die Überschrift eines Kapitels separat auf einer 
Seite ausgegeben werden. Der nachfolgende Text wird auf der nächsten 
beziehungsweise bei doppelseitigem Satz und rechts öffnenden Kapiteln%
\footnote{%
  \Option{twoside} und \Option{open}[right], Standard für \Class{tudscrbook}
}
auf der übernächsten Seite ausgegeben. Die in diesem Fall erzeugte Rückseite 
wird in ihrer Ausprägung~-- wie auch Teileseiten~-- durch die Einstellung von 
\Option{cleardoublespecialpage} bestimmt. Beim farbigen Layout ist diese Option 
standardmäßig aktiviert. \notudscrartcl
%
\begin{values}
\itemfalse
  Es gibt keine Sonderstellung von Kapiteln, der nachfolgende Text wird direkt 
  unter der Überschrift auf der gleichen Seite ausgegeben.
\itemtrue*
  Die Kapitelüberschrift wird auf einer separaten Seite gesetzt, der folgende
  Text wird erst auf der nächsten beziehungsweise übernächsten Seite 
  ausgegeben. Siehe dazu auch die Option \Option{cleardoublespecialpage}.
\end{values}
%
\ChangedAt{v2.02}
Der Seitenstil von Kapiteln lässt sich übrigens~-- unabhängig von der Option 
\Option*{chapterpage}~-- ändern, indem der Befehl \Macro*{chapterpagestyle} 
umdefiniert wird. Außerdem können mit der Option \Option{chapterprefix} 
Kapitelüberschriften mit einer Vorsatzzeile aktiviert werden. Dabei wird 
zunächst in einer Zeile \enquote{Kapitel} gefolgt von der Kapitelnummer 
ausgegeben, in der nächsten Zeile wird anschließend die Überschrift in 
linksbündigem Flattersatz ausgegeben. Insbesondere bei der Verwendung von 
separaten Kapitelseiten ist die Nutzung dieser Option empfehlenswert. Genaueres 
hierzu ist in der \KOMAScript"=Dokumentation nachzulesen.
\end{Declaration}


\subsection{Satzspiegel und Kolumnentitel}
\begin{Declaration}{\Option{geometry}[\PSet]}[true]%
\printdeclarationlist%
\index{Seitenstil}\index{Layout!Seitenstil}%
\index{Satzspiegel}\index{Satzspiegel!doppelseitig}\index{Layout!Satzspiegel}%
\index{Layout!Seitenränder}%
%
Diese Option ist für die Aufteilung beziehungsweise die Berechnung des 
Satzspiegels verantwortlich. Das Maß der Seitenränder ist im \CD fest 
vorgegeben und wird standardmäßig von den \TUDScript-Klassen eingehalten. 
Allerdings lassen sich die Seitenränder anpassen, um beispielsweise einen 
vernünftigen doppelseitigen Satz zu ermöglichen.%
\footnote{Hierbei sollte der innere Rand schmaler als der äußere sein}
Des Weiteren besteht die Möglichkeit, auf das Standardverhalten von 
\KOMAScript{} zurückzufallen und die Satzspiegelberechnung durch das Paket
\Package{typearea} vornehmen zu lassen. Hier hat insbesondere die Klassenoption 
\Option{DIV}[\PSet] maßgeblichen Einfluss auf den Satzspiegel. Siehe dazu die 
Dokumentation von \KOMAScript{}.
%
\begin{values}
\itemfalse
  Die Satzspiegelberechnung erfolgt via \Package{typearea}, die Vorgaben des 
  \CDs bezüglich der Seitenränder werden ignoriert.
\itemtrue*[tud/cd/asymmetric]
  Die Seitenränder werden im asymmetrischen Stil des \CDs fest definiert und 
  auch für den doppelseitigen Satz (\Option{twoside}[true]) genutzt.%
  \footnote{links: 30\,mm, rechts: 20\,mm, oben: 25\,mm, unten: 30\,mm}
\item[symmetric/centred/centered]
  Der Satzspiegel wird im einseitigen sowie doppelseitigen Satz auf der Seite 
  zentriert.%
  \footnote{links: 25\,mm, rechts: 25\,mm, oben: 25\,mm, unten: 30\,mm}
\item[balanced/twoside]
  Im einseitigen Layout ist das Verhalten der Einstellung identisch zu
  \Option*{geometry}[symmetric]. Beim doppelseitigen Satz wird der Satzspiegel 
  derart verändert, dass die Ränder der inneren Seiten schmaler sind als die 
  der äußeren.%
  \footnote{innen: 20\,mm, außen: 30\,mm, oben: 25\,mm, unten: 30\,mm}
  \Attention{%
    Der so erzeugte Satzspiegel ist allerdings nicht sehr vorteilhaft. Es ist 
    zu beachten, dass dabei das Logo der \TnUD sehr nah am inneren Seitenrand 
    des Dokumentes gesetzt wird, folglich insbesondere auf rechten respektive 
    ungeraden Seiten sehr weit an den Blattrand rückt.
  }
\end{values}
%
Für die Festlegung der Seitenränder wird das Paket \Package{geometry} 
verwendet. Ist \Option*{geometry}[false] gewählt, erfolgt die Berechnung des 
Satzspiegels durch \Package{typearea}. Die damit berechneten Werte werden 
anschließend an \Package{geometry} weitergereicht und durch dieses umgesetzt.
\end{Declaration}

\subsubsection{Bindekorrektur}
\index{Bindekorrektur|!}\index{Layout!Bindekorrektur}%
%
Zu erwähnen im Zusammenhang mit Seitenrändern und Satzspiegel ist die durch 
\Package{typearea} angebotene Option \Option{BCOR}[\PName{Länge}], mit der bei 
der Satzspiegelberechnung ein Heftrand beziehungsweise eine Bindekorrektur 
berücksichtigt wird. Die \TUDScript-Klassen reichen diesen Wert auch an 
\Package{geometry} weiter, so dass der Benutzer unabhängig von der Auswahl zur 
Satzspiegelgestaltung diese Option nutzen kann. So kann beispielsweise eine 
Bindekorrektur von \unit[5]{mm} mit der Klassenoption \Option{BCOR}[5mm] 
gesetzt werden.

Eine Anpassung der Bindekorrektur hat natürlich \emph{immer} eine Änderung der 
verfügbaren Breite des Textbereichs zur Folge hat und führt somit zwingend zu 
einer Anpassung des Satzspiegels. Da die Bindekorrektur jedoch abhängig von der 
Höhe des Buchblocks gewählt werden sollte, welche letztendlich erst mit dem 
Druck des fertiggestellten Dokumentes bestimmt werden kann, muss diese zu 
Beginn abgeschätzt werden.
%
\begin{Example}
Als Faustregel gilt, dass die erforderliche Bindekorrektur in etwa der halben 
Höhe des Buchblocks entsprechen sollte. Dessen Höhe wiederum ist abhängig von 
der Anzahl der Seiten sowie der Dichte des verwendeten Papiers. Wird normales 
Papier mit einer Dichte von \unit[80]{g/m²} verwendet, so entsprechen 100~Blatt 
in etwa einer Höhe von \unit[10]{mm}, bei \unit[100]{g/m²} ca. \unit[12]{mm}. 
Dementsprechend wäre eine Bindekorrektur von \Option{BCOR}[5mm] beziehungsweise 
\Option{BCOR}[6mm] bei diesem Beispiel zu wählen.
\end{Example}

\subsubsection{Kopf"~ und Fußzeile im Zusammenspiel mit dem Satzspiegel}
\index{Kopfzeile|!}\index{Layout!Kopfzeile}%
\index{Fußzeile|!}\index{Layout!Fußzeile}%
Da im \CD nicht festgelegt ist, wie die Gestaltung der Kopf"~ und Fußzeilen in 
einer wissenschaftlichen Arbeit auszuführen ist, bleibt dem Nutzer dafür eine 
gewisse Freiheit. Dafür sollte idealerweise das zu \KOMAScript{} gehörige Paket 
\Package{scrlayer-scrpage} genutzt werden. 

In der Dokumentation zu \Package{typearea} wird auch darauf eingegangen, wann 
Kopf"~ und Fußzeile bei der Satzspiegelkonstruktion entweder dem Rand oder dem 
Textkörper zugeschlagen werden sollten. Dies sollte bei der Erstellung eigener 
Kopf"~ und Fußzeilen beachtet werden. Die Einstellung dafür erfolgt mit den 
beiden \KOMAScript"=Optionen \Option{headinclude}[\PBoolean] sowie 
\Option{footinclude}[\PBoolean]. Diese können~-- unabhängig von der gewählten 
Einstellung zur Satzspiegelgestaltung über \Option{geometry}~-- verwendet 
werden.

\begin{Declaration}{\Option{cdfoot}[\PBoolean]}[false]%
\printdeclarationlist%
\index{Kolumnentitel}\index{Layout!Kolumnentitel}
\index{Satzspiegel!doppelseitig}%

Eine Möglichkeit zur Gestaltung der Kolumnentitel zeigt das Handbuch für das 
\CD der \TnUD. Dieses wird ohne Kopf"~ und mit einer einfachen Fußzeile 
gesetzt. Diese enthält dabei den aktuellen Kolumnentitel sowie die Paginierung. 
Eine derartige Ausprägung ist nicht explizit durch das \CD vorgegeben, wurde 
jedoch innerhalb der alten \Class{tudbook}"=Klasse exakt so umgesetzt.

Die neuen \TUDScript-Klassen sind~-- insbesondere aufgrund der Möglichkeit zur 
Verwendung des Paketes \Package{scrlayer-scrpage}~-- bei der Gestaltung der 
Kopf"~ und Fußzeilen wesentlich flexibler. Dennoch kann mit dieser Option das 
beschriebene Verhalten aktiviert werden. Hierbei wird beim doppelseitigen Satz 
(\Option{twoside}[true]) die Seitenzahl außen gesetzt.
%
\begin{values}
\itemfalse
  Die Kopf"~ und Fußzeilen zeigen Standardverhalten, zur manuellen Änderung 
  dieser sollte unbedingt das \KOMAScript"=Paket \Package{scrlayer-scrpage} 
  verwendet werden.
\itemtrue*
  Die Kopf"~ und Fußzeilen des Dokumentes werden wie im Handbuch des \CDs der 
  \TnUD beziehungsweise der \Class{tudbook}"=Klasse gesetzt.
\end{values}
%
Der Inhalt der Kolumnentitel kann durch den Anwender frei gewählt werden. Wird 
die Klassenoption \Option{automark} angegeben, werden für das automatische 
Setzen der Marken die Titel der Gliederungsebenen verwendet. Genaueres hierzu 
sowie der Möglichkeit, die Kolumnentitel manuell festzulegen, ist dem Handbuch 
von \KOMAScript{} zu entnehmen.
\end{Declaration}


\subsection{Die Farben des \CDs}
\index{Farben}%
% 
Zur Verwendung der Farben des \CDs wird das Paket \Package{tudscrcolor} 
genutzt. Falls dieses nicht in der Präambel geladen wird~-- um beispielsweise 
zusätzliche Optionen aufzurufen~-- binden die \TUDScript"=Klassen dieses 
automatisch ein. Detaillierte Informationen sind in der Dokumentation von 
\Package'{tudscrcolor} zu finden.



\section{Zusätzliche Optionen und Erweiterungen}
Neben den Befehlen für die Anpassung des Layouts an das \CD der \TnUD stellen 
die \TUDScript-Klassen weitere Befehle und Umgebungen zur Verfügung, um die 
Anwendung insbesondere für wissenschaftliche Arbeiten zu erleichtern.


\subsection{Zusammenfassung}
\begin{Declaration}[%
  v2.02!Wert \PValue{double} mit \PValue{multi} ersetzt%
]{\Option{abstract}[\PSet]}%
\printdeclarationlist%
\index{Zusammenfassung|!(}%
\index{Zweispaltensatz}%
%
Diese Option wird bereits durch \KOMAScript{} für die Klassen \Class{scrartcl} 
und \Class{scrreprt} standardmäßig bereitgestellt. Für die Klasse 
\Class{scrbook} geschieht dies nicht. Dazu heißt es im Handbuch:
%
\begin{quoting}
Bei Büchern wird in der Regel eine andere Art der Zusammenfassung verwendet. 
Dort setzt man ein entsprechendes Kapitel an den Anfang oder Ende des Werks. 
Oft wird diese Zusammenfassung entweder mit der Einleitung oder einem weiteren 
Ausblick verknüpft. Daher gibt es bei \Class{scrbook} generell keine 
\Environment{abstract}"=Umgebung. Bei Berichten im weiteren Sinne, etwa einer 
Studien- oder Diplomarbeit, ist ebenfalls eine Zusammenfassung in dieser Form 
zu empfehlen.
\end{quoting}
%
Durch die \TUDScript-Klassen wird die \Option*{abstract}"=Option erweitert. 
Neben der standardmäßigen Auswahlmöglichkeit innerhalb der Klassen 
\Class{tudscrartcl} und \Class{tudscrreprt}, ob keine oder eine kleine und 
zentrierte Überschrift innerhalb der \Environment{abstract}"=Umgebung gesetzt 
werden soll, kann die Überschrift für die Zusammenfassung außerdem in Gestalt 
eines Unterkapitels oder für die Klassen \Class{tudscrreprt} und 
\Class{tudscrbook} in der Form eines Kapitels ausgegeben werden.

Abhängig von der gewählten Gliederungsebene der Überschrift wird das 
Standardverhalten für das Setzen eines Eintrages ins Inhaltsverzeichnis 
festgelegt. Ohne oder mit zentrierter Überschrift (\Option*{abstract}[true]) 
wird per Voreinstellung kein Eintrag im Inhaltsverzeichnis erzeugt. Wird die 
Überschrift jedoch in Form einer Gliederungsebene gewählt, so erscheint die 
Zusammenfassung für gewöhnlich im Inhaltsverzeichnis auf der obersten Ebene. 
Dieses Verhalten kann jederzeit mit der Option \Option*{abstract}[toc/notoc] 
durch den Anwender überschrieben werden.
%
\begin{values}
\itemfalse[][nur für \Class*{tudscrartcl} und \Class*{tudscrreprt} verfügbar]
  Es wird keine Überschrift für die \Environment{abstract}"=Umgebung ausgegeben.
\itemtrue*[][nur für \Class*{tudscrartcl} und \Class*{tudscrreprt} verfügbar]
  Wie bei den \KOMAScript"=Klassen wird eine zentrierte Überschrift mit dem 
  Bezeichner \Term{abstractname} vor der eigentlichen Zusammenfassung gesetzt.
\item[section]
  Die Überschrift verwendet den Gliederungsbefehl \Macro{addsec}.
\item[chapter][%
    (Säumniswert für \Class*{tudscrbook})
    nur für \Class*{tudscrreprt} und \Class*{tudscrbook} verfügbar%
  ]
  Es wird der Befehl \Macro{addchap} für das Setzen der Überschrift genutzt. 
\item[heading]
  Es wird die höchstmögliche Gliederungsebene verwendet. Für 
  \Class{tudscrartcl} entspricht dies \Option*{abstract}[section], bei 
  \Class{tudscrreprt} und \Class{tudscrbook} \Option*{abstract}[chapter].
\item[toc/totoc]
  Unabhängig von der Wahl der Überschrift erhält die Zusammenfassung einen nicht
  nummerierten Eintrag im Inhaltsverzeichnis auf der obersten Gliederungsebene. 
\item[notoc/nottotoc]
  Die Zusammenfassung wird definitiv nicht ins Inhaltsverzeichnis eingetragen.
\end{values}
%
Häufig wird für Abschlussarbeiten verlangt, neben der deutschsprachigen 
auch noch eine englischsprachige Zusammenfassung zu verfassen. Es kann 
eingestellt werden, beide zusammen auf einer Seite auszugeben~-- sofern 
genügend Platz vorhanden ist. Außerdem kann die standardmäßige vertikale 
Zentrierung der \Environment{abstract}"=Umgebung auf einer Seite unterdrückt 
werden. Damit kann der Anwender gegebenenfalls die Positionierung selbstständig 
vornehmen. 
%
\begin{values}
\item[one/simple/single]Jede Zusammenfassung wird auf einer eigenen Seite
  beziehungsweise im zweispaltigen Satz in einer neuen Spalte ausgegeben.
\item[multi/multiple/all]
  \ChangedAt{v2.02}
  Zusammenfassungen, welche mit \Macro{nextabstract} getrennt wurden, werden 
  direkt nacheinander auf der gleichen Seite ausgegeben, wenn ausreichend Platz 
  auf dieser vorhanden sein sollte. Ist die Option \Option{twocolumn} aktiviert,
  erfolgt die Ausgabe der aller Zusammenfassungen ohne Spaltenumbruch.
\item[nofil/nofill/novfil/novfill]
  Die Ausgabe erfolgt wie im normalen Textsatz auch.
\item[fil/fill/vfil/vfill]
  Alle Zusammenfassungen auf einer Ausgabeseite werden vertikal zentriert. Für 
  den zweispaltigen Satz steht diese Option nicht zur Verfügung.
\end{values}
Es ist zu beachten, dass die zuvor genannten Einstellungen zur Positionierung 
der Zusammenfassungen mit \Option{abstract}[simple/multiple/fill/nofill] 
innerhalb der \Environment*{abstract}"=Umgebung nur wirksam sind, wenn eine 
Titelseite (\Option{titlepage}[true]) und \emph{keine} Überschriften in Form 
von Kapiteln (\Option{abstract}[chapter]) verwendet werden.
\end{Declaration}

\begin{Declaration}[%
  v2.02!\Macro*{nextabstract} zur Trennung der einzelnen Teile%
]{\Environment{abstract}[\OLParameter{Sprache}]}
\begin{Declaration}[v2.02]{\Macro{nextabstract}\OLParameter{Sprache}}
\begin{Declaration}{\Key{\Environment{abstract}}{language}[\PName{Sprache}]}
\begin{Declaration}[v2.02]{%
  \Key{\Environment{abstract}}{pagestyle}[\PName{Seitenstil}]%
}
\begin{Declaration}{\Key{\Environment{abstract}}{columns}[\PName{Anzahl}]}
\begin{Declaration}{\Key{\Environment{abstract}}{option}[\PSet]}
\printdeclarationlist%
\index{Zweispaltensatz}%
%
Diese Umgebung dient speziell für die Ausgabe einer Zusammenfassung. Mit der 
Option \Option{abstract} kann eingestellt werden, in welcher Gestalt diese 
ausgegeben werden soll. Wird keine Titelseite sondern ein Titelkopf verwendet 
(\Option{titlepage}[false]), so wird für den Fall, dass die Zusammenfassung 
\emph{nicht} mit einer Überschrift einer Gliederungsebene gesetzt wird, diese 
wie bei den \KOMAScript"=Klassen in einer \Environment{quotation}"=Umgebung 
gesetzt, um die Zusammenfassung abzuheben. Diese hat jedoch den Nachteil, dass 
in dieser die Option \Option{parskip} nicht beachtet wird. Um dieses Problem zu 
beheben, kann das Paket \Package{quoting} geladen werden, wodurch stattdessen 
die \Environment{quoting}"=Umgebung verwendet wird.

Zusätzlich können weitere Parameter als optionales Argument angegeben werden. 
Wird das Paket \Package{babel} verwendet, kann mit dem Parameter 
\Key*{\Environment{abstract}}{language}[\PName{Sprache}] die Sprache innerhalb 
der \Environment{abstract}"=Umgebung geändert werden. Dafür muss die gewünschte 
Sprache bereits mit dem Laden von \Package{babel} entweder als Paketoption oder 
besser noch als Klassenoption angegeben worden sein. Dadurch werden lokal 
innerhalb der Umgebung die Bezeichnung \Term{abstractname} und die 
Trennungsmuster sprachspezifisch angepasst. Die gewünschte Sprache kann auch 
direkt und ohne den Parameter \Key*{\Environment{abstract}}{language} als 
optionales Argument übergeben werden.

Wurde das Paket \Package{multicol} geladen, kann mit dem Parameter 
\Key*{\Environment{abstract}}{columns}[\PName{Anzahl}] die Zusammenfassung 
mehrspaltig gesetzt werden.
\ChangedAt{v2.02}
Der für die Umgebung zu verwendende Seitenstil kann mit dem Parameter 
\Key*{\Environment{abstract}}{pagestyle} angegeben werden. Dieser unterstützt 
auch die \PageStyle{tudheadings}"=Seitenstile.

Dem Parameter \Key*{\Environment{abstract}}{option} können alle gültigen, 
bereits erläuterten Werte der Option \Option{abstract} übergeben werden. 
Die damit gemachten Einstellungen wirken sich~-- im Gegensatz zur Angabe als 
Klassenoption oder über die Variante der späten Optionenwahl%
\footnote{%
  \Macro{TUDoption}\PParameter{abstract}\Parameter{Einstellung} oder
  \Macro{TUDoptions}\PParameter{abstract=\PName{Einstellung}}
}~-- lediglich lokal auf die verwendete \Environment{abstract}"=Umgebung aus.

\ChangedAt{v2.02}
Sollen mehrere Zusammenfassungen im gleichen Stil erzeugt und die Einstellungen 
der Option \Option{abstract}[simple/multiple/fill/nofill] beachtet werden, so 
ist die \Environment*{abstract}"=Umgebung nur einmal zu verwenden. Innerhalb 
dieser müssen die einzelnen Zusammenfassungen mit \Macro*{nextabstract} 
voneinander getrennt werden. Der Befehl akzeptiert dabei im optionalen Argument 
alle Parameter, die auch von der \Environment*{abstract}"=Umgebung selbst 
unterstützt werden. Das Minimalbeispiel in \autoref{sec:exmpl:dissertation} 
zeigt hierfür das notwendige Vorgehen.
%
%Sollte die \Environment{abstract}"=Umgebung innerhalb des Argumentes der 
%Befehle \Macro{partpreamble} beziehungsweise \Macro{chapterpreamble} verwendet 
%werden, wird die Überschrift~-- im Fall, dass nicht \Option{abstract}[false] 
%gewählt ist~-- \emph{immer} in Textgröße und zentriert gesetzt.
\end{Declaration}
\end{Declaration}
\end{Declaration}
\end{Declaration}
\end{Declaration}
\end{Declaration}
\index{Zusammenfassung|!)}%


\subsection{Selbstständigkeitserklärung und Sperrvermerk}
\begin{Declaration}[%
  v2.02!Wert \PValue{double} mit \PValue{multi} ersetzt%
]{\Option{declaration}[\PSet]}[true]%
\printdeclarationlist%
\index{Selbstständigkeitserklärung|!}\index{Sperrvermerk|!}%
%
Mit dieser Einstellung kann äquivalent zur Option \Option{abstract} die 
Gestaltung von Selbstständigkeitserklärung und Sperrvermerk angepasst werden.
Zur Ausgabe der Erklärungen werden die Umgebung \Environment{declarations} 
sowie die Befehle \Macro{declaration} beziehungsweise \Macro{confirmation} und 
\Macro{blocking} bereitgestellt. 

Abhängig von der gewählten Gliederungsebene der Überschrift wird das 
Standardverhalten für das Setzen eines Eintrages ins Inhaltsverzeichnis 
festgelegt. Ohne oder mit zentrierter Überschrift (\Option*{declaration}[true]) 
wird per Voreinstellung kein Eintrag im Inhaltsverzeichnis erzeugt. Wird die 
Überschrift jedoch in Form einer Gliederungsebene gewählt, so erscheint die 
Erklärung für gewöhnlich im Inhaltsverzeichnis auf der obersten Ebene. Dieses 
Verhalten kann jederzeit mit der Option \Option*{declaration}[toc/notoc] durch 
den Anwender überschrieben werden.
%
\begin{values}
\itemfalse
  Es wird keine Überschrift über den Erklärungen selbst ausgegeben.
\itemtrue*
  Eine zentrierte Überschrift mit dem Bezeichner \Term{confirmationname} vor 
  der Selbstständigkeitserklärung beziehungsweise \Term{blockingname} vor dem 
  Sperrvermerk wird gesetzt. 
\item[section]
  Die Überschrift verwendet den Gliederungsbefehl \Macro{addsec}.
\item[chapter]
  Es wird der Befehl \Macro{addchap} für das Setzen der Überschrift genutzt. 
\item[heading]
  Es wird die höchstmögliche Gliederungsebene verwendet. Für 
  \Class{tudscrartcl} entspricht dies \Option*{declaration}[section], bei 
  \Class{tudscrreprt} und \Class{tudscrbook} \Option*{declaration}[chapter].
\item[toc/totoc]
  Unabhängig von der Wahl der Überschrift erhält die Erklärung einen nicht
  nummerierten Eintrag im Inhaltsverzeichnis auf der obersten Gliederungsebene. 
\item[notoc/nottotoc]
  Die Erklärung wird definitiv nicht ins Inhaltsverzeichnis eingetragen.
\end{values}
%
Die folgenden Einstellungen haben lediglich Auswirkungen, wenn die Überschrift 
der Erklärung \emph{nicht} im Form eines Kapitels ausgegeben und eine 
Titelseite (\Option{titlepage}[true]) verwendet wird.
%
\begin{values}
\item[one/simple/single]Jede Erklärung wird auf einer separaten Seite
  beziehungsweise im zweispaltigen Satz in einer neuen Spalte ausgegeben.
\item[multi/multiple/all]
  \ChangedAt{v2.02}
  Erklärungen, welche in der \Environment{declarations}"=Umgebung mit den 
  Befehlen \Macro{confirmation}, \Macro{blocking} und \Macro{declaration} oder 
  außerhalb dieser mit \Macro{declaration} gesetzt wurden, werden direkt 
  nacheinander auf der gleichen Seite ausgegeben, wenn ausreichend Platz auf 
  dieser vorhanden sein sollte. Ist die Option \Option{twocolumn} aktiviert, 
  erfolgt die Ausgabe ohne Spaltenumbruch.
\item[nofil/nofill/novfil/novfill]
  Die Ausgabe erfolgt wie im normalen Textsatz auch.
\item[fil/fill/vfil/vfill]
  Alle Erklärungen auf einer Ausgabeseite werden vertikal zentriert. Für 
  den zweispaltigen Satz steht diese Option nicht zur Verfügung.
\end{values}
%
Es ist zu beachten, dass die Einstellungen zur Positionierung der Erklärungen 
mit der Option \Option{declaration}[simple/multiple/fill/nofill] innerhalb der 
Umgebung \Environment*{declarations} nur wirksam sind, wenn eine Titelseite 
(\Option{titlepage}[true]) und \emph{keine} Überschriften in Form von Kapiteln 
(\Option{declaration}[chapter]) verwendet werden.
\end{Declaration}

\begin{Declaration}[v2.02]{\Environment{declarations}[\LParameter]}
\begin{Declaration}{\Key{\Environment{declarations}}{language}[\PName{Sprache}]}
\begin{Declaration}[v2.02]{%
  \Key{\Environment{declarations}}{pagestyle}[\PName{Seitenstil}]%
}
\begin{Declaration}[v2.02]{%
  \Key{\Environment{declarations}}{columns}[\PName{Anzahl}]%
}
\begin{Declaration}{\Key{\Environment{declarations}}{option}[\PSet]}
\begin{Declaration}{%
  \Key{\Environment{declarations}}{supporter}[\PName{Unterstützer}]
}
\begin{Declaration}{\Key{\Environment{declarations}}{place}[\PName{Ort}]}
\begin{Declaration}{\Key{\Environment{declarations}}{closing}[\PName{Ende}]}
\begin{Declaration}{\Key{\Environment{declarations}}{company}[\PName{Firma}]}

\printdeclarationlist%
\index{Selbstständigkeitserklärung}\index{Sperrvermerk}%
%
Innerhalb dieser Umgebung können Selbstständigkeitserklärung und Sperrvermerk 
mit dem Befehl \Macro{declaration} direkt nacheinander folgend beziehungsweise 
mit \Macro{confirmation} und \Macro{blocking} auch separat ausgegeben werden. 
Dies kann in beliebiger Reihenfolge und auch mehrmals geschehen, um diese 
beispielsweise mehrsprachig zu setzen. Die im Folgenden beschriebenen Parameter 
können dabei sowohl für die \Environment*{declarations}"=Umgebung selbst als 
auch für die zuvor genannten Befehle als optionales Argument verwendet werden.

Mit \Key*{\Environment{declarations}}{language} kann die Sprache der Erklärung 
geändert werden. Der verwendete Seitenstil lässt sich mit dem Parameter 
\Key*{\Environment{declarations}}{pagestyle}[\PName{Seitenstil}] angeben. 
Dieser unterstützt auch die \PageStyle{tudheadings}"=Seitenstile. Mit dem 
Parameter \Key*{\Environment{declarations}}{columns}[\PName{Anzahl}] kann die 
Erklärung mehrspaltig gesetzt werden, wenn das Paket \Package{multicol} geladen 
wurde.

Die Verwendung der Parameter \Key{\Macro{confirmation}}{supporter} sowie
\Key{\Macro{confirmation}}{place} und \Key{\Macro{confirmation}}{closing} ist 
in der Dokumentation des Befehls \Macro{confirmation} zu finden, der Parameter 
\Key{\Macro{blocking}}{company} ist für \Macro{blocking} erläutert. Für den 
Parameter \Key*{\Environment{declarations}}{option} können alle gültigen Werte 
der Option \Option{declaration} angegeben werden. Diese wirken sich nur lokal 
innerhalb der \Environment*{declarations}"=Umgebung aus.
\end{Declaration}
\end{Declaration}
\end{Declaration}
\end{Declaration}
\end{Declaration}
\end{Declaration}
\end{Declaration}
\end{Declaration}
\end{Declaration}

\begin{Declaration}{\Macro{declaration}\LParameter}
\begin{Declaration}{\Key{\Macro{declaration}}{language}[\PName{Sprache}]}
\begin{Declaration}[v2.02]{%
  \Key{\Macro{declaration}}{pagestyle}[\PName{Seitenstil}]%
}
\begin{Declaration}[v2.02]{\Key{\Macro{declaration}}{columns}[\PName{Anzahl}]}
\begin{Declaration}{\Key{\Macro{declaration}}{option}[\PSet]}
\begin{Declaration}{\Key{\Macro{declaration}}{supporter}[\PName{Unterstützer}]}
\begin{Declaration}{\Key{\Macro{declaration}}{place}[\PName{Ort}]}
\begin{Declaration}{\Key{\Macro{declaration}}{closing}[\PName{Ende}]}
\begin{Declaration}{\Key{\Macro{declaration}}{company}[\PName{Firma}]}
\printdeclarationlist%
\index{Selbstständigkeitserklärung}\index{Sperrvermerk}%
%
Dieser Befehl gibt die Selbstständigkeitserklärung und den Sperrvermerk direkt 
aufeinanderfolgend aus. Dabei werden die Einstellungen zur Positionierung der 
einzelnen Erklärungen, welche über die Wertzuweisungen an die Option 
\Option{declaration}[simple/multiple/fill/nofill] erfolgen, beachtet. Er kann 
sowohl innerhalb der \Environment{declarations}"=Umgebung als auch außerhalb 
direkt im Dokument verwendet werden und akzeptiert im optionalen Argument dabei 
alle für die genannte Umgebung beschriebenen Parameter.
\end{Declaration}
\end{Declaration}
\end{Declaration}
\end{Declaration}
\end{Declaration}
\end{Declaration}
\end{Declaration}
\end{Declaration}
\end{Declaration}

\begin{Declaration}{\Macro{confirmation}\OLParameter{supporter}}
\begin{Declaration}{\Key{\Macro{confirmation}}{supporter}[\PName{Unterstützer}]}
\begin{Declaration}{\Key{\Macro{confirmation}}{place}[\PName{Ort}]}
\begin{Declaration}{\Key{\Macro{confirmation}}{closing}[\PName{Ende}]}
\begin{Declaration}{\Key{\Macro{confirmation}}{language}[\PName{Sprache}]}
\begin{Declaration}[v2.02]{%
  \Key{\Environment{confirmation}}{pagestyle}[\PName{Seitenstil}]%
}
\begin{Declaration}[v2.02]{%
  \Key{\Environment{confirmation}}{columns}[\PName{Anzahl}]%
}
\begin{Declaration}{\Key{\Macro{confirmation}}{option}[\PSet]}
\printdeclarationlist%
\index{Selbstständigkeitserklärung}\index{Datum}%
%
Mit diesem Befehl wird ein sprachspezifischer Standardtext für eine 
Selbstständigkeitserklärung ausgegeben, welcher in \Term{confirmationtext} 
gespeichert ist. Wie dieser angepasst beziehungsweise geändert werden kann, ist 
unter \autoref{sec:localization} zu finden. Er kann sowohl innerhalb der 
\Environment{declarations}"=Umgebung als auch außerhalb direkt im Dokument 
verwendet werden. 

Wird er in seiner ursprünglichen Form belassen, kann er im optionalen Argument 
über die deklarierten Parameter angepasst werden. Im Standardtext der 
Selbstständigkeitserklärung werden sowohl der Titel als auch der Typ der 
Abschlussarbeit~-- falls dieser mit \Macro{thesis} oder \Macro{subject} und 
einem speziellen Wert aus \autoref{tab:thesis} beziehungsweise mit der Option 
\Option{subjectthesis} angegeben wurde~-- aufgeführt. Über den Parameter 
\Key*{\Macro{confirmation}}{supporter} können weitere an der Arbeit beteiligte 
Personen angegeben werden. Dies ist auch mit dem Befehl \Macro{supporter} 
möglich, wenn dieser \emph{vor} \Macro*{confirmation} verwendet wird. Mehrere 
zu nennende Personen sind auch hier durch \Macro{and} zu trennen. Das Feld der 
Unterstützer kann auch mit dem bloßen optionalen Argument ohne die Angabe eines 
Parameters angepasst werden.

Nach dem eigentlichen Text der Selbstständigkeitserklärung wird der mit 
\Key*{\Macro{confirmation}}{place} beziehungsweise \Macro{place} angegebene Ort 
sowie das mit \Macro{date} eingestellte Datum ausgegeben. Als Voreinstellung 
ist für den Ort \enquote{Dresden} gewählt. Danach folgen~-- mit etwas 
vertikalem Freiraum für die notwendige Unterschrift~-- der Autor oder die 
Autoren, angegeben durch den Befehl \Macro{author}. Soll anstelle dessen etwas 
anderes nach dem Text der Selbstständigkeitserklärung gesetzt werden, kann dies 
mit dem Parameter \Key*{\Macro{confirmation}}{closing} oder zuvor mit dem 
Befehl \Macro{confirmationclosing} angepasst werden. Die Parameter 
\Key*{\Environment{declarations}}{language}, 
\Key*{\Environment{declarations}}{pagestyle}, 
\Key*{\Environment{declarations}}{columns} und 
\Key*{\Environment{declarations}}{option} entsprechen in ihrem Verhalten denen 
der \Environment{declarations}"=Umgebung.
\end{Declaration}
\end{Declaration}
\end{Declaration}
\end{Declaration}
\end{Declaration}
\end{Declaration}
\end{Declaration}
\end{Declaration}

\Rename[macros]{v2.02}{\Macro{restriction}}{\Macro{blocking}}
\begin{Declaration}{\Macro{blocking}\OLParameter{company}}
\begin{Declaration}{\Key{\Macro{blocking}}{company}[\PName{Firma}]}
\begin{Declaration}{\Key{\Macro{blocking}}{language}[\PName{Sprache}]}
\begin{Declaration}[v2.02]{%
  \Key{\Environment{blocking}}{pagestyle}[\PName{Seitenstil}]%
}
\begin{Declaration}[v2.02]{%
  \Key{\Environment{blocking}}{columns}[\PName{Anzahl}]%
}
\begin{Declaration}{\Key{\Macro{blocking}}{option}[\PSet]}
\printdeclarationlist%
\index{Sperrvermerk}%
%
Beim Sperrvermerk verhält es sich äquivalent zur Selbstständigkeitserklärung.
Es wird der in \Term{blockingtext} hinterlegte Standardtext in der gewählten 
Sprache ausgegeben. Dieser kann durch den Anwender geändert werden. Wie genau 
ist in \autoref{sec:localization} beschrieben. Der Befehl \Macro*{blocking} 
kann sowohl innerhalb der Umgebung \Environment{declarations} als auch 
außerhalb direkt im Dokument verwendet werden. 

In seiner ursprünglichen Definition, kann er im optionalen Argument über die 
deklarierten Parameter angepasst werden. Im Standardtext des Sperrvermerks 
werden sowohl der Titel als auch der Typ der Abschlussarbeit~-- falls dieser 
mit \Macro{thesis} oder \Macro{subject} und einem speziellen Wert aus 
\autoref{tab:thesis} beziehungsweise mit der Option \Option{subjectthesis} 
angegeben wurde~-- aufgeführt. Mit \Key*{\Macro{blocking}}{company}~-- oder 
\emph{vorher} mit dem Befehl \Macro{company}~-- kann zusätzlich eine im 
Sperrvermerk zu nennende Firma oder ähnliches angegeben werden. Dieses Feld 
kann auch direkt im optionalen Argument ohne die Verwendung eines Parameters 
gesetzt werden. Die weiteren Parameter 
\Key*{\Environment{declarations}}{language}, 
\Key*{\Environment{declarations}}{pagestyle}, 
\Key*{\Environment{declarations}}{columns} und 
\Key*{\Environment{declarations}}{option} entsprechen in ihrem Verhalten denen 
der \Environment{declarations}"=Umgebung.
\end{Declaration}
\end{Declaration}
\end{Declaration}
\end{Declaration}
\end{Declaration}
\end{Declaration}

\begin{Declaration}{\Macro{supporter}\Parameter{Unterstützer}}
\begin{Declaration}{\Macro{place}\Parameter{Ort}}
\begin{Declaration}{\Macro{confirmationclosing}\Parameter{Ende}}
\begin{Declaration}{\Macro{company}\Parameter{Firma}}
\printdeclarationlist%
\index{Selbstständigkeitserklärung}\index{Sperrvermerk}%
%
Diese Makros ändern~-- im Gegensatz zu den Parametern der bereits vorgestellten 
Befehle \Macro{confirmation} und \Macro{blocking}~-- die entsprechenden 
Feldwerte global für das gesamte Dokument. Genutzt werden kann dies 
beispielsweise wenn ein Erklärungstyp in unterschiedlichen Sprachen ausgegeben 
wird. Dann kann man sich mit diesen Makros die mehrfache Angabe eines 
Parameters sparen.
\end{Declaration}
\end{Declaration}
\end{Declaration}
\end{Declaration}


\subsection{Lesezeichen}
\begin{Declaration}{\Option{tudbookmarks}[\PBoolean]}[true]%
\begin{Declaration}{%
  \Macro{tudbookmark}\OParameter{Ebene}\Parameter{Text}\Parameter{Ankername}%
}%
\printdeclarationlist%
\index{Lesezeichen}%
\index{Umschlagseite}\index{Titel}\index{Inhaltsverzeichnis}%
\index{Aufgabenstellung}\index{Gutachten}\index{Aushang}%
%
Diese Option wird wirksam, wenn \Package{hyperref} geladen wurde. Es werden für 
die Umschlag- und Titelseite, das Inhaltsverzeichnis sowie~-- bei der 
Verwendung des Paketes \Package{tudscrsupervisor}~-- die Aufgabenstellung 
jeweils Lesezeichen oder auch Outline"=Einträge im PDF-Dokument erzeugt.
%
\begin{values}
\itemfalse
  Es erfolgt kein Eintrag von ergänzenden Lesezeichen.
\itemtrue*
  Es werden automatisch zusätzliche Lesezeichen eingetragen.
\end{values}
%
Der Befehl \Macro*{tudbookmark} arbeitet wie \Macro{pdfbookmark} aus 
\Package{hyperref} mit dem Unterschied, dass die Lesezeichen nur generiert 
werden, wenn die Option \Option*{tudbookmarks} aktiviert ist.
\end{Declaration}
\end{Declaration}



\section{Sprachabhängige Bezeichner}
\label{sec:localization}
\index{Bezeichner|!(}%
%
Durch \KOMAScript{} werden Befehle, mit denen sprachabhängige Bezeichner 
erzeugt oder geändert werden können, zur Verfügung gestellt. Diese werden durch 
das \TUDScript-Bundle genutzt, um lokalisierte Begriffe für die Sprachen 
Englisch und Deutsch bereitzustellen. Ein Großteil davon betrifft Bezeichnungen 
für Felder auf der Titelseite (\autoref{sec:title}). Hierfür wird
\Macro{providecaptionname}\Parameter{Sprache}\Parameter{Makro}\Parameter{Inhalt}
verwendet, wobei \PName{Sprache} dem geladenen Sprachpaket~-- normalerweise das 
Paket \Package{babel}~-- bekannt sein muss.

Sollte der Anwender die im Folgenden erläuterten oder auch andere Bezeichner, 
welche von einem beliebigen (Sprach"~)Paket bereitgestellt werden, ändern 
wollen, ist hierfür der Befehl
\Macro{renewcaptionname}\Parameter{Sprache}\Parameter{Makro}\Parameter{Inhalt} 
zu verwenden. Seit der Version~v3.12 stellt \KOMAScript sicher, dass die 
Bezeichner erst \textbf{nach} \Macro*{begin}\PParameter{document} angepasst 
werden und somit nicht durch ein später geladenes Paket abermals geändert 
werden können. Es sollte natürlich dabei eine \PName{Sprache} angegeben werden, 
welche im Dokument durch \Package{babel} oder ein anderes Sprachpaket verwendet 
wird, beispielsweise \PValue{ngerman} oder \PValue{english}. 

Die Makros der Bezeichner und deren Verwendung werden folgend kurz beschrieben 
und tabellarisch aufgeführt. Dabei wurde versucht, alle Befehle der Bezeichner 
für bestimmte Begriffe auf \PValue{\dots{}name} und beschreibende Texte auf 
\PValue{\dots{}text} enden zu lassen.

\begin{Declaration}{\Term{dissertationname}}
\begin{Declaration}{\Term{diplomathesisname}}
\begin{Declaration}{\Term{masterthesisname}}
\begin{Declaration}{\Term{bachelorthesisname}}
\begin{Declaration}{\Term{studentresearchname}}
\begin{Declaration}{\Term{projectpapername}}
\begin{Declaration}{\Term{seminarpapername}}
\begin{Declaration}{\Term{researchname}}
\begin{Declaration}{\Term{logname}}
\begin{Declaration}{\Term{internshipname}}
\begin{Declaration}{\Term{reportname}}
\printdeclarationlist%
\index{Titel}\index{Abschlussarbeit}\index{Typisierung}%
%
Diese Bezeichner dienen zur Typisierung speziell für eine Abschlussarbeit. Wie 
diese genutzt werden können, ist bei der Erläuterung von \Macro{thesis} und 
\Macro'{subject} beziehungsweise in \autoref{tab:thesis} zu finden.
\TermTable{%
  dissertationname,diplomathesisname,masterthesisname,bachelorthesisname,%
  studentresearchname,projectpapername,seminarpapername,researchname,%
  logname,internshipname,reportname%
}
\end{Declaration}
\end{Declaration}
\end{Declaration}
\end{Declaration}
\end{Declaration}
\end{Declaration}
\end{Declaration}
\end{Declaration}
\end{Declaration}
\end{Declaration}
\end{Declaration}

\begin{Declaration}{\Term{supervisorname}}
\begin{Declaration}{\Term{supervisorothername}}
\begin{Declaration}[%
  v2.02!Unterscheidung zwischen einem und mehreren Gutachtern%
]{\Term{refereename}}
\begin{Declaration}{\Term{refereeothername}}
\begin{Declaration}{\Term{advisorname}}
\begin{Declaration}{\Term{advisorothername}}
\begin{Declaration}[%
  v2.02!Unterscheidung zwischen einem und mehreren Professoren%
]{\Term{professorname}}
\begin{Declaration}[v2.02]{\Term{professorothername}}
\printdeclarationlist%
\index{Titel}%
\index{Betreuer}\index{Gutachter}\index{Hochschullehrer}%
\index{Referent}%
%
Diese sprachabhängigen Begriffe sind die Bezeichner für die Titelseitenfelder 
von Betreuer (\Macro{supervisor}), Gutachter (\Macro{referee}) und Fachreferent 
(\Macro{advisor}). Soll innerhalb eines dieser Felder mehr als eine Person 
angegeben werden, so sind die Einzelpersonen jeweils mit dem Befehl \Macro{and} 
voneinander zu trennen. In diesem Fall werden alle nach der erstgenannten 
folgenden Personen durch den Bezeichner \PValue{\bsc\dots{}othername} ergänzt.

\ChangedAt{v2.02}
Bei der Bezeichnung des Gutachters wird übrigens unterschieden, ob einer oder 
mehrere angegeben wurden. Wird lediglich einer genannt, so ist eine 
Unterscheidung nicht notwendig. Werden jedoch zwei Gutachter angegeben, so 
werden diese auch mit Erst- und Zweitgutachter betitelt. Für den betreuenden 
Hochschullehrer (\Macro{professor}) wird änlich verfahren. Hier wird allerdings 
lediglich automatisch die Bezeichnung gegebenenfalls vom Singular in den Plural 
geändert.

\renewcaptionname{ngerman}{\refereename}{Gutachter/Erstgutachter}
\renewcaptionname{english}{\refereename}{Referee/First referee}
\TermTable{%
  supervisorname,supervisorothername,refereename,refereeothername,%
  advisorname,advisorothername,professorname,professorothername%
}
\end{Declaration}
\end{Declaration}
\end{Declaration}
\end{Declaration}
\end{Declaration}
\end{Declaration}
\end{Declaration}
\end{Declaration}

\begin{Declaration}{\Term{dateofbirthtext}}
\begin{Declaration}{\Term{placeofbirthtext}}
\begin{Declaration}{\Term{matriculationnumbername}}
\begin{Declaration}{\Term{matriculationyearname}}
\printdeclarationlist%
\index{Titel}\index{Autorenangaben}\index{Datum!Geburtsdatum}%
%
Werden für den Autor oder die Autoren das Geburtsdatum (\Macro{dateofbirth}), 
der Geburtsort (\Macro{placeofbirth}) sowie die
Matrikelnummer (\Macro{matriculationnumber}) und/oder das Immatrikulationsjahr 
(\Macro{matriculationyear}) angegeben, werden sowohl auf der Titelseite als 
auch auf der gegebenenfalls mit \Package{tudscrsupervisor} erstellten 
Aufgabenstellung die dazugehörigen Bezeichner vorangestellt. Auf dem Titel 
werden diese dabei mit dem durch \Macro{titledelimiter} gegebenen Trennzeichen 
vom eigentlichen Feld abgegrenzt.
\TermTable{%
  dateofbirthtext,placeofbirthtext,matriculationnumbername,%
  matriculationyearname%
}
\end{Declaration}
\end{Declaration}
\end{Declaration}
\end{Declaration}

\Rename[terms]{v2.02}{\Term{degreetext}}{\Term{graduationtext}}
\begin{Declaration}{\Term{graduationtext}}
\printdeclarationlist%
\index{Titel}\index{Abschlussarbeit}\index{Typisierung}%
%
Wurde erkannt, dass das Dokument eine Abschlussarbeit ist,%
\footnote{%
  Entweder wurde \Macro{thesis} oder \Macro{subject} mit einem speziellen Wert 
  oder der Option \Option{subjectthesis} verwendet.
}
so kann der zu erlangende akademische Grad mit dem Befehl \Macro{graduation} 
angegeben werden. Bei dessen Ausgabe auf dem Titel wird dabei der entsprechende 
Text dazu angegeben.
\TermTable*{graduationtext}{.78\textwidth}
\end{Declaration}

\begin{Declaration}{\Term{datetext}}
\begin{Declaration}{\Term{defensedatetext}}
\printdeclarationlist%
\index{Titel}\index{Abschlussarbeit}%
\index{Datum}\index{Datum!Verteidigungsdatum}%
%
Wird mit \Macro{date} das Datum und mit \Macro{defensedate} ein Datum der 
Verteidigung für eine Abschlussarbeit angegeben, so werden auch diese Felder 
durch einen einleitenden Text beschrieben.
\TermTable{datetext,defensedatetext}
\end{Declaration}
\end{Declaration}

\begin{Declaration}{\Term{coverpagename}}
\begin{Declaration}{\Term{titlepagename}}
\printdeclarationlist%
\index{Lesezeichen}\index{Umschlagseite}\index{Titel!Umschlagseite}\index{Titel}
%
Diese beiden Bezeichner werden bei aktivierter \Option{tudbookmarks} für das 
Eintragen von Lesezeichen in ein PDF"=Dokument genutzt.
\TermTable{coverpagename,titlepagename}
\end{Declaration}
\end{Declaration}

\begin{Declaration}{\Term{listingname}}
\begin{Declaration}{\Term{listlistingname}}
\printdeclarationlist%
%
Sollte ein Paket zur Einbindung von externem Quelltext~-- beispielsweise 
das Paket \Package{listings}~-- verwendet werden, so werden diese Bezeichnungen 
für Quelltextausschnitte und das Quelltextverzeichnis verwendet.
\TermTable{listingname,listlistingname}
\end{Declaration}
\end{Declaration}

\begin{Declaration}{\Term{abstractname}}
\printdeclarationlist%
%
Dieser Bezeichner wird lediglich für \Class{tudscrbook} definiert, da dieser 
von \KOMAScript{} für die Buchklasse nicht vorgesehen wird.
\TermTable{abstractname}
\end{Declaration}

\Rename[terms]{v2.02}{\Term{restrictionname}}{\Term{blockingname}}
\begin{Declaration}{\Term{confirmationname}}
\begin{Declaration}{\Term{blockingname}}
\printdeclarationlist%
\index{Selbstständigkeitserklärung}\index{Sperrvermerk}%
%
Es werden die Bezeichnungen für Selbstständigkeitserklärung und Sperrvermerk 
für die dazugehörigen Überschriften definiert.
\TermTable{confirmationname,blockingname}
\end{Declaration}
\end{Declaration}

\Rename[terms]{v2.02}{\Term{restrictiontext}}{\Term{blockingtext}}
\begin{Declaration}{\Term{confirmationtext}}
\begin{Declaration}{\Term{blockingtext}}
\printdeclarationlist%
%
Die Texte der Erklärungen selbst sind derart aufgebaut, dass sie in 
Abhängigkeit von den angegebenen Informationen unterschiedlich ausgeführt 
werden. Innerhalb der Selbstständigkeitserklärung (\Macro{confirmation}) werden 
gegebenenfalls die Felder für den Titel (\Macro{title}) und die Typisierung der 
Abschlussarbeit%
\footnote{%
  entweder \Macro{thesis} oder \Macro{subject}\Parameter{\autoref{tab:thesis}}
  beziehungsweise Option \Option{subjectthesis}[true]
}
sowie die angegebenen Unterstützer%
\footnote{%
  \Macro{confirmation}\POParameter{\Key{\Macro{confirmation}}{supporter}=\dots}
  oder \Macro{supporter}\PParameter{\dots}%
}
beachtet. Für den Sperrvermerk (\Macro{blocking}) wird neben dem Titel 
(\Macro{title}) optional außerdem noch das Feld der externen Firma%
\footnote{%
  \Macro{blocking}\POParameter{\Key{\Macro{blocking}}{company}=\dots} oder 
  \Macro{company}\PParameter{\dots}%
}
verwendet. Der Vollständigkeit halber werden im Folgenden noch die Texte für 
die Selbstständigkeitserklärung und den Sperrvermerk aufgeführt~-- allerdings 
lediglich die deutschsprachige Version. Dabei werden alle möglichen Felder 
angezeigt.

\begingroup
  \makeatletter
  \def\@@title{\PName{Titel}}
  \def\@@thesis{\PName{Abschlussarbeit}}
  \def\@supporter{\PName{Vorname Nachname} \and \PName{Vorname Nachname}}
  \def\@company{\PName{Firma}}
  \makeatother
  
  \vskip\baselineskipglue\noindent
  \textbf{Bezeichner}\quad\Term*{confirmationtext}%
  \par\vskip\baselineskipglue%
  \begin{center}\begin{minipage}{.8\textwidth}
  \confirmationtext
  \end{minipage}\end{center}
  
  \vskip\baselineskipglue\noindent
  \textbf{Bezeichner}\quad\Term*{blockingtext}%
  \par\vskip\baselineskipglue%
  \begin{center}\begin{minipage}{.8\textwidth}
  \blockingtext
  \end{minipage}\end{center}
\endgroup
\end{Declaration}
\end{Declaration}
\index{Bezeichner|!)}
\setchapterpreamble{%
  Zusätzlich zu den eigentlichen Hauptklassen werden im \TUDScript-Bundle 
  weitere Paket bereitgestellt. Diese werden im Folgenden vorgestellt.%
}
\chapter{Zusätzliche Pakete im \TUDScript-Bundle}
\section{Das Paket \Package*{tudscrsupervisor} -- Studentische Betreuung}
\DeclarePackage*{tudscrsupervisor}
%
Dieses Paket stellt für das Erstellen von Aufgabenstellungen und Gutachten  
wissenschaftlicher Arbeiten sowie offiziellen Aushängen im \CD passende 
Umgebungen und Befehle für den Anwender bereit. Deshalb richtet es sich 
vornehmlich an Mitarbeiter an der \TnUD, kann jedoch natürlich auch von 
Studenten genutzt werden.


\subsection{Aufgabenstellung für eine wissenschaftliche Arbeit}
\begin{Declaration}{\Environment{task}[\OLParameter{Überschrift}]}{%
  \Environment[autoref]{tudpage}%
}
\begin{Declaration}{\Key{\Environment{task}}{headline}[\PName{Überschrift}]}
\printdeclarationlist%
\index{Aufgabenstellung|!(}%
%
Mit der \Environment{task}"=Umgebung kann ein Aufgabenstellung für eine 
wissenschaftliche Arbeit ausgegeben werden. Sie basiert auf der Umgebung 
\Environment{tudpage} und akzeptiert deshalb im optionalen Argument alle 
Parameter, welche bei der Beschreibung von \Environment'{tudpage} erläutert 
wurden.

Für die Aufgabenstellung wird normalerweise eine Überschrift gesetzt, welche 
sich aus \Term{taskname} und~-- falls der Typ der Abschlussarbeit angegeben 
wurde~-- noch aus \Term{tasktext} und \Macro{thesis} zusammensetzt. Der 
Parameter \Key*{\Environment{task}}{headline} kann genutzt werden, um diese 
automatisch generierte Überschrift anzupassen.

Zu Beginn der Aufgabenstellung erscheint eine Tabelle mit den angegebenen 
Informationen zum Autor respektive zu den Autoren der Abschlussarbeit. Zwingend 
anzugeben sind dafür lediglich der Name des oder der Verfasser (\Macro{author}) 
sowie der Titel der Arbeit (\Macro{title}), welcher am Ende der Tabelle in 
fetter Schrift aufgeführt wird. Optional werden noch die Felder für den 
Studiengang (\Macro{course}), die Fachrichtung (\Macro{discipline}) sowie die 
Matrikelnummer (\Macro{matriculationnumber}) und das Immatrikulationsjahr 
(\Macro{matriculationyear}) ausgegeben, wobei nicht angegebene Felder bei der 
Ausgabe ignoriert werden. Der eigentliche Inhalt der Umgebung~-- sprich die 
Aufgabenstellung selbst~-- wird nach dem generierten Kopf ausgegeben

Nach der Ausgabe des Inhaltes der Aufgabenstellung werden der oder die mit 
\Macro{supervisor} definierten Betreuer aufgelistet. Dabei wird unter dem 
jeweiligen Namen selbst der sprachabhängige Bezeichner (\Term{supervisorname}, 
\Term{supervisorothername}) gesetzt. Darauf folgend erscheint das Ausgabedatum 
(\Macro{issuedate}) und der verpflichtende Abgabetermin (\Macro{duedate}). Zum 
Schluss wird die Unterschriftzeile für den Prüfungsausschussvorsitzenden 
(\Macro{chairman}) und den betreuenden Hochschullehrer (\Macro{professor}) 
gesetzt. Für genannte Personen werden unter dem Namen selbst die Bezeichner 
ausgegeben (\Term{chairmanname} und \Term{professorname}).
\end{Declaration}
\end{Declaration}

\begin{Declaration}{\Macro{taskform}\LParameter%
  \Parameter{Ziele}\Parameter{Schwerpunkte}%
}
\printdeclarationlist%
%
Zusätzlich zur der frei gestaltbaren Umgebung \Environment*{task} zur Erstellung
einer Aufgabenstellung wird ein separater Befehl für eine standardisierte 
Ausgabe zur Verfügung gestellt. Dieser strukturiert die Aufgabenstellung in die 
zwei Bereiche \emph{Ziele} und \emph{Schwerpunkte} der Arbeit mit dazugehörigen 
Überschriften (\Term{objectivesname}, \Term{focusname}).

Im optionalen Argument können alle Parameter der Umgebung \Environment*{task} 
verwendet werden. Im ersten obligatorischen Argument sollte ein Text mit einer 
kurzen thematischen Einordnung und dem eigentlichen Ziel der Arbeit erscheinen. 
im zweiten Argument sollen die thematischen Schwerpunkte in Stichpunkten 
benannt werden. Der Inhalt des zweiten notwendigen Argumentes wird in einer 
\Environment{itemize}"=Umgebung gesetzt. Deshalb \emph{muss} jedem Stichpunkt 
\Macro*{item} vorangestellt werden.
\index{Aufgabenstellung|!)}%
\end{Declaration}
%
\begin{Example}
Die empfohlene Verwendung des Befehls \Macro*{taskform} ist wie folgt:
\begin{Code}[escapechar=§]
\taskform{%
  Motivation der Arbeit im ersten Absatz§\dots§
  
  Ziele der Arbeit im zweiten Absatz§\dots§
}{%
  \item Schwerpunkt 1
  \item Schwerpunkt 2
}
\end{Code}
Hierzu sei auch auf das Minimalbeispiel in \autoref{sec:exmpl:task} verwiesen.
\end{Example}

\Rename[macros]{v2.02}{\Macro{branch}}{\Macro{discipline}}
\begin{Declaration}{\Macro{course}\Parameter{Studiengang}}
\begin{Declaration}{\Macro{discipline}\Parameter{Studienrichtung}}
\printdeclarationlist%
\index{Kollaboratives Schreiben}%
%
Mit diesen beiden Befehlen kann der Studiengang sowie die Studienrichtung für 
den Autor oder die Autoren angegeben werden. Diese Informationen werden zu 
Beginn der \Environment{task}"=Umgebung gesetzten Tabelle ausgegeben. Werden 
diese Befehle innerhalb des Makros \Macro{author} verwendet, können auch 
unterschiedliche Angaben für mehrere Autoren gemacht werden. Dabei sind die 
Autoren mit \Macro{and} voneinander zu trennen.
\end{Declaration}
\end{Declaration}

\begin{Declaration}{\Macro{chairman}\Parameter{Prüfungsausschussvorsitzender}}
\printdeclarationlist%
%
Wird dieses Feld genutzt, wird neben dem betreuenden Hochschullehrer 
(\Macro{professor}) auch der Vorsitzende des Prüfungsausschusses am Ende der 
Aufgabenstellung aufgeführt. Dies wird zumeist für Abschlussarbeiten wie 
beispielsweise \masterthesisname{} oder \diplomathesisname{} benötigt.
\end{Declaration}

\begin{Declaration}{\Macro{issuedate}\Parameter{Ausgabedatum}}
\begin{Declaration}{\Macro{duedate}\Parameter{Abgabetermin}}
\printdeclarationlist%
%
Mit diesen beiden Befehlen sollte das Datum der Ausgabe der Aufgabenstellung 
sowie der spätest mögliche Abgabetermin angegeben werden. Ist das Paket 
\Package{isodate} geladen, wird die damit eingestellte Formatierung des Datums 
durch den Befehl \Macro{printdate} aus diesem Paket für \Macro*{issuedate} und 
\Macro*{duedate} verwendet.
\end{Declaration}
\end{Declaration}


\subsection{Gutachten für wissenschaftliche Arbeiten}
\begin{Declaration}{\Environment{evaluation}[\OLParameter{Überschrift}]}{%
  \Environment[autoref]{tudpage}%
}
\begin{Declaration}{%
  \Key{\Environment{evaluation}}{headline}[\PName{Überschrift}]%
}
\begin{Declaration}{\Key{\Environment{evaluation}}{grade}[\PName{Note}]}
\printdeclarationlist%
\index{Gutachten|!(}%
%
Diese Umgebung wird für das Erstellen eines Gutachtens einer wissenschaftlichen 
Arbeit bereitgestellt. Auch diese unterstützt alle Parameter, welche für die 
Umgebung \Environment'{tudpage} beschrieben wurden.

Für ein Gutachten wird gewöhnlich eine Überschrift aus \Term{evaluationname} 
und~-- falls der Abschlussarbeitstyp angegeben wurde~-- \Term{evaluationtext} 
sowie \Macro{thesis} generiert. Diese automatisch generierte Überschrift kann 
mit dem Parameter \Key*{\Environment{evaluation}}{headline} ersetzt werden. Am 
Ende des Gutachtens wird die mit \Key*{\Environment{evaluation}}{grade} 
gegebene Note in fetter Schrift ausgezeichnet.

Am Anfang der \Environment*{evaluation}"=Umgebung wird die gleiche Tabelle mit 
Autorenangaben ausgegeben, wie dies bei der \Environment{task}"=Umgebung der 
Fall ist. Nach dem Tabellenkopf folgt auch hier der eigentliche Inhalt, sprich 
das Gutachten der Abschlussarbeit. Abgeschlossen wird die Umgebung mit der 
gegebenen Note~-- welche innerhalb von \Term{gradetext} ausgegeben wird~-- 
sowie der Orts- und Datumsangabe (\Macro{place}, \Macro{date}) und der 
darauffolgenden Unterschriftzeile für den oder die Gutachter (\Macro{referee}), 
welche wiederum mit den entsprechenden sprachabhängigen Bezeichner 
(\Term{refereename}, \Term{refereeothername}) ergänzt werden.
\end{Declaration}
\end{Declaration}
\end{Declaration}

\begin{Declaration}{\Macro{evaluationform}\LParameter%
  \Parameter{Aufgabe}\Parameter{Inhalt}\Parameter{Bewertung}\Parameter{Note}%
}
\printdeclarationlist%
%
Neben der individuell nutzbaren Umgebung \Environment*{evaluation} wird ein 
separater Befehl zur Erstellung eines standardisierten Gutachtens 
bereitgestellt. Dieser strukturiert die Ausgabe in die vier Bereiche 
\emph{Aufgabe}, \emph{Inhalt}, \emph{Bewertung} und \emph{Note} und versieht 
diese jeweils mit der dazugehörigen Überschrift beziehungsweise Textausgabe 
(\Term{taskname}, \Term{contentname}, \Term{assessmentname} und 
\Term{gradetext}). Das optionale Argument unterstützt alle Parameter der 
\Environment{evaluation}"=Umgebung.
\index{Gutachten|!)}%
\end{Declaration}
%
\begin{Example}
Die empfohlene Verwendung des Befehls \Macro*{evaluationform} ist wie folgt:
\begin{Code}[escapechar=§]
\evaluationform{%
  Kurzbeschreibung der Aufgabenstellung§\dots§
}{%
  Zusammenfassung von Inhalt und Struktur§\dots§
}{%
  Bewertung der schriftlichen Abschlussarbeit§\dots§
}{%
  Zahl (Note)
}
\end{Code}
Hierzu sei auch auf das Minimalbeispiel in \autoref{sec:exmpl:evaluation} 
verwiesen.
\end{Example}

\begin{Declaration}{\Macro{grade}\Parameter{Note}}
\printdeclarationlist%
%
Neben der Angabe der Note für ein Gutachten über den Parameter 
\Key*{\Environment{evaluation}}{grade} kann dafür auch dieser global wirkende 
Befehl verwendet werden.
\end{Declaration}


\subsection{Aushang}
\begin{Declaration}{\Environment{notice}[\OLParameter{Überschrift}]}{%
  \Environment[autoref]{tudpage}%
}
\begin{Declaration}{\Key{\Environment{notice}}{headline}[\PName{Überschrift}]}
\printdeclarationlist%
\index{Aushang|!(}%
%
Für das Anfertigen eines Aushangs kann diese Umgebung verwendet werden. Sie 
basiert abermals auf der Umgebung \Environment{tudpage} und unterstützt alle 
deren Parameter.

Wurde ein Datum angegeben, wird dieses in der oberen rechten Ecke gesetzt. 
Anschließend wird die Überschrift ausgegeben, welche für gewöhnlich dem Inhalt 
von \Term{noticename} entspricht und mit \Key*{\Environment{notice}}{headline} 
geändert werden kann. Nach der Überschrift wird bereits der Inhalt der Umgebung 
ausgegeben. Wurde mit \Macro{contactperson} ein oder mehrere Ansprechpartner 
angegeben, werden diese Informationen am Ende der Umgebung ausgegeben.
\end{Declaration}
\end{Declaration}

\begin{Declaration}{\Macro{noticeform}\LParameter%
  \Parameter{Inhalt}\Parameter{Schwerpunkte}%
}
\printdeclarationlist%
%
Auch für diese Umgebung gibt es einen Befehl für eine normierte Form. Diese 
soll vor allem Verwendung für den Aushang studentischer Arbeitsthemen finden. 
Für das optionale Argument können sämtliche Parameter verwendet werden, die 
auch die \Environment*{notice}"=Umgebung unterstützt.

Das erste obligatorische Argument sollte für eine kurze Inhaltsbeschreibung 
verwendet werden. Neben dem textuellen Teil sollte hier wenn möglich eine 
thematisch passende Abbildung eingebunden werden (\Macro{includegraphics}). Das 
zweite Argument wird~--wie schon bei \Macro{taskform}~-- dazu verwendet, einige 
Schwerpunkte aufzuzählen. Auch hier kommt nach der gliedernden Überschrift 
(\Term{focusname}) eine \Environment{itemize}"=Umgebung zum Einsatz, allen 
Schwerpunkten muss ein \Macro*{item} vorangestellt werden.
\index{Aushang|!)}%
\end{Declaration}
%
\begin{Example}
Die empfohlene Verwendung des Befehls \Macro*{noticeform} ist wie folgt:
\begin{Code}[escapechar=§]
\noticeform{%
  Kurzbeschreibung des Inhaltes der studentischen Arbeit§\dots§
  
  Bild (optional), einzubinden mit:
    \includegraphics[§\PName{Optionen}§]{§\PName{Datei}§}
}{%
  \item Schwerpunkt 1
  \item Schwerpunkt 2
}
\end{Code}
Hierzu sei auch auf das Minimalbeispiel in \autoref{sec:exmpl:notice} verwiesen.
\end{Example}

\Rename[macros]{v2.02}{\Macro{contact}}{\Macro{contactperson}}
\Rename[macros]{v2.02}{\Macro{phone}}{\Macro{telephone}}
\Rename[macros]{v2.02}{\Macro{email}}{\Macro{emailaddress}}
\begin{Declaration}{\Macro{contactperson}\Parameter{Kontaktperson(en)}}
\begin{Declaration}{\Macro{office}\Parameter{Dienstsitz}}
\begin{Declaration}{\Macro{telephone}\Parameter{Telefonnummer}}
\begin{Declaration}{\Macro{emailaddress}\Parameter{E-Mail-Adresse}}
\printdeclarationlist%
%
Am Ende eines Aushangs können mit \Macro*{contactperson} Kontaktinformationen 
für eine oder mehrere Ansprechpartner angegeben werden. Soll mehr als eine 
Kontaktperson genannt werden, so müssen diese innerhalb des Befehls
\Macro*{contactperson} mit dem Befehl \Macro{and} getrennt werden. Für jede 
Person kann innerhalb von \Macro*{contactperson} der Dienstsitz 
(\Macro*{office}), die dienstliche Telefonnummer (\Macro*{telephone}) sowie die 
geschäftliche E-Mail"=Adresse (\Macro*{email}) angegeben werden. Sollte das 
Paket \Package*{hyperref} geladen werden, wird die gegebene E-Mail"=Adresse 
direkt in einen entsprechenden Link gewandelt.
\end{Declaration}
\end{Declaration}
\end{Declaration}
\end{Declaration}


\subsection{Zusätzliche sprachabhängige Bezeichner}
\index{Bezeichner|!(}
Für das Paket \Package*{tudscrsupervisor} werden für die zusätzlichen 
Umgebungen weiter Bezeichner definiert. Für eine etwaige Anpassung dieser sei 
auf \autoref{sec:localization} verwiesen.

\begin{Declaration}{\Term{taskname}}
\begin{Declaration}{\Term{tasktext}}
\printdeclarationlist%
%
Die Bezeichnung der Aufgabenstellung selbst ist in \Term*{taskname} enthalten. 
Für die Generierung einer Überschrift wird dieser verwendet. Wurde außerdem mit 
\Macro{thesis} oder \Macro{subject} der Typ der Abschlussarbeit%
\footnote{%
  \Option{subjectthesis} oder spezieller Wert aus \autoref{tab:thesis}
}
angegeben, wird die Überschrift zusammen mit dem Bezeichner \Term*{tasktext}
um die Typisierung erweitert. Falls gewünscht, kann die automatisch generierte 
Überschrift mit dem Parameter \Key{\Environment{task}}{headline} der Umgebung 
\Environment{task} überschrieben werden.
\TermTable{taskname,tasktext}
\end{Declaration}
\end{Declaration}

\Rename[terms]{v2.02}{\Term{branchname}}{\Term{disciplinename}}
\begin{Declaration}{\Term{authorname}}
\begin{Declaration}{\Term{titlename}}
\begin{Declaration}{\Term{coursename}}
\begin{Declaration}{\Term{disciplinename}}
\printdeclarationlist%
%
Diese Bezeichner werden in der Tabelle mit den Autoreninformationen zu Beginn 
der Aufgabenstellung verwendet. Dabei werden \Term*{coursename} und 
\Term*{disciplinename} nur genutzt, wenn für mindestens einen Autor die Befehle 
\Macro{course} und/oder \Macro{discipline} verwendet wurden.
\TermTable{authorname,titlename,coursename,disciplinename}
\end{Declaration}
\end{Declaration}
\end{Declaration}
\end{Declaration}

\begin{Declaration}{\Term{issuedatetext}}
\begin{Declaration}{\Term{duedatetext}}
\printdeclarationlist%
%
Am Ende der Aufgabenstellung wird nach dem oder der Betreuer das Ausgabe- und 
Abgabedatum (\Macro{issuedate}, \Macro{duedate}) der Abschlussarbeit mit 
folgenden Bezeichner erläutert.
\TermTable{issuedatetext,duedatetext}
\end{Declaration}
\end{Declaration}

\begin{Declaration}{\Term{chairmanname}}
\printdeclarationlist%
%
Wurde der Prüfungsausschussvorsitzende (\Macro{chairman}) angegeben, erfolgt 
unter dem Namen selbst die Ausgabe des Bezeichners.
\TermTable{chairmanname}
\end{Declaration}

\begin{Declaration}{\Term{focusname}}
\begin{Declaration}{\Term{objectivesname}}
\printdeclarationlist%
%
Die Vorlagen für Aufgabenstellung (\Macro{taskform}) beziehungsweise Aushang 
(\Macro{noticeform}) nutzen für die gesetzten Überschriften diese Bezeichner.
\TermTable{focusname,objectivesname}
\end{Declaration}
\end{Declaration}

\begin{Declaration}{\Term{evaluationname}}
\begin{Declaration}{\Term{evaluationtext}}
\printdeclarationlist%
%
Die Bezeichnung des Gutachten selbst ist in \Term*{evaluationname} enthalten. 
Für die Generierung der Überschrift wird der Bezeichner \Term*{evaluationtext} 
sowie der mit \Macro{thesis} oder gegebenenfalls mit \Macro{subject} gegebenen 
Typ der Abschlussarbeit verwendet. Diese automatisch generierte Überschrift 
kann mit dem Parameter \Key{\Environment{evaluation}}{headline} der 
Umgebung \Environment{evaluation} durch den Anwender überschrieben werden.
\TermTable{evaluationname,evaluationtext}
\end{Declaration}
\end{Declaration}

\begin{Declaration}{\Term{contentname}}
\begin{Declaration}{\Term{assessmentname}}
\printdeclarationlist%
%
Bei der standardisierten Form des Gutachten (\Macro{evaluationform}) werden die 
darin~-- zur strukturierter Gliederung~-- erzeugten Überschriften mit den 
Bezeichnern \Term*{taskname}, \Term*{contentname} und \Term*{assessmentname} 
gesetzt.
\TermTable{taskname,contentname,assessmentname}
\end{Declaration}
\end{Declaration}

\begin{Declaration}{\Term{gradetext}}
\printdeclarationlist%
%
Wird für ein Gutachten mit dem Befehl \Macro{grade}\Parameter{Note} oder mit 
dem Parameter \Key{\Environment{evaluation}}{grade}[\PName{Note}] die Note 
angegeben, so wird diese innerhalb von \Term*{gradetext} verwendet.
\grade{\PName{Note}}
\TermTable*{gradetext}{.7\textwidth}
\end{Declaration}

\Rename[terms]{v2.02}{\Term{contactname}}{\Term{contactpersonname}}
\begin{Declaration}{\Term{noticename}}
\begin{Declaration}{\Term{contactpersonname}}
\printdeclarationlist%
%
Die Bezeichnung des Aushangs selbst ist in \Term*{noticename} enthalten. Für 
die Generierung einer Überschrift wird dieser verwendet. Falls gewünscht, kann 
diese mit dem Parameter \Key{\Environment{notice}}{headline} der Umgebung 
\Environment{notice} überschrieben werden. Wurde eine Kontaktperson mit dem 
Befehl \Macro{contactperson} angegeben, wird als Überschrift der Kontaktdaten 
der Bezeichner \Term*{contactpersonname} verwendet.

\TermTable{noticename,contactpersonname}
\end{Declaration}
\end{Declaration}
\index{Bezeichner|!)}



\section{Das Paket \Package*{tudscrcolor} -- Farben im \CD}%
\DeclarePackage*{tudscrcolor}
\index{Farben|!}%
%
Zur Verwendung der Farben des \CDs wird das Paket \Package{tudscrcolor} 
genutzt. Falls dieses nicht in der Präambel geladen wird~-- um beispielsweise 
zusätzliche Optionen aufzurufen~-- binden die \TUDScript"=Klassen dieses 
automatisch ein.

Für das \CD sind mehrere Farben vorgesehen. Die prägnanteste aller ist die 
Hausfarbe \Color*{HKS41}, danach folgen die Farben für Auszeichnungen der ersten
(\Color*{HKS44}) und der zweiten Kategorie (\Color*{HKS36}, \Color*{HKS33}, 
\Color*{HKS57}, \Color*{HKS65}) sowie eine Ausnahmefarbe (\Color*{HKS07}). 
Diese Farben dürfen sowohl in ihrer Grundform als auch in helleren Tönen mit 
einer Abstufung in 10\,\%"~Schritten verwendet werden. Das ohnehin verwendete 
Paket \Package{xcolor} stellt genau diese Funktionalität zur Verfügung. Jede 
der Farben kann sowohl über \Color*{HKS}\PName{Zahl} als auch über ein 
Pseudonym \Color*{cd}\PName{Farbe} angesprochen werden. Diese werden in diesem 
\autorefname dokumentiert und dargestellt.
%
\begin{Example*}
Die Grundfarbe \Color*{HKS44} soll in der auf 20\% reduzierten, helleren 
Abstufung genutzt werden. Innerhalb eines Befehls, der als Argument eine 
gültige Farbe erwartet, muss lediglich \Color*{HKS44}\PValue{!20} angegeben 
werden. Dies wird hier exemplarisch mit dieser \colorbox{HKS44!20}{%
  Box \Macro*{colorbox}\PParameter{HKS44!20}\PParameter{Box}%
} demonstriert.
\end{Example*}
%
Bei der farbigen Gestaltung des \CDs (\Option{cd}[color]) ist der Hintergrund 
von Umschlagseite, Titel sowie Teilen in \Color*{HKS41} und die Schrift auf 
selbigen in \Color*{HKS41}\PValue{!30} gehalten. Der Kapitelseitenhintergrund
erscheint in \Color*{HKS41}\PValue{!10}, die Schrift in \Color*{HKS41}. Bei 
geringerem Farbeinsatz werden lediglich die Schriften der Gliederungsseiten auf 
\Color*{HKS41} gesetzt.

Sollen bestimmte Optionen an das Paket \Package{xcolor} weitergereicht werden, 
gibt es dafür zwei Möglichkeiten. Diese kann entweder vor dem Laden der Klasse 
direkt an \Package{xcolor} übergeben werden%
\footnote{%
  \Macro*{PassOptionsToPackage}\Parameter{Option}\PParameter{xcolor} gefolgt von
  \Macro*{documentclass}\OParameter{Klassenoptionen}\PParameter{tudscr\dots}
} oder es wird \Package*{tudscrcolor} mit der entsprechenden Option geladen.%
\footnote{
  \Macro*{usepackage}\OParameter{Option}\PParameter{\Package*{tudscrcolor}};
  \Package*{tudscrcolor} reicht \PName{Option} an \Package{xcolor} weiter
}
\newcommand*\cdcolorcalc{}
\newcommand*\cdcolorname{}
\newcommand*\cdcolorvalue{}
\newcommand*\cdcolortext{}
\newcommand*\cdcolor[2][0]{%
  \noindent%
  \begin{tikzpicture}[every node/.style={%
    rectangle,minimum height=.1\linewidth,minimum width=25mm%
  }]%
  \def\cdcolorcalc##1##2{%
    \pgfmathparse{100-##1*10}%
    \xdef\cdcolorname{HKS##2!\pgfmathresult}%
    \xdef\cdcolorvalue{\pgfmathresult}%
    \pgfmathparse{10+##1*10}%
  }%
  \foreach \x in {0,1,...,9}{%
    \cdcolorcalc{\x}{#2}%
    \ifnum\x<#1%
      \def\cdcolortext{white}%
    \else%
      \def\cdcolortext{black}%
    \fi%
    \node [fill=\cdcolorname,rotate=90] at (.\x\linewidth,0)%
      {\textcolor{\cdcolortext}{HKS#2!\pgfmathprintnumber\cdcolorvalue}};%
  };%
  \end{tikzpicture}%
}


\subsection{Primäre Hausfarbe}
\begin{Declaration}{\Color{HKS41}[cddarkblue]}
\printdeclarationlist%
\cdcolor[6]{41}
\end{Declaration}


\subsection{Sekundäre Hausfarbe (Geschäftsausstattung)}
\begin{Declaration}{\Color{HKS92}[cdgray]}
\printdeclarationlist%
\cdcolor[4]{92}
\end{Declaration}


\subsection{Auszeichnungsfarbe 1.Kategorie}
\begin{Declaration}{\Color{HKS44}[cdblue]}
\printdeclarationlist%
\cdcolor[4]{44}
\end{Declaration}


\subsection{Auszeichnungsfarbe 2.Kategorie}
\begin{Declaration}{\Color{HKS36}[cdindigo]}
\begin{Declaration}{\Color{HKS33}[cdpurple]}
\begin{Declaration}{\Color{HKS57}[cddarkgreen]}
\begin{Declaration}{\Color{HKS65}[cdgreen]}
\printdeclarationlist%
\cdcolor[4]{36}\vskip\baselineskipglue
\cdcolor[4]{33}\vskip\baselineskipglue
\cdcolor[2]{57}\vskip\baselineskipglue
\cdcolor{65}
\end{Declaration}
\end{Declaration}
\end{Declaration}
\end{Declaration}


\subsection{Ausnahmefarbe}
\begin{Declaration}{\Color{HKS07}[cdorange]}
\printdeclarationlist%
\cdcolor{07}
\end{Declaration}


\subsection{Zusätzliche Farbdefinitionen}
Das Paket \Package*{tudscrcolor} definiert im Normalfall lediglich die zuvor 
beschriebenen Grundfarben \Color*{HKS41}, \Color*{HKS92}, \Color*{HKS44}, 
\Color*{HKS36}, \Color*{HKS33}, \Color*{HKS57}, \Color*{HKS65} sowie 
\Color*{HKS07}. Alle anderen farblichen Abstufungen können mit den beschrieben 
Möglichkeiten des Paketes \Package{xcolor} generiert werden.

\begin{Declaration}{\Option{full}}
\printdeclarationlist%
%
In den letzten Jahren sind viele verschiedene Klassen und Pakete für das \CD 
der \TnUD entstanden. Innerhalb dieser existieren abweichende Farbdefinitionen. 
Um eine Migration von anderen Klassen und Paketen auf das \TUDScript-Bundle zu 
ermöglichen, existiert die Paketoption \Option*{full}. Wird diese aktiviert, so 
werden zusätzlich weitere Farben nach dem Schema \Color*{HKS41K}\PName{Zahl} 
und \Color*{HKS41-}\PName{Zahl} bereitgestellt, wobei der hinten angestellte 
Zahlenwert aus der 10er-Reihe kommen muss.
\end{Declaration}



\subsection{Umstellung des Farbmodells}
\index{Farben!Farbmodell}%
Normalerweise verwendet \Package*{tudscrcolor} das CMYK"=Farbmodell. Außerdem 
wird weiterhin noch der RGB"=Farbraum unterstützt. Eine Umschaltung des 
Farbmodells ist beispielsweise für gewisse Funktionen des Paketes 
\Package{tikz} notwendig.

\begin{Declaration}{\Option{RGB}}
\printdeclarationlist%
%
Mit dieser Option werden bereits beim Laden des Paketes \Package*{tudscrcolor} 
die Farben nicht nach dem CMYK"=Farbmodell sondern im RGB"=Farbraum global 
definiert.
\end{Declaration}

\begin{Declaration}{\Macro{setcdcolors}\Parameter{Farbmodell}}
\printdeclarationlist%
%
Mit diesem Befehl kann innerhalb des Dokumentes das verwendete Farbmodell 
angepasst werden. Damit ist es möglich, lokal innerhalb einer Umgebung den 
Farbmodus zu ändern und so nur in bestimmten Situationen beispielsweise aus dem 
CMYK"=Farbmodell in den RGB"=Farbraum zu wechseln. Unterstützte Werte für 
\PName{Farbmodell} sind \PValue{CMYK} und \PValue{RGB} beziehungsweise 
\PValue{rgb}.
\end{Declaration}

\bigskip\noindent
\Attention{%
  Beachten Sie, dass die Darstellung der Farben im jeweiligen Farbmodus 
  (\PValue{CMYK} oder \PValue{RGB}) je nach verwendeter Bildschirm- Drucker- 
  und Softwarekonfiguration verschieden ausfallen kann. Die verwendeten 
  RGB-Werte entstammen aus dem Handbuch zum \CD und sind lediglich 
  Näherungswerte. Abweichungen vom gedruckten \Color*{HKS}-Farbregister und 
  selbst ermittelten Werten sind technisch nicht zu vermeiden.
}



\section{Das Paket \Package*{tudscrfonts} -- Schriften im \CD}
\DeclarePackage*[v2.02]{tudscrfonts}
%
Dieses Paket stellt die Schriften des \CDs für \hologo{LaTeXe}-Klassen bereit, 
welche \emph{nicht} zum \TUDScript-Bundle gehören. Das Paket unterstützt einen 
Großteil der in \fullref{sec:fonts} beschriebenen Optionen und Befehle. Die 
nutzbaren Paketoptionen sind für den Fließtext \Option{cdfont}~-- ohne die 
Einstellungsmöglichkeiten für den Querbalken des \CDs (\Option{barfont})~-- und 
für die mathematischen Schriften \Option{sansmath} sowie \Option{slantedgreek}. 
Da von \Package*{tudscrfonts} intern das Paket \Package{tudscrbase} geladen 
wird, können diese entweder als Paketoptionen im optionalen Argument von 
\Macro*{usepackage}\OParameter{Paketoption}\PParameter{\Package*{tudscrfonts}} 
oder direkt als Klassenoption angegeben werden. Zusätzlich ist nach dem Laden 
des Paketes die späte Optionenwahl mit \Macro{TUDoption} beziehungsweise 
\Macro{TUDoptions} möglich.

Des Weiteren wird das Paket \Package{textcase} geladen, welches die Befehle 
\Macro{MakeTextUppercase} und \Macro{NoCaseChange} zur Verfügung stellt. Der 
Befehl \Macro{ifdin} wird ebenso wie die in \autoref{sec:fonts} beschriebenen 
Textschalter und "~kommandos sowie die Befehle für die griechischen Buchstaben 
bereitgestellt.

Prinzipiell kann das Paket \Package*{tudscrfonts} mit jeder beliebigen 
\hologo{LaTeXe}-Klasse verwendet werden. Sollen allerdings die Überschriften~-- 
wie im \CD vorgesehen~-- in Majuskeln der \DIN gesetzt werden, muss der 
Anwender dies für die jeweilige Klasse selbst umsetzen. Dafür sei noch einmal 
auf die Textauswahlbefehle \Macro{textdbn} und \Macro{dinbn} sowie den Befehl 
\Macro{MakeTextUppercase} zur automatisierten Großschreibung hingewiesen.

Das Paket \Package*{tudscrfonts} ist insbesondere für die Verwendung zusammen 
mit den Klassen \Class{tudbook}, \Class{tudbeamer}, \Class{tudletter}, 
\Class{tudfax}, \Class{tudhaus} sowie \Class{tudform} vorgesehen. Die 
Schriftinstallation für das \TUDScript-Bundle unterscheidet sich von der für 
die gerade genannten Klassen sehr stark. Dabei wurde auch die Bezeichnung der 
Schriftfamilien geändert. Dies hatte zwei Gründe, wobei der letztere der 
entscheidende ist:
%
\begin{enumerate}
\item
  Die bisherige Schriftbenennung entsprach nicht dem offiziellen     
  \hrfn{http://mirrors.ctan.org/info/fontname/fontname.pdf}%
  {\hologo{TeX}-Namensschema}
\item
  Bei der Installation für das \TUDScript-Bundle werden sowohl die Metriken
  als auch das Kerning der Schriften für Fließtext und den Mathematikmodus 
  angepasst, was das Ergebnis der erzeugten Ausgabe beeinflusst. Damit jedoch
  Dokumente, die mit den Klassen von Klaus Bergmann erstellt wurden, weiterhin 
  genauso ausgegeben werden wie bisher, mussten die Schriftfamilien einen neuen 
  Namen erhalten.
\end{enumerate}
%
Wird nun das Paket \Package*{tudscrfonts} zusammen mit den alten Klassen von 
Klaus Bergmann verwendet hat dies den Vorteil, dass auch in diesen das 
angepasste Kerning der Schriften sowie der stark verbesserte Mathematiksatz zum 
Tragen kommen. Außerdem kann bei der Verwendung von \Package*{tudscrfonts} auf 
eine Installation der Schriften des \CDs in der alten Variante verzichtet 
werden.
\Attention{%
  In diesem Fall kann sich das Ausgabeergebnis im Vergleich zu der Varianten 
  mit den alten Schriften ändern. Alternativ zur Verwendung des Paketes 
  \Package*{tudscrfonts} können die alten Schriftfamilien auch parallel zu den 
  neuen installiert werden. Hierfür werden die Skripte
  \hrfn{https://github.com/tud-cd/tudscrold/releases/download/fonts/tudfonts_install.bat}{\File{tudfonts\_install.bat}}
  beziehungsweise
  \hrfn{https://github.com/tud-cd/tudscrold/releases/download/fonts/tudfonts_install.sh}{\File{tudfonts\_install.sh}}
  bereitgestellt.
}



\section{Das Paket \Package*{tudscrcomp} -- Kompatibilität zu alten Klassen}
\DeclarePackage*{tudscrcomp}
\index{Version!v1.0|(}%
\index{Kompatibilität!Version v1.0|(}%
\index{Kompatibilität!\Class*{tudbook}|(}%
%
\noindent\Attention{%
  Sollten Sie die \Class{tudbook}-Klasse oder \TUDScript in der Version~v1.0 
  nie genutzt haben, können Sie dieses \autorefname ohne Weiteres überspringen.
  Sämtliche hier vorgestellten Optionen und Befehle sind in der aktuellen 
  Version von \TUDScript obsolet.
}

\bigskip\noindent
Zu Beginn der Entwicklung von \TUDScript bildete die Klasse \Class{tudbook}
die Basis. Ziel war es, sämtliche Funktionalitäten dieser beizubehalten und 
zusätzlich den vollen Funktionsumfang der \KOMAScript-Klassen nutzbar zu 
machen. Bei der kompletten Neuimplementierung der \TUDScript-Klassen wurde sehr 
viel verändert und verbessert. Einige der Optionen und Befehle waren jedoch 
bereits in der \TUDScript-Version~v1.0 Relikte, um die Kompatibilität zur 
\Class{tudbook}-Klasse zu gewährleisten. Mit der Version~v2.00 wurden einige 
aus Gründen der Konsistenz lediglich umbenannt, andere wiederum wurden 
vollständig entfernt oder über neue Befehle und Optionen in ihrer 
Funktionalität ersetzt und erweitert. 

Das Paket \Package{tudscrcomp} dient zum Umstieg von \TUDScript der Version~v1.0
beziehungsweise der \Class{tudbook}-Klasse. Nach dem Laden des Paketes stehen 
Optionen und Befehle bereit, welche von den alten Klassen definiert wurden und 
das entsprechende Verhalten nachahmen. Die Intention ist, alte Dokumente 
möglichst schnell und einfach auf die \TUDScript-Klassen portieren zu können. 
Des Weiteren ist beschrieben, wie sich die Funktionalität ohne Verwendung des 
Paketes mit den Mitteln von \TUDScript umsetzen lassen. Für den Satz neuer 
Dokumente wird empfohlen, auf den Einsatz dieses Paketes komplett zu verzichten 
und stattdessen die neuen Befehle zu nutzen.

\begin{Declaration}{\Option{serifmath}}{%
  identisch zu \Option{sansmath}[false]%
}
\printdeclarationlist%
%
Die Funktionalität wird durch die Option \Option{sansmath} bereitgestellt.
\end{Declaration}

\begin{Declaration}{\Option{colortitle}}{%
  identisch zu \Option{cdtitle}[color]%
}
\begin{Declaration}{\Option{nocolortitle}}{%
  identisch zu \Option{cdtitle}[true]%
}
\printdeclarationlist%
%
Die Funktionalität wird durch die Option \Option{cdtitle} bereitgestellt.
\end{Declaration}
\end{Declaration}

\begin{Declaration}{\Macro{einrichtung}\Parameter{Fakultät}}{%
  identisch zu \Macro{faculty}
}
\begin{Declaration}{\Macro{fachrichtung}\Parameter{Einrichtung}}{%
  identisch zu \Macro{department}
}
\begin{Declaration}{\Macro{institut}\Parameter{Institut}}{%
  identisch zu \Macro{institute}
}
\begin{Declaration}{\Macro{professur}\Parameter{Lehrstuhl}}{%
  identisch zu \Macro{chair}
}
\printdeclarationlist%
%
Dies sind die deutschsprachigen Befehle für den Kopf im \CD.
\end{Declaration}
\end{Declaration}
\end{Declaration}
\end{Declaration}

\begin{Declaration}{\Macro{moreauthor}\Parameter{Autorenzusatz}}{%
  identisch zu \Macro{authormore}%
}
\printdeclarationlist%
%
Ursprünglich war diese Befehl für das Unterbringen aller möglichen, 
zusätzlichen Autoreninformationen gedacht. Auch der Befehl \Macro{authormore} 
ist ein Rudiment davon. Empfohlen wird die Verwendung der Befehle 
\Macro{dateofbirth}, \Macro{placeofbirth}, \Macro{matriculationnumber} und 
\Macro{matriculationyear} sowie für die Aufgabenstellung einer 
wissenschaftlichen Arbeit \Macro{course} und \Macro{discipline} aus dem Paket 
\Package{tudscrsupervisor}.
\end{Declaration}

\begin{Declaration}{\Macro{submitdate}\Parameter{Datum}}{%
  identisch zu \Macro{date}%
}
\printdeclarationlist%
%
Die Funktionalität wird durch den erweiterten Standardbefehl \Macro{date} 
abgedeckt.
\end{Declaration}

\begin{Declaration}{\Macro{supervisorII}\Parameter{Name}}{%
  identisch zur Verwendung von \Macro{and} innerhalb von \Macro{supervisor}%
}
\printdeclarationlist%
%
Es sollte \Macro{supervisor}\PParameter{\PName{Name} \Macro{and} \PName{Name}}
anstelle des Befehls \Macro*{supervisorII}\Parameter{Name} verwendet werden.
\end{Declaration}

\begin{Declaration}{\Macro{supervisedby}\Parameter{Bezeichnung}}{%
  siehe \Term{supervisorname}%
}
\begin{Declaration}{\Macro{supervisedIIby}\Parameter{Bezeichnung}}{%
  siehe \Term{supervisorothername}%
}
\begin{Declaration}{\Macro{submittedon}\Parameter{Bezeichnung}}{%
  siehe \Term{datetext}%
}
\printdeclarationlist%
%
Zur Änderung der Bezeichnung der Betreuer sollten die sprachabhängigen 
Bezeichner wie in \autoref{sec:localization} beschrieben angepasst werden. Eine 
Verwendung der alten Befehle entfernt die Abhängigkeit der Bezeichner von der 
verwendeten Sprache.
\end{Declaration}
\end{Declaration}
\end{Declaration}

\begin{Declaration}{\Macro{dissertation}}
\printdeclarationlist%
%
Die Funktionalität kann durch die Befehle \Macro{thesis}\PParameter{diss} und 
\Macro{referee} sowie die Bezeichner \Term{refereename} und 
\Term{refereeothername} dargestellt werden.
\end{Declaration}

\begin{Declaration}{\Option{ddcfooter}}{%
  identisch zu \Option{ddcfoot}[true]%
}
\printdeclarationlist%
%
Die Funktionalität wird durch die Option \Option{ddcfoot} bereitgestellt.
\end{Declaration}

\begin{Declaration}{\Macro{chapterpage}}
\printdeclarationlist%
%
Durch diesen Befehl können Kapitelseiten konträr zur eigentlichen Einstellung 
aktiviert oder deaktiviert werden. Prinzipiell ist dies auch durch Änderung der 
Option \Option{chapterpage} möglich. Allerdings wird davon abgeraten, da dies 
zu einem inkonsistenten Layout innerhalb des Dokumentes führt.
\end{Declaration}

\begin{Declaration}{\Environment{theglossary}[\OParameter{Präambel}]}
\begin{Declaration}{\Macro{glossitem}\Parameter{Begriff}}
\printdeclarationlist%
%
Die \Class{tudbook}-Klasse stellt eine rudimentäre Umgebung für ein Glossar 
bereit. Allerdings gibt es dafür bereits zahlreiche und besser implementierte 
Pakete. Daher wird für diese Umgebung keine Portierung vorgenommen, sondern 
lediglich die ursprüngliche Definition übernommen. Allerdings sein an dieser 
Stelle auf wesentlich bessere Lösungen wie beispielsweise das Paket 
\Package{glossaries} oder~-- mit Abstrichen~-- das nicht ganz so umfangreiche 
Paket \Package{nomencl} verwiesen. 
\end{Declaration}
\end{Declaration}
\index{Version!v1.0|)}
\index{Kompatibilität!Version v1.0|)}
\index{Kompatibilität!\Class*{tudbook}|)}



\section{Das Paket \Package*{mathswap}}
\DeclarePackage*{mathswap}
\index{Trennzeichen}\index{Mathematiksatz}%
\index{Trennzeichen!Dezimaltrennzeichen}%
\index{Trennzeichen!Tausendertrennzeichen}%
%
Die Verwendung von Dezimal- und Tausendertrennzeichen im mathematischen Satz 
sind regional sehr unterschiedlich. In den meisten englischsprachigen Ländern 
wird der Punkt als Dezimaltrennzeichen und das Komma zur Zifferngruppierung 
verwendet, im restlichen Europa wird dies genau entgegengesetzt praktiziert.
Dieses Paket soll dazu dienen, beliebige formatierte Zahlen in ihrer Ausgabe 
anzupassen. Dafür werden die Zeichen Punkt (\ .\ ) und Komma (\ ,\ ) als 
aktive Zeichen im Mathematikmodus definiert.

Ähnliche Funktionalitäten werden bereits durch die Pakete \Package{icomma} und 
\Package{ziffer} bereitgestellt. Bei \Package{icomma} muss jedoch beim
Verfassen des Dokumentes durch den Autor beachtet werden, ob das verwendete
Komma einem Dezimaltrennzeichen entspricht ($t=1,\!2$) oder einem normalen 
Komma im Mathematiksatz ($z=f(x,y)$), wo ein gewisser Abstand nach dem Komma 
durchaus gewünscht ist. Das Paket \Package{ziffer} liefert dafür die gewünschte 
Funktionalität,%
\footnote{kein Leerraum nach Komma, wenn direkt danach eine Ziffer folgt}
ist allerdings etwas unflexibel, was den Umgang mit den Trennzeichen anbelangt.
Als Alternative zu diesem Paket kann außerdem \Package{ionumbers} verwendet 
werden.

Das Paket \Package*{mathswap} sorgt dafür, dass Trennzeichen direkt vor einer 
Ziffer erkannt und nach bestimmten Vorgaben ersetzt werden. Sollte sich jedoch 
zwischen Trennzeichen und Ziffer Leerraum befinden, wird dieser als solcher
auch gesetzt. Für ein Beispiel zur Verwendung des Paketes sei auf das Tutorial 
\Tutorial{mathswap} in \autoref{sec:exmpl:mathswap} hingewiesen.

\begin{Declaration}{\Macro{commaswap}\Parameter{Trennzeichen}}
\begin{Declaration}{\Macro{dotswap}\Parameter{Trennzeichen}}
\printdeclarationlist%
%
Die beiden Befehle \Macro*{commaswap} und \Macro*{dotswap} sind die zentrale 
Benutzerschnittstelle des Paketes. Das Makro \Macro*{commaswap} definiert das 
Trennzeichen oder den Inhalt, wodurch ein Komma ersetzt werden soll, auf 
welches direkt danach eine Ziffer folgt. Normalerweise setzt \hologo{LaTeX}
nach einem Komma im mathematischen Satz zusätzlich einen horizontalen Abstand.
Bei der Ersetzung durch \Macro*{commaswap} entfällt dieser. Die Voreinstellung
für \Macro*{commaswap} ist deshalb auf ein Komma (,) gesetzt. Mit dem Makro 
\Macro*{dotswap} kann definiert werden, wodurch der Punkt im mathematischen 
Satz ersetzt werden soll, wenn auf diesen direkt anschließend eine Ziffer 
folgt. Da der Punkt im deutschsprachigem Raum zur Gruppierung von Ziffern 
genutzt wird, ist hierfür standardmäßig ein halbes geschütztes Leerzeichen 
definiert (\Macro*{,}).
\end{Declaration}
\end{Declaration}
\begin{Declaration}[v2.02]{\Macro{mathswapon}}
\begin{Declaration}[v2.02]{\Macro{mathswapoff}}
\printdeclarationlist%
%
Die Funktionalität von \Package*{mathswap} kann innerhalb des Dokumentes mit 
diesen beiden Befehlen an- und abgeschaltet werden. Beim Laden des Paketes ist 
es standardmäßig aktiviert.
\end{Declaration}
\end{Declaration}



\section{Das Paket \Package*{twocolfix}}
\DeclarePackage*{twocolfix}%
\index{Zweispaltensatz|?}%
%
Der \hologo{LaTeXe}-Kernel enthält einen Fehler, der Kapitelüberschriften im
zweispaltigen Layout höher setzt, als im einspaltigen. Der 
\hrfn{http://www.latex-project.org/cgi-bin/ltxbugs2html?pr=latex/3126}{Fehler}
ist zwar schon länger bekannt, allerdings noch nicht in \Package{ltxfix2e} 
übernommen worden. Das Paket \Package{twocolfix} behebt das Problem. Eine 
Integration dieses Bugfixes in \KOMAScript{} wurde bereits bei Markus Kohm 
angefragt, jedoch von ihm bis jetzt \hrfn{http://www.komascript.de/node/1681} 
{nicht weiter verfolgt}.



\section{Zukünftige Arbeiten}
Diese Dinge sollen langfristig in das \TUDScript-Bundle eingearbeitet werden.
\subsection*{\Class*{tudscrlttr} auf Basis von \Class*{scrlttr2}}
\DeclareClass{tudscrlttr}
\ToDo[imp,nxt]{Klasse \Class*{tudscrlttr}}[v2.x]
%\chapter[Die Briefklasse \Class*{tudscrlttr}]{Die Briefklasse}
Es soll die Klasse
%
\begin{description}
  \item \Class*{tudscrlttr}
\end{description}
%
für Briefe im \CD der \TnUD entstehen. Auch Vorlagen für Fax und 
Hausmitteilungen sollen dabei abfallen.
\subsection*{Das Paket \Package*{tudscrposter}}
\DeclarePackage{tudscrposter}
\ToDo[imp,nxt]{%
  Paket \Package*{tudscrposter} als Ersatz für \Class{tudmathposter}%
}[v2.x]
Die Funktionalität der Klasse \Class{tudmathposter} soll in ein eigenständiges 
Paket mit dem Namen \Package*{tudscrposter} o.\,ä. überführt werden.
\setpartpreamble{%
  \begin{abstract}
    \hypersetup{linkcolor=red}
    \noindent Für die Verwendung des \TUDScript-Bundles ist es nicht notwendig,
    diesen Teil zu lesen. In \autoref{sec:exmpl} sind insbesondere für 
    \hologo{LaTeX}"=Neulinge sowie neue Anwender des \TUDScript-Bundles mehrere 
    einfache Beispiele sowie umfangreichere Tutorials für dessen Verwendung zu 
    sehen. In \autoref{sec:packages} werden Einsteigern~-- und auch dem bereits 
    versierten \hologo{LaTeX}-Nutzer~-- meiner Meinung nach empfehlenswerte 
    Pakete kurz vorgestellt.
    
    Anwendungshinweise sowie der eine oder andere allgemeine Hinweis bei der 
    Verwendung von \hologo{LaTeXe} wird in \autoref{sec:tips} gegeben. Dabei 
    sind diese durchaus für die Verwendung sowohl des \TUDScript-Bundles als 
    auch anderer \hologo{LaTeX}-Klassen interessant. Für Anregungen, Hinweise, 
    Ratschläge oder Empfehlungen zu weiteren Pakete sowie Tipps bin ich 
    jederzeit empfänglich.
  \end{abstract}
}
\part{Ergänzungen und Hinweise}
\label{part:additional}
\setchapterpreamble{%
  \begin{abstract}
    \hypersetup{linkcolor=red}
    Dieses Kapitel soll den Einstieg und den ersten Umgang mit \TUDScript 
    erleichtern. Dafür werden einige Minimalbeispiele gegeben, die einzelne 
    Funktionalitäten darstellen. Diese sind so reduziert ausgeführt, dass sie 
    sich dem Anwender direkt erschließen sollten. Des Weiteren werden 
    weiterführende und kommentierte Anwendungsbeispiele bereitgestellt. Diese 
    Tutorials sind nicht unmittelbar im Handbuch enthalten sondern werden als 
    externe Dateien bereitgehalten, welche direkt via Hyperlink geöffnet werden 
    können.
  \end{abstract}
}
\chapter{Minimalbeispiele und Tutorials}
\label{sec:exmpl}
\index{Minimalbeispiel|(}
\section{Dokument}
\index{Minimalbeispiel!Dokument}
Hier wird gezeigt, wie die Präambel eines minimalen \hologo{LaTeXe}-Dokumentes 
gestaltet werden sollte. Dieser Ausschnitt kann prinzipiell als Grundlage für 
ein neu zu erstellendes Dokument verwendet werden. Lediglich das Einbinden des 
Paketes \Package*{blindtext} mit \Macro*{usepackage}\PParameter{blindtext} und 
der daraus stammende Befehl \Macro*{blinddocument} kann weggelassen werden.
\includeexample{document}
\section{Dissertation}
\label{sec:exmpl:dissertation}
\index{Minimalbeispiel!Dissertation}
Eine Abschlussarbeit oder ähnliches könnte wie hier gezeigt begonnen werden.
\includeexample{dissertation}
\section{Abschlussarbeit (kollaborativ)}
\label{sec:exmpl:thesis}
\index{Minimalbeispiel!Abschlussarbeit}
\index{Minimalbeispiel!Kollaboratives Schreiben}
Alle zusätzlichen Angaben außerhalb des Argumentes von \Macro{author} werden 
für beide Autoren gleichermaßen übernommen.%
\footnote{In diesem Beispiel \Macro{matriculationyear}.}
Die Angaben innerhalb des Argumentes von \Macro{author} werden den jeweiligen, 
mit \Macro{and} getrennten Autoren zugeordnet.%
\footnote{%
  In diesem Beispiel \Macro{matriculationnumber}, \Macro{dateofbirth} und 
  \Macro{placeofbirth}.
}
Ohne die Verwendung von \Macro{and} kann natürlich auch nur ein Autor 
aufgeführt werden. Außerdem sei auf die Verwendung von \Macro{subject} anstelle 
von \Macro{thesis} mit einem speziellen Wert aus \autoref{tab:thesis} 
hingewiesen.
\includeexample{thesis}
\section{Aufgabenstellung (kollaborativ)}
\label{sec:exmpl:task}
\index{Minimalbeispiel!Aufgabenstellung}
\index{Minimalbeispiel!Kollaboratives Schreiben}
Eine Aufgabenstellung für eine wissenschaftliche Arbeit ist mithilfe der 
Umgebung \Environment{task} oder dem Befehl \Macro{taskform} aus dem Paket 
\Package{tudscrsupervisor} folgendermaßen dargestellt werden.
\includeexample{task}
\section{Gutachten}
\label{sec:exmpl:evaluation}
\index{Minimalbeispiel!Gutachten}
Nach dem Laden des Paketes \Package{tudscrsupervisor} kann ein Gutachten für 
eine wissenschaftliche Arbeit mit der \Environment{evaluation}"=Umgebung oder 
dem Befehl \Macro{evaluationform} erstellt werden.
\includeexample{evaluation}
\section{Aushang}
\label{sec:exmpl:notice}
\index{Minimalbeispiel!Aushang}
Das Paket \Package{tudscrsupervisor} stellt die Umgebung \Environment{notice} 
für das Anfertigen allgemeiner Aushänge sowie den Befehl \Macro*{noticeform} 
für die Ausschreibung wissenschaftlicher Arbeiten bereit.
\includeexample{notice}
\index{Minimalbeispiel|)}
\index{Tutorial|(}
\section{Vorlage für eine wissenschaftlichen Arbeit}
\label{sec:exmpl:treatise}
\index{Tutorial!Abschlussarbeit}
Die meisten Anwender der \TUDScript-Klassen sind Studenten oder angehörige der 
\TnUD, die ihre ersten Schritte mit \hologo{LaTeXe} beim Verfassen einer 
wissenschaftlichen Arbeit oder ähnlichem machen. Während der Einstiegsphase in 
\hologo{LaTeXe} kann ein Anfänger sehr schnell aufgrund der großen Anzahl an 
empfohlenen Pakete sowie der teilweise diametral zueinander stehenden Hinweise 
überfordert sein. Mit dem Tutorial \Tutorial{treatise} soll versucht werden, 
ein wenig Licht ins Dunkel zu bringen. Es erhebt jedoch keinerlei Anspruch, 
vollständig oder perfekt zu sein. Einige der darin vorgestellten Möglichkeiten 
lassen sich mit Sicherheit auch anders, einfacher und/oder besser lösen. 
Dennoch ist es gerade für Neulinge~-- vielleicht auch für den einen oder 
anderen \hologo{LaTeX}"=Veteran~-- als Leitfaden für die Erstellung einer 
wissenschaftlichen Arbeit gedacht.
\ToDo[imp]{Tutorial für eine wissenschaftlichen Arbeit}[v2.02]
\ToDo[imp]{Einarbeiten von paragraph in treatise}[v2.02]
\section{Ein Beitrag zum mathematischen Satz in \NoCaseChange{\hologo{LaTeXe}}}
\label{sec:exmpl:mathtype}
\index{Tutorial!Mathematiksatz}
Das Tutorial \Tutorial{mathtype} richtet sich an alle Anwender, die in ihrem 
\hologo{LaTeX}"=Dokument mathematische Formeln setzen wollen. In diesem wird 
ausführlich darauf eingegangen, wie mit wenigen Handgriffen ein typographisch 
sauberer Mathematiksatz zu bewerkstelligen ist.
\section{Änderung der Trennzeichen im Mathematikmodus}
\label{sec:exmpl:mathswap}
\index{Tutorial!Trennzeichen Mathematikmodus}
Sollen beim Verfassen eines \hologo{LaTeX}"=Dokumentes Daten in einem 
Zahlenformat importiert werden, welches nicht den Gepflogenheiten der 
Dokumentsprache entspricht, kommt es meist zu unschönen Ergebnissen bei der 
Ausgabe. Einfachstes Beispiel sind Daten, in denen als Dezimaltrennzeichen ein 
Punkt verwendet wird, wie es im englischsprachigen Raum der Fall ist. Sollen 
diese in einem Dokument deutscher Sprache eingebunden werden, müssten diese 
normalerweise allesamt angepasst und das ursprüngliche Dezimaltrennzeichen 
durch ein Komma ersetzt werden. Dieser Schritt wird mit dem \TUDScript-Paket 
\Package{mathswap} automatisiert. Wie dies genau funktioniert, wird im Tutorial 
\Tutorial{mathswap} erläutert.
\index{Tutorial|)}
\chapter{Unerlässliche und beachtenswerte Pakete}
\label{sec:packages}
\index{Kompatibilität!Pakete}
\section{Von den neuen Hauptklassen benötigte Pakete}
\label{sec:packages:needed}
\subsection{Erforderliche Pakete bei der Schriftinstallation}
Für die Installation der Schriften sind die folgend genannten Pakete von
\emph{essentieller} Bedeutung und daher \emph{zwingend} notwendig. Das 
Vorhandensein dieser wird durch die jeweiligen Schriftinstallationsskripte
(siehe~\autoref{sec:install}) geprüft und die Installation beim Fehlen eines 
oder mehrerer Pakete mit einer entsprechenden Warnung abgebrochen.
%
\begin{packages}
\item[fontinst]
  Dieses Paket wird für die Installation der Schriften \Univers sowie \DIN 
  benötigt.
\item[cmbright]
  Alle mathematischen Glyphen und Symbole, die nicht in den \Univers-Schriften 
  enthalten sind, werden diesem Paket entnommen.
\item[iwona]
  Sowohl für die \Univers-Schriftfamilie als auch für \DIN werden fehlenden 
  Glyphen und Symbole hieraus entnommen.
\end{packages}
%
Zusätzlich werden die \texttt{Schreibmaschinenschriften} aus dem Paket 
\Package{lmodern} verwendet.

\subsection{Notwendige Pakete für die Verwendung der Hauptklassen}
In diesem \autorefname werden alle Pakete genannt, die von den neuen Klassen 
zwingend benötigt und geladen werden, um den Anwender das mehrmalige Laden 
dieser Pakete oder mögliche Konflikte mit anderen Paketen zu ersparen.
%
\begin{packages}
\item[scrbase]
  Dieses Paket gehört zum \KOMAScript-Bundle und erlaubt das Definieren von 
  Klassenoptionen im Stil von \KOMAScript, welche auch noch nach dem Laden der 
  Klasse mit den Befehlen \Macro*{TUDoption} und \Macro*{TUDoptions} geändert 
  werden können. Von diesem wird das Paket \Package{keyval} geladen, welches 
  das Definieren von Klassen"~ und Paketoptionen sowie Parametern nach dem 
  Schlüssel"=Wert"=Prinzip ermöglicht.
\item[kvsetkeys]
  Hiermit wird das von \Package*{scrbase} geladene Paket \Package*{keyval} 
  verbessert. Unter anderem kann das Verhalten für einen übergebenen, 
  unbekannten Schlüssel festgelegt werden.
\item[scrlayer-scrpage]
  Dieses \KOMAScript-Paket wird für die \PageStyle{tudheadings}-Seitenstile 
  benötigt.
\item[etoolbox]
  Es werden viele Funktionen zum Testen und zur Ablaufkontrolle bereitgestellt 
  und das einfache Manipulieren vorhandener Makros ermöglicht.
\item[geometry]\index{Satzspiegel}
  Das Paket ist essentiell für die \TUDScript-Klassen. Es wird zum Festlegen 
  der Seitenränder respektive des Satzspiegels verwendet.
\item[environ]\index{Befehle!Deklaration}
  Es wird eine verbesserte Deklaration von Umgebungen ermöglicht, bei der auch 
  beim Abschluss der Umgebung auf die übergebenen Parameter zugegriffen werden 
  kann. Dies wird die Neugestaltung der \Environment{abstract}"=Umgebung 
  benötigt. Das Paket lädt \Package{trimspaces}, womit das Entfernen von 
  überflüssigen Leerraum um einen Strings ermöglicht wird.
\item[textcase]\index{Schriftauszeichnung}
  Mit \Macro{MakeTextUppercase} wird die Großschreibung der Überschriften in 
  \DIN erzwungen. Im \emph{Ausnahmefall} kann dies mit \Macro{NoCaseChange} 
  unterbunden werden.
\item[graphicx]\index{Grafiken}
  Dies ist das De-facto-Standard-Paket zum Einbinden von Grafiken. Zum Setzen 
  des Logos der \TnUD im Kopf wird \Macro{includegraphics} genutzt.
\item[xcolor]\index{Farben}
  Damit werden die Farben des \CDs zur Verwendung im Dokument definiert. 
  Genaueres ist bei der Beschreibung von \Package{tudscrcolor} in 
  \autoref{files:tudscrcolor} zu finden.
\end{packages}
%
Soll eines der hier aufgezählten Pakete mit bestimmten Optionen geladen werden, 
so müssen diese bereits \emph{vor} der Definition der Dokumentklasse an das 
entsprechende Paket werden.
%
\begin{Example}
Das Weiterreichen von Optionen an Pakete muss folgendermaßen erfolgen:
\begin{Code}[escapechar=§]
\PassOptionsToPackage§\Parameter{Optionenliste}\Parameter{Paket}§
\documentclass§\OParameter{Klassenoptionen}\PParameter{tudscr\dots}§
\end{Code}
\end{Example}\vspace{-\baselineskip}%



\section{Durch \TUDScript direkt unterstütze Pakete}
%
\begin{packages}
\item[hyperref]\index{Lesezeichen}\index{Querverweise}
  Hiermit können in einem PDF-Dokument Lesezeichen, Querverweise und   
  Hyperlinks erstellt werden. Wird es geladen, sind die Option 
  \Option{tudbookmarks} sowie \Macro{tudbookmark} nutzbar. Das Paket 
  \Package{bookmark} erweitert die Unterstützung nochmals. \Package*{hyperref} 
  beziehungsweise \Package{bookmark} sollte zuletzt in der Präambel eingebunden 
  werden.%
  \footnote{%
    \Package{glossaries} ist eine von wenigen Ausnahmen und muss \textbf{nach} 
    \Package*{hyperref} geladen werden.
  }
\item[isodate]\index{Datum|?}
  Dieses Paket formatiert mit \Macro{printdate}\Parameter{Datum} die Ausgabe 
  eines Datums automatisch in ein spezifiziertes Format. Wird es geladen, 
  werden alle Datumsfelder, welche durch die \TUDScript-Klassen definiert 
  wurden,%
  \footnote{%
    \Macro{date}, \Macro{dateofbirth}, \Macro{defensedate}, \Macro{duedate}, 
    \Macro{issuedate}
  }
  in diesem Format ausgegeben.
\item[multicol]\index{Zweispaltensatz|?}
  Hiermit kann jeglicher beliebiger Inhalt in zwei oder mehr Spalten ausgegeben 
  werden, wobei~-- im Gegensatz zur \hologo{LaTeX}-Option \Option{twocolumn}~-- 
  für einen Spaltenausgleich gesorgt wird. Unterstützt wird das Paket innerhalb 
  der Umgebungen \Environment{abstract} und \Environment{tudpage}.
\item[quoting]\index{Zitate}
  \hologo{LaTeX} bietet von Haus aus \emph{zwei} verschiedene Umgebungen für 
  Zitate und ähnliches. Beide sind in ihrer Ausprägung starr und ignorieren 
  beispielsweise die Einstellungen von \Option{parskip}. Dies wird durch die 
  Umgebung \Environment{quoting} verbessert. Wird das Paket geladen, kommt 
  diese gegebenenfalls innerhalb der \Environment{abstract}"=Umgebung zum 
  Einsatz.
\item[ragged2e]\index{Worttrennung}
  Das Paket verbessert den Flattersatz, indem für diesen die Worttrennung 
  aktiviert wird.
\item[pagecolor]\index{Farben}
  Mit dem Paket kann die Hintergrundfarbe der Seiten im Dokument geändert 
  werden. Nach der Ausgabe einer farbigen Titel"~, Teile"~, oder Kapitelseite 
  wird auf diese zurückgeschaltet.
\item[afterpage]
  Der Befehl \Macro*{afterpage}\Parameter{\dots} kann genutzt werden, um den 
  Inhalt aus dessen Argument direkt nach der Ausgabe der aktuellen Seite 
  auszuführen. Wird es geladen, wird der Titelkopf im \CD für das
  zweispaltige Layout verbessert (\autoref{sec:title}).
\end{packages}



\section{Empfehlenswerte Pakete}
\label{sec:packages:recommended}
In diesem \autorefname wird eine Vielzahl an Paketen genannt und~-- zumeist 
kurz~-- charakterisiert, welche sich bei meiner Arbeit mit \hologo{LaTeX} 
bewährt haben. Für detaillierte Informationen sowie bei Fragen zu den einzelnen 
Paketen sollte die jeweilige Dokumentation zu Rate gezogen werden,%
\footnote{Kommandozeile/Terminal: \Path{texdoc\,\PName{Paketname}}}
das Lesen der hier gegebenen Kurzbeschreibung ersetzt dies in keinem Fall.


\subsection{Pakete zur Verwendung in jedem Dokument}
Die hier vorgestellten Pakete gehören meiner Meinung nach in die Präambel eines 
jeden Dokumentes. Egal, in welcher Sprache das Dokument verfasst wird, sollte 
diese mit dem Paket \Package*{babel} definiert werden~-- auch wenn dies 
Englisch ist. Für deutschsprachige Dokumente ist für eine annehmbare 
Worttrennung das Paket \Package*{hyphsubst} unbedingt zu verwenden.

\begin{packages}
\item[fontenc]\index{Zeichensatzkodierung}
  Das Paket erlaubt Festlegung der Zeichensatzkodierung des Ausgabefonts. Als 
  Voreinstellung ist die Ausgabe als 7"~bit kodierte Schrift gewählt, was unter 
  anderem dazu führt, dass keine echten Umlaute im erzeugten PDF-Dokument 
  verwendet werden. Um auf 8"~bit"~Schriften zu schalten, sollte man
  \Macro*{usepackage}\POParameter{T1}\PParameter{fontenc} nutzen.
\item[selinput]\index{Eingabekodierung}
  Hiermit erfolgt die (automatische) Festlegung der Eingabekodierung. Diese ist 
  vom genutzten \hyperref[sec:tips:editor]{Editors (\autoref{sec:tips:editor})} 
  und den darin gewählten Einstellung abhängig. Zu verwenden ist es wie folgt:
  \begin{Code}
  \usepackage{selinput}
  \SelectInputMappings{adieresis={ä},germandbls={ß}}
  \end{Code}\vspace{-\baselineskip}%
  Dies macht den Quelltext portabel. Außerdem kann so beispielsweise ganz 
  einfach via Copy~\&~Paste ein \hrfn{http://www.komascript.de/minimalbeispiel}%
  {Minimalbeispiel} bei Problemstellungen in einem Forum bereitgestellt werden. 
  Alternativ dazu lässt sich mit dem Paket \Package{inputenc} die zu 
  verwendende Eingabekodierung manuell einstellen
  (\Macro*{usepackage}\OParameter{Eingabekodierung})\PParameter{inputenc}).
\item[microtype]\index{Typographie}
  Dieser Paket kümmert sich um den optischen Randausgleich%
  \footnote{englisch: protrusion, margin kerning}
  und das Nivellieren der Wortzwischenräume%
  \footnote{englisch: font expansion}
  im Dokument. Es funktioniert nicht mit der klassischen \hologo{TeX}-Engine, 
  wohl jedoch mit \hologo{pdfTeX} als auch \hologo{LuaTeX} sowie \hologo{XeTeX}.
\item[babel]\index{Sprachunterstützung}\index{Bezeichner}
  Mit diesem Paket erfolgt die Einstellung der im Dokument verwendeten 
  Sprache(n). Bei mehreren angegebenen Sprachen ist die zuletzt geladene die 
  Hauptsprache des Dokumentes. Die gewünschten Sprachen sollten als nicht als 
  Paketoption sondern als Klassenoption und gesetzt werden, damit auch andere 
  Pakete auf die Spracheinstellungen zugreifen können. Für deutschsprachige 
  Dokumente ist die Option \Option*{ngerman} für die neue oder \Option*{german} 
  für die alte deutsche Rechtschreibung zu verwenden. 
  
  Mit dem Laden von \Package*{babel} und der dazugehörigen Sprachen werden 
  sowohl die Trennmuster als auch die sprachabhängigen Bezeichner angepasst.
  Von einer Verwendung der obsoleten Pakete \Package*{german} beziehungsweise 
  \Package*{ngerman} anstelle von \Package*{babel} wird abgeraten. Für 
  \hologo{LuaLaTeX} und \hologo{XeLaTeX} kann das Paket \Package{polyglossia} 
  genutzt werden.
\item[hyphsubst]\index{Worttrennung|!}
  Die möglichen Trennstellen von Wörtern wird von \hologo{LaTeX} mithilfe 
  eines Algorithmus berechnet. Dieser ist jedoch in seiner ursprünglichen Form 
  für die englische Sprache konzipiert worden. Für deutschsprachige Texte wird 
  die Worttrennung~-- insbesondere für zusammengeschriebenen Wörtern~-- mit dem 
  Paket \Package*{hyphsubst} entscheidend verbessert. Dafür wird ein um 
  Trennstellen ergänztes Wörterbuch aus dem Paket \Package{dehyph-exptl} 
  genutzt. \Package*{hyphsubst} muss bereits \emph{vor} den Dokumentklassen 
  selbst wie folgt geladen werden:
  \begin{Code}[escapechar=§]
  \RequirePackage§\POParameter{ngerman=ngerman-x-latest}\PParameter{hyphsubst}§
  \documentclass§\OParameter{Klassenoptionen}\PParameter{tudscr\dots}§
  \end{Code}\vspace{-\baselineskip}%
  Sollte trotzdem einmal ein bestimmtes Wort falsch getrennt werden, so kann 
  die Worttrennung dieses Wortes manuell und global geändert werden. Dies wird 
  mit dem Befehl \Macro{hyphenation}\PParameter{Sil-ben-tren-nung} gemacht. Es 
  ist zu beachten, dass dies für alle Flexionsformen des Wortes erfolgen 
  sollte. Für eine lokale/temporäre Worttrennung kann mit Befehlen aus dem 
  Paket \Package{babel} gearbeitet werden. Diese sind:
  %
  \vskip\topsep\noindent
  \begin{tabular}{@{}ll}
  \textbf{Beschreibung} & \textbf{Befehl}\tabularnewline
  ausschließliche Trennstellen & \textbackslash-\tabularnewline
  zusätzliche Trennstellen & "'-\tabularnewline
  Umbruch ohne Trennstrich & "'"'\tabularnewline
  Bindestrich, welcher weitere Trennstellen erlaubt & "'=\tabularnewline
  geschützter Bindestrich ohne Umbruch & "'\textasciitilde\tabularnewline
  \end{tabular}
\end{packages}

\subsection{Pakete zur situativen Verwendung}
\subsubsection{Verzeichnisse aller Art}
\index{Verzeichnisse|?}
Neben dem Erstellen des eigentlichen Dokumentes sind für eine wissenschaftliche 
Arbeit meist auch allerhand Verzeichnisse gefordert. Fester Bestandteil ist 
dabei das Literaturverzeichnis, auch ein Abkürzungs- und Formelzeichen- 
beziehungsweise Symbolverzeichnis werden häufig gefordert. Gegebenenfalls wird 
auch noch ein Glossar benötigt. Hier werden die passenden Pakete vorgestellt. 
Für das Erstellen eines Quelltextverzeichnisses sei auf \Package'{listings} 
verwiesen.

\begin{packages}
\item[biblatex]\index{Literaturverzeichnis}
  Das Paket gibt es seit geraumer Zeit und es kann als legitimer Nachfolger zu 
  \Package*{bibtex} gesehen werden. Ähnlich dazu bietet \Package*{biblatex} 
  die Möglichkeit, Literaturdatenbanken einzubinden und verschiedene Stile der 
  Referenzierung und Darstellung des Literaturverzeichnisses auszuwählen. 
  
  Mit \Package*{biblatex} ist die Anpassung eines bestimmten Stiles wesentlich 
  besser umsetzbar als mit \Package*{bibtex}. Wird \Application{biber} für die 
  Sortierung des Literaturverzeichnisses genutzt, ist die Verwendung einer 
  UTF-8-kodierten Literaturdatenbank problemlos möglich. In Verbindung mit 
  \Package*{biblatex} wird die zusätzliche Nutzung des Paketes 
  \Package*{csquotes} sehr empfohlen.
\item[acro]\index{Abkürzungsverzeichnis}
  Soll lediglich ein Abkürzungsverzeichnis erstellt werden, ist dieses Paket 
  die erste Wahl. Es stellt Befehle zur Definition von Abkürzungen sowie zu 
  deren Verwendung im Text und zur sortierten Ausgabe eines Verzeichnisses 
  bereit. Alternativ dazu kann das Paket \Package{acronym} verwendet werden. 
  Die Sortierung des Abkürzungsverzeichnisses muss hier allerdings manuell 
  durch den Anwender erfolgen.
\item[glossaries]\index{Glossar}\index{Abkürzungsverzeichnis}%
  \index{Formelzeichenverzeichnis}\index{Symbolverzeichnis}%
  Dies ist ein sehr mächtiges Paket zum Erstellen eines Glossars sowie 
  Abkürzungs- und Symbolverzeichnisses. Die mannigfaltige Anzahl an Optionen 
  ist zu Beginn eventuell etwas abschreckend. Insbesondere wenn Verzeichnisse 
  für Abkürzungen \emph{und} Formelzeichen beziehungsweise Symbole benötigt 
  werden, sollte man dieses Paket in Erwägung ziehen.
  
  Alternativ dazu kann für ein Symbolverzeichnis auch lediglich eine manuell 
  gesetzte Tabelle genutzt werden. Das hierfür sehr häufig empfohlene Paket 
  \Package{nomencl} bietet meiner Meinung nach demgegenüber keinerlei Vorteile.
\end{packages}

\subsubsection{Grafiken und Abbildungen}
\index{Grafiken|?}
Grafiken für wissenschaftliche Arbeiten sollten als Vektorgrafiken erstellt 
werden, um Skalierbarkeit und hohe Druckqualität zu gewährleisten. Bestenfalls 
folgen diese auch dem Stil der dazugehörigen Arbeit.%
\footnote{%
  Aus anderen Arbeiten übernommene Grafiken sollten meiner Meinung nach für 
  qualitativ hochwertige Dokumente nicht direkt kopiert sondern nach der 
  Vorlage im entsprechenden Format neu erstellt und mit der Referenz auf die 
  Quelle ins Dokument eingebunden werden.
}
Für das Erstellen eigener Vektorgrafiken in \hologo{LaTeX}, die unter anderem 
die \hologo{LaTeX}"=Schriften und das Layout des Hauptdokumentes nutzen, gibt
es zwei mögliche Wege. Entweder, man \enquote{programmiert} die Grafiken, 
ähnlich wie auch das Dokument selber oder man nutzt Zeichenprogramme, die 
wiederum die Ausgabe oder das Weiterreichen von Text an \hologo{LaTeX} 
unterstützen. Für das Programmieren von Grafiken sollen hier die wichtigsten 
Pakete vorgestellt werden. Wie diese zu verwenden sind, ist den dazugehörigen 
Paketdokumentationen zu entnehmen.

\begin{packages}
\item[tikz]
  Dies ist ein sehr mächtiges Paket für das Programmieren von Vektorgrafiken 
  und höchstwahrscheinlich die erste Wahl bei der Verwendung von 
  \hologo{pdfLaTeX}.
\item[pstricks]
  Das Paket \Package*{pstricks} stellt die zweite Variante zum Programmieren 
  von Grafiken dar. Mit diesem Paket hat man \emph{noch} mehr Möglichkeiten bei 
  der Erstellung eigener Grafiken, da man mit \Package*{pstricks} auf 
  PostScript zugreifen kann und einige der bereitgestellten Befehle davon rege 
  Gebrauch machen. Der daraus resultierende Nachteil ist, dass mit 
  \Package*{pstricks} die direkte Verwendung von \hologo{pdfLaTeX} nicht 
  möglich ist.
  
  Die Grafiken aus den \Environment{pspicture}"=Umgebungen müssen deshalb erst 
  über den Pfad \Path{latex \textrightarrow{} dvips \textrightarrow{} ps2pdf}
  in PDF"~Dateien gewandelt werden. Diese lassen sich von \hologo{pdfLaTeX} 
  anschließend als Abbildungen einbinden. Um dieses Vorgehen zu ermöglichen, 
  können folgende Pakete genutzt werden:
  %
  \begin{packages}
  \item[pst-pdf]
    Dieses Paket stellt die prinzipiellen Methoden für den Export bereit. Die 
    einzelnen Aufrufe zur Kompilierung von DVI über PostScript zu PDF müssen 
    manuell durchgeführt werden.
  \item[auto-pst-pdf]
    Das Paket automatisiert die Erzeugung der \Package*{pstricks}"=Grafiken. 
    Dafür muss \hologo{pdfLaTeX} mit der Option \Path{-{}-shell-escape} 
    aufgerufen werden.
  \item[pdftricks2]
    Ein weiteres Paket mit der gleichen Intention wie \Package*{auto-pst-pdf}, 
    allerdings anders implementiert. Auch hier ist für \hologo{pdfLaTeX} die 
    Option \Path{-{}-shell-escape} notwendig.
  \end{packages}
  %
\item[standalone]
  Sollte \Package*{tikz} und/oder \Package*{pstricks} eingesetzt werden, kann 
  das Paket \Package*{standalone} genutzt werden, um die Grafiken einerseits 
  als eigenständiges Dokument übersetzen zu können und andererseits diese 
  Grafiken mit dem Hauptdokument zu kompilieren. Damit muss beim Erstellen oder 
  Ändern einer Grafik nicht immer das vollständige Hauptdokument mit kompiliert 
  werden.
\end{packages}
%
Als Möglichkeit des Zeichnens einer Grafik mit einem Bildbearbeitungsprogramm, 
welches die Weiterverarbeitung durch \hologo{LaTeX} erlaubt, möchte ich auf die 
freien Programme \Application{LaTeXDraw} und \Application{Inkscape} verweisen. 
Insbesondere das zuletzt genannte Programm ist sehr empfehlenswert. Für die 
erstellten Grafiken kann man den Export für die Einbindung in \hologo{LaTeX} 
manuell durchführen. In \autoref{sec:tips:svg} wird vorgestellt, wie sich dies 
automatisieren lässt.

\subsubsection{Gleitobjekte}
\index{Gleitobjekte|?}
\index{Tabellen}\index{Grafiken}
Es werden Pakete für die Beeinflussung von Aussehen, Beschriftung und 
Positionierung von Gleitobjekten vorgestellt. Unter \autoref{sec:tips:floats} 
sind außerdem Hinweise zur manuellen Manipulation der Gleitobjektplatzierung zu 
finden.

\begin{packages}
\item[caption]\index{Gleitobjekte!Beschriftung}
  Die \KOMAScript-Klassen bietet bereits einige Möglichkeiten zum Setzen der 
  Beschriftungen für Gleitobjekte. Dieses Paket ist daher meist nur in gewissen
  Ausnahmefällen für spezielle Anweisungen notwendig, allerdings auch bei der 
  Verwendung unbedenklich.
\item[subcaption]\index{Gleitobjekte!Beschriftung}
  Diese Paket kann zum einfachen Setzen von Unterabbildungen oder "~tabellen 
  mit den entsprechenden Beschriftungen genutzt werden. Das dazu alternative 
  Paket \Package{subfig} sollte vermieden werden, da es nicht mehr gepflegt 
  wird und es mit diesem im Zusammenspiel mit anderen Paketen des Öfteren zu 
  Problemen kommt. Sollte der Funktionsumfang von \Package*{subcaption} nicht 
  ausreichen, kann anstelle dessen das Paket \Package*{floatrow} verwendet 
  werden, welches ähnliche Funktionalitäten wie \Package{subfig} bereitstellt.
\item[floatrow]\index{Gleitobjekte!Beschriftung}
  Mit diesem Paket können global wirksame Einstellungen und Formatierungen für 
  \emph{alle} Gleitobjekte eines Dokumentes vorgenommen werden. So kann unter 
  anderem die verwendete Schrift (\Macro*{floatsetup}\PParameter{font=\dots}) 
  innerhalb der Umgebungen \Environment*{float} und \Environment*{table} 
  eingestellt werden. Das typographisch richtige Setzen der Beschriftungen von 
  Abbildungen als Unterschriften 
  (\Macro*{floatsetup}\POParameter{figore}\PParameter{capposition=bottom})
  sowie Tabellen als Überschriften 
  (\Macro*{floatsetup}\POParameter{table}\PParameter{capposition=top})
  kann automatisch erzwungen werden~-- unabhängig von der Position des Befehls 
  zur Beschriftung \Macro{caption} innerhalb der Gleitobjektumgebung.
  
  Wird das Verhalten so wie empfohlen mit dem \Package*{floatrow}-Paket 
  eingestellt, sollte für eine richtige Platzierung der Tabellenüberschriften  
  die \KOMAScript-Option \Option{captions}[tableheading] genutzt werden.
\item[placeins]\index{Gleitobjekte!Platzierung}
  Mit diesem Paket kann die Ausgabe von Gleitobjekten vor Kapiteln und wahlweise
  Unterkapiteln erzwungen werden.
\item[flafter]\index{Gleitobjekte!Platzierung}
  Dieses Paket erlaubt die frühestmögliche Platzierung von Gleitobjekten im 
  ausgegeben Dokument erst an der Stelle ihres Auftretens im Quelltext. Sie 
  werden dementsprechend nie vor ihrer Definition am Anfang der Seite 
  erscheinen.
\end{packages}

\subsubsection{Listen und Tabellen}
\index{Listen|?}\index{Tabellen|?}
Für den Tabellensatz in \hologo{LaTeX} werden von Haus aus die Umgebungen 
\Environment*{tabbing} und \Environment*{tabular} beziehungsweise 
\Environment*{tabular*} bereitgestellt, welche in ihrer Funktionalität meist 
für einen qualitativ hochwertigen Tabellensatz nicht ausreichen. Es werden 
deshalb Pakete vorgestellt, die zusätzlich verwendet werden können. Ebenfalls 
können die Umgebungen für Auflistungen in \hologo{LaTeX} verbessert werden.

\begin{packages}
\item[enumitem]
  Das Paket \Package*{enumitem} erweitert die rudimentären Funktionalitäten der 
  \hologo{LaTeX}"=Standardlisten \Environment{itemize}, \Environment{enumerate}
  sowie \Environment{description} und ermöglicht die individuelle Anpassung 
  dieser durch die Bereitstellung vieler optionale Parameter nach dem
  Schlüssel"=Wert"=Prinzip. 
  
  Eine von mir sehr häufig genutzte Funktion ist beispielsweise die Entfernung 
  des zusätzlichen Abstand zwischen den einzelnen Einträgen einer Liste mit 
  \Macro*{setlist}\PParameter{noitemsep}.
\item[array]
  Dieses Paket ermöglicht die erweiterte Definition von Tabellenspalten sowie 
  das Erstellen neuer Spaltentypen mit \Macro*{newcolumntype}. Außerdem kann 
  mit \Macro*{extrarowheight} die Höhe der Zeilen einer Tabelle angepasst 
  werden.
\item[multirow]
  Es wird der Befehl \Macro*{multirow} definiert, der~-- ähnlich zum Makro 
  \Macro*{multicolumn}~-- das Zusammenfassen von mehreren Zeilen in einer 
  Spalte ermöglicht.
\item[booktabs]
  Für einen guten Tabellensatz mit \hologo{LaTeX} gibt es bereits zahlreiche 
  \hrfn{http://userpage.fu-berlin.de/latex/Materialien/tabsatz.pdf}{Tipps} im 
  Internet zu finden. Zwei Regeln sollten dabei definitiv beachtet werden:
  %
  \begin{enumerate}[itemindent=0pt,labelwidth=*,labelsep=1em,label=\Roman*.]
  \makeatletter
  \item@packages keine vertikalen Linien
  \item@packages keine doppelten Linien
  \makeatother
  \end{enumerate}
  %
  Das Paket \Package*{booktabs} ist für den Satz von hochwertigen Tabellen~-- 
  zusammen mit der deutschsprachigen Dokumentation \Package*{booktabs-de}~-- 
  eine große Hilfe und stellt neue Befehle für horizontale Linien bereit.
\item[widetable]
  Mit der Standard"=\hologo{LaTeX}"=Umgebung \Environment*{tabular*} kann eine 
  Tabelle mit einer definierten Breite gesetzt werden. Dieses Paket stellt die 
  Umgebung \Environment*{widetable} zur Verfügung, die als Alternative genutzt 
  werden kann und eine symmetrische Tabelle erzeugt.
\item[tabularx]
  Auch mit diesem Paket kann die Breite eine Tabelle spezifiziert werden. Dafür 
  wird der neue Spaltentyp \PValue{X} definiert, welcher als Argument der 
  \Environment*{tabularx}"=Umgebung beliebig häufig angegeben werden kann
  (\Macro*{begin}\PParameter{tabularx}\Parameter{Breite}\Parameter{Spalten}). 
  \PValue{X}"~Spalten entsprechen denen vom Typ~\PValue{p}\OParameter{Breite}, 
  die Breite wird allerdings aus der gegebenen Tabellenbreite und dem 
  benötigten Platz der Standardspalten automatisch berechnet.
\item[longtable]
  Sollen mehrseitige Tabellen mit Seitenumbruch erstellt werden, ist dieses 
  Paket das Mittel der ersten Wahl. Für die Kombination mehrseitiger Tabellen 
  mit einer \Environment*{tabularx}"=Umgebung können die Pakete 
  \Package{ltablex} oder besser noch \Package*{ltxtable} verwendet werden.
\item[ltxtable]
  Wie bereits erwähnt sollte dieses Paket für mehrseitige Tabellen, die mit der 
  Umgebung \Environment*{tabularx} erstellt wurden, verwendet werden. 
  Alternativ dazu kann man auch \Package*{tabu} nutzen.
\item[tabu]
  Dies ist ein relativ neues Paket, welches versucht, viele der zuvor genannten 
  Funktionalitäten zu implementieren und weitere bereitzustellen. Dafür werden 
  die Umgebungen \Environment*{tabu} und \Environment*{longtabu} definiert. Es 
  kann als Alternative für \Package*{tabularx} verwendet werden und ist 
  insbesondere als Ersatz für das Paket \Package*{ltxtable} empfehlenswert.
\item[tabularborder]
  Bei Tabellen wird zwischen Spalten automatisch ein horizontaler Abstand 
  (\Length{tabcolsep}) gesetzt~-- besser gesagt jeweils vor und nach einer 
  Spalte. Dies geschieht auch \emph{vor} der ersten und \emph{nach} der letzten 
  Spalte. Dieser zusätzliche Platz an den äußeren Rändern kann störend wirken, 
  insbesondere wenn die Tabelle über die komplette Textbreite gesetzt wird. Mit 
  dem Paket \Package*{tabularborder} kann dieser Platz automatisch entfernt 
  werden.
  
  Dies funktioniert allerdings nur mit der \Environment*{tabular}"=Umgebung. 
  Die Tabellen aus den Paketen \Package*{tabularx} und \Package*{tabu} werden 
  nicht unterstützt. Wie dieser Abstand bei diesen manuell entfernt werden 
  kann, ist in einem Beispiel unter \autoref{sec:tips:table} zu finden.
\end{packages}

\subsubsection{Typographie und Layout}
\index{Typographie}
%
\begin{packages}
\item[setspace]\index{Zeilenabstand}
  Die Vergrößerung des Zeilenabstandes wird:
  %
  \begin{enumerate}[itemindent=0pt,labelwidth=*,labelsep=1em,label=\Roman*.]
  \makeatletter
  \item@packages viel zu häufig und völlig unnötig gefordert und
  \item@packages schließlich auch noch zu groß gewählt.
  \makeatother
  \end{enumerate}
  %
  Die Forderung nach Erhöhung des Zeilenabstandes~-- in der Typographie als 
  Durchschuss bezeichnet~-- kommt noch aus den Zeiten der Textverarbeitung mit 
  der Schreibmaschine. Ein einzeiliger Zeilenabstand bedeutete hier, dass die 
  Unterlängen der oberen Zeile genau auf der Höhe der Oberlängen der folgenden 
  Zeile lagen. Ein anderthalbzeiliger Zeilenabstand erzielte hier somit einen 
  akzeptablen Durchschuss. Eine Erhöhung des Durchschusses bei der Verwendung 
  von \hologo{LaTeX} ist an und für sich nicht notwendig. Sinnvoll ist dies 
  nur, wenn im Fließtext serifenlose Schriften zum Einsatz kommen, um die damit 
  verbundene schlechte Lesbarkeit etwas zu verbessern.
  
  Ist die Erhöhung des Durchschusses wirklich notwendig, sollte das Paket 
  \Package*{setspace} verwendet werden. Dieses stellt den Befehl 
  \Macro*{setstretch}\Parameter{Faktor} zur Verfügung, mit dem der Durchschuss 
  beziehungsweise Zeilenabstand angepasst werden kann. Der Wert des Faktors 
  ist standardmäßig auf~1 gestellt und sollte maximal bis~1.25 vergrößert 
  werden. Der Befehl \Macro*{onehalfspacing} aus diesem Paket setzt diesen Wert 
  auf eben genau~1.25. Allerdings ist hier anzumerken, dass die Vergrößerung 
  des Zeilenabstandes~-- so wie ich es mir angelesen habe~-- aus der Sicht 
  eines Typographen keine Spielerei ist sondern vielmehr allein der Lesbarkeit 
  des Textes dient und möglichst gering ausfallen sollte.
  
  Ziel ist es, beim Lesen nach dem Beenden der aktuellen Zeile das Auffinden 
  der neuen Zeile zu vereinfachen. Bei Serifen ist dies durch die Betonung der 
  Grundlinie sehr gut möglich. Bei serifenlosen Schriften~-- wie der im \CD der 
  \TnUD verwendeten \Univers~-- ist dies schwieriger und ein erweiterter 
  Abstand der   Zeilen kann dabei durchaus hilfreich sein. Jedoch sollte nicht 
  nach dem Motto \enquote{viel hilft viel} verfahren werden. In diesem Dokument 
  wurde als Faktor für den Zeilenabstand \Macro*{setstretch}\PParameter{1.1} 
  gewählt. Nach einer Einstellung des Zeilenabstandes sollte der Satzspiegel 
  unbedingt mit \Macro{recalctypearea} neu berechnet werden. Siehe dazu auch 
  \autoref{sec:tips:headings} sowie \autoref{sec:tips:headline}.
\item[csquotes]\index{Zitate}
  Das Paket stellt unter anderem den Befehl \Macro{enquote}\Parameter{Zitat} 
  zur Verfügung, welcher Anführungszeichen in Abhängigkeit der gewählten 
  Sprache setzt. Außerdem werden weitere Kommandos und Optionen für die 
  spezifischen Anforderungen des Zitierens bei wissenschaftlichen Arbeiten 
  angeboten. Außerdem wird es durch \Package{biblatex} unterstützt und sollte 
  zumindest bei dessen Verwendung geladen werden.
\item[xspace]\index{Befehle!Deklaration}
  Mit \Package*{xspace} kann bei der Definition eigener Makros der Befehl 
  \Macro*{xspace} genutzt werden. Dieser setzt ein gegebenenfalls notwendiges 
  Leerzeichen automatisch. In \autoref{sec:tips:xspace} ist die Definition 
  eines solchen Befehls exemplarisch ausgeführt.
\item[xpunctuate]\index{Befehle!Deklaration}
  Die Funktionalität von \Package*{xspace} wird um die Beachtung von 
  Interpunktionen erweitert.
\item[ellipsis]\index{Befehle!Deklaration}
  In \hologo{LaTeX} folgten den Befehlen für Auslassungspunkte (\Macro*{dots} 
  und \Macro*{textellipsis}) \emph{immer} ein Leerzeichen. Dies kann unter 
  Umständen unerwünscht sein. Mit dem Paket \Package*{ellipsis} wird das 
  folgende Leerzeichen~-- im Gegensatz zum Standardverhalten~-- nur gesetzt, 
  wenn ein Satzzeichen und kein Buchstabe folgt.
\makeatletter\item@packages[\Application{DeLig}]\makeatother
  \index{Typographie}\index{Ligaturen}
  Hierbei handelt es sich um ein Java-Script, welches anhand eines Wörterbuches 
  falsche Ligaturen innerhalb eines Dokumentes automatisiert entfernt. Wird 
  \Univers verwendet ist dies jedoch nicht notwendig, da diese keinerlei 
  Ligaturen enthält, die insbesondere in deutschen Texten für einen guten Satz 
  manuell aufgelöst werden müssten.%
  \footnote{%
    Das sind ff, fi, fl, ffi, und ffl bei den \hologo{LaTeX}"=Standardschriften.
  }
  Mit \hologo{LuaLaTeX} als Dokumentprozessor kann alternativ dazu auch 
  \Package{selnolig} verwendet werden.
\item[noindentafter]
  Mit diesem Paket lassen sich automatische Absatzeinzüge für selbst zu 
  bestimmende Befehle und Umgebungen unterdrücken.
\item[balance]\index{Zweispaltensatz}
  Dieses Paket ermöglicht einen Spaltenausgleich im Zweispaltensatz auf der 
  letzten Dokumentseite. Alternativ dazu kann auch \Package{multicol} verwendet 
  werden.
\end{packages}

\subsubsection{Schriften, Sonderzeichen und Rechtschreibung}
%
\begin{packages}
\item[lmodern]\index{Schriftart}
  Soll mit den klassischen \hologo{LaTeX}"=Standardschriften gearbeitet werden, 
  empfiehlt sich die Verwendung des Paketes \Package*{lmodern}. Dieses 
  verbessert die Darstellung der Computer~Modern sowohl am Bildschirm als auch 
  beim finalen Druck.
\item[cfr-lm]\index{Schriftart}
  Dieses experimentelle Paket liefert weitere Schriftschnitte für das Paket 
  \Package{lmodern}.
\item[libertine]\index{Schriftart}
  Das Paket stellt die Schriften Linux~Libertine und Linux~Biolinum zur 
  Verfügung.
  %
  \begin{packages}
    \item[libgreek]
      Es werden griechische Buchstaben für Linux~Libertine bereitgestellt.
    \item[newtxmath]
      Das Paket aus dem \Package*{newtx}-Bundle erlaubt die Verwendung der 
      Linux~Libertine im Mathematikmodus. Es wird mit
      \Macro*{usepackage}\POParameter{libertine}\PParameter{newtxmath} geladen.
  \end{packages}
  %
\item[mweights]\index{Schrift!Stärke}
  Werden Schriften aus unterschiedlichen Paketen verwendet, kann es unter 
  Umständen zu Problemen bei den Schriftstärken der Schriften kommen. 
  Normalerweise gibt es bei \hologo{LaTeXe} die Schriftfamilien für 
  Serifenschriften (\Macro*{rmfamily}), serifenlose Schrift (\Macro*{sffamily}) 
  sowie die Schreibmaschinenschriften (\Macro*{ttfamily}). Die Schriftstärke 
  dieser drei Schriftfamilien wird für gewöhnlich einheitlich über die Beiden 
  Befehle \Macro*{mddefault} und \Macro*{bfdefault} festgelegt. Mit dem Paket 
  \Package*{mweights} kann die Schriftstärke für jede der drei Schriftfamilien 
  individuell festgelegt werden.
\item[relsize]\index{Schrift!Größe}
  Mithilfe dieses Paketes kann die Größe einer Textauszeichnung relativ zur 
  aktuell gewählten Schriftgröße gesetzt werden.
\item[textcomp]\index{Sonderzeichen}
  Es werden verschiedene zusätzliche Symbole wie beispielsweise das Promille- 
  oder Eurozeichen sowie Pfeile im Text zur Verfügung gestellt.
\item[fontspec]
  Wird als Dokumentprozessor nicht \hologo{pdfLaTeX} sondern \hologo{XeLaTeX} 
  oder \hologo{LuaLaTeX} verwendet, können mit dem Paket \Package*{fontspec} 
  auch Schriften im OpenType-Format eingebunden werden, womit sich die Auswahl 
  der verwendbaren Schriften in einem \hologo{LaTeX}"=Dokument stark erweitert. 
  Für die Verwendung von OpenType"=Schriften müssen diese lediglich für das 
  Betriebssystem jedoch nicht speziell für \hologo{LaTeX} installiert sein.
\item[spelling]\index{Rechtschreibung}
  Wird \hologo{LuaLaTeX} als Prozessor verwendet, wird mit diesem Paket der 
  reine Textanteil aus dem \hologo{LaTeX}"~Dokument extrahiert~-- wobei Makros 
  und aktive Zeichen entfernt werden~-- und in eine separate Textdatei 
  geschireben. Anschließend kann diese Datei mit einer externen Software zur 
  Rechtschreibprüfung wie \Application{GNU Aspell} oder \Application{Hunspell} 
  analysiert werden. Wird durch dieses Programm eine Liste falsch geschriebener 
  Wörter ausgegeben, können diese mit \Package*{spelling} im PDF"~Dokument 
  hervorgehoben werden.
\item[lua-check-hyphen]\index{Worttrennung}
  Mit diesem Paket lassen sich bei der Verwendung \hologo{LuaLaTeX} 
  Trennstellen am Zeilenende zur Prüfung markieren.
\end{packages}

\subsubsection{Mathematiksatz}
\index{Mathematiksatz}
Dies sind Pakete, die Umgebungen und Befehle für den Mathematiksatz sowie das 
Setzen von Einheiten und Zahlen im Allgemeinen anbieten.

\begin{packages}
  \item[mathtools]
    Dieses Paket stellt für das De-facto-Standard-Paket \Package{amsmath} für 
    Mathematikumgebungen Bugfixes zur Verfügung und erweitert dieses.
  \item[sansmath]
    Sollten die normalen \hologo{LaTeX}-Schriften Computer~Modern verwendet 
    werden, kann man dieses Paket zum serifenlosen Setzen mathematischer 
    Ausdrücke nutzen. Für die \TUDScript-Hauptklassen sei hierzu auf die Option
    \Option{sansmath} verwiesen.
  \item[sfmath]
    Diese Paket verfolgt ein ähnliches Ziel, kann jedoch im Gegensatz zu 
    \Package*{sansmath} nicht nur für Computer~Modern sondern mit der 
    entsprechenden Option auch für Latin~Modern, Helvetica und 
    Computer~Modern~Bright verwendet werden.
  \item[mathastext]
     Mit dem Paket wird das Ziel verfolgt, aus der genutzten Schrift für den 
     Fließtext alle notwendigen Zeichen für den Mathematiksatz zu extrahieren.
  \item[bm]
    Das Paket bietet mit \Macro*{bm} eine Alternative zu \Macro*{boldsymbol} im 
    \hrfn{http://tex.stackexchange.com/questions/3238}{Mathematiksatz}.
\end{packages}
%
Die korrekte Formatierung von Zahlen ist häufig ein Problem bei der Verwendung 
von \hologo{LaTeX}. Insbesondere, wenn in einem deutschsprachigen Dokument 
Daten im typischen englischsprachigen Format verwendet werden, kommt es zu 
Problemen. Dafür wird im \TUDScript-Bundle das Paket \Package{mathswap} 
bereitgestellt. Dennoch gibt es zu diesem auch Alternativen.
%
\begin{packages}\index{Trennzeichen}
  \item[icomma]
    Wird im Mathematikmodus nach dem Komma ein Leerzeichen gesetzt, wird dies 
    bei der Ausgabe beachtet. Der Verfasser muss sich demzufolge jederzeit 
    selbst um die typographisch korrekte Ausgabe kümmern.
  \item[ziffer]
    Für deutschsprachige Dokumente wird das Komma als Dezimaltrennzeichen 
    zwischen zwei Ziffern definiert. Folgt dem Komma keine Ziffer, wird 
    jederzeit der obligatorische Freiraum gesetzt, was meiner Meinung nach 
    besser als das Verhalten von \Package*{icomma} ist.
  \item[ionumbers]
    Dieses Paket ist mir tatsächlich erst bei der Arbeit an \Package*{mathswap} 
    bekannt geworden. Es bietet mehr Funktionalitäten und kann als Alternative 
    dazu betrachtet werden.
\end{packages}
%
Für das typographisch korrekte Setzen von Einheiten~-- ein halbes Leerzeichen 
zwischen Zahl und \emph{aufrecht} gesetzter Einheit~-- gibt es zwei gut 
nutzbare Pakete.
%
\begin{packages}\index{Einheiten}
\item[units]
  Dies ist ein einfaches und sehr zweckdienliches Paket zum Setzen von 
  Einheiten und für die meisten Anforderungen völlig ausreichend.
\item[siunitx]
  Dieses Paket ist in seinem Umfang im Vergleich deutlich erweitert. Mir hat 
  sich persönlich noch nicht erschlossen, was genau die daraus resultierenden 
  Vorteile sind. Damit das Paket in deutschsprachigen Dokumenten alle Ziffern 
  richtig setzt, muss zumindest die Lokalisierung angegeben werden. Mehr dazu 
  in \autoref{sec:tips:siunitx}.
\end{packages}
%
Weitere Hinweise und Anwendungsfälle zur mathematischen Typographie werden in 
\autoref{sec:exmpl:mathtype} sowie \autoref{sec:exmpl:mathswap} gegeben.

\subsubsection{Die kleinen und großen Helfer\dots}
Hier taucht alles auf, was sich nicht eignete, in die vorherigen Kategorien 
eingeordnet zu werden.
%
\begin{packages}
\item[xparse]\index{Befehle!Deklaration}
  Dieses mächtige Paket entstammt dem \hologo{LaTeX3}-Projekt und bietet für 
  die Erstellung eigener Befehle und Umgebungen einen alternativen Ansatz zu 
  den bekannten \hologo{LaTeX}"=Deklarationsbefehlen \Macro*{newcommand} und 
  \Macro*{newenvironment} sowie deren Derivaten. Mit \Package*{xparse} wird es 
  möglich, obligatorische und optionale Argumente an beliebigen Stellen 
  innerhalb des Befehlskonstruktes zu definieren. Auch die Verwendung anderer 
  Zeichen als eckige Klammern für die Spezifizierung eines optionalen 
  Argumentes ist möglich.
\item[calc]\index{Berechnungen}
  Normalerweise können Berechnungen nur mit Low-Level-\hologo{TeX}-Primitiven 
  im Dokument durchgeführt werden. Dieses Paket stellt eine einfachere Syntax 
  für Rechenoperationen und Befehle zur Bestimmung der Höhe und Breite  
  bestimmter Textauszüge bereit.
\item[bookmark]\index{Lesezeichen}\index{Querverweise}
  Dieses Paket verbessert und erweitert die von \Package{hyperref} angebotenen 
  Möglichkeiten zur Erstellung von Lesezeichen~-- auch Outline"=Einträge~-- im 
  PDF-Dokument. Beispielsweise können Schriftfarbe- und "~stil geändert werden.
\item[mwe]\index{Minimalbeispiel|!}
  Hiermit lassen sich sehr einfach Minimalbeispiele mit Abbildungen erzeugen.
\item[varioref]\index{Querverweise}
  Mit diesem Paket lassen sich sehr gute Verweise auf Seiten erzeugen. 
  Insbesondere, wenn der Verweis auf die aktuelle, die vorhergehende oder 
  nachfolgende sowie im zweiseitigen Satz auf die gegenüberliegende Seite 
  erfolgt, werden passende Textbausteine für diesen verwendet.
\item[marginnote]\index{Randnotizen}
  Randnotizen, welche mit \Macro*{marginpar} erzeugt werden, sind spezielle 
  Gleitobjekte in \hologo{LaTeX}. Dies kann dazu führen, dass eine Notiz am 
  Blattrand nicht direkt da gesetzt wird, wo diese intendiert war. Dieses Paket 
  stellt den Befehl \Macro*{marginnote} für nicht"~gleitende Randnotizen zur 
  Verfügung. Alternativ dazu kann man auch \Package*{mparhack} verwenden.
\item[todonotes]\index{Randnotizen}
  Mit \Package*{todonotes} können noch offene Aufgaben in unterschiedlicher 
  Formatierung am Blattrand oder im direkt Fließtext ausgegeben werden. Aus 
  allen Anmerkungen lässt sich eine Liste aller offenen Punkte erzeugen.
\item[listings]\index{Quelltexte einbinden}\index{Quelltextverzeichnis}%
  Dieses Paket eignet sich hervorragend zur Quelltextdokumentation in 
  \hologo{LaTeX}. Es bietet die Möglichkeit, externe Quelldateien einzulesen 
  und darzustellen sowie die Syntax in Abhängigkeit der verwendeten 
  Programmiersprache hervorzuheben. Zusätzlich lässt sich ein Verzeichnis mit 
  allen eingebundenen sowie direkt im Dokument angegebenen Quelltextauszügen 
  erstellen.
  
  Wird \Package*{listings} in Dokumenten mit UTF-8-Kodierung verwendet, sollte 
  direkt nach dem Laden des Paketes in der Präambel Folgendes hinzugefügt 
  werden:
  \begin{Code}
  \lstset{%
    inputencoding=utf8,extendedchars=true,
    literate=%
      {ä}{{\"a}}1 {ö}{{\"o}}1 {ü}{{\"u}}1
      {Ä}{{\"A}}1 {Ö}{{\"O}}1 {Ü}{{\"U}}1
      {~}{{\textasciitilde}}1 {ß}{{\ss}}1
  }
  \end{Code}\vspace{-\baselineskip}%
\item[chngcntr]\index{Zählermanipulation}
  Das Paket erlaubt die Manipulation aller möglichen, bereits definierten 
  \hologo{LaTeX}-Zähler. Es können Zähler so umdefiniert werden, dass sie bei 
  der Änderung eines anderen Zählers automatisch zurückgesetzt werden oder eben 
  nicht. Ein kleines Beispiel dazu ist in \autoref{sec:tips:counter} zu finden.
\item[filemod]
  Wird entweder \hologo{pdfLaTeX} oder \hologo{LuaLaTeX} als Prozessor 
  eingesetzt, können mit diesem Paket das Änderungsdatum zweier Dateien 
  miteinander verglichen und in Abhängigkeit davon definierbare Aktionen 
  ausgeführt werden.
\item[coseoul]
  Mit diesem Paket kann man die Struktur der Gliederung relativ angeben. Es 
  wird keine absolute Gliederungsebene (\Macro*{chapter}, \Macro*{section}) 
  angegeben sondern die Relation zwischen vorheriger und aktueller Ebene 
  (\Macro*{levelup}, \Macro*{levelstay}, \Macro*{leveldown}).
\item[dprogress]\index{Debugging}
  Das Paket schreibt bei der Kompilierung des Dokumentes die Gliederung in die 
  Logdatei. Dies kann im Fehlerfall beim Auffinden des Problems im Dokument 
  helfen. Allerdings werden dafür die Gliederungsebenen so umdefiniert, dass 
  diese keine optionalen Argumente mehr unterstützen,was jedoch für die 
  \TUDScript-Klassen von essentieller Bedeutung ist. Zum Debuggen kann es 
  trotzdem sporadisch eingesetzt werden.
\end{packages}

\subsubsection{Bugfixes}
%
\begin{packages}
\item[scrhack]
  Das Paket behebt Kompatibilitätsprobleme der \KOMAScript-Klassen mit den 
  Paketen \Package{hyperref}, \Package{float}, \Package{floatrow} und
  \Package{listings}. Es ist durchaus empfehlenswert, jedoch sollte man 
  unbedingt die Dokumentation beachten.
\item[fixltx2e]
  Dieses Paket enthält Bugfixes für \hologo{LaTeXe}. Da diese eventuell zu 
  Inkompatibilitäten mit früheren Versionen führen könnten, wurden diese nicht 
  in den \hologo{LaTeXe}-Kernel eingepflegt.
\item[mparhack]
  Zur Behebung falsch gesetzter Randnotizen wird ein Bugfix für 
  \Macro*{marginpar} bereitgestellt. Alternativ dazu kann man auch 
  \Package*{marginnote} verwenden.
\item[etex]
  Das Paket kann genutzt werden, falls die standardmäßig maximale Anzahl der 
  \hologo{LaTeX}-Register für Längen, Zähler etc. überschritten wurde.
\end{packages}
\chapter{Praktische Tipps \& Tricks}
\label{sec:tips}
\section{\NoCaseChange{\hologo{LaTeX}}-Editoren}
\label{sec:tips:editor}
Hier werden die gängigsten Editoren zum Erzeugen von \hologo{LaTeX}"=Dateien 
genannt. Ich persönlich bin mittlerweile sehr überzeugter Nutzer von 
\Application{\hologo{TeX}studio}, da dieser viele Unterstützungs- und 
Assistenzfunktionen bietet. Neben diesen gibt es noch weitere, gut nutzbare 
\hologo{LaTeX}-Editoren. Egal, für welchen Editor man sich letztendlich 
entscheidet, sollte dieser auf jeden Fall eine Unicode"=Unterstützung~(UTF-8) 
enthalten:
%
\begin{itemize}
\item \Application{\hologo{TeX}maker}
\item \Application{Kile}
\item \Application{\hologo{TeX}works}
\item \Application{\hologo{TeX}lipse}~-- Plug-in für \Application{Eclipse}
\item \Application{\hologo{TeX}nicCenter}
\item \Application{WinEdt}
\item \Application{LEd}~-- früher \hologo{LaTeX}~Editor
\item \Application{\hologo{LyX}}~-- grafisches Front"~End für \hologo{LaTeX}
\end{itemize}
%
Für den Editor\Application{\hologo{TeX}studio} werden im GitHub-Repository
\hrfn{https://github.com/tud-cd/tudscr/tree/master/addon/texstudio}{tudscr/addon/texstudio}
Dateien zur Erweiterung der automatischen Befehlsvervollständigung für das 
\TUDScript-Bundle bereitgestellt. Diese müssen unter Windows in
\Path{\%APPDATA\%\textbackslash texstudio} beziehungsweise unter unixoiden 
Betriebssystemen in \Path{.config/texstudio} eingefügt werden.

Außerdem findet man für die Verwendung des \TUDScript-Bundles zusammen mit 
\Application{\hologo{LyX}} unter
\hrfn{https://github.com/tud-cd/tudscr/tree/master/addon/tudscr4lyx}{tudscr/addon/tudscr4lyx}
die notwendigen Dateien und ein Minimalbeispiel. Die Layout-Dateien müssen 
dafür im \Application{\hologo{LyX}}"=Installationspfad in den passenden 
Unterordner kopiert werden. Dieser ist bei Windows
\Path{\%PROGRAMFILES(X86)\%\textbackslash{}LyX~2.0\textbackslash{}Resources\textbackslash{}layouts}
beziehungsweise bei unixoiden Betriebssystemen \Path{/usr/share/lyx/layouts}.



\section{Literaturverwaltung in \NoCaseChange{\hologo{LaTeX}}}
\ChangedAt{v2.02}%
%
Die simpelste Variante, eine Literaturdatenbank in \hologo{LaTeX} zu verwalten, 
ist dies mit dem verwendeten \hologo{LaTeX}-Editors manuell zu erledigen. Dies 
ist allerdings nicht sonderlich komfortabel. Einfacher ist es, dies mit einer 
darauf spezialisierten Anwendung zu bewerkstelligen. Für die Referenzverwaltung 
in \hologo{LaTeX} gibt es dafür zwei sehr gute Programme
%
\begin{itemize}
\item \Application{Citavi}
\item \Application{JabRef}
\end{itemize}
%
Das Programm \Application{Citavi} ermöglicht den Import von bibliographischen 
Informationen aus dem Internet. Allerdings sind diese teilweise unvollständig 
oder mangelhaft. Mit \Application{JabRef} hingegen muss die Literaturdatenbank 
manuell erstellt werden. Allerdings lassen sich einzelne Einträge aus 
.bib-Dateien sehr importieren. Beide Anwendungen unterstützen den Export 
beziehungsweise die Erstellung von Datenbanken im Stil von \Package{biblatex}. 
Für \Application{JabRef} muss diese durch den Anwender explizit aktiviert 
werden.\footnote{Optionen/Einstellungen/Erweitert/BibLaTeX-Modus}

Zur Verwendung der beiden Programme in Verbindung mit \Package{biblatex} und 
\Application{biber} gibt es ein gutes Tutorial unter diesem
\href{http://www.suedraum.de/latex/stammtisch/degenkolb_latex_biblatex_folien-final.pdf}{Link}.



\section{Finden von unbekannten \NoCaseChange{\hologo{LaTeX}}-Symbolen}
\index{Symbole}
Für \hologo{LaTeX} stehen jede Menge Symbole zur Verfügung, die allerdings 
nicht immer einfach zu finden sind. In der Zusammenfassung
\hrfn{http://mirrors.ctan.org/info/symbols/comprehensive/symbols-a4.pdf}{\File{symbols-a4.pdf}}
werden viele Symbole aus mehreren Paketen aufgeführt. Allerdings ist das 
Auffinden eines speziellen Symbols nicht sehr komfortabel. Alternativ dazu kann 
dieser \hrfn{http://detexify.kirelabs.org/classify.html}{Link} verwendet 
werden. Auf dieser Seite wird das gesuchte Symbol einfach gezeichnet, die dazu 
ähnlichsten werden zurückgegeben.



\section{Zeilenabstände in Überschriften}
\label{sec:tips:headings}
Mit dem Paket \Package{setspace} kann der Zeilenabstand beziehungsweise der 
Durchschuss innerhalb des Dokumentes geändert werden. Sollte dieser erhöht 
worden sein, können die Abstände bei mehrzeiligen Überschriften als zu groß 
erscheinen. Um dies zu korrigieren kann mit dem Befehl \Macro{addtokomafont}%
\PParameter{disposition}\PParameter{\Macro*{setstretch}\PParameter{1}} der 
Zeilenabstand aller Überschriften auf einzeilig zurückgeschaltet werden. Soll 
dies nur für eine bestimmte Gliederungsebene erfolgen, so ist 
\PParameter{disposition} durch das entsprechende Schriftelement zu ersetzen.



\section{Unterdrückung des Einzuges eines Absatzes}
\index{Absatzauszeichnung}
Verwendet man~-- wie es aus typographischer Sicht zumeist sinnvoll ist~-- 
Einzüge und keine vertikalen Abstände zur Auszeichnung von Absätzen im Dokument
(\Option{parskip}[false]), kann es vorkommen, dass ein bestimmter Absatz~-- 
beispielsweise der nach einer gewissen Umgebung folgende~-- ungewollt 
eingerückt ist. Dies kann sehr einfach behoben werden, indem direkt zu Beginn 
des Absatzes das Makro \Macro{noindent} verwendet wird. Möchte man das für 
bestimmte Umgebungen oder Befehle automatisiert gestalten, ist das Paket
\Package{noindentafter} zu empfehlen.



\section{Unterbinden des Zurücksetzens von Fußnoten}%
\label{sec:tips:counter}
\index{Fußnoten}
Oft taucht die Frage auf, wie man auch über Kapitel fortlaufende Fußnoten 
erhalten kann. Dies ist sehr einfach mit dem Paket \Package{chngcntr} möglich. 
Nach dem Laden des Paketes, kann das Rücksetzen des Zählers nach einem Kapitel 
mit \Macro*{counterwithout*}\PParameter{footnote}\PParameter{chapter} 
deaktiviert werden. Auch andere \hologo{LaTeX}-Zähler~-- wie beispielsweise der 
bereits vorgestellte \Counter{symbolheadings}~-- lassen sich mit diesem 
Paket manipulieren.



\section{URL-Umbrüche im Literaturverzeichnis mit \Package{biblatex}}
\index{Literaturverzeichnis}
%
\ChangedAt{v2.02}
Wird das Paket \Package{biblatex} verwendet, kann es unter Umständen dazu 
kommen, das eine URL nicht vernünftig umbrochen werden. Ist dies der Fall, 
können die Zählern \Counter*{biburlnumpenalty}, \Counter*{biburlucpenalty} und 
\Counter*{biburllcpenalty} erhöht werden. Die möglichen Werte liegen zwischen 0 
und 10000, wobei es bei höheren Werte der Zähler zu mehr URL-Umbrüchen an 
Ziffern (\Counter*{biburlnumpenalty}), Groß- (\Counter*{biburlucpenalty}) und 
Kleinbuchstaben (\Counter*{biburllcpenalty}) kommt. Genaueres hierzu ist der 
Dokumentation des \Package{biblatex}"=Paketes zu entnehmen.



\section{Bezeichnungen der Gliederungsebenen durch \Package{hyperref}}
\index{Querverweise}
%
\ChangedAt{v2.02}
Das Paket \Package{hyperref} stellt für Querverweise unter anderem den Befehl 
\Macro*{autoref}\Parameter{label} zur Verfügung. Mit diesem wird~-- im 
Gegensatz zur Verwendung von \Macro*{ref}~-- bei einer Referenz nicht nur die 
Nummerierung selber sondern auch das entsprechende Element wie Kapitel oder 
Abbildung vorangestellt. Bei der Benennung des referenzierten Elementes wird 
sequentiell geprüft, ob das Makro \Macro*{}\PName{Element}\PValue{autorefname}
oder \Macro*{}\PName{Element}\PValue{name} existiert. Soll die Bezeichnung 
eines Elementes geändert werden, muss man den entsprechende Bezeichner anpassen.
%
\begin{Example}
Bezeichnungen von Gliederungsebenen können folgendermaßen verändert werden.
\begin{Code}
\renewcaptionname{ngerman}{\sectionautorefname}{Unterkapitel}
\renewcaptionname{ngerman}{\subsectionautorefname}{Abschnitt}
\renewcaptionname{ngerman}{\subsubsectionautorefname}{Unterabschnitt}
\end{Code}
\end{Example}



\section{Setzen von Einheiten mit \Package{siunitx}}
\label{sec:tips:siunitx}
\index{Einheiten}
Wenn \Package*{siunitx} in einem deutschsprachigen Dokument genutzt soll
werden, muss zumindest mit \Macro*{sisetup}\PParameter{locale = DE} die 
richtige Lokalisierung angegeben werden. Sollen auch die Zahlen richtig 
formatiert sein, müssen weitere Einstellungen vorgenommen werden. Die meiner 
Meinung nach besten sind die folgenden.
%
\begin{quoting}
\begin{Code}
\sisetup{%
  locale = DE,%
  input-decimal-markers={,},%
  input-ignore={.},%
  group-separator={\,},%
  group-minimum-digits=3%
}
\end{Code}
\end{quoting}
%
Das Komma kommt als Dezimaltrennzeichen zum Einsatz. Des Weiteren werden Punkte 
innerhalb der Zahlen ignoriert und eine Gruppierung von jeweils drei Ziffern 
vorgenommen. Alternativ zu diesem Paket kann übrigens auch \Package{units} 
verwendet werden.



\section{Leer- und Satzzeichen nach \NoCaseChange{\hologo{LaTeX}}-Befehlen}%
\label{sec:tips:xspace}
\index{Typographie}
Normalerweise \enquote{schluckt} \hologo{LaTeX} die Leerzeichen nach einem 
Makro ohne Argumente. Dies ist jedoch nicht immer~-- genau genommen in den 
seltensten Fällen~-- erwünscht. Für dieses Handbuch ist beispielsweise der 
Befehl \Macro*{TUD} definiert worden, um \enquote{\TUD{}} nicht ständig 
ausschreiben zu müssen. Um sich bei der Verwendung des Befehl innerhalb eines 
Satzes sich für den Erhalt eines folgenden Leerzeichens das Setzen der 
geschweiften Klammer nach dem Befehl zu sparen (\Macro*{TUD}\PParameter{}), 
kann \Macro*{xspace} aus dem Paket \Package{xspace} genutzt werden. Damit wird 
ein folgendes Leerzeichen erhalten. Der Befehl \Macro*{TUD} ist wie folgt 
definiert:
%
\begin{quoting}
\begin{Code}
\newcommand*\TUD{Technische Universit\"at Dresden\xspace}
\end{Code}
\end{quoting}
%
Das Paket \Package{xpunctuate} erweitert die Funktionalität nochmals. Damit 
können auch Abkürzungen so definiert werden, dass ein versehentlicher Punkt 
ignoriert wird:
%
\begin{quoting}
\begin{Code}
\newcommand*\zB{z.\,B\xperiod}
\end{Code}
\end{quoting}



\section{Automatisiertes Einbinden von \Application{Inkscape}-Grafiken }
\label{sec:tips:svg}
\index{Grafiken}
In \hrfn{http://www.ctan.org/pkg/svg-inkscape}{\Package{svg-inkscape}} wird das 
automatisierte Einbinden von \Application{Inkscape}-Grafiken in ein 
\hologo{LaTeX}"=Dokument erläutert. Hier wird ein daraus abgeleiteter und 
verbesserter Ansatz vorgestellt. Nutzer von unixartigen Systemen können 
alternativ auch das Paket \Package{svg} nutzen, welches den folgend erläuterten 
Befehl \Macro{includesvg} definiert.

Die mit \Application{Inkscape} erstellte Grafik soll automatisch kompiliert und 
eingebunden werden. Dies soll allerdings nicht bei jeder Kompilierung des 
Hauptdokumentes erfolgen, sondern lediglich, wenn die originale Bilddatei 
geändert beziehungsweise aktualisiert wurde. Hierfür wird \Package{filemod} 
verwendet. Die automatisierte Übersetzung einer Grafik im SVG"~Format in eine 
PDF"~Datei und die daran anschließende Einbindung dieser in das Dokument ist 
mit der Definition von \Macro{includesvg}\OParameter{Breite}\Parameter{Datei} 
in der Präambel des Dokumentes wie folgt möglich:
%
\makeatletter
\label{macros:includesvg}%
\Hy@raisedlink{\hyperdef{\jobname}{macros:includesvg}{}}%
\makeatother
\begin{quoting}
\begin{Code}[escapechar=§]
\usepackage{filemod}
\newcommand*{\includesvg}[2][\textwidth]{%
  \def\svgwidth{#1}
  \filemodCmp{#2.pdf}{#2.svg}{}{%
    \immediate\write18{%
      inkscape -z -D --file=#2.svg --export-pdf=#2.pdf --export-latex
    }%
  }%
  \input{#2.pdf_tex}%
}
\end{Code}
\end{quoting}
%
Mit \Macro*{immediate}\Macro*{write18}\Parameter{externer Aufruf} wird das 
zwischenzeitliche Ausführen eines externen Programms beim Durchlauf von 
\hologo{pdfLaTeX}~-- in diesem Fall von \File{inkscape.exe}~-- möglich. Damit 
der externe Aufruf auch tatsächlich durchgeführt wird, muss \hologo{pdfLaTeX} 
mit der Option \Path{-{}-shell-escape} ausgeführt werden. Außerdem muss der 
Pfad zur Datei \File{inkscape.exe} dem System bekannt sein.%
\footnote{%
  Genauer gesagt, muss der Pfad zu \File{inkscape.exe} in der 
  \texttt{PATH}-Variable des Betriebssystems enthalten sein.
}
Bei der Verwendung des Befehls \Macro{includesvg} \emph{muss} der Dateiname 
ohne Endung angegeben werden. Die einzubindende SVG"~Datei sollte sich hierbei 
im gleichen Pfad wie das Hauptdokument befinden. Ist die SVG"~Datei in einem 
Unterordner relativ zum Pfad des Hauptdokumentes, kann dieser einfach mit 
\Macro{includesvg}\PParameter{\PName{Ordner}/\PName{Datei}} im Argument 
angegeben werden.



\section{Warnung wegen zu geringer Höhe der Kopf-/Fußzeile}
\label{sec:tips:headline}
Wird das Paket \Package{setspace} verwendet, kann es passieren, dass nach der 
Änderung des Zeilenabstandes \emph{innerhalb} des Dokumentes eine oder beide 
der folgenden Warnungen erscheinen:
%
\begin{quoting}
\begin{Code}
scrlayer-scrpage Warning: \headheight to low.
scrlayer-scrpage Warning: \footheight to low.
\end{Code}
\end{quoting}
%
Dies liegt an dem durch den vergrößerten Zeilenabstand erhöhten Bedarf für die
Kopf- und Fußzeile, die Höhen können in diesem Fall direkt mit der Verwendung 
von \Macro{recalctypearea} angepasst werden. Allerdings ändert das den 
Satzspiegel im Dokument, was eine andere und durchaus berechtigte Warnung von 
\Package{typearea} zur Folge hat. Falls die Änderung des Durchschusses wirklich 
nötig ist, sollte dies in der Präambel des Dokumentes einmalig passieren. Dann 
entfallen auch die Warnungen.



\section{Warnung beim Erzeugen des Inhaltsverzeichnisses}
\index{Inhaltsverzeichnis}%
\ChangedAt{v2.02}%
%
Erstellt man ein Inhaltsverzeichnis für ein Dokument mit einer dreistelligen 
Seitenanzahl, so erhält man bei der Verwendung von \Macro*{tableofcontents} 
viele Warnungen mit der Meldung \enquote{\texttt{overfull }\Macro*{hbox}}. Das 
liegt daran, dass die Seitenzahl in einer Box mit der Breite \Macro*{@pnumwidth}
gesetzt wird. Der hierfür standardmäßig verwendete Wert von \PValue{1.55em} ist 
in diesem Fall zu klein. Dieser kann folgendermaßen geändert werden:
%
\begin{quoting}
\begin{Code}
\makeatletter
\renewcommand*\@pnumwidth{2em}
\makeatother
\end{Code}
\end{quoting}
%
Dabei sollte der eingesetzte Wert nicht zu groß ausfallen.



\section{Änderung des Papierformates}
\index{Papierformat}
Es kann vorkommen, dass man innerhalb eines Dokumentes kurzzeitig das 
Papierformat ändern möchte, um beispielsweise eine Konstruktionsskizze in der 
digitalen PDF"~Datei einzubinden. Dabei ist es sowohl möglich, lediglich die 
Ausrichtung mit \Option{paper}[landscape] in ein Querformat zu ändern, als 
auch die Größe des Papierformates selber.
%
\begin{Example}
Ein Dokument im A4"~Format soll kurzzeitig auf ein A3"=Querformat geändert 
werden. Das folgende Minimalbeispiel zeigt, wie das Papierformat mit den 
Mitteln von \KOMAScript{} geändert werden kann.
\begin{Code}
\documentclass[paper=a4,pagesize]{tudscrreprt}
\usepackage{selinput}
\SelectInputMappings{adieresis={ä},germandbls={ß}}
\usepackage[T1]{fontenc}
\usepackage[ngerman]{babel}
\usepackage{blindtext}

\begin{document}
\chapter{Überschrift Eins}
\Blindtext

\cleardoublepage
\storeareas\PotraitArea% speichert den aktuellen Satzspiegel
\KOMAoptions{paper=A3,paper=landscape,DIV=current}
\chapter{Überschrift Zwei}
\Blindtext

\cleardoublepage
\PotraitArea% lädt den gespeicherten Satzspiegel
\chapter{Überschrift Drei}
\Blindtext
\end{document}
\end{Code}
\end{Example}



\section{Einrückung von Tabellenspalten verhindern}%
\label{sec:tips:table}
\index{Tabellen}
Bei Tabellen wird vor und nach Spalte durch \hologo{LaTeX} ein horizontaler 
Abstand von \Length{tabcolsep} gesetzt. Dies geschieht auch \emph{vor} der 
ersten und \emph{nach} der letzten Spalte. Diese Einrückung an den äußeren 
Rändern kann insbesondere bei Tabellen, welche die komplette Seitenbreite 
überspannen, stören. Das Paket \Package{tabularborder} versucht, dieses Problem 
automatisiert zu beheben, ist jedoch nicht zu allen \hologo{LaTeX}-Paketen für 
den Tabellensatz kompatibel.

Dieses Problem lässt sich auch manuell durch den Anwender lösen. Bei der 
Deklaration einer Tabelle kann mit \PValue{@}\PParameter{\dots} vor und 
nach dem Spaltentyp angegeben werden, was anstelle von \Length{tabcolsep} vor 
beziehungsweise nach der eigentlichen Spalte eingeführt werden soll. Dies kann 
für das Entfernen der Einrückungen genutzt werden.
%
\begin{Example}
Eine Tabelle mit zwei Spalten, wobei bei einer die Breite automatisch berechnet 
wird, soll über die komplette Textbreite gesetzt werden. Dabei soll der Rand 
vor der ersten und nach der letzten entfernt werden.
\begin{Code}[escapechar=§]
\begin{tabularx}{\textwidth}{@{}lX@{}}
§\dots§ & §\dots§ \tabularnewline
§\dots§
\end{tabularx}
\end{Code}
\end{Example}






\section{Warnungen bei der Verwendung von \Package{multicol}}
%
\ChangedAt{v2.02}
Das einzig momentan bekannte Problem der \TUDScript"=Klasse tritt in Verbindung 
mit dem Paket \Package[?]{multicol} auf. Dieses greift~-- ähnlich zu dem für 
die Seitenstile verwendeten Paket \Package{scrlayer-scrpage}~-- sehr stark in 
die Ausgaberoutine von \hologo{LaTeXe} ein. Bei Spaltenumbrüchen innerhalb der 
\Environment*{multicols}"=Umgebung kommt es sehr häufig zu folgender Warnung:
%
\begin{quoting}
\begin{Code}
Underfull \hbox (badness 10000) has occurred while \output is active
\end{Code}
\end{quoting}
%
Wird \Package{scrlayer-scrpage} nicht verwendet, unterbleibt die Warnung. Beide 
Pakete haben in der Kombination miteinander augenscheinlich ein Problem. Die 
Autoren der beiden Pakete wurden über dieses Problem informiert, signalisierten 
jedoch verständlicherweise wenig Bereitschaft, der Ursache auf den Grund zu 
gehen. Leider kann für die \TUDScript-Klassen auf \Package{scrlayer-scrpage} 
nicht verzichtet werden, es ist von essentieller Bedeutung für die Seitenstile 
im \CD der \TnUD. Am Ausgabeergebnis ändert sich nichts. Im Zweifelsfall werden 
jedoch eine Menge bedeutungsloser Warnungen generiert. 

Es gibt allerdings eine Möglichkeit, diese zu unterdrücken. Hierfür wird auf 
die Funktionalitäten von \Package{etoolbox} zurückgegriffen, was ohnehin von 
den \TUDScript-Klassen geladen wird. In der Präambel kann folgender Quelltext 
verwendet werden:
%
\begin{quoting}
\begin{Code}
\makeatletter
\apptocmd{\prepare@multicols}{\hbadness10000}{}{}
\makeatother
\end{Code}
\end{quoting}
%
Dadurch werden die durch \hologo{LaTeXe} generierten Warnungen innerhalb der 
\Environment*{multicols}"=Umgebung deaktiviert. Es muss dabei beachtet werden, 
dass dies für alle Boxen~-- und nicht nur jene beim Spaltenumbruch~-- gilt.



\section{Lokale Änderungen von Befehlen und Einstellungen}
\index{Gruppierungen}
%
\ChangedAt{v2.02}
Ein zentraler Bestandteil von \hologo{LaTeX} ist die Verwendung von Gruppen 
oder Gruppierungen. Innerhalb dieser bleiben alle vorgenommenen Änderungen an 
Befehlen, Umgebungen oder Einstellungen lokal. Dies kann sehr nützlich sein, 
wenn beispielsweise das Verhalten eines bestimmten Makros einmalig oder 
innerhalb von selbst definierten Befehlen oder Umgebungen geändert werden, im 
Normalfall jedoch die ursprüngliche Funktionalität behalten soll.
\begin{Example}
\index{Schriftauszeichnung}
Der Befehl \Macro*{emph} wird von \hologo{LaTeX} für Hervorhebungen im Text 
bereitgestellt und führt normalerweise zu einer kursiven oder~-- falls kein 
Schriftschnitt mit echten Kursiven vorhanden ist~-- kursivierten Auszeichnung. 
Soll nun in einem bestimmten Abschnitt die Auszeichnung mit fetter Schrift 
erfolgen, kann der Befehl \Macro*{emph} innerhalb einer Gruppierung geändert 
und verändert werden. Wird diese beendet, verhält sich der Befehl wie gewohnt.
\begin{Code}
In diesem Text wird ein bestimmtes \emph{Wort} hervorgehoben.

\begingroup
\renewcommand*\emph[1]{\textbf{#1}}%
In diesem Text wird ein bestimmtes \emph{Wort} hervorgehoben.
\endgroup

In diesem Text wird ein bestimmtes \emph{Wort} hervorgehoben.
\end{Code}
\end{Example}
Eine Gruppierung kann entweder mit \Macro*{begingroup} und \Macro*{endgroup} 
oder einfach mit einem geschweiften Klammerpaar \texttt{\{\dots\}} definiert 
werden.



\section{Platzierung von Gleitobjekten}
\label{sec:tips:floats}
\index{Gleitobjekte|?}
Die standardmäßige Platzierung von Gleitobjekten wird durch die im Folgenden 
aufgezählten Befehle beeinflusst. Diese können mit 
\Macro*{renewcommand*}\Parameter{Befehl}\Parameter{Wert} geändert werden.

\begin{Declaration}{\Macro{floatpagefraction}}[0\floatpagefraction]
\begin{Declaration}{\Macro{dblfloatpagefraction}}[0\dblfloatpagefraction]
\printdeclarationlist*%
%
Der Wert gibt die relative Größe eines Gleitobjektes bezogen auf die Texthöhe 
(\Macro*{textheight}) an, die mindestens erreicht sein muss, damit für dieses 
gegebenenfalls vor dem Beginn eines neuen Kapitels eine separate Seite erzeugt 
wird. Dabei wird einspaltiges (\Macro*{floatpagefraction}) und zweispaltiges 
(\Macro*{dblfloatpagefraction}) Layout unterschieden. Der Wert für beide 
Befehle sollte im Bereich von \PValue{0.5\dots 0.8} liegen.
\end{Declaration}
\end{Declaration}

\begin{Declaration}{\Macro{topfraction}}[0\topfraction]
\begin{Declaration}{\Macro{dbltopfraction}}[0\dbltopfraction]
\printdeclarationlist*%
%
Diese Werte geben den maximalen Seitenanteil für Gleitobjekte, die am oberen 
Seitenrand platziert werden, für einspaltiges und zweispaltiges Layout an. Er 
sollte im Bereich von \PValue{0.5\dots 0.8} liegen und größer als 
\Macro*{floatpagefraction} beziehungsweise \Macro*{dblfloatpagefraction} sein.
\end{Declaration}
\end{Declaration}

\begin{Declaration}{\Macro{bottomfraction}}[0\bottomfraction]
\printdeclarationlist*%
%
Dies ist der maximale Seitenanteil für Gleitobjekte, die am unteren Seitenrand 
platziert werden. Er sollte zwischen \PValue{0.2} und \PValue{0.5} betragen.
\end{Declaration}

\begin{Declaration}{\Macro{textfraction}}[0\textfraction]
\printdeclarationlist*%
%
Dies ist der Mindestanteil an Text, der auf einer Seite mit Gleitobjekten 
vorhanden sein muss, wenn diese nicht auf einer eigenen Seite ausgegeben 
werden. Er sollte im Bereich von \PValue{0.1\dots 0.3} liegen.
\end{Declaration}

\begin{Declaration}{\Counter{totalnumber}}[\arabic{totalnumber}]
\begin{Declaration}{\Counter{topnumber}}[\arabic{topnumber}]
\begin{Declaration}{\Counter{dbltopnumber}}[\arabic{dbltopnumber}]
\begin{Declaration}{\Counter{bottomnumber}}[\arabic{bottomnumber}]
\printdeclarationlist*%
%
Außerdem gibt es noch Zähler, welche die maximale Anzahl an Gleitobjekten pro 
Seite insgesamt (\Counter*{totalnumber}), am oberen (\Counter*{topnumber}) und 
am unteren Rand der Seite (\Counter*{bottomnumber}) sowie im Zweispaltensatz 
beide Spalten überspannend (\Counter*{dbltopnumber}) festlegen. Die Werte 
können mit \Macro*{setcounter}\Parameter{Zähler}\Parameter{Wert} geändert 
werden.
\end{Declaration}
\end{Declaration}
\end{Declaration}
\end{Declaration}

\begin{Declaration}{\Length{@fptop}}
\begin{Declaration}{\Length{@fpsep}}
\begin{Declaration}{\Length{@fpbot}}
\begin{Declaration}{\Length{@dblfptop}}
\begin{Declaration}{\Length{@dblfpsep}}
\begin{Declaration}{\Length{@dblfpbot}}
\printdeclarationlist*%
%
Sind vor Beginn eines Kapitels noch Gleitobjekte verblieben, so werden diese 
durch \hologo{LaTeX} normalerweise auf einer separaten vertikal zentriert Seite 
ausgegeben. Dabei bestimmen diese Längen jeweils den Abstand vor dem ersten 
Gleitobjekt zum oberen Seitenrand (\Length*{@fptop}, \Length*{@dblfptop}), 
zwischen den einzelnen Objekten (\Length*{@fpsep}, \Length*{@dblfpsep}) sowie 
zum unteren Seitenrand (\Length*{@fpbot}, \Length*{@dblfpbot}). Soll dies nicht 
geschehen, können die Längen durch den Anwender geändert werden.
\end{Declaration}
\end{Declaration}
\end{Declaration}
\end{Declaration}
\end{Declaration}
\end{Declaration}
%
\begin{Example}
Alle Gleitobjekte auf einer dafür speziell gesetzten Seite sollen direkt zu 
Beginn dieser ausgegeben werden. In der Dokumentpräambel kann man dafür 
schreiben:
\begin{Code}
\makeatletter
\setlength{\@fptop}{0pt}
\setlength{\@dblfptop}{0pt} % twocolumn
\makeatother
\end{Code}
\end{Example}
\appendix
\part{Anhang}
\chapter{Weiterführende Installationshinweise}
\label{sec:install:ext}
%
Bis zur Version~v2.01 wurde \TUDScript ausschließlich über das \Forum zur 
lokalen Nutzerinstallation angeboten. In erster Linie hat das historische 
Hintergründe und hängt mit der Entstehungsgeschichte von \TUDScript zusammen. 

Eine lokale Nutzerinstallation bietet mehr oder weniger genau einen Vorteil. 
Treten bei der Verwendung von \TUDScript Probleme auf, können diese im Forum 
gemeldet und diskutiert werden. Ist für ein solches Problem tatsächlich eine 
Fehlerkorrektur respektive eine Aktualisierung von \TUDScript nötig, kann 
diese schnell und unkompliziert über das 
\hrfn{https://github.com/tud-cd/tudscr}{GitHub-Repository \Package*{tudscr}} 
bereitgestellt und durch den Anwender sofort genutzt werden.

Dies hat allerdings für Anwender, welche das Forum relativ wenig oder gar 
nicht besuchen, den großen Nachteil, dass diese nicht von Aktualisierungen, 
Verbesserungen und Fehlerkorrekturen neuer Versionen profitieren können. Auch 
sämtliche nachfolgenden Bugfixes und Aktualisierungen des \TUDScript-Bundles 
müssen durch den Anwender manuell durchgeführt werden. Daher wird in Zukunft 
die Verbreitung via \hrfn{http://www.ctan.org/pkg/tudscr}{CTAN} präferiert, so 
dass \TUDScript stets in der aktuellen Version verfügbar ist~-- eine durch den 
Anwender aktuell gehaltene \hologo{LaTeX}"=Distribution vorausgesetzt. Der 
einzige Nachteil bei diesem Ansatz ist, dass die Verbreitung eines Bugfixes 
über das \hrfn{http://www.ctan.org/}{Comprehensive TeX Archive Network (CTAN)} 
und die anschließende Bereitstellung durch die verwendete Distribution für 
gewöhnlich mehrere Tage dauert.

Die gängigen \hologo{LaTeX}"=Distributionen durchsuchen im Regelfall zuerst das 
lokale \Path{texmf}"=Nutzerverzeichnis nach Klassen und Paketen und erst daran 
anschließend den \Path{texmf}"=Pfad der Distribution selbst. Dabei spielt es 
keine Rolle, in welchem Pfad die neuere Version einer Klasse oder eines Paketes 
liegt. Sobald im Nutzerverzeichnis die gesuchte Datei gefunden wurde, wird die 
Suche beendet.
\Attention{%
  In der Konsequenz bedeutet dies, dass sämtliche Aktualisierungen über CTAN 
  nicht zum Tragen kommen, falls \TUDScript als lokale Nutzerversion  
  installiert wurde.
}

Deshalb wird Anwendern, die \TUDScript in der Version~v2.01 oder älter nutzen 
und sich nicht \emph{bewusst} für eine lokale Nutzerinstallation entschieden 
haben, empfohlen, diese zu deinstallieren. Der Prozess der Deinstallation wird 
in \autoref{sec:local:uninstall} erläutert. Wird diese einmalig durchgeführt, 
können Updates des \TUDScript-Bundles durch die Aktualisierungsfunktion der 
Distribution erfolgen. Wie das \TUDScript-Bundle trotzdem als lokale 
Nutzerversion installiert oder aktualisiert werden kann, ist in 
\autoref{sec:local:install} beziehungsweise \autoref{sec:local:update} zu 
finden. Der Anwender sollte in diesem Fall allerdings genau wissen, was er 
damit bezweckt, da er in diesem Fall für die Aktualisierung von \TUDScript 
selbst verantwortlich ist.



\section{Lokale Deinstallation von \TUDScript}
\label{sec:local:uninstall}
\index{Deinstallation}
%
Um die lokale Nutzerinstallation zu entfernen, kann für Windows
\hrfn{https://github.com/tud-cd/tudscr/releases/download/uninstall/tudscr\_uninstall.bat}%
{\File{tudscr\_uninstall.bat}} sowie für unixartige Betriebssysteme
\hrfn{https://github.com/tud-cd/tudscr/releases/download/uninstall/tudscr\_uninstall.sh}%
{\File{tudscr\_uninstall.sh}} verwendet werden. Nach der Ausführung des 
jeweiligen Skriptes kann in der Konsole beziehungsweise im Terminal mit
%
\begin{quoting}
\Path{kpsewhich --all tudscrbase.sty}
\end{quoting}
%
überprüft werden, ob die Deinstallation erfolgreich war oder immer noch eine 
lokale Nutzerinstallation vorhanden ist. Es werden alle Pfade ausgegeben, in 
welchen die Datei \File*{tudscrbase.sty} gefunden wird. Sollte nur noch der 
Pfad der Distribution erscheinen, ist ab sofort die \TUDScript-Version von CTAN 
aktiv und der Anwender kann mit dem \TUDScript-Bundle arbeiten. Falls es nicht 
schon passiert ist, müssen dafür lediglich die Schriften des \CDs installiert 
werden (\autoref{sec:install}).

Wird \emph{nur} das lokale Nutzerverzeichnis oder gar kein Verzeichnis 
gefunden, so wird höchstwahrscheinlich eine veraltete Distribution 
verwendet. In diesem Fall \emph{muss} das \TUDScript-Bundle lokal aktualisiert 
(\autoref{sec:local:update}) beziehungsweise bei der erstmaligen Verwendung 
lokal installiert (\autoref{sec:local:install}) werden. Sollte neben dem 
Pfad der Distribution immer noch mindestens ein weiterer Pfad angezeigt werden,
so ist weiterhin eine lokale Nutzerversion installiert. In diesem Fall hat der 
Anwender zwei Möglichkeiten:
%
\begin{enumerate}
\item Entfernen der lokalen Nutzerinstallation (manuell)
\item Update der lokalen Nutzerversion
\end{enumerate}
%
Die erste Variante wird nachfolgend erläutert, die zweite Möglichkeit wird in 
\autoref{sec:local:update} beschrieben. Nur die manuelle Deinstallation der 
lokalen Nutzerversion \TUDScript ermöglicht dabei die Verwendung der jeweils 
aktuellen CTAN"=Version. Hierfür ist etwas Handarbeit vonnöten. Der in der 
Konsole beziehungsweise im Terminal mit \Path{kpsewhich --all tudscrbase.sty} 
gefundene~-- zum Ordner der Distribution \emph{zusätzliche}~-- Pfad hat die 
folgende Struktur:
%
\begin{quoting}
\Path{\emph{<Installationspfad>}/tex/latex/tudscr/tudscrbase.sty}
\end{quoting}
%
Um die Nutzerinstallation vollständig zu entfernen, muss als erstes zu 
\Path{\emph{<Installationspfad>}} navigiert werden. Anschließend ist in diesem 
Pfad Folgendes durchzuführen:
%
\settowidth{\tempdim}{\Path{tex/latex/tudscr/}}%
\begin{description}[labelwidth=\tempdim,labelsep=1em]
\item[\Path{tex/latex/tudscr/}]alle .cls- und .sty-Dateien löschen
\item[\Path{tex/latex/tudscr/}]Ordner \Path{logo} vollständig löschen
\item[\Path{doc/latex/}] Ordner \Path{tudscr} vollständig löschen
\item[\Path{source/latex/}] Ordner \Path{tudscr} vollständig löschen
\end{description}
%
Das Verzeichnis \Path{\emph{<Installationspfad>}/tex/latex/tudscr/fonts} 
\textbf{sollte erhalten bleiben}. Andernfalls müssen die Schriften des \CDs 
abermals wie unter \autoref{sec:install} beschrieben installiert werden.
Zum Abschluss ist in der Kommandozeile beziehungsweise im Terminal der Befehl 
\Path{texhash} aufzurufen. Damit wurde die lokale Version entfernt und es wird 
von nun an die Version von \TUDScript genutzt, welche durch die verwendete 
Distribution bereitgestellt wird.

\section{Lokale Installation des \TUDScript-Bundles}
\label{sec:local:install}
\index{Installation!Nutzerinstallation}\index{Nutzerinstallation (lokal)}
\subsection{Lokale Installation von \TUDScript unter Windows}
\index{Installation!Nutzerinstallation}\index{Nutzerinstallation (lokal)}
Für eine lokale Installation sowohl des \TUDScript-Bundles als auch der 
dazugehörigen Schriften für die Distributionen \Distribution{\hologo{TeX}~Live} 
oder \Distribution{\hologo{MiKTeX}} werden neben Schriftarchiven die Dateien aus
\hrfn{https://github.com/tud-cd/tudscr/releases/download/\vTUDScript/TUD-KOMA-Script\_\vTUDScript\_Windows\_full.zip}%
{\File*{TUD-KOMA-Script\_\vTUDScript\_Windows\_full.zip}} benötigt. Vor der 
Verwendung des Skripts \File{tudscr\_\vTUDScript\_install.bat} sollte 
sichergestellt werden, dass sich \emph{alle} der folgenden Dateien im selben 
Verzeichnis befinden:
%
\settowidth{\tempdim}{\File{tudscr\_\vTUDScript\_install.bat}}%
\begin{description}[labelwidth=\tempdim,labelsep=1em]
  \item[\File{tudscr\_\vTUDScript.zip}]Archiv mit Klassen- und Paketdateien
  \item[\File{tudscr\_\vTUDScript\_install.bat}]Installationsskript
  \item[\File{Univers\_PS.zip}]Archiv mit Schriftdateien für \Univers
  \item[\File{DIN\_Bd\_PS.zip}]Archiv mit Schriftdateien für \DIN
  \item[\File{tudscrfonts.zip}]Archiv mit Metriken für die
    Schriftinstallation via \Package{fontinst}
  \item[\File{7za.exe}]Stand-Alone-Version von 7-zip zum Entpacken der Archive%
    \footnote{%
      Windows stellt keine Bordmittel zum Extrahieren von Archiven auf 
      Kommandozeilen-/Skript-Ebene zur Verfügung.%
    }%
\end{description}
%
Beim Ausführen des Installationsskripts werden alle Schriften in das lokale 
Nutzerverzeichnis der jeweiligen Distribution installiert, falls kein anderes 
Verzeichnis explizit angegeben wird. Für Hinweise bei Problemen mit der 
Schriftinstallation sei auf \autoref{sec:install:fonts:win} verwiesen.



\subsection{Lokale Installation von \TUDScript unter Linux und OS~X}
\index{Installation!Nutzerinstallation}\index{Nutzerinstallation (lokal)}
Für eine lokale Installation sowohl des \TUDScript-Bundles als auch der 
dazugehörigen Schriften für die Distributionen \Distribution{\hologo{TeX}~Live} 
oder \Distribution{Mac\hologo{TeX}} werden neben Schriftarchiven die Dateien aus
\hrfn{https://github.com/tud-cd/tudscr/releases/download/\vTUDScript/TUD-KOMA-Script\_\vTUDScript\_Unix\_full.zip}%
{\File*{TUD-KOMA-Script\_\vTUDScript\_Unix\_full.zip}} benötigt. Vor der 
Verwendung des Skripts \File{tudscr\_\vTUDScript\_install.sh} sollte 
sichergestellt werden, dass sich \emph{alle} der folgenden Dateien im selben 
Verzeichnis befinden:
%
\begin{description}[labelwidth=\tempdim,labelsep=1em]
\settowidth{\tempdim}{\File{tudscr\_\vTUDScript\_install.sh}}%
  \item[\File{tudscr\_\vTUDScript.zip}]Archiv mit Klassen- und Paketdateien
  \item[\File{tudscr\_\vTUDScript\_install.sh}]Installationsskript
  \item[\File{Univers\_PS.zip}]Archiv mit Schriftdateien für \Univers
  \item[\File{DIN\_Bd\_PS.zip}]Archiv mit Schriftdateien für \DIN
  \item[\File{tudscrfonts.zip}]Archiv mit Metriken für die
    Schriftinstallation via \Package{fontinst}
\end{description}
%
Beim Ausführen des Installationsskripts werden alle Schriften in das lokale 
Nutzerverzeichnis der jeweiligen Distribution installiert. Für Hinweise bei 
Problemen mit der Schriftinstallation sei auf \autoref{sec:install:fonts:unix} 
verwiesen.



\section{Lokales Update des \TUDScript-Bundles}
\label{sec:local:update}
\index{Update!Nutzerinstallation}\index{Nutzerinstallation (lokal)}
\subsection{Update des \TUDScript-Bundles ab Version~\NoCaseChange{v}2.00}
Mit der Version~v2.02 gab es unter anderem auch einige Änderungen an den 
verwendeten Logos. Deshalb wird für diese Version kein dediziertes Update 
angeboten. Es können entweder, wie in \autoref{sec:local:install} erläutert, 
lokal via Skript neu installiert werden oder das Update erfolgt manuell. 
Hierfür muss der Inhalt des Archivs
\hrfn{https://github.com/tud-cd/tudscr/releases/download/v2.02/tudscr\_v2.02.zip}%
{\File{tudscr\_v2.02.zip}} in das lokale \Path{texmf}"=Nutzerverzeichnis 
kopiert werden.
%Für eine lokale Aktualisierung von \TUDScript auf \vTUDScript{} muss das Archiv
%\hrfn{https://github.com/tud-cd/tudscr/releases/download/\vTUDScript/TUD-KOMA-Script\_\vTUDScript\_Windows\_update.zip}%
%{TUD-KOMA-Script\_\vTUDScript\_Windows\_update.zip} respektive 
%\hrfn{https://github.com/tud-cd/tudscr/releases/download/\vTUDScript/TUD-KOMA-Script\_\vTUDScript\_Unix\_update.zip}%
%{TUD-KOMA-Script\_\vTUDScript\_Unix\_update.zip} entpackt und anschließend
%\File{tudscr\_\vTUDScript\_update.bat} oder 
%\File{tudscr\_\vTUDScript\_update.sh} ausgeführt werden.
%\Attention{%
%  Das lokale Update funktioniert nur, wenn bereits mindestens Version~v2.02 
%  entweder lokal oder über die Distribution installiert ist.%
%}



\DeclareClass{tudscrbookold}%
\DeclareClass{tudscrreprtold}%
\DeclareClass{tudscrartclold}%
\subsection{Update des \TUDScript-Bundles von Version \NoCaseChange{v}1.0}
\index{Update!Version v1.0}%
\index{Hauptklassen}
\index{Version!v1.0}%
Ist \TUDScript in der Version~v1.0 installiert, so wird dringend zu einer 
Deinstallation dieser geraten. Geschieht dies nicht, wird es zu Problemen 
kommen. Dafür werden die Skripte 
\hrfn{https://github.com/tud-cd/tudscr/releases/download/uninstall/tudscr\_uninstall.bat}{\File{tudscr\_uninstall.bat}}
beziehungsweise
\hrfn{https://github.com/tud-cd/tudscr/releases/download/uninstall/tudscr_uninstall.sh}{\File{tudscr\_uninstall.sh}}
bereitgestellt. Die aktuelle Version~\vTUDScript{} kann nach der Deinstallation 
der Version~v1.0 wie unter \autoref{sec:install} beschrieben installiert werden.

Sollen die obsoleten \TUDScript-Klassen in der Version~v1.0 nach einer 
Aktualisierung weiterhin genutzt werden, so müssen diese erst wie zuvor 
beschrieben de"~~und anschließend neu installiert werden. Dafür kann das Archiv 
\hrfn{https://github.com/tud-cd/tudscrold/releases/download/v1.0/TUD-KOMA-Script_v1.0old.zip}%
{\File*{TUD-KOMA-Script\_v1.0old.zip}} verwendet werden, welches sowohl die 
genannten Skripte zur Deinstallation als auch die zur neuerlichen Installation 
der veralteten Klassen benötigten \File{tudscr\_v1.0old\_install.bat} oder 
\File{tudscr\_v1.0old\_install.sh} enthält. Nach Abschluss des Vorgangs sind 
die alten Klassen der Version~v1.0 mit \Class*{tudscrbookold}, 
\Class*{tudscrreprtold} und \Class*{tudscrartclold} verwendbar und können 
parallel zur aktuellen Version~\vTUDScript{} genutzt werden.

Im Vergleich zur Version~v1.0 hat sich an der Benutzerschnittstelle nicht sehr 
viel verändert. Treten nach dem Umstieg von der Version~v1.0 auf die 
Version~\vTUDScript{} dennoch Probleme auf, sollte der Anwender als erstes in 
\autoref{sec:comp} sehen. Hier werden die gemachten Änderungen erläutert und im 
alten Dokument gegebenenfalls notwendige Anpassungen beschrieben. Sollten 
dennoch Fehler oder Probleme beim Umstieg auf die neue \TUDScript-Version 
auftreten, ist eine Meldung im Forum die beste Möglichkeit, um Hilfe zu 
erhalten.
\chapter{Obsolete sowie vollständig entfernte Optionen und Befehle}
\label{sec:obsolete}%
%
Einige Optionen und Befehle waren während der Weiterentwicklung von \TUDScript
in ihrer ursprünglichen Form nicht mehr umsetzbar oder wurden schlichtweg 
verworfen. Dennoch wird hier \emph{teilweise} gezeigt, wie die Funktionalität 
mit \TUDScript in der Version \vTUDScript{} darstellbar ist.

\begin{Declaration}{\Option{cd}[alternative]}{entfällt}
\begin{Declaration}{\Option{cdtitle}[alternative]}{entfällt}
\begin{Declaration}{\Length{titlecolwidth}}{entfällt}
\begin{Declaration}{\Term{authortext}}{entfällt}
\printdeclarationlist*%
\index{Titel!alternativer}%
%
Die alternative Titelseite ist komplett aus dem \TUDScript-Bundle entfernt 
worden. Dementsprechend entfallen auch die dazugehörigen Optionen sowie Länge 
und Bezeichner.
\end{Declaration}
\end{Declaration}
\end{Declaration}
\end{Declaration}

\begin{Declaration}{\Option{color}[\PBoolean]}{siehe \Option*{cd}[color]}
\printdeclarationlist*%
%
Die Einstellungen der farbigen Ausprägung des Dokumentes erfolgt über die 
Option \Option*{cd}.
\end{Declaration}

\begin{Declaration}{\Option{tudfonts}[\PBoolean]}{siehe \Option*{cdfont}[\PSet]}
\printdeclarationlist*%
%
Die Option zur Schrifteinstellung ist wesentlich erweitert worden. Aus Gründen 
der Konsistenz wurde diese umbenannt.
\end{Declaration}

\begin{Declaration}{\Option{tudfoot}[\PBoolean]}{
  siehe \Option*{cdfoot}[\PBoolean]%
}
\printdeclarationlist*%
%
Ebenso wurde diese Option umbenannt, um dem Namensschema der restlichen 
Optionen zu entsprechen.
\end{Declaration}

\begin{Declaration}{\Option{headfoot}[\PSet]}{entfällt}
\printdeclarationlist*%
%
Diese Option war in der Version~v1.0 notwendig, um die parallele Verwendung von 
\Package*{typearea} und \Package*{geometry} zu ermöglichen. Dies wurde komplett 
überarbeitet, an das Paket \Package*{geometry} werden die Einstellungen für die 
\KOMAScript"=Optionen \Option*{headinclude} und \Option*{footinclude} jetzt 
direkt weitergereicht. Damit ist die Option \Option*{headfoot} nicht mehr 
notwendig und wurde entfernt.
\end{Declaration}

\begin{Declaration}{\Option{partclear}[\PBoolean]}{%
  entfällt, siehe \Option*{cleardoublespecialpage}%
}
\begin{Declaration}{\Option{chapterclear}[\PBoolean]}{%
  entfällt, siehe \Option*{cleardoublespecialpage}%
}
\printdeclarationlist*%
%
Beide Optionen sind in der neuen Option \Option*{cleardoublespecialpage} 
aufgegangen, womit ein konsistentes Layout erreicht wird. Die ursprünglichen 
Optionen entfallen. 
\end{Declaration}
\end{Declaration}

\begin{Declaration}{\Option{abstracttotoc}[\PBoolean]}{%
  entfällt, siehe \Option*{abstract}[\PSet]%
}
\begin{Declaration}{\Option{abstractdouble}[\PBoolean]}{%
  entfällt, siehe \Option*{abstract}[\PSet]%
}
\printdeclarationlist*%
%
Beide Optionen wurden in die Option \Option*{abstract} integriert und sind 
deshalb überflüssig.
\end{Declaration}
\end{Declaration}

\begin{Declaration}{\Macro{confirmationandrestriction}}{%
  entfällt, siehe \Macro*{declaration}%
}
\begin{Declaration}{\Macro{restrictionandconfirmation}}{%
  entfällt, siehe \Macro*{declaration}%
}
\begin{Declaration}{\Macro{location}\Parameter{Ort}}{%
  siehe \Macro*{place} sowie auch Parameter \Key*{\Macro{declaration}}{place}%
}
\printdeclarationlist*%
%
Die ersten beiden Befehle entfallen, \Macro*{declaration} kann alternativ dazu 
verwendet werden. In Anlehnung an andere \hologo{LaTeX}-Pakete und "~Klassen 
wurde \Macro*{location} in \Macro*{place} umbenannt.
\end{Declaration}
\end{Declaration}
\end{Declaration}

\begin{Declaration}{\Macro{logofile}\Parameter{Dateiname}}%
  {siehe \Macro*{headlogo}\Parameter{Dateiname}%
}
\begin{Declaration}{\Macro{logofilename}\Parameter{Dateiname}}%
  {siehe \Macro*{headlogo}\Parameter{Dateiname}%
}
\printdeclarationlist*%
%
Der Befehl \Macro*{logofile} wurde in \Macro*{headlogo} umbenannt.
\end{Declaration}
\end{Declaration}

\begin{Declaration}{\Length{chapterheadingvskip}}{%
  siehe \Length*{pageheadingsvskip} sowie \Length*{headingsvskip}
}
\begin{Declaration}{\Length{signatureheight}}{entfällt}
\printdeclarationlist*%
%
Die vertikale Positionierung von Überschriften wurde zweigeteilt. Die Höhe für 
die Zeile der Unterschriften wurde dehnbar gestaltet. Eine Anpassung durch den 
Anwender ist nicht vonnöten.
\end{Declaration}
\end{Declaration}

\begin{Declaration}{\Term{titlecoldelim}}{%
  entfällt, siehe \Macro*{titledelimiter}%
}
\printdeclarationlist*%
%
Das Trennzeichen für Bezeichnungen beziehungsweise beschreibende Texte und dem 
eigentlichen Feld auf der Titelseite ist nicht mehr sprachabhängig und wurde 
umbenannt.
\end{Declaration}

\begin{Declaration}{\Macro{submissiondate}\Parameter{Datum}}{%
  Alias für \Macro*{date}%
}
\begin{Declaration}{\Macro{birthday}\Parameter{Geburtsdatum}}{%
  Alias für \Macro*{dateofbirth}%
}
\begin{Declaration}{\Macro{birthplace}\Parameter{Geburtsort}}{%
  Alias für \Macro*{placeofbirth}
}
\begin{Declaration}{\Macro{studentid}\Parameter{Matrikelnummer}}{%
  Alias für \Macro*{matriculationnumber}
}
\begin{Declaration}{\Macro{enrolmentyear}\Parameter{Immatrikulationsjahr}}{%
  Alias für \Macro*{matriculationyear}%
}
\printdeclarationlist*%
%
Alle Befehle wurden umbenannt und sind jetzt für die Titelseite im \CD nutzbar.
\end{Declaration}
\end{Declaration}
\end{Declaration}
\end{Declaration}
\end{Declaration}

\begin{Declaration}{\Term{submissiontext}}{umbenannt, siehe \Term*{datetext}}
\begin{Declaration}{\Term{birthdaytext}}{%
  umbenannt, siehe \Term*{dateofbirthtext}%
}
\begin{Declaration}{\Term{birthplacetext}}{%
  umbenannt, siehe \Term*{placeofbirthtext}%
}
\begin{Declaration}{\Term{studentidname}}{%
  umbenannt, siehe \Term*{matriculationnumbername}%
}
\begin{Declaration}{\Term{enrolmentname}}{%
  umbenannt, siehe \Term*{matriculationyearname}%
}
\begin{Declaration}{\Term{supervisorIIname}}{%
  umbenannt, siehe \Term*{supervisorothername}%
}
\begin{Declaration}{\Term{defensetext}}{%
  umbenannt, siehe \Term*{defensedatetext}%
}
\printdeclarationlist*%
%
Die Bezeichner wurden in Anlehnung an die dazugehörigen Befehlsnamen umbenannt.
\end{Declaration}
\end{Declaration}
\end{Declaration}
\end{Declaration}
\end{Declaration}
\end{Declaration}
\end{Declaration}


\minisec{Aufgabenstellung}
Die Umgebung für die Erstellung einer Aufgabenstellung für eine 
wissenschaftliche Arbeit wurde in das Paket \Package*{tudscrsupervisor} 
ausgelagert. Dieses muss für die Verwendung der Umgebung \Environment*{task} 
und der daraus abgeleiteten standardisierten Form zwingend geladen werden.

\begin{Declaration}{\Option{cdtask}[\PSet]}{entfällt, siehe \Environment*{task}}
\begin{Declaration}{\Option{taskcompact}[\PBoolean]}{entfällt}
\begin{Declaration}{\Macro{tasks}\Parameter{Ziele}\Parameter{Schwerpunkte}}{%
  umbenannt, siehe \Macro*{taskform}%
}
\begin{Declaration}{\Length{taskcolwidth}}{entfällt}
\printdeclarationlist*%
%
Die Klassenoption \Option*{cdtask} ist komplett entfernt worden, alle 
Einstellungen, welche \Environment*{task} betreffen erfolgen direkt über das 
optionale Argument der Umgebung. Die Variante eines kompakten Kopfes mit der 
Option \Option*{taskcompact} wird nicht mehr bereitgestellt. Der Befehl 
\Macro*{tasks} wurde in \Macro*{taskform} umbenannt und in der Funktionalität 
erweitert. Die manuelle Einstellung der Spaltenbreite für den Kopf der 
Aufgabenstellung mit \Length*{taskcolwidth} wurde aufgrund der verbesserten 
automatischen Berechnung entfernt.
\end{Declaration}
\end{Declaration}
\end{Declaration}
\end{Declaration}

\begin{Declaration}{\Macro{startdate}\Parameter{Ausgabedatum}}{%
  Alias für \Macro*{issuedate}%
}
\begin{Declaration}{\Macro{enddate}\Parameter{Abgabetermin}}{%
  Alias für \Macro*{duedate}%
}
\begin{Declaration}{\Term{starttext}}{umbenannt, siehe \Term*{issuedatetext}}
\begin{Declaration}{\Term{duetext}}{umbenannt, siehe \Term*{duedatetext}}
\begin{Declaration}{\Term{focustext}}{umbenannt, siehe \Term*{focusname}}
\begin{Declaration}{\Term{objectivestext}}{%
  umbenannt, siehe \Term*{objectivesname}%
}
\printdeclarationlist*%
%
Alle genannten Befehle und Bezeichner wurden umbenannt.
\end{Declaration}
\end{Declaration}

\end{Declaration}
\end{Declaration}
\end{Declaration}
\end{Declaration}
\Index{Optionen}{options}%
\Index{Befehle}[Befehle etc.]{macros}%
\Index{Befehle!Parameter}[Parameter]{keys}%
\Index{Makro}[Befehle etc.]{macros}%
\Index{Umgebungen}[Befehle etc.]{macros}%
\Index{Umgebungen!Parameter}[Parameter]{keys}%
\Index{Parameter}{keys}%
\Index{Bezeichner}{terms}%
\Index{Schriftelemente}{fonts}%
\Index{Farben}{colors}%
\Index{Klassen}[Dateien etc.]{files}%
\Index{Pakete}[Dateien etc.]{files}%
\Index{Dateien}[Dateien etc.]{files}%
\Changelog{Änderungen}
\Changelog{Changelog}
\Changelog{Version}
\index{Abbildungen|see{Grafiken}}%
\index{Aktualisierung|see{Update}}%
\index{Aufzählungen|see{Listen}}%
\index{Cover|see{Umschlagseite}}%
\index{Dezimaltrennzeichen|see{Trennzeichen}}%
\index{Distribution!\hologo{TeX}~Live|see{\hologo{TeX}~Live}}
\index{Distribution!\hologo{MiKTeX}|see{\hologo{MiKTeX}}}
\index{Fachreferent|see{Referent}}%
\index{Farbraum|see{Farben!Farbmodell}}%
\index{Grafiken!Beschriftung|see{Gleitobjekte}}
\index{Großbuchstaben|see{Schriftauszeichnung}}%
\index{Klassenoptionen|see{Optionen}}%
\index{Kleinbuchstaben|see{Schriftauszeichnung}}%
\index{Kurzfassung|see{Zusammenfassung}}%
\index{Lokalisierung|see{Bezeichner}}%
\index{Majuskeln|see{Schriftauszeichnung}}%
\index{Mathematiksatz!Einheiten|see{Einheiten}}
\index{Mathematiksatz!Trennzeichen|see{Trennzeichen}}
\index{Minuskeln|see{Schriftauszeichnung}}%
\index{Outline-Eintrag|see{Lesezeichen}}%
\index{Professor|see{Hochschullehrer}}%
\index{Seitenränder|see{Satzspiegel}}
\index{Silbentrennung|see{Worttrennung}}%
\index{Sprachunterstützung!Lokalisierung|see{Bezeichner}}%
\index{Sprachunterstützung!Worttrennung|see{Worttrennung}}%
\index{Sprungmarken|see{Lesezeichen}}
\index{Tabellen!Beschriftung|see{Gleitobjekte}}
\index{Tausendertrennzeichen|see{Trennzeichen}}%
\index{Trennmuster|see{Worttrennung}}%
\index{Vakatseiten|see{Leerseiten}}%
\index{Vektorgrafiken|see{Grafiken}}%
\setchapterpreamble{%
  \begin{abstract}
    \noindent Die Formatierung der Einträge in allen aufgeführten Indizes ist 
    folgendermaßen aufzufassen: \textbf{Zahlen in fetter Schrift} verweisen auf 
    die \textbf{Erklärung} zu einem Stichwort, wobei in der digitalen Fassung 
    dieses Handbuchs dieser Eintrag selbst ein Hyperlink zu seiner Erläuterung 
    ist. Seitenzahlen in normaler Schriftstärke hingegen deuten auf zusätzliche 
    Informationen, wobei diese für \textit{kursiv hervorgehobene Zahlen} als 
    besonders \textit{wichtig} erachtet werden.
    
    Bei Einträgen für \hyperref[idx:options]{Klassen- und Paketoptionen} 
    beziehungsweise für \hyperref[idx:macros]{Umgebungen und Befehle}, zu denen 
    keine direkte \textbf{Erklärung} gegeben ist sondern lediglich zusätzliche 
    Hinweise vorhanden sind, handelt es sich um \KOMAScript"=Optionen. Diese
    sind gegebenenfalls im dazugehörigen Handbuch nachzulesen 
    (\File{scrguide.pdf}).
  \end{abstract}
}
\addchap*{\indexname}
\addxcontentsline{toc}{part}{\indexname}
\PrintIndexPrologue{%
  Die aufgelisteten Schlagworte sollen sowohl Antworten bei generellen Fragen 
  als auch Lösungen für typische Probleme beim Umgang mit \hologo{LaTeX} sowie 
  dem \TUDScript-Bundle liefern. 
%  Falls ein gesuchter Begriff hier nicht auftaucht, ist das Forum erster 
%  Anlaufpunkt.
}
\PrintIndex
\PrintChangelog
\ListOfToDo
\ToDo[man]{Layout und Umbrüche kontrollieren}
\ToDo[man]{Datum in tudscr-version.dtx, Handbuch und README aktualisieren}
\ToDo{Hinweis auf falschen Link auf CTAN}
\ToDo{Homepage überarbeiten}
\end{document}