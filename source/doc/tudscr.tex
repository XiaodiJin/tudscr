\RequirePackage[ngerman=ngerman-x-latest]{hyphsubst}
\documentclass[english,ngerman]{tudscrmanual}
\usepackage{selinput}\SelectInputMappings{adieresis={ä},germandbls={ß}}
\usepackage[T1]{fontenc}
\lstset{%
  inputencoding=utf8,extendedchars=true,
  literate=%
    {ä}{{\"a}}1 {ö}{{\"o}}1 {ü}{{\"u}}1
    {Ä}{{\"A}}1 {Ö}{{\"O}}1 {Ü}{{\"U}}1
    {~}{{\textasciitilde}}1 {ß}{{\ss}}1
}
\usepackage{bookmark}

\TUDoption{ToDo}{notnxt}

%\tracinglabels[all]
%\tracingmarkup
%\tracingbundle

\begin{document}
\newcommand*\cdurl{%
  \begingroup%
    \hypersetup{hidelinks}%
    \href{http://tu-dresden.de/cd}{http://tu-dresden.de/cd}%
  \endgroup%
}
\faculty{\protect\cdurl}
\date{06.10.2015}
\author{Falk Hanisch\thanks{\noexpand\Email{\tudscrmail}}}
\subject{\TUDScript{} \vTUDScript{} basierend auf \KOMAScript{}}
\title{%
  Ein \NoCaseChange{\hologo{LaTeXe}}-Bundle für Dokumente im~neuen \CD der \TnUD
}
\addtokomafont{subtitle}{\univbn}
\ifdef{\tudprintflag}{%
  \subtitle{Benutzerhandbuch\thanks{\href{tudscr}{Online-Version}}}%
}{%
  \subtitle{Benutzerhandbuch\thanks{\href{tudscr_print}{Druckversion}}}%
}
\makeatletter
\begingroup%
  \def\and{, }%
  \let\thanks\@gobble%
  \let\footnote\@gobble%
  \hypersetup{%
    pdfauthor = {\@author},%
    pdftitle = {\@title},%
    pdfsubject = {Benutzerhandbuch für \TUDScript},%
    pdfkeywords = {LaTeX, \TUDScript, Benutzerhandbuch},%
  }%
\endgroup%
\renewcommand*\@pnumwidth{1.7em}
\renewcommand*\@tocrmarg{2.7em}
\makeatother

\ChangedAt*{%
  v2.00:Generalüberholung und komplette Neuimplementierung von \TUDScript%
}%
\maketitle
\addchap{\prefacename}
Die im Folgenden beschriebenen Klassen und Pakete wurden für das Erstellen von 
\hologo{LaTeX}"=Dokumenten im \CD der \TnUD entwickelt.%
\footnote{%
  \url{http://tu-dresden.de/cd}\hfill
  \url{http://tu-dresden.de/service/publizieren/cd/6_handbuch/index.html}%
}
Diese basieren auf den gerade im deutschsprachigen Raum häufig verwendeten 
\KOMAScript"=Klassen, welche eine Vielzahl von Einstellmöglichkeiten bieten, 
die weit über die Möglichkeiten der \hologo{LaTeX}"=Standardklassen 
hinausgehen. Zusätzlich bietet das hier dokumentierten \TUDScript-Bundle 
weitere, insbesondere das Dokumentlayout betreffende Auswahlmöglichkeiten.

Es sei angemerkt, dass die hier beschriebenen Klassen eine Abweichung vom \CD 
der \TnUD zulassen, da dieses gerade unter typografischen Gesichtspunkten 
durchaus als diskussionswürdig zu erachten ist. Prinzipiell ist es mit den 
entsprechenden Einstellungen möglich, auf das standardmäßige Layout der 
\KOMAScript"=Klassen zurückzuschalten. Ohne die gezielte Verwendung dieser 
Optionen durch den Anwender werden per Voreinstellung alle Vorgaben des \CDs 
umgesetzt.

In diesem Handbuch werden dem Anwender Hinweise für eine einfache Installation 
der benötigten Schriften des \CDs sowie eine Beschreibung der zusätzlich zu den 
\KOMAScript"=Klassen nutzbaren Optionen und Befehle gegeben. Dabei werden 
Grundkenntnisse in der Verwendung von \hologo{LaTeXe} vorausgesetzt. Sollten 
diese nicht vorhanden sein, wird das Lesen der \hologo{LaTeXe}-Kurzbeschreibung
\hrfn{http://mirrors.ctan.org/info/lshort/german/l2kurz.pdf}{\File{l2kurz.pdf}}
dringend empfohlen. Für den stärker vertieften Einstieg in \hologo{LaTeX} gibt 
es eine \hrfn{http://www.fadi-semmo.de/latex/workshop/}{Workshop-Reihe} von 
Fadi~Semmo. Außerdem stellt Nicola~L.~C.~Talbot sehr ausführliche Tutorials für 
\hrfn{http://www.dickimaw-books.com/latex/novices/}{\hologo{LaTeX}-Novizen} und 
\hrfn{http://www.dickimaw-books.com/latex/thesis/}{Dissertationen} zur freien 
Verfügung. Außerdem werden in \autoref{part:additional} dieses Handbuchs 
Minimalbeispiele sowie etwas ausführlichere Tutorials angeboten.

Des Weiteren sollte \emph{jeder} Anwender das \hologo{LaTeXe}"=Sündenregister 
\hrfn{http://mirrors.ctan.org/info/l2tabu/german/l2tabu.pdf}{\File{l2tabu.pdf}}
kennen, um typische Fehler zu vermeiden. Antworten auf häufig gestellte Fragen 
liefert \hrfn{http://projekte.dante.de/DanteFAQ/WebHome}{DANTE"~FAQ}. Falls der 
Nutzer unerfahren bei der Verwendung von \KOMAScript{} sein sollte, so ist ein 
Blick in das \scrguide[dazugehörige Handbuch] sehr zu empfehlen, wenn nicht 
sogar unumgänglich.

Der aktuelle Stand der Klassen und Pakete aus dem \TUDScript-Bundle in der 
Version~\vTUDScript{} wurde nach bestem Wissen und Gewissen auf Herz und Nieren 
getestet. Dennoch kann nicht für das Ausbleiben von Fehlern garantiert werden. 
Beim Auftreten eines Problems sollte dieses genauso wie Inkompatibilitäten mit 
anderen Paketen im Forum unter
\begin{quoting}
\Forum*%
\end{quoting}
gemeldet werden. Für eine schnelle und erfolgreiche Fehlersuche sollte bitte 
ein \hrfn{http://www.komascript.de/minimalbeispiel}{\textbf{Minimalbeispiel}} 
bereitgestellt werden. Auf Anfragen ohne dieses werde ich gegebenenfalls 
verspätet oder gar nicht reagieren. Ebenso sind im genannten Forum auch 
\emph{Fragen}, \emph{Kritik} und \emph{Verbesserungsvorschläge}~-- sowohl das 
Bundle selbst als auch die Dokumentation betreffend~-- gerne gesehen. Da dieses 
Bundle in meiner Freizeit entstanden ist und auch gepflegt wird, bitte ich um 
Nachsicht, falls ich nicht sofort antworte und/oder eine Fehlerkorrektur 
vornehmen kann.

\makeatletter
\medskip
\noindent Falk Hanisch\newline
Dresden, \@date
\makeatother

\tableofcontents
\chapter{Einleitung}
%
Zur fehlerfreien Verwendung der \TUDScript-Klassen der Version~\vTUDScript{} 
werden sowohl die \KOMAScript"=Klassen~\vKOMAScript{} oder später als auch die 
beiden Hausschriften des \CDs \Univers und \DIN zwingend benötigt. Außerdem 
müssen weitere Pakete durch die genutzte \hologo{LaTeX}"=Distribution 
bereitgestellt werden. 

Bei den aktuellen Distributionen
\index{Distribution}%
\Distribution{\hologo{TeX}~Live}[2016]|?|,
\Distribution{Mac\hologo{TeX}}[2016]|?| und 
\Distribution{\hologo{MiKTeX}}[2.9]|?|
ist das mit großer Sicherheit kein Problem. Nutzen Sie jedoch eine frühere 
Distribution, könnte dies zu Problemen führen. Dann sollte bestenfalls eine der 
aktuellen Distributionen installiert werden. Ist dies nicht möglich, müssen die 
unter \autoref{sec:packages:needed} aufgeführten Pakete sowie \TUDScript 
(\autoref{sec:local:install}) in der jeweils benötigten Version lokal 
installiert werden.

Das Vorlagenpaket von Klaus Bergmann ist für die Verwendung nicht notwendig. 
Allerdings beinhaltet dieses weitere Klassen zum Erstellen von Folien 
und Briefen.%
\footnote{%
  \Class{tudbook}, \Class{tudbeamer}, \Class{tudletter}, \Class{tudfax}, 
  \Class{tudhaus}, \Class{tudform}%
}
Das \TUDScript"=Bundle ist hauptsächlich für das Erstellen wissenschaftlicher 
Texte und Arbeiten gedacht und soll die ursprünglichen Klassen \emph{momentan} 
nicht ersetzen sondern vielmehr ergänzen. 

Eine Umsetzung des \CDs für die \Class{beamer}"=Klasse sowie für Briefe und 
Geschäftsschreiben auf Basis der \KOMAScript"=Brief"=Klasse \Class{scrlttr2} 
ist bis jetzt leider noch nicht entstanden, soll jedoch langfristig 
bereitgestellt werden. Allerdings existieren bereits im Bundle 
\Class{tudmathposter} für die \Class{beamer}"=Klasse mehrere Stile. Das Bundle 
ist sowohl bei \hrfn{https://github.com/tud-cd/tud-cd}{GitHub} als auf der 
\hrfn{https://tu-dresden.de/cd/4_latex}{\hologo{LaTeX}-Seite der \TnUD} zu 
finden.



\section{Zur Verwendung dieses Handbuchs}
Sämtliche neu definierten Optionen, Umgebungen und Befehle der 
\TUDScript-Klassen und \TUDScript-Pakete werden im Handbuch aufgeführt und 
beschrieben. Am Ende des Dokumentes befinden sich mehrere Indexe, die das 
Nachschlagen oder Auffinden von bisher unbekannten Befehlen oder Optionen 
erleichtern sollen.

Die im Folgenden beschriebenen Optionen können~-- wie ein Großteil aller 
Einstellungen für \KOMAScript~-- in der Syntax des \Package{keyval}"=Paketes 
als Schlüssel"=Wert"=Paare bei der Wahl der Dokumentklasse angegeben werden:
\Macro*{documentclass}[%
  \POParameter{\PName{Schlüssel}\PValue{=}\PName{Wert}}\Parameter{Klasse}%
]

Des Weiteren eröffnen die \KOMAScript"=Klassen die Möglichkeit der späten 
Optionenwahl. Damit können Optionen nicht nur direkt beim Laden als sogenannte 
Klassenoptionen angegeben werden, sondern lassen sich auch noch innerhalb des 
Dokumentes nach dem Laden der Klasse ändern. Die \KOMAScript"=Klassen sehen 
hierfür zwei Befehle vor. Mit 
\Macro{KOMAoptions}[\Parameter{Optionenliste}](\Package{koma-script})'none'
lassen sich beliebig viele Schlüsseln jeweils genau einen Wert zuweisen, 
\Macro{KOMAoption}[%
  \Parameter{Option}\Parameter{Werteliste}%
](\Package{koma-script})'none'
erlaubt das gleichzeitige Setzen mehrere Werte für genau einen Schlüssel. 
Äquivalent dazu werden für die von \TUDScript \emph{zusätzlich} zur Verfügung 
gestellten Optionen die Befehle \Macro{TUDoptions}[\Parameter{Optionenliste}] 
und \Macro{TUDoption}[\Parameter{Option}\Parameter{Werteliste}] definiert. 
Damit kann das Verhalten von Optionen im Dokument~-- innerhalb einer Gruppe 
auch lokal~-- geändert werden.

Die Voreinstellung jeder Option wird mit \enquote{Standardwert:\,\PName{Wert}} 
bei deren Beschreibung angeführt. Einige dieser Voreinstellungen sind nicht 
immer gleich sondern werden in Abhängigkeit der genutzten Benutzereinstellungen 
und Optionen gesetzt. Diese bedingten Voreinstellungen werden durch 
\enquote{%
  Standardwert:\,\PName{Wert}%
  \PValue{\,|\,}Bedingung:\,\PName{bedingter~Wert}%
}
angegeben. Wird ein Schlüssel durch den Benutzer \emph{ohne} eine Wertzuweisung 
genutzt, so wird~-- falls vorhanden~-- ein vordefinierter Säumniswert gesetzt, 
welcher in der Beschreibung aller Optionen durch die~\PValue{\emph{kursive}} 
Schreibweise innerhalb der Werteliste gekennzeichnet ist. In den meisten Fällen 
ist der Säumniswert eines Schlüssels \PValue{true}, er entspricht folglich der 
Angabe \PName{Schlüssel}\PValue{=true}. Mit der expliziten Wertzuweisung eines 
Schlüssels durch den Benutzer werden immer sowohl normale als auch bedingte 
Voreinstellungen überschrieben. Die neben den Optionen neu eingeführten Befehle 
und Umgebungen der Klassen werden im gleichen Stil erläutert.



\section{Installation des \TUDScript-Bundles}
\tudhyperdef*{sec:install}%
\index{Installation|!(}%
\index{Update}%
%
\ChangedAt{%
  v2.01:\TUDScript-Bundle auf CTAN veröffentlicht;%
  v2.02:Installationsroutine der PostScript-Schriften angepasst;%
  v2.04:Installationsskripte verbessert und robuster gestaltet sowie für die 
  beiden portablen Distributionen \Distribution*{\hologo{TeX}~Live~Portable} 
  und \Distribution*{\hologo{MiKTeX}~Portable} erweitert
}
%
Das \TUDScript-Bundle ist seit der Version~v2.01~-- aufgrund lizenzrechtlicher 
Bedingungen \emph{ohne} die geschützten Schriften \Univers und \DIN~-- im \CTAN
zu finden und kann dadurch die aktuellen \hologo{LaTeX}"=Distributionen wie 
\Distribution{\hologo{TeX}~Live}[2016]|?|, 
\Distribution{Mac\hologo{TeX}}[2016]|?| oder auch
\Distribution{\hologo{MiKTeX}}[2.9]|?| genutzt werden. Es besteht momentan aus 
den drei Hauptklassen \Class{tudscrbook}, \Class{tudscrreprt} sowie 
\Class{tudscrartcl}~-- dokumentiert in \autoref{sec:mainclasses}~-- und sowohl 
mit den \TUDScript-Klassen assoziierten als auch eigenständigen Paketen, welche 
in \autoref{sec:bundle} beschrieben sind.

Zur problemlosen Verwendung des \TUDScript-Bundles ist~-- neben \KOMAScript{} 
mindestens in der Version~\vKOMAScript{} sowie den in 
\autoref{sec:packages:needed} aufgeführten \hologo{LaTeX}"=Paketen~-- lediglich 
eine Installation der PostScript"=Schriften des \TUDCDs notwendig. Diese müssen 
über das Universitätsmarketing auf 
\hrfn{https://tu-dresden.de/cd/1_basiselemente/03_hausschrift/}{Anfrage} mit 
dem Hinweis auf die Verwendung von \hologo{LaTeX} bestellt werden. Sobald Sie 
die notwendigen Archive \File{Univers\_PS.zip} und \File{DIN\_Bd\_PS.zip} 
erhalten haben, können die Schriften für Windows (\autoref{sec:install:win}) 
beziehungsweise unixoide Betriebssysteme (\autoref{sec:install:unix}) 
installiert werden. Die benötigten Skripte werden als 
\hrfn{https://github.com/tud-cd/tudscr/releases/tag/fonts}{Release} 
im \GitHubRepo* bereitgestellt.%
\footnote{%
  Die Verwendung von installierten Systemschriften im Open"~Type"~Format mit 
  dem Paket \Package{fontspec} für \Engine{LuaLaTeX} oder \Engine{XeLaTeX} 
  wird mittlerweile unterstützt, mehr dazu in \autoref{sec:fonts:fontspec}.%
}

\Attention{%
  Die Skripte erzeugen für die Schriftinstallation \textbf{temporär} einige 
  Verzeichnisse und Dateien, weshalb diese nur fehlerfrei auf einem Laufwerk 
  ausgeführt werden können, auf dem der Anwender selbst Schreibrechte besitzt.
}


\minisec{Anmerkung zu Windows}
Sollte Windows genutzt werden und noch keine \hologo{LaTeX}"=Distribution auf 
ihrem System installiert sein, so rate ich persönlich zur Verwendung von 
\Distribution{\hologo{TeX}~Live}|?| statt \Distribution{\hologo{MiKTeX}}|?|. 
Der Vorteil dieser Distribution liegt zum einen in der Wartung durch mehrere 
Autoren sowie der früheren Verfügbarkeit aller Updates über CTAN. Zum anderen 
liefert \Distribution{\hologo{TeX}~Live}|?| zusätzlich zu \hologo{LaTeXe} einen 
\textsc{Perl}"=Interpreter sowie \textsc{Ghostscript}, wodurch die 
Ad"=hoc"=Verwendung einiger Pakete wie beispielsweise \Package{glossaries} 
vereinfacht beziehungsweise verbessert wird. 


\minisec{Anmerkung zu Linux und OS~X}
Die Installation von \Distribution{\hologo{TeX}~Live}|?| beziehungsweise 
\Distribution{Mac\hologo{TeX}}|?| sollte direkt über die im Internet 
angebotenen Installationspakete (\url{https://tug.org/texlive/} respektive 
\url{https://tug.org/mactex/}) und nicht über \Path{apt-get install} erfolgen. 
%
Die Installation der Schriften des \CDs \emph{muss zwingend} über das Terminal 
ausgeführt werden. Nach dem Entpacken eines Release-Archivs sollte im passenden 
Pfad (beispielsweise \Path{cd~"\$HOME/Downloads/\PName{Unterordner}"}) das 
Skript aus dem Terminal mit \Path{bash \PName{Skript}.sh} direkt ausgeführt 
werden. Mehr dazu ist in \autoref{sec:install:unix} zu finden.


\minisec{Anmerkung zu \NoCaseChange{\hologo{MiKTeX}}}
Vor der Installation der Schriften für \TUDScript sollte unbedingt ein Update 
von \Distribution{\hologo{MiKTeX}}|?| durchgeführt werden. Andernfalls wird 
es mit großer Sicherheit zu Problemen kommen. Außerdem ist es sehr ratsam, die 
Installation von \Distribution{\hologo{MiKTeX}}|?| in der Mehrbenutzervariante 
mit Administratorrechten durchzuführen, da die Einzelbenutzervariante relativ
unregelmäßig und nicht immer nachvollziehbar zu Problemen führen kann. 

Möglicherweise sind einige der für den Schriftinstallationsprozess notwendigen 
Pakete \Package{fontinst}, \Package{fontware} sowie \Package{cmbright}, 
\Package{hfbright}, \Package{cm-super}, \Package{lmodern} und \Package{iwona} 
noch nicht installiert. Ist die automatische Nachinstallation fehlender Pakete 
aktiviert, so reicht es im Normalfall das Installationsskript zu starten. 
Andernfalls müssen diese Pakete manuell durch den Benutzer über den 
\Distribution{\hologo{MiKTeX}}"=Paketmanager hinzugefügt werden.

Das Installationsskript scheitert außerdem bei einigen Anwendern~-- aufgrund 
eingeschränkter Nutzerzugriffsrechte~-- beim Eintragen der Schriften in die 
Map"~Datei. Dies muss gegebenenfalls durch den Anwender über die Kommandozeile 
\Path{initexmf -{}-edit-config-file updmap} erfolgen. In der sich öffnenden 
Datei sollte sich der Eintrag \Path{Map tudscr.map} befinden. Ist dies nicht 
der Fall, muss diese Zeile manuell eingetragen und die Datei anschließend 
gespeichert werden. Danach muss der Nutzer in der Kommandozeile noch 
\Path{initexmf -{}-mkmaps} ausführen.


\minisec{Anmerkung zu 
  \NoCaseChange{\hologo{TeX}}~Live und Mac\NoCaseChange{\hologo{TeX}}%
}
Für den Schriftinstallationsprozess werden die Pakete \Package{fontinst}, 
\Package{fontware} sowie \Package{cmbright}, \Package{hfbright}, 
\Package{cm-super}, \Package{lmodern} und \Package{iwona} benötigt. Sollte 
keine Vollinstallation von \Distribution{\hologo{TeX}~Live} durchgeführt worden 
sein, müssen diese Pakete sehr wahrscheinlich manuell durch den Benutzer über 
den \Distribution{\hologo{TeX}~Live}"=Paketmanager hinzugefügt werden.

Sind nach einem fehlerfreien Durchlauf des Installationsskriptes die Schriften 
dennoch nicht verfügbar, so lässt sich mit \Path{updmap-sys -{}-syncwithtrees} 
die Synchronisierung aller Schriftdateien anstoßen. Daran anschließend muss mit 
\Path{updmap-sys -{}-enable Map=tudscr.map} die Map"~Datei und die dazugehörigen
Schriftdateien mit \Path{updmap-sys -{}-force} registriert werden.

Sind die Schriften danach immer noch nicht verfügbar, so wurden bestimmt schon 
weitere Schriften auf dem System \emph{lokal} installiert. In diesem Fall 
sollte der Vorgang nochmals für eine lokale Schriftinstallation mit 
\Path{updmap -{}-syncwithtrees}, \Path{updmap -{}-enable Map=tudscr.map} und 
\Path{updmap -{}-force} ausgeführt werden. Dieses Vorgehen macht allerdings den 
Befehl \Path{updmap-sys} von nun an wirkungslos. Nach einer systemweiten 
Installation neuer Schriften~-- beispielsweise bei der Aktualisierung der 
Distribution~-- müssen diese über den manuellen Aufruf von \Path{updmap} 
zukünftig durch den Anwender lokal bei \Distribution{\hologo{TeX}~Live}|?| 
respektive \Distribution{Mac\hologo{TeX}}|?| registriert werden.

\Attention{%
  Für die Schriftinstallation werden die Skripte \Path{tftopl}, \Path{pltotf} 
  und \Path{vptovf} benötigt, welche bei \Distribution{\hologo{TeX}~Live}|?| 
  beziehungsweise \Distribution{Mac\hologo{TeX}}|?| über das Paket 
  \Package*{fontware} aus \Package*{collection-fontutils}<> 
  bereitgestellt werden und zwingend installiert sein müssen.
}


\minisec{Weiterführende Installationshinweise}
In \autoref{sec:install:ext} sind zusätzliche Varianten der Installation von 
\TUDScript zu finden. Soll eine ältere lokale Nutzerinstallation entfernt 
werden, um zukünftig alle Aktualisierungen direkt über die jeweils verwendete 
Distribution durchzuführen, werden entsprechende Skripte zu Deinstallation in 
\autoref{sec:local:uninstall} zur Verfügung gestellt.

Will der Anwender \emph{bewusst} eine lokale Nutzerinstallation anlegen, werden 
hierfür in \autoref{sec:local:install} entsprechende Skripte bereitgestellt. 
Soll eine bereits vorhandene, lokale Nutzerinstallation aktualisiert werden, 
finden Sie die entsprechenden Skripte in \autoref{sec:local:update}.

Für die portablen Varianten von \Distribution{\hologo{TeX}~Live~Portable}
respektive \Distribution{\hologo{MiKTeX}~Portable} sind zusätzliche 
Installationshinweise in \autoref{sec:install:portable} zu finden.


\subsection{Installation der PostScript-Schriften unter Windows}
\tudhyperdef*{sec:install:win}%
%
Zur Installation der Schriften des \CDs für das \TUDScript-Bundle ist das Archiv
\hrfn{\Download{fonts/TUD-Script-fonts-Windows.zip}}{\File*{TUD-Script\_fonts\_Windows.zip}}
vorgesehen. Dieses ist sowohl für \Distribution{\hologo{TeX}~Live}|?| als auch
\Distribution{\hologo{MiKTeX}}|?| nutzbar und enthält~-- bis auf die jeweiligen 
Schriftarchive selbst~-- alle benötigten Dateien. Diese sollten nach dem 
Entpacken des Archivs in das gleiche Verzeichnis kopiert werden. Vor der 
Verwendung des Skripts \File{tudscr\_fonts\_install.bat} sollte sichergestellt 
werden, dass sich \emph{alle} der folgenden Dateien im selben Verzeichnis 
befinden:
%
\settowidth\tempdim{\File{tudscr\_fonts\_install.zip}~}%
\begin{description}[labelwidth=\tempdim,labelsep=1em]
  \item[\File{tudscr\_fonts\_install.bat}]Installationsskript
  \item[\File{Univers\_PS.zip}]Archiv mit Schriftdateien für \Univers
  \item[\File{DIN\_Bd\_PS.zip}]Archiv mit Schriftdateien für \DIN
  \item[\File{tudscr\_fonts\_install.zip}]Archiv mit Metriken für die
    Schriftinstallation via \Package{fontinst}
\end{description}
%
Beim Ausführen des Installationsskripts werden alle Schriften standardmäßig in 
ein lokales Nutzerverzeichnis installiert. Wird das Skript über das Kontextmenü 
mit Administratorrechten ausgeführt, erfolgt die Installation in einem Pfad, 
der \emph{für alle Nutzer} gültig und lesbar ist.



\subsection{Installation der PostScript-Schriften unter Linux und OS~X}
\tudhyperdef*{sec:install:unix}%
%
Für die Erstellung des Installationsskripts für Linux und OS~X geht mein Dank 
an Jons"~Tobias Wamhoff, der sich für die erstmalige Portierung des Skripts 
von Windows zu unixartigen Systemen freiwillig zur Verfügung stellte.
Zur Installation der Schriften des \CDs für das \TUDScript-Bundle ist das Archiv
\hrfn{\Download{fonts/TUD-Script_fonts_Unix.zip}}{\File*{TUD-Script\_fonts\_Unix.zip}}
 vorgesehen. Dieses ist sowohl für \Distribution{\hologo{TeX}~Live}|?| als auch 
\Distribution{Mac\hologo{TeX}}|?| nutzbar und enthält~-- bis auf die jeweiligen 
Schriftdateien selbst~-- alle benötigten Dateien. Diese sollten nach dem 
Entpacken des Archivs in das gleiche Verzeichnis kopiert werden. Vor der 
Verwendung des Skripts \File{tudscr\_fonts\_install.sh} sollte sichergestellt 
werden, dass sich \emph{alle} der folgenden Dateien im selben Verzeichnis 
befinden:
%
\begin{description}[labelwidth=\tempdim,labelsep=1em]
  \item[\File{tudscr\_fonts\_install.sh}]Installationsskript
    (Terminal: \Path{bash tudscr\_fonts\_install.sh})
  \item[\File{Univers\_PS.zip}]Archiv mit Schriftdateien für \Univers
  \item[\File{DIN\_Bd\_PS.zip}]Archiv mit Schriftdateien für \DIN
  \item[\File{tudscr\_fonts\_install.zip}]Archiv mit Metriken für die
    Schriftinstallation via \Package{fontinst}
\end{description}
%
\Attention{%
  Das Installationsskript \textbf{muss} mit \Path{bash \PName{Skript}.sh} im 
  Terminal im Pfad mit den benötigten Dateien aufgerufen werden.
}
Dabei werden alle Schriften standardmäßig in das lokale Nutzerverzeichnis 
(\Path{\$TEXMFHOME}) installiert. Wird das Skript mit \Path{sudo} verwendet, 
erfolgt die Installation \emph{für alle Nutzer} in den lokalen Systempfad 
(\Path{\$TEXMFLOCAL}).

Es ist unbedingt darauf zu Achten, das beim Ausführen des Skriptes das Terminal 
im richtigen Verzeichnis aktiv ist. Bei den meisten unixoiden Betriebssystemen 
ist es problemlos möglich, das Terminal aus der Benutzeroberfläche heraus über 
das Kontextmenü im gewünschten Pfad zu öffnen. Geht dies nicht, so muss nach 
dem Öffnen des Terminals mit dem Befehl \Path{cd} erst zum entsprechenden 
Pfad~-- exemplarisch \Path{cd~"\$HOME/Downloads/\PName{Unterordner}"}~-- 
navigiert werden. Ein beispielhafter Aufruf im Terminal könnte also lauten:
%
\begin{quoting}
\Path{%
  cd~"\$HOME/Downloads/TUD-Script\_fonts\_Unix"{}\,\POParameter{ENTER}
}\newline
\Path{bash tudscr\_fonts\_install.sh\,\POParameter{ENTER}}
\end{quoting}



\subsection{Probleme bei der Installation der PostScript-Schriften}
%
Wird Windows verwendet, kann es unter Umständen vorkommen, dass notwendige 
Befehlsaufrufe für das Installationsskript nicht ausgeführt werden können. In 
diesem Fall ist der Pfad zu den benötigten Dateien, welche normalerweise unter 
\Path{\%SystemRoot\%\textbackslash System32} zu finden sind, nicht in der 
Umgebungsvariable \Path{PATH} enthalten. Einen Hinweis zur Problemlösung ist 
\hrfn{http://latex.wcms-file3.tu-dresden.de/phpBB3/viewtopic.php?t=359}{%
  in diesem Beitrag im Forum%
}
zu finden.

Treten bei der Installation wider Erwarten Probleme auf, so ist zur Lösung eine 
Logdatei zu erstellen. Hierfür sollte unter \textbf{Windows} das Skript, 
welches Probleme verursacht, \emph{nicht} aus der Kommandozeile oder dem 
Explorer heraus sondern über \emph{Windows PowerShell} ausgeführt werden. 
Hierfür ist die Eingabe von \enquote{PowerShell} im Startmenü von Windows mit 
einem nachfolgenden Öffnen mittels \POParameter{ENTER}"~Taste ausreichend. 
Danach muss mit \Path{cd} zum Ordner des Skriptes navigiert und dieses mit 
\Path{.\textbackslash\PName{Skript}.bat|Tee-Object -file \PName{Skript}.log} 
ausgeführt werden. Ein Aufruf aus der PowerShell"~Konsole könnte lauten:
%
\begin{quoting}[rightmargin=0pt]
  \Path{%
    cd~"\$env:USERPROFILE\textbackslash{}Downloads\textbackslash{}%
    TUD-Script\_fonts\_Windows"{}\,\POParameter{ENTER}%
  }\newline%
  \Path{%
    .\textbackslash{}tudscr\_fonts\_install.bat%
    |Tee-Object\,-file\,tudscr\_fonts\_install.log\,\POParameter{ENTER}%
  }%
\end{quoting}
%
Für \textbf{unixartige Systeme} ist der Aufruf \Path{bash \PName{Skript}.sh > 
\PName{Skript}.log} aus dem Terminal heraus zu verwenden. Ein exemplarischer  
Aufruf im könnte lauten:
%
\begin{quoting}
  \Path{%
    cd~"\$HOME/Downloads/TUD-Script\_fonts\_Unix"{}\,\POParameter{ENTER}
  }\newline
  \Path{%
    bash tudscr\_fonts\_install.sh > %
    tudscr\_fonts\_install.log\,\POParameter{ENTER}%
  }%
\end{quoting}
%
Die so erstellte Logdatei kann \emph{mit einer kurzen Fehlerbeschreibung} 
entweder im \Forum* gepostet oder direkt per E"~Mail an \mailto{\tudscrmail} 
gesendet werden.
\index{Installation|!)}%


\section{Schnelleinstieg}
Das Handbuch gliedert sich in drei Teile. In \autoref{part:main} ist die 
Dokumentation von \TUDScript zu finden. Hier werden alle neuen Optionen, 
Umgebungen und Befehle, die über die Funktionalität von \KOMAScript{} 
hinausgehen, erläutert. \autoref{part:additional} enthält zum einen einfache 
Minimalbeispiele, um den prinzipiellen Umgang und die Funktionalitäten von 
\TUDScript zu demonstrieren. Zum anderen werden hier auch ausführliche und 
dokumentierte Tutorials vor allem für \hologo{LaTeX}"=Neulinge angeboten. 
Insbesondere das Tutorial \Tutorial{treatise} ist mehr als einen Blick wert, 
wenn eine wissenschaftliche Arbeit mit \hologo{LaTeXe} verfasst werden soll.
Abschließend werden verschiedenste Pakete vorgestellt, die nicht speziell für 
das \TUDScript-Bundle selber sondern auch für andere \hologo{LaTeX}"=Klassen
verwendet werden können und demzufolge für alle \hologo{LaTeX}"=Anwender 
interessant sein könnten. Außerdem werden hier einige Tipps \& Tricks beim 
Umgang mit \hologo{LaTeX} beschrieben, um kleinere oder größere Probleme zu 
lösen.

Die Klassen \Class{tudscrbook}, \Class{tudscrreprt} und \Class{tudscrartcl} 
sind Wrapper"=Klassen der bekannten \KOMAScript-Klassen \Class{scrbook}, 
\Class{scrreprt} sowie \Class{scrartcl} und können einfach anstelle deren 
verwendet werden. Auf diesen basierende Dokumente können durch das Umstellen 
der Dokumentklasse einfach in das \TUDCD überführt werden. Bei Fragestellungen 
bezüglich Layout, Schriften oder ähnlichem ist in jedem Fall ein weiterer Blick 
in das hier vorliegende Handbuch empfehlenswert.




\setpartpreamble{%
  \begin{abstract}
    \hypersetup{linkcolor=red}
    \noindent Dies ist der Hauptteil des \TUDScript-Bundles. Hier findet der 
    Anwender alle verfügbaren Optionen, Umgebungen und Befehle, die über 
    die Funktionalität von \KOMAScript{} hinausgehen.
  \end{abstract}
}%
\part{Das \TUDScript-Bundle}
\tudhyperdef*{part:main}
\chapter[Die Klassen tudscrbook, tudscrreprt und tudscrartcl]{Die Hauptklassen}
\label{sec:mainclasses}
\ChangedAt*{%
  v2.00!Robustheit vieler Befehle und Optionen erhöht,%
  v2.02!Umbenennung einiger Befehle für Kompatibilität mit anderen Paketen,%
  v2.03!Anpassungen interner Befehle an \KOMAScript-Version~v3.15%
}
\begin{Declaration*}{\Class{tudscrbook}}
\begin{Declaration*}{\Class{tudscrreprt}}
\begin{Declaration*}{\Class{tudscrartcl}}
\index{Hauptklassen|!}
Es werden die drei neuen Hauptklassen
%
\begin{description}
\item \Class{tudscrbook}
\item \Class{tudscrreprt}
\item \Class{tudscrartcl}
\end{description}
%
eingeführt, welche auf den \KOMAScript-Klassen basieren und grundsätzlich alle
deren Optionen, Umgebungen und Befehle~-- beispielsweise \Option{parskip} für 
die Absatzeinstellungen oder \Option{BCOR} zur Festlegung der Bindekorrektur~-- 
unterstützen. Zusätzlich zu den \KOMAScript"=Klassen werden weitere Pakete 
zwingend benötigt, welche unter \autoref{sec:packages:needed} aufgeführt sind 
und auf jeden Fall durch \TUDScript geladen werden.

Es sei hier abermals auf die Anwenderdokumentation \scrguide von \KOMAScript{} 
hingewiesen, viele der folgend beschriebenen Befehle und Optionen beziehen sich 
auf die darin vorgestellten Einstellungsmöglichkeiten. Die Anpassungen und 
Erweiterungen der \KOMAScript"=Klassen an das \CD und die neu definierten 
beziehungsweise geänderten Befehle und Optionen werden im Folgenden erläutert.
\end{Declaration*}
\end{Declaration*}
\end{Declaration*}

\begin{Declaration}{\Macro{TUDoptions}\Parameter{Optionenliste}}
\begin{Declaration}{\Macro{TUDoption}\Parameter{Option}\Parameter{Werteliste}}
\printdeclarationlist%
\index{Optionen|!}%
%
Mit diesen Befehlen hat man bei den meisten der neuen Klassenoptionen die 
Möglichkeit, den Wert der Optionen noch nach dem Laden der Klasse zu ändern.
Man kann wahlweise mit der Anweisung \Macro{TUDoptions} die Werte einer Reihe 
von Optionen ändern. Jede Option der Optionenliste hat dabei die Form
\PName{Option}\PValue{=}\PName{Wert}. Die meisten Optionen besitzen auch einen 
Säumniswert\footnote{engl.: default value}. Versäumt man die Angabe eines 
Wertes~-- verwendet demzufolge einfach die Form \PName{Option}~-- so wird 
automatisch dieser Säumniswert angenommen.

Manche Optionen können gleichzeitig mehrere Werte besitzen. Für diese besteht 
die Möglichkeit, mit \Macro{TUDoption} der einen Option nacheinander eine 
Reihe von Werten zuzuweisen. Die einzelnen Werte sind dabei in der Werteliste 
durch Komma voneinander getrennt.

Mit diesen beiden Befehlen kann im Bedarfsfall das Verhalten von einer Option 
oder mehreren Optionen im Dokument geändert werden. Werden diese Befehle in 
einer Umgebung oder einer Gruppe verwendet, bleiben die gemachten Einstellungen 
innerhalb dieser lokal begrenzt.
\end{Declaration}
\end{Declaration}



\section{Die Schriften des \CDs}
\label{sec:fonts}
\index{Schrift|?(}
%
\ChangedAt*{%
  v2.00!Schriften~-- insbesondere für den mathematischen Satz~-- verbessert,%
  v2.01!Unterschneidung (Kerning) der Ziffern für \Univers verbessert%
}
Das \CD der \TnUD gibt die Verwendung der Schriften \Univers für den Fließtext 
sowie \DIN für das Setzen von Überschriften vor, was durch \TUDScript in der 
Standardkonfiguration auch so umgesetzt wird. Da jedoch in längeren Texten die 
Verwendung von Serifenschriften zu empfehlen ist, gibt es die Möglichkeit, die 
ursprünglich vorgesehenen Schriften des \CDs nicht zu laden und stattdessen die 
\hologo{LaTeX}-Standardschriften beziehungsweise ein anderes Schriftpaket zu 
verwenden. Die Erläuterungen dazu sind in \autoref{sec:text} zu finden.

Durch das \CD werden keine Schriften für den Mathematiksatz festgelegt. Das ist 
insbesondere für sowohl mathematische Abhandlungen als auch ingenieur- und 
naturwissenschaftliche Dokumente nicht tragbar. Dieser Mangel wird behoben, 
indem im Mathematikmodus die lateinischen Buchstaben der Hausschriften mit 
griechischen Lettern und mathematischen Symbolen aus anderen Paketen ergänzt 
werden.%
\footnote{%
  \Package{iwona} für die Schrift \DIN und zusätzlich \Package{cmbright} für 
  die \Univers"=Schriftfamilie%
}
Diese Grundeinstellung lässt sich ebenfalls deaktivieren, wodurch die 
Standardschriften oder gegebenenfalls die eines zusätzlich geladenen Paketes 
für den mathematischen Satz genutzt werden. Die dafür relevanten Einstellungen 
werden in \autoref{sec:math} erläutert. In \autoref{sec:exmpl:mathtype} sowie 
\autoref{sec:exmpl:mathswap} sind zusätzliche Hinweise zum typografisch guten 
Mathematiksatz zu finden.


\subsection{Schriften für den Textsatz}
\begin{Declaration}[%
  v2.02!Werte für Option \Option{cdhead} ergänzt%
]{\Option{cdfont}[\PSet]}[true]%
\printdeclarationlist%
\label{sec:text}%
\index{Schrift!Fließtext}\index{Schriftstärke}%
%
Mit der Option \Option{cdfont} können durch den Anwender alle zentralen 
Schrifteinstellungen für die \TUDScript-Klassen vorgenommen werden. Dies 
betrifft sowohl die Schriften für Überschriften als auch den Fließtext sowie 
die Mathematikschriften. Die verwendete Schriftstärke im charakteristischen 
Querbalken der Kopfzeile lässt sich hiermit ebenfalls einstellen.
%
\begin{values}
\itemfalse
  Es werden die \hologo{LaTeX}"=Standardschriften und nicht die Hausschriften 
  des \CDs verwendet. Der Anwender kann beliebige Schriftpakete nutzen.%
  \footnote{%
    Für die Verwendung der klassischen \hologo{LaTeX}"=Schriften, ist das Paket 
    \Package{lmodern} sehr empfehlenswert.%
  }
  Sollte das Layout des \CDs aktiviert sein (\see*{\Option{cd}}), werden die 
  Überschriften in Großbuchstaben und \DIN gesetzt und nur die Schriftart des 
  Fließtextes kann angepasst werden.
\itemtrue*[light/lightfont/noheavyfont]
  Es werden die Hausschriften im Stil des \CDs der \TnUD genutzt. Überschriften 
  der obersten Gliederungsebenen bis einschließlich \Macro*{subsubsection} 
  verwenden \DIN, darunter liegende%
  \footnote{\Macro{paragraph} und \Macro{subparagraph}} 
  \textubn{Univers~65~Bold}. Für den Fließtext im Dokument kommt 
  \textuln{Univers~45~Light} zum Einsatz. Aus \Package{lmodern} wird die
  \texttt{Schreibmaschinenschrift} verwendet.
\item[heavy/heavyfont]
  Die Schriftstärke der Hausschriften wird erhöht. Die beiden untersten 
  Gliederungsebenen werden in \textuxn{Univers~75~Black} gesetzt, der Fließtext 
  in \texturn{Univers~55~Regular}. Ansonsten entspricht alles der Option 
  \Option{cdfont}[true]. Die Mathematikschriften werden nicht beeinflusst, 
  diese lassen sich gegebenenfalls mit \Macro{boldmath} auf den fetten Schnitt 
  umschalten.
\item[nodin]
  Für die Überschriften wird \DIN nicht verwendet. Für \Option{cdfont}[true] 
  wird \Univers genutzt. Die Schriftstärke ist dabei abhängig von der   
  Einstellung \Option{cdfont}[light/heavy]. Ist die Verwendung der Schriften 
  des \CDs deaktiviert (\Option{cdfont}[false]), kommt die fette Schriftstärke 
  der eingestellten serifenlosen Schriftfamilie zum Einsatz.
\item[din]
  Mit dieser Einstellung wird die Schrift \DIN in den Überschriften verwendet. 
  Diese ist standardmäßig aktiviert.
\end{values}
%
\ChangedAt{v2.02}
Für den Text im Querbalken gibt es folgende Einstellmöglichkeiten:
%
\begin{values}
\item[head/lighthead/lightfonthead/noheavyfonthead]
  Für den Querbalken der Kopfzeile wird unabhängig von der Verwendung der 
  Hausschriften die Schrift \Univers in normaler Schriftstärke verwendet,
  \see*{\Option{cdhead}[true]}.
\item[heavyhead/heavyfonthead]
  Die im Querbalken verwendete Schrift ist \Univers in erhöhter Stärke, 
  \see*{\Option{cdhead}[heavy]}.
\end{values}
%
Die verwendeten Mathematikschriften lassen sich mit folgenden Werte 
beeinflussen:
%
\begin{values}
\item[nomath/nocdmath]  
  Diese Einstellung deaktiviert die Verwendung von serifenlosen Schriften für 
  den mathematischen Satz. Es werden die \hologo{LaTeX}"=Standardschriften 
  verwendet und der Benutzer kann beliebige Schriftpakete für den 
  Mathematikmodus nutzen, \see*{\Option{cdmath}[false]}.
\item[math/cdmath]
  Es werden serifenlose Mathematikschriften für lateinische und griechische 
  Lettern genutzt, \see*{\Option{cdmath}[true]}.
\item[upgreek/uprightgreek]
  Die großen griechischen Buchstaben werden im Mathematikmodus aufrecht gesetzt,
  \see*{\Option{slantedgreek}[false]}.
\item[slgreek/slantedgreek]
  In mathematischen Umgebungen erfolgt die Ausgabe der griechischen Majuskeln 
  kursiv, \see*{\Option{slantedgreek}[true]}.
\end{values}
%
\ChangedAt{v2.02,v2.04!Einfachere Verwendung von \Package{fontspec}}
Normalerweise kommen die Schriften des \CDs im PostScript"=Format zum Einsatz, 
wenn diese wie unter \autoref{sec:install} beschrieben installiert wurden.
Wird entweder \hologo{LuaLaTeX} oder \hologo{XeLaTeX} als Dokumentprozessor 
verwendet und das Paket \Package{fontspec} geladen, so werden die Schriften des 
\CDs im OpenType"=Format verwendet. Hierfür sind die Hinweise in 
\fullref{sec:fonts:fontspec} unbedingt zu beachten.
\end{Declaration}


\subsubsection{Auszeichnungen in Überschriften}
\index{Schriftelemente}
Für die Schriftauswahl der Überschriften aller Gliederungsebenen sind die durch 
\KOMAScript{} bereitgestellten Schriftelemente verantwortlich. Mehr dazu ist in 
\autoref{sec:fonts:elements} zu finden. Da die Überschriften der obersten 
Gliederungsebenen bis einschließlich \Macro*{subsubsection} normalerweise in 
Majuskeln gesetzt werden, bestehen für den Anwender mit den folgenden Befehlen 
gewisse Einflussmöglichkeiten, deren Ausprägung anzupassen.

\begin{Declaration}{\Macro{ifdin}\Parameter{Dann-Teil}\Parameter{Sonst-Teil}}%
\printdeclarationlist%
\index{Überschriften}\index{Schrift!Überschriften}\index{Schriftauszeichnung}%
\index{Kolumnentitel}\index{Layout!Kolumnentitel}
%
Der Befehl \Macro{ifdin} prüft, ob die Schriftfamilie \DIN aktiv ist und führt 
in diesem Fall \Parameter{Dann-Teil} aus, andernfalls \Parameter{Sonst-Teil}. 
Dies ist beispielsweise bei Überschriften sinnvoll, wenn innerhalb des 
obligatorischen Argumentes zwischen der Ausgabe im Dokument selber und dem 
Eintrag für das Inhaltsverzeichnis sowie der Ausprägung der automatischen 
Kolumnentitel unterschieden werden soll.
\end{Declaration}

\begin{Declaration}{\Macro{MakeTextUppercase}\Parameter{Text}}%
\begin{Declaration}{\Macro{NoCaseChange}\Parameter{Text}}%
\printdeclarationlist%
\index{Überschriften}\index{Schrift!Überschriften}\index{Schriftauszeichnung}%
%
Der Befehl \Macro{MakeTextUppercase} stammt aus dem Paket \Package{textcase} 
und setzt den Text seines Argumentes in Majuskeln. Die Überschriften der 
Gliederungsebenen bis einschließlich \Macro*{subsubsection} werden damit in 
Großbuchstaben der Schrift \DIN gesetzt, wenn das Layout des \CDs nicht 
deaktiviert wurde (\Option{cd}[false]). Sollen in einer Überschrift bestimmte 
Kleinbuchstaben erhalten bleiben, ist der Befehl \Macro{NoCaseChange} zu 
nutzen, welcher ebenfalls von besagtem Paket bereitgestellt wird.
\end{Declaration}
\end{Declaration}
%
\begin{Example}
In einer Kapitelüberschrift wird ein einzelnes Wort in Kleinbuchstaben 
geschrieben:
\begin{Code}[escapechar=§]
\chapter{§Ü§berschrift mit \NoCaseChange{kleinem} Wort}
\end{Code}
\end{Example}

\subsubsection{Auszeichnungen im Text}
\index{Schrift!Fließtext}\index{Schriftstärke}%
\index{Schrift!Befehle}\index{Schrift!Schalter}%
%
Unabhängig davon, welche Schriftfamilie verwendet wird, können die Schriften 
des \CDs jederzeit entweder mit einem der hier aufgeführten Textschalter oder 
Textkommandos innerhalb des Dokumentes genutzt werden. Ein Textschalter wirkt 
sich~-- wenn er nicht in einer Gruppe oder einer Umgebung verwendet und damit 
lokal begrenzt wird~-- global auf das Dokument aus, wie etwa beispielsweise 
\Macro*{bfseries}. Bei einem Textkommando hingegen erfolgt die Änderung der 
Schriftart nur für das nachfolgend angegebene Argument, wie zum Beispiel bei
\Macro*{textbf}\Parameter{Text}. Darauf ist bei der Nutzung zu achten. 
%
\begin{Declaration}{\Macro{univln}}
\begin{Declaration}{\Macro{textuln}\Parameter{Text}}
\begin{Declaration}{\Macro{univrn}}
\begin{Declaration}{\Macro{texturn}\Parameter{Text}}
\begin{Declaration}{\Macro{univbn}}
\begin{Declaration}{\Macro{textubn}\Parameter{Text}}
\begin{Declaration}{\Macro{univxn}}
\begin{Declaration}{\Macro{textuxn}\Parameter{Text}}
\begin{Declaration}{\Macro{univls}}
\begin{Declaration}{\Macro{textuls}\Parameter{Text}}
\begin{Declaration}{\Macro{univrs}}
\begin{Declaration}{\Macro{texturs}\Parameter{Text}}
\begin{Declaration}{\Macro{univbs}}
\begin{Declaration}{\Macro{textubs}\Parameter{Text}}
\begin{Declaration}{\Macro{univxs}}
\begin{Declaration}{\Macro{textuxs}\Parameter{Text}}
\begin{Declaration}{\Macro{dinbn}}
\begin{Declaration}{\Macro{textdbn}\Parameter{Text}}
\settowidth\tempdim{\Macro{textuln}\Parameter{Text}}%
\addtolength\tempdim{\dimexpr 2\tabcolsep+2\arrayrulewidth-\textwidth}%
\printdeclarationlist(%
  \begin{minipage}{-\tempdim}%
  \centering%
  \begin{tabularm}{3}%
    \toprule%
    \textbf{Schriftart}                  & \textbf{Schalter}
      & \textbf{Textkommando}\tabularnewline
    \midrule
    \textuln{Univers 45 Light}           & \Macro{univln}{}
      & \Macro{textuln}\Parameter{Text}\tabularnewline
    \texturn{Univers 55 Regular}         & \Macro{univrn}{}
      & \Macro{texturn}\Parameter{Text}\tabularnewline
    \textubn{Univers 65 Bold}            & \Macro{univbn}{}
      & \Macro{textubn}\Parameter{Text}\tabularnewline
    \textuxn{Univers 75 Black}           & \Macro{univxn}{}
      & \Macro{textuxn}\Parameter{Text}\tabularnewline
    \textuls{Univers 45 Light Oblique}   & \Macro{univls}{}
      & \Macro{textuls}\Parameter{Text}\tabularnewline
    \texturs{Univers 55 Regular Oblique} & \Macro{univrs}{}
      & \Macro{texturs}\Parameter{Text}\tabularnewline
    \textubs{Univers 65 Bold Oblique}    & \Macro{univbs}{}
      & \Macro{textubs}\Parameter{Text}\tabularnewline
    \textuxs{Univers 75 Black Oblique}   & \Macro{univxs}{}
      & \Macro{textuxs}\Parameter{Text}\tabularnewline
    \DIN & \Macro{dinbn}{}
      & \Macro{textdbn}\Parameter{Text}\tabularnewline
    \bottomrule%
    \allcolumnpar{\footnotesize\vskip0pt%
       Die Schrift \DIN darf laut \CD nur mit Majuskeln (Großbuchstaben) 
       verwendet werden. Wird diese Schrift manuell verwendet, sollte dies mit 
       \Macro{MakeTextUppercase}\PParameter{\Macro{textdbn}\Parameter{Text}}  
       geschehen. Sollen dabei im Argument einzelne Teile zwingend klein 
       geschrieben werden, wird der Befehl \Macro{NoCaseChange} benötigt.
    }
  \end{tabularm}%
  \end{minipage}%
)%
Alternativ zu den beschriebenen Textschaltern und -kommandos können seit der 
Version~v2.04 auch die beiden Befehle \Macro{cdfont} und \Macro{textcdfont} 
verwendet werden, welche die gleiche Funktionalität wesentlich komfortabler 
bereitstellen.
\end{Declaration}
\end{Declaration}
\end{Declaration}
\end{Declaration}
\end{Declaration}
\end{Declaration}
\end{Declaration}
\end{Declaration}
\end{Declaration}
\end{Declaration}
\end{Declaration}
\end{Declaration}
\end{Declaration}
\end{Declaration}
\end{Declaration}
\end{Declaration}
\end{Declaration}
\end{Declaration}

\begin{Declaration}[v2.04]{\Macro{cdfont}\Parameter{Schriftart}}
\begin{Declaration}[v2.04]{%
  \Macro{textcdfont}\Parameter{Schriftart}\Parameter{Text}%
}
\printdeclarationlist
Diese beiden Befehle dienen ebenfalls zu gezielten Aktivierung einer Schriftart 
des \CDs in Stärke und Schnitt. Hierbei entspricht \Macro{cdfont} einem 
Textschalter und ändert die aktuell verwendete Schriftart im aktuellen 
Geltungsbereich auf \PName{Schriftart}, wohingegen \Macro{textcdfont} als 
Textkommando fungiert und den im zweiten Argument gegebenen \PName{Text} in 
\PName{Schriftart} setzt ohne dabei die Dokumentschriftart selbst zu ändern.

Für die Schriftauswahl muss im ersten Argument der Name der zu verwendenden 
Schriftart angegeben werden. Dieser ist der obigen Tabelle zu entnehmen. Für 
die Auswahl der Schriftfamilie \Univers kann der Vorsatz \PValue{Univers} im 
Argument \PName{Schriftart} entfallen. Ebenso sind weder Leerzeichen noch die 
passende Groß- und Kleinschreibung notwendig. Für die Wahl der Schriftstärke 
ist die entsprechende Zahl \emph{oder} die Bezeichnung allein ausreichend.%
\footnote{\PValue{45/55/65/75} oder \PValue{Light/Regular/Bold/Black}}
Anstelle des Suffix' \PValue{Oblique} ist auch die Nutzung von \PValue{Italic} 
oder \PValue{Slanted} als Alias für die geneigten Schriftschnitte möglich. Zur 
Auswahl von \DIN ist \PValue{din} als Argument hinreichend.

\end{Declaration}
\end{Declaration}


\subsection{Schriften für den Mathematiksatz}
\begin{Declaration}[v2.03]{\Option{cdmath}[\PBoolean]}%
  [true][\Option{cdfont}[false]:false]
\printdeclarationlist%
\label{sec:math}
\index{Schrift!Mathematiksatz}\index{Mathematiksatz|!}
\index{Schrift!Griechische Buchstaben}\index{Griechische Buchstaben}
%
Diese Option dient zur Anpassung der Mathematikschriften. Wird diese aktiviert, 
so werden zu den Hausschriften passende im Mathematikmodus genutzt, mit 
\Option{cdmath}[false] wird auf die Standardschriften zurückgeschaltet. Ein 
Umschalten innerhalb des Dokumentes ist~-- beispielsweise für Abbildungen oder 
Tabellen~-- durch \Macro{TUDoptions}\PParameter{\Option{cdmath}[true/false]} 
möglich. Mit \Macro{boldmath} kann auf fette Mathematikschriften umgeschaltet 
werden. Gültige Werte für die Option \Option{cdmath} sind:
%
\begin{values}
\itemfalse
  Es werden die normalen \hologo{LaTeX}"=Serifenschriften beziehungsweise die 
  Schriften beliebig nutzbarer Pakete für den Mathematiksatz verwendet.
\itemtrue*
  Im Mathematikmodus wird \Univers genutzt. Außerdem kommen die griechischen 
  Buchstaben aus dem Paket \Package{cmbright} sowie Symbole aus dem Paket 
  \Package{iwona} zum Einsatz.
\item[upgreek/uprightgreek]
  Die griechischen Majuskeln werden aufrecht gesetzt, 
  \see*{\Option{slantedgreek}[false]}.
\item[slgreek/slantedgreek]
  Die Ausgabe der griechischen Großbuchstaben erfolgt kursiv, 
  \see*{\Option{slantedgreek}[true]}.
\end{values}
\end{Declaration}

\subsubsection{Griechischen Buchstaben}
\vskip-\lastskip%
\label{sec:greek}%
\index{Schrift!Griechische Buchstaben}\index{Griechische Buchstaben}%
%
\begin{Declaration}{\Macro{varDelta}}
\begin{Declaration}{\Macro{varTheta}}
\begin{Declaration}{\Macro{varLambda}}
\begin{Declaration}{\Macro{varXi}}
\begin{Declaration}{\Macro{varPi}}
\begin{Declaration}{\Macro{varSigma}}
\begin{Declaration}{\Macro{varUpsilon}}
\begin{Declaration}{\Macro{varPhi}}
\begin{Declaration}{\Macro{varPsi}}
\begin{Declaration}{\Macro{varOmega}}
\begin{Declaration}{\Macro{upDelta}}
\begin{Declaration}{\Macro{upTheta}}
\begin{Declaration}{\Macro{upLambda}}
\begin{Declaration}{\Macro{upXi}}
\begin{Declaration}{\Macro{upPi}}
\begin{Declaration}{\Macro{upSigma}}
\begin{Declaration}{\Macro{upUpsilon}}
\begin{Declaration}{\Macro{upPhi}}
\begin{Declaration}{\Macro{upPsi}}
\begin{Declaration}{\Macro{upOmega}}
\index{Schrift!Griechische Buchstaben}\index{Griechische Buchstaben}%
\settowidth\tempdim{\Macro{varUpsilon}}%
\addtolength\tempdim{\dimexpr 2\tabcolsep+2\arrayrulewidth-\textwidth\relax}%
\printdeclarationlist(%
  \begin{minipage}{-\tempdim}%
    \newcommand\tablecontent{}%
    \newcommand*\greekLetters{%
      Delta,Theta,Lambda,Xi,Pi,Sigma,Upsilon,Phi,Psi,Omega%
    }%
    \def\do#1{\appto\tablecontent{%
      \Macro*{var#1} & $\csuse{var#1}$ & & 
      \Macro*{up#1} & $\csuse{up#1}$\tabularnewline
    }}%
    \expandafter\docsvlist\expandafter{\greekLetters}%
    \centering%
    \vspace{\intextsep}\noindent
    \begin{tabularm}{5}
      \toprule%
      \textbf{Befehl (kursiv)} & \textbf{Symbol} & &
      \textbf{Befehl (aufrecht)} & \textbf{Symbol}
      \tabularnewline\midrule\tablecontent\bottomrule%
      \allcolumnpar{\footnotesize\vskip0pt%
        Die Befehle \Macro*{up}\PName{Name} und \Macro*{var}\PName{Name}
        werden normalerweise durch einige Pakete, unter anderem auch von 
        \Package{cmbright} oder \Package{amsmath}, bereitgestellt.
      }
    \end{tabularm}
  \end{minipage}%
)%
%
Unabhängig von den beiden Optionen \Option{cdmath} und \Option{slantedgreek} 
können sowohl kursive als auch aufrechte griechischen Großbuchstaben im 
Mathematikmodus mit diesen Befehlen direkt verwendet werden. Dies ist nützlich, 
um zwischen kursiven Variablen und aufrechten Konstanten zu unterscheiden. Die 
griechischen Minuskeln sind leider nur in der kursiven Variante verfügbar.
\end{Declaration}
\end{Declaration}
\end{Declaration}
\end{Declaration}
\end{Declaration}
\end{Declaration}
\end{Declaration}
\end{Declaration}
\end{Declaration}
\end{Declaration}
\end{Declaration}
\end{Declaration}
\end{Declaration}
\end{Declaration}
\end{Declaration}
\end{Declaration}
\end{Declaration}
\end{Declaration}
\end{Declaration}
\end{Declaration}

\begin{Declaration}{\Option{slantedgreek}[\PBoolean]}[false]
\printdeclarationlist%
%
Die Option ändert die standardmäßige Neigung der griechischen Großbuchstaben im 
Mathematikmodus bei der Verwendung der Befehle \Macro*{Delta}, \Macro*{Theta}, 
\Macro*{Lambda}, \Macro*{Xi}, \Macro*{Pi}, \Macro*{Sigma}, \Macro*{Upsilon}, 
\Macro*{Phi}, \Macro*{Psi} und \Macro*{Omega}. Wie unabhängig von der Option 
\Option{slantedgreek} gezielt kursive und aufrechte Buchstaben gesetzt werden 
können, wird \vpageref{sec:greek} beschrieben.
%
\begin{values}
\itemfalse
  Die griechischen Majuskeln werden wie bei den Standardklassen aufrecht 
  gesetzt.
\itemtrue*
  Die Ausgabe der griechischen Großbuchstaben erfolgt kursiv.
\end{values}
\end{Declaration}


\subsubsection{Zusätzliche Hinweise zum Mathematiksatz}
Weitere Hinweise zum typografisch guten Mathematiksatz sind außerdem in 
\autoref{sec:exmpl:mathswap} sowie \autoref{sec:exmpl:mathtype} zu finden.


\subsection{Die Schriften des \CDs im OpenType-Format}
\label{sec:fonts:fontspec}
\index{OpenType-Schriften}
%
\ChangedAt{v2.02!OpenType-Schriften mit \Package{fontspec} verwendbar}
Das \TUDScript-Bundle unterstützt die Nutzung der Schriften des \CDs sowohl 
im PostScript- als auch im OpenType"=Format. Letztere müssen ebenfalls über das 
\hrfn{http://tu-dresden.de/service/publizieren/cd/1_basiselemente/03_hausschrift/schriftbestellung.html}%
{Universitätsmarketing auf Anfrage} bestellt werden. Die in den beiden Archiven 
\File{Univers\_8\_TTF.zip} und \File{DIN\_TTF.zip} enthaltenen Schriften müssen 
für das Betriebssystem installiert werden und lassen sich anschließend mit dem 
Paket \Package{fontspec} verwenden.

Auf die Installation der PostScript"=Schriften kann dennoch nicht ohne Weiteres 
verzichtet werden. Denn einerseits sind diese für die Kompilierung eines 
\hologo{LaTeX}"~Dokumentes über den klassischen Prozesspfad via
\Path{latex \textrightarrow{} dvips \textrightarrow{} ps2pdf}~-- wie es 
beispielsweise für die Erstellung von Grafiken mit \Package{pstricks} notwendig 
ist~-- nötig. Andererseits liefern die Schriftfamilien des \CDs keinerlei 
mathematische Glyphen, sodass diese bei der PostScript"=Schriftinstallation aus 
den Schriftpaketen \Package{cmbright} und \Package{iwona} entnommen werden. Bei 
der Nutzung der Schriften im OpenType"=Format ist dies leider nicht so einfach 
möglich, da es schlichtweg an passenden Schriftpaketen für den Mathematiksatz 
im OpenType"=Format mangelt. Weiterhin kommt es auch beim Kerning der Schriften 
zu Problemen.

Die Verwendung der Schriften des \CDs im OpenType"=Format sollte folglich nur 
erfolgen, wenn eine Installation der PostScript"=Schriften \emph{absolut} nicht 
möglich ist beziehungsweise \hologo{LuaLaTeX} oder \hologo{XeLaTeX} zwingend 
genutzt werden müssen. Hierfür ist es~-- nach der systemweiten Installation der 
OpenType"=Schriften~-- ausreichend, das Paket \Package{fontspec} zu laden.
\index{Schrift|?)}%



\section{Das Layout des \CDs}
Das Hauptaugenmerk der neuen Klassen liegt auf der Umsetzung des \CDs der
\TnUD für \hologo{LaTeX}. Ein großer Teil der definierten Optionen und Befehle
dient genau dazu und wird nachfolgend beschrieben.

Einige spezielle Seiten werden im prägnanten Stil mit dem Logo der \TnUD und 
der dazugehörigen Kopfzeile mit Querbalken gesetzt. Dies betrifft insbesondere 
\hyperref[sec:title]{die Umschlagseite und den Titel aus \autoref{sec:title}}, 
die \hyperref[sec:part]{Teileseiten in \autoref{sec:part}} sowie die
\hyperref[sec:chapter]{Kapitelseiten in \autoref{sec:chapter}}. Mit den 
\PageStyle{tudheadings}"=Seitenstilen oder der \Environment{tudpage}-Umgebung  
können weitere Seiten in diesem Stil erzeugt werden. Wird das Paket 
\Package{tudscrsupervisor} verwendet und mit den bereitgestellten Befehlen oder 
Umgebungen eine Aufgabenstellung, ein Gutachten oder ein Aushang erstellt, so 
erscheinen auch diese in besagtem Seitenstil.


\subsection{Die Gestalt von Titel, Umschlagseite, Teilen und Kapiteln}
\begin{Declaration}[%
  v2.03!neue Farbvarianten
    (\protect\PValue{bicolor} und \protect\PValue{fullcolor}),%
  v2.04!farbiger Querbalken möglich (\protect\PValue{barcolor})%
]{\Option{cd}[\PSet]}[true]
\printdeclarationlist%
\index{Layout}%
%
Mit dieser Option wird festgelegt, ob und wie das \CD der \TnUD verwendet wird. 
Sie hat Einfluss auf die Ausprägung von Titel"~, Teil"~, und Kapitelseiten 
sowie die Überschriften der weiteren Gliederungsebenen.
%
\begin{values}
\itemfalse
  Diese Einstellung erzeugt das Standard"=Verhalten der \KOMAScript"=Klassen, 
  es wird kein \CD genutzt.
\itemtrue*[nocolor/monochrome]
  Das Layout für Titel"~, Teil"~ und Kapitelseiten ist im \CD, es wird 
  schwarze Schrift für Titel, Teil"~ und Kapitelüberschriften verwendet. Die 
  Ausprägung des Seitenkopfes ist abhängig von der Option \Option{cdhead}.
\item[lightcolor/pale]
  Die Einstellung entspricht weitestgehend der Option \Option{cd}[true], 
  allerdings wird die primäre Hausfarbe \Color{HKS41} für den Kopf des 
  \PageStyle{tudheadings}"=Seitenstils und die Überschriften genutzt.
\item[barcolor]
  \ChangedAt{v2.04} Zusätzlich zur vorherigen Einstellung wird außerdem der 
  Querbalken farblich abgesetzt.
\item[bicolor/bichrome]
  \ChangedAt{v2.03} Der Kopf wird mit einem farbigen Hintergrund in der 
  Hausfarbe gesetzt, auch der Querbalken wird farbig hinterlegt. Für die 
  Überschriften wird die primären Hausfarbe verwendet.
\item[color]
  Der Titel sowie Teil"~ und Kapitelseiten werden allesamt farbig und im \CD 
  gestaltet, der Seitenkopf wird in der primären Hausfarbe \Color{HKS41} 
  gesetzt, der Querbalken erhält Linien als Begrenzung.
\item[full/fullcolor]
  \ChangedAt{v2.03} Entspricht der vorherigen Einstellung, allerdings wird der 
  Querbalken nicht durch Linien begrenzt sondern farbig hinterlegt.
\end{values}
\end{Declaration}


\subsubsection{Einstellungen für Titel, Umschlagseite, Teile und Kapitel}
Das Verhalten aller für das Layout relevanten Elemente wird von der eben zuvor 
erläuterten Option \Option{cd}[\PSet] bestimmt. Dies betrifft zum einen sowohl 
den Titel~(\Macro{maketitle}) als auch die Umschlagseite~(\Macro{makecover}) 
und zum anderen alle Teileseiten~(\Macro{part}, \Macro{addpart}) und 
Kapitelseiten~(\Macro{chapter}, \Macro{addchap}).

Soll ein bestimmtes Element des Layouts abweichend erscheinen, so kann eine der 
folgenden Optionen genutzt werden, um dieses individuell anzupassen. Die 
gültigen Wertzuweisungen für die einzelnen Elemente entsprechend dabei den 
möglichen Werten für die Option \Option{cd}.

\begin{Declaration}[%
  v2.03!neue Farbvarianten
    (\protect\PValue{bicolor} und \protect\PValue{fullcolor}),%
  v2.04!farbiger Querbalken möglich (\protect\PValue{barcolor})%
]{\Option{cdtitle}[\PSet]}
\printdeclarationlist%
\index{Titel}\index{Layout!Titel}%
%
Mit der Option \Option{cdtitle} kann die allgemeine Einstellung für den Titel 
überschrieben werden. Es kann zwischen dem normalen (\Option{cdtitle}[false]) 
und dem im \CD umgeschaltet werden. Die neue Titelseite unterstützt alle durch 
\KOMAScript{} definierten Befehle für den Titel.%
\footnote{\raggedright%
  \Macro{extratitle}\Parameter{Schmutztitel},\Macro{titlehead}\Parameter{Kopf},
  \Macro{subject}\Parameter{Typisierung},\Macro{title}\Parameter{Titel},
  \Macro{subtitle}\Parameter{Untertitel},\Macro{author}\Parameter{Autor},
  \Macro{date}\Parameter{Datum},\Macro{publishers}\Parameter{Verlag},
  \Macro{and} und \Macro{thanks}\Parameter{Fußnote} sowie
  \Macro{uppertitleback}\Parameter{Titelrückseitenkopf},
  \Macro{lowertitleback}\Parameter{Titelrückseitenfuß}
  und \Macro{dedication}\Parameter{Widmung}
}
Zusätzlich werden viele neue Felder definiert, welche vor allem für eine 
wissenschaftliche Arbeit von Relevanz sind. Genaueres dazu 
ist in \autoref{sec:title} nachzulesen. Unabhängig von der gewählten Variante 
der Titelseite wird diese immer mit \Macro{maketitle} erzeugt.
\end{Declaration}

\begin{Declaration}[%
  v2.02,%
  v2.03!neue Farbvarianten
    (\protect\PValue{bicolor} und \protect\PValue{fullcolor}),%
  v2.04!farbiger Querbalken möglich (\protect\PValue{barcolor})%
]{\Option{cdcover}[\PSet]}
\printdeclarationlist%
\index{Umschlagseite|!}\index{Titel!Umschlagseite}\index{Layout!Umschlagseite}%
%
Die \TUDScript-Klassen führen zusätzlich den Befehl \Macro{makecover} ein, mit 
dem sich neben dem Titel eine separate Umschlagseite erzeugen lässt. Diese ist 
in ihrer Gestalt der Titelseite sehr ähnlich, wird normalerweise jedoch in 
einem anderen Satzspiegel als dem des Buchblocks gesetzt. Mit der Option 
\Option{cdcover} kann~-- unabhängig von \Option{cd}~-- das Erscheinungsbild 
der Umschlagseite geändert werden. Wird \Option{cdcover}[false] gewählt, 
entspricht die Umschlagseite dem originalen \KOMAScript-Titel. Die Verwendung 
des Befehls \Macro{makecover} sowie die dazugehörigen Parameter werden 
detailliert in \autoref{sec:title} erläutert.
\end{Declaration}

\begin{Declaration}[%
  v2.03!neue Farbvarianten
    (\protect\PValue{bicolor} und \protect\PValue{fullcolor}),%
  v2.04!farbiger Querbalken möglich (\protect\PValue{barcolor})%
]{\Option{cdpart}[\PSet]}
\printdeclarationlist%
\index{Teileseiten}\index{Layout!Teileseiten}%
%
Für die Teileseiten kann der Wert des Schlüssels \Option{cd} separat 
überschrieben und somit deren Layout respektive Erscheinungsbild beeinflusst 
werden, welches bei der Benutzung der Befehle \Macro{part} beziehungsweise 
\Macro{addpart} und deren Sternversionen genutzt wird.
\end{Declaration}

\begin{Declaration}[%
  v2.03!neue Farbvarianten
    (\protect\PValue{bicolor} und \protect\PValue{fullcolor}),%
  v2.04!farbiger Querbalken möglich (\protect\PValue{barcolor})%
]{\Option{cdchapter}[\PSet]}
\printdeclarationlist%
\index{Kapitelseiten}\index{Layout!Kapitelseiten}%
%
Für Kapitelseiten kann der Schlüsselwert \Option{cd} ebenfalls angepasst und 
damit das Layout respektive Erscheinungsbild geändert werden, das bei der 
Verwendung von \Macro{chapter} beziehungsweise \Macro{addchap} und den 
dazugehörigen Sternversionen genutzt wird.
\end{Declaration}

\begin{Declaration}[v2.05]{\Option{cdsection}[\PSet]}
\printdeclarationlist%
\index{Überschriften}
%
Für Überschriften der Gliederungsebenen \Macro{section}, \Macro{subsection}, 
\Macro{subsubsection} sowie \Macro{paragraph} und \Macro{subparagraph} werden 
in der primären Hausfarbe \Color{HKS41} ausgegeben, falls über die Option 
\Option{cd} eine farbige Ausprägung des Layouts eingestellt wurde. Mit der 
Angabe von \Option{cdsection}[true] erschienen die Überschriften der genannten 
Gliederungsebenen ohne Farbeinsatz.
\end{Declaration}

\begin{Example}
Soll die Titelseite in Farbe, der Rest des Dokumentes allerdings in schwarzer 
Schrift gesetzt werden, so kann dies folgendermaßen erreicht werden:
\begin{Code}[escapechar=§]
\documentclass[cd=true,cdtitle=color]{§\PName{Dokumentklasse}§}
\end{Code}
\end{Example}


\subsubsection{Position von Überschriften}
\begin{Declaration}[v2.02]{\Length{pageheadingsvskip}}
\begin{Declaration}[v2.02]{\Length{headingsvskip}}
\printdeclarationlist%
\index{Überschriften!Position}\index{Titel!Position}%
\index{Teileüberschriften}\index{Kapitelüberschriften}%
\index{Kapitelseiten}\index{Layout!Kapitelseiten}\index{Layout!Überschriften}%
%
Diese beiden Längen haben Auswirkung auf die vertikale Position bestimmter
Überschriften. Mit \Length{pageheadingsvskip} lassen sich sowohl der Titel auf
der Titelseite (\Option{titlepage}[true]) als auch die Überschriften von Teilen 
und Kapiteln, welche als einzelne Kapitelseite (\Option{chapterpage}[true]) 
gesetzt werden, verschieben. Demgegenüber erlaubt es \Length{headingsvskip}, 
sowohl den Titel innerhalb eines Titelkopfes (\Option{titlepage}[false]) als 
auch die Überschrift eines Kapitels bei deaktivierter Kapitelseite 
(\Option{chapterpage}[false]) in ihrer vertikalen Position anzupassen.

Die zuvor genannten Überschriften werden normalerweise im Layout relativ tief 
im Textbereich gesetzt. Mit negativen Werten werden die Überschriften nach oben 
verschoben, wobei darauf geachtet werden sollte, dass diese sich danach noch 
innerhalb des Satzspiegels befinden. Positive Werte setzen die Überschriften 
dementsprechend tiefer.
\end{Declaration}
\end{Declaration}



\subsection{Seiten im Stil des \CDs}
\begin{Declaration}[v2.02]{\PageStyle{tudheadings}}
\begin{Declaration}[v2.02]{\PageStyle{plain.tudheadings}}
\begin{Declaration}[v2.02]{\PageStyle{empty.tudheadings}}
\printdeclarationlist%
\label{sec:tudheadings}
%
\ChangedAt*{%
  v2.03!Seitenstile um zweifarbigen Kopf und farbigen Fuß erweitert%
}%
Ein zentrales Element des \CDs der \TnUD ist der prägnante Seitenkopf mit der 
Angabe von Fakultät~(\Macro{faculty}), Einrichtung~(\Macro{department}), 
Institut~(\Macro{institute}) und Lehrstuhl~(\Macro{chair}) im dazugehörigen 
Querbalken. Durch die Nutzung des Paketes \Package{scrlayer-scrpage} lassen 
sich entweder einzelne Seiten oder auch ganze Dokumente sehr einfach in diesem 
Stil setzen. Hierzu muss lediglich mit \Macro{pagestyle}\Parameter{Seitenstil} 
einer der Seitenstile geladen werden. 

Allen Seitenstilen gemein ist der typische Kopf mit dem charakteristischen 
Querbalken, dessen Gestalt für \emph{alle} Seitenstile gleichermaßen über die 
Option \Option{cdhead} angepasst werden kann. Mit dem Befehl \Macro{headlogo} 
lässt sich ein zusätzliches Zweitlogo im Kopfbereich ausgegeben.
\Attention{%
  Für die speziellen Layout-Elemente Titel und Umschlagseite sowie Teile- und 
  Kapitelseiten wird die Einstellung von \Option{cdhead} durch die Nutzung der 
  Option~\Option{cd} überschrieben.
}

Die Ausprägung des Fußes unterscheidet sich bei den einzelnen Seitenstilen. 
Dieser ist beim Seitenstil \PageStyle{empty.tudheadings} immer leer. Die beiden 
Stile~-- oder vielmehr das Seitenstil-Paar~-- \PageStyle{tudheadings} und 
\PageStyle{plain.tudheadings} übernehmen die Einstellungen für die Fußzeile aus 
der Anwenderschnittstelle von \Package{scrlayer-scrpage}.%
\footnote{%
  Es können die Befehle \Macro{lefoot}, \Macro{cefoot} und \Macro{refoot} sowie 
  \Macro{lofoot}, \Macro{cofoot} und \Macro{rofoot} respektive \Macro{ofoot}, 
  \Macro{cfoot} und \Macro{ifoot} genutzt werden.
}
Wie diese zu verwenden ist, kann der \KOMAScript"=Anleitung entnommen werden. 
Alternativ zu einer eigenen Definition der Fußzeile lässt sich außerdem die 
Option \Option{cdfoot} verwenden. Zusätzlich kann über \Macro{footcontent} ein 
freier Inhalt für den Fußbereich definiert werden, mit \Macro{footlogo} ist die 
Ausgabe von einem oder mehreren Logos in diesem möglich. Die verwendete Schrift 
im Fußbereich wird durch das Schriftelement~\Font{tudheadings} festgelegt.

Sobald einer der definierten Stile mit \Macro{pagestyle}\Parameter{Seitentil} 
aktiviert wurde, sind die beiden Seitenstile \PageStyle{tudheadings} sowie 
\PageStyle{plain.tudheadings} zusätzlich unter den Namen \PageStyle{headings} 
respektive \PageStyle{plain} verwendbar. Dies hat den Vorteil, dass bei 
Optionen oder Befehlen, welche automatisch zwischen den beiden Seitenstilen 
\PageStyle{headings} und \PageStyle{plain} umschalten, durch die einmalige 
Auswahl von einem der \PageStyle{tudheadings}-Stilen nun zwischen diesen  
umgeschaltet wird.

Der Seitenstil \PageStyle{empty} erzeugt allerdings weiterhin eine komplett 
leere Seite. Soll eine Seite mit der prägnanten Kopfzeile der \TnUD und leerem 
Seitenfuß erschienen, so muss \Macro{pagestyle}\PParameter{empty.tudheadings} 
manuell aufgerufen werden. Um auf das normale Verhalten von \KOMAScript{} 
zurückzuschalten, muss einer der beiden Stile \PageStyle{scrheadings} 
beziehungsweise \PageStyle{plain.scrheadings} aktiviert werden.

\Attention{%
  Die beschriebenen Seitenstile werden erst \emph{nach} dem Laden des Paketes 
  \Package{scrlayer-scrpage} definiert. Wird dieses nicht durch den Anwender 
  geladen, sollte \Macro{pagestyle}\Parameter{Seitenstil} erst nach 
  \Macro*{begin}\PParameter{document} verwendet werden.
}%
\end{Declaration}
\end{Declaration}
\end{Declaration}

\begin{Declaration}[v2.04]{\Font{tudheadings}}
\printdeclarationlist%
\index{Schriftelemente}
%
Im Fußbereich der Seiten im \PageStyle{tudheadings}-Seitenstil wird das  
Schriftelement~\Font{tudheadings} verwendet. Dieses wirkt sich auf die 
Seitenzahlen, den Kolumnentitel und die mit \Macro{footcontent} angegebenen 
Inhalte aus. Hierüber wird die Wahl der richtigen Schriftfarbe in Abhängigkeit 
vom Seitenhintergrund und den Einstellungen für die Optionen \Option{cdhead} 
sowie \Option{cdfoot} realisiert. Wie \Font{tudheadings} angepasst werden kann, 
ist in \autoref{sec:fonts:elements} zu finden.
\end{Declaration}

\begin{Declaration}{\Macro{faculty}\Parameter{Fakultät}}
\begin{Declaration}{\Macro{department}\Parameter{Einrichtung}}
\begin{Declaration}{\Macro{institute}\Parameter{Institut}}
\begin{Declaration}{\Macro{chair}\Parameter{Lehrstuhl}}
\begin{Declaration}{\Macro{extraheadline}\Parameter{Textzeile}}
\printdeclarationlist%
\index{Kopfzeile}\index{Layout!Kopfzeile}%
\index{Querbalken}\index{Layout!Querbalken}%
%
Für den Seitenstil des \CDs der \TnUD typisch ist die Kopfzeile mit dem 
charakteristischen Querbalken. In dieser wird~-- falls angegeben~-- in fetter 
Schrift die Fakultät ausgegeben, danach folgen durch Kommas getrennt die 
Einrichtung, das Institut und der Lehrstuhl beziehungsweise die Professur. 
Sollte der Platz in der ersten Zeile nicht ausreichen, erfolgt ein 
automatischer Zeilenumbruch.

In besonderen Ausnahmefällen erlaubt das \CD die Angabe einer zusätzlichen
zweiten beziehungsweise dritten Zeile unterhalb der Angaben des Bereichs an der 
\TnUD, welche weitere, frei wählbare Angaben enthält. Diese kann mit dem Befehl 
\Macro{extraheadline}\Parameter{Textzeile} definiert werden.
\end{Declaration}
\end{Declaration}
\end{Declaration}
\end{Declaration}
\end{Declaration}

\begin{Declaration}[%
  v2.03,%
  v2.04!farbiger Querbalken möglich (\PValue{barcolor})%
]{\Option{cdhead}[\PSet]}[true]
\printdeclarationlist%
\index{Kopfzeile}\index{Layout!Kopfzeile}%
\index{Querbalken}\index{Layout!Querbalken}%
%
Mit dieser Option lassen sich für die \PageStyle{tudheadings}"=Seitenstile 
sowohl die Gestalt des Logos sowie des Querbalkens als auch die darin 
verwendete Schrift beeinflussen. Die folgenden Werte können für eine Anpassung 
der Schriftart im Balken verwendet werden:
%
\begin{values}
\itemfalse
  Sollte mit \Option{cdfont}[false] die Verwendung der Hausschrift im Stil des 
  \CDs der \TnUD deaktiviert worden sein, wird die Kopfzeile im Querbalken in
  den Serifenlosen der genutzten Schrift gesetzt. Sind die Hausschriften 
  aktiviert, hat diese Einstellung keinen Einfluss.
\itemtrue*[light/lightfont/noheavyfont]
  Im Querbalken wird für \Macro{faculty} \textubn{Univers~65~Bold} verwendet, 
  die Felder \Macro{department}, \Macro{institute}, \Macro{chair} und 
  \Macro{extraheadline} kommt \textuln{Univers~45~Light} zum Einsatz.
\item[heavy/heavyfont]
  Der Inhalt von \Macro{faculty} wird weiterhin in \textubn{Univers~65~Bold} 
  gesetzt, für die restlichen Felder wird \texturn{Univers~55~Regular} genutzt.
\end{values}
%
Bei der Ausprägung des Kopfes und des Querbalkens gibt es mehrere Varianten. 
Einerseits kann der Querbalken mit zwei Außenlinien dargestellt werden:
%
\begin{values}
\item[nocolor/monochrome]
  Der Kopf und die Linien des Querbalkens erscheinen in schwarzer Farbe.
\item[lightcolor/pale]
  Sowohl Kopf als auch Querbalken werden in der primären Hausfarbe gesetzt.
\end{values}
%
Andererseits ist auch eine Darstellung mit mehr Farbeinsatz möglich, bei 
welcher der Querbalken und gegebenenfalls der ganze Seitenkopf farblich 
abgesetzt wird.
%
\begin{values}
\item[barcolor]
  \ChangedAt{v2.04} Im Gegensatz zur vorherigen Einstellung wird der 
  Querbalken mit farbigem Hintergrund verwendet.
\item[bicolor/bichrome]
  \ChangedAt{v2.03} Die Kopfzeile wird farblich abgesetzt, wobei der 
  Hintergrund des Logos und der Querbalken unterschiedlich ausfallen. Die 
  Außenlinien der Querbalkens entfallen.
\end{values}
%
Für den Fall, dass der Querbalken lediglich mit zwei Außenlinien dargestellt 
wird, kann zusätzlich dessen Laufweite festgelegt werden:
%
\begin{values}
\item[textwidth/slim]
  Der Querbalken im Kopf erstreckt sich nur über den Textbereich. Diese 
  Einstellung ist insbesondere sinnvoll, wenn ein randloser Ausdruck technisch 
  nicht möglich ist.   
\item[paperwidth/wide]
  Die horizontale Ausdehnung des Querbalkens erstreckt sich über die komplette 
  Seitenbreite bis an den Blattrand. Dieses Verhalten ist standardmäßig im 
  farbigen Layout aktiviert.
\end{values}
\end{Declaration}

\begin{Declaration}[%
  v2.03!farbiger Hintergrund der Fußzeile möglich%
]{\Option{cdfoot}[\PSet]}[false]%
\printdeclarationlist%
\index{Kolumnentitel}\index{Layout!Kolumnentitel}
\index{Satzspiegel!doppelseitig}%

Die \TUDScript-Klassen sind~-- insbesondere aufgrund der Möglichkeit zur 
Verwendung des Paketes \Package{scrlayer-scrpage}~-- bei der Gestaltung der 
Kopf"~ und Fußzeilen sehr flexibel und individuell anpassbar. Die Ausprägung 
und der Inhalt dieser ist nicht explizit durch das \CD vorgegeben und können 
durch den Anwender beliebig gewählt und geändert werden. Wird die Klassenoption 
\Option{automark} angegeben, werden für das automatische Setzen der Marken die 
Titel der Gliederungsebenen verwendet. Genaueres hierzu sowie der Möglichkeit, 
die Kolumnentitel manuell festzulegen, ist dem Handbuch von \KOMAScript{} zu 
entnehmen.

Eine Möglichkeit für deren Gestaltung zeigt das Handbuch für das \CD der \TnUD. 
Dieses wird ohne Kopf"~ und mit einer einfachen Fußzeile gesetzt, welche den 
aktuellen Kolumnentitel sowie die Paginierung enthält. Mit \Option{cdfoot} kann 
diese Ausprägung aktiviert werden, was auch für dieses Anwenderhandbuch 
geschehen ist.
%
\begin{values}
\itemfalse
  Die Kopf"~ und Fußzeilen zeigen Standardverhalten, zur manuellen Änderung 
  dieser sollte unbedingt das \KOMAScript"=Paket \Package{scrlayer-scrpage} 
  verwendet werden.
\itemtrue*
  Die Fußzeilen des Dokumentes werden wie im Handbuch des \CDs der \TnUD 
  mit Kolumnentitel und Seitenzahl gesetzt. Im doppelseitigen Satz 
  (\Option{twoside}[true]) wird die Paginierung außen platziert.
\end{values}
%
\ChangedAt{v2.03} 
Sollte einer der \PageStyle{tudheadings}-Seitenstil aktiviert sein und es wird 
auf der erzeugten Seite ein farbiges Layout~--  beispielsweise der zweifarbige 
Kopf (\Option{cdhead}[bicolor]) oder ein farbiger Seitenhintergrund~-- genutzt, 
so kann auch die Fußzeile einen farbigen Hintergrund erhalten.
%
\begin{values}
\item[nocolor/monochrome]
  Der Fuß wird immer ohne farbigen Hintergrund gesetzt.
\item[color/bicolor/bichrome]
  Die Fußzeile wird farblich abgesetzt, falls entweder der zweifarbige Kopf
  (\Option{cdhead}[bicolor]) oder zumindest eine Seite mit einem farbigen 
  Hintergrund in der Hausfarbe (Titel oder Kapitelseite) verwendet wurde.
\end{values}
%
Wird der Option \Option{cdfoot} eine Längenmaß übergeben, entspricht dies der 
Verwendung von \Option{extrabottommargin}.
\end{Declaration}

\begin{Declaration}{\Macro{headlogo}\LParameter\Parameter{Dateiname}}
\printdeclarationlist%
\index{Kopfzeile}\index{Layout!Kopfzeile}%
\index{Zweitlogo|?}\index{Layout!Zweitlogo}\index{\DDC-Logo}%
%
Neben dem Logo der \TnUD darf zusätzlich ein Zweitlogo im Kopf verwendet 
werden. Dieses lässt sich mit diesem Befehl einbinden. Normalerweise wird es 
auf die Höhe der Erstlogos skaliert. Über das optionale Argument können weitere 
Formatierungsbefehle an den verwendeten Befehl \Macro{includegraphics} 
durchgereicht werden, um beispielsweise die Größe des Zweitlogos anzupassen.
Welche Parameter angepasst werden können, ist der Dokumentation des
\Package{graphicx}-Paketes zu entnehmen.

Sollte die Option \Option{ddc} aktiviert sein, wird das \DDC-Logo nicht im Kopf 
sondern automatisch im Fuß gesetzt. Die Option \Option{ddchead} setzt dieses 
auf jeden Fall im Kopf und überschreibt damit das mit \Macro{headlogo} 
angegebene Zweitlogo.
\end{Declaration}

\begin{Declaration}[v2.03]{%
  \Macro{footlogo}\LParameter\Parameter{Dateinamenliste}%
}
\begin{Declaration}[v2.03]{\Macro{footlogosep}}%
\begin{Declaration}[v2.03]{\Length{footlogoheight}}%
\printdeclarationlist%
\index{Drittlogo|?}\index{Layout!Drittlogo}\index{\DDC-Logo}%
\index{Fußzeile}\index{Layout!Fußzeile}%

Laut den Richtlinien des \CDs dürfen im Fußbereich weitere Logos erscheinen, 
beispielsweise von kooperierenden Unternehmen oder Sponsoren. Die Dateinamen 
der gewünschten Logos können als kommaseparierte Liste im obligatorischen 
Argument des Befehls \Macro{footlogo} angegeben werden. Sollte tatsächlich 
nicht nur ein Dateiname sondern eine Liste übergeben worden sein, so wird bei 
der Ausgabe der Logos zwischen diesen jeweils der in \Macro{footlogosep} 
gespeicherte Separator~-- standardmäßig \Macro*{hfill}~-- gesetzt. Dieser kann 
mit \Macro*{renewcommand*}\Macro{footlogosep}\PParameter{\dots} beliebig durch 
den Anwender angepasst werden. Der Separator wird auch gesetzt, wenn in der 
\Parameter{Dateinamenliste} lediglich ein Komma verwendet wurde. So kann man 
beispielsweise ein Logo mit \Macro{footlogo}\PParameter{,\PName{Dateiname},} 
zentriert im Fuß setzen.

Das optionale Argument von \Macro{footlogo} wird an \Macro{includegraphics} 
weitergereicht. Dies geschieht für alle angegeben Dateien aus der Liste 
gleichermaßen. Sollen für einzelne Logos individuelle Einstellungen vorgenommen 
werden, so sind die entsprechenden Parameter im obligatorischen Argument nach 
dem jeweiligen Dateinamen mit einem Doppelpunkt~\enquote{\PValue{:}} als 
Separator (\Macro{footlogo}\PParameter{\PName{Dateiname}:\PName{Parameter}}) zu 
übergeben, wobei diese \emph{nach} den allgemeinen Einstellungen für alle Logos 
angewendet werden. Die möglichen Parameter und Werte für die optionalen 
Argumente sind der Dokumentation des \Package{graphicx}-Paketes zu entnehmen.

Ohne die Angabe eines optionalen Argumentes für die Größe werden alle Logos im 
Fuß auf die auf die Höhe des Logos der \TnUD skaliert. Der Anwender kann dies 
ändern, indem der Wert der Länge \Length{footlogoheight} mit \Macro*{setlength} 
auf einen von \PValue{0pt} verschiedenen gesetzt wird. Sollte die Höhe des 
Fußbereiches nicht ausreichen, um alle Logos in der gewünschten Größe 
darstellen zu können, kann diese über die Option \Option{extrabottommargin} 
beziehungsweise \Option{cdfoot} angepasst werden.
\end{Declaration}
\end{Declaration}
\end{Declaration}

\begin{Declaration}[v2.04]{\Macro{footcontent}%
  \OParameter{Anweisungen}\Parameter{Inhalt}\OParameter{Inhalt}%
}
\begin{Declaration}[v2.04]{\Macro*{footcontent*}
    \OParameter{Anweisungen}\Parameter{Inhalt}\OParameter{Inhalt}%
}
\printdeclarationlist%
\index{Fußzeile}\index{Layout!Fußzeile}%
%
Mit diesem Befehl kann beliebiger Inhalt entweder einspaltig oder zweispaltig 
im Fußbereich der \PageStyle{tudheadings}"=Seitenstile gesetzt werden. In der 
Form \Macro{footcontent}\Parameter{Inhalt} wird der Inhalt über die komplette 
Textbreite im Fuß ausgegeben. Wird der Befehl jedoch in der zweiten Variante 
\Macro{footcontent}\Parameter{linker Inhalt}\OParameter{rechter Inhalt} mit 
einem optionalen \emph{nach} dem obligatorischen Argument verwendet, so 
erscheint der Fußbereich zweispaltig, wobei der Inhalt aus dem ersten, 
obligatorischen Argument in der linken und der Inhalt aus dem zweiten, 
optionalen Argument entsprechend in der rechten Fußspalte gesetzt wird.

Im Normalfall wird das Schriftelement \Font{tudheadings} für die Schrift im 
Fußbereich verwendet. Mit dem ersten optionalen Argument können weitere 
Schrifteinstellungen respektive Anweisungen vor der eigentlichen Ausgabe des 
Inhaltes erfolgen. Soll die Definition des Inhalts für den Fußbereich gänzlich 
ohne die automatische Schriftformatierungen erfolgen, so kann die Sternversion 
\Macro{footcontent*} genutzt und gegebenenfalls die Schriftformatierung über 
das optionale Argument vorgenommen werden.
\end{Declaration}
\end{Declaration}

\begin{Declaration}[%
  v2.02!\protect\DDC-Logo automatisch in Kopf oder Fuß%
]{\Option{ddc}[\PSet]}[false]
\begin{Declaration}[v2.02]{\Option{ddchead}[\PSet]}[false]
\begin{Declaration}[%
  v2.02!neue Werte für die Farbwahl des Logos von \protect\DDC%
]{\Option{ddcfoot}[\PSet]}[false]
\printdeclarationlist%
\index{Zweitlogo}\index{Layout!Zweitlogo}\index{\DDC-Logo}%
%
Diese Optionen fügen das Logo von \DDC entweder im Kopf oder im Fuß der Seiten
mit dem Stil \PageStyle{tudheadings} ein. Mit \Option{ddc} wird dieses 
automatisch entweder im Kopf oder~-- falls ein Zweitlogo mit \Macro{headlogo} 
angegeben wurde~-- im Fuß gesetzt. Die anderen beiden Optionen setzen das Logo 
zwingend entweder im Kopf (\Option{ddchead}) oder im Fuß (\Option{ddcfoot}), 
wobei erstgenannte ein optionales Zweitlogo dabei unterdrückt. Die Verwendung 
einer der drei Optionen führt zur Deaktivierung der anderen beiden, sie 
schließen sich folglich gegenseitig aus. Die möglichen Werte für diese Optionen 
sind:
%
\begin{values}
\itemfalse
  Bei den \PageStyle{tudheadings}-Seitenstile erscheint kein Logo von \DDC.
\itemtrue*
  Das Logo von \DDC wird im Kopf beziehungsweise im Fuß verwendet. Die Wahl der 
  Farbe des Logos geschieht passend zur farblichen Ausprägung der Seite selbst.
\end{values}
%
Soll die Farbe des \DDC-Logos manuell erfolgen, können folgende Werte verwendet 
werden:
%
\begin{values}
\item[color]
  Im Kopf oder Fuß wird die achtfarbige 4C"~Variante des \DDC-Logos genutzt.
\item[colorblack]
  Es wird das achtfarbige Logo mit schwarzem \DDC-Schriftzug anstelle des 
  grauen verwendet. Für den Fuß wird der grüne Claim ebenfalls durch einen 
  schwarzen ersetzt. Dies ist insbesondere für kleine Darstellungen des Logos 
  im Fuß sinnvoll.
\item[gray/grey]
  Dies Ausgabe des \DDC-Logos erfolgt in Graustufen.
\item[black]
  Verwendung des Logos in Graustufen mit schwarzem Schriftzug.
\item[blue]
  Der Schriftzug und das Logo werden in der primären Hausfarbe \Color{HKS41} 
  und den entsprechenden Abstufungen gesetzt
\item[white]
  Das \DDC-Logo sowie der dazugehörige Schriftzug sind vollständig weiß.
\end{values}
%
Die Größe des \DDC-Logos ist vorgegeben. Es wird sowohl im Kopf als auch im Fuß 
in der gleichen Höhe gesetzt, wie das Logo der \TnUD. Wird es im Fuß gesetzt, 
lässt sich die Größe allerdings über die Länge \Length{footlogoheight} ändern. 
Sollte nach einer Vergrößerung der Darstellung die Höhe des Fußbereiches nicht 
ausreichen, so kann diese über die Option \Option{cdfoot} beziehungsweise 
\Option{extrabottommargin} angepasst werden.
\end{Declaration}
\end{Declaration}
\end{Declaration}

\ToDo[imp,nxt]{%
  Spezialseite zur freien Gestaltung mit Hintergrundebene für Bilder und Texte  
  (CD-Handbuch S. 80 ff.); Seitenstil: \PageStyle{special.tudheadings}
}[v2.06]

\begin{Declaration}[%
  v2.02!\Key{\Environment{tudpage}}{head} \emph{entfällt},
  v2.02!\Key{\Environment{tudpage}}{foot} \emph{entfällt},
  v2.03!\Key{\Environment{tudpage}}{color} \emph{entfällt}
]{\Environment{tudpage}[\OLParameter{Sprache}]}
\begin{Declaration}{\Key{\Environment{tudpage}}{language}[\PName{Sprache}]}
\begin{Declaration}{\Key{\Environment{tudpage}}{columns}[\PName{Anzahl}]}
\begin{Declaration}[v2.02]{\Key{\Environment{tudpage}}{pagestyle}[\PSet]}
\begin{Declaration}{\Key{\Environment{tudpage}}{cdfont}[\PSet]}{%
  \see*{\Option{cdfont}'ppage'}%
}
\begin{Declaration}[v2.03]{\Key{\Environment{tudpage}}{cdhead}[\PSet]}{%
  \see*{\Option{cdhead}'ppage'}%
}
\begin{Declaration}[v2.03]{\Key{\Environment{tudpage}}{cdfoot}[\PSet]}{%
  \see*{\Option{cdfoot}'ppage'}%
}
\begin{Declaration}{\Key{\Environment{tudpage}}{headlogo}[\PName{Dateiname}]}{%
  \see*{\Macro{headlogo}'ppage'}%
}
\begin{Declaration}[v2.03]{%
  \Key{\Environment{tudpage}}{footlogo}[\PName{Dateinamenliste}]
}{\see*{\Macro{footlogo}'ppage'}}
\begin{Declaration}[v2.02]{\Key{\Environment{tudpage}}{ddc}[\PSet]}{%
  \see*{\Option{ddc}'ppage'}%
}
\begin{Declaration}[v2.02]{\Key{\Environment{tudpage}}{ddchead}[\PSet]}{%
  \see*{\Option{ddchead}'ppage'}%
}
\begin{Declaration}[v2.02]{\Key{\Environment{tudpage}}{ddcfoot}[\PSet]}{%
  \see*{\Option{ddcfoot}'ppage'}%
}
\printdeclarationlist%
\index{Layout}\index{Layout!Seitenstil}%
\index{Kopfzeile}\index{Layout!Kopfzeile}%
\index{Fußzeile}\index{Layout!Fußzeile}%
%
Die \Environment{tudpage}"=Umgebung hat ihren Ursprung in einer früheren 
Version, als die \PageStyle{tudheadings}"=Seitenstile noch nicht verfügbar 
waren, welche mittlerweile anstelle dieser Umgebung verwendet werden können.
Für die \Environment{tudpage}"=Umgebung lassen sich verschiedene Parameter als 
optionales Argument angegeben. Wird das Paket \Package{babel} genutzt, kann die 
verwendete Sprache mit \Key{\Environment{tudpage}}{language}[\PName{Sprache}] 
geändert werden, was zur Anpassung der sprachspezifischen Trennungsmuster und 
Bezeichner führt. Wurde das Paket \Package{multicol} geladen, wird mit dem 
Parameter \Key{\Environment{tudpage}}{columns}[\PName{Anzahl}] der Inhalt der 
Umgebung mehrspaltig gesetzt. Mit \Key{\Environment{tudpage}}{pagestyle} kann 
der Seitenstil angepasst werden, wobei \PValue{headings}, \PValue{plain} und 
\PValue{empty} gültige Werte sind. 

Die weiteren Parameter entsprechen in ihrem Verhalten prinzipiell den 
gleichnamigen Klassenoptionen respektive Befehlen, wirken sich jedoch nur lokal 
innerhalb der \Environment{tudpage}"=Umgebung aus. Das Verhalten sowie die 
jeweils gültigen Wertzuweisungen können in den entsprechenden Abschnitten des 
Handbuchs nachgelesen werden.
\end{Declaration}
\end{Declaration}
\end{Declaration}
\end{Declaration}
\end{Declaration}
\end{Declaration}
\end{Declaration}
\end{Declaration}
\end{Declaration}
\end{Declaration}
\end{Declaration}
\end{Declaration}



\subsection{Der Titel und die Umschlagseite}
\label{sec:title}%\label{sec:cover}%
\index{Titel|!(}\index{Umschlagseite|!}\index{Layout!Umschlagseite}%
%
\ChangedAt*{%
  v2.03!Bugfix für Umschlagseite und Titel beim Satzspiegel%
}
Für den Titel werden alle Felder unterstützt, die bereits durch \KOMAScript{} 
bereitgestellt werden. Darüber hinaus werden für die \TUDScript-Klassen weitere 
Felder definiert, die Auswirkungen auf die Gestalt des Titels haben. Diese 
werden nachfolgend in diesem \autorefname erläutert. Der Titel~-- bestehend aus 
möglichem Schmutztitel, der eigentlichen Titelseite und der nachgelagerten 
Elementen~-- kann mit dem Befehl \Macro{maketitle} ausgegeben werden. Außerdem 
kann im zweispaltigen Satz der \Macro{maketitleonecolumn} verwendet werden, 
welcher einen einspaltigen Einfügung nach dem Titel selbst ermöglicht.

Zusätzlich zum Titel lässt sich mit \Macro{makecover} eine Umschlagseite 
erzeugen. Diese kann insbesondere für gebundene Arbeiten verwendet werden. Es 
wird~-- im Vergleich zum Titel~-- lediglich einer reduzierte Anzahl an Feldern 
auf dieser ausgegeben.

\ChangedAt{v2.02}
Für alle Felder des Titels und der Umschlagseite lässt sich die verwendete 
Schrift anpassen. Dabei werden sowohl die bereits durch \KOMAScript{} 
bereitgestellten Schriftelemente \Font{titlehead}, \Font{subject}, 
\Font{title}, \Font{subtitle}, \Font{author}, \Font{date}, \Font{publishers} 
und \Font{dedication} als auch die neuen \Font{titlepage} und \Font{thesis} 
unterstützt.
%
\begin{Example}
In diesem Dokument wurde der Untertitel derart geändert, dass dieser nicht 
standardmäßig in \DIN sondern in \textubn{Univers~65~Bold} ausgegeben wird.
\begin{Code}[escapechar=§]
\addtokomafont{subtitle}{\univbn}
\end{Code}
\end{Example}

\begin{Declaration}[%
  v2.01!Bugfix für Schriftstärke auf Titelseite,%
  v2.02!Unterstützung der Schriftelemente \Font*{titlehead}{,} 
    \Font*{subject}{,} \Font*{title}{,} \Font*{subtitle}{,} \Font*{author}{,} 
    \Font*{date}{,} \Font*{publishers}{,} \Font*{dedication}{,} 
    \Font*{titlepage} und \Font*{thesis}%
]{\Macro{maketitle}\OLParameter{Seitenzahl}}
\begin{Declaration}[v2.02]{%
  \Key{\Macro{maketitle}}{pagenumber}[\PName{Seitenzahl}]%
}
\begin{Declaration}[v2.02]{\Key{\Macro{maketitle}}{cdfont}[\PSet]}{%
  \see*{\Option{cdfont}'ppage'}%
}
\begin{Declaration}[v2.03]{\Key{\Macro{maketitle}}{cdhead}[\PSet]}{%
  \see*{\Option{cdhead}'ppage'}%
}
\begin{Declaration}[v2.03]{\Key{\Macro{maketitle}}{cdfoot}[\PSet]}{%
  \see*{\Option{cdfoot}'ppage'}%
}
\begin{Declaration}[v2.03]{%
  \Key{\Macro{maketitle}}{headlogo}[\PName{Dateiname}]%
}{\see*{\Macro{headlogo}'ppage'}}
\begin{Declaration}[v2.03]{%
  \Key{\Macro{maketitle}}{footlogo}[\PName{Dateinamenliste}]
}{\see*{\Macro{footlogo}'ppage'}}
\begin{Declaration}[v2.03]{\Key{\Macro{maketitle}}{ddc}[\PSet]}{%
  \see*{\Option{ddc}'ppage'}%
}
\begin{Declaration}[v2.03]{\Key{\Macro{maketitle}}{ddchead}[\PSet]}{%
  \see*{\Option{ddchead}'ppage'}%
}
\begin{Declaration}[v2.03]{\Key{\Macro{maketitle}}{ddcfoot}[\PSet]}{%
  \see*{\Option{ddcfoot}'ppage'}%
}
\printdeclarationlist%
\index{Layout!Titel}%
\index{Satzspiegel!doppelseitig}%
%
Der Befehl \Macro{maketitle} setzt für \Option{cdtitle}[false] den normalen 
\KOMAScript"=Titel{}, ansonsten wird die Titelseite im \CD der \TnUD erzeugt. 
Die letztere Variante ist im Vergleich zum Standardtitel um eine Vielzahl von 
Feldern erweitert worden und erlaubt insbesondere die Angabe von Daten für das 
Deckblatt einer akademischen Abschlussarbeit. Die einzelnen Felder werden 
später in diesem \autorefname erläutert. Wird das Dokument doppelseitig und mit 
rechts öffnenden Kapiteln gesetzt,%
\footnote{%
  \Option{twoside} und \Option{open}[right], Standard für \Class{tudscrbook}
}
so wird zusätzlich die Option \Option{clearcolor} beachtet. Dies gilt es 
insbesondere bei der Verwendung der Befehle \Macro{uppertitleback} respektive 
\Macro{lowertitleback} für die Titelrückseite zu beachten.

Das optionale Argument erlaubt~-- ebenso wie bei den \KOMAScript"=Klassen~-- 
die Änderung der Seitenzahl der Titelseite. Diese wird jedoch nicht ausgegeben, 
sondern beeinflusst lediglich die Zählung. Sie sollten hier unbedingt eine 
ungerade Zahl wählen, da sonst die gesamte Zählung durcheinander gerät. 
Wird eine Titelseite (\Option{titlepage}[true]) im \CD der \TnUD gesetzt 
(\Option{cdtitle}[true]), können auch die weiterhin aufgeführten Parameter im 
optionalen Argument verwendet werden. Diese entsprechen in ihrem Verhalten den 
gleichnamigen Optionen respektive Befehlen, wirken sich jedoch nur lokal und 
einzig auf die Titelseite aus. So kann beispielsweise die Nutzung eines 
\DDC-Logos auf den Titel beschränkt bleiben.
\end{Declaration}
\end{Declaration}
\end{Declaration}
\end{Declaration}
\end{Declaration}
\end{Declaration}
\end{Declaration}
\end{Declaration}
\end{Declaration}
\end{Declaration}

\begin{Declaration}{%
  \Macro{maketitleonecolumn}\OLParameter{Seitenzahl}\Parameter{Einspaltentext}%
}
\printdeclarationlist%
\index{Layout!Titel}%
\index{Satzspiegel!doppelseitig}%
\index{Zweispaltensatz}%
%
Im zweispaltigen Satz (Option~\Option{twocolumn}[true]) wird mit 
\Macro{maketitle} die Titelseite selbst immer einspaltig gesetzt. Direkt nach 
dem Titel folgt normalerweise der zweispaltige Fließtext. Mit dem 
\TUDScript-Befehl \Macro{maketitleonecolumn} kann nach dem Titel zusätzlich 
noch weiterer Inhalt~-- beispielsweise eine Zusammenfassung beziehungsweise 
eine Kurfassung~-- einspaltig gesetzt werden.

Wird der Befehl bei einer Titelseite (Option~\Option{titlepage}[true]) 
verwendet, wird der Inhalt des Argumentes (\PName{Einspaltentext}) direkt nach 
dieser auf einer oder gegebenenfalls mehreren neuen Seiten ebenfalls einspaltig 
ausgegeben. Kommt jedoch ein Titelkopf (Option~\Option{titlepage}[false]) zum 
Einsatz, so folgt diesem die einspaltige Textpassage aus dem obligatorischen 
Argument direkt. Dabei erfolgt gegebenenfalls ein automatischer Seitenumbruch, 
falls der Inhalt nicht auf eine einzelne Seite passt. Nach dem obligatorischen 
Argument \PName{Einspaltentext} wird direkt und ohne zusätzlichen Seitenumbruch 
auf das zweispaltige Layout umgeschaltet.

Der optionale Parameter von \Macro{maketitleonecolumn} kann äquivalent zu 
\Macro{maketitle} für die Änderung der Seitenzahl, der verwendeten Schrift 
sowie zur Anpassung von Kopf und Fuß verwendet werden. Dabei ist zu beachten, 
dass ein Großteil der Parameter nur Auswirkungen haben, falls eine Titelseite
(\Option{titlepage}[true]) verwendet wird.
\end{Declaration}

\begin{Declaration}[%
  v2.02!Umschlagseite für Layout ohne \noexpand\CD hinzugefügt,%
  v2.02!Unterstützung der Schriftelemente \Font*{titlehead}{,} 
    \Font*{subject}{,} \Font*{title}{,} \Font*{subtitle}{,} \Font*{author}{,} 
    \Font*{publishers}{,} \Font*{titlepage} und \Font*{thesis}%
]{\Macro{makecover}\OLParameter{Seitenzahl}}
\begin{Declaration}[v2.02]{%
  \Key{\Macro{makecover}}{pagenumber}[\PName{Seitenzahl}]%
}
\begin{Declaration}{\Key{\Macro{makecover}}{cdgeometry}[\PBoolean]}
\begin{Declaration}[v2.02]{\Key{\Macro{makecover}}{cdfont}[\PSet]}{%
  \see*{\Option{cdfont}'ppage'}%
}
\begin{Declaration}[v2.03]{\Key{\Macro{makecover}}{cdhead}[\PSet]}{%
  \see*{\Option{cdhead}'ppage'}%
}
\begin{Declaration}[v2.03]{\Key{\Macro{makecover}}{cdfoot}[\PSet]}{%
  \see*{\Option{cdfoot}'ppage'}%
}
\begin{Declaration}[v2.03]{%
  \Key{\Macro{makecover}}{headlogo}[\PName{Dateiname}]%
}{\see*{\Macro{headlogo}'ppage'}}
\begin{Declaration}[v2.03]{%
  \Key{\Macro{makecover}}{footlogo}[\PName{Dateinamenliste}]
}{\see*{\Macro{footlogo}'ppage'}}
\begin{Declaration}[v2.03]{\Key{\Macro{makecover}}{ddc}[\PSet]}{%
  \see*{\Option{ddc}'ppage'}%
}
\begin{Declaration}[v2.03]{\Key{\Macro{makecover}}{ddchead}[\PSet]}{%
  \see*{\Option{ddchead}'ppage'}%
}
\begin{Declaration}[v2.03]{\Key{\Macro{makecover}}{ddcfoot}[\PSet]}{%
  \see*{\Option{ddcfoot}'ppage'}%
}
\printdeclarationlist%
%
Eine Umschlagseite wird zumeist für gebundene Abschlussarbeiten verlangt, um 
diese beispielsweise für einen Prägedruck auf dem Buchdeckel zu verwenden. 
Deshalb ist die farbige Ausprägung der Umschlagseite auch deaktiviert, wenn 
diese für das restliche Dokument aktiv ist (\Option{cd}[color]). Dies kann 
jedoch jederzeit mit \Option{cdcover}[\PSet] überschrieben werden.

Wird \Option{cdcover}[true] gewählt, so wird die Umschlagseite im \CD der 
\TnUD gesetzt. Auf dieser werden der Titel des Dokumentes, die Typisierung 
durch \Macro{thesis} und/oder \Macro{subject} sowie der Autor oder respektive 
die Autoren und gegebenenfalls der mit \Macro{publishers} angegebene Verlag 
ausgegeben.
\ChangedAt{v2.02}
Für die Einstellung \Option{cdcover}[false] wird lediglich der normale 
\KOMAScript"=Titel als separate Umschlagseite ausgegeben. 

Die Titelseite selbst gehört immer zum Buchblock und wird daher im gleichen 
Satzspiegel gesetzt. Dem entgegen steht die Umschlagseite, welche zumeist in 
einem anderen Layout erscheint. Normalerweise wird das Cover~-- unabhängig von 
der Option \Option{cdgeometry}~-- im asymmetrischen Satzspiegel des \CDs 
gesetzt. Mit \Key{\Macro{makecover}}{cdgeometry}[false] im optionalen Argument 
kann das Verhalten geändert werden. In diesem Fall erscheint auch die 
Umschlagseite im Buchblock des restlichen Dokumentes. Allerdings können für 
diese Einstellung die Seitenränder mit den Befehlen \Macro{coverpagetopmargin}, 
\Macro{coverpageleftmargin}, \Macro{coverpagerightmargin} sowie 
\Macro{coverpagebottommargin} durch den Nutzer frei angepasst werden. Mehr dazu 
ist im \KOMAScript"=Handbuch \scrguide zu finden.

Außerdem kann mit dem optionalen Argument die Seitenzahl der Umschlagseite 
geändert werden. Diese wird jedoch nicht ausgegeben, sondern beeinflusst 
lediglich die Zählung. Sie sollten hier unbedingt eine ungerade Zahl wählen, da 
sonst die gesamte Zählung durcheinander gerät. Die weiterhin aufgeführten 
Parameter entsprechen in ihrem Verhalten beziehungsweise ihrer Funktion den 
gleichnamigen Optionen respektive Befehlen, wirken sich jedoch nur lokal und 
einzig auf die Umschlagseite aus.

\end{Declaration}
\end{Declaration}
\end{Declaration}
\end{Declaration}
\end{Declaration}
\end{Declaration}
\end{Declaration}
\end{Declaration}
\end{Declaration}
\end{Declaration}
\end{Declaration}

\begin{Declaration}{\Macro{titledelimiter}\Parameter{Trennzeichen}}
\printdeclarationlist%
\index{Titel!Felder}\index{Titel!Trennzeichen}%%
%
Für den Titel und die Umschlagseite werden durch die \TUDScript-Klassen
eine Reihe von zusätzlichen Feldern bereitgestellt. Einigen dieser Felder wird 
eine Beschreibung (\see*{\autoref{sec:localization}}) vorangestellt. Dazwischen 
wird bei der Ausgabe ein Trennzeichen eingefügt. Ein Doppelpunkt gefolgt von 
einem Leerzeichen (:\Macro*{nobreakspace}) ist hierfür die Voreinstellung. Mit 
dem Befehl \Macro{titledelimiter} lässt sich dieses Trennzeichen beliebig an 
die individuellen Wünsche des Anwenders anpassen.
\end{Declaration}

\begin{Declaration}{\Macro{author}\Parameter{Autor(en)}}
\begin{Declaration}{\Macro{authormore}\Parameter{Autorenzusatz}}
\begin{Declaration}{\Macro{dateofbirth}\Parameter{Geburtsdatum}}
\begin{Declaration}{\Macro{placeofbirth}\Parameter{Geburtsort}}
\begin{Declaration}{\Macro{matriculationnumber}\Parameter{Matrikelnummer}}
\begin{Declaration}{\Macro{matriculationyear}\Parameter{Immatrikulationsjahr}}
\printdeclarationlist%
\index{Titel!Felder}\index{Autorenangaben|?}%
\index{Datum!Geburtsdatum|?}%
%
Mit dem Befehl \Macro{author} wird der Autor angegeben. Innerhalb des 
Argumentes können auch mehrere Autoren aufgeführt werden, wobei diese in diesem 
Fall jeweils mit \Macro{and} zu trennen sind. Alle weiteren hier vorgestellten 
Befehle können selbst im Argument von \Macro{author} verwendet werden, wodurch 
für jeden Autor individuelle Angaben möglich sind.

Mit \Macro{authormore} wird unter dem Autor eine Zeile ausgegeben, welche 
durch den Anwender frei belegt werden kann. Sollte das Paket \Package{isodate} 
geladen sein, so wird die damit eingestellte Formatierung des Datums durch 
\Macro{dateofbirth}~-- wie übrigens bei jedem anderem Datumsfeld der 
\TUDScript-Klassen auch~-- verwendet. Dafür der Befehl \Macro{printdate} aus 
diesem Paket verwendet. Die weiteren Befehle als zusätzliche Angabe erklären 
sich von selbst.
\end{Declaration}
\end{Declaration}
\end{Declaration}
\end{Declaration}
\end{Declaration}
\end{Declaration}

\begin{Declaration}{\Macro{and}}
\printdeclarationlist%
\index{Kollaboratives Schreiben|?}\index{Titel!Kollaboratives Schreiben}%
\index{Autorenangaben!kollaborativ}
%
Dieser Befehl wird sowohl bei den \hologo{LaTeX}"=Standardklassen als auch bei 
den \KOMAScript"=Klassen lediglich auf der Titelseite dazu verwendet, mehrere 
Autoren im Argument von \Macro{author} voneinander zu trennen.

Bei den \TUDScript-Klassen hingegen ist dieser Befehl derart in seiner Funktion 
erweitert worden, dass damit die Angabe einer kollaborativen Autorenschaft für 
Abschlussarbeiten innerhalb des Befehls \Macro{author} möglich ist. Außerdem 
kann er noch im Argument von \Macro{supervisor}, \Macro{referee} sowie 
\Macro{advisor} verwendet werden, um mehrere Betreuer beziehungsweise Gutachter 
und Fachreferenten anzugeben. Er ist dabei nicht auf die Verwendung für den 
Titel allein beschränkt. Auch bei den Umgebungen \Environment{task}, 
\Environment{evaluation} und \Environment{notice} kann er eingesetzt werden.
\end{Declaration}
%
\begin{Example}
Angenommen, es soll eine Abschussarbeit von zwei unterschiedlichen Autoren in 
kollaborativer Gemeinschaft erstellt werden, so könnte man die Autorenangaben 
folgendermaßen gestalten:
\begin{Code}
\author{%
  Mickey Mouse
  \matriculationnumber{12345678}
  \dateofbirth{2.1.1990}
  \placeofbirth{Dresden}
\and%
  Donald Duck
  \matriculationnumber{87654321}
  \dateofbirth{1.2.1990}
  \placeofbirth{Berlin}
}
\matriculationyear{2010}
\end{Code}
Alle zusätzlichen Angaben außerhalb des Argumentes von \Macro{author} werden 
für beide Autoren gleichermaßen übernommen. Angaben innerhalb des Argumentes 
von \Macro{author} werden den jeweiligen, mit \Macro{and} getrennten Autoren 
zugeordnet. Mehr dazu ist im Minimalbeispiel in \autoref{sec:exmpl:thesis}.
\end{Example}

\begin{Declaration}{\Macro{thesis}\Parameter{Typisierung}}
\begin{Declaration}{\Macro{subject}\Parameter{Typisierung}}
\printdeclarationlist%
\index{Titel!Felder}%
\index{Abschlussarbeit|!}\index{Typisierung}%
%
Mit diesen beiden Befehlen kann der Typ der Dokumentes beziehungsweise der 
Abschlussarbeit angegeben werden. Während der Befehl \Macro{thesis} den Inhalt 
des Feldes unter dem Titel vertikal zentriert und in \DIN auf der Titelseite 
ausgibt, erscheint der Inhalt des Befehls \Macro{subject} in \Univers oberhalb 
des Titels. Es können auch beide Befehle parallel mit unterschiedlichen 
Inhalten verwendet werden. Der Befehl \Macro{thesis} dient den 
\TUDScript"=Dokumentklassen außerdem zur Erkennung von Abschlussarbeiten 
gedacht, da für diese spezielle Felder bereitgehalten werden und auch die 
Titelseite leicht geändert gesetzt wird.

Des Weiteren ist es bei beiden Befehlen möglich, spezielle Werte als Argument 
zur Typisierung des Dokumentes zu verwenden. Diese werden entsprechend der 
gewählten Dokumentensprache~-- entweder Deutsch oder Englisch~-- entschlüsselt 
und gesetzt. Die möglichen Werte sind \autoref{tab:thesis} zu entnehmen. Dabei 
ist zu beachten, dass das Setzen eines speziellen Wertes für \emph{entweder} 
\Macro{thesis} \emph{oder} \Macro{subject} möglich ist. Die Verwendung eines 
der genannten Werte führt immer dazu, dass das Dokument als Abschlussarbeiten 
erkannt und die erweiterte Titelseite aktiviert wird. Gleichzeitig wird damit 
die Option \Option{subjectthesis} beeinflusst. Sollte vom Anwender kein 
explizites Verhalten für \Option{subjectthesis} definiert sein, so führt die 
Verwendung von \Macro{thesis}\Parameter{Wert} zu \Option{subjectthesis}[false] 
und \Macro{subject}\Parameter{Wert} zu \Option{subjectthesis}[true].
%
\begin{table}
\index{Bezeichner}\index{Bezeichner!Typisierung}%\\
\index{Abschlussarbeit!Typisierung}%
\caption{%
  Spezielle Werte zur Typisierung des Dokumentes für
  \Macro{thesis} und \Macro{subject}%
}
\label{tab:thesis}%
\centering%
\makeatletter%
\def\@tempa#1{%
  \Term{#1} & \@nameuse{#1} & \selectlanguage{english}\@nameuse{#1}%
  \tabularnewline%
}%
\begin{tabular}{llll}
  \toprule
  \textbf{Wert} & \textbf{Bezeichner}
    & \textbf{Deutsch} & \textbf{Englisch} \tabularnewline
  \midrule
  diss & \@tempa{dissertationname}
  doctoral & \@tempa{dissertationname}
  phd & \@tempa{dissertationname}
  diploma & \@tempa{diplomathesisname}
  master & \@tempa{masterthesisname}
  bachelor & \@tempa{bachelorthesisname}
  student & \@tempa{studentresearchname}
  project & \@tempa{projectpapername}
  seminar & \@tempa{seminarpapername}
  research & \@tempa{researchname}
  log & \@tempa{logname}
  report & \@tempa{reportname}
  internship & \@tempa{internshipname}
  \bottomrule
\end{tabular}
\makeatother%
\end{table}
\end{Declaration}
\end{Declaration}

\begin{Declaration}{\Option{subjectthesis}[\PBoolean]}%
  [false][\Macro{subject}\Parameter{\autoref{tab:thesis}}:true]
\printdeclarationlist%
%
Der Befehl \Macro{thesis} dient den \TUDScript"=Hauptklassen zur Unterscheidung 
zwei unterschiedlicher Ausprägungen der Titelseite und ist speziell für 
Abschlussarbeiten gedacht. Außerdem kann bei der Nutzung spezieller Werte 
aus \autoref{tab:thesis} innerhalb des Argumentes von \Macro{subject} ebenfalls 
das Verhalten für Abschlussarbeiten aktiviert werden, wobei hierdurch die 
Einstellung \Option{subjectthesis}[true] automatisch vorgenommen wird.

Für den Standardfall~-- bekanntlich \Option{subjectthesis}[false]~-- wird der 
durch \Macro{thesis} gegebene Typ der Abschlussarbeit sowie der gegebenenfalls 
durch \Macro{graduation} gesetzte angestrebte Abschluss in großen Lettern und 
sehr zentral auf der Titelseite gesetzt. Die Verwendung von \Macro{subject} ist 
hierbei weiterhin möglich.
%
Wird die Option mit \Option{subjectthesis}[true] aktiviert, so wird die mit 
\Macro{thesis} gesetzte Bezeichnung nicht unterhalb sondern oberhalb des Titels 
an der Stelle von \Macro{subject} ausgegeben. Der mit \Macro{graduation} 
angegebene Abschluss wird weiterhin unter dem Titel, allerdings in schlankerer 
Schrift gesetzt. Eine etwaige Verwendung des Befehls \Macro{subject} wird in 
diesem Fall ignoriert.
%
\begin{values}
\itemfalse
  Die Ausgabe des Typs der Abschlussarbeit (\Macro{thesis}) selbst sowie des 
  angestrebten Abschlusses (\Macro{graduation}) erfolgt in großen Lettern in 
  \DIN zentral auf der Titelseite.
\itemtrue*
  Der Typ der Abschlussarbeit (\Macro{thesis}) wird oberhalb des Titels in der 
  Betreffzeile gesetzt. Der angestrebte Abschluss (\Macro{graduation}) wird 
  zentral in der schlankeren \Univers ausgegeben.
\end{values}
\end{Declaration}

\begin{Declaration}[v2.02]{\Macro{graduation}\OParameter{Kurzform}\Parameter{Grad}}
\printdeclarationlist%
\index{Titel!Felder}%
%
Mit diesem Befehl wird der angestrebte akademische Grad auf der Titelseite 
ausgegeben. Da dies nur mit einer Abschlussarbeit erreicht werden kann erfolgt 
die Ausgabe nur, wenn entweder \Macro{thesis} oder \Macro{subject} verwendet 
wurde, wobei bei letzterem Befehl im Argument zwingend ein Wert aus 
\autoref{tab:thesis} verwendet werden muss.

Die Option \Option{subjectthesis} hat Einfluss auf die Ausgabe auf der 
Titelseite. Für die Einstellung \Option{subjectthesis}[false] wird der 
Abschuss~-- ähnlich wie 
der Typ der Abschlussarbeit~-- zentral und in relativ großen Lettern gesetzt. 
Für \Option{subjectthesis}[true] erfolgt die Ausgabe kleiner und in weniger 
starken Buchstaben.
\end{Declaration}

\begin{Declaration}{\Macro{supervisor}\Parameter{Name(n)}}
\begin{Declaration}{\Macro{referee}\Parameter{Name(n)}}
\begin{Declaration}{\Macro{advisor}\Parameter{Name(n)}}
\begin{Declaration}{\Macro{professor}\Parameter{Name}}
\printdeclarationlist%
\index{Titel!Felder}%
\index{Betreuer|?}\index{Gutachter|?}\index{Referent|?}%
%
Mit \Macro{supervisor}, \Macro{referee} und \Macro{advisor} werden die Betreuer 
einer Abschlussarbeit beziehungsweise die Gutachter und Fachreferenten einer 
Dissertation angegeben. Zusätzlich kann mit \Macro{professor} der betreuende 
Hochschullehrer beziehungsweise die betreuenden Professoren für studentische 
Arbeiten angegeben werden. Die Angabe mehrerer Person erfolgt wie beim Befehl 
\Macro{author} durch die Trennung mittels \Macro{and}.
\end{Declaration}
\end{Declaration}
\end{Declaration}
\end{Declaration}

\begin{Declaration}{\Macro{date}\OParameter{Ergänzung}\Parameter{Datum}}
\begin{Declaration}{\Macro{defensedate}\Parameter{Verteidigungsdatum}}
\printdeclarationlist%
\index{Titel!Felder}
\index{Datum|?}\index{Datum!Verteidigungsdatum|?}%
%
Mit \Macro{date} kann das Datum angegeben werden. Das optionale Argument 
erlaubt eine zusätzliche Anmerkung, welche nach dem Datum ausgegeben wird. Das 
Datum wird bei normalen Dokumenten direkt nach dem Autor beziehungsweise den 
Autoren ausgegeben. Bei Abschlussarbeiten~-- aktiviert durch die Verwendung von 
\Macro{thesis} oder \Option{subjectthesis}~-- erscheint dieses am Ende der 
Titelseite als Abgabedatum. Außerdem kann in diesem Fall mit  dem Befehl
\Macro{defensedate} das Datum der Verteidigung angegeben werden, wie es 
beispielsweise bei dem Druck von Dissertationen üblich ist.

Sollte das Paket \Package{isodate} geladen sein, so wird die damit eingestellte 
Formatierung des Datums durch den Befehl \Macro{printdate} aus diesem Paket für 
alle Datumsfelder des Dokumentes und folglich auch für die beiden Felder 
\Macro{date} und \Macro{defensedate} verwendet.
\end{Declaration}
\end{Declaration}

\begin{Declaration}{\Macro{extratitle}\Parameter{Schmutztitel}}
\begin{Declaration}{\Macro{titlehead}\Parameter{Kopf}}
\begin{Declaration}{\Macro{title}\Parameter{Titel}}
\begin{Declaration}[%
  v2.01!Bugfix für Schriftstärke bei Verwendung des Untertitels%
]{\Macro{subtitle}\Parameter{Untertitel}}
\begin{Declaration}{\Macro{publishers}\Parameter{Verlag}}
\begin{Declaration}{\Macro{thanks}\Parameter{Fußnote}}
\begin{Declaration}{\Macro{uppertitleback}\Parameter{Titelrückseitenkopf}}
\begin{Declaration}{\Macro{lowertitleback}\Parameter{Titelrückseitenfuß}}
\begin{Declaration}{\Macro{dedication}\Parameter{Widmung}}
\printdeclarationlist%
\index{Titel!Felder}%
%
Diese Befehle entsprechen den in ihrem Verhalten den originalen Pendants der 
\KOMAScript"=Klassen{} und sollen hier der Vollständigkeit halber erwähnt 
werden.

Die Ausgabe des mit \Macro{extratitle} definierten Schmutztitels~-- welcher 
beliebig gestaltet und formatiert werden kann~-- erfolgt als Bestandteil der 
Titelei mit \Macro{maketitle} vor der eigentlichen Titelseite. Mit dem Befehl 
\Macro{titlehead} kann ein zusätzlicher, beliebig formatierbarer Text oberhalb 
der Typisierung und des Titels ausgegeben werden. Da die vertikale Position des 
Dokumenttitels durch das \CD fest vorgegeben ist, kann es~-- im Gegensatz zu 
den \KOMAScript"=Klassen~-- passieren, dass der Kopf des Haupttitels selbst in 
die Kopfzeile ragt. Dies wird durch die \TUDScript-Klassen nicht geprüft und 
muss gegebenenfalls vom Anwender kontrolliert werden.

Die Befehle \Macro{title} und \Macro{subtitle} bedürfen keiner weiteren 
Erklärung. Anzumerken ist, dass sowohl Titel als auch Untertitel normalerweise 
in Majuskeln und \DIN gesetzt werden. Der mit dem Befehl \Macro{publishers} 
definierte Inhalt muss nicht zwingende einen Verlag bezeichnen sondern kann 
auch andere Informationen beinhalten, welche am Ende der Titelseite ausgegeben 
werden sollen.

Fußnoten werden auf dem Titel nicht mit \Macro{footnote}, sondern mit der 
Anweisung \Macro{thanks} erzeugt. Diese dienen in der Regel für Anmerkungen bei 
Titel oder den Autoren. Als Fußnotenzeichen werden dabei Symbole statt Zahlen 
verwendet. Der Befehl \Macro{thanks} kann nur innerhalb des Arguments einer 
der Anweisungen für die Titelseite wie beispielsweise \Macro{author} oder 
\Macro{title} verwendet werden.

\index{Satzspiegel!doppelseitig}%
Im doppelseitigen Druck lässt sich die Rückseite der Haupttitelseite für 
weitere Angaben nutzen. Sowohl den Titelrückseitenkopf als auch den
Titelrückseitenfuß kann der Anwender mit \Macro{uppertitleback} und 
\Macro{lowertitleback} frei gestalten.

Mit \Macro{dedication} lässt eine separate Widmungsseite zentriert und in etwas 
größerer Schrift setzen. Die Rückseite ist~-- wie auch die des Schmutztitels~-- 
grundsätzlich leer. Die Widmung wird mit der restlichen Titelei ausgegeben und 
muss daher vor der Nutzung von \Macro{maketitle} angegeben werden.
\end{Declaration}
\end{Declaration}
\end{Declaration}
\end{Declaration}
\end{Declaration}
\end{Declaration}
\end{Declaration}
\end{Declaration}
\end{Declaration}

\begin{Declaration}[v2.02]{\Font{titlepage}}
\begin{Declaration}[v2.02]{\Font{thesis}}
\printdeclarationlist%
\index{Schriftelemente}
%
Die \TUDScript-Klassen definieren diese neuen Schriftelemente. Dabei wird 
\Font{titlepage} auf der Titelseite für alle Felder verwendet, welche kein 
spezielles Schriftelement verwenden, welches ohnehin durch \KOMAScript{} 
bereitgestellt wird. Das mit \Macro{thesis} angegebene Feld, in welchem der Typ 
einer Abschlussarbeit angegeben wird, nutzt das Schriftelement~\Font{thesis}. 
Wie diese Elemente angepasst werden können, ist in \autoref{sec:fonts:elements} 
zu finden. 
\end{Declaration}
\end{Declaration}
\index{Titel|!)}


\subsection{Die Teileseite}
\label{sec:part}
%
\ChangedAt{%
  v2.02!\PageStyle{plain.tudheadings} wird genutzt!\Macro{partpagestyle}%
}[Implementierung]
Wird für die Teileseiten das Layout des \CDs verwendet, so wird der Seitenstil 
dieser (\Macro{partpagestyle}) auf \PageStyle{plain.tudheadings} gesetzt. 
Möchten Sie stattdessen einen anderen Seitenstil nutzen, so kann dieser mit 
\Macro*{renewcommand*}\PParameter{\Macro{partpagestyle}}\Parameter{Seitenstil} 
angepasst werden.

\begin{Declaration}{\Option{parttitle}[\PBoolean]}[false]%
\printdeclarationlist%
\index{Teileseiten|?}\index{Layout!Teileseiten}%
%
Diese Option ermöglicht es, den mit \Macro{title} gegebenen Titel des 
Dokumentes selbst in großer Schrift auf einer Teileseite auszugeben, die 
Bezeichnung des mit \Macro{part}\Parameter{Bezeichnung} erzeugten Teils wird 
in diesem Fall in kleiner Schrift direkt darunter gesetzt. Diese 
Layout"=Variante findet sich im Handbuch für das \CD der \TnUD. \notudscrartcl
%
\begin{values}
\itemfalse
  Die Bezeichnung des Teils erscheint in großer Schrift auf der Seite, der 
  Titel des Dokumentes gar nicht.
\itemtrue*
  Der Titel wird in großer Auszeichnung auf der Teileseite gesetzt, die 
  Bezeichnung des Teils selber in kleinerer.
\end{values}
\end{Declaration}

\begin{Declaration}[v2.02]{\Font{parttitle}}
\printdeclarationlist%
\index{Schriftelemente}
%
Mit dem Schriftelement~\Font{parttitle} lässt sich~-- bei aktivierter 
\Option{parttitle}-Option~-- die Schrift für die Bezeichnung des Teils 
beeinflussen. In \autoref{sec:fonts:elements} ist zu finden, wie dieses 
angepasst werden kann.
\end{Declaration}


\subsection{Die Kapitelseite}
\begin{Declaration}{\Option{chapterpage}[\PBoolean]}%
  [false][\Option{cd}[color]:true]%
\printdeclarationlist%
\label{sec:chapter}%
\index{Kapitelseiten|?}\index{Layout!Kapitelseiten|?}%
\index{Satzspiegel!doppelseitig}\index{Vakatseiten}%
%
Mit dieser Einstellung kann die Überschrift eines Kapitels separat auf einer 
Seite ausgegeben werden. Der nachfolgende Text wird auf der nächsten 
beziehungsweise bei doppelseitigem Satz und rechts öffnenden Kapiteln%
\footnote{%
  \Option{twoside} und \Option{open}[right], Standard für \Class{tudscrbook}
}
auf der übernächsten Seite ausgegeben. Die in diesem Fall erzeugte Rückseite 
wird in ihrer Ausprägung~-- wie auch Teileseiten~-- durch die Einstellung von 
\Option{cleardoublespecialpage} bestimmt. Beim farbigen Layout ist diese Option 
standardmäßig aktiviert. \notudscrartcl
%
\begin{values}
\itemfalse
  Es gibt keine Sonderstellung von Kapiteln, der nachfolgende Text wird direkt 
  unter der Überschrift respektive nach der mit \Macro{setchapterpreamble} 
  erzeugten Kapitelpräambel auf der gleichen Seite ausgegeben.
\itemtrue*
  Die Kapitelüberschrift und gegebenenfalls die Kapitelpräambel werden auf 
  einer separaten Seite gesetzt. Der folgende Text erscheint auf der nächsten   
  respektive übernächsten Seite, \seealso*{\Option{cleardoublespecialpage}}.
\end{values}
%
\ChangedAt{%
  v2.02!nicht mehr abhängig von \Macro{partpagestyle}!\Macro{chapterpagestyle}
}[Implementierung]
Mit \Macro*{renewcommand*}\PParameter{\Macro{chapterpagestyle}}%
\Parameter{Seitenstil} lässt sich übrigens~-- unabhängig von der Option 
\Option{chapterpage}~-- der Seitenstil von Kapiteln anpassen. Bei der 
Verwendung von separaten Kapitelseiten ist außerdem das Aktivieren der 
\KOMAScript-Option \Option{chapterprefix} empfehlenswert. Damit werden die
Kapitelüberschriften mit einer Vorsatzzeile gesetzt. Falls ein nummeriertes 
Kapitel erzeugt wird, so wird zunächst in einer Zeile \enquote{Kapitel} gefolgt 
von der aktuellen Kapitelnummer ausgegeben, in der nächsten Zeile wird 
anschließend die eigentliche Überschrift in linksbündigem Flattersatz 
ausgegeben. Genaueres hierzu ist in der \KOMAScript"=Dokumentation 
nachzulesen.
\end{Declaration}



\subsection{Vakatseiten}
\index{Vakatseiten}%
Automatisch erzeugte Vakatseiten~-- auch absichtliche Leerseiten genannt~-- 
findet man in Dokumenten mit den aktivierten Optionen \Option{twoside} und 
\Option{open}[right]\footnote{Standard bei \Class{tudscrbook}} beziehungsweise 
\Option{open}[left] beim Beginn von Teilen und Kapiteln. Für diese kann der 
Seitenstil mit der \KOMAScript"=Option \Option{cleardoublepage} eingestellt 
werden.

\ToDo[doc,imp,nxt]{%
  Rückseite bei Kapitelseiten (auch bei Teilen?) im zweiseitigen Satz per 
  Option nicht als Vakatseite setzen. Implementierung von farbigen Seiten davor 
  (\Option*{open}[right]) sowie danach (\Option*{open}[left]). Setzen von 
  speziellen Inhalten auf diesen Seiten äquivalent zu \Macro*{setpartpreamble} 
  bzw. \Macro*{setchapterpreamble}; ggf. temporär umschalten bzw. Warnung bei 
  Konflikt.
  \url{http://latex.wcms-file3.tu-dresden.de/phpBB3/viewtopic.php?f=11&t=396}
}[v2.06]
\begin{Declaration}{\Option{cleardoublespecialpage}[\PSet]}[true]%
\printdeclarationlist%
\index{Teileseiten}\index{Layout!Teileseiten}%
\index{Kapitelseiten}\index{Layout!Kapitelseiten}%
\index{Satzspiegel!doppelseitig}\index{Layout!Rückseiten}%
%
Diese Option wirkt sich lediglich bei aktiviertem doppelseitigem Satz und 
ausschließlich rechts eröffnenden Seiten für Teile beziehungsweise Kapitel
aus.%
\footnote{\Option{twoside} und \Option{open}[right]}
In diesem Fall kann der Stil der darauffolgenden, linken Seite~-- sprich der 
Rückseite~-- beeinflusst werden. Das Normalverhalten sieht vor, dass nach einem 
Teil die Rückseite unabhängig von der Einstellung für \Option{cleardoublepage} 
immer als vollständig leere Seite ohne Kopf"~ oder Fußzeilen gesetzt wird.

Diese Einstellung erlaubt es, dieses Normalverhalten zu deaktivieren und für 
die Seite nach der Teileseite~-- und abhängig von \Option{chapterpage} 
auch nach einem Kapitelanfang auf einer separaten Seite~-- den Seitenstil der 
Option \Option{cleardoublepage} zu übernehmen. Des Weiteren kann auch ein 
anderer, beliebiger, bereits definierter Seitenstil gewählt werden. Außerdem
kann im farbigen Layout die Rückseite in der gleichen Farbe wie die 
Vorderseite von Teil oder Kapitel gesetzt werden. \notudscrartcl
%
\begin{values}
\itemfalse
  Die Rückseiten sind vollständig leere Seiten, unabhängig von Option
  \Option{cleardoublepage}.
\itemtrue*
  Der Seitenstil der Rückseite von Teilen und gegebenenfalls Kapiteln 
  entspricht der Einstellung von \Option{cleardoublepage} für Vakatseiten.
\item[current]
  Es wird der aktuell definierte Seitenstil (\Macro{pagestyle}) für die 
  erzeugte Rückseite verwendet.
\itemvalues[\PName{Seitenstil}:]
  Mit der Angabe von \Option{cleardoublespecialpage}[\PName{Seitenstil}] 
  kann ein beliebiger, bereits definierter Seitenstil für die Rückseite nach 
  Teilen und Kapiteln verwendet werden.
\item[color]
  Im farbigen Layout ist auch die Rückseite von Teilen und Kapiteln farbig, 
  \see*{\Option{clearcolor}}.
\end{values}
\ToDo[imp,nxt]{\PValue{nocolor}}[v2.06]
\end{Declaration}

\ToDo[doc,imp,nxt]{%
  \Option{clearcolor} in \Option{cleardoublespecialpage}%
}[v2.06]
\begin{Declaration}{\Option{clearcolor}[\PBoolean]}[false]%
\printdeclarationlist%
\index{Titel}\index{Layout!Titel}%
\index{Teileseiten}\index{Layout!Teileseiten}%
\index{Kapitelseiten}\index{Layout!Kapitelseiten}%
\index{Satzspiegel!doppelseitig}\index{Vakatseiten}%
%
Sollten beim farbigen Layout die Optionen \Option{twoside} sowie auch
\Option{open}[right] gesetzt sein, so werden beim Aktivieren dieser Option die 
Rückseiten von Teilen~-- und je nach Einstellung von \Option{chapterpage} 
gegebenenfalls auch von Kapiteln~-- farbig gesetzt.%
\footnote{%
  Dies führt bei der Ausgabe zu farbigen Blättern (Vorder- und Rückseite) der 
  entsprechenden Elemente des Layouts.
}
Die Option wirkt sich ebenfalls auf die Rückseite des Titels aus.%
\footnote{%
  \see*{\Macro{uppertitleback} und \Macro{lowertitleback}} der 
  \KOMAScript"=Dokumentation (\scrguide*)
}
Der Stil dieser zusätzlich eingefügten Rückseiten ist abhängig von der Option
\Option{cleardoublespecialpage}.
%
\begin{values}
\itemfalse
  Es werden weiße Rückseiten bei Titel, Teilen und gegebenenfalls Kapiteln 
  erzeugt.
\itemtrue*
  Die rückwärtigen Seiten der genannten Elemente des Layouts sind farbig.
\end{values}
\end{Declaration}




\subsection{Der Satzspiegel}
\begin{Declaration}[v2.03]{\Option{cdgeometry}[\PSet]}[true]%
\printdeclarationlist%
\index{Satzspiegel|(}\index{Layout!Satzspiegel|(}%
\index{Seitenstil}\index{Layout!Seitenstil}%
\index{Satzspiegel!doppelseitig}%
\index{Layout!Seitenränder}%
%
Diese Option ist für die Aufteilung beziehungsweise die Berechnung des 
Satzspiegels verantwortlich. Das Maß der Seitenränder ist im \CD fest 
vorgegeben und wird standardmäßig von den \TUDScript-Klassen eingehalten. 
Allerdings lassen sich die Seitenränder anpassen, um beispielsweise einen 
vernünftigen doppelseitigen Satz zu ermöglichen.%
\footnote{Hierbei sollte der innere Rand schmaler als der äußere sein}
Des Weiteren besteht die Möglichkeit, auf das Standardverhalten von 
\KOMAScript{} zurückzufallen und die Satzspiegelberechnung durch das Paket
\Package{typearea} vornehmen zu lassen. Hier hat insbesondere die Klassenoption 
\Option{DIV}[\PSet] maßgeblichen Einfluss auf den Satzspiegel. Mehr dazu ist in 
der Dokumentation von \KOMAScript{} zu finden.
%
\begin{values}
\itemfalse
  Die Satzspiegelberechnung erfolgt via \Package{typearea}, die Vorgaben des 
  \CDs bezüglich der Seitenränder werden ignoriert.
\itemtrue*[asymmetric/cd]
  Die Seitenränder werden im asymmetrischen Stil des \CDs fest definiert und 
  auch für den doppelseitigen Satz (\Option{twoside}[true]) genutzt.%
  \footnote{links: 30\,mm, rechts: 20\,mm, oben: 25\,mm, unten: 30\,mm}
\item[symmetric/centred/centered]
  Der Satzspiegel wird im einseitigen sowie doppelseitigen Satz auf der Seite 
  zentriert.%
  \footnote{links: 25\,mm, rechts: 25\,mm, oben: 25\,mm, unten: 30\,mm}
\item[twoside/balanced]
  Diese Einstellung aktiviert den doppelseitigen Satz (\Option{twoside}[true]) 
  und verändert den Satzspiegel derart, dass die Ränder der inneren Seiten 
  schmaler sind als die der äußeren.%
  \footnote{innen: 20\,mm, außen: 30\,mm, oben: 25\,mm, unten: 30\,mm}
  \Attention{%
    Der so erzeugte Satzspiegel ist jedoch nicht sehr vorteilhaft. Es ist zu 
    beachten, dass dabei das Logo der \TnUD sehr nah am inneren Seitenrand 
    des Dokumentes gesetzt wird, folglich insbesondere auf rechten respektive 
    ungeraden Seiten sehr weit an den Blattrand rückt.
  }%
  Diesem Problem kann zumindest teilweise~-- bei den Klassen \Class{tudscrbook} 
  und \Class{tudscrartcl}~-- aus dem Weg gegangen werden, indem Titel, Teile 
  und Kapitel mit der \KOMAScript-Option \Option{open}[left] immer auf einer 
  der nächsten linken Seite begonnen werden.
\end{values}
%
Für die Festlegung der Seitenränder wird das Paket \Package{geometry} genutzt. 
Ist \Option{cdgeometry}[false] gewählt, erfolgt die Berechnung des Satzspiegels 
durch \Package{typearea}. Die damit berechneten Werte werden anschließend an 
\Package{geometry} weitergereicht und durch dieses umgesetzt.
\end{Declaration}

\begin{Declaration}[v2.03]{\Option{extrabottommargin}[\PName{Höhe}]}[0pt]%
\printdeclarationlist%
\index{Fußzeile|?}
%
Mit dieser Option kann die Größe des unteren Seitenrandes angepasst werden, 
wenn der Satzspiegel des \CDs (\Option{cdgeometry}[true/symmetric/balanced]) 
verwendet wird. Insbesondere für den Fall, dass im Fußbereich der Seiten im 
Stil \PageStyle{tudheadings} entweder mit \Macro{footlogo} Drittlogos verwendet 
werden und diese über das optionale Argument oder via \Length{footlogoheight} 
über die Standardhöhe hinaus vergrößert wurden oder mit \Macro{footcontent} ein 
übergroßer Inhalt angegeben wurde, kann dieser unter Umständen etwas zu klein 
sein. Mit der Option \Option{extrabottommargin} wird der Fußbereich durch 
positive Werte vergrößert, negative Werte verkleinern diesen entsprechend. 

Alternativ zu \Option{extrabottommargin} kann auch die Option \Option{cdfoot} 
mit einer Längenangabe verwendet werden. Dabei spielt es für beide Optionen 
keine Rolle, ob eine \hologo{LaTeX}-Länge, ein \hologo{TeX}-Abstand oder eine 
\hologo{TeX}-Ausdehnung als Länge bei der Wertzuweisung verwendet wird.
\end{Declaration}

\minisec{Kopf"~ und Fußzeile im Zusammenspiel mit dem Satzspiegel}
\index{Kopfzeile|?}\index{Layout!Kopfzeile}%
\index{Fußzeile|?}\index{Layout!Fußzeile}%
Da im \CD nicht festgelegt ist, wie die Gestaltung der Kopf"~ und Fußzeilen in 
einer wissenschaftlichen Arbeit auszuführen ist, bleibt dem Nutzer dafür eine 
gewisse Freiheit. Dafür sollte idealerweise das zu \KOMAScript{} gehörige Paket 
\Package{scrlayer-scrpage} genutzt werden. 

In der Dokumentation zu \Package{typearea} wird auch darauf eingegangen, wann 
Kopf"~ und Fußzeile bei der Satzspiegelkonstruktion entweder dem Rand oder dem 
Textkörper zugeschlagen werden sollten. Dies sollte bei der Erstellung eigener 
Kopf"~ und Fußzeilen beachtet werden. Die Einstellung dafür erfolgt mit den 
beiden \KOMAScript"=Optionen \Option{headinclude}[\PBoolean] sowie 
\Option{footinclude}[\PBoolean]. Diese können~-- unabhängig von der gewählten 
Einstellung zur Satzspiegelgestaltung für die Option \Option{cdgeometry}~-- 
verwendet werden.

\minisec{Bindekorrektur}
\index{Bindekorrektur|!}\index{Layout!Bindekorrektur}%
%
Zu erwähnen im Zusammenhang mit Seitenrändern und Satzspiegel ist die durch 
\Package{typearea} angebotene Option \Option{BCOR}[\PName{Länge}], mit der bei 
der Satzspiegelberechnung ein Heftrand beziehungsweise eine Bindekorrektur 
berücksichtigt wird. Die \TUDScript-Klassen reichen diesen Wert auch an 
\Package{geometry} weiter, so dass der Benutzer unabhängig von der Auswahl zur 
Satzspiegelgestaltung diese Option nutzen kann. So kann beispielsweise eine 
Bindekorrektur von \unit[5]{mm} mit der Klassenoption \Option{BCOR}[5mm] 
gesetzt werden.

Eine Anpassung der Bindekorrektur hat natürlich \emph{immer} eine Änderung der 
verfügbaren Breite des Textbereichs zur Folge hat und führt somit zwingend zu 
einer Anpassung des Satzspiegels. Da die Bindekorrektur jedoch abhängig von der 
Höhe des Buchblocks gewählt werden sollte, welche letztendlich erst mit dem 
Druck des fertiggestellten Dokumentes bestimmt werden kann, muss diese zu 
Beginn abgeschätzt werden.
%
\begin{Example}
Als Faustregel gilt, dass die erforderliche Bindekorrektur in etwa der halben 
Höhe des Buchblocks entsprechen sollte. Dessen Höhe wiederum ist abhängig von 
der Anzahl der Seiten sowie der Dichte des verwendeten Papiers. Wird normales 
Papier mit einer Dichte von \unit[80]{g/m²} verwendet, so entsprechen 100~Blatt 
in etwa einer Höhe von \unit[10]{mm}, bei \unit[100]{g/m²} ca. \unit[12]{mm}. 
Dementsprechend wäre eine Bindekorrektur von \Option{BCOR}[5mm] beziehungsweise 
\Option{BCOR}[6mm] bei diesem Beispiel zu wählen.
\end{Example}
\index{Satzspiegel|)}\index{Layout!Satzspiegel|)}%


\subsection{Verwendung von Schriftelementen}
\label{sec:fonts:elements}%
\index{Schriftelemente|!}
Von \TUDScript werden weitere Schriftelemente~-- in Ergänzung zu den bereits
durch \KOMAScript{} bereitgestellten~-- definiert. Dies sind \Font{titlepage}, 
\Font{thesis}, \Font{tudheadings} sowie \Font{parttitle}. Sowohl die bereits 
durch \KOMAScript{} definierten als auch alle hier genannten Schriftelemente 
und später erläuterten sollten im Bedarfsfall durch den Anwender über den 
Befehl \Macro{addtokomafont}\Parameter{Schriftelement}\Parameter{Einstellungen}
angepasst werden. Mehr dazu ist im \KOMAScript"=Handbuch \scrguide im 
Abschnitt \emph{Textauszeichnungen} zu finden.


\subsection{Die Farben des \CDs}
\index{Farben}%
% 
Zur Verwendung der Farben des \CDs wird das Paket \Package{tudscrcolor} 
genutzt. Falls dieses nicht in der Präambel geladen wird~-- um beispielsweise 
zusätzliche Optionen aufzurufen~-- binden die \TUDScript"=Klassen dieses 
automatisch ein. Detaillierte Informationen sind in der Dokumentation von 
\Package{tudscrcolor}'full' zu finden.



\section{Zusätzliche Optionen und Erweiterungen}
\ChangedAt*{%
  v2.03!Bugfix für \abstractname{,} \confirmationname{} und \blockingname{} 
    bei Seitenstil und Kolumnentitel%
}%
Neben den Befehlen für die Anpassung des Layouts an das \CD der \TnUD stellen 
die \TUDScript-Klassen weitere Befehle und Umgebungen zur Verfügung, um die 
Anwendung insbesondere für wissenschaftliche Arbeiten zu erleichtern.


\subsection{Zusammenfassung/Kurzfassung}
\begin{Declaration}[%
  v2.02!Wert \PValue{double} mit \PValue{multi} ersetzt,%
  v2.02!Wert \PValue{tocleveldown} neu,%
  v2.02!Wert \PValue{markboth} neu,%
  v2.04!Wert \PValue{tocmultiple} neu%
]{\Option{abstract}[\PSet]}%
\printdeclarationlist%
\index{Zusammenfassung|!(}%
\index{Zweispaltensatz}%
%
Diese Option wird bereits durch \KOMAScript{} für die Klassen \Class{scrartcl} 
und \Class{scrreprt} standardmäßig bereitgestellt. Für die Klasse 
\Class{scrbook} geschieht dies nicht. Dazu heißt es im Handbuch:
%
\begin{quoting}
Bei Büchern wird in der Regel eine andere Art der Zusammenfassung verwendet. 
Dort setzt man ein entsprechendes Kapitel an den Anfang oder Ende des Werks. 
Oft wird diese Zusammenfassung entweder mit der Einleitung oder einem weiteren 
Ausblick verknüpft. Daher gibt es bei \Class{scrbook} generell keine 
\Environment{abstract}"=Umgebung. Bei Berichten im weiteren Sinne, etwa einer 
Studien- oder Diplomarbeit, ist ebenfalls eine Zusammenfassung in dieser Form 
zu empfehlen.
\end{quoting}
%
Durch die \TUDScript-Klassen wird die \Option{abstract}"=Option erweitert. 
Neben den Auswahlmöglichkeit, welche bereits \KOMAScript{} für die Klassen 
\Class{tudscrartcl} und \Class{tudscrreprt} anbietet, kann die Überschrift für 
die Zusammenfassung außerdem in Gestalt eines \sectionautorefname{}s oder für 
\Class{tudscrreprt} und \Class{tudscrbook} in der Form eines 
\chapterautorefname{}s ausgegeben werden.
%
\begin{values}
\itemfalse[][nur für \Class{tudscrartcl} und \Class{tudscrreprt} verfügbar]
  Es wird keine Überschrift für die \Environment{abstract}"=Umgebung ausgegeben.
\itemtrue*[][nur für \Class{tudscrartcl} und \Class{tudscrreprt} verfügbar]
  Wie bei den \KOMAScript"=Klassen wird eine zentrierte Überschrift mit dem 
  Bezeichner \Term{abstractname} vor der eigentlichen Zusammenfassung gesetzt.
\item[section/addsec]
  Die Überschrift (\Term{abstractname}) verwendet den Gliederungsbefehl 
  \Macro{section}.
\item[chapter/addchap][%
    (Säumniswert für \Class{tudscrbook})
    nur für \Class{tudscrreprt} und \Class{tudscrbook} verfügbar%
  ]
  Es wird der Befehl \Macro{chapter} für das Setzen der Überschrift 
  (\Term{abstractname}) genutzt. 
\item[heading]
  Es wird die höchstmögliche Gliederungsebene verwendet. Für 
  \Class{tudscrartcl} entspricht dies \Option{abstract}[section], bei 
  \Class{tudscrreprt} und \Class{tudscrbook} \Option{abstract}[chapter].
\end{values}
%
Abhängig von der gewählten Gliederungsebene der Überschrift wird das Verhalten 
für das Setzen eines Eintrages ins Inhaltsverzeichnis festgelegt. Ohne oder mit 
zentrierter Überschrift wird per Voreinstellung kein Eintrag erzeugt. Wird die 
Überschrift jedoch in Form einer Gliederungsebene gewählt, so erscheint die 
Zusammenfassung für gewöhnlich im Inhaltsverzeichnis auf der obersten Ebene. 
Das voreingestellte Verhalten für die Einträge ins Inhaltsverzeichnis kann 
jederzeit mit folgenden Werten durch den Anwender überschrieben werden.
%
\begin{values}
\item[notoc/nottotoc]
  Die Zusammenfassung wird definitiv nicht ins Inhaltsverzeichnis eingetragen.
\item[toc/totoc]
  Es wird ein nicht nummerierten Eintrag im Inhaltsverzeichnis auf der obersten 
  Gliederungsebene der verwendeten Dokumentklasse für die Zusammenfassung 
  gesetzt.
\item[leveldown/tocleveldown/totocleveldown]
  \ChangedAt{v2.02}
  Der Inhaltsverzeichniseintrag wird eine Gliederungsebene unterhalb der 
  obersten erzeugt.
\item[tocmultiple/totocmultiple/tocaggregate/totocaggregate]
  \ChangedAt{v2.04}
  Es wird ein \emph{einziger} Inhaltsverzeichniseintrag für \emph{alle} 
  Zusammenfassungen erstellt.
\end{values}
%
\ChangedAt{v2.02}
Außerdem kann das Verhalten für die Kolumnentitel durch den Nutzer beeinflusst 
werden. Normalerweise werden diese nur gesetzt, wenn automatische Kolumnentitel 
aktiviert sind (\Option{automark}) und sind von der Gliederungsebene der 
Überschrift abhängig. Werden manuelle Kolumnentitel genutzt, müssen diese auch 
für die Zusammenfassung manuell gesetzt werden. Mit \Option{abstract}[markboth] 
lässt sich das Setzen der Kolumnentitel jedoch forcieren.
%
\begin{values}
\item[markboth]
  Unabhängig von der Verwendung manueller oder automatischer Kolumnentitel 
  werden diese auf rechten sowie linken Seiten mit \Term{abstractname} gesetzt.
\item[nomarkboth]
  Die Einstellung für manuelle oder automatische Kolumnentitel werden beachtet 
  und abhängig von der verwendeten Gliederungsebene der Überschrift gesetzt.
\end{values}
%
Mit dem optionalen Parameter \Key{\Environment{abstract}}{markboth} der 
\Environment{abstract}"=Umgebung kann der Kolumnentitel mit einem beliebigen 
Inhalt gesetzt werden.

Häufig wird für Abschlussarbeiten verlangt, neben der deutschsprachigen auch 
noch eine englischsprachige Zusammenfassung zu verfassen. Mit der Einstellung 
\Option{abstract}[multiple] lassen sich mehrere Zusammenfassungen auf einer 
Seite ausgeben~-- sofern genügend Platz vorhanden ist. Außerdem kann die 
standardmäßige vertikale Zentrierung der \Environment{abstract}"=Umgebung 
auf einer Seite unterdrückt werden. Diese Einstellungen zur Positionierung der 
Zusammenfassungen innerhalb der \Environment{abstract}"=Umgebung werden nur 
wirksam, wenn eine Titelseite (\Option{titlepage}[true]) und \emph{keine} 
Überschriften in Form von Kapiteln (\Option{abstract}[chapter]) verwendet 
werden.
%
\begin{values}
\item[single/one/simple]
  Jede Zusammenfassung wird auf einer eigenen Seite
  beziehungsweise im zweispaltigen Satz in einer neuen Spalte ausgegeben.
\item[multiple/multi/all/aggregate]
  \ChangedAt{v2.02}
  Zusammenfassungen, welche mit \Macro{nextabstract} getrennt wurden, werden 
  direkt nacheinander auf der gleichen Seite ausgegeben, wenn ausreichend Platz 
  auf dieser vorhanden sein sollte. Ist die Option \Option{twocolumn} aktiviert,
  erfolgt die Ausgabe aller Zusammenfassungen ohne Spaltenumbruch.
\item[fil/fill/vfil/vfill]
  Alle Zusammenfassungen auf einer Ausgabeseite werden vertikal zentriert. Für 
  den zweispaltigen Satz mit \Option{twocolumn} steht diese Einstellung nicht 
  zur Verfügung.
\item[nofil/nofill/novfil/novfill]
  Die Ausgabe erfolgt wie im normalen Fließtext auch.
\end{values}
\end{Declaration}

\begin{Declaration}[%
  v2.02!\Macro{nextabstract} zur Trennung der einzelnen Teile%
]{\Environment{abstract}[\OLParameter{Sprache}]}
\begin{Declaration}[v2.02]{\Macro{nextabstract}\OLParameter{Sprache}}
\begin{Declaration}{\Key{\Environment{abstract}}{language}[\PName{Sprache}]}
\begin{Declaration}[v2.02]{%
  \Key{\Environment{abstract}}{markboth}[\PBName{Kolumnentitel}]%
}
\begin{Declaration}[v2.02]{%
  \Key{\Environment{abstract}}{pagestyle}[\PName{Seitenstil}]%
}
\begin{Declaration}{\Key{\Environment{abstract}}{columns}[\PName{Anzahl}]}
\begin{Declaration}{\Key{\Environment{abstract}}{option}[\PSet]}{%
  \see*{\Option{abstract}'ppage'}%
}
\printdeclarationlist%
\index{Zweispaltensatz}%
%
Die \Environment{abstract}-Umgebung dient speziell für die Ausgabe einer 
Zusammenfassung, entweder zu Beginn eines Dokumentes oder beispielsweise vor 
einem Teil oder Kapitel. Wird ein Titelkopf (\Option{titlepage}[false]) und 
keine Titelseite verwendet, so wird für den Fall, dass die Zusammenfassung 
\emph{nicht} mit der Überschrift einer Gliederungsebene gesetzt wird, diese wie 
bei den \KOMAScript"=Klassen in einer \Environment{quotation}"=Umgebung 
gesetzt, um diese vom restlichen Fließtext abzuheben. Diese hat jedoch den 
Nachteil, dass in besagter Umgebung die Option \Option{parskip} nicht beachtet 
wird. Um dieses Problem zu beheben, kann das Paket \Package{quoting} geladen 
werden, wodurch stattdessen die Umgebung \Environment{quoting} verwendet wird.

Mit der zuvor erläuterten Option \Option{abstract} kann eingestellt werden, in 
welcher Gestalt die Zusammenfassung ausgegeben werden soll. Des Weiteren lässt 
sich jede \Environment{abstract}"=Umgebung individuell über weitere Parameter 
als optionales Argument anpassen. Damit lassen sich gegebenenfalls für eine 
bestimmte \Environment{abstract}"=Umgebung die globalen Einstellungen 
der Option \Option{abstract} lokal ändern und gezielt anpassen. 

Wird das Paket \Package{babel} durch den Anwender geladen, kann mit dem 
optionalen Parameter \Key{\Environment{abstract}}{language}[\PName{Sprache}] 
die Sprache innerhalb der \Environment{abstract}"=Umgebung geändert werden. 
Dafür muss die gewünschte Sprache bereits mit dem Laden von \Package{babel} 
entweder als Paketoption oder besser noch als Klassenoption angegeben worden 
sein. Dadurch werden innerhalb der Umgebung die Bezeichnung \Term{abstractname} 
und die Trennungsmuster sprachspezifisch angepasst. Die gewünschte Sprache kann 
auch ohne die Verwendung des Parameters \Key{\Environment{abstract}}{language} 
direkt als optionales Argument übergeben werden.

\ChangedAt{v2.02}
Mit \Key{\Environment{abstract}}{markboth} können die gesetzten Kolumnentitel 
beeinflusst werden. Wird \Key{\Environment{abstract}}{markboth}[false] 
angegeben, werden automatische respektive manuelle Kolumnentitel verwendet. Die 
Einstellung \Key{\Environment{abstract}}{markboth}[true] wiederum setzt diese 
für linke und rechte Seiten auf \Term{abstractname}. Außerdem lässt sich der 
Kolumnentitel mit \Key{\Environment{abstract}}{markboth}[\PName{Kolumnentitel}] 
auch direkt festlegen. So können die Kolumnen beispielsweise mit der Verwendung 
von \Key{\Environment{abstract}}{markboth}[\PParameter{}] auch gelöscht werden. 
Sollte \Key{\Environment{abstract}}{markboth} aktiviert werden, so wird in der
Umgebung automatisch der Seitenstil \PageStyle{headings} genutzt~-- falls eine 
Titelseite und kein Titelkopf (\Option{titlepage}[true]) verwendet wird. Mit 
dem Parameter \Key{\Environment{abstract}}{pagestyle} kann dieser auch manuell 
angegeben werden, wobei die \PageStyle{tudheadings}"=Seitenstile ebenfalls 
unterstützt werden.

Wurde das Paket \Package{multicol} geladen, kann mit dem Parameter 
\Key{\Environment{abstract}}{columns}[\PName{Anzahl}] die Zusammenfassung 
mehrspaltig gesetzt werden. Dem Parameter \Key{\Environment{abstract}}{option} 
können alle gültigen, bereits erläuterten Werte der Option \Option{abstract} 
übergeben werden. Die damit gemachten Einstellungen wirken sich~-- im Gegensatz 
zur Angabe als Klassenoption oder über die Variante der späten Optionenwahl%
\footnote{%
  \Macro{TUDoption}\PParameter{abstract}\Parameter{Einstellung} oder
  \Macro{TUDoptions}\PParameter{abstract=\PName{Einstellung}}
}~-- lediglich lokal auf die verwendete \Environment{abstract}"=Umgebung aus.

\ChangedAt{v2.02}
Sollen mehrere Zusammenfassungen im gleichen Stil erzeugt und die Einstellungen 
der Option \Option{abstract}[simple/multiple/fill/nofill] beachtet werden, so 
ist die \Environment{abstract}"=Umgebung nur einmal zu verwenden. Innerhalb 
dieser müssen die einzelnen Zusammenfassungen mit \Macro{nextabstract} 
voneinander getrennt werden. Der Befehl akzeptiert dabei im optionalen Argument 
alle Parameter, die auch von der \Environment{abstract}"=Umgebung selbst 
unterstützt werden. Das Minimalbeispiel in \fullref{sec:exmpl:dissertation} 
zeigt hierfür das notwendige Vorgehen.

Wird die \Environment{abstract}"=Umgebung innerhalb des Argumentes der Befehle 
\Macro{setpartpreamble} beziehungsweise \Macro{setchapterpreamble} verwendet, 
so wird die Überschrift~-- im Fall, dass nicht \Option{abstract}[false] gewählt 
ist~-- \emph{immer} in Textgröße und zentriert gesetzt.
\end{Declaration}
\end{Declaration}
\end{Declaration}
\end{Declaration}
\end{Declaration}
\end{Declaration}
\end{Declaration}

\minisec{Umbenennung der \abstractname}
Mit dem \KOMAScript-Befehl \Macro{renewcaptionname} kann der Bezeichner~-- 
sprich der Wortlaut~-- der für die \Environment{abstract}-Umgebung verwendeten 
Überschrift verändert werden. Mehr dazu ist in \autoref{sec:localization} zu 
finden.
%
\begin{Example}
Die Überschrift der \Environment{abstract}-Umgebung soll für die Sprache 
\PValue{ngerman} von \enquote{\abstractname} in \enquote{Kurzfassung} umbenannt 
werden. Der Befehl \Macro{renewcaptionname} erwartet die drei obligatorischen 
Argumente \Parameter{Sprache}\Parameter{Makro}\Parameter{Inhalt}:
\begin{Code}[escapechar=§]
\renewcaptionname{ngerman}{\abstractname}{Kurzfassung}
\end{Code}
\end{Example}
%
\index{Zusammenfassung|!)}%


\subsection{Selbstständigkeitserklärung und Sperrvermerk}
\begin{Declaration}[%
  v2.02!Wert \PValue{double} mit \PValue{multi} ersetzt,%
  v2.02!Wert \PValue{tocleveldown} neu,%
  v2.02!Wert \PValue{markboth} neu,%
  v2.04!Wert \PValue{tocmultiple} neu%
]{\Option{declaration}[\PSet]}[true]%
\printdeclarationlist%
\index{Selbstständigkeitserklärung|!}\index{Sperrvermerk|!}%
%
Mit \Option{declaration} kann äquivalent zur Option \Option{abstract} die 
Gestaltung von Selbstständigkeitserklärung und Sperrvermerk angepasst werden.
Zur Ausgabe der Erklärungen werden die Umgebung \Environment{declarations} 
sowie die Befehle \Macro{declaration} beziehungsweise \Macro{confirmation} und 
\Macro{blocking} bereitgestellt. 

Die beiden Optionen \Option{abstract} und \Option{declaration} ähneln sich sehr 
stark. Alle möglichen Wertzuweisungen für \Option{declaration} wurden bereits 
bei der Beschreibung von \Option{abstract} ausführlich erläutert. Deshalb 
geschieht dies hier in einer etwas kürzeren Ausführung. Sollte Ihnen eine 
Erläuterung etwas dürftig erscheinen, so hilft mit Sicherheit ein Blick zur 
Erklärung der Option \Option{abstract}'full'.

Die möglichen Werte für die Gestaltung der Überschrift werden nachfolgend 
genannt. Im Gegensatz zur Option \Option{abstract} stehen die Einstellungen 
\Option{declaration}[true/false] auch für die Klasse \Class{tudscrbook} zur 
Verfügung.
%
\begin{values}
\itemfalse
  Es wird keine Überschrift über den Erklärungen selbst ausgegeben.
\itemtrue*
  Eine zentrierte Überschrift mit dem Bezeichner \Term{confirmationname} vor 
  der Selbstständigkeitserklärung beziehungsweise \Term{blockingname} vor dem 
  Sperrvermerk wird gesetzt. 
\item[section/addsec]
  Die Überschrift verwendet den Gliederungsbefehl \Macro{section}.
\item[chapter/addchap][%
    (Säumniswert für \Class{tudscrbook})
    nur für \Class{tudscrreprt} und \Class{tudscrbook} verfügbar%
  ]
  Es wird der Befehl \Macro{chapter} für das Setzen der Überschrift genutzt. 
\item[heading]
  Es wird die höchstmögliche Gliederungsebene verwendet. Für 
  \Class{tudscrartcl} entspricht dies \Option{declaration}[section], bei 
  \Class{tudscrreprt} und \Class{tudscrbook} \Option{declaration}[chapter].
\end{values}
%
Abhängig von der gewählten Gliederungsebene der Überschrift wird das Verhalten 
für das Setzen eines Eintrages ins Inhaltsverzeichnis festgelegt. Normalerweise 
wird nur für Überschriften in Form einer Gliederungsebene ein Eintrag der 
Erklärung ins Inhaltsverzeichnis erstellt, für \Option{declaration}[true/false] 
geschieht dies standardmäßig nicht. Das voreingestellte Verhalten kann mit 
folgenden Werten überschrieben werden.
%
\begin{values}
\item[notoc/nottotoc]
  Die Erklärung wird definitiv nicht ins Inhaltsverzeichnis eingetragen.
\item[toc/totoc]
  Unabhängig von der Wahl der Überschrift erhält jede Erklärung einen nicht
  nummerierten Eintrag im Inhaltsverzeichnis auf der obersten Gliederungsebene  
  der aktuell gerade verwendeten Dokumentklasse. 
\item[leveldown/tocleveldown/totocleveldown]
  \ChangedAt{v2.02}
  Der Inhaltsverzeichniseintrag wird eine Gliederungsebene unterhalb der 
  obersten erzeugt.
\item[tocmultiple/totocmultiple/tocaggregate/totocaggregate]
  \ChangedAt{v2.04}
  Es wird ein \emph{einziger} Inhaltsverzeichniseintrag für \emph{alle} 
  Erklärungen erstellt.
\end{values}
%
\ChangedAt{v2.02}
Normalerweise werden die automatischen Kolumnentitel in Abhängigkeit von der 
Gliederungsebene der Überschrift gesetzt, falls diese denn aktiviert sind 
(\Option{automark}). Werden manuelle Kolumnentitel genutzt, müssen diese auch 
für die Erklärungen manuell gesetzt werden. Mit \Option{declaration}[markboth] 
lässt sich außerdem das Setzen der Kolumnentitel auf linken und rechten Seiten 
forcieren, wobei hierfür der Titel der Überschrift genutzt wird.
%
\begin{values}
\item[markboth]
  Unabhängig von der Verwendung manueller oder automatischer Kolumnentitel 
  werden diese auf rechten sowie linken Seiten mit den Bezeichnern 
  \Term{confirmationname} beziehungsweise \Term{blockingname} gesetzt.
\item[nomarkboth]
  Die Einstellung für manuelle oder automatische Kolumnentitel werden beachtet.
\end{values}
%
Für \Macro{declaration} respektive \Macro{confirmation} und \Macro{blocking} 
sowie die \Environment{declaration}"=Umgebung lässt sich mit dem Parameter 
\Key{\Environment{declaration}}{markboth} ein beliebiger Kolumnentitel setzen. 

Die folgenden Einstellungen zur Positionierung der Erklärungen haben lediglich 
Auswirkungen, wenn die Überschrift der Erklärung \emph{nicht} im Form eines 
Kapitels ausgegeben und eine Titelseite (\Option{titlepage}[true]) verwendet 
wird.
%
\begin{values}
\item[single/one/simple]
  Jede Erklärung wird auf einer separaten Seite
  beziehungsweise im zweispaltigen Satz in einer neuen Spalte ausgegeben.
\item[multiple/multi/all/aggregate]
  \ChangedAt{v2.02}
  Erklärungen, welche in der \Environment{declarations}"=Umgebung mit den 
  Befehlen \Macro{confirmation}, \Macro{blocking} und \Macro{declaration} oder 
  außerhalb dieser mit \Macro{declaration} gesetzt wurden, werden direkt 
  nacheinander auf der gleichen Seite ausgegeben, wenn ausreichend Platz auf 
  dieser vorhanden sein sollte. Ist die Option \Option{twocolumn} aktiviert, 
  erfolgt die Ausgabe aller Erklärungen ohne Spaltenumbruch.
\item[fil/fill/vfil/vfill]
  Alle Erklärungen auf einer Ausgabeseite werden vertikal zentriert. Für 
  den zweispaltigen Satz mit \Option{twocolumn} steht diese Einstellung nicht 
  zur Verfügung.
\item[nofil/nofill/novfil/novfill]
  Die Ausgabe erfolgt wie im normalen Fließtext auch.
\end{values}
\end{Declaration}

\begin{Declaration}[v2.02]{\Environment{declarations}[\OLParameter{Sprache}]}
\begin{Declaration}[v2.04]{%
  \Macro{nextdeclaration}%
  \OLParameter{Sprache}\Parameter{Überschrift}\Parameter{Erklärung}
}
\begin{Declaration}{\Key{\Environment{declarations}}{language}[\PName{Sprache}]}
\begin{Declaration}[v2.02]{%
  \Key{\Environment{declarations}}{markboth}[\PBName{Kolumnentitel}]%
}
\begin{Declaration}[v2.02]{%
  \Key{\Environment{declarations}}{pagestyle}[\PName{Seitenstil}]%
}
\begin{Declaration}[v2.02]{%
  \Key{\Environment{declarations}}{columns}[\PName{Anzahl}]%
}
\begin{Declaration}{\Key{\Environment{declarations}}{option}[\PSet]}
\begin{Declaration}{%
  \Key{\Environment{declarations}}{supporter}[\PName{Unterstützer}]
}
\begin{Declaration}{\Key{\Environment{declarations}}{place}[\PName{Ort}]}
\begin{Declaration}{\Key{\Environment{declarations}}{closing}[\PName{Ende}]}
\begin{Declaration}{\Key{\Environment{declarations}}{company}[\PName{Firma}]}
\printdeclarationlist%
\index{Selbstständigkeitserklärung}\index{Sperrvermerk}%
%
Für Selbstständigkeitserklärung und Sperrvermerk sollten im einfachsten Fall 
die Befehle \Macro{declaration} beziehungsweise \Macro{confirmation} und 
\Macro{blocking} verwendet werden. Sobald diese jedoch in anderer Reihenfolge,  
mehrfacher Ausführung, unterschiedlichen Sprachen oder durch zusätzliche  
Erklärungen ergänzt werden, so bietet die \Environment{declarations}-Umgebung 
die notwendigen Freiheiten.

Innerhalb dieser Umgebung können Selbstständigkeitserklärung und Sperrvermerk 
mit dem Befehl \Macro{declaration} direkt nacheinander folgend beziehungsweise 
mit \Macro{confirmation} und \Macro{blocking} auch separat ausgegeben werden. 
Dies kann in beliebiger Reihenfolge und auch mehrmals geschehen, um diese 
beispielsweise mehrsprachig zu setzen.
\ChangedAt{v2.04} Des Weiteren gibt es mit \Macro{nextdeclaration} die 
Möglichkeit, eine Erklärung völlig frei zu verfassen. Dieser Befehl kann 
\emph{ausschließlich} innerhalb der \Environment{declarations}"=Umgebung 
genutzt werden, wobei im ersten Argument die gewünschte Überschrift und im 
zweiten der Inhalt respektive Text der Erklärung selbst angegeben werden muss.

Die im Folgenden beschriebenen Parameter können sowohl für die Umgebung 
\Environment{declarations} selbst als auch für die zuvor genannten Befehle als 
optionales Argument verwendet werden. Ähnlich wie die gleichnamigen Optionen 
sind auch die Umgebungen \Environment{abstract} und \Environment{declaration} 
sehr ähnlich zueinander. Deshalb werden die Erläuterungen relativ kurz 
gehalten. Ist ein Erklärung für einen Parameter etwas unverständlich, kann 
diese bei der Umgebung \Environment{abstract}'full' nachgelesen werden.

Wurde das Paket \Package{babel} geladen, kann die Sprache~-- sofern diese als 
Paketoption oder besser noch als Klassenoption angegeben wurde~-- mit dem 
Parameter \Key{\Environment{declarations}}{language}[\PName{Sprache}] für die 
\Environment{declarations}"=Umgebung geändert werden. Dadurch werden die 
Bezeichner~-- unter anderem \Term{confirmationname} und \Term{blockingname}~-- 
sowie die Trennungsmuster innerhalb der Umgebung sprachspezifisch angepasst. 

\ChangedAt{v2.02}
Die Kolumnentitel können mit \Key{\Environment{declarations}}{markboth} 
beeinflusst werden. Mit \Key{\Environment{abstract}}{markboth}[true] werden 
für diese auf linker und rechter Seite \Term{confirmationname} respektive 
\Term{blockingname} verwendet. Außerdem kann der Anwender selbige mit 
\Key{\Environment{declarations}}{markboth}[\PName{Kolumnentitel}] auch direkt 
festlegen. Sollte \Key{\Environment{declarations}}{markboth} verwendet werden, 
wird der Seitenstil automatisch auf \PageStyle{headings} gesetzt. Mit dem 
Parameter \Key{\Environment{declarations}}{pagestyle} lässt sich dieser für die 
Umgebung auch manuell angegeben. Wurde das Paket \Package{multicol} geladen, 
wird mit \Key{\Environment{declarations}}{columns}[\PName{Anzahl}] der Inhalt 
der Umgebung mehrspaltig gesetzt. Für \Key{\Environment{declarations}}{option} 
können alle gültigen Werte der Option \Option{declaration} angegeben werden. 
Die Verwendung der Parameter \Key{\Macro{confirmation}}{supporter} sowie
\Key{\Macro{confirmation}}{place} und \Key{\Macro{confirmation}}{closing} ist 
in der Dokumentation des Befehls \Macro{confirmation} zu finden, der Parameter 
\Key{\Macro{blocking}}{company} ist für \Macro{blocking} erläutert. 
\end{Declaration}
\end{Declaration}
\end{Declaration}
\end{Declaration}
\end{Declaration}
\end{Declaration}
\end{Declaration}
\end{Declaration}
\end{Declaration}
\end{Declaration}
\end{Declaration}

\begin{Declaration}{\Macro{confirmation}\OLParameter{Unterstützer}}
\begin{Declaration}{\Key{\Macro{confirmation}}{supporter}[\PName{Unterstützer}]}
\begin{Declaration}{\Key{\Macro{confirmation}}{place}[\PName{Ort}]}
\begin{Declaration}{\Key{\Macro{confirmation}}{closing}[\PName{Ende}]}
\begin{Declaration}{\Key{\Macro{confirmation}}{language}[\PName{Sprache}]}
\begin{Declaration}[v2.02]{%
  \Key{\Macro{confirmation}}{markboth}[\PBName{Kolumnentitel}]%
}
\begin{Declaration}[v2.02]{%
  \Key{\Macro{confirmation}}{pagestyle}[\PName{Seitenstil}]%
}
\begin{Declaration}[v2.02]{%
  \Key{\Macro{confirmation}}{columns}[\PName{Anzahl}]%
}
\begin{Declaration}{\Key{\Macro{confirmation}}{option}[\PSet]}
\printdeclarationlist%
\index{Selbstständigkeitserklärung}\index{Datum}%
%
Mit diesem Befehl wird ein sprachspezifischer Standardtext für eine 
Selbstständigkeitserklärung ausgegeben, welcher in \Term{confirmationtext} 
gespeichert ist. Wie dieser angepasst beziehungsweise geändert werden kann, ist 
unter \autoref{sec:localization} zu finden. Er kann sowohl innerhalb der 
\Environment{declarations}"=Umgebung als auch außerhalb dieser direkt im 
Dokument verwendet werden. 

Wird \Term{confirmationtext} nicht geändert, kann dieser über das optionale 
Argument von \Macro{confirmation} und die deklarierten Parameter angepasst 
werden. Im Standardtext der Selbstständigkeitserklärung werden sowohl der Titel 
als auch der Typ der Abschlussarbeit~-- falls dieser mit \Macro{thesis}, 
\Macro{subject}\Parameter{\autoref{tab:thesis}} beziehungsweise mit der Option 
\Option{subjectthesis} angegeben wurde~-- aufgeführt. Über den Parameter 
\Key{\Macro{confirmation}}{supporter} oder \emph{zuvor} mit dem Befehl 
\Macro{supporter} können weitere an der Arbeit beteiligte Personen angegeben 
werden. Mehrere zu nennende Personen sind auch hier durch \Macro{and} zu 
trennen. Das Feld der Unterstützer kann auch mit dem bloßen optionalen Argument 
ohne die Angabe eines Parameters angepasst werden.

Nach dem eigentlichen Text der Selbstständigkeitserklärung wird der mit 
\Key{\Macro{confirmation}}{place} beziehungsweise \Macro{place} angegebene Ort 
sowie das mit \Macro{date} eingestellte Datum ausgegeben. Als Voreinstellung 
ist für den Ort \enquote{Dresden} gewählt. Danach folgen~-- mit etwas 
vertikalem Freiraum für die notwendige Unterschrift~-- der Autor oder die 
Autoren, angegeben durch den Befehl \Macro{author}. Soll anstelle dessen etwas 
anderes nach dem Text der Selbstständigkeitserklärung gesetzt werden, kann dies 
mit dem Parameter \Key{\Macro{confirmation}}{closing} oder zuvor mit dem 
Befehl \Macro{confirmationclosing} angepasst werden. Die Parameter 
\Key{\Environment{declarations}}{language}, 
\Key{\Environment{declarations}}{markboth}, 
\Key{\Environment{declarations}}{pagestyle}, 
\Key{\Environment{declarations}}{columns} und 
\Key{\Environment{declarations}}{option} entsprechen in ihrem Verhalten denen 
der \Environment{declarations}"=Umgebung.
\end{Declaration}
\end{Declaration}
\end{Declaration}
\end{Declaration}
\end{Declaration}
\end{Declaration}
\end{Declaration}
\end{Declaration}
\end{Declaration}

\begin{Declaration}[v2.02]{\Macro{blocking}\OLParameter{Firma}}
\begin{Declaration}{\Key{\Macro{blocking}}{company}[\PName{Firma}]}
\begin{Declaration}{\Key{\Macro{blocking}}{language}[\PName{Sprache}]}
\begin{Declaration}[v2.02]{%
  \Key{\Macro{blocking}}{markboth}[\PBName{Kolumnentitel}]%
}
\begin{Declaration}[v2.02]{\Key{\Macro{blocking}}{pagestyle}[\PName{Seitenstil}]}
\begin{Declaration}[v2.02]{\Key{\Macro{blocking}}{columns}[\PName{Anzahl}]}
\begin{Declaration}{\Key{\Macro{blocking}}{option}[\PSet]}
\printdeclarationlist%
\index{Sperrvermerk}%
%
Beim Sperrvermerk verhält es sich äquivalent zur Selbstständigkeitserklärung.
Es wird der in \Term{blockingtext} hinterlegte Standardtext in der gewählten 
Sprache ausgegeben. Dieser kann durch den Anwender geändert werden. Wie genau 
ist in \autoref{sec:localization} beschrieben. Der Befehl \Macro{blocking} 
kann sowohl innerhalb der Umgebung \Environment{declarations} als auch 
außerhalb direkt im Dokument verwendet werden. 

In seiner ursprünglichen Definition, kann er im optionalen Argument über die 
deklarierten Parameter angepasst werden. Im Standardtext des Sperrvermerks 
werden sowohl der Titel als auch der Typ der Abschlussarbeit~-- falls dieser 
mit \Macro{thesis}, \Macro{subject}\Parameter{\autoref{tab:thesis}} respektive  
mit der Option \Option{subjectthesis} angegeben wurde~-- aufgeführt. Mit 
\Key{\Macro{blocking}}{company} oder \emph{vorher} mit \Macro{company} kann 
zusätzlich eine im Sperrvermerk zu nennende Firma oder ähnliches angegeben 
werden. Dieses Feld kann auch direkt im optionalen Argument ohne die Verwendung 
eines Parameters gesetzt werden. Die weiteren Parameter 
\Key{\Environment{declarations}}{language}, 
\Key{\Environment{declarations}}{markboth}, 
\Key{\Environment{declarations}}{pagestyle}, 
\Key{\Environment{declarations}}{columns} und 
\Key{\Environment{declarations}}{option} entsprechen in ihrem Verhalten denen 
der \Environment{declarations}"=Umgebung.
\end{Declaration}
\end{Declaration}
\end{Declaration}
\end{Declaration}
\end{Declaration}
\end{Declaration}
\end{Declaration}

\begin{Declaration}{\Macro{declaration}\LParameter}
\begin{Declaration}{\Key{\Macro{declaration}}{language}[\PName{Sprache}]}
\begin{Declaration}[v2.02]{%
  \Key{\Macro{declaration}}{markboth}[\PBName{Kolumnentitel}]%
}
\begin{Declaration}[v2.02]{%
  \Key{\Macro{declaration}}{pagestyle}[\PName{Seitenstil}]%
}
\begin{Declaration}[v2.02]{\Key{\Macro{declaration}}{columns}[\PName{Anzahl}]}
\begin{Declaration}{\Key{\Macro{declaration}}{option}[\PSet]}
\begin{Declaration}{\Key{\Macro{declaration}}{supporter}[\PName{Unterstützer}]}
\begin{Declaration}{\Key{\Macro{declaration}}{place}[\PName{Ort}]}
\begin{Declaration}{\Key{\Macro{declaration}}{closing}[\PName{Ende}]}
\begin{Declaration}{\Key{\Macro{declaration}}{company}[\PName{Firma}]}
\printdeclarationlist%
\index{Selbstständigkeitserklärung}\index{Sperrvermerk}%
%
Dieser Befehl gibt die Selbstständigkeitserklärung und den Sperrvermerk direkt 
aufeinanderfolgend aus. Dabei werden die Einstellungen zur Positionierung der 
einzelnen Erklärungen, welche über die Wertzuweisungen an die Option 
\Option{declaration}[simple/multiple/fill/nofill] erfolgen, beachtet. Er kann 
sowohl innerhalb der \Environment{declarations}"=Umgebung als auch außerhalb 
direkt im Dokument verwendet werden und akzeptiert im optionalen Argument dabei 
alle für die \Environment{declarations}"=Umgebung beschriebenen Parameter.
\end{Declaration}
\end{Declaration}
\end{Declaration}
\end{Declaration}
\end{Declaration}
\end{Declaration}
\end{Declaration}
\end{Declaration}
\end{Declaration}
\end{Declaration}

\begin{Declaration}{\Macro{supporter}\Parameter{Unterstützer}}
\begin{Declaration}{\Macro{place}\Parameter{Ort}}
\begin{Declaration}{\Macro{confirmationclosing}\Parameter{Ende}}
\begin{Declaration}{\Macro{company}\Parameter{Firma}}
\printdeclarationlist%
\index{Selbstständigkeitserklärung}\index{Sperrvermerk}%
%
Diese Makros ändern~-- im Gegensatz zu den Parametern der bereits vorgestellten 
Befehle \Macro{confirmation} und \Macro{blocking}~-- die entsprechenden 
Feldwerte für das gesamte Dokument. Genutzt werden kann dies beispielsweise 
wenn ein Erklärungstyp in unterschiedlichen Sprachen ausgegeben wird. Hiermit 
kann man sich die mehrfache Angabe eines Parameters sparen.
\end{Declaration}
\end{Declaration}
\end{Declaration}
\end{Declaration}


\subsection{Fußnoten in Überschriften}
\begin{Declaration}[%
  v2.02!Fußnoten in Überschriften können mit Symbolen gesetzt werden%
]{\Option{footnotes}[\PSet]}[nosymbolheadings]%
\begin{Declaration}[v2.02]{\Counter{symbolheadings}}%
\printdeclarationlist%
\index{Überschriften!Fußnoten}\index{Fußnoten}%
%
\ToDo[imp]{Fehler mit \Macro{addchap} beheben, Paket \Package{footmisc}}[v2.06]
\ToDo[imp]{Zähler auch bei Sternversionen von Kapiteln zurücksetzen}[v2.06]
\ToDo[imp]{Fußnoten nicht ins Inhaltsverzeichnis?}[v2.06]
\ToDo[imp]{Problem mit \Package{hyperref} lösbar?}[v2.06]
Für die Überschriften wird die \KOMAScript-Option \Option{footnotes} erweitert.
Normalerweise kann diese die Werte \PValue{multiple} und \PValue{nomultiple} 
annehmen, wobei Letzteres der Standardfall ist. Die \TUDScript-Hauptklassen 
erweitern die Option dahingehend, dass auf die Verwendung von Symbolen anstelle 
von Zahlen innerhalb der Überschriften umgeschaltet werden kann. Hierfür wird 
der Zähler \Counter{symbolheadings} definiert, der mit dem Beginn eines neuen 
Kapitels zurückgesetzt wird.
%
\begin{values}
\item[nosymbolheadings/numberheadings]
  Die Fußnoten der Überschriften werden fortlaufend mit denen des Fließtextes 
  gesetzt.
\item[symbolheadings]
  Für die Überschriften werden symbolische Fußnoten mit einem eigenen Zähler 
  verwendet.
\end{values}
\end{Declaration}
\end{Declaration}


\subsection{Lesezeichen}
\begin{Declaration}{\Option{tudbookmarks}[\PBoolean]}[true]%
\printdeclarationlist%
\index{Lesezeichen}%
\index{Titel}\index{Umschlagseite}\index{Inhaltsverzeichnis}%
\index{Aufgabenstellung}\index{Gutachten}\index{Aushang}%
%
Diese Option wird wirksam, wenn \Package{hyperref} geladen wurde. Es werden für 
die Umschlag- und Titelseite, das Inhaltsverzeichnis sowie~-- bei der 
Verwendung des Paketes \Package{tudscrsupervisor}~-- die Aufgabenstellung 
Lesezeichen oder auch Outline"=Einträge im PDF-Dokument erzeugt.
%
\begin{values}
\itemfalse
  Es erfolgt kein Eintrag von ergänzenden Lesezeichen.
\itemtrue*
  Es werden automatisch zusätzliche Lesezeichen eingetragen.
\end{values}
\end{Declaration}

\begin{Declaration}{%
  \Macro{tudbookmark}\OParameter{Ebene}\Parameter{Text}\Parameter{Ankername}%
}%
\printdeclarationlist%
%
Der Befehl \Macro{tudbookmark} arbeitet wie \Macro{pdfbookmark} aus 
\Package{hyperref} mit dem Unterschied, dass die Lesezeichen nur generiert 
werden, wenn die Option \Option{tudbookmarks} aktiviert ist.
\end{Declaration}



\section{Sprachabhängige Bezeichner}
\label{sec:localization}
\index{Bezeichner|!(}%
%
Durch \KOMAScript{} werden Befehle, mit denen sprachabhängige Bezeichner 
erzeugt oder geändert werden können, zur Verfügung gestellt. Diese werden durch 
\TUDScript genutzt, um lokalisierte Begriffe für die Sprachen Englisch und 
Deutsch bereitzustellen. Ein Großteil davon betrifft Bezeichnungen für Felder 
auf der Titelseite (\autoref{sec:title}). Hierfür wird
\Macro{providecaptionname}\Parameter{Sprache}\Parameter{Makro}\Parameter{Inhalt}
verwendet, wobei \PName{Sprache} dem geladenen Sprachpaket~-- normalerweise das 
Paket \Package{babel}~-- bekannt sein muss.

Sollte der Anwender die im Folgenden erläuterten oder auch andere Bezeichner, 
welche von einem beliebigen (Sprach"~)Paket bereitgestellt werden, ändern 
wollen, ist hierfür der Befehl
\Macro{renewcaptionname}\Parameter{Sprache}\Parameter{Makro}\Parameter{Inhalt} 
zu verwenden. Es sollte natürlich dabei eine \PName{Sprache} angegeben werden, 
welche im Dokument durch \Package{babel} oder ein anderes Sprachpaket verwendet 
wird, beispielsweise \PValue{ngerman} oder \PValue{english}. 

Die Makros der Bezeichner und deren Verwendung werden folgend kurz beschrieben 
und tabellarisch aufgeführt. Dabei wurde versucht, alle Befehle der Bezeichner 
für bestimmte Begriffe auf \PValue{\dots{}name} und beschreibende Texte auf 
\PValue{\dots{}text} enden zu lassen.

\begin{Declaration}{\Term{supervisorname}}
\begin{Declaration}{\Term{supervisorothername}}
\begin{Declaration}[%
  v2.02!Unterscheidung von einem und mehreren Gutachtern%
]{\Term{refereename}}
\begin{Declaration}{\Term{refereeothername}}
\begin{Declaration}{\Term{advisorname}}
\begin{Declaration}{\Term{advisorothername}}
\begin{Declaration}[%
  v2.02!Unterscheidung von einem und mehreren Professoren%
]{\Term{professorname}}
\begin{Declaration}[v2.02]{\Term{professorothername}}
\printdeclarationlist%
\index{Titel}%
\index{Betreuer}\index{Gutachter}\index{Hochschullehrer}%
\index{Referent}%
%
Diese sprachabhängigen Begriffe sind die Bezeichner für die Titelseitenfelder 
von Betreuer (\Macro{supervisor}), Gutachter (\Macro{referee}) und Fachreferent 
(\Macro{advisor}). Soll innerhalb eines dieser Felder mehr als eine Person 
angegeben werden, so sind die Einzelpersonen jeweils mit dem Befehl \Macro{and} 
voneinander zu trennen. In diesem Fall werden alle nach der erstgenannten 
folgenden Personen durch den Bezeichner \PValue{\textbackslash\dots{}othername} 
ergänzt.

\ChangedAt{v2.02}
Bei der Bezeichnung des Gutachters wird unterschieden, ob einer oder mehrere 
angegeben wurden. Wird lediglich einer genannt, so ist eine Unterscheidung 
nicht notwendig. Werden jedoch zwei Gutachter angegeben, so werden diese auch 
mit Erst- und Zweitgutachter betitelt. Für den betreuenden Hochschullehrer 
(\Macro{professor}) wird ähnlich verfahren. Hier wird allerdings lediglich 
die Bezeichnung vom Singular in den Plural gegebenenfalls automatisch geändert.


\renewcaptionname{ngerman}{\refereename}{Gutachter/Erstgutachter}
\renewcaptionname{english}{\refereename}{Referee/First referee}
\TermTable{%
  supervisorname,supervisorothername,refereename,refereeothername,%
  advisorname,advisorothername,professorname,professorothername%
}
\end{Declaration}
\end{Declaration}
\end{Declaration}
\end{Declaration}
\end{Declaration}
\end{Declaration}
\end{Declaration}
\end{Declaration}

\begin{Declaration}{\Term{dissertationname}}
\begin{Declaration}{\Term{diplomathesisname}}
\begin{Declaration}{\Term{masterthesisname}}
\begin{Declaration}{\Term{bachelorthesisname}}
\begin{Declaration}{\Term{studentresearchname}}
\begin{Declaration}{\Term{projectpapername}}
\begin{Declaration}{\Term{seminarpapername}}
\begin{Declaration}{\Term{researchname}}
\begin{Declaration}{\Term{logname}}
\begin{Declaration}{\Term{internshipname}}
\begin{Declaration}{\Term{reportname}}
\printdeclarationlist%
\index{Titel}\index{Abschlussarbeit}\index{Typisierung}%
%
Diese Bezeichner dienen zur Typisierung speziell für eine Abschlussarbeit. Wie 
diese genutzt werden können, ist bei der Erläuterung von \Macro{thesis} und 
\Macro{subject}'full' beziehungsweise in \autoref{tab:thesis} zu finden.
\TermTable{%
  dissertationname,diplomathesisname,masterthesisname,bachelorthesisname,%
  studentresearchname,projectpapername,seminarpapername,researchname,%
  logname,internshipname,reportname%
}
\end{Declaration}
\end{Declaration}
\end{Declaration}
\end{Declaration}
\end{Declaration}
\end{Declaration}
\end{Declaration}
\end{Declaration}
\end{Declaration}
\end{Declaration}
\end{Declaration}

\begin{Declaration}{\Term{dateofbirthtext}}
\begin{Declaration}{\Term{placeofbirthtext}}
\begin{Declaration}{\Term{matriculationnumbername}}
\begin{Declaration}{\Term{matriculationyearname}}
\printdeclarationlist%
\index{Titel}\index{Autorenangaben}\index{Datum!Geburtsdatum}%
%
Werden für den Autor oder die Autoren das Geburtsdatum (\Macro{dateofbirth}), 
der Geburtsort (\Macro{placeofbirth}) sowie die
Matrikelnummer (\Macro{matriculationnumber}) und/oder das Immatrikulationsjahr 
(\Macro{matriculationyear}) angegeben, werden sowohl auf der Titelseite als 
auch auf der gegebenenfalls mit \Package{tudscrsupervisor} erstellten 
Aufgabenstellung die dazugehörigen Bezeichner vorangestellt. Auf dem Titel 
werden diese dabei mit dem durch \Macro{titledelimiter} gegebenen Trennzeichen 
vom eigentlichen Feld abgegrenzt.
\TermTable{%
  dateofbirthtext,placeofbirthtext,matriculationnumbername,%
  matriculationyearname%
}
\end{Declaration}
\end{Declaration}
\end{Declaration}
\end{Declaration}

\begin{Declaration}[v2.02]{\Term{graduationtext}}
\printdeclarationlist%
\index{Titel}\index{Abschlussarbeit}\index{Typisierung}%
%
Wurde erkannt, dass das Dokument eine Abschlussarbeit ist,%
\footnote{%
  Entweder wurde \Macro{thesis} oder \Macro{subject} mit einem speziellen Wert 
  oder der Option \Option{subjectthesis} verwendet.
}
so kann der zu erlangende akademische Grad mit dem Befehl \Macro{graduation} 
angegeben werden. Bei dessen Ausgabe auf dem Titel wird dabei der entsprechende 
Text dazu angegeben.
\TermTable*{graduationtext}{.78\textwidth}
\end{Declaration}

\begin{Declaration}{\Term{datetext}}
\begin{Declaration}{\Term{defensedatetext}}
\printdeclarationlist%
\index{Titel}\index{Abschlussarbeit}%
\index{Datum}\index{Datum!Verteidigungsdatum}%
%
Wird mit \Macro{date} das Datum und mit \Macro{defensedate} ein Datum der 
Verteidigung für eine Abschlussarbeit angegeben, so werden auch diese Felder 
durch einen einleitenden Text beschrieben.
\TermTable{datetext,defensedatetext}
\end{Declaration}
\end{Declaration}

\begin{Declaration}{\Term{abstractname}}
\printdeclarationlist%
%
Dieser Bezeichner wird lediglich für \Class{tudscrbook} definiert, da dieser 
von \KOMAScript{} für die Buchklasse nicht vorgesehen wird.
\TermTable{abstractname}
\end{Declaration}

\begin{Declaration}{\Term{confirmationname}}
\begin{Declaration}[v2.02]{\Term{blockingname}}
\printdeclarationlist%
\index{Selbstständigkeitserklärung}\index{Sperrvermerk}%
%
Es werden die Bezeichnungen für Selbstständigkeitserklärung und Sperrvermerk 
für die dazugehörigen Überschriften definiert.
\TermTable{confirmationname,blockingname}
\end{Declaration}
\end{Declaration}

\begin{Declaration}{\Term{confirmationtext}}
\begin{Declaration}[v2.02]{\Term{blockingtext}}
\printdeclarationlist%
%
Die Texte der Erklärungen selbst sind derart aufgebaut, dass sie in 
Abhängigkeit von den angegebenen Informationen unterschiedlich ausgeführt 
werden. Innerhalb der Selbstständigkeitserklärung (\Macro{confirmation}) werden 
gegebenenfalls die Felder für den Titel (\Macro{title}) und die Typisierung der 
Abschlussarbeit%
\footnote{%
  entweder \Macro{thesis} oder \Macro{subject}\Parameter{\autoref{tab:thesis}}
  beziehungsweise Option \Option{subjectthesis}[true]
}
sowie die angegebenen Unterstützer%
\footnote{%
  \Macro{confirmation}\POParameter{\Key{\Macro{confirmation}}{supporter}=\dots}
  oder \Macro{supporter}\PParameter{\dots}%
}
beachtet. Für den Sperrvermerk (\Macro{blocking}) wird neben dem Titel 
(\Macro{title}) optional außerdem noch das Feld der externen Firma%
\footnote{%
  \Macro{blocking}\POParameter{\Key{\Macro{blocking}}{company}=\dots} oder 
  \Macro{company}\PParameter{\dots}%
}
verwendet. Der Vollständigkeit halber werden im Folgenden noch die Texte für 
die Selbstständigkeitserklärung und den Sperrvermerk aufgeführt~-- allerdings 
lediglich die deutschsprachige Version. Dabei werden alle möglichen Felder 
angezeigt.

\begingroup
  \makeatletter
  \def\@@title{\PName{Titel}}
  \def\@@thesis{\PName{Abschlussarbeit}}
  \def\@supporter{\PName{Vorname Nachname} \and \PName{Vorname Nachname}}
  \def\@company{\PName{Firma}}
  \makeatother
  \vskip\baselineskipglue\noindent
  \textbf{Bezeichner}\quad\Term*{confirmationtext}%
  \begin{quoting}
  \confirmationtext
  \end{quoting}
  \textbf{Bezeichner}\quad\Term*{blockingtext}%
  \begin{quoting}
  \blockingtext
  \end{quoting}
\endgroup
\end{Declaration}
\end{Declaration}

\begin{Declaration}{\Term{coverpagename}}
\begin{Declaration}{\Term{titlepagename}}
\printdeclarationlist%
\index{Lesezeichen}\index{Titel}\index{Umschlagseite}%
%
Diese beiden Bezeichner werden bei aktivierter \Option{tudbookmarks} für das 
Eintragen von Lesezeichen in ein PDF"=Dokument genutzt.
\TermTable{coverpagename,titlepagename}
\end{Declaration}
\end{Declaration}

\begin{Declaration}{\Term{listingname}}
\begin{Declaration}{\Term{listlistingname}}
\printdeclarationlist%
%
Sollte ein Paket zur Einbindung von externem Quelltext~-- beispielsweise 
das Paket \Package{listings}~-- verwendet werden, so werden diese Bezeichnungen 
für Quelltextausschnitte und das Quelltextverzeichnis verwendet.
\TermTable{listingname,listlistingname}
\end{Declaration}
\end{Declaration}
\index{Bezeichner|!)}

\section{Kompatibilitätseinstellungen zu früheren Versionen}
Bei der Entwicklung von \TUDScript lässt es sich nicht immer vermeiden, dass 
Verbesserungen sowie Korrekturen an den Klassen und Paketen zu Änderungen am 
Ergebnis der Ausgabe führen, insbesondere bei Umbruch und Layout. Für bereits
archivierte Dokumente, welche mit einer früheren Version erstellt wurden ist 
dies jedoch bei einer erneuten Kompilierung unter Umständen eher unerwünscht.

\begin{Declaration}[v2.03]{\Option{tudscrver}[%
  \PName{Version}\textOR\PValue{first}\textOR\PValue{last}%
]}[last]
\printdeclarationlist%
\index{Kompatibilität|!}%
%
Mit dieser Option wird es möglich, auf das (Umbruch-)Verhalten einer älteren 
respektive früheren Version von \TUDScript umzuschalten, um nach der 
Kompilierung das erwartete Ergebnis zu erhalten. Neue Möglichkeiten, die sich 
nicht auf den Umbruch oder das Layout auswirken, sind auch für den Fall 
verfügbar, dass per Option die Kompatibilität zu einer älteren Version 
ausgewählt wurde. 

Bei der Angabe einer unbekannten Version als Wert wird eine Warnung ausgegeben 
und \Option{tudscrver}[first] angenommen. Mit \Option{tudscrver}[last] wird die 
jeweils aktuell verfügbare Version ausgewählt und folglich auf die zukünftige 
Kompatibilität des Dokumentes zu der aktuell genutzten Version verzichtet. 
Dieses Verhalten entspricht der Voreinstellung. Es ist zu beachten, dass die 
Nutzung von \Option{tudscrver} nur als Klassenoption möglich ist.
%
\begin{values}
\item[\PValue{2.02}\textOR\PValue{first}]
  \ChangedAt{%
    v2.03!Satzspiegel im \CD geändert{,} \protect\DDC-Logo im Fußbereich 
    wird ohne vergrößerten Seitenrand verwendet
  }
  Der Satzspiegel im Layout des \CDs (\see*{\Option{cdgeometry}}) wurde in der 
  Version~v2.03 leicht geändert. Der obere Seitenrand wurde verkleinert, der 
  untere im gleichen Maße vergrößert. Der verfügbare Textbereich ist folglich 
  identisch. Bei der Aktivierung des \DDC-Logos im Fußbereich der Seite
  (\see*{\Option{ddcfoot}}) wird im Gegensatz zur Version~v2.02 der gleiche 
  Satzspiegel genutzt. Mit \Option{tudscrver}[2.02] kann dieses Verhalten 
  deaktiviert werden.
\item[\PValue{2.03}]
  \ChangedAt{%
    v2.04!Werte bestimmter Längen abhängig von der verwendeten Schriftgröße%
  }\index{Schriftgröße}%
  Seit der Version~v2.04 werden mehrere Längen in Abhängigkeit der gewählten 
  Schriftgröße (Option \Option{fontsize}) definiert. Dies betrifft sowohl die 
  dehnbaren Längen \Length{smallskipamount}, \Length{medskipamount} und 
  \Length{bigskipamount}, die von den Befehlen \Macro{smallskip},   
  \Macro{medskip} sowie \Macro{bigskip} für das Einfügen vertikaler Abstände 
  genutzt werden, als auch die beiden Längen \Length{abovecaptionskip} und 
  \Length{belowcaptionskip} für den Abstand zwischen einem Gleitobjekt und 
  dessen mit \Macro{caption} gesetzten Beschreibung sowie \Length{columnsep} 
  als Maß für den Abstand der einzelnen Textspalten im zwei- oder mehrspaltigen 
  Layout. Mit der Wahl \Option{tudscrver}[2.03] lässt sich diese Funktionalität 
  deaktivieren.
\item[\PValue{2.04}]
  \ChangedAt{%
    v2.05!Einstellungen für den Satzspiegel für die jeweilige ISO/DIN-Klasse 
    des verwendeten Papierformates identisch%
  }\index{Satzspiegel}%
  Mit der Version~v2.05 werden die vorgegebenen Einstellungen zum Satzspiegel 
  anhand der B-ISO/DIN-Reihe vorgenommen. Damit sind für alle Papierformate 
  einer spezifischen ISO/DIN-Klasse die Seitenränder identisch. Mit der Wahl 
  \Option{tudscrver}[2.04] ist der Satzspiegel von der A-ISO/DIN-Reihe 
  abhängig, sodass die B- und C-Papierformate der gleichen Klasse größere 
  Seitenränder erhalten, als die D- und A-Formate.
\item[\PValue{2.05}]
  Dies ist Kompatibilitätseinstellung für \TUDScript~\vTUDScript{} und wird für 
  zukünftige Änderungen bereits vorgehalten. Soll ein mit der momentan 
  aktuellen Version erzeugtes Dokument auch mit einer späteren Version von 
  \TUDScript nach einem \hologo{LaTeX}-Lauf das gleiche Ausgabeergebnis 
  liefern, muss dies mit \Option{tudscrver}[2.05] angegeben werden.
\item[\PValue{last}]
  Es werden keine Kompatibilitätseinstellungen für das Dokument vorgenommen. 
  Mit einer späteren Version von \TUDScript kann ein anderes Umbruchverhalten 
  innerhalb des Dokumentes auftreten. Dies ist die Standardeinstellung.
\end{values}
\end{Declaration}

% \CheckSum{168}
% \iffalse meta-comment
% 
% ============================================================================
% 
%  TUD-KOMA-Script
%  Copyright (c) Falk Hanisch <tudscr@gmail.com>, 2012-2015
% 
% ============================================================================
% 
%  This work may be distributed and/or modified under the conditions of the
%  LaTeX Project Public License, version 1.3c of the license. The latest
%  version of this license is in http://www.latex-project.org/lppl.txt and 
%  version 1.3c or later is part of all distributions of LaTeX 2005/12/01
%  or later and of this work. This work has the LPPL maintenance status 
%  "author-maintained". The current maintainer and author of this work
%  is Falk Hanisch.
% 
% ----------------------------------------------------------------------------
% 
% Dieses Werk darf nach den Bedingungen der LaTeX Project Public Lizenz
% in der Version 1.3c, verteilt und/oder veraendert werden. Die aktuelle 
% Version dieser Lizenz ist http://www.latex-project.org/lppl.txt und 
% Version 1.3c oder spaeter ist Teil aller Verteilungen von LaTeX 2005/12/01 
% oder spaeter und dieses Werks. Dieses Werk hat den LPPL-Verwaltungs-Status 
% "author-maintained", wird somit allein durch den Autor verwaltet. Der 
% aktuelle Verwalter und Autor dieses Werkes ist Falk Hanisch.
% 
% ============================================================================
%
% \fi
%
% \CharacterTable
%  {Upper-case    \A\B\C\D\E\F\G\H\I\J\K\L\M\N\O\P\Q\R\S\T\U\V\W\X\Y\Z
%   Lower-case    \a\b\c\d\e\f\g\h\i\j\k\l\m\n\o\p\q\r\s\t\u\v\w\x\y\z
%   Digits        \0\1\2\3\4\5\6\7\8\9
%   Exclamation   \!     Double quote  \"     Hash (number) \#
%   Dollar        \$     Percent       \%     Ampersand     \&
%   Acute accent  \'     Left paren    \(     Right paren   \)
%   Asterisk      \*     Plus          \+     Comma         \,
%   Minus         \-     Point         \.     Solidus       \/
%   Colon         \:     Semicolon     \;     Less than     \<
%   Equals        \=     Greater than  \>     Question mark \?
%   Commercial at \@     Left bracket  \[     Backslash     \\
%   Right bracket \]     Circumflex    \^     Underscore    \_
%   Grave accent  \`     Left brace    \{     Vertical bar  \|
%   Right brace   \}     Tilde         \~}
%
% \iffalse
%%% From File: tudscr-poster.dtx
%<*driver>
\ifx\ProvidesFile\undefined\def\ProvidesFile#1[#2]{}\fi
\ProvidesFile{tudscr-poster.dtx}[%
  2015/07/03 v2.05 TUD-KOMA-Script (corporate design posters)%
]
\RequirePackage[ngerman=ngerman-x-latest]{hyphsubst}
\documentclass[english,ngerman]{tudscrdoc}
\usepackage{selinput}\SelectInputMappings{adieresis={ä},germandbls={ß}}
\usepackage[T1]{fontenc}
\usepackage{babel}
\usepackage{tudscrfonts} % only load this package, if the fonts are installed
\KOMAoptions{parskip=half-}
\CodelineIndex
\RecordChanges
\GetFileInfo{tudscr-poster.dtx}
\begin{document}
  \maketitle
  \DocInput{\filename}
\end{document}
%</driver>
% \fi
%
% \selectlanguage{ngerman}
%
% \section{Poster}
%
% Diese Klasse stellt auf Basis von \cls{tudscrartcl} das Layout für ein Poster
% im \CD der \TnUD zur Verfügung. dabei gibt es Einstellungen und Optionen, die 
% sich in ihrer Handhabung mit den Hauptklassen überschneiden. Diese sind in 
% den entsprechenden Quelldateien zu finden.
%
% \StopEventually{\PrintIndex\PrintChanges}
%
% \iffalse
%<*class&body>
% \fi
%
% \subsection{Die Klasse \cls{tudscrposter}}
%
% \begin{macro}{\tud@cdstyle@set}
% \changes{v2.04}{2015/05/18}{neu}^^A
% Mit diesem Makro erfolgt die Zuweisung des Seitenstils. Dieser wird über die 
% Option \opt{cdstyle} gesetzt, welche in \file{tudscr-layout.dtx} definiert 
% wird.
%    \begin{macrocode}
\newcommand*\tud@cdstyle@set{%
  \ifcase\tud@cdstyle\relax% false
    \footcontent{}%
  \else% !false
    \pagestyle{empty.tudheadings}%
    \footcontent{\tud@foot@poster@left}[\tud@foot@poster@right]%
    \ifcase\tud@cdstyle\relax\or% true
      \TUDoptions{cdhead=nocolor,cdfoot=true}%
    \or% litecolor
      \TUDoptions{cdhead=litecolor,cdfoot=true}%
    \or% barcolor
      \TUDoptions{cdhead=barcolor,cdfoot=true}%
    \else% bicolor/color/full
      \TUDoptions{cdhead=bicolor,cdfoot=bicolor}%
    \fi%
  \fi%
}
\AtBeginDocument{\tud@cdstyle@set}
%    \end{macrocode}
% \end{macro}^^A \tud@cdstyle@set
% \begin{macro}{\tud@foot@line@add}
% \changes{v2.04}{2015/05/12}{neu}^^A
% \begin{macro}{\tud@foot@line@write}
% \changes{v2.04}{2015/05/12}{neu}^^A
% Mit \cs{tud@foot@line@add} wird der Inhalt eines Feldes in \cs{@\meta{Feld}} 
% gespeichert. Der Befehl erwartet als erstes obligatorisches Argument den
% Feldnamen und als zweites den Inhalt. Entspricht das dritte obligatorische 
% Argument \cs{@empty}, so wird in \cs{@\meta{Feld}@foot} ebenfalls das zweite 
% Argument abgelegt, andernfalls das dritte.
%
% Damit wird es für Poster möglich, die Befehle \cs{faculty}, \cs{department}, 
% \cs{institute}, \cs{chair} und \cs{professor} dahingehend zu erweitern, dass 
% unterschiedliche Angaben für die Kopf- und Fußzeile gemacht werden können.
% Wird eines der zuvor genannten Makros lediglich mit einem obligatorischen
% Argument verwendet, so enthalten Kopf und Fuß den gleichen Eintrag. Wird
% jedoch zusätzlich das optionale Argument genutzt, so wird dessen Inhalt im 
% Fußbereich mit \cs{tud@foot@line@write} ausgegeben.
%    \begin{macrocode}
\newcommand*\tud@foot@line@add[3]{%
  \csgdef{@#1}{\trim@spaces{#2}}%
  \ifx#3\@empty\relax%
    \global\csletcs{@#1@foot}{@#1}%
  \else%
    \csgdef{@#1@foot}{\trim@spaces{#3}}%
  \fi%
}
\newcommand*\tud@foot@line@write[1]{%
  \protected@edef\@tempa{\csuse{@#1@foot}}%
  \ifx\@tempa\@empty\else\newline{\csuse{@#1@foot}}\fi%
}
%    \end{macrocode}
% \end{macro}^^A \tud@foot@line@write
% \end{macro}^^A \tud@foot@line@add
% \begin{macro}{\tud@foot@poster@left}
% \changes{v2.04}{2015/05/12}{neu}^^A
% \begin{macro}{\tud@foot@poster@right}
% \changes{v2.04}{2015/05/12}{neu}^^A
% \begin{macro}{\tud@newline}
% \begin{macro}{\tud@split@author}
% \begin{macro}{\tud@split@contactperson}
% Mit diesen beiden Hilfsmakros werden die linke und die rechte Spalte des 
% Standard-Seitenfußes eines Posters festgelegt. In der linken Spalte werden 
% dabei Fakultät, Einrichtung, Institut und Lehrstuhl sowie der Professor 
% ausgegeben, wobei die Angaben über das optionale Argument der entsprechenden 
% Feldbefehle, die in den Klassen normalerweise nur für den Seitenkopf genutzt 
% werden, variiert werden können.
%    \begin{macrocode}
\newcommand*\tud@foot@poster@left{%
  \ifx\contactname\@empty\else{\tud@head@font@bold\contactname}\newline\fi%
  Technische Universit\"at Dresden%
  \tud@foot@line@write{faculty}%
  \tud@foot@line@write{department}%
  \tud@foot@line@write{institute}%
  \tud@foot@line@write{chair}%
  \tud@foot@line@write{professor}%
}
%    \end{macrocode}
% In der rechten Spalte werden der Autor oder die Autoren (\cs{author}) und 
% die Kontaktperson(en) (\cs{contactperson}) ausgegeben. Zu jeder Person können
% individuelle Angaben bzgl. Büro, Telefonnummer und E-Mail-Adresse gemacht 
% werden. 
%    \begin{macrocode}
\newcommand*\tud@foot@poster@right{%
  \def\tud@newline{%
    \ifx\@office\@empty\else\newline\@office\fi%
    \ifx\@telephone\@empty\else\newline\@telephone\fi%
    \ifx\@emailaddress\@empty\else\newline\@emailaddress\fi%
  }%
%    \end{macrocode}
% Wurde kein Autor angegeben, wird in diesem Fall die normalerweise erzeugte 
% Warnung bei der Verwendung des Feldes \cs{@author} unterdrückt.
%    \begin{macrocode}
  \ifpatchable{\@author}{\@latex@warning@no@line}{%
    \let\@tempa\@empty%
  }{%
    \let\@tempa\@author%
  }%
%    \end{macrocode}
% Der temporäre Schalter wird verwendet, um die gleichzeitige Angabe von Autor 
% und Kontaktperson zu erkennen und zwischen den Angaben eine Leerzeile 
% einzufügen.
%    \begin{macrocode}
  \@tempswafalse%
  \ifx\@tempa\@empty\else%
    \ifx\authorname\@empty\else%
      {\tud@head@font@bold\authorname}\newline%
    \fi%
%    \end{macrocode}
% Das Makro zum Aufteilen der Autorenangaben wird für die hier benötigte Form 
% definiert. Dabei wird die Ausgabe aller nicht \emph{lokal} angegebenen Felder
% unterdrückt, indem der Befehl \cs{tud@multiple@fields@preset} im zweiten 
% Argument mit einem \val{*} aufgerufen wird.
%    \begin{macrocode}
    \renewcommand*\tud@split@author[2]{%
      \tud@multiple@fields@store{@author}{##1}%
      \tud@multiple@fields@preset{@author}{*}{##1}%
      \ignorespaces##1\tud@newline%
      \tud@multiple@fields@restore{@author}%
      \tud@multiple@@@split{##2}{\newline}%
    }%
    \noindent\tud@multiple@split{@author}%
    \tud@multiple@fields@restore{@author}%
%    \end{macrocode}
% Wurde gültige Felder außerhalb von \cs{@author} global angegeben, so werden 
% diese \emph{nach} allen Autoren ausgegeben.
%    \begin{macrocode}
    \tud@newline%
    \@tempswatrue%
  \fi%
%    \end{macrocode}
% Die Ausgabe der Kontaktperson(en) erfolgt analog zu der Autorenausgabe.
%    \begin{macrocode}
  \ifx\@contactperson\@empty\else%
    \if@tempswa\newline\fi%
    \ifx\contactpersonname\@empty\else%
      {\tud@head@font@bold\contactpersonname}\newline%
    \fi%
    \renewcommand*\tud@split@contactperson[2]{%
      \tud@multiple@fields@store{@contactperson}{##1}%
      \tud@multiple@fields@preset{@contactperson}{*}{##1}%
      \ignorespaces##1\tud@newline%
      \tud@multiple@fields@restore{@contactperson}%
      \tud@multiple@@@split{##2}{\newline}%
    }%
    \noindent\tud@multiple@split{@contactperson}%
    \tud@multiple@fields@restore{@contactperson}%
    \tud@newline%
  \fi%
%    \end{macrocode}
% Zu guter letzt noch eine mögliche Homepage.
%    \begin{macrocode}
  \ifx\@webpage\@empty\else\newline\@webpage\fi%
}
%    \end{macrocode}
% \end{macro}^^A \tud@split@contactperson
% \end{macro}^^A \tud@split@author
% \end{macro}^^A \tud@newline
% \end{macro}^^A \tud@foot@poster@right
% \end{macro}^^A \tud@foot@poster@left
% \begin{macro}{\tud@split@author@list}
% Der Befehl \cs{tud@split@author@list} wird um die in der Poster-Klasse 
% \cls{tudscrposter} zusätzlich definierten Felder erweitert.
%    \begin{macrocode}
\patchcmd{\tud@split@author@list}{authormore}{%
  authormore,office,telephone,emailaddress%
}{}{\tud@patch@wrn{tud@split@author@list}}
%    \end{macrocode}
% \end{macro}^^A \tud@split@author@list
%
% \iffalse
%</class&body>
% \fi
%
% \Finale
%
\endinput

\setchapterpreamble{%
  \begin{abstract}
    Zusätzlich zu den bisher im Anwenderhandbuch vorgestellten Klassen und 
    Paketen werden im \TUDScript-Bundle weitere Paket bereitgestellt. Diese 
    sind nicht zwingend an die Verwendung einer der \TUDScript-Klassen 
    angewiesen sondern können prinzipiell mit jeder \hologo{LaTeXe}-Klasse 
    genutzt werden.%
  \end{abstract}
}
\chapter{Zusätzliche Pakete im \TUDScript-Bundle}
\tudhyperdef*{sec:bundle}%
%
\section{Das Paket \Package{tudscrcolor} -- Farben im \CD}%
\index{Farben|(}%
\index{Layout!Farben|?(}%
%
\begin{Bundle*}{\Package{tudscrcolor}}
Zur Verwendung der Farben des \CDs wird das Paket \Package{tudscrcolor} 
genutzt. Falls dieses nicht in der Präambel geladen wird~-- um beispielsweise 
zusätzliche Optionen aufzurufen~-- binden die \TUDScript"=Klassen dieses 
automatisch ein.

Für das \CD sind mehrere Farben vorgesehen. Die prägnanteste aller ist die 
Hausfarbe \Color{HKS41}, danach folgen die Farben für Auszeichnungen der ersten
(\Color{HKS44}) und der zweiten Kategorie (\Color{HKS36}, \Color{HKS33}, 
\Color{HKS57}, \Color{HKS65}) sowie eine Ausnahmefarbe (\Color{HKS07}). 
Diese Farben dürfen sowohl in ihrer Grundform als auch in helleren Tönen mit 
einer Abstufung in 10\,\%"~Schritten verwendet werden. Das ohnehin verwendete 
Paket \Package{xcolor} stellt genau diese Funktionalität zur Verfügung. Jede 
der Farben kann sowohl mit \Color*{HKS\PName{Zahl}}() als auch über ein 
Pseudonym \Color*{cd\PName{Farbe}}() genutzt werden.
%
\begin{Example*}
Die Grundfarbe \Color{HKS44} soll in der auf 20\% reduzierten, helleren 
Abstufung genutzt werden. Innerhalb eines Befehls, der als Argument eine 
gültige Farbe erwartet, muss lediglich \PValue{HKS44!20} angegeben werden. 
Dies wird hier exemplarisch mit der folgenden \colorbox{HKS44!20}{%
  Box \Macro{colorbox}[%
    \PParameter{HKS44!20}\PParameter{Box}%
  ](\Package{xcolor})'none'%
} demonstriert.
\end{Example*}
%
Bei der farbigen Gestaltung des \CDs (\Option{cd=color}) ist der Hintergrund 
von Umschlagseite, Titel sowie Teilen in \Color{HKS41} und die Schrift auf 
selbigen in \Color{HKS41}[!30] gehalten. Der Hintergrund von Kapitelseiten 
erscheint in \Color{HKS41}[!10], die Schrift in \Color{HKS41}. Bei geringerem 
Farbeinsatz werden lediglich die Schriften der Gliederungsseiten auf 
\Color{HKS41} gesetzt.

Sollen bestimmte Optionen an das Paket \Package{xcolor} weitergereicht werden, 
gibt es dafür zwei Möglichkeiten. Diese kann entweder vor dem Laden der Klasse 
direkt an \Package{xcolor} übergeben werden%
\footnote{%
  \Macro{PassOptionsToPackage}[\Parameter{Option}\PParameter{xcolor}] vor 
  \Macro*{documentclass}[\OParameter{Klassenoptionen}\PParameter{tudscr\dots}]
} oder es wird \Package{tudscrcolor} mit der entsprechenden Option geladen.%
\footnote{
  \Macro*{usepackage}[\OParameter{Option}\PParameter{tudscrcolor}];
  \Package{tudscrcolor} reicht \PName{Option} an \Package{xcolor} weiter
}
\newcommand*\cdcolorcalc{}
\newcommand*\cdcolorname{}
\newcommand*\cdcolorvalue{}
\newcommand*\cdcolortext{}
\newcommand*\cdcolor[2][0]{%
  \noindent%
  \begin{tikzpicture}[every node/.style={%
    rectangle,minimum height=.1\linewidth,minimum width=25mm%
  }]%
  \def\cdcolorcalc##1##2{%
    \pgfmathparse{100-##1*10}%
    \xdef\cdcolorname{HKS##2!\pgfmathresult}%
    \xdef\cdcolorvalue{\pgfmathresult}%
    \pgfmathparse{10+##1*10}%
  }%
  \foreach \x in {0,1,...,9}{%
    \cdcolorcalc{\x}{#2}%
    \ifnum\x<#1%
      \def\cdcolortext{white}%
    \else%
      \def\cdcolortext{black}%
    \fi%
    \node [fill=\cdcolorname,rotate=90] at (.\x\linewidth,0)%
      {\textcolor{\cdcolortext}{HKS#2!\pgfmathprintnumber\cdcolorvalue}};%
  }%
  \end{tikzpicture}%
}

\subsection{Generelle Farbdefinitionen}
\minisec{Primäre Hausfarbe}
\begin{Declaration}{\Color{HKS41}[cddarkblue]}
\printdeclarationlist%
\cdcolor[6]{41}
\end{Declaration}

\minisec{Sekundäre Hausfarbe (Geschäftsausstattung)}
\begin{Declaration}{\Color{HKS92}[cdgray]}
\printdeclarationlist%
\cdcolor[4]{92}
\end{Declaration}

\minisec{Auszeichnungsfarbe 1.Kategorie}
\begin{Declaration}{\Color{HKS44}[cdblue]}
\printdeclarationlist%
\cdcolor[4]{44}
\end{Declaration}

\minisec{Auszeichnungsfarbe 2.Kategorie}
\begin{Declaration}{\Color{HKS36}[cdindigo]}
\begin{Declaration}{\Color{HKS33}[cdpurple]}
\begin{Declaration}{\Color{HKS57}[cddarkgreen]}
\begin{Declaration}{\Color{HKS65}[cdgreen]}
\printdeclarationlist%
\cdcolor[4]{36}\vskip\medskipamount
\cdcolor[4]{33}\vskip\medskipamount
\cdcolor[2]{57}\vskip\medskipamount
\cdcolor{65}
\end{Declaration}
\end{Declaration}
\end{Declaration}
\end{Declaration}

\minisec{Ausnahmefarbe}
\begin{Declaration}{\Color{HKS07}[cdorange]}
\printdeclarationlist%
\cdcolor{07}
\end{Declaration}


\subsection{Zusätzliche Farbdefinitionen}
Das Paket \Package{tudscrcolor} definiert im Normalfall lediglich die zuvor 
beschriebenen Grundfarben \Color{HKS41}, \Color{HKS92}, \Color{HKS44}, 
\Color{HKS36}, \Color{HKS33}, \Color{HKS57}, \Color{HKS65} sowie \Color{HKS07}. 
Alle anderen farblichen Abstufungen können mit den beschrieben Möglichkeiten 
des Paketes \Package{xcolor} generiert werden.

\begin{Declaration}{\Option{oldcolors}}
\printdeclarationlist%
%
In den letzten Jahren sind viele verschiedene Klassen und Pakete für das \TUDCD 
entstanden. Innerhalb dieser existieren abweichende Farbdefinitionen. Um eine 
Migration von den benannten Klassen und Paketen auf \TUDScript zu ermöglichen, 
existiert die Paketoption \Option{oldcolors}. Wird diese genutzt, so werden 
zusätzliche Farben nach dem Schema \Color*{HKS41K\PName{Zahl}}() und 
\Color*{HKS41-\PName{Zahl}}() definiert, wobei der hinten angestellte 
Zahlenwert aus der 10er-Reihe kommen muss.
\end{Declaration}



\subsection{Umstellung des Farbmodells}
\index{Farben!Farbmodell}%
%
Normalerweise verwendet \Package{tudscrcolor} das CMYK"=Farbmodell. Außerdem 
wird weiterhin noch der RGB"=Farbraum unterstützt. Eine Umschaltung des 
Farbmodells ist beispielsweise für gewisse Funktionen des Paketes 
\Package{tikz} notwendig.

\begin{Declaration}{\Option{RGB}}
\printdeclarationlist%
%
Mit dieser Option werden bereits beim Laden des Paketes \Package{tudscrcolor} 
die Farben nicht nach dem CMYK"=Farbmodell sondern im RGB"=Farbraum global 
definiert.
\end{Declaration}

\begin{Declaration}{\Macro{setcdcolors}[\Parameter{Farbmodell}]}
\printdeclarationlist%
%
Mit diesem Befehl kann innerhalb des Dokumentes das verwendete Farbmodell 
angepasst werden. Damit ist es möglich, lokal innerhalb einer Umgebung den 
Farbmodus zu ändern und so nur in bestimmten Situationen beispielsweise aus dem 
CMYK"=Farbmodell in den RGB"=Farbraum zu wechseln. Unterstützte Werte für 
\PName{Farbmodell} sind \PValue{CMYK} und \PValue{RGB} beziehungsweise 
\PValue{rgb}.
\end{Declaration}

\bigskip\noindent
\Attention{%
  Beachten Sie, dass die Darstellung der Farben im jeweiligen Farbmodus 
  (\PValue{CMYK} oder \PValue{RGB}) je nach verwendeter Bildschirm- Drucker- 
  und Softwarekonfiguration verschieden ausfallen kann. Die verwendeten 
  RGB-Werte entstammen aus dem Handbuch zum \CD und sind lediglich 
  Näherungswerte. Abweichungen vom gedruckten HKS-Farbregister und selbst 
  ermittelten Werten sind technisch nicht zu vermeiden.
}%
\index{Farben|)}%
\index{Layout!Farben|?)}%
\end{Bundle*}



\section{Das Paket \Package{tudscrfonts} -- Schriften im \CD}
\begin{Bundle*}[v2.02]{\Package{tudscrfonts}}
\printchangedatlist%
%
Dieses Paket stellt die Schriften des \CDs für \hologo{LaTeXe}-Klassen bereit, 
welche \emph{nicht} zum \TUDScript-Bundle gehören. Das Paket unterstützt einen 
Großteil der für die \TUDScript-Klassen bereitgestellten Optionen und Befehle 
für die Schriftauswahl. Um Dopplungen in der Dokumentation zu vermeiden, wird 
auf eine abermalige Erläuterung der im Paket \Package{tudscrfonts} verfügbaren 
Optionen und Befehle verzichtet. Diese werden im Folgenden lediglich noch 
einmal genannt, die dazugehörigen Erläuterungen sind in \fullref{sec:fonts} zu 
finden.

Die nutzbaren Paketoptionen sind für den Fließtext \Option{cdfont}~-- ohne die 
Einstellungsmöglichkeiten für den Querbalken des \CDs (\Option{cdhead})~-- und 
für die mathematischen Schriften \Option{cdmath} sowie \Option{slantedgreek}. 
Weiterhin wird die Option \Option{vspacing} bereitgestellt. Besagte Optionen 
können dabei entweder als Paketoptionen im optionalen Argument von 
\Macro*{usepackage}[\OParameter{Paketoption}\PParameter{tudscrfonts}] oder 
direkt als Klassenoption angegeben werden. Zusätzlich ist nach dem Laden des 
Paketes die späte Optionenwahl mit \Macro{TUDoption} beziehungsweise 
\Macro{TUDoptions} möglich.

Die in \autoref{sec:fonts} beschriebenen Textschalter und "~kommandos zur 
expliziten Auswahl einzelnen Schnitte der Hausschriften sowie die Befehle für 
griechische Buchstaben werden ebenso wie der Befehl \Macro{ifdin} zur Prüfung 
auf die Verwendung von \DIN bereitgestellt. Dabei muss der Anwender das Setzen 
der Gliederungsüberschriften in Majuskeln der \DIN~-- wie es im \CD vorgesehen 
ist~-- selbst umsetzen. Hierfür sollten die Textauswahlbefehle \Macro{textdbn} 
und \Macro{dinbn} sowie \Macro{MakeTextUppercase}(\Package{textcase})'none' zur 
automatisierten Großschreibung genutzt werden. Der letztgenannte Befehl wird 
zusammen mit \Macro{NoCaseChange}(\Package{textcase}) von \Package{textcase} 
zur Verfügung stellt, welches durch \Package{tudscrfonts} geladen wird.

\ChangedAt{%
  v2.04:Unterstützung der Klassen \Class{tudposter} und \Class{tudmathposter};%
  v2.05:Neues Schriftpaket \Package{fix-tudscrfonts} für Dokumentklassen 
  im \CD der \TnUD, welche nicht zu \TUDScript gehören%
}
Ursprünglich war das Paket \Package{tudscrfonts} für die Verwendung zusammen 
mit einer der Klassen \Class{tudbook}, \Class{tudbeamer}, \Class{tudletter}, 
\Class{tudfax}, \Class{tudhaus}, \Class{tudform} und seit der Version~v2.04 
auch \Class{tudmathposter} sowie \Class{tudposter} vorgesehen. Allerdings 
traten bei der Verwendung des Paketes mit einer dieser Klassen einige kleinere 
Unzulänglichkeiten auf. Deshalb wird seit der Version~v2.05 empfohlen, für 
diese Klassen das Paket \Package{fix-tudscrfonts}'full' zu verwenden.
\end{Bundle*}



\section{Das Paket \Package{fix-tudscrfonts} -- Schriftkompatibilität}
\begin{Bundle*}[v2.05]{\Package{fix-tudscrfonts}}
\printchangedatlist%
%
Dieses Paket ist für die alleinige Verwendung mit einer der folgenden Klassen 
vorgesehen:
\begin{itemize}
\item \Class{tudbook}
\item \Class{tudbeamer}
\item \Class{tudletter}
\item \Class{tudfax}
\item \Class{tudhaus}
\item \Class{tudform}
\item \Class{tudmathposter}
\item \Class{tudposter}
\end{itemize}
%
Die Schriftinstallation für das \TUDScript-Bundle unterscheidet sich von der 
für die gerade genannten Klassen sehr stark. Dabei wurde auch die Bezeichnung 
der Schriftfamilien geändert. Dies hatte zwei Gründe, wobei letzterer von 
entscheidender Bedeutung ist:
%
\begin{enumerate}
\item
  Die bisherige Schriftbenennung entsprach nicht dem offiziellen 
  \hrfn{http://mirrors.ctan.org/info/fontname/fontname.pdf}%
  {\hologo{TeX}-Namensschema}
\item
  Bei der Installation für das \TUDScript-Bundle werden sowohl die Metriken
  als auch das Kerning der Schriften für Fließtext und den Mathematikmodus 
  angepasst, was das Ergebnis der erzeugten Ausgabe beeinflusst. Damit jedoch
  Dokumente, die mit den zuvor genannten, älteren Klassen erstellt wurden, 
  weiterhin genauso ausgegeben werden wie bisher, mussten die Schriftfamilien 
  einen neuen Namen erhalten.
\end{enumerate}
%
Wird nun das Paket \Package{fix-tudscrfonts} zusammen mit einer der zuvor 
genannten Klassen verwendet, hat dies den Vorteil, dass auch bei diesen sowohl 
das angepasste Kerning der Schriften als auch der stark verbesserte 
Mathematiksatz zum Tragen kommen. Außerdem kann bei der Verwendung von 
\Package{fix-tudscrfonts} auf eine Installation der Schriften des \CDs in der 
alten Variante verzichtet werden.
\Attention{%
  In diesem Fall kann sich das Ausgabeergebnis im Vergleich zu der Varianten 
  mit den alten Schriften ändern. Alternativ zur Verwendung des Paketes 
  \Package{fix-tudscrfonts} können die alten Schriftfamilien auch parallel zu 
  den neuen installiert werden. Hierfür werden die Skripte
  \hrfn{https://github.com/tud-cd/tudscrold/releases/download/fonts/tudfonts_install.bat}{\File{tudfonts\_install.bat}}
  beziehungsweise
  \hrfn{https://github.com/tud-cd/tudscrold/releases/download/fonts/tudfonts_install.sh}{\File{tudfonts\_install.sh}}
  bereitgestellt.
}%
%
Um alle notwendigen Einstellung korrekt und ohne unnötige Warnungen vornehmen 
zu können, muss das Paket \Package{fix-tudscrfonts} bereits \emph{vor} der 
Dokumentklasse geladen werden, wobei die gleichen Paketoptionen wie für das 
Paket \Package{tudscrfonts} verwendet werden können:
%
\begin{Code}[escapechar=§]
\RequirePackage[§\dots§]{fix-tudscrfonts}
\documentclass[§\dots§]{tudbook}
§\dots§
\begin{document}
§\dots§
\end{document}
\end{Code}
%
Dabei wird spätestens zum Ende der Präambel das Paket \Package{tudscrfonts} 
geladen. Alternativ kann dies auch durch den Benutzer in der Dokumentpräambel 
erfolgen.
\end{Bundle*}


\section{Das Paket \Package{mathswap}}
\index{Mathematiksatz|(}%
\index{Zifferngruppierung|(}%
%
\begin{Bundle*}{\Package{mathswap}}
Die Verwendung von Dezimal- und Tausendertrennzeichen im mathematischen Satz 
sind regional sehr unterschiedlich. In den meisten englischsprachigen Ländern 
wird der Punkt als Dezimaltrennzeichen und das Komma zur Zifferngruppierung 
verwendet, im restlichen Europa wird dies genau entgegengesetzt praktiziert.
Dieses Paket soll dazu dienen, beliebige formatierte Zahlen in ihrer Ausgabe 
anzupassen. Dafür werden die Zeichen Punkt (\ .\ ) und Komma (\ ,\ ) als 
aktive Zeichen im Mathematikmodus definiert.

Ähnliche Funktionalitäten werden bereits durch die Pakete \Package{icomma} und 
\Package{ziffer} bereitgestellt. Bei \Package{icomma} muss jedoch beim
Verfassen des Dokumentes durch den Autor beachtet werden, ob das verwendete
Komma einem Dezimaltrennzeichen entspricht ($t=1,\!2$) oder einem normalen 
Komma im Mathematiksatz ($z=f(x,y)$), wo ein gewisser Abstand nach dem Komma 
durchaus gewünscht ist. Das Paket \Package{ziffer} liefert dafür die gewünschte 
Funktionalität,%
\footnote{kein Leerraum nach Komma, wenn direkt danach eine Ziffer folgt}
ist allerdings etwas unflexibel, was den Umgang mit den Trennzeichen anbelangt.
Als Alternative zu diesem Paket kann außerdem \Package{ionumbers} verwendet 
werden.

Das Paket \Package{mathswap} sorgt dafür, dass Trennzeichen direkt vor einer 
Ziffer erkannt und nach bestimmten Vorgaben ersetzt werden. Sollte sich jedoch 
zwischen Trennzeichen und Ziffer Leerraum befinden, wird dieser als solcher
auch gesetzt. Für ein Beispiel zur Verwendung des Paketes sei auf das Tutorial 
\Tutorial{mathswap} in \autoref{sec:exmpl:mathswap} hingewiesen.

\begin{Declaration}{\Macro{commaswap}[\Parameter{Trennzeichen}]}
\begin{Declaration}{\Macro{dotswap}[\Parameter{Trennzeichen}]}
\printdeclarationlist%
%
Die beiden Befehle \Macro{commaswap} und \Macro{dotswap} sind die zentrale 
Benutzerschnittstelle des Paketes. Das Makro \Macro{commaswap} definiert das 
Trennzeichen oder den Inhalt, wodurch ein Komma ersetzt werden soll, auf 
welches direkt danach eine Ziffer folgt. Normalerweise setzt \hologo{LaTeX}
nach einem Komma im mathematischen Satz zusätzlich einen horizontalen Abstand.
Bei der Ersetzung durch \Macro{commaswap} entfällt dieser. Die Voreinstellung
für \Macro{commaswap} ist deshalb auf ein Komma (,) gesetzt. Mit dem Makro 
\Macro{dotswap} kann definiert werden, wodurch der Punkt im mathematischen 
Satz ersetzt werden soll, wenn auf diesen direkt anschließend eine Ziffer 
folgt. Da der Punkt im deutschsprachigem Raum zur Gruppierung von Ziffern 
genutzt wird, ist hierfür standardmäßig ein halbes geschütztes Leerzeichen 
definiert (\Macro*{,}).
\end{Declaration}
\end{Declaration}

\ChangedAt*{v2.02:Funktionalität im Dokument umschaltbar}%
\begin{Declaration}[v2.02]{\Macro{mathswapon}}
\begin{Declaration}[v2.02]{\Macro{mathswapoff}}
\printdeclarationlist%
%
Die Funktionalität von \Package{mathswap} kann innerhalb des Dokumentes mit 
diesen beiden Befehlen an- und abgeschaltet werden. Beim Laden des Paketes ist 
es standardmäßig aktiviert.
\end{Declaration}
\end{Declaration}
\index{Mathematiksatz|)}%
\index{Zifferngruppierung|)}%
\end{Bundle*}



\section{Das Paket \Package{twocolfix}}
\begin{Bundle*}{\Package{twocolfix}}
\index{Satzspiegel!zweispaltig|?}%
%
Der \hologo{LaTeXe}-Kernel enthält einen Fehler, der Kapitelüberschriften im
zweispaltigen Layout höher setzt, als im einspaltigen. Der 
\hrfn{http://www.latex-project.org/cgi-bin/ltxbugs2html?pr=latex/3126}{Fehler}
ist zwar schon länger bekannt, allerdings bisher noch nicht in den 
\hologo{LaTeXe}-Kernel übernommen worden. Das Paket \Package{twocolfix} behebt 
das Problem. Eine Integration dieses Bugfixes in \KOMAScript{} wurde bereits 
bei Markus Kohm angefragt, jedoch von ihm bis jetzt 
\hrfn{http://www.komascript.de/node/1681}{nicht weiter verfolgt}.
\end{Bundle*}


\addsec*{Zukünftige Arbeiten}
Diese Dinge sollen langfristig in das \TUDScript-Bundle eingearbeitet werden:

%\chapter{Das Paket \Package{tudscrletter} -- Briefe im \CD}
%\begin{Bundle*}{\Package{tudscrletter}}
\ToDo[imp]{Paket \Package*{tudscrletter} für Briefe im \CD}[v2.07]
Es soll das Paket \Package*{tudscrletter}<tudscr> für Briefe im \TUDCD 
entstehen. Auch Klassen für Fax und Hausmitteilungen sollen dabei abfallen.
%\end{Bundle*}

%\chapter{Das Paket \Package{tudscrbeamer} -- Präsentationen im \CD}
%\begin{Bundle*}{\Package{tudscrbeamer}}
\ToDo[imp]{Paket \Package*{tudscrbeamer} für Präsentationen im \CD}[v2.08]
Mit \Package*{tudscrbeamer}<tudscr> soll ein Paket entstehen, mit dem 
\hologo{LaTeX}-Beamer-Präsentationen im Stil des \TUDCDs erstellt werden können.
%\end{Bundle*}

%\chapter{Das Paket \Package{tudscrlayout} -- Seitenstil und Satzspiegel im \CD}
%\begin{Bundle*}{\Package{tudscrlayout}}
\ToDo[imp]{%
  Paket \Package*{tudscrlayout} \url{https://github.com/tud-cd/tud-cd/issues/6}
  für Satzspiegel, evtl. auch für die Klasse über \Option*{cdgeometry=forced}
}[v2.09]
Außerdem ist ein Paket \Package*{tudscrlayout}<tudscr> vorstellbar, welches
den durch das \CD vorgegebene Satzspiegel aktiviert ohne den Seitenstil selbst 
zu verwenden, um beispielsweise bereits mit dem Kopf der \TnUD bedrucktes 
Papier nutzen zu können. Ebenfalls wäre es denkbar, für andere Klassen die 
\PageStyle{tudheadings}-Seitenstile verfügbar zu machen ohne dabei den 
Satzspiegel des \CDs umzusetzen.
%\end{Bundle*}




\setpartpreamble{%
  \begin{abstract}
    \hypersetup{linkcolor=red}
    \noindent Für die Verwendung des \TUDScript-Bundles ist es nicht notwendig,
    diesen Teil zu lesen. In \autoref{sec:exmpl} sind insbesondere für 
    \hologo{LaTeX}"=Neulinge sowie neue Anwender des \TUDScript-Bundles 
    mehrere einfache Beispiele sowie umfangreichere Tutorials für dessen
    Verwendung zu sehen. In \autoref{sec:packages} werden Einsteigern~-- und
    auch dem bereits versierten \hologo{LaTeX}-Nutzer~-- meiner Meinung nach
    empfehlenswerte Pakete kurz vorgestellt.
    
    Anwendungshinweise sowie der eine oder andere allgemeine Hinweis bei der 
    Verwendung von \hologo{LaTeXe} wird in \autoref{sec:tips} gegeben. Dabei 
    sind diese durchaus für die Verwendung sowohl des \TUDScript-Bundles als 
    auch anderer \hologo{LaTeX}-Klassen interessant. Für Anregungen, Hinweise, 
    Ratschläge oder Empfehlungen zu weiteren Pakete sowie Tipps bin ich 
    jederzeit empfänglich.
  \end{abstract}
}
\part{Ergänzungen und Hinweise}
\tudhyperdef*{part:additional}
\setchapterpreamble{%
  \begin{abstract}
    \hypersetup{linkcolor=red}
    Dieses Kapitel soll den Einstieg und den ersten Umgang mit \TUDScript 
    erleichtern. Dafür werden einige Minimalbeispiele gegeben, die einzelne 
    Funktionalitäten darstellen. Diese sind so reduziert ausgeführt, dass sie 
    sich dem Anwender direkt erschließen sollten. Des Weiteren werden 
    weiterführende und kommentierte Anwendungsbeispiele bereitgestellt. Diese 
    Tutorials sind nicht unmittelbar im Handbuch enthalten sondern werden als 
    externe Dateien bereitgehalten, welche direkt via Hyperlink geöffnet werden 
    können.
  \end{abstract}
}
\chapter{Minimalbeispiele und Tutorials}
\label{sec:exmpl}
\index{Minimalbeispiel|(}

\section{Dokument}
\index{Minimalbeispiel!Dokument}
Hier wird gezeigt, wie die Präambel eines minimalen \hologo{LaTeXe}-Dokumentes 
gestaltet werden sollte. Dieser Ausschnitt kann prinzipiell als Grundlage für 
ein neu zu erstellendes Dokument verwendet werden. Lediglich das Einbinden des 
Paketes \Package*{blindtext} mit \Macro*{usepackage}\PParameter{blindtext} und 
die Verwendung des daraus stammenden Befehls \Macro*{blinddocument} können 
weggelassen werden.
\includeexample{document}

\section{Dissertation}
\label{sec:exmpl:dissertation}
\index{Minimalbeispiel!Dissertation}
Eine Abschlussarbeit oder ähnliches könnte wie hier gezeigt begonnen werden.
\includeexample{dissertation}

\section{Abschlussarbeit (kollaborativ)}
\label{sec:exmpl:thesis}
\index{Minimalbeispiel!Abschlussarbeit}
\index{Minimalbeispiel!Kollaboratives Schreiben}
Alle zusätzlichen Angaben außerhalb des Argumentes von \Macro{author} werden 
für beide Autoren gleichermaßen übernommen.%
\footnote{In diesem Beispiel \Macro{matriculationyear}.}
Die Angaben innerhalb des Argumentes von \Macro{author} werden den jeweiligen, 
mit \Macro{and} getrennten Autoren zugeordnet.%
\footnote{%
  In diesem Beispiel \Macro{matriculationnumber}, \Macro{dateofbirth} und 
  \Macro{placeofbirth}.
}
Ohne die Verwendung von \Macro{and} kann natürlich auch nur ein Autor 
aufgeführt werden. Außerdem sei auf die Verwendung von \Macro{subject} anstelle 
von \Macro{thesis} mit einem speziellen Wert aus \autoref{tab:thesis} 
hingewiesen.
\includeexample{thesis}

\section{Aufgabenstellung (kollaborativ)}
\label{sec:exmpl:task}
\index{Minimalbeispiel!Aufgabenstellung}
\index{Minimalbeispiel!Kollaboratives Schreiben}
Eine Aufgabenstellung für eine wissenschaftliche Arbeit ist mithilfe der 
Umgebung \Environment{task} oder dem Befehl \Macro{taskform} aus dem Paket 
\Package{tudscrsupervisor} folgendermaßen dargestellt werden.
\includeexample{task}

\section{Gutachten}
\label{sec:exmpl:evaluation}
\index{Minimalbeispiel!Gutachten}
Nach dem Laden des Paketes \Package{tudscrsupervisor} kann ein Gutachten für 
eine wissenschaftliche Arbeit mit der \Environment{evaluation}"=Umgebung oder 
dem Befehl \Macro{evaluationform} erstellt werden.
\includeexample{evaluation}

\section{Aushang}
\label{sec:exmpl:notice}
\index{Minimalbeispiel!Aushang}
Das Paket \Package{tudscrsupervisor} stellt die Umgebung \Environment{notice} 
für das Anfertigen allgemeiner Aushänge sowie den Befehl \Macro{noticeform} 
für die Ausschreibung wissenschaftlicher Arbeiten bereit.
\includeexample{notice}
\index{Minimalbeispiel|)}

\section{Vorlage für eine wissenschaftlichen Arbeit}
\label{sec:exmpl:treatise}
\index{Tutorial|(}
\index{Tutorial!Abschlussarbeit}
Die meisten Anwender der \TUDScript-Klassen sind Studenten oder angehörige der 
\TnUD, die ihre ersten Schritte mit \hologo{LaTeXe} beim Verfassen einer 
wissenschaftlichen Arbeit oder ähnlichem machen. Während der Einstiegsphase in 
\hologo{LaTeXe} kann ein Anfänger sehr schnell aufgrund der großen Anzahl an 
empfohlenen Pakete sowie der teilweise diametral zueinander stehenden Hinweise 
überfordert sein. Mit dem Tutorial \Tutorial{treatise} soll versucht werden, 
ein wenig Licht ins Dunkel zu bringen. Es erhebt jedoch keinerlei Anspruch, 
vollständig oder perfekt zu sein. Einige der darin vorgestellten Möglichkeiten 
lassen sich mit Sicherheit auch anders, einfacher und/oder besser lösen. 
Dennoch ist es gerade für Neulinge~-- vielleicht auch für den einen oder 
anderen \hologo{LaTeX}"=Veteran~-- als Leitfaden für die Erstellung einer 
wissenschaftlichen Arbeit gedacht.

\section{Ein Beitrag zum mathematischen Satz in \NoCaseChange{\hologo{LaTeXe}}}
\label{sec:exmpl:mathtype}
\index{Tutorial!Mathematiksatz}
Das Tutorial \Tutorial{mathtype} richtet sich an alle Anwender, die in ihrem 
\hologo{LaTeX}"=Dokument mathematische Formeln setzen wollen. In diesem wird 
ausführlich darauf eingegangen, wie mit wenigen Handgriffen ein typographisch 
sauberer Mathematiksatz zu bewerkstelligen ist.

\section{Änderung der Trennzeichen im Mathematikmodus}
\label{sec:exmpl:mathswap}
\index{Tutorial!Trennzeichen Mathematikmodus}
Sollen beim Verfassen eines \hologo{LaTeX}"=Dokumentes Daten in einem 
Zahlenformat importiert werden, welches nicht den Gepflogenheiten der 
Dokumentsprache entspricht, kommt es meist zu unschönen Ergebnissen bei der 
Ausgabe. Einfachstes Beispiel sind Daten, in denen als Dezimaltrennzeichen ein 
Punkt verwendet wird, wie es im englischsprachigen Raum der Fall ist. Sollen 
diese in einem Dokument deutscher Sprache eingebunden werden, müssten diese 
normalerweise allesamt angepasst und das ursprüngliche Dezimaltrennzeichen 
durch ein Komma ersetzt werden. Dieser Schritt wird mit dem \TUDScript-Paket 
\Package{mathswap} automatisiert. Wie dies genau funktioniert, wird im Tutorial 
\Tutorial{mathswap} erläutert.
\index{Tutorial|)}

\chapter{Unerlässliche und beachtenswerte Pakete}
\manualhyperdef{sec:packages}
\index{Kompatibilität!Pakete}
\section{Von den neuen Hauptklassen benötigte Pakete}
\label{sec:packages:needed}
\subsection{Erforderliche Pakete bei der Schriftinstallation}
%
Für die Installation der Schriften sind die folgend genannten Pakete von
\emph{essentieller} Bedeutung und daher \emph{zwingend} notwendig. Das 
Vorhandensein dieser wird durch die jeweiligen Schriftinstallationsskripte
(\see*{\autoref{sec:install}}) geprüft und die Installation beim Fehlen eines 
oder mehrerer Pakete mit einer entsprechenden Warnung abgebrochen.
%
\begin{packages}
\item[fontinst]
  Dieses Paket wird für die Installation der Schriften \Univers sowie \DIN 
  benötigt.
\item[cmbright]
  Alle mathematischen Glyphen und Symbole, die nicht in den \Univers-Schriften 
  enthalten sind, werden diesem Paket entnommen. Aßerdem werden für die 
  \PValue{T1}"~Schriftkodierung die beiden Pakete \Package{cm-super} und 
  \Package{hfbright} benötigt.
\item[iwona]
  Sowohl für die \Univers-Schriftfamilie als auch für \DIN werden fehlenden 
  Glyphen und Symbole hieraus entnommen.
\end{packages}
%
Zusätzlich werden die \texttt{Schreibmaschinenschriften} aus dem Paket 
\Package{lmodern} verwendet.



\subsection{Notwendige Pakete für die Verwendung der Hauptklassen}
Diese Pakete werden von den neuen Klassen zwingend benötigt und automatisch 
geladen.
%
\begin{packages}
\item[koma-script][scrlayer-scrpage,scrbase]
  \ChangedAt{%
    v2.02!\Package{scrlayer-scrpage}: Für Verwendung von \TUDScript notwendig%
  }%
  Das \KOMAScript-Bundle ist die zentrale Basis für \TUDScript. Neben den 
  Klassen \Class{scrbook}, \Class{scrreprt} und \Class{scrartcl} wird 
  das Paket \Package{scrbase} benötigt. Dieses 
  erlaubt das Definieren von Klassenoptionen im Stil von \KOMAScript, welche 
  auch noch nach dem Laden der Klasse mit den Befehlen \Macro{TUDoption} und 
  \Macro{TUDoptions} geändert werden können. Für die Bereitstellung der 
  \PageStyle{tudheadings}-Seitenstile ist das Paket \Package{scrlayer-scrpage} 
  notwendig. Wenn es nicht durch den Anwender~-- mit beliebigen Optionen~--  
  geladen wird, erfolgt dies am Ende der Präambel automatisch durch \TUDScript.
\item[kvsetkeys]
  Hiermit wird das von \Package{scrbase} geladene Paket \Package{keyval} 
  verbessert, welches das Definieren von Klassen"~ und Paketoptionen sowie 
  Parametern nach dem Schlüssel"=Wert"=Prinzip ermöglicht. Das Verhalten für 
  unbekannte Schlüssel  kann mit \Package{kvsetkeys} festgelegt werden.
\item[etoolbox]
  Es werden viele Funktionen zum Testen und zur Ablaufkontrolle bereitgestellt 
  und das einfache Manipulieren vorhandener Makros ermöglicht.
\item[geometry]\index{Satzspiegel}
  Das Paket wird zum Festlegen der Seitenränder respektive des Satzspiegels 
  verwendet. Ein Weiterreichen zusätzlicher Optionen an das Paket wird 
  dringlich nicht empfohlen.
\item[textcase]\index{Schriftauszeichnung}
  Mit \Macro{MakeTextUppercase} wird die Großschreibung der Überschriften in 
  \DIN erzwungen. Im \emph{Ausnahmefall} kann dies mit \Macro{NoCaseChange} 
  unterbunden werden.
\item[graphicx]\index{Grafiken}
  Dies ist das De-facto-Standard-Paket zum Einbinden von Grafiken. Zum Setzen 
  des Logos der \TnUD im Kopf wird \Macro{includegraphics} genutzt. Es kann 
  auch durch den Anwender in der Präambel geladen werden.
\item[xcolor]\index{Farben}
  Damit werden die Farben des \CDs zur Verwendung im Dokument definiert. 
  Genaueres ist bei der Beschreibung von \Package{tudscrcolor}'auto' zu finden. 
  Ein Laden beider Pakete in der  Präambel durch den Nutzer ist problemlos 
  möglich.
\item[environ]\index{Befehle!Deklaration}
  Es wird eine verbesserte Deklaration von Umgebungen ermöglicht, bei der auch 
  beim Abschluss der Umgebung auf die übergebenen Parameter zugegriffen werden 
  kann. Dies wird die Neugestaltung der \Environment{abstract}"=Umgebung 
  benötigt.
\item[trimspaces]
  Bei mehreren Eingabefeldern für den Anwender werden die Argumente mithilfe 
  dieses Paketes um eventuell angegebene, unnötige Leerzeichen befreit.
\end{packages}
%
Möchten Sie eines der hier aufgezählten Pakete selber nutzen, es jedoch mit 
bestimmten Optionen laden, so sollten diese bereits \emph{vor} der Definition 
der Dokumentklasse an das Paket weitergereicht werden, falls bei der 
Beschreibung nichts anderweitiges angegeben worden ist.
%
\begin{Example}
Das Weiterreichen von Optionen an Pakete muss folgendermaßen erfolgen:
\begin{Code}[escapechar=§]
\PassOptionsToPackage§\Parameter{Optionenliste}\Parameter{Paket}§
\documentclass§\OParameter{Klassenoptionen}\PParameter{tudscr\dots}§
\end{Code}
\end{Example}



\section{Durch \TUDScript direkt unterstütze Pakete}
%
Wird eines der genannten Pakete geladen, so wird die Funktionalität von 
\TUDScript durch dieses verbessert beziehungsweise erweitert oder es wird ein 
Bugfix bereitgestellt.
%
\begin{packages}
\item[hyperref]\index{Lesezeichen}\index{Querverweise}
  Hiermit können in einem PDF-Dokument Lesezeichen, Querverweise und 
  Hyperlinks erstellt werden. Wird es geladen, sind außerdem die Option 
  \Option{tudbookmarks} sowie der Befehl \Macro{tudbookmark} nutzbar. Das 
  Paket \Package{bookmark} erweitert die Unterstützung nochmals. Beide 
  genannten Pakete sollten~-- bis auf sehr wenige Ausnahmen wie beispielsweise 
  \Package{glossaries}~-- als letztes in der Präambel eingebunden werden.
\item[isodate]\index{Datum|?}
  Dieses Paket formatiert mit \Macro{printdate}\Parameter{Datum} die Ausgabe 
  eines Datums automatisch in ein spezifiziertes Format. Wird es geladen, 
  werden alle Datumsfelder, welche durch die \TUDScript-Klassen definiert 
  wurden,%
  \footnote{%
    \Macro{date}, \Macro{dateofbirth}, \Macro{defensedate}, \Macro{duedate}, 
    \Macro{issuedate}
  }
  in diesem Format ausgegeben.
\item[multicol]\index{Zweispaltensatz|?}
  Hiermit kann jeglicher beliebiger Inhalt in zwei oder mehr Spalten ausgegeben 
  werden, wobei~-- im Gegensatz zur \hologo{LaTeX}-Option \Option{twocolumn}~-- 
  für einen Spaltenausgleich gesorgt wird. Unterstützt wird das Paket innerhalb 
  der Umgebungen \Environment{abstract} und \Environment{tudpage}.
\item[quoting]\index{Zitate}
  \hologo{LaTeX} bietet von Haus aus \emph{zwei} verschiedene Umgebungen für 
  Zitate und ähnliches. Beide sind in ihrer Ausprägung starr und ignorieren 
  beispielsweise die Einstellungen von \Option{parskip}. Dies wird durch die 
  Umgebung \Environment{quoting} verbessert. Wird das Paket geladen, kommt 
  diese gegebenenfalls innerhalb der \Environment{abstract}"=Umgebung zum 
  Einsatz.
\item[ragged2e]\index{Worttrennung}
  Das Paket verbessert den Flattersatz, indem für diesen die Worttrennung 
  aktiviert wird.
\end{packages}



\section{Empfehlenswerte Pakete}
\label{sec:packages:recommended}
In diesem \autorefname wird eine Vielzahl an Paketen~-- zumeist kurz~-- 
vorgestellt, welche sich für mich persönlich bei der Arbeit mit \hologo{LaTeX} 
bewährt haben. Einige davon werden außerdem im Tutorial \Tutorial{treatise} in 
ihrer Anwendung beschrieben. Für detaillierte Informationen sowie bei Fragen zu 
den einzelnen Paketen sollte die jeweilige Dokumentation zu Rate gezogen
werden,\footnote{Kommandozeile/Terminal: \Path{texdoc\,\PName{Paketname}}}
das Lesen der hier gegebenen Kurzbeschreibung ersetzt dies in keinem Fall. 


\subsection{Pakete zur Verwendung in jedem Dokument}
Die hier vorgestellten Pakete gehören meiner Meinung nach in die Präambel eines 
jeden Dokumentes. Egal, in welcher Sprache das Dokument verfasst wird, sollte 
diese mit dem Paket \Package{babel} definiert werden~-- auch wenn dies 
Englisch ist. Für deutschsprachige Dokumente ist für eine annehmbare 
Worttrennung das Paket \Package{hyphsubst} unbedingt zu verwenden.

\begin{packages}
\item[fontenc]\index{Zeichensatzkodierung}
  Das Paket erlaubt Festlegung der Zeichensatzkodierung des Ausgabefonts. Als 
  Voreinstellung ist die Ausgabe als 7"~bit kodierte Schrift gewählt, was unter 
  anderem dazu führt, dass keine echten Umlaute im erzeugten PDF-Dokument 
  verwendet werden. Um auf 8"~bit"~Schriften zu schalten, sollte man
  \Macro*{usepackage}\POParameter{T1}\PParameter{fontenc} nutzen.
\item[selinput]\index{Eingabekodierung}
  Hiermit erfolgt die (automatische) Festlegung der Eingabekodierung. Diese ist 
  vom genutzten \hyperref[sec:tips:editor]{Editors (\autoref{sec:tips:editor})} 
  und den darin gewählten Einstellung abhängig. Mit:
  \begin{Code}
    \usepackage{selinput}
    \SelectInputMappings{adieresis={ä},germandbls={ß}}
  \end{Code}\vspace{-\baselineskip}%
  wird es verwendet. Dies macht den Quelltext portabel, womit beispielsweise 
  einfach via Copy~\&~Paste ein \hrfn{http://www.komascript.de/minimalbeispiel}%
  {Minimalbeispiel} bei Problemstellungen in einem Forum bereitgestellt werden 
  kann. Alternativ dazu lässt sich mit dem Paket \Package{inputenc} die zu 
  verwendende Eingabekodierung manuell einstellen
  (\Macro*{usepackage}\OParameter{Eingabekodierung})\PParameter{inputenc}).
\item[babel]\index{Sprachunterstützung}\index{Bezeichner}
  Mit diesem Paket erfolgt die Einstellung der im Dokument verwendeten 
  Sprache(n). Bei mehreren angegebenen Sprachen ist die zuletzt geladene die 
  Hauptsprache des Dokumentes. Die gewünschten Sprachen sollten als nicht als 
  Paketoption sondern als Klassenoption und gesetzt werden, damit auch andere 
  Pakete auf die Spracheinstellungen zugreifen können. Für deutschsprachige 
  Dokumente ist die Option \Option{ngerman} für die neue oder \Option{german} 
  für die alte deutsche Rechtschreibung zu verwenden. 
  
  Mit dem Laden von \Package{babel} und der dazugehörigen Sprachen werden 
  sowohl die Trennmuster als auch die sprachabhängigen Bezeichner angepasst.
  Von einer Verwendung der obsoleten Pakete \Package{german} beziehungsweise 
  \Package{ngerman} anstelle von \Package{babel} wird abgeraten. Für 
  \hologo{LuaLaTeX} und \hologo{XeLaTeX} kann das Paket \Package{polyglossia} 
  genutzt werden.
\item[microtype]\index{Typografie}
  Dieses Paket kümmert sich um den optischen Randausgleich%
  \footnote{englisch: protrusion, margin kerning}
  und das Nivellieren der Wortzwischenräume%
  \footnote{englisch: font expansion}
  im Dokument. Es funktioniert nicht mit der klassischen \hologo{TeX}-Engine, 
  wohl jedoch mit \hologo{pdfTeX} als auch \hologo{LuaTeX} sowie \hologo{XeTeX}.
\item[hyphsubst]\index{Worttrennung|!}
  Die möglichen Trennstellen von Wörtern wird von \hologo{LaTeX} mithilfe 
  eines Algorithmus berechnet. Dieser wird für deutschsprachige Texte mit dem 
  Paket \Package{hyphsubst} entscheidend verbessert. Es muss bereits \emph{vor} 
  der Dokumentklasse wie folgt geladen werden:
  \begin{Code}[escapechar=§]
    \RequirePackage[ngerman=ngerman-x-latest]{hyphsubst}
  \end{Code}\vspace{-\baselineskip}%
  In \autoref{sec:tips:hyphenation} wird genauer auf das Zusammenspiel von 
  \Package{hyphsubst} und \Package{babel} sowie \Package{fontenc} eingegangen,
  ein Blick dahin wird dringend empfohlen. Zusätzliche werden dort weitere 
  Hinweise für eine verbesserte Worttrennung gegeben. 
\end{packages}


\subsection{Pakete zur situativen Verwendung}
Die nachfolgenden Pakete sollten nicht zwangsweise in jedem Dokument geladen 
werden sondern nur, falls dies auch tatsächlich notwendig ist. Zur besseren 
Übersicht wurde versucht, diese thematisch passend zu gruppieren. Daraus lässt 
sich keinerlei Wertung bezüglich ihrer Nützlichkeit oder meiner persönlichen 
Wertschätzung ableiten.

\subsubsection{Typografie und Layout}
\index{Typografie}
%
\begin{packages}
\item[setspace]\index{Zeilenabstand}
  Die Vergrößerung des Zeilenabstandes wird:
  %
  \begin{enumerate}[itemindent=0pt,labelwidth=*,labelsep=1em,label=\Roman*.]
  \itempackages viel zu häufig und völlig unnötig gefordert und
  \itempackages schließlich auch noch zu groß gewählt.
  \end{enumerate}
  %
  Die Forderung nach Erhöhung des Zeilenabstandes~-- in der Typografie als 
  Durchschuss bezeichnet~-- kommt noch aus den Zeiten der Textverarbeitung mit 
  der Schreibmaschine. Ein einzeiliger Zeilenabstand bedeutete hier, dass die 
  Unterlängen der oberen Zeile genau auf der Höhe der Oberlängen der folgenden 
  Zeile lagen. Ein anderthalbzeiliger Zeilenabstand erzielte hier somit einen 
  akzeptablen Durchschuss. Eine Erhöhung des Durchschusses bei der Verwendung 
  von \hologo{LaTeX} ist an und für sich nicht notwendig. Sinnvoll ist dies 
  nur, wenn im Fließtext serifenlose Schriften zum Einsatz kommen, um die damit 
  verbundene schlechte Lesbarkeit etwas zu verbessern.
  
  Ist die Erhöhung des Durchschusses wirklich notwendig, sollte das Paket 
  \Package{setspace} genutzt werden. Dieses stellt den Befehl 
  \Macro{setstretch}\Parameter{Faktor} zur Verfügung, mit dem der Durchschuss 
  respektive Zeilenabstand angepasst werden kann. Der Wert des Faktors ist 
  standardmäßig auf~1 gestellt und sollte maximal bis~1.25 vergrößert werden. 
  Der Befehl \Macro*{onehalfspacing} aus diesem Paket setzt diesen Wert auf 
  eben genau~1.25. Allerdings ist hier anzumerken, dass die Vergrößerung des 
  Zeilenabstandes~-- so wie ich es mir angelesen habe~-- aus der Sicht eines 
  Typographen keine Spielerei ist sondern vielmehr allein der Lesbarkeit des 
  Textes dient und möglichst gering ausfallen sollte.
  
  Ziel ist es, beim Lesen nach dem Beenden der aktuellen Zeile das Auffinden 
  der neuen Zeile zu vereinfachen. Bei Serifen ist dies durch die Betonung der 
  Grundlinie sehr gut möglich. Bei serifenlosen Schriften~-- wie der im \CD der 
  \TnUD verwendeten \Univers~-- ist dies schwieriger und ein erweiterter 
  Abstand der   Zeilen kann dabei durchaus hilfreich sein. Jedoch sollte nicht 
  nach dem Motto \enquote{viel hilft viel} verfahren werden. In diesem Dokument 
  wurde als Faktor für den Zeilenabstand \Macro{setstretch}\PParameter{1.1} 
  gewählt. Nach einer Einstellung des Zeilenabstandes sollte der Satzspiegel 
  unbedingt mit \Macro{recalctypearea} neu berechnet werden. Weitere Tipps sind 
  in \autoref{sec:tips:headings} sowie \autoref{sec:tips:headline} zu finden.
\item[csquotes]\index{Zitate}
  Das Paket stellt unter anderem den Befehl \Macro{enquote}\Parameter{Zitat} 
  zur Verfügung, welcher Anführungszeichen in Abhängigkeit der gewählten 
  Sprache setzt. Zusätzlich werden weitere Kommandos und Optionen für die 
  spezifischen Anforderungen des Zitierens bei wissenschaftlichen Arbeiten 
  angeboten. Außerdem wird es durch \Package{biblatex} unterstützt und sollte 
  zumindest bei dessen Verwendung geladen werden.
\item[noindentafter]
  \ChangedAt{v2.02!\Package{noindentafter}: Beschreibung hinzugefügt}
  Mit diesem Paket lassen sich automatische Absatzeinzüge für selbst zu 
  bestimmende Befehle und Umgebungen unterdrücken.
\item[xspace]\index{Befehle!Deklaration}
  Mit \Package{xspace} kann bei der Definition eigener Makros der Befehl 
  \Macro{xspace} genutzt werden. Dieser setzt ein gegebenenfalls notwendiges 
  Leerzeichen automatisch. In \autoref{sec:tips:xspace} ist die Definition 
  eines solchen Befehls exemplarisch ausgeführt.
\item[ellipsis]\index{Befehle!Deklaration}
  In \hologo{LaTeX} folgen den Befehlen für Auslassungspunkte (\Macro{dots} und 
  \Macro{textellipsis}) \emph{immer} ein Leerzeichen. Dies kann unter Umständen 
  unerwünscht sein. Mit dem Paket \Package{ellipsis} wird das nachfolgende 
  Leerzeichen~-- im Gegensatz zum Standardverhalten~-- nur gesetzt, wenn ein 
  Satzzeichen und kein Buchstabe folgt, \seealso*{\autoref{sec:tips:dots}}.
\item[xpunctuate]\index{Befehle!Deklaration}
  Die Funktionalität von \Package{xspace} wird um die Beachtung von 
  Interpunktionen erweitert.
\itempackages[\href{http://www.ctan.org/pkg/delig}{\Application{DeLig}}]
  \index{Typografie}\index{Ligaturen}
  Hierbei handelt es sich um ein Java-Script, welches anhand eines Wörterbuches 
  falsche Ligaturen innerhalb eines Dokumentes automatisiert entfernt. Wird 
  \Univers verwendet ist dies jedoch nicht notwendig, da diese keinerlei 
  Ligaturen enthält, die insbesondere in deutschen Texten für einen guten Satz 
  manuell aufgelöst werden müssten.%
  \footnote{%
    Das sind ff, fi, fl, ffi, und ffl bei den \hologo{LaTeX}"=Standardschriften.
  }
  Mit \hologo{LuaLaTeX} als Dokumentprozessor kann alternativ dazu auch 
  \Package{selnolig} verwendet werden.
\item[balance]\index{Zweispaltensatz}
  Dieses Paket ermöglicht einen Spaltenausgleich im Zweispaltensatz auf der 
  letzten Dokumentseite. Alternativ dazu kann auch \Package{multicol} verwendet 
  werden.
\end{packages}

\subsubsection{Rechtschreibung}
Für die Rechtschreibkontrolle zeichnet im Normalfall der verwendete Editor 
verantwortlich. Dennoch gibt es einige wenige Pakete, welche sich diesem Thema 
widmen. Diese sind jedoch lediglich nutzbar, wenn \hologo{LuaLaTeX} als 
Dokumentprozessor genutzt wird.
\begin{packages}
\item[lua-check-hyphen]\index{Worttrennung}
  Hiermit lassen sich mit \hologo{LuaLaTeX} Trennstellen am Zeilenende zur 
  Prüfung markieren. Zum Thema der \textit{korrekten Worttrennung} sei außerdem 
  auf \autoref{sec:tips:hyphenation} verwiesen.
\item[spelling]\index{Rechtschreibung}
  Wird \hologo{LuaLaTeX} als Prozessor verwendet, wird mit diesem Paket der 
  reine Textanteil aus dem \hologo{LaTeX}"~Dokument extrahiert~-- wobei Makros 
  und aktive Zeichen entfernt werden~-- und in eine separate Textdatei 
  geschrieben. Anschließend kann diese Datei mit einer externen Software zur 
  Rechtschreibprüfung wie \Application{GNU Aspell} oder \Application{Hunspell} 
  analysiert werden. Wird durch dieses Programm eine Liste falsch geschriebener 
  Wörter ausgegeben, können diese mit \Package{spelling} im PDF"~Dokument 
  hervorgehoben werden.
\end{packages}

\subsubsection{Schriften und Sonderzeichen}
\begin{packages}
\item[lmodern](lm)\index{Schriftart}
  Soll mit den klassischen \hologo{LaTeX}"=Standardschriften gearbeitet werden, 
  empfiehlt sich die Verwendung des Paketes \Package{lmodern}. Dieses 
  verbessert die Darstellung der Computer~Modern sowohl am Bildschirm als auch 
  beim finalen Druck.
\item[cfr-lm]\index{Schriftart}
  Dieses experimentelle Paket liefert weitere Schriftschnitte für das Paket 
  \Package{lmodern}.
\item[newtx]
  Es werden einige alternative Schriften sowohl für den Fließtext 
  (\textit{Times} und \textit{Helvetica}) als auch den Mathematikmodus 
  bereitgestellt.
\item[libertine]\index{Schriftart}
  Das Paket stellt die Schriften Linux~Libertine und Linux~Biolinum zur 
  Verfügung. Um diese Schriftart auch für den Mathematikmodus verwenden zu 
  können, sollte \Package{newtxmath} aus dem \Package{newtx}-Bundle mit     
  \Macro*{usepackage}\POParameter{libertine}\PParameter{newtxmath} in der 
  Präambel eingebunden werden. Das Paket \Package{libgreek} enthält griechische 
  Buchstaben für Linux~Libertine.
\item[mweights]\index{Schriftstärke}
  In \hologo{LaTeXe} existieren die drei Schriftfamilien für Serifenschriften 
  (\Macro{rmfamily}), serifenlose Schriften (\Macro{sffamily}) sowie die 
  Schreibmaschinenschriften (\Macro{ttfamily}). Deren Schriftstärke wird für 
  gewöhnlich mit den beiden Befehlen \Macro{mddefault} und \Macro{bfdefault} 
  einheitlich festgelegt. Bei der Verwendung unterschiedlicher Schriftpakete 
  kann es unter Umständen zu Problemen bei den Schriftstärken kommen. Das Paket 
  \Package{mweights} erlaubt die individuelle Definition der Schriftstärke für 
  jede der drei Schriftfamilien.
\item[fontspec]
  \ChangedAt{v2.02!\Package{fontspec}: Beschreibung hinzugefügt}%
  Wird als Dokumentprozessor nicht \hologo{pdfLaTeX} sondern \hologo{XeLaTeX} 
  oder \hologo{LuaLaTeX} verwendet, können mit diesem Paket Systemschriften im 
  OpenType-Format eingebunden werden, womit sich die Auswahl der verwendbaren 
  Schriften in einem \hologo{LaTeX}"=Dokument stark erweitert. Das Paket wird 
  durch \TUDScript unterstützt.
\item[relsize]\index{Schriftgröße}
  Die Größe einer Textauszeichnung kann relativ zur aktuellen Schriftgröße 
  gesetzt werden.
\item[textcomp]\index{Sonderzeichen}
  Es werden zusätzliche Symbole und Sonderzeichen wie beispielsweise das 
  Promille- oder Eurozeichen sowie Pfeile für den Fließtext zur Verfügung 
  gestellt.
\end{packages}
%
Auch für (serifenlose) Mathematikschriften gibt es einige nützliche Pakete. 
Werden die Schriften des \CDs genutzt, sei auf die Option \Option{cdmath} 
verwiesen.\index{Mathematiksatz!Schrift}\index{Schrift!Mathematiksatz}
%
\begin{packages}
\item[sansmathfonts]
  Sollten die normalen \hologo{LaTeX}-Schriften Computer~Modern verwendet 
  werden, kann man dieses Paket zum serifenlosen Setzen mathematischer 
  Ausdrücke nutzen. Ein alternatives Paket mit der gleichen Zielstellung ist 
  \Package{sansmath}
\item[sfmath]
  Diese Paket verfolgt ein ähnliches Ziel, kann jedoch im Gegensatz zu 
  \Package{sansmath} nicht nur für Computer~Modern sondern mit der 
  entsprechenden Option auch für Latin~Modern, Helvetica und 
  Computer~Modern~Bright verwendet werden.
\item[mathastext]
   Mit dem Paket wird das Ziel verfolgt, aus der genutzten Schrift für den 
   Fließtext alle notwendigen Zeichen für den Mathematiksatz zu extrahieren.
\end{packages}

\subsubsection{Mathematiksatz}
\index{Mathematiksatz}
Dies sind Pakete, die Umgebungen und Befehle für den Mathematiksatz sowie das 
Setzen von Einheiten und Zahlen im Allgemeinen anbieten.

\begin{packages}
  \item[mathtools]
    Dieses Paket stellt für das De-facto-Standard-Paket \Package{amsmath} für 
    Mathematikumgebungen Bugfixes zur Verfügung und erweitert dieses.
  \item[bm]
    Das Paket bietet mit \Macro{bm} eine Alternative zu \Macro{boldsymbol} im 
    \hrfn{http://tex.stackexchange.com/questions/3238}{Mathematiksatz}.
\end{packages}
%
Für das typografisch korrekte Setzen von Einheiten~-- ein halbes Leerzeichen 
zwischen Zahl und \emph{aufrecht} gesetzter Einheit~-- gibt es zwei gut 
nutzbare Pakete.
%
\begin{packages}\index{Einheiten}
\item[units]
  Dies ist ein einfaches und sehr zweckdienliches Paket zum Setzen von 
  Einheiten und für die meisten Anforderungen völlig ausreichend.
\item[siunitx]
  Dieses Paket ist in seinem Umfang im Vergleich deutlich erweitert. Neben 
  Einheiten können zusätzlich auch Zahlen typografisch korrekt gesetzt werden. 
  Die Ausgabe lässt sich in vielerlei Hinsicht an individuelle Bedürfnisse 
  anpassen. Für deutschsprachige Dokumenten sollte die Lokalisierung angegeben 
  werden. Mehr dazu in \autoref{sec:tips:siunitx}.
\end{packages}
%
Die korrekte Formatierung von Zahlen ist häufig ein Problem bei der Verwendung 
von \hologo{LaTeX}. Insbesondere, wenn in einem deutschsprachigen Dokument 
Daten im englischsprachigen Format verwendet werden, kommt es zu Problemen. 
Dafür wird im \TUDScript-Bundle das Paket \Package{mathswap} bereitgestellt. 
Dennoch gibt es zu diesem auch Alternativen.
%
\begin{packages}\index{Trennzeichen}
  \item[icomma]
    Wird im Mathematikmodus nach dem Komma ein Leerzeichen gesetzt, wird dies 
    bei der Ausgabe beachtet. Der Verfasser muss sich demzufolge jederzeit 
    selbst um die typografisch korrekte Ausgabe kümmern.
  \item[ziffer]
    Für deutschsprachige Dokumente wird das Komma als Dezimaltrennzeichen 
    zwischen zwei Ziffern definiert. Folgt dem Komma keine Ziffer, wird 
    jederzeit der obligatorische Freiraum gesetzt, was meiner Meinung nach 
    besser als das Verhalten von \Package{icomma} ist.
  \item[ionumbers]
    Dieses Paket ist mir tatsächlich erst bei der Arbeit an \Package{mathswap} 
    bekannt geworden. Es bietet mehr Funktionalitäten und kann als Alternative 
    dazu betrachtet werden.
\end{packages}
%
Weitere Hinweise und Anwendungsfälle zur mathematischen Typografie werden in 
\autoref{sec:exmpl:mathtype} sowie \autoref{sec:exmpl:mathswap} gegeben.

\subsubsection{Verzeichnisse aller Art}
\index{Verzeichnisse|?}
Neben dem Erstellen des eigentlichen Dokumentes sind für eine wissenschaftliche 
Arbeit meist auch allerhand Verzeichnisse gefordert. Fester Bestandteil ist 
dabei das Literaturverzeichnis, auch ein Abkürzungs- und Formelzeichen- 
beziehungsweise Symbolverzeichnis werden häufig gefordert. Gegebenenfalls wird 
auch noch ein Glossar benötigt. Hier werden die passenden Pakete vorgestellt. 
Sollen im Dokument komplette Quelltexte oder auch nur Auszüge daraus erscheinen 
und für diese auch gleich ein entsprechendes Verzeichnis generiert werden, so 
sei auf das Paket \Package{listings}'full' verwiesen.

\begin{packages}
\item[biblatex]\index{Literaturverzeichnis}
  Das Paket kann als legitimer Nachfolger zu \hologo{BibTeX} gesehen werden. 
  Ähnlich dazu bietet \Package{biblatex} die Möglichkeit, Literaturdatenbanken 
  einzubinden und verschiedene Stile der Referenzierung und Darstellung des 
  Literaturverzeichnisses auszuwählen. 
  
  Mit \Package{biblatex} ist die Anpassung eines bestimmten Stiles wesentlich 
  besser umsetzbar als mit \hologo{BibTeX}. Wird \Application{biber} für die 
  Sortierung des Literaturverzeichnisses genutzt, ist die Verwendung einer 
  UTF"~8-kodierten Literaturdatenbank problemlos möglich. In Verbindung mit 
  \Package{biblatex} wird die zusätzliche Nutzung des Paketes 
  \Package{csquotes} sehr empfohlen.
\item[acro]\index{Abkürzungsverzeichnis}
  Soll lediglich ein Abkürzungsverzeichnis erstellt werden, ist dieses Paket 
  die erste Wahl. Es stellt Befehle zur Definition von Abkürzungen sowie zu 
  deren Verwendung im Text und zur sortierten Ausgabe eines Verzeichnisses 
  bereit. Alternativ dazu kann das Paket \Package{acronym} verwendet werden. 
  Die Sortierung des Abkürzungsverzeichnisses muss hier allerdings manuell 
  durch den Anwender erfolgen.
\item[glossaries]\index{Glossar}\index{Abkürzungsverzeichnis}%
  \index{Formelzeichenverzeichnis}\index{Symbolverzeichnis}%
  Dies ist ein sehr mächtiges Paket zum Erstellen eines Glossars sowie 
  Abkürzungs- und Symbolverzeichnisses. Die mannigfaltige Anzahl an Optionen 
  ist zu Beginn eventuell etwas abschreckend. Insbesondere wenn Verzeichnisse 
  für Abkürzungen \emph{und} Formelzeichen beziehungsweise Symbole benötigt 
  werden, sollte man dieses Paket in Erwägung ziehen.
  
  Alternativ dazu kann für ein Symbolverzeichnis auch lediglich eine manuell 
  gesetzte Tabelle genutzt werden. Das hierfür sehr häufig empfohlene Paket 
  \Package{nomencl} bietet meiner Meinung nach demgegenüber keinerlei Vorteile.
\end{packages}

\subsubsection{Listen}
\index{Listen|?}
\begin{packages}
\item[enumitem]
  Das Paket \Package{enumitem} erweitert die rudimentären Funktionalitäten der 
  \hologo{LaTeX}"=Standardlisten \Environment{itemize}, \Environment{enumerate}
  sowie \Environment{description} und ermöglicht die individuelle Anpassung 
  dieser durch die Bereitstellung vieler optionale Parameter nach dem
  Schlüssel"=Wert"=Prinzip. Eine von mir sehr häufig genutzte Funktion ist 
  beispielsweise die Entfernung des zusätzlichen Abstand zwischen den einzelnen 
  Einträgen einer Liste mit \Macro{setlist}\PParameter{noitemsep}.
\end{packages}

\subsubsection{Tabellen}
\index{Tabellen|?}
Für den Tabellensatz in \hologo{LaTeX} werden von Haus aus die Umgebungen 
\Environment{tabbing} und \Environment{tabular} beziehungsweise 
\Environment{tabular*} bereitgestellt, welche in ihrer Funktionalität meist 
für einen qualitativ hochwertigen Tabellensatz nicht ausreichen. Es werden 
deshalb Pakete vorgestellt, die zusätzlich verwendet werden können. 
\begin{packages}
\item[array]
  Dieses Paket ermöglicht mit dem Befehl \Macro{newcolumntype} das Erstellen 
  eigener Spaltentypen sowie die erweiterte Definition von Tabellenspalten
  (\PValue{>\{\dots\}}\PName{Spaltentyp}\PValue{<\{\dots\}}), wobei mithilfe 
  sogenannter \enquote{Hooks} vor und nach Einträgen innerhalb einer Spalte 
  gezielt Anweisungen gesetzt werden können. Außerdem kann die Höhe der Zeilen 
  einer Tabelle mit \Macro{extrarowheight} angepasst werden.  
\item[multirow]
  Es wird der Befehl \Macro{multirow} definiert, der~-- ähnlich zum Makro 
  \Macro{multicolumn}~-- das Zusammenfassen von mehreren Zeilen in einer 
  Spalte ermöglicht.
\item[widetable]
  Mit der Standard"=\hologo{LaTeX}"=Umgebung \Environment{tabular*} kann eine 
  Tabelle mit einer definierten Breite gesetzt werden. Dieses Paket stellt die 
  Umgebung \Environment{widetable} zur Verfügung, die als Alternative genutzt 
  werden kann und eine symmetrische Tabelle erzeugt.
\item[booktabs]
  Für einen guten Tabellensatz mit \hologo{LaTeX} gibt es bereits zahlreiche 
  \hrfn{http://userpage.fu-berlin.de/latex/Materialien/tabsatz.pdf}{Tipps} im 
  Internet zu finden. Zwei Regeln sollten dabei definitiv beachtet werden:
  %
  \begin{enumerate}[itemindent=0pt,labelwidth=*,labelsep=1em,label=\Roman*.]
  \itempackages keine vertikalen Linien
  \itempackages keine doppelten Linien
  \end{enumerate}
  %
  Das Paket \Package{booktabs} (deutsche Dokumentation \Package*{booktabs-de}) 
  ist für den Satz von hochwertigen Tabellen eine große Hilfe und stellt die 
  Befehle \Macro{toprule}, \Macro{midrule} sowie \Macro{cmidrule} und 
  \Macro{bottomrule} für unterschiedliche horizontale Linien bereit.
\item[tabularborder]
  Bei Tabellen wird zwischen Spalten automatisch ein horizontaler Abstand 
  (\Length{tabcolsep}) gesetzt~-- besser gesagt jeweils vor und nach einer 
  Spalte. Dies geschieht auch \emph{vor} der ersten und \emph{nach} der letzten 
  Spalte. Dieser zusätzliche Platz an den äußeren Rändern kann störend wirken, 
  insbesondere wenn die Tabelle über die komplette Textbreite gesetzt wird. Mit 
  dem Paket \Package{tabularborder} kann dieser Platz automatisch entfernt 
  werden.
  
  Dies funktioniert allerdings nur mit der \Environment{tabular}"=Umgebung. 
  Die Tabellen aus den Paketen \Package{tabularx}, \Package{tabulary} und 
  \Package{tabu} werden nicht unterstützt. Wie dieser Abstand bei diesen 
  manuell entfernt werden kann, ist unter \autoref{sec:tips:table} zu finden.
\item[tabularx]
  Auch mit diesem Paket kann die Gesamtbreite einer Tabelle spezifiziert 
  werden. Dafür wird der Spaltentyp \PValue{X} definiert, welcher als Argument 
  der \Environment{tabularx}"=Umgebung beliebig häufig angegeben werden kann
  (\Macro*{begin}\PParameter{tabularx}\Parameter{Breite}\Parameter{Spalten}). 
  Die \PValue{X}"~Spalten ähneln denen vom Typ~\PValue{p}\Parameter{Breite}, 
  wobei die Breite dieser aus der gewünschten Tabellengesamtbreite und dem 
  benötigten Platz der gegebenenfalls vorhandenen Standardspalten automatisch 
  berechnet wird.
\item[tabulary]
  Dies ist ein weiteres Paket zur automatischen Berechnung von Spaltenbreiten. 
  Der zur Verfügung stehende Platz~-- gewünschte Gesamtbreite abzüglich der 
  notwendigen Breite für die Standardspalten~-- wird jedoch nicht wie bei der 
  Umgebung \Environment{tabularx} auf alle Spalten gleichmäßig verteilt sondern 
  in der \Environment{tabulary}"=Umgebung für die Spaltentypen~\PValue{LCRJ} 
  anhand ihres Zellinhaltes gewichtet vergeben. 
  (\Macro*{begin}\PParameter{tabulary}\Parameter{Breite}\Parameter{Spalten}). 
\item[longtable]
  Sollen mehrseitige Tabellen mit Seitenumbruch erstellt werden, ist dieses 
  Paket das Mittel der ersten Wahl. Für die Kombination mehrseitiger Tabellen 
  mit einer \Environment{tabularx}"=Umgebung können die Pakete 
  \Package{ltablex} oder besser noch \Package{ltxtable} verwendet werden.
\item[ltxtable]
  Wie bereits erwähnt sollte dieses Paket für mehrseitige Tabellen, die mit der 
  Umgebung \Environment{tabularx} erstellt wurden, verwendet werden. 
  Alternativ dazu kann man auch \Package{tabu} nutzen.
\item[tabu]
  \ChangedAt{v2.02!\Package{tabu}: Verwendung nur bedingt empfehlenswert}
  Dies ist ein relativ neues Paket, welches versucht, viele der zuvor genannten 
  Funktionalitäten zu implementieren und weitere bereitzustellen. Dafür werden 
  die Umgebungen \Environment{tabu} und \Environment{longtabu} definiert. Es 
  kann alternativ zu \Package{tabularx} verwendet werden und ist insbesondere 
  als Ersatz für das Paket \Package{ltxtable} empfehlenswert.
  
  \Attention{%
    Leider wären für das Paket in der aktuellen Version~v2.8 seit geraumer Zeit 
    ein paar kleinere Bugfixes notwendig. Außerdem wird sich die 
    \hrfn{https://groups.google.com/d/topic/comp.text.tex/xRGJTC74uCI}{%
      Benutzerschnittstelle in einer zukünftigen Version
    } sehr stark ändern, weshalb es momentan mit Vorsicht zu genießen ist. 
    Zumindest sollte sich der Anwender bewusst sein, dass er mit dieser Version 
    gesetzte Dokumente gegebenenfalls später anpassen muss.
  }%
\end{packages}



\subsubsection{Gleitobjekte}
\index{Gleitobjekte|?}
\index{Tabellen}\index{Grafiken}
Es werden Pakete für die Beeinflussung von Aussehen, Beschriftung und 
Positionierung von Gleitobjekten vorgestellt. Unter \autoref{sec:tips:floats} 
sind außerdem Hinweise zur manuellen Manipulation der Gleitobjektplatzierung zu 
finden.

\begin{packages}
\item[placeins]\index{Gleitobjekte!Platzierung}
  Mit diesem Paket kann die Ausgabe von Gleitobjekten vor Kapiteln und wahlweise
  Unterkapiteln erzwungen werden.
\item[flafter]\index{Gleitobjekte!Platzierung}
  Dieses Paket erlaubt die frühestmögliche Platzierung von Gleitobjekten im 
  ausgegeben Dokument erst an der Stelle ihres Auftretens im Quelltext. Diese 
  werden dementsprechend nie vor ihrer Definition am Anfang der Seite 
  erscheinen.
\item[caption]\index{Gleitobjekte!Beschriftung}
  Die \KOMAScript-Klassen bietet bereits einige Möglichkeiten zum Setzen der 
  Beschriftungen für Gleitobjekte. Dieses Paket ist daher meist nur in gewissen
  Ausnahmefällen für spezielle Anweisungen notwendig, allerdings auch bei der 
  Verwendung unbedenklich.
\item[subcaption]\index{Gleitobjekte!Beschriftung}
  Diese Paket kann zum einfachen Setzen von Unterabbildungen oder "~tabellen 
  mit den entsprechenden Beschriftungen genutzt werden. Das dazu alternative 
  Paket \Package{subfig} sollte vermieden werden, da es nicht mehr gepflegt 
  wird und es mit diesem im Zusammenspiel mit anderen Paketen des Öfteren zu 
  Problemen kommt. Sollte der Funktionsumfang von \Package{subcaption} nicht 
  ausreichen, kann anstelle dessen das Paket \Package{floatrow} verwendet 
  werden, welches ähnliche Funktionalitäten wie \Package{subfig} bereitstellt.
\item[floatrow]\index{Gleitobjekte!Beschriftung}
  Mit diesem Paket können global wirksame Einstellungen und Formatierungen für 
  \emph{alle} Gleitobjekte eines Dokumentes vorgenommen werden. So kann unter 
  anderem die verwendete Schrift (\Macro{floatsetup}\PParameter{font=\dots}) 
  innerhalb der Umgebungen \Environment{float} und \Environment{table} 
  eingestellt werden. Das typografisch richtige Setzen der Beschriftungen von 
  Abbildungen als Unterschriften 
  (\Macro{floatsetup}\POParameter{figore}\PParameter{capposition=bottom})
  sowie Tabellen als Überschriften 
  (\Macro{floatsetup}\POParameter{table}\PParameter{capposition=top})
  kann automatisch erzwungen werden~-- unabhängig von der Position des Befehls 
  zur Beschriftung \Macro{caption} innerhalb der Gleitobjektumgebung. Wird das 
  Verhalten so wie empfohlen mit dem \Package{floatrow}-Paket eingestellt, 
  sollte für eine richtige Platzierung der Tabellenüberschriften außerdem die 
  \KOMAScript-Option \Option{captions}[tableheading] genutzt werden.
\end{packages}

\subsubsection{Grafiken und Abbildungen}
\index{Grafiken|?}
Grafiken für wissenschaftliche Arbeiten sollten als Vektorgrafiken erstellt 
werden, um die Skalierbarkeit und hohe Druckqualität zu gewährleisten. 
Bestenfalls folgen diese auch dem Stil der dazugehörigen Arbeit.%
\footnote{%
  Für qualitativ hochwertige Dokumente sollten übernommene Grafiken nicht 
  direkt kopiert oder gescannt sondern im gewünschten Zielformat neu erstellt 
  und mit der Referenz auf die Quelle ins Dokument eingebunden werden.
}
Für das Erstellen eigener Vektorgrafiken, welche die \hologo{LaTeX}"=Schriften 
und das Layout des Hauptdokumentes nutzen, gibt es zwei mögliche Ansätze. 
Entweder man \enquote{programmiert} die Grafiken ähnlich wie das Dokument 
selber oder man nutzt Zeichenprogramme, die wiederum die Ausgabe oder das 
Weiterreichen von Text an \hologo{LaTeX} unterstützen. Für das Programmieren 
von Grafiken sollen hier die wichtigsten Pakete vorgestellt werden. Wie diese 
zu verwenden sind, ist den dazugehörigen Paketdokumentationen zu entnehmen. 
Außerdem wird im Tutorial \Tutorial{treatise} für beide Pakete jeweils ein 
Beispiel gegeben.

\begin{packages}
\item[tikz](pgf)
  Dies ist ein sehr mächtiges Paket für das Programmieren von Vektorgrafiken 
  und sehr häufig~-- insbesondere bei Einsteigern~-- die erste Wahl bei der 
  Verwendung von \hologo{pdfLaTeX}.
\item[pstricks]
  Das Paket \Package{pstricks} stellt die zweite Variante zum Programmieren 
  von Grafiken dar. Mit diesem Paket hat man \emph{noch} mehr Möglichkeiten bei 
  der Erstellung eigener Grafiken, da man mit \Package{pstricks} auf 
  PostScript zugreifen kann und einige der bereitgestellten Befehle davon rege 
  Gebrauch machen. Der daraus resultierende Nachteil ist, dass mit 
  \Package{pstricks} die direkte Verwendung von \hologo{pdfLaTeX} nicht 
  möglich ist.
  
  Die Grafiken aus den \Environment{pspicture}"=Umgebungen müssen deshalb erst 
  über den Pfad \Path{latex \textrightarrow{} dvips \textrightarrow{} ps2pdf}
  in PDF"~Dateien gewandelt werden. Diese lassen sich von \hologo{pdfLaTeX} 
  anschließend als Abbildungen einbinden. Um dieses Vorgehen zu ermöglichen, 
  können folgende Pakete genutzt werden:
  %
  \begin{packages}
  \item[pst-pdf]
    Dieses Paket stellt Methoden für den Export von PostSript-Grafiken in 
    PDF-Datien bereit. Die einzelnen Aufrufe zur Kompilierung von DVI über 
    PostScript zu PDF müssen durch den Anwender manuell beziehungsweise über 
    die Ausgaberoutinen des verwendeten Editors durchgeführt werden.
  \item[auto-pst-pdf]
    Das Paket automatisiert die Erzeugung der \Package{pstricks}"=Grafiken mit 
    dem Paket \Package{pst-pdf}. Dafür muss \hologo{pdfLaTeX} per Option mit 
    Schreibrechten ausgeführt werden. Dazu ist der Aufruf von \Path{pdflatex} 
    mit der Option \Path{-{}-shell-escape} beziehungsweise für Nutzer von 
    \Distribution{\hologo{MiKTeX}} mit \Path{-{}-enable-write18} notwendig. 
    Bitte beachten Sie die Hinweise in \autoref{sec:tips:auto-pst-pdf}. Eine 
    Alternative dazu ist das Paket \Package{pdftricks2}.
  \end{packages}
\end{packages}
%
Um bei der Erstellung von Grafiken mit \Package{pstricks} oder \Package{tikz} 
nicht bei jeder Änderung das komplette Dokument kompilieren zu müssen, können 
diese in separate Dateien ausgelagert werden. Hierfür sind die beiden Pakete 
\Package{standalone} oder \Package{subfiles} sehr nützlich.

Für das Zeichnen einer Grafik mit einem Bildbearbeitungsprogramm, welches die 
Weiterverarbeitung durch \hologo{LaTeX} erlaubt, möchte ich auf die freien 
Programme \Application{LaTeXDraw} und \Application{Inkscape} verweisen. 
Insbesondere das zuletzt genannte Programm ist sehr empfehlenswert. Für die 
erstellten Grafiken kann man den Export für die Einbindung in \hologo{LaTeX} 
manuell durchführen. In \autoref{sec:tips:svg} wird vorgestellt, wie sich dies 
automatisieren lässt.

\subsubsection{Aufteilung des Hauptdokumentes in Unterdateien}
Um während des Entwurfes eines Dokumentes die Zeitdauer für das Kompilieren zu 
verkürzen, kann dieses in Unterdokumente gegliedert werden. Dadurch wird es 
möglich, nur den momentan bearbeiteten Dokumentteil~-- respektive die aktuelle 
\Package{tikz}- oder \Package{pstricks}-Grafik~-- zu kompilieren. Die meiner 
Meinung nach besten Pakete für dieses Unterfangen werden folgend vorgestellt.
%
\begin{packages}
\item[standalone]
  \ChangedAt{v2.02!\Class{standalone}: Probleme behoben}
  Dieses Paket ist für das Erstellen eigenständiger (Unter)"~Dokumente gedacht, 
  welche später in ein Hauptdokument eingebunden werden können. Jedes dieser 
  Teildokumente benötigt eine eigene Präambel. Optional lassen sich die 
  Präambeln der Unterdokumente automatisch in ein Hauptdokument einbinden.
\item[subfiles]
\ChangedAt{v2.02!\Package{subfiles}: Beschreibung hinzugefügt}
  Dieses Paket wählt einen etwas anderen Ansatz als \Package{standalone}. Es 
  ist von Anfang an dafür gedacht, ein dediziertes Hauptdokument zu verwenden. 
  Die darin mit \Macro{subfiles} eingebundenen Unterdateien nutzen bei der 
  autarken Kompilierung dessen Präambel.
\end{packages}
%
Unabhängig davon, ob Sie eines der beiden Pakete nutzen oder alles in einem 
Dokument belassen, ist es ratsam, eigens definierte Befehle, Umgebungen und 
ähnliches in ein separates Paket auszulagern. Dafür müssen Sie lediglich ein 
leeres \hologo{LaTeX}-Dokument erzeugen und es unter \File*{mypreamble.sty} 
oder einem anderen Namen im gleichen Ordner wie das Hauptdokument speichern. 
Dann können Sie in dieser Datei ihre Deklarationen ausführen und diese mit 
\Macro*{usepackage}\PParameter{mypreamble} in das Dokument einbinden. Dies hat 
den Vorteil, dass das Hauptdokument zum einen übersichtlich bleibt und Sie zum 
anderen Ihre persönliche Präambel generisch wachsen lassen und für andere 
Dokumente wiederverwenden können.

\subsubsection{Die kleinen und großen Helfer\dots}
Hier taucht alles auf, was nicht in die vorherigen Kategorien eingeordnet 
werden konnte.
%
\begin{packages}
\item[bookmark]\index{Lesezeichen}\index{Querverweise}
  Dieses Paket verbessert und erweitert die von \Package{hyperref} angebotenen 
  Möglichkeiten zur Erstellung von Lesezeichen~-- auch Outline"=Einträge~-- im 
  PDF-Dokument. Beispielsweise können Schriftfarbe- und "~stil geändert werden.
\item[calc]\index{Berechnungen}
  Normalerweise können Berechnungen nur mit Low-Level-\hologo{TeX}-Primitiven 
  im Dokument durchgeführt werden. Dieses Paket stellt eine einfachere Syntax 
  für Rechenoperationen der vier Grundrechenarten zur Verfügung. Zusätzlich 
  werden neue Befehle zur Bestimmung der Höhe und Breite bestimmter Textauszüge 
  definiert.
\item[chngcntr]\index{Zählermanipulation}
  Das Paket erlaubt die Manipulation aller möglichen, bereits definierten 
  \hologo{LaTeX}-Zähler. Es können Zähler so umdefiniert werden, dass sie bei 
  der Änderung eines anderen Zählers automatisch zurückgesetzt werden oder eben 
  nicht. Ein kleines Beispiel dazu ist in \autoref{sec:tips:counter} zu finden.
\item[varioref]\index{Querverweise}
  Mit diesem Paket lassen sich sehr gute Verweise auf bestimmte Seiten 
  erzeugen. Insbesondere, wenn der Querverweis auf die aktuelle, die 
  vorhergehende oder nachfolgende sowie im zweiseitigen Satz auf die 
  gegenüberliegende Seite erfolgt, werden passende Textbausteine für diesen 
  verwendet.
\item[cleveref]\index{Querverweise}
  Dieses Paket vereint die Vorzüge von \Package{varioref} mit der automatischen 
  Benennung der referenzierten Objekte mit dem Befehl \Macro{autoref} aus dem 
  Paket \Package{hyperref}.
\item[marginnote]\index{Randnotizen}
  Randnotizen, welche mit \Macro{marginpar} erzeugt werden, sind spezielle 
  Gleitobjekte in \hologo{LaTeX}. Dies kann dazu führen, dass eine Notiz am 
  Blattrand nicht direkt da gesetzt wird, wo diese intendiert war. Dieses Paket 
  stellt den Befehl \Macro{marginnote} für nicht"~gleitende Randnotizen zur 
  Verfügung. Alternativ dazu kann man auch \Package{mparhack} verwenden.
\item[todonotes]\index{Randnotizen}
  Mit \Package{todonotes} können noch offene Aufgaben in unterschiedlicher 
  Formatierung am Blattrand oder im direkt Fließtext ausgegeben werden. Aus 
  allen Anmerkungen lässt sich eine Liste aller offenen Punkte erzeugen.
\item[listings]\index{Quelltextdokumentation}%
  Dieses Paket eignet sich hervorragend zur Quelltextdokumentation in 
  \hologo{LaTeX}. Es bietet die Möglichkeit, externe Quelldateien einzulesen 
  und darzustellen sowie die Syntax in Abhängigkeit der verwendeten 
  Programmiersprache hervorzuheben. Zusätzlich lässt sich ein Verzeichnis mit 
  allen eingebundenen sowie direkt im Dokument angegebenen Quelltextauszügen 
  erstellen.
  \ChangedAt{v2.02!\Package{listings}: Beschreibung ergänzt}
  Wird \Package{listings} in Dokumenten mit UTF"~8-Kodierung verwendet, sollte 
  direkt nach dem Laden des Paketes in der Präambel Folgendes hinzugefügt 
  werden:
  \begin{Code}
    \lstset{%
      inputencoding=utf8,extendedchars=true,
      literate=%
        {ä}{{\"a}}1 {ö}{{\"o}}1 {ü}{{\"u}}1
        {Ä}{{\"A}}1 {Ö}{{\"O}}1 {Ü}{{\"U}}1
        {~}{{\textasciitilde}}1 {ß}{{\ss}}1
    }
  \end{Code}\vspace{-\baselineskip}%
\item[xparse]\index{Befehle!Deklaration}
  Dieses mächtige Paket entstammt dem \hologo{LaTeX3}-Projekt und bietet für 
  die Erstellung eigener Befehle und Umgebungen einen alternativen Ansatz zu 
  den bekannten \hologo{LaTeX}"=Deklarationsbefehlen \Macro*{newcommand} und 
  \Macro*{newenvironment} sowie deren Derivaten. Mit \Package{xparse} wird es 
  möglich, obligatorische und optionale Argumente an beliebigen Stellen 
  innerhalb des Befehlskonstruktes zu definieren. Auch die Verwendung anderer 
  Zeichen als eckige Klammern für die Spezifizierung eines optionalen 
  Argumentes ist möglich.
\item[xkeyval]\index{Befehle!Deklaration}
  \ChangedAt{v2.02!\Package{xkeyval}: Beschreibung hinzugefügt}
  Das \KOMAScript"=Bundle lädt das Paket \Package{keyval}, um Optionen mit 
  einer Schlüssel"=Wert"=Syntax deklarieren zu können. Zusätzlich wird von 
  \TUDScript das Paket \Package{kvsetkeys} geladen, um auf nicht definierte 
  Schlüssel reagieren zu können. Die Schlüssel"=Wert"=Syntax kann auch für 
  eigens definierte Makros genutzt werden, um sich das exzessive Verwenden von 
  optionalen Argumenten zu ersparen. Damit wäre folgende Definition möglich:
  \Macro*{newcommand}\Macro*{Befehl}\OParameter{Schlüssel"=Wert"=Liste}%
  \Parameter{Argument}.
  
  Das Paket \Package{xkeyval} erweitert insbesondere die Möglichkeiten zur 
  Deklaration unterschiedlicher Typen von Schlüsseln. Sollten die bereits durch 
  \TUDScript geladenen Pakete \Package{keyval} und \Package{kvsetkeys} in ihrer 
  Funktionalität nicht ausreichen, kann dieses Paket verwendet werden. Für die 
  Entwicklung eigener Pakete, deren Optionen das Schlüssel"=Wert"=Format 
  unterstützen, kann das Paket \Package{scrbase} genutzt werden. Soll aus einem 
  Grund auf \KOMAScript{} gänzlich verzichtet werden, sind die beiden Pakete 
  \Package{kvoptions} oder \Package{pgfkeys} eine Alternative.
\item[afterpage]
  Der Befehl \Macro{afterpage}\Parameter{\dots} kann genutzt werden, um den 
  Inhalt aus dessen Argument direkt nach der Ausgabe der aktuellen Seite 
  auszuführen.
\item[pagecolor]\index{Farben}
  Mit dem Paket kann die Hintergrundfarbe der Seiten im Dokument geändert 
  werden.
\item[pdfpages]
  Das Paket ermöglicht die Einbindung von einzelnen oder mehreren PDF"~Dateien.
\item[mwe]\index{Minimalbeispiel|!}
  \ChangedAt{v2.02!\Package{mwe}: Beschreibung hinzugefügt}
  Mit diesem Paket lassen sich sehr einfach Minimalbeispiele erzeugen, die 
  sowohl Blindtexte respektive Abbildungen enthalten sollen.
\item[filemod]
  Wird entweder \hologo{pdfLaTeX} oder \hologo{LuaLaTeX} als Prozessor 
  eingesetzt, können mit diesem Paket das Änderungsdatum zweier Dateien 
  miteinander verglichen und in Abhängigkeit davon definierbare Aktionen 
  ausgeführt werden.
\item[coseoul]
  Mit diesem Paket kann man die Struktur der Gliederung relativ angeben. Es 
  wird keine absolute Gliederungsebene (\Macro*{chapter}, \Macro*{section}) 
  angegeben sondern die Relation zwischen vorheriger und aktueller Ebene 
  (\Macro*{levelup}, \Macro*{levelstay}, \Macro*{leveldown}).
\item[dprogress]\index{Debugging}
  Das Paket schreibt bei der Kompilierung des Dokumentes die Gliederung in die 
  Logdatei. Dies kann im Fehlerfall beim Auffinden des Problems im Dokument 
  helfen. Allerdings werden dafür die Gliederungsebenen so umdefiniert, dass 
  diese keine optionalen Argumente mehr unterstützen,was jedoch für die 
  \TUDScript-Klassen von essentieller Bedeutung ist. Zum Debuggen kann es 
  trotzdem sporadisch eingesetzt werden.
\end{packages}

\subsubsection{Bugfixes}
\begin{packages}
\item[scrhack](koma-script)
  Das Paket behebt Kompatibilitätsprobleme der \KOMAScript-Klassen mit den 
  Paketen \Package{hyperref}, \Package{float}, \Package{floatrow} und
  \Package{listings}. Es ist durchaus empfehlenswert, jedoch sollte man 
  unbedingt die Dokumentation beachten.
\item[mparhack]
  Zur Behebung falsch gesetzter Randnotizen wird ein Bugfix für 
  \Macro{marginpar} bereitgestellt. Alternativ dazu kann man auch 
  \Package{marginnote} verwenden.
\end{packages}

\newcommand*\TaT{\hyperref[sec:tips]{Tipps \& Tricks}:\xspace}
\chapter{Praktische Tipps \& Tricks}
\tudhyperdef*{sec:tips}%
\section{\NoCaseChange{\hologo{LaTeX}}-Editoren}
\tudhyperdef*{sec:tips:editor}%
%
Hier werden die gängigsten Editoren zum Erzeugen von \hologo{LaTeX}"=Dateien 
genannt. Ich persönlich bin mittlerweile sehr überzeugter Nutzer von 
\Application{\hologo{TeX}studio}, da dieser viele Unterstützungs- und 
Assistenzfunktionen bietet. Neben diesen gibt es noch weitere, gut nutzbare 
\hologo{LaTeX}"=Editoren. Unabhängig von der Auswahl des Editors, sollte dieser 
auf jeden Fall eine Unicode"=Unterstützung~(UTF"~8) enthalten:
%
\begin{itemize}
\item \Application{\hologo{TeX}maker}
\item \Application{Kile}
\item \Application{\hologo{TeX}works}
\item \Application{\hologo{TeX}lipse}~-- Plug-in für \Application{Eclipse}
\item \Application{\hologo{TeX}nicCenter}
\item \Application{WinEdt}
\item \Application{LEd}~-- früher \hologo{LaTeX}~Editor
\item \Application{\hologo{LyX}}~-- grafisches Front"~End für \hologo{LaTeX}
\end{itemize}
%
Für \Application{\hologo{TeX}studio} wird im \GitHubRepo* das Archiv 
\hrfn{https://github.com/tud-cd/tudscr/releases/download/TeXstudio/tudscr4texstudio.zip}{\File{tudscr4texstudio.zip}}
bereitgestellt, welches Dateien zur Erweiterung der automatischen 
Befehlsvervollständigung für \TUDScript enthält. Diese müssen unter Windows in 
\Path{\%APPDATA\%\textbackslash texstudio} beziehungsweise unter unixoiden 
Betriebssystemen in \Path{.config/texstudio} eingefügt werden.

Möchten Sie das grafische \hologo{LaTeX}"~Frontend~\Application{\hologo{LyX}} 
für das Erstellen eines Dokumentes mit den \TUDScript-Klassen nutzen, so werden 
dafür spezielle Layout-Dateien benötigt, um die Klassendateien verwenden zu 
können. Diese sind zusammen mit einem \Application{\hologo{LyX}}"~Dokument als 
Archiv 
\hrfn{https://github.com/tud-cd/tudscr/releases/download/LyX/tudscr4lyx.zip}{\File{tudscr4lyx.zip}}
im \GitHubRepo* verfügbar. Die Layout-Dateien müssen dafür im 
\Application{\hologo{LyX}}"=Installationspfad in den passenden Unterordner 
kopiert werden. Dieser ist bei Windows
\Path{\%PROGRAMFILES(X86)\%\textbackslash{}LyX~2.1\textbackslash{}Resources\textbackslash{}layouts}
beziehungsweise bei unixoiden Betriebssystemen \Path{/usr/share/lyx/layouts}.
Anschließend muss LyX über den Menüpunkt \emph{Werkzeuge} neu konfiguriert 
werden. 



\section{Literaturverwaltung in \NoCaseChange{\hologo{LaTeX}}}
\ChangedAt{v2.02:\TaT Literaturverwaltung}
%
Die simpelste Variante, eine \hologo{LaTeX}"=Literaturdatenbank zu verwalten, 
ist dies mit dem Editor manuell zu erledigen. Wesentlich komfortabler ist es 
jedoch, die Referenzverwaltung mit einer darauf spezialisierten Anwendung zu 
bewerkstelligen. Dafür gibt es zwei sehr gute Programme:
%
\begin{itemize}
\item \Application{Citavi}
\item \Application{JabRef}
\end{itemize}
%
Das Programm \Application{Citavi} ermöglicht den Import von bibliografischen 
Informationen aus dem Internet. Allerdings sind diese teilweise unvollständig 
oder mangelhaft. Mit \Application{JabRef} hingegen muss die Literaturdatenbank 
manuell erstellt werden. Allerdings lassen sich einzelne Einträge aus 
\File*{.bib}"~Dateien importieren. Beide Anwendungen unterstützen den Export 
beziehungsweise die Erstellung von Datenbanken im Stil von \Package{biblatex}. 
Für \Application{JabRef} muss diese durch den Anwender explizit aktiviert 
werden.\footnote{Optionen/Einstellungen/Erweitert/BibLaTeX-Modus} 
Zur Verwendung der beiden Programme in Verbindung mit \Package{biblatex} und 
\Application{biber} gibt es ein gutes Tutorial unter diesem
\href{http://www.suedraum.de/latex/stammtisch/degenkolb_latex_biblatex_folien-final.pdf}{Link}.



\section{Worttrennungen in deutschsprachigen Texten}
\tudhyperdef*{sec:tips:hyphenation}
\ChangedAt{v2.02:\TaT Worttrennungen}
%
Die möglichen Trennstellen von Wörtern werden von \hologo{LaTeXe} mithilfe 
eines Algorithmus berechnet. Dieser ist jedoch in seiner ursprünglichen Form 
für die englische Sprache konzipiert worden. Für deutschsprachige Texte wird 
die Worttrennung~-- insbesondere bei zusammengeschriebenen Wörtern~-- mit dem 
Paket \Package{hyphsubst} entscheidend verbessert. Dafür wird ein um vielerlei 
Trennungsmuster ergänztes Wörterbuch aus dem Paket \Package{dehyph-exptl} 
genutzt. 

Das Paket \Package{hyphsubst} muss bereits \emph{vor} der Dokumentklasse selbst 
geladen werden. Außerdem wird das Paket \Package{babel} benötigt. Damit auch 
Wörter mit Umlauten richtig getrennt werden, ist zusätzlich die Verwendung des 
Paketes \Package{fontenc} mit der \PValue{T1}"~Schriftkodierung erforderlich. 
Der Beginn einer Dokumentpräambel könnte folgendermaßen aussehen:
%
\begin{quoting}[rightmargin=0pt]
\begin{Code}[escapechar=§]
\RequirePackage[ngerman=ngerman-x-latest]{hyphsubst}
\documentclass[ngerman,§\PName{Klassenoptionen}§]§\Parameter{Dokumentklasse}§
\usepackage{selinput}\SelectInputMappings{adieresis={ä},germandbls={ß}}
\usepackage[T1]{fontenc}
\usepackage{babel}
§\dots§
\end{Code}
\end{quoting}
%
Eine Anmerkung noch zur Trennung von Wörtern mit Bindestrichen. Normalerweise 
sind die beiden von \hologo{LaTeXe} verwendeten Zeichen für Bindestrich und 
Trennstrich identisch. Leider wird der Trennungsalgorithmus von \hologo{LaTeXe} 
bei Wörtern, welche bereits einen Bindestrich enthalten, außer Kraft gesetzt. 
In der Folge werden~-- in der deutschen Sprache durchaus öfter anzutreffende~-- 
Wortungetüme wie die \enquote{Donaudampfschifffahrts-Gesellschafterversammlung} 
normalerweise nur direkt nach dem angegebenen Bindestrich getrennt. 

Allerdings gibt es die Möglichkeit, das genutzte Zeichen für den Trennstrich 
zu ändern. Dafür ist das Laden der \PValue{T1}"~Schriftkodierung mit dem Paket 
\Package{fontenc} zwingend erforderlich. Wenn von der verwendeten Schrift 
nichts anderes eingestellt ist, liegen sowohl Binde- als auch Trennstrich auf 
Position~\PValue{45} der Zeichentabelle. In der \PValue{T1}"~Schriftkodierung 
befindet sich auf der Position~\PValue{127} glücklicherweise für gewöhnlich das 
gleiche Zeichen noch einmal. Dies ist jedoch von der verwendeten Schrift 
abhängig. Wird der Ausdruck \Macro*{defaulthyphenchar}[\PValue{=127}] in der 
Dokumentpräambel verwendet, kann dieses Zeichen für den Trennstrich genutzt 
werden. Bei den Schriften des \TUDCDs ist dies bereits automatisch eingestellt.

Sollte trotz aller Maßnahmen dennoch einmal ein bestimmtes Wort falsch getrennt 
werden, so kann die Worttrennung dieses Wortes manuell und global geändert 
werden. Dies wird mit \Macro{hyphenation}[\PParameter{Sil-ben-tren-nung}] 
gemacht. Es ist zu beachten, dass dies für alle Flexionsformen des Wortes 
erfolgen sollte. Für eine lokale/temporäre Worttrennung kann mit Befehlen aus 
dem Paket \Package{babel} gearbeitet werden. Diese sind: 

\vskip\medskipamount\noindent
\begingroup
\newcommand*\listhyphens[2]{#1 & \PValue{#2} \tabularnewline}%
\begin{tabular}{@{}ll}
  \textbf{Beschreibung} & \textbf{Befehl} \tabularnewline
  \listhyphens{ausschließliche Trennstellen}{\textbackslash-}
  \listhyphens{zusätzliche Trennstellen}{"'-} 
  \listhyphens{Umbruch ohne Trennstrich}{"'"'}
  \listhyphens{Bindestrich ohne Umbruch}{"'\textasciitilde} 
  \listhyphens{Bindestrich, der weitere Trennstellen erlaubt}{"'=}
\end{tabular}
\endgroup



\section{Lokale Änderungen von Befehlen und Einstellungen}
\index{Befehlsdeklaration!Geltungsbereich}%
\ChangedAt{v2.02:\TaT Lokale Änderungen}
%
Ein zentraler Bestandteil von \hologo{LaTeX} ist die Verwendung von Gruppen 
oder Gruppierungen. Innerhalb dieser bleiben alle vorgenommenen Änderungen an 
Befehlen, Umgebungen oder Einstellungen lokal. Dies kann sehr nützlich sein, 
wenn beispielsweise das Verhalten eines bestimmten Makros einmalig oder 
innerhalb von selbst definierten Befehlen oder Umgebungen geändert werden, im 
Normalfall jedoch die ursprüngliche Funktionalität behalten soll.
\begin{Example}
\index{Schriftauszeichnung}%
Der Befehl \Macro{emph} wird von \hologo{LaTeX} für Hervorhebungen im Text 
bereitgestellt und führt normalerweise zu einer kursiven oder~-- falls kein 
Schriftschnitt mit echten Kursiven vorhanden ist~-- kursivierten oder auch 
geneigten Auszeichnung. Soll nun in einem bestimmten Abschnitt die Auszeichnung 
mit fetter Schrift erfolgen, kann der Befehl \Macro{emph} innerhalb einer 
Gruppierung geändert und verändert werden. Wird diese beendet, verhält sich der 
Befehl wie gewohnt.
\begin{Code}
In diesem Text wird ein bestimmtes \emph{Wort} hervorgehoben.

\begingroup
  \renewcommand*{\emph}[1]{\textbf{#1}}%
  In diesem Text wird ein bestimmtes \emph{Wort} hervorgehoben.
\endgroup

In diesem Text wird ein bestimmtes \emph{Wort} hervorgehoben.
\end{Code}
\end{Example}
Eine Gruppierung kann entweder mit \Macro*{begingroup} und \Macro*{endgroup} 
oder einfach mit einem geschweiften Klammerpaar \PParameter{\dots} definiert 
werden.



\section{Bezeichnung der Gliederungsebenen durch \Package{hyperref}}
\tudhyperdef*{sec:tips:references}
\index{Querverweise}%
\ChangedAt{v2.02:\TaT Bezeichnungen der Gliederungsebenen}
%
Das Paket \Package{hyperref} stellt für Querverweise unter anderem den Befehl 
\Macro{autoref}[\Parameter{label}](\Package{hyperref})'none' zur Verfügung. Mit 
diesem wird~-- im Gegensatz zur Verwendung von \Macro{ref}~-- bei einer 
Referenz nicht nur die Nummerierung selber sondern auch das entsprechende 
Element wie Kapitel oder Abbildung vorangestellt. Bei der Benennung des 
referenzierten Elementes wird sequentiell geprüft, ob das Makro 
\Macro*{\PName{Element}\PValue{autorefname}} oder 
\Macro*{\PName{Element}\PValue{name}} existiert. Soll die Bezeichnung eines 
Elementes geändert werden, muss der entsprechende Bezeichner angepasst werden.
%
\begin{Example}
Bezeichnungen von Gliederungsebenen können folgendermaßen verändert werden.
\begin{Code}
\renewcaptionname{ngerman}{\sectionautorefname}{Unterkapitel}
\renewcaptionname{ngerman}{\subsectionautorefname}{Abschnitt}
\renewcaptionname{ngerman}{\subsubsectionautorefname}{Unterabschnitt}
\end{Code}
\end{Example}



\section{URL-Umbrüche im Literaturverzeichnis mit \Package{biblatex}}
\index{Literaturverzeichnis}%
\ChangedAt{v2.02:\TaT URL-Umbrüche im Literaturverzeichnis}
%
Wird das Paket \Package{biblatex} verwendet, kann es unter Umständen dazu 
kommen, das eine URL nicht vernünftig umbrochen werden. Ist dies der Fall, 
können die Zählern \Counter{biburlnumpenalty}(\Package{biblatex})'none', 
\Counter{biburlucpenalty}(\Package{biblatex})'none' und 
\Counter{biburllcpenalty}(\Package{biblatex})'none' erhöht werden. Das 
Manipulieren eines Zähler kann mit \Macro*{setcounter}[\Parameter{Zähler}] oder 
lokal mit \Macro*{defcounter}[\Parameter{Zähler}] aus dem Paket 
\Package{etoolbox} erfolgen. Die möglichen Werte liegen zwischen 0~und~10000, 
wobei es bei höheren Zählerwerten zu mehr URL-Umbrüchen an 
Ziffern~(\Counter{biburlnumpenalty}(\Package{biblatex})'none'), 
Groß-~(\Counter{biburlucpenalty}(\Package{biblatex})'none') und 
Kleinbuchstaben~(\Counter{biburllcpenalty}(\Package{biblatex})'none') kommt. 
Genaueres hierzu in der Dokumentation zu \Package{biblatex}.



\section{Zeilenabstände in Überschriften}
\tudhyperdef*{sec:tips:headings}%
%
Mit dem Paket \Package{setspace} kann der Zeilenabstand beziehungsweise der 
Durchschuss innerhalb des Dokumentes geändert werden. Sollte dieser erhöht 
worden sein, können die Abstände bei mehrzeiligen Überschriften als zu groß 
erscheinen. Um dies zu korrigieren kann mit dem Befehl 
\Macro{addtokomafont}[%
  \PParameter{disposition}\PParameter{%
    \Macro{setstretch}[\PParameter{1}](\Package{setspace})'none'%
  }%
](\Package{koma-script})'none'
der Zeilenabstand aller Überschriften auf einzeilig zurückgeschaltet werden. 
Soll dies nur für eine bestimmte Gliederungsebene erfolgen, so ist der 
Parameter \PValue{disposition} durch das dazugehörige Schriftelement zu 
ersetzen.



\section{Warnung wegen zu geringer Höhe der Kopf-/Fußzeile}
\tudhyperdef*{sec:tips:headline}%
%
Wird das Paket \Package{setspace} verwendet, kann es passieren, dass nach der 
Änderung des Zeilenabstandes \emph{innerhalb} des Dokumentes eine oder beide 
der folgenden Warnungen erscheinen:
%
\begin{quoting}
\begin{Code}
scrlayer-scrpage Warning: \headheight to low.
scrlayer-scrpage Warning: \footheight to low.
\end{Code}
\end{quoting}
%
Dies liegt an dem durch den vergrößerten Zeilenabstand erhöhten Bedarf für die
Kopf- und Fußzeile, die Höhen können in diesem Fall direkt mit der Verwendung 
von \Macro{recalctypearea}(\Package{typearea})'none' angepasst werden. 
Allerdings ändert das den Satzspiegel im Dokument, was eine andere und durchaus 
berechtigte Warnung von \Package{typearea} zur Folge hat. Falls die Änderung 
des Durchschusses wirklich nötig ist, sollte dies in der Präambel des 
Dokumentes einmalig passieren. Dann entfallen auch die Warnungen.



\section{Einrückung von Tabellenspalten verhindern}%
\tudhyperdef*{sec:tips:table}
\index{Tabellen}%
%
Normalerweise wird in einer Tabelle vor \emph{und} nach jeder Spalte durch 
\hologo{LaTeXe} etwas horizontaler Raum mit \Macro*{hskip}\Length{tabcolsep} 
eingefügt.%
\footnote{%
  Der Abstand zweier Spalten beträgt folglich \PValue{2}\Length{tabcolsep}.%
}
Dies geschieht auch \emph{vor} der ersten und \emph{nach} der letzten Spalte. 
Diese optische Einrückung an den äußeren Rändern kann unter Umständen stören, 
insbesondere bei Tabellen, die willentlich~-- beispielsweise mit den Paketen 
\Package{tabularx}, \Package{tabulary} oder auch \Package{tabu}~-- über die 
komplette Seitenbreite aufgespannt werden.

Das Paket \Package{tabularborder} versucht, dieses Problem automatisiert zu 
beheben, ist jedoch nicht zu allen \hologo{LaTeXe}"=Paketen für den 
Tabellensatz kompatibel, unter anderem auch nicht zu den drei zuvor genannten. 
Allerdings lässt sich dieses Problem manuell durch den Anwender lösen. 

Bei der Deklaration einer Tabelle kann mit~\PValue{@}\PParameter{\dots} vor und 
nach dem Spaltentyp angegeben werden, was anstelle von \Length{tabcolsep} vor 
beziehungsweise nach der eigentlichen Spalte eingeführt werden soll. Dies kann 
für das Entfernen der Einrückungen genutzt werden, indem an den entsprechenden 
Stellen~\PValue{@\PParameter{}} bei der Angabe der Spaltentypen vor der ersten 
und nach der letzten Tabellenspalte verwendet wird.
%
\begin{Example}
Eine Tabelle mit zwei Spalten, wobei bei einer die Breite automatisch berechnet 
wird, soll über die komplette Textbreite gesetzt werden. Dabei soll der Rand 
vor der ersten und nach der letzten entfernt werden.
\begin{Code}[escapechar=§]
\begin{tabularx}{\textwidth}{@{}lX@{}}
§\dots§ & §\dots§ \tabularnewline
§\dots§
\end{tabularx}
\end{Code}
\end{Example}




\section{Unterdrückung des Einzuges eines Absatzes}
\index{Absatzauszeichnung}%
%
Werden zur Absatzauszeichnung im Dokument~-- wie es aus typografischer Sicht 
zumeist sinnvoll ist~-- Einzüge und keine vertikalen Abstände verwendet
(\KOMAScript-Option \Option*{parskip=false}(\Package{koma-script})'none'), kann 
es vorkommen, dass ein ganz bestimmter Absatz~-- beispielsweise der nach einer 
zuvor genutzten Umgebung folgende~-- ungewollt eingerückt ist. Dies kann sehr 
einfach manuell behoben werden, indem direkt zu Beginn des Absatzes das Makro 
\Macro{noindent} aufgerufen wird. Soll das Einrücken von Absätzen nach ganz 
bestimmten Umgebungen oder Befehlen automatisiert unterbunden werden, ist das 
Paket \Package{noindentafter} zu empfehlen.



\section{Unterbinden des Zurücksetzens von Fußnoten}%
\tudhyperdef*{sec:tips:counter}%
\index{Fußnoten}%
%
Oft taucht die Frage auf, wie sich über Kapitel fortlaufende Fußnoten 
realisieren lassen. Dies ist sehr einfach mit dem Paket \Package{chngcntr} 
möglich. Nach dem Laden des Paketes, kann das Zurücksetzen des Zählers nach 
einem Kapitel mit \Macro{counterwithout*}[%
  \PParameter{footnote}\PParameter{chapter}%
](\Package{chngcntr})'none' deaktiviert werden. Auch andere 
\hologo{LaTeXe}"=Zähler~-- wie beispielsweise der bereits vorgestellte 
\Counter{symbolheadings}~-- lassen sich mit diesem Paket manipulieren.



\section{Setzen von Einheiten mit \Package{siunitx}}
\tudhyperdef*{sec:tips:siunitx}%
\index{Einheiten}%
%
Wenn \Package{siunitx} in einem deutschsprachigen Dokument genutzt soll
werden, muss zumindest die richtige Lokalisierung mit
\Macro{sisetup}[\PParameter{locale = DE}](\Package{siunitx})'none' angegeben 
werden. Sollen auch die Zahlen richtig formatiert sein, müssen weitere 
Einstellungen vorgenommen werden. Die meiner Meinung nach besten sind die 
folgenden.
%
\begin{quoting}
\begin{Code}
\sisetup{%
  locale = DE,%
  input-decimal-markers={,},input-ignore={.},%
  group-separator={\,},group-minimum-digits=3%
}
\end{Code}
\end{quoting}
%
Das Komma kommt als Dezimaltrennzeichen zum Einsatz. Des Weiteren werden Punkte 
innerhalb der Zahlen ignoriert und eine Gruppierung von jeweils drei Ziffern 
vorgenommen. Alternativ zu diesem Paket kann übrigens auch \Package{units} 
verwendet werden.



\section{Warnung beim Erzeugen des Inhaltsverzeichnisses}
\index{Inhaltsverzeichnis}%
\ChangedAt{v2.02:\TaT Warnung beim Erzeugen des Inhaltsverzeichnisses}
\ChangedAt{v2.02:\TaT Hinweis auf Rand bei mehrzeiligen Einträgen ergänzt}
%
Wird mit \Macro{tableofcontents}(\Package{koma-script})'none' das 
Inhaltsverzeichnis für ein Dokument mit einer dreistelligen Seitenanzahl 
erstellt, so erscheinen unter Umständen viele Warnungen mit der Meldung:
%
\begin{quoting}
\begin{Code}
overfull \hbox
\end{Code}
\end{quoting}
%
Die Seitenzahlen im Verzeichnis werden in einer Box mit einer festen Breite 
von~\PValue{1.55em} gesetzt, welche im Makro \Macro*{@pnumwidth} hinterlegt ist 
und im Zweifel vergrößert werden sollte. Dabei ist auch der rechte Rand für 
mehrzeilige Einträge im Verzeichnis \Macro*{@tocrmarg} zu vergrößern, welcher 
mit~\PValue{2.55em} voreingestellt ist. Die Werte sollten nur minimal geändert 
werden:
%
\begin{quoting}
\begin{Code}
\makeatletter
\renewcommand*{\@pnumwidth}{1.7em}\renewcommand*{\@tocrmarg}{2.7em}
\makeatother
\end{Code}
\end{quoting}



\section{Leer- und Satzzeichen nach \NoCaseChange{\hologo{LaTeX}}-Befehlen}%
\tudhyperdef*{sec:tips:xspace}%
\index{Typografie}%
%
Normalerweise \enquote{schluckt} \hologo{LaTeX} die Leerzeichen nach einem 
Makro ohne Argumente. Dies ist jedoch nicht immer~-- genau genommen in den 
seltensten Fällen~-- erwünscht. Für dieses Handbuch ist beispielsweise der 
Befehl \Macro*{TUD} definiert worden, um \enquote{\TUD{}} nicht ständig 
ausschreiben zu müssen. Um sich bei der Verwendung des Befehl innerhalb eines 
Satzes für den Erhalt eines folgenden Leerzeichens das Setzen der geschweiften 
Klammer nach dem Befehl zu sparen (\Macro*{TUD}[\PParameter{}]), kann 
\Macro{xspace}(\Package{xspace})'none' aus dem Paket \Package{xspace} genutzt 
werden. Damit wird ein folgendes Leerzeichen erhalten. Der Befehl \Macro*{TUD} 
ist wie folgt definiert:
%
\begin{quoting}
\begin{Code}
\newcommand*{\TUD}{Technische Universit\"at Dresden\xspace}
\end{Code}
\end{quoting}
%
Das Paket \Package{xpunctuate} erweitert die Funktionalität nochmals. Damit 
können auch Abkürzungen so definiert werden, dass ein versehentlicher Punkt 
ignoriert wird:
%
\begin{quoting}
\begin{Code}
\newcommand*{\zB}{z.\,B\xperiod}
\end{Code}
\end{quoting}



\section{Das Setzen von Auslassungspunkten}
\tudhyperdef*{sec:tips:dots}%
\index{Typografie}%
\ChangedAt{v2.02:\TaT Das Setzen von Auslassungspunkten}
%
Auslassungspunkte werden mit \hologo{LaTeXe} mit den Befehlen \Macro{dots} oder 
\Macro{textellipsis} gesetzt. Für gewöhnlich folgt diesen \emph{immer} ein 
Leerzeichen, was nicht in jedem Fall gewollt ist. Das Paket \Package{ellipsis} 
schafft hier Abhilfe, wobei die Option \Option*{xspace} führt dazu, dass nach 
der Verwendung eines der beiden Befehle automatisch ein Leerzeichen gesetzt 
wird. 
%
\begin{quoting}
\begin{Code}
\usepackage[xspace]{ellipsis}
\end{Code}
\end{quoting}
%
Im Ursprung ist es für das Setzen englischsprachiger Texte gedacht, wo zwischen 
Auslassungspunkten und Satzzeichen ein Leerzeichen gesetzt wird. Im Deutschen 
ist dies anders:
%
\begin{quoting}
\enquote{%
  Um eine Auslassung in einem Text zu kennzeichnen, werden drei Punkte gesetzt. 
  Vor und nach den Auslassungspunkten wird jeweils ein Wortzwischenraum 
  gesetzt, wenn sie für ein selbständiges Wort oder mehrere Wörter stehen. Bei 
  Auslassung eines Wortteils werden sie unmittelbar an den Rest des Wortes 
  angeschlossen. Am Satzende wird kein zusätzlicher Schlusspunkt gesetzt. 
  Satzzeichen werden ohne Zwischenraum angeschlossen.%
}~[Duden, 23. Aufl.]
\end{quoting} 
%
Um dieses Verhalten zu erreichen, sollte noch Folgendes in der Präambel 
eingefügt werden:
%
\begin{quoting}
\begin{Code}
\let\ellipsispunctuation\relax
\newcommand*{\qdots}{[\dots{}]\xspace}
\end{Code}
\end{quoting}
%
Der Befehl \Macro*{qdots} wird definiert, um Auslassungspunkte in eckigen 
Klammern (\POParameter{\dots}) setzen zu können, wie sie für das Kürzen von 
wörtlichen Zitaten häufig verwendet werden.



\section{Finden von unbekannten \NoCaseChange{\hologo{LaTeX}}-Symbolen}
\index{Symbole}%
%
Für \hologo{LaTeX} stehen jede Menge Symbole zur Verfügung, die allerdings 
nicht immer einfach zu finden sind. In der Zusammenfassung
\hrfn{http://mirrors.ctan.org/info/symbols/comprehensive/symbols-a4.pdf}{\File{symbols-a4.pdf}}
werden viele Symbole aus mehreren Paketen aufgeführt. Alternativ kann 
\hrfn{http://detexify.kirelabs.org/classify.html}{Detexify} verwendet werden. 
Auf dieser Web-Seite wird das gesuchte Symbol einfach gezeichnet, die dazu 
ähnlichsten werden zurückgegeben.



\section{Änderung des Papierformates}
\index{Papierformat}%
%
Es kann vorkommen, dass innerhalb eines Dokumentes kurzzeitig das Papierformat 
geändert werden soll, um beispielsweise eine Konstruktionsskizze in der 
digitalen PDF"~Datei einzubinden. Dabei ist es mit der \KOMAScript-Option 
\Option{paper=\PSet}(\Package{typearea})'none' sowohl möglich, lediglich die 
Ausrichtung in ein Querformat zu ändern, als auch die Größe des Papierformates 
selber.
%
\begin{Example}
Ein Dokument im A4"~Format soll kurzzeitig auf ein A3"=Querformat geändert 
werden. Das folgende Minimalbeispiel zeigt, wie dies mit \KOMAScript-Mitteln 
über die Optionen \Option*{paper=landscape}(\Package{typearea})'none' und 
\Option*{paper=A3}(\Package{typearea})'none' geändert werden kann.
\begin{Code}
\documentclass[paper=a4,pagesize]{tudscrreprt}
\usepackage{selinput}
\SelectInputMappings{adieresis={ä},germandbls={ß}}
\usepackage[T1]{fontenc}
\usepackage[ngerman]{babel}
\usepackage{blindtext}

\begin{document}
\chapter{Überschrift Eins}
\Blindtext

\cleardoublepage
\storeareas\PotraitArea% speichert den aktuellen Satzspiegel
\KOMAoptions{paper=A3,paper=landscape,DIV=current}
\chapter{Überschrift Zwei}
\Blindtext

\cleardoublepage
\PotraitArea% lädt den gespeicherten Satzspiegel
\chapter{Überschrift Drei}
\Blindtext
\end{document}
\end{Code}
\end{Example}



\section{Vermeiden des Skalierens einer PDF-Datei beim Druck}
\ChangedAt{v2.04:\TaT Vermeiden des Skalierens einer PDF-Datei beim Druck}
%
Beim Erzeugen eines Druckauftrages einer PDF-Datei kann es unter Umständen dazu 
führen, dass diese durch den verwendeten PDF-Betrachter unnötigerweise vorher 
skaliert wird und dabei die Seitenränder vergrößert werden. Um dieses Verhalten 
für Dokumente, die mit \Engine{pdfTeX} erzeugt werden, zu unterdrücken, gibt es 
zwei Möglichkeiten:
%
\begin{enumerate}
\item Wenn im Dokument ohnehin das Paket \Package{hyperref} verwendet wird, 
  ist der simple Aufruf von
  \Macro{hypersetup}[%
    \PParameter{pdfprintscaling=None}%
  ](\Package{hyperref})'none'
  ausreichend.
\item Der Low-Level-Befehl
  \Macro*{pdfcatalog}[\PParameter{/ViewerPreferences<{}</PrintScaling/None>{}>}]
  hat das gleiche Verhalten und kann auch ohne das Paket \Package{hyperref} 
  genutzt werden.
\end{enumerate}
%
Weitere Informationen dazu sind unter \url{http://www.komascript.de/node/1897} 
zu finden.


\section{Beschnittzugabe und Schnittmarken}
\tudhyperdef*{sec:tips:crop}%
\index{Beschnittzugabe|!}%
\index{Schnittmarken|!}%
\ChangedAt{v2.05:\TaT Beschnittzugabe und Schnittmarken}
%
Beim Plotten von Postern oder anderen farbigen Druckerzeugnissen besteht 
oftmals das Problem, dass ein randloses Drucken nur schwer realisierbar ist. 
Deshalb wird zu oftmals damit beholfen, dass der Druck des fertigen Dokumentes 
auf einem größeren Papierbogen erfolgt und anschließend auf das gewünschte 
Zielformat zugeschnitten wird, womit das Problem des nicht bedruckbaren Randes 
entfällt. Dies kann über zwei verschiedene Wege realisiert werden.

Der einfachste Weg ist die Verwendung des Paketes \Package{crop}. Mit diesem 
kann das Dokument ganz normal im gewünschten Zielformat erstellt werden. Vor 
dem Druck wird dieses Paket geladen und einfach das gewünschte Format des 
Papierbogens angegeben. 
%
\begin{quoting}
\begin{Code}[escapechar=§]
\RequirePackage{fix-cm}
\documentclass[%
  paper=a1,
  fontsize=36pt
]{tudscrposter}
\usepackage{selinput}
\SelectInputMappings{adieresis={ä},germandbls={ß}}
\usepackage[T1]{fontenc}
§\dots§
\usepackage{graphicx}
\usepackage[b1,center,cam]{crop}
\begin{document}
§\dots§
\end{document}
\end{Code}
\end{quoting}
%
Alternativ dazu kann für die \TUDScript-Klassen auf die Funktionalität des 
Paketes \Package{geometry} zurückgegriffen werden. Dieses Paket stellt den 
Befehl \Macro{geometry}(\Package{geometry})'none' bereit, in dessen Argument mit
\Key{\Macro{geometry}(\Package{geometry})}{paper=\PName{Papierformat}}|default|
das Papierformat festgelegt werden kann. Wird zusätzlich noch der Parameter 
\Key{\Macro{geometry}(\Package{geometry})}{layout=\PName{Zielformat}}|default|
angegeben, so wird damit das gewünschte Zielformat definiert. Dabei sollte mit
\Key{\Macro{geometry}(\Package{geometry})}{layoutoffset=\PName{Längenwert}}|default|
dieser Bereich gegebenenfalls etwas eingerückt werden. Die Angabe von 
\Key{\Macro{geometry}(\Package{geometry})}{showcrop=\PBoolean}|default|
generiert außerdem noch visuelle Schnittmarken. 
%
\begin{quoting}
\begin{Code}[escapechar=§]
\RequirePackage{fix-cm}
\documentclass[%
  paper=a1,
  fontsize=36pt
]{tudscrposter}
\usepackage{selinput}
\SelectInputMappings{adieresis={ä},germandbls={ß}}
\usepackage[T1]{fontenc}
§\dots§
\geometry{paper=b1,layout=a1,layoutoffset=1in,showcrop}
\begin{document}
§\dots§
\end{document}
\end{Code}
\end{quoting}
%
Für genauere Erläuterungen sowie weitere Einstellmöglichkeiten sei auf die 
Dokumentation von \Package{crop} beziehungsweise \Package{geometry} verwiesen.
Mit der \TUDScript-Option \Option{bleedmargin}(\Class{tudscrposter}) können
zusätzlich ie farbigen Bereiche der \PageStyle{tudheadings}-Seitenstile 
erweitert werden, um ein \enquote{Zuschneiden in die Farbe} zu ermöglichen.



\section{Warnung bei der Schriftgrößenwahl}
\tudhyperdef*{sec:tips:fontsize}%
\ChangedAt{v2.04:\TaT Warnung bei der Schriftgrößenwahl}
%
Die im Dokument verwendete Schriftgröße kann bei den \KOMAScript-Klassen sehr 
einfach über die Option~\Option{fontsize}(\Package{koma-script})'none' 
eingestellt werden, wobei diese immer als Klassenoption angegeben werden 
sollte. Bei relativ großen und kleinen Schriftgrößen kann dabei eine Warnung in 
der Gestalt 
%
\begin{quoting}
\begin{Code}[escapechar=§]
LaTeX Font Warning: Font shape `§\dots§' in size <xx> not available
\end{Code}
\end{quoting}
%
auftreten. Dies daran, dass zum Zeitpunkt des Ladens einer Klasse immer nach 
den Computer-Modern-Standardschriften gesucht wird, unabhängig davon, ob im 
Nachhinein ein anderes Schriftpaket geladen wird. Diese sind de-facto nicht in 
alle Größen skalierbar. Um die Warnungen zu beseitigen, sollte das Paket 
\Package{fix-cm} mit \Macro*{RequirePackage} \emph{vor} der Dokumentklasse 
geladen werden:
% 
\begin{quoting}[rightmargin=0pt]
\begin{Code}[escapechar=§]
\RequirePackage{fix-cm}
\documentclass§\OParameter{Klassenoptionen}\Parameter{Klasse}§
\usepackage{selinput}\SelectInputMappings{adieresis={ä},germandbls={ß}}
\usepackage[T1]{fontenc}
§\dots§
\begin{document}
§\dots§
\end{document}
\end{Code}
\end{quoting}
%
Damit werden die Warnungen behoben.


\section{Platzierung von Gleitobjekten}
\tudhyperdef*{sec:tips:floats}%
\index{Gleitobjekte!Platzierung|?}%
%
Mit den beiden Paketen \Package{flafter} sowie \Package{placeins} gibt es die 
Möglichkeit, den für \hologo{LaTeX} zur Verfügung stehenden Raum für die 
Platzierung von Gleitobjekten einzuschränken. Darüber hinaus kann diese auch 
durch die im Folgenden aufgezählten Befehle beeinflusst werden. Die Makros 
lassen sich mit \Macro*{renewcommand*}[\Parameter{Befehl}\Parameter{Wert}] 
sehr einfach ändern.

\begin{Declaration}{\Macro*{floatpagefraction}}[0\floatpagefraction]
\begin{Declaration}{\Macro*{dblfloatpagefraction}}[0\dblfloatpagefraction]
\printdeclarationlist%
%
Der Wert gibt die relative Größe eines Gleitobjektes bezogen auf die Texthöhe 
(\Length*{textheight}) an, die mindestens erreicht sein muss, damit für dieses 
gegebenenfalls vor dem Beginn eines neuen Kapitels eine separate Seite erzeugt 
wird. Dabei wird einspaltiges (\Macro*{floatpagefraction}) und zweispaltiges 
(\Macro*{dblfloatpagefraction}) Layout unterschieden. Der Wert für beide 
Befehle sollte im Bereich von~\PValue{0.5}\dots\PValue{0.8} liegen.
\end{Declaration}
\end{Declaration}

\begin{Declaration}{\Macro*{topfraction}}[0\topfraction]
\begin{Declaration}{\Macro*{dbltopfraction}}[0\dbltopfraction]
\printdeclarationlist%
%
Diese Werte geben den maximalen Seitenanteil für Gleitobjekte, die am oberen 
Seitenrand platziert werden, für einspaltiges und zweispaltiges Layout an. Er 
sollte im Bereich von \PValue{0.5}\dots\PValue{0.8} liegen und größer als 
\Macro*{floatpagefraction} beziehungsweise \Macro*{dblfloatpagefraction} sein.
\end{Declaration}
\end{Declaration}

\begin{Declaration}{\Macro*{bottomfraction}}[0\bottomfraction]
\printdeclarationlist%
%
Dies ist der maximale Seitenanteil für Gleitobjekte, die am unteren Seitenrand 
platziert werden. Er sollte zwischen~\PValue{0.2} und~\PValue{0.5} betragen.
\end{Declaration}

\begin{Declaration}{\Macro*{textfraction}}[0\textfraction]
\printdeclarationlist%
%
Dies ist der Mindestanteil an Text, der auf einer Seite mit Gleitobjekten 
vorhanden sein muss, wenn diese nicht auf einer eigenen Seite ausgegeben 
werden. Er sollte im Bereich von~\PValue{0.1}\dots\PValue{0.3} liegen.
\end{Declaration}

\begin{Declaration}{\Counter*{totalnumber}}[\arabic{totalnumber}]
\begin{Declaration}{\Counter*{topnumber}}[\arabic{topnumber}]
\begin{Declaration}{\Counter*{dbltopnumber}}[\arabic{dbltopnumber}]
\begin{Declaration}{\Counter*{bottomnumber}}[\arabic{bottomnumber}]
\printdeclarationlist%
%
Außerdem gibt es noch Zähler, welche die maximale Anzahl an Gleitobjekten pro 
Seite insgesamt (\Counter*{totalnumber}), am oberen (\Counter*{topnumber}) und 
am unteren Rand der Seite (\Counter*{bottomnumber}) sowie im zweispaltigen Satz
beide Spalten überspannend (\Counter*{dbltopnumber}) festlegen. Die Werte 
können mit \Macro*{setcounter}[\Parameter{Zähler}\Parameter{Wert}] geändert 
werden.
\end{Declaration}
\end{Declaration}
\end{Declaration}
\end{Declaration}

\begin{Declaration}{\Length*{@fptop}}
\begin{Declaration}{\Length*{@fpsep}}
\begin{Declaration}{\Length*{@fpbot}}
\begin{Declaration}{\Length*{@dblfptop}}
\begin{Declaration}{\Length*{@dblfpsep}}
\begin{Declaration}{\Length*{@dblfpbot}}
\printdeclarationlist%
%
Sind vor Beginn eines Kapitels noch Gleitobjekte verblieben, so werden diese 
durch \hologo{LaTeX} normalerweise auf einer separaten vertikal zentriert Seite 
ausgegeben. Dabei bestimmen diese Längen jeweils den Abstand vor dem ersten 
Gleitobjekt zum oberen Seitenrand (\Length*{@fptop}, \Length*{@dblfptop}), 
zwischen den einzelnen Objekten (\Length*{@fpsep}, \Length*{@dblfpsep}) sowie 
zum unteren Seitenrand (\Length*{@fpbot}, \Length*{@dblfpbot}). Soll dies nicht 
geschehen, können die Längen durch den Anwender geändert werden.
\end{Declaration}
\end{Declaration}
\end{Declaration}
\end{Declaration}
\end{Declaration}
\end{Declaration}
%
\begin{Example}
Alle Gleitobjekte auf einer dafür speziell gesetzten Seite sollen direkt zu 
Beginn dieser ausgegeben werden. In der Dokumentpräambel lässt sich für dieses 
Unterfangen Folgendes nutzen:
\begin{Code}
\makeatletter
\setlength{\@fptop}{0pt}
\setlength{\@dblfptop}{0pt}% twocolumn
\makeatother
\end{Code}
\end{Example}



\section{Automatisiertes Einbinden von \Application{Inkscape}-Grafiken }
\tudhyperdef*{sec:tips:svg}%
\index{Grafiken}%
%
Das Einbinden von \Application{Inkscape}"=Grafiken in \hologo{LaTeX}"=Dokumente
wird auf \CTAN[pkg/svg-inkscape]{\Package*{svg-inkscape}'none'} erläutert. Hier 
wird ein daraus abgeleiteter und verbesserter Ansatz vorgestellt, um den Export 
der \Application{Inkscape}"=Grafiken \textbf{automatisiert} direkt bei der 
Kompilierung durch \Engine{pdfLaTeX} auszuführen und diese in das Dokument 
einzubinden. Nutzer von unixartigen Systemen können alternativ dazu auch das 
Paket \Package{svg} nutzen, welches ebenfalls den folgend erläuterten Befehl 
\Macro{includesvg}~-- in etwas abgewandelter Form~-- definiert.

Die mit \Application{Inkscape}|?| erstellte Grafik soll automatisch exportiert
und eingebunden werden. Um die benötigte Zeit für die Kompilierung möglichst 
gering zu halten, sollte der Export nicht bei jedem \Engine{pdfLaTeX}"=Lauf 
erfolgen, sondern lediglich, falls die originale Bilddatei geändert respektive 
aktualisiert wurde. Hierfür wird das Paket \Package{filemod} verwendet. Die 
automatisierte Übersetzung einer Grafik im SVG"~Format in eine PDF"~Datei und 
die daran anschließende Einbindung dieser in das Dokument ist mit der 
\ToDo[imp]{Bugfix in \Package{svg} oder in tudscr oder eigenes Paket?}[v2.06]
%\ChangedAt{%
%  v2.06:\TaT\Macro*{includesvg} Bugfix für \Application{Inkscape} Version~0.91;
%  v2.06:\TaT\Macro*{includesvg} in Funktionalität erweitert%
%}
Definition von \Macro{includesvg}[\OParameter{Breite}\Parameter{Datei}] 
in der Präambel des Dokumentes wie folgt möglich:
%\usepackage{filemod}
%\usepackage{kvsetkeys}
%\usepackage{graphicx}
%\graphicspath{{bla/}{blubb/}}
%\DeclareUnicodeCharacter{2212}{\ensuremath{-}}
%\makeatletter
%\newcommand*{\includesvg}[2][\textwidth]{%
%  \def\svgwidth{#1}%
%  \filemodCmp{#2.pdf}{#2.svg}{}{%
%    \immediate\write18{%
%      inkscape -z -D --file=#2.svg --export-pdf=#2.pdf --export-latex%
%    }%
%  }%
%  \IfFileExists{#2.pdf}{%
%    \begingroup%
%      \pdfximage{#2.pdf}%
%      \let\saved@includegraphics\includegraphics%
%      \let\includegraphics\patch@includegraphics%
%      \InputIfFileExists{#2.pdf_tex}{}{}%
%    \endgroup%
%  }{}%
%}
%\newcommand*\saved@includegraphics{}
%\newcommand*\patch@includegraphics[2][]{%
%  \define@key{svg}{page}{\@tempcnta=##1\relax}%
%  \kv@set@family@handler{svg}{}%
%  \kvsetkeys{svg}{#1}%
%  \ifnum\@tempcnta>\pdflastximagepages\relax\else%
%    \saved@includegraphics[#1]{#2}%
%  \fi%
%}
%\makeatother
%
\begin{Declaration*}{\Macro{includesvg}}
\begin{quoting}
\begin{Code}[escapechar=§]
\usepackage{filemod}
\newcommand*{\includesvg}[2][\textwidth]{%
  \def\svgwidth{#1}
  \filemodCmp{#2.pdf}{#2.svg}{}{%
    \immediate\write18{%
      inkscape -z -D --file=#2.svg --export-pdf=#2.pdf --export-latex
    }%
  }%
  \InputIfFileExists{#2.pdf_tex}{}{}%
}
\end{Code}
\end{quoting}
\end{Declaration*}
%
Mit \Macro*{immediate}\Macro*{write18}[\Parameter{externer Aufruf}] wird das 
zwischenzeitliche Ausführen eines externen Programms beim Durchlauf von 
\Engine{pdfLaTeX}~-- in diesem Fall von \File{inkscape.exe}~-- möglich. Damit 
der externe Aufruf auch tatsächlich durchgeführt wird, ist die Ausführung von 
\Engine{pdfLaTeX} mit der Option \Path{-{}-shell-escape} beziehungsweise 
\Path{-{}-enable-write18} zwingend notwendig. Außerdem muss der Pfad zur Datei 
\File{inkscape.exe} dem System bekannt sein.%
\footnote{%
  Der Pfad zu \File{inkscape.exe} in der Umgebungsvariable \Path{PATH} des 
  Betriebssystems enthalten sein.%
}
Bei der Verwendung des Befehls \Macro{includesvg} \emph{muss} der Dateiname 
\emph{ohne Endung} angegeben werden. Die einzubindende SVG"~Datei sollte sich 
hierbei im gleichen Pfad wie das Hauptdokument befinden. Ist die SVG"~Datei in 
einem Unterordner relativ zum Pfad des Hauptdokumentes, kann dieser einfach mit 
\Macro{includesvg}[\PParameter{\PName{Ordner}/\PName{Datei}}] im Argument 
angegeben werden.



\section{Fehlermeldung: ! No room for a new \textbackslash write}
\tudhyperdef*{sec:tips:write}
\ChangedAt{v2.02:\TaT Fehler beim Schreiben von Hilfsdateien}
%
Für das Erstellen und Schreiben externer Hilfsdateien steht \hologo{LaTeXe} nur 
eine begrenzte Anzahl sogenannter Ausgabe-Streams zur Verfügung. Allein für 
jedes zu erstellende Verzeichnis reserviert \hologo{LaTeX} selbst jeweils einen 
neuen Stream. Auch einige bereits zuvor in diesem Handbuch vorgestellte, sehr 
hilfreiche Pakete~-- wie beispielsweise \Package{hyperref}, \Package{biblatex}, 
\Package{glossaries}, \Package{todonotes} oder auch \Package{filecontents}~-- 
benötigen eigene Hilfsdateien und öffnen für das Erstellen dieser einen 
Ausgabe-Stream oder mehr. Lädt der Anwender mehrere, in eine Hilfsdatei 
schreibende Pakete, kann es zur der Fehlermeldung
%
\begin{quoting}
\begin{Code}
! No room for a new \write .
\end{Code}
\end{quoting}
%
kommen. Abhilfe schafft hier das Paket \Package{scrwfile}<koma-script>, welches 
einige Änderungen am \hologo{LaTeX}"=Kernel vornimmt, um die Anzahl der 
benötigten Hilfsdateien für das Schreiben aller Verzeichnisse zu reduzieren. Es 
muss einfach in der Präambel eingebunden werden. Sollten mit diesem Paket 
unerwarteter Weise Probleme auftreten, ist dessen Anleitung im \scrguide zu 
finden. Eine weitere Möglichkeit, das beschriebene Problem der geringen Menge 
an Ausgabe-Streams zu umgehen, stellt das Paket \Package{morewrites} dar. 
Allerdings ist dessen Verwendung nicht in allen Fällen von Erfolg gekrönt.



\section{Fehlermeldung beim Laden eines Paketes mit Optionen}
\ChangedAt{v2.05:\TaT Fehler beim Laden eines Paketes mit Optionen}
%
Es kann unter Umständen passieren, dass beim Laden eines Paketes mit bestimmten 
Optionen via \Macro*{usepackage}[\OParameter{Paketoptionen}\Parameter{Paket}] 
folgender Fehler ausgegeben wird:
%
\begin{quoting}
\begin{Code}[escapechar=§]
! LaTeX Error: Option clash for package <§\dots§>.
\end{Code}
\end{quoting}
%
Mit großer Sicherheit wird das angeforderte Paket bereits durch die verwendete 
Dokumentklasse oder ein anderes Paket geladen. Normalerweise genügt es, bereits 
vor dem Laden der Dokumentklasse mit \Macro*{documentclass} durch 
\Macro{PassOptionsToPackage}[\Parameter{Paketoptionen}\Parameter{Paket}] die 
gewünschten Optionen an das Paket weiterzureichen, welches den Konflikt meldet.



\section{Probleme bei der Verwendung von \Package{auto-pst-pdf}}
\tudhyperdef*{sec:tips:auto-pst-pdf}
\ChangedAt{v2.02:\TaT Hinweise zum Paket \Package{auto-pst-pdf}}
%
Bei der Verwendung von \Engine{pdfLaTeX} liest das Paket \Package{auto-pst-pdf} 
die Präambel ein und erstellt anschließend über den PostScript"=Pfad 
\Path{latex \textrightarrow{} dvips \textrightarrow{} ps2pdf} eine PDF-Datei, 
welche lediglich alle in den vorhandenen 
\Environment{pspicture}(\Package{pstricks})'none'"=Umgebungen erstellten 
Grafiken enthält. Das Paket \Package{ifpdf} stellt das Makro 
\Macro{ifpdf}(\Package{ifpdf})'none' bereit, mit welchem unterschieden werden 
kann, ob \Engine{pdfLaTeX} als Textsatzsystem verwendet wird. Abhängig davon 
können unterschiedliche Quelltexte ausgeführt werden, was genutzt wird, um die 
nachfolgend beschriebenen Probleme zu beheben.
%
\begin{quoting}
\begin{Code}
\usepackage{ifpdf}
\end{Code}
\end{quoting}

\minisec{Die gleichzeitige Verwendung von \Package{floatrow}}
Das Paket \Package{floatrow} stellt Befehle bereit, mit denen die Beschriftung 
von Gleitobjekten sehr bequem gesetzt werden können. Diese Setzen ihren Inhalt 
erst in einer Box, um deren Breite zu ermitteln und diese anschließend 
auszugeben. In Kombination mit \Package{auto-pst-pdf} führt das zu einer 
doppelten Erstellung der gewünschten Abbildung. Um dies zu vermeiden, müssen 
die durch \Package{floatrow} bereitgestellten Befehle \enquote{unschädlich} 
gemacht werden. Die fraglichen Befehlen akzeptieren allerdings bis zu drei 
optionale Argumente \emph{vor} den beiden obligatorischen, was für die 
Benutzerschnittstelle für die (Re-)Definition durch \hologo{LaTeXe} 
normalerweise nicht vorgesehen ist. Deshalb wird das Paket \Package{xparse} 
geladen, mit welchem dies möglich wird. Genaueres dazu ist der dazugehörigen 
Paketdokumentation zu entnehmen. Mit folgendem Quelltextauszug lassen sich die 
\Package{floatrow}"=Befehle zusammen mit der 
\Environment{pspicture}(\Package{pstricks})'none'"=Umgebung wie gewohnt 
verwenden.
%
\begin{quoting}
\begin{Code}
\usepackage{floatrow}
\usepackage{xparse}
\ifpdf\else
  \RenewDocumentCommand{\fcapside}{ooo+m+m}{#4#5}
  \RenewDocumentCommand{\ttabbox}{ooo+m+m}{#4#5}
  \RenewDocumentCommand{\ffigbox}{ooo+m+m}{#4#5}
\fi
\end{Code}
\end{quoting}

\minisec{Die parallele Nutzung von \Package{tikz} und \Package{todonotes}}
Mit dem Paket \Package{tikz}~-- und auch allen anderen Paketen die 
selbiges nutzen wie beispielsweise \Package{todonotes}~-- gibt es in Verbindung 
mit \Package{auto-pst-pdf} ebenfalls Probleme. Lösen lässt sich dieses Dilemma, 
indem die fraglichen Pakete lediglich geladen werden, wenn \Engine{pdfLaTeX} 
aktiv ist.
%
\begin{quoting}[rightmargin=0pt]
\begin{Code}
\ifpdf
  \usepackage{tikz}%\dots gegebenenfalls weitere auf tikz basierende Pakete
\fi
\end{Code}
\end{quoting}




\part{Anhang}
\appendix
\chapter{Weiterführende Installationshinweise}
\label{sec:install:ext}
%
\noindent\Attention{%
  Im Folgenden werden unterschiedliche Varianten erläutert, wie die jeweils 
  aktuelle Version von \TUDScript genutzt werden kann, falls bereits eine 
  frühere Variante als \textbf{lokale Nutzerinstallation} verwendet wird. 
  Unabhängig davon, für welche Möglichkeit Sie sich entscheiden, ist eine 
  abermalige Installation der PostScript"=Schriften für eine fehlerfreie 
  Verwendung von \TUDScript zwingend notwendig, wenn Sie bisher \TUDScript vor 
  der Version~v2.02 verwendet haben, da der Installationsprozess der Schriften 
  des \CDs für die Version~v2.02 nochmals angepasst werden musste. Dieser 
  Schritt ließ sich leider nicht vermeiden. Für zukünftige Versionen kann 
  darauf hoffentlich verzichtet werden, sodass Aktualisierungen ausschließlich 
  über CTAN eingespielt werden können. In \autoref{sec:install} wird der  
  Installationsprozess der PostScript"=Schriften beschrieben.
}

\bigskip\noindent
Bis zur Version~v2.01 wurde \TUDScript ausschließlich über das \Forum zur 
lokalen Nutzerinstallation angeboten. In erster Linie hat das historische 
Hintergründe und hängt mit der Entstehungsgeschichte von \TUDScript zusammen. 

Eine lokale Nutzerinstallation bietet mehr oder weniger genau einen Vorteil. 
Treten bei der Verwendung von \TUDScript Probleme auf, können diese im Forum 
gemeldet und diskutiert werden. Ist für ein solches Problem tatsächlich eine 
Fehlerkorrektur respektive eine Aktualisierung von \TUDScript nötig, kann 
diese schnell und unkompliziert über das \GitHubRepo bereitgestellt und durch 
den Anwender sofort genutzt werden.

Dies hat allerdings für Anwender, welche das Forum relativ wenig oder gar 
nicht besuchen, den großen Nachteil, dass diese nicht von Aktualisierungen, 
Verbesserungen und Fehlerkorrekturen neuer Versionen profitieren können. Auch 
sämtliche nachfolgenden Bugfixes und Aktualisierungen des \TUDScript-Bundles 
müssen durch den Anwender manuell durchgeführt werden. Daher wird in Zukunft 
die Verbreitung via \hrfn{http://www.ctan.org/pkg/tudscr}{CTAN} präferiert, so 
dass \TUDScript stets in der aktuellen Version verfügbar ist~-- eine durch den 
Anwender aktuell gehaltene \hologo{LaTeX}"=Distribution vorausgesetzt. Der 
einzige Nachteil bei diesem Ansatz ist, dass die Verbreitung eines Bugfixes 
über das \hrfn{http://www.ctan.org/}{Comprehensive TeX Archive Network (CTAN)} 
und die anschließende Bereitstellung durch die verwendete Distribution für 
gewöhnlich mehrere Tage dauert.

Die gängigen \hologo{LaTeX}"=Distributionen durchsuchen im Regelfall zuerst das 
lokale \Path{texmf}"=Nutzerverzeichnis nach Klassen und Paketen und erst daran 
anschließend den \Path{texmf}"=Pfad der Distribution selbst. Dabei spielt es 
keine Rolle, in welchem Pfad die neuere Version einer Klasse oder eines Paketes 
liegt. Sobald im Nutzerverzeichnis die gesuchte Datei gefunden wurde, wird die 
Suche beendet.
\Attention{%
  In der Konsequenz bedeutet dies, dass sämtliche Aktualisierungen über CTAN 
  nicht zum Tragen kommen, falls \TUDScript als lokale Nutzerversion  
  installiert wurde.
}

Deshalb wird Anwendern, die \TUDScript in der Version~v2.01 oder älter nutzen 
und sich nicht \emph{bewusst} für eine lokale Nutzerinstallation entschieden 
haben, empfohlen, diese zu deinstallieren. Der Prozess der Deinstallation wird 
in \autoref{sec:local:uninstall} erläutert. Wird diese einmalig durchgeführt, 
können Updates des \TUDScript-Bundles durch die Aktualisierungsfunktion der 
Distribution erfolgen. Wie das \TUDScript-Bundle trotzdem als lokale 
Nutzerversion installiert oder aktualisiert werden kann, ist in 
\autoref{sec:local:install} beziehungsweise \autoref{sec:local:update} zu 
finden. Der Anwender sollte in diesem Fall allerdings genau wissen, was er 
damit bezweckt, da er in diesem Fall für die Aktualisierung von \TUDScript 
selbst verantwortlich ist.



\section{Lokale Deinstallation des \TUDScript-Bundles}
\label{sec:local:uninstall}
\index{Deinstallation}
%
Um die lokale Nutzerinstallation zu entfernen, kann für Windows
\hrfn{https://github.com/tud-cd/tudscr/releases/download/uninstall/tudscr\_uninstall.bat}%
{\File{tudscr\_uninstall.bat}} sowie für unixartige Betriebssysteme
\hrfn{https://github.com/tud-cd/tudscr/releases/download/uninstall/tudscr\_uninstall.sh}%
{\File{tudscr\_uninstall.sh}} verwendet werden. Nach der Ausführung des 
jeweiligen Skriptes kann in der Konsole beziehungsweise im Terminal mit
%
\begin{quoting}
\Path{kpsewhich --all tudscrbase.sty}
\end{quoting}
%
überprüft werden, ob die Deinstallation erfolgreich war oder immer noch eine 
lokale Nutzerinstallation vorhanden ist. Es werden alle Pfade ausgegeben, in 
welchen die Datei \File*{tudscrbase.sty} gefunden wird. Erscheint nur noch der 
Pfad der Distribution, ist die \TUDScript-Version von CTAN aktiv und der 
Anwender kann mit dem \TUDScript-Bundle arbeiten. Falls es nicht 
schon passiert ist, müssen dafür lediglich die Schriften des \CDs installiert 
werden (\autoref{sec:install}).

Wird \emph{nur} das lokale Nutzerverzeichnis oder gar kein Verzeichnis 
gefunden, so wird höchstwahrscheinlich eine veraltete Distribution 
verwendet. In diesem Fall wird eine Aktualisierung dieser \emph{unbedingt} 
empfohlen. Sollte dies nicht möglich sein, \emph{muss} \TUDScript als lokale 
Nutzerversion aktualisiert (\autoref{sec:local:update}) beziehungsweise bei der 
erstmaligen Verwendung installiert (\autoref{sec:local:install}) werden. Sollte 
neben dem Pfad der Distribution immer noch mindestens ein weiterer Pfad 
angezeigt werden, so ist weiterhin eine lokale Nutzerversion installiert. In 
diesem Fall hat der Anwender zwei Möglichkeiten:
%
\begin{enumerate}
\item Entfernen der lokalen Nutzerinstallation (manuell)
\item Aktualisierung der lokalen Nutzerversion
\end{enumerate}
%
Die erste Variante wird nachfolgend erläutert, die zweite Möglichkeit wird in 
\autoref{sec:local:update} beschrieben. Nur die manuelle Deinstallation der 
lokalen Nutzerversion \TUDScript ermöglicht dabei die Verwendung der jeweils 
aktuellen CTAN"=Version. Hierfür ist etwas Handarbeit durch den Anwender 
vonnöten. Der in der Konsole beziehungsweise im Terminal mit
%
\begin{quoting}
\Path{kpsewhich --all tudscrbase.sty}
\end{quoting}
%
gefundene~-- zum Ordner der Distribution \emph{zusätzliche}~-- Pfad hat die 
folgende Struktur:
%
\begin{quoting}
\Path{\emph{<Installationspfad>}/tex/latex/tudscr/tudscrbase.sty}
\end{quoting}
%
Um die Nutzerinstallation vollständig zu entfernen, muss als erstes zu 
\Path{\emph{<Installationspfad>}} navigiert werden. Anschließend ist in diesem 
Pfad Folgendes durchzuführen:
%
\settowidth\tempdim{\Path{tex/latex/tudscr/}}%
\begin{description}[labelwidth=\tempdim,labelsep=1em]
\item[\Path{tex/latex/tudscr/}]alle .cls- und .sty-Dateien löschen
\item[\Path{tex/latex/tudscr/}]Ordner \Path{logo} vollständig löschen
\item[\Path{doc/latex/}] Ordner \Path{tudscr} vollständig löschen
\item[\Path{source/latex/}] Ordner \Path{tudscr} vollständig löschen
\end{description}
%
Das Verzeichnis \Path{\emph{<Installationspfad>}/tex/latex/tudscr/fonts} 
\textbf{sollte erhalten bleiben}. Andernfalls müssen die Schriften des \CDs 
abermals wie unter \autoref{sec:install} beschrieben installiert werden.
Zum Abschluss ist in der Kommandozeile beziehungsweise im Terminal der Befehl 
\Path{texhash} aufzurufen. Damit wurde die lokale Nutzerversion entfernt und es 
wird von nun an die Version von \TUDScript genutzt, welche durch die verwendete 
Distribution bereitgestellt wird.



\section{Lokale Installation des \TUDScript-Bundles}
\label{sec:local:install}
\index{Installation!Nutzerinstallation}\index{Nutzerinstallation (lokal)}
%
Für die lokale Nutzerinstallation von \TUDScript inklusive der Schriften des 
\CDs werden für Windows sowie unixartige Betriebssysteme die passenden Skripte 
angeboten. Eine lokale Installation sollte nur von Anwender ausgeführt werden, 
die genau wissen, aus welchen Gründen dies geschehen soll.


\subsection{Lokale Installation von \TUDScript unter Windows}
\index{Installation!Nutzerinstallation}\index{Nutzerinstallation (lokal)}
Für eine lokale Nutzerinstallation des \TUDScript-Bundles und der dazugehörigen 
Schriften für die Distributionen \Distribution{\hologo{TeX}~Live} oder 
\Distribution{\hologo{MiKTeX}} werden neben den Schriftarchiven die Dateien aus
\hrfn{https://github.com/tud-cd/tudscr/releases/download/\vTUDScript/TUD-KOMA-Script\_\vTUDScript\_Windows\_full.zip}%
{\File*{TUD-KOMA-Script\_\vTUDScript\_Windows\_full.zip}} benötigt. Vor der 
Verwendung des Skripts \File{tudscr\_\vTUDScript\_install.bat} sollte 
sichergestellt werden, dass sich \emph{alle} der folgenden Dateien im selben 
Verzeichnis befinden:
%
\settowidth\tempdim{\File{tudscr\_\vTUDScript\_install.bat}}%
\begin{description}[labelwidth=\tempdim,labelsep=1em]
  \item[\File{tudscr\_\vTUDScript.zip}]Archiv mit Klassen- und Paketdateien
  \item[\File{tudscr\_\vTUDScript\_install.bat}]Installationsskript
  \item[\File{Univers\_PS.zip}]Archiv mit Schriftdateien für \Univers
  \item[\File{DIN\_Bd\_PS.zip}]Archiv mit Schriftdateien für \DIN
  \item[\File{tudscrfonts.zip}]Archiv mit Metriken für die
    Schriftinstallation via \Package{fontinst}
\end{description}
%
Beim Ausführen des Installationsskripts werden alle Schriften in das lokale 
Nutzerverzeichnis der jeweiligen Distribution installiert, falls kein anderes 
Verzeichnis explizit angegeben wird. Für Hinweise bei Problemen mit der 
Schriftinstallation sei auf \autoref{sec:install:win} verwiesen.
\Attention{%
  Wird die \hologo{LaTeX}"=Distribution \Distribution{\hologo{MiKTeX}} genutzt, 
  sollte in jedem Fall vor der Ausführung der Installationsskripte ein Update 
  der Distribution durchgeführt werden. Andernfalls wird es unter Umständen im 
  Installationsprozess oder bei der Nutzung von \TUDScript zu Problemen kommen.
}



\subsection{Lokale Installation von \TUDScript unter Linux und OS~X}
\index{Installation!Nutzerinstallation}\index{Nutzerinstallation (lokal)}
Für eine lokale Nutzerinstallation des \TUDScript-Bundles und der dazugehörigen 
Schriften für die Distributionen \Distribution{\hologo{TeX}~Live} oder 
\Distribution{Mac\hologo{TeX}} werden neben den Schriftarchiven die Dateien aus
\hrfn{https://github.com/tud-cd/tudscr/releases/download/\vTUDScript/TUD-KOMA-Script\_\vTUDScript\_Unix\_full.zip}%
{\File*{TUD-KOMA-Script\_\vTUDScript\_Unix\_full.zip}} benötigt. Vor der 
Verwendung des Skripts \File{tudscr\_\vTUDScript\_install.sh} sollte 
sichergestellt werden, dass sich \emph{alle} der folgenden Dateien im selben 
Verzeichnis befinden:
%
\begin{description}[labelwidth=\tempdim,labelsep=1em]
\settowidth\tempdim{\File{tudscr\_\vTUDScript\_install.sh}}%
  \item[\File{tudscr\_\vTUDScript.zip}]Archiv mit Klassen- und Paketdateien
  \item[\File{tudscr\_\vTUDScript\_install.sh}]Installationsskript
    (Terminal: \Path{bash tudscr\_\vTUDScript\_install.sh})
  \item[\File{Univers\_PS.zip}]Archiv mit Schriftdateien für \Univers
  \item[\File{DIN\_Bd\_PS.zip}]Archiv mit Schriftdateien für \DIN
  \item[\File{tudscrfonts.zip}]Archiv mit Metriken für die
    Schriftinstallation via \Package{fontinst}
\end{description}
%
Beim Ausführen des Installationsskripts werden alle Schriften in das lokale 
Nutzerverzeichnis der jeweiligen Distribution installiert. Für Hinweise bei 
Problemen mit der Schriftinstallation sei auf \autoref{sec:install:unix} 
verwiesen.



\section{Lokales Update des \TUDScript-Bundles}
\label{sec:local:update}
\index{Update!Nutzerinstallation}\index{Nutzerinstallation (lokal)}



\subsection{Update des \TUDScript-Bundles ab Version~\NoCaseChange{v}2.02}
Für eine lokale Aktualisierung von \TUDScript auf \vTUDScript{} muss das Archiv
\hrfn{https://github.com/tud-cd/tudscr/releases/download/\vTUDScript/TUD-KOMA-Script\_\vTUDScript\_Windows\_update.zip}%
{TUD-KOMA-Script\_\vTUDScript\_Windows\_update.zip} respektive 
\hrfn{https://github.com/tud-cd/tudscr/releases/download/\vTUDScript/TUD-KOMA-Script\_\vTUDScript\_Unix\_update.zip}%
{TUD-KOMA-Script\_\vTUDScript\_Unix\_update.zip} entpackt und anschließend
\File{tudscr\_\vTUDScript\_update.bat} oder 
\File{tudscr\_\vTUDScript\_update.sh} ausgeführt werden.
\Attention{%
  Die lokale Aktualisierung funktioniert nur, wenn \TUDScript bereits 
  mindestens in der Version~v2.02 entweder als lokale Nutzerversion oder über 
  die Distribution sowie die PostScript"=Schriften installiert sind.%
}

\subsection{Update des \TUDScript-Bundles ab Version~\NoCaseChange{v}2.00}
Mit der Version~v2.02 gab es einige tiefgreifende Änderungen. Deshalb wird für 
vorausgehende Versionen~-- sprich v2.00 und v2.01~-- kein dediziertes Update 
angeboten. Die Aktualisierung kann durch den Anwender entweder~-- wie in 
\autoref{sec:local:install} erläutert~-- mit einer skriptbasierten oder mit 
einer manuellen Neuinstallation erfolgen. Für die zweite Variante muss 
der Inhalt des Archivs
\hrfn{https://github.com/tud-cd/tudscr/releases/download/\vTUDScript/tudscr\_\vTUDScript.zip}%
{\File{tudscr\_\vTUDScript.zip}} in das lokale \Path{texmf}"=Nutzerverzeichnis 
kopiert werden. Des Weiteren wurde die Installation der Schriften überarbeitet. 
Deshalb wird auch für diese eine Neuinstallation (\autoref{sec:install}) sehr 
empfohlen.


\subsection{Update des \TUDScript-Bundles von Version \NoCaseChange{v}1.0}
\begin{Declaration*}{\Class{tudscrbookold}}
\begin{Declaration*}{\Class{tudscrreprtold}}
\begin{Declaration*}{\Class{tudscrartclold}}
%
Ist \TUDScript in der veralteten Version~v1.0 installiert, so wird dringend zu 
einer Deinstallation geraten. Andernfalls wird es zu Problemen kommen. Dafür 
werden die Skripte 
\hrfn{https://github.com/tud-cd/tudscr/releases/download/uninstall/tudscr\_uninstall.bat}{\File{tudscr\_uninstall.bat}}
respektive
\hrfn{https://github.com/tud-cd/tudscr/releases/download/uninstall/tudscr_uninstall.sh}{\File{tudscr\_uninstall.sh}}
bereitgestellt. Die aktuelle Version~\vTUDScript{} kann nach der Deinstallation 
wie in \autoref{sec:install} beschrieben installiert werden.

Möchten Sie die obsoleten \TUDScript-Klassen in der Version~v1.0 nach einer 
Aktualisierung weiterhin nutzen, so müssen diese nach der Deinstallation neu 
installiert werden. Dafür steht das Archiv 
\hrfn{https://github.com/tud-cd/tudscrold/releases/download/v1.0/TUD-KOMA-Script_v1.0old.zip}%
{\File*{TUD-KOMA-Script\_v1.0old.zip}} bereit, welches sowohl die genannten 
Skripte zur Deinstallation als auch die zur neuerlichen Installation der 
veralteten Klassen benötigten \File{tudscr\_v1.0old\_install.bat} oder 
\File{tudscr\_v1.0old\_install.sh} enthält. Nach Abschluss des Vorgangs sind 
die alten Klassen der Version~v1.0 mit \Class{tudscrbookold}, 
\Class{tudscrreprtold} und \Class{tudscrartclold} parallel zur aktuellen 
Version~\vTUDScript{} verwendbar.

Im Vergleich zur Version~v1.0 hat sich an der Benutzerschnittstelle nicht sehr 
viel verändert. Treten nach dem Umstieg auf die Version~\vTUDScript{} dennoch 
Probleme auf, sollte der Anwender als erstes die Beschreibung des Paketes 
\Package{tudscrcomp}'full'(tudscr) lesen, welches eine Schnittstelle zur 
Nutzung alter und ursprünglich nicht mehr vorgesehener Befehle sowie Optionen 
bereitstellt. Allerdings werden einige von diesen auch durch das Paket 
\Package{tudscrcomp}(tudscr) nicht mehr bereitgestellt. Aufgeführt sind diese 
in \autoref{sec:obsolete}. Sollten trotz aller Hinweise dennoch Fehler oder 
Probleme beim Umstieg auf die neue \TUDScript-Version auftreten, ist eine 
Meldung im \Forum die beste Möglichkeit, um Hilfe zu erhalten.
\end{Declaration*}
\end{Declaration*}
\end{Declaration*}



\section{Installationshinweise für portable Installationen}
\label{sec:install:portable}

Prinzipiell ist die Installation der PostScript-Schriften bei der Nutzung von 
\Distribution{\hologo{TeX}~Live~Portable}|?| beziehungsweise 
\Distribution{\hologo{MiKTeX}~Portable}|?| äquivalent zur nicht-portablen 
Variante, welche in \autoref{sec:install} beschrieben wird. Alle dort gegebenen 
Hinweise sollten sorgfältig berücksichtigt werden. Darüberhinaus ist bei 
\Distribution{\hologo{MiKTeX}~Portable} darauf zu achten, den Installationspfad 
nicht unbedingt auf der obersten Verzeichnisebene des externen Speichermediums 
zu wählen.
Zur Installation der Schriften des \CDs wird das Archiv
\hrfn{https://github.com/tud-cd/tudscr/releases/download/fonts/TUD-KOMA-Script_fonts_Windows.zip}%
{\File*{TUD-KOMA-Script\_fonts\_Windows.zip}} benötigt. Dieses kann entweder 
auf der lokalen Festplatte oder auf dem externen Speichermedium entpackt 
werden. Danach wird folgendes Vorgehen empfohlen:

\minisec{\NoCaseChange{\hologo{TeX}}~Live~Portable}
Das folgende Vorgehen wurde mit Windows getestet. Empfehlungen für die portable 
Installation für unixoide Betriebssysteme können gerne an \Email{\tudscrmail} 
gesendet werden.
\begin{enumerate}
\item Installation von \Distribution{\hologo{TeX}~Live~Portable} in den Pfad
  \Path{\PName{Laufwerk}:\textbackslash texlive}
\item Die Datei \File{tl-tray-menu.exe} im Installationspfad öffnen
\item Im Infobereich der Taskleiste mit einem Rechtsklick auf das Symbol von  
  \Distribution{\hologo{TeX}~Live~Portable} das Kontextmenü öffnen und ein 
  Update entweder über die grafische Oberfläche (\emph{Package Manager}) oder 
  die Kommandozeile (\emph{Command Prompt}) durchführen
\item Über das Kontextmenü die Kommandozeile starten und über diese das 
  Installationsskript für die Schriften \File{tudscrfonts\_install.bat} 
  ausführen. Der voreingestellte Installationspfad kann im Normalfall so 
  belassen werden. Wird dieser geändert, so sollte dieser sich logischerweise 
  auf dem externen Speichermedium befinden.
  \Attention{%
    Ein Ausführen ohne die über \Distribution{\hologo{TeX}~Live~Portable} 
    geöffnete Kommandozeile führt zu Fehlern.
  }%
\end{enumerate}

\minisec{\NoCaseChange{\hologo{MiKTeX}}~Portable}
\begin{enumerate}
\item Installation von \Distribution{\hologo{MiKTeX}~Portable} in den Pfad
  \Path{\PName{Laufwerk}:\textbackslash LaTeX\textbackslash MiKTeXportable}
\item Die Datei \File{miktex-portable.cmd} im Installationspfad öffnen
\item Im Infobereich der Taskleiste mit einem Rechtsklick auf das Symbol von  
  \Distribution{\hologo{MiKTeX}~Portable} das Kontextmenü öffnen und ein Update 
  durchführen
\item Über das Kontextmenü die Kommandozeile starten und über diese das 
  Installationsskript für die Schriften \File{tudscrfonts\_install.bat} 
  ausführen. Der voreingestellte Installationspfad kann im Normalfall so 
  belassen werden. Wird dieser geändert, so sollte dieser sich logischerweise 
  auf dem externen Speichermedium befinden. Bei diesem Schritt werden die 
  Pakete \Package{fontinst}, \Package{cmbright} und \Package{iwona} unter 
  Umständen nachinstalliert.
  \Attention{%
    Ein Ausführen ohne die über \Distribution{\hologo{MiKTeX}~Portable} 
    geöffnete Kommandozeile führt zu Fehlern.
  }%
\item Bei der erstmaligen Verwendung einer der \TUDScript-Dokumentklassen 
  werden die Pakete \Package{tudscr}, \Package{koma-script}, 
  \Package{etoolbox}, \Package{textcase}, \Package{environ}, 
  \Package{trimspaces}, \Package{xcolor}, \Package{mptopdf}() durch 
  \Distribution{\hologo{MiKTeX}~Portable} nachinstalliert, falls diese nicht 
  schon vorhanden sind und die automatische Nachinstallation von Paketen 
  aktiviert ist.
\end{enumerate}

\chapter{Obsolete sowie vollständig entfernte Optionen und Befehle}
\label{sec:obsolete}%
%
\section{Veraltete Optionen und Befehle in \TUDScript}
Einige Optionen und Befehle waren während der Weiterentwicklung von \TUDScript
in ihrer ursprünglichen Form nicht mehr umsetzbar oder wurden~-- unter anderem 
aus Gründen der Kompatibilität zu anderen Paketen~-- schlichtweg verworfen. 
Dennoch besteht für die meisten entfallenen Direktiven eine Möglichkeit, deren 
Funktionalität ohne größere Aufwände mit \TUDScript in der aktuellen Version 
\vTUDScript{} darzustellen. Ist dies der Fall, wird hier entsprechend kurz 
darauf hingewiesen.


\ToDo[imp]{vskip-lastskip überprüfen}[v2.05]
\subsection{Änderungen für \TUDScript~v2.00}
\vskip-\lastskip
\ChangedAt*{v2.00:Änderungen gegenüber der vorhergehenden Version}%
\begin{Obsolete}{v2.00}{\Option{cd}[alternative]}
\begin{Obsolete}{v2.00}{\Option{cdtitle}[alternative]}
\begin{Obsolete}{v2.00}{\Length{titlecolwidth}}
\begin{Obsolete}{v2.00}{\Term{authortext}}
\printobsoletelist%
%
Die alternative Titelseite ist komplett aus dem \TUDScript-Bundle entfernt 
worden. Dementsprechend entfallen auch die dazugehörigen Optionen sowie Länge 
und Bezeichner.
\end{Obsolete}
\end{Obsolete}
\end{Obsolete}
\end{Obsolete}

\begin{Obsolete}{v2.00:\Option{cd}}{\Option{color}[\PBoolean]}
\printobsoletelist%
%
Die Einstellungen der farbigen Ausprägung des Dokumentes erfolgt über die 
Option \Option*{cd}.
\end{Obsolete}

\begin{Obsolete}{v2.00:\Option{cdfont}}{\Option{tudfonts}[\PBoolean]}
\begin{Obsolete}{v2.00:\Option{cdfont}}{\Option{cdfonts}[\PBoolean]}
\printobsoletelist%
%
Die Option zur Schrifteinstellung ist wesentlich erweitert worden. Aus Gründen 
der Konsistenz wurde diese umbenannt.
\end{Obsolete}
\end{Obsolete}

\begin{Obsolete}{v2.00:\Option{cdfoot}}{\Option{tudfoot}[\PBoolean]}
\printobsoletelist%
%
Ebenso wurde die Option \Option*{tudfoot} umbenannt, um dem Namensschema der 
restlichen Optionen von \TUDScript zu entsprechen.
\end{Obsolete}

\begin{Obsolete}{v2.00}{\Option{headfoot}[\PSet]}{%
  \seeref{\KOMAScript-Optionen \Option*{headinclude} und \Option*{footinclude}}%
}
\printobsoletelist%
%
Diese Option war für \TUDScript in der Version~v1.0 notwendig, um die parallele 
Verwendung der beiden Pakete \Package*{typearea} und \Package*{geometry} zu 
ermöglichen. Die Erstellung des Satzspiegels wurde komplett überarbeitet. 
Mittlerweile werden an das Paket \Package*{geometry} die Einstellungen für die 
\KOMAScript"=Optionen \Option*{headinclude} und \Option*{footinclude} direkt 
weitergereicht, so dass die Option \Option*{headfoot} nicht mehr notwendig ist 
und deshalb entfernt wurde.
\end{Obsolete}

\begin{Obsolete}{v2.00:\Option{cleardoublespecialpage}}{%
  \Option{partclear}[\PBoolean]%
}
\begin{Obsolete}{v2.00:\Option{cleardoublespecialpage}}{%
  \Option{chapterclear}[\PBoolean]%
}
\printobsoletelist%
%
Beide Optionen sind in der neuen Option \Option*{cleardoublespecialpage} 
aufgegangen, womit ein konsistentes Layout erreicht wird. Die ursprünglichen 
Optionen entfallen. 
\end{Obsolete}
\end{Obsolete}

\begin{Obsolete}{v2.00:\Option{abstract}}{\Option{abstracttotoc}[\PBoolean]}
\begin{Obsolete}{v2.00:\Option{abstract}}{\Option{abstractdouble}[\PBoolean]}
\printobsoletelist%
%
Beide Optionen wurden in die Option \Option*{abstract} integriert und sind 
deshalb überflüssig.
\end{Obsolete}
\end{Obsolete}

\begin{Obsolete}{v2.00:\Macro{headlogo}}{\Macro{logofile}\Parameter{Dateiname}}
\printobsoletelist%
%
Der Befehl \Macro*{logofile} wurde in \Macro*{headlogo} umbenannt, wobei die 
Funktionalität weiterhin bestehen bleibt.
\end{Obsolete}

\begin{Obsolete}{v2.00:\Option{tudbookmarks}}{\Option{bookmarks}[\PBoolean]}
\printobsoletelist%
%
Umbenannt, um Überschneidungen mit \Package*{hyperref} zu vermeiden.
\end{Obsolete}

\begin{Obsolete}{v2.00}{\Length{signatureheight}}
\printobsoletelist%
%
Die Höhe für die Zeile der Unterschriften wurde dehnbar gestaltet, eine etwaige 
Anpassung durch den Anwender ist nicht vonnöten.
\end{Obsolete}

\begin{Obsolete}{v2.00:\Macro{titledelimiter}}{\Term{titlecoldelim}}%
\printobsoletelist%
%
Das Trennzeichen für Bezeichnungen beziehungsweise beschreibende Texte und dem 
eigentlichen Feld auf der Titelseite ist nicht mehr sprachabhängig und wurde 
umbenannt.
\end{Obsolete}

\begin{Obsolete}{v2.00:\Macro{declaration}}{\Macro{confirmationandrestriction}}
\begin{Obsolete}{v2.00:\Macro{declaration}}{\Macro{restrictionandconfirmation}}
\printobsoletelist%
%
Die beiden Befehle entfallen, stattdessen sollte entweder der Befehl 
\Macro*{declaration} oder die Umgebung \Environment*{declarations} zusammen mit 
den Befehlen \Macro*{confirmation} und \Macro*{blocking} verwendet werden, 
wobei sich diese in der Umgebung in beliebiger Reihenfolge anordnen lassen.
\end{Obsolete}
\end{Obsolete}

\begin{Obsolete}{v2.00:\Macro{place}}{\Macro{location}\Parameter{Ort}}
\printobsoletelist%
%
In Anlehnung an andere \hologo{LaTeX}-Pakete und "~Klassen wurde 
\Macro*{location} in \Macro*{place} umbenannt.
\end{Obsolete}

\minisec{\taskname}
\begin{Bundle}{\Package{tudscrsupervisor}}
Die Umgebung für die Erstellung einer Aufgabenstellung für eine 
wissenschaftliche Arbeit wurde in das Paket \Package{tudscrsupervisor} 
ausgelagert. Dieses muss für die Verwendung der Umgebung \Environment*{task} 
und der daraus abgeleiteten standardisierten Form zwingend geladen werden.

\begin{Obsolete}{v2.00:\Environment{task}}{\Option{cdtask}[\PSet]}
\begin{Obsolete}{v2.00}{\Option{taskcompact}[\PBoolean]}
\begin{Obsolete}{v2.00}{\Length{taskcolwidth}}
\printobsoletelist%
%
Die Klassenoption \Option*{cdtask} ist komplett entfernt worden, alle 
Einstellungen, erfolgen direkt über das optionale Argument der Umgebung 
\Environment*{task}. Die Variante eines kompakten Kopfes mit der Option 
\Option*{taskcompact} wird nicht mehr bereitgestellt. Die Möglichkeit zur 
manuellen Festlegung der Spaltenbreite für den Kopf der Aufgabenstellung mit 
\Length*{taskcolwidth} wurde aufgrund der verbesserten automatischen Berechnung 
entfernt.
\end{Obsolete}
\end{Obsolete}
\end{Obsolete}

\begin{Obsolete}{v2.00:\Macro{taskform}}{%
  \Macro{tasks}\Parameter{Ziele}\Parameter{Schwerpunkte}%
}
\begin{Obsolete}{v2.00:\Term{focusname}}{\Term{focustext}}
\begin{Obsolete}{v2.00:\Term{objectivesname}}{\Term{objectivestext}}
\printobsoletelist%
%
Der Befehl \Macro*{tasks} wurde in \Macro*{taskform} umbenannt und in der 
Funktionalität erweitert. Die darin verendeten Bezeichner wurden ebenfalls 
leicht abgewandelt.
\end{Obsolete}
\end{Obsolete}
\end{Obsolete}

\begin{Obsolete}{v2.00:\Macro{matriculationnumber}}{%
  \Macro{studentid}\Parameter{Matrikelnummer}%
}
\begin{Obsolete}{v2.00:\Macro{matriculationyear}}{%
  \Macro{enrolmentyear}\Parameter{Immatrikulationsjahr}%
}
\begin{Obsolete}{v2.00:\Macro{date}}{\Macro{submissiondate}\Parameter{Datum}}
\begin{Obsolete}{v2.00:\Macro{dateofbirth}}{%
  \Macro{birthday}\Parameter{Geburtsdatum}%
}
\begin{Obsolete}{v2.00:\Macro{placeofbirth}}{%
  \Macro{birthplace}\Parameter{Geburtsort}%
}
\begin{Obsolete}{v2.00:\Macro{issuedate}}{%
  \Macro{startdate}\Parameter{Ausgabedatum}%
}
\printobsoletelist%
%
Alle Befehle wurden umbenannt und sind jetzt neben der \taskname{} auch für die 
Titelseite im \CD nutzbar.
\end{Obsolete}
\end{Obsolete}
\end{Obsolete}
\end{Obsolete}
\end{Obsolete}
\end{Obsolete}

\begin{Obsolete}{v2.00:\Term{matriculationnumbername}}{\Term{studentidname}}
\begin{Obsolete}{v2.00:\Term{matriculationyearname}}{\Term{enrolmentname}}
\begin{Obsolete}{v2.00:\Term{datetext}}{\Term{submissiontext}}
\begin{Obsolete}{v2.00:\Term{dateofbirthtext}}{\Term{birthdaytext}}
\begin{Obsolete}{v2.00:\Term{placeofbirthtext}}{\Term{birthplacetext}}
\begin{Obsolete}{v2.00:\Term{supervisorothername}}{\Term{supervisorIIname}}
\begin{Obsolete}{v2.00:\Term{defensedatetext}}{\Term{defensetext}}
\begin{Obsolete}{v2.00:\Term{issuedatetext}}{\Term{starttext}}
\begin{Obsolete}{v2.00:\Term{duedatetext}}{\Term{duetext}}
\printobsoletelist%
%
Die Bezeichner wurden in Anlehnung an die dazugehörigen Befehlsnamen umbenannt.
\end{Obsolete}
\end{Obsolete}
\end{Obsolete}
\end{Obsolete}
\end{Obsolete}
\end{Obsolete}
\end{Obsolete}
\end{Obsolete}
\end{Obsolete}
\end{Bundle}

\subsection{Änderungen für \TUDScript~v2.02}
\vskip-\lastskip
\ChangedAt*{v2.02:Änderungen gegenüber der vorhergehenden Version}%
\begin{Obsolete}{v2.02:\Length{pageheadingsvskip}}{\Length{chapterheadingvskip}}
\printobsoletelist%
%
Die vertikale Positionierung von Überschriften wurde aufgeteilt. Zum einen kann 
diese für Titel-, Teile- und Kapitelseiten (\Option*{chapterpage}[true]) über 
die Länge \Length*{pageheadingsvskip} geändert werden. Für Kapitelüberschriften 
(\Option*{chapterpage}[false]) sowie den Titelkopf (\Option*{titlepage}[false]) 
kann dies unabhängig davon mit \Length*{headingsvskip} erfolgen.
\end{Obsolete}

\begin{Obsolete}{v2.02:\Macro{graduation}}{%
  \Macro{degree}\OParameter{Abk.}\Parameter{Grad}%
}
\begin{Obsolete}{v2.02:\Term{graduationtext}}{\Term{degreetext}}
\printobsoletelist%
%
Der Befehl wurde zur Erhöhung der Kompatibilität mit anderen Paketen umbenannt, 
der dazugehörige Bezeichner dahingehend angepasst.
\end{Obsolete}
\end{Obsolete}

\begin{Obsolete}{v2.02:\Macro{blocking}}{%
  \Macro{restriction}\OLParameter{Firma}%
}
\begin{Obsolete}{v2.02:\Term{blockingname}}{\Term{restrictionname}}
\begin{Obsolete}{v2.02:\Term{blockingtext}}{\Term{restrictiontext}}
\printobsoletelist%
%
Der Befehl wurde zur Erhöhung der Kompatibilität mit anderen Paketen umbenannt, 
die dazugehörigen Bezeichner dahingehend angepasst.
\end{Obsolete}
\end{Obsolete}
\end{Obsolete}

\begin{Obsolete}{}{\Environment{tudpage}[\OLParameter{Sprache}]}
\begin{Obsolete}{v2.02:\Key{\Environment{tudpage}}{pagestyle}}{%
  \Key{\Environment{tudpage}}{head}[\PSet]
}
\begin{Obsolete}{v2.02:\Key{\Environment{tudpage}}{pagestyle}}{%
  \Key{\Environment{tudpage}}{foot}[\PSet]
}
\printobsoletelist%
%
Diese beiden Parameter der Umgebung \Environment*{tudpage} wurden in ihrer 
Funktionalität durch den Parameter \Key*{\Environment{tudpage}}{pagestyle} 
ersetzt.
\end{Obsolete}
\end{Obsolete}
\end{Obsolete}



\minisec{Änderungen im Paket \Package*{tudscrsupervisor}}
Im Paket \Package{tudscrsupervisor} gab es ein paar kleinere Anpassungen.
\begin{Bundle}{\Package{tudscrsupervisor}}
\vskip-\lastskip
\begin{Obsolete}{v2.02:\Macro{discipline}}{%
  \Macro{branch}\Parameter{Studienrichtung}%
}
\begin{Obsolete}{v2.02:\Term{disciplinename}}{\Term{branchname}}
\printobsoletelist%
%
Für die \taskname{} wurden der Befehl sowie der dazugehörige Bezeichner 
umbenannt.
\end{Obsolete}
\end{Obsolete}

\begin{Obsolete}{v2.02:\Macro{contactperson}}{%
  \Macro{contact}\Parameter{Kontaktperson(en)}%
}
\begin{Obsolete}{v2.02:\Term{contactpersonname}}{\Term{contactname}}
\begin{Obsolete}{v2.02:\Macro{telephone}}{%
  \Macro{phone}\Parameter{Telefonnummer}%
}
\begin{Obsolete}{v2.02:\Macro{emailaddress}}{%
  \Macro{email}\Parameter{E-Mail-Adresse}%
}
\printobsoletelist%
%
Alle genannten Befehle und Bezeichner wurden für den \noticename{} umbenannt.
\end{Obsolete}
\end{Obsolete}
\end{Obsolete}
\end{Obsolete}
\end{Bundle}

\subsection{Änderungen für \TUDScript~v2.03}
\vskip-\lastskip
\ChangedAt*{v2.03:Änderungen gegenüber der vorhergehenden Version}%
\begin{Obsolete}{v2.03:\Option{cdgeometry}}{\Option{geometry}[\PBoolean]}
\printobsoletelist%
%
Die Option \Option*{geometry} wurde aus Gründen der Konsistenz und dem 
Vermeiden eines möglichen Konfliktes mit einer späteren \KOMAScript-Version 
umbenannt. An der Funktionalität wurde nichts geändert.
\end{Obsolete}

\begin{Obsolete}{v2.03:\Option{cdmath}}{\Option{sansmath}[\PBoolean]}
\printobsoletelist%
%
Die Option \Option*{sansmath} wurde aus Gründen der Konsistenz umbenannt. 
Zusätzlich wurde die Funktionalität erweitert.
\end{Obsolete}

\begin{Obsolete}{v2.03:\Option{cdhead}}{\Option{barfont}[\PSet]}
\begin{Obsolete}{v2.03:\Option{cdhead}}{\Option{widehead}[\PBoolean]}
\printobsoletelist%
%
Die Funktionalitäten der Optionen \Option*{barfont} und \Option*{widehead} 
wurden in der Option \Option*{cdhead} zusammengefasst und erweitert.
\end{Obsolete}
\end{Obsolete}

\begin{Obsolete}{}{\Environment{tudpage}[\OLParameter{Sprache}]}
\begin{Obsolete}{v2.03}{\Key{\Environment{tudpage}}{color}[\PName{Farbe}]}
\printobsoletelist%
%
Dieser Parameter der \Environment*{tudpage}-Umgebung wurde ersatzlos entfernt.
\end{Obsolete}
\end{Obsolete}


\subsection{Änderungen für \TUDScript~v2.04}
\vskip-\lastskip
\ChangedAt*{v2.04:Änderungen gegenüber der vorhergehenden Version}%
\begin{Obsolete}{v2.04}{\Option{fontspec}[\PBoolean]}[false]%
\printobsoletelist%
%
Anstatt die Option \Option*{fontspec} zu aktivieren, kann einfach das Paket 
\Package{fontspec} in der Dokumentpräambel geladen werden. Dadurch können  
anschließend zusätzliche Pakete genutzt werden, die auf die Verwendung von 
\Package{fontspec} angewiesen sind. Sollte die Option \Option*{fontspec} 
dennoch genutzt werden, müssen alle auf das Paket \Package{fontspec} aufbauende 
Einstellungen mit \Macro*{AfterPackage}\PParameter{fontspec}\PParameter{\dots} 
durch den Anwender verzögert werden. In \fullref{sec:fonts:fontspec} sind 
weitere Hinweise zur Verwendung des Paketes \Package{fontspec} zu finden.
\begin{values}
\itemfalse*
  Die Hausschriften im Stil des \CDs der \TnUD werden im PostScript"=Format 
  eingebunden. Sowohl Kerning als auch der mathematische Satz funktionieren 
  problemlos.
\itemtrue*
   Es werden die OpenType"=Varianten der Hausschriften verwendet. Dazu wird das 
   Paket \Package{fontspec} geladen, welches lediglich mit \hologo{LuaLaTeX} 
   oder \hologo{XeLaTeX} jedoch nicht mit \hologo{pdfLaTeX} als genutzt werden 
   kann. Sowohl beim mathematischen Satz als auch beim Kerning der Schriften 
   kann es zu Problemen kommen. Hierfür müssen die OpenType"=Schriften auf dem 
   Betriebssystem installiert sein. Die Verwendung dieser Einstellung sollte 
   nur für den Fall erfolgen, dass eine Installation der PostScript"=Schriften 
   nicht möglich ist. 
\end{values}
\end{Obsolete}


\section{Das Paket \Package*{tudscrcomp} -- Umstieg von anderen Klassen}
\begin{Bundle!}{\Package{tudscrcomp}}
\index{Kompatibilität!\Class{tudbook}|(}%
\index{Kompatibilität!\Class{tudmathposter}|(}%
%
\ToDo[imp]{Unterstützung für alle Klassen von Klaus Bergmann prüfen}[v2.05]
\ToDo[doc]{keine Sternversion von printdeclarationlist}[v2.05]
\ToDo[imp]{Unterstützung für \Class{tudmathposter}}[v2.05]
%
\noindent\Attention{%
  Sollten Sie eine der Klassen \Class{tudbook}|?|, \Class{tudbeamer}|?|, 
  \Class{tudletter}|?|, \Class{tudfax}|?|, \Class{tudhaus}|?| und 
  \Class{tudform}|?| sowie \Class{tudmathposter}|?| oder \TUDScript in der 
  Version~v1.0 nie genutzt haben, können Sie dieses \autorefname ohne Weiteres 
  überspringen. Sämtliche hier vorgestellten Optionen und Befehle sind in der 
  aktuellen Version von \TUDScript obsolet.
}

\bigskip\noindent
Zu Beginn der Entwicklung von \TUDScript bildete die Klasse \Class{tudbook}
die Basis. Ziel war es, sämtliche Funktionalitäten dieser beizubehalten und 
zusätzlich den vollen Funktionsumfang der \KOMAScript-Klassen nutzbar zu 
machen. Bei der kompletten Neuimplementierung der \TUDScript-Klassen wurde sehr 
viel verändert und verbessert. Einige der Optionen und Befehle waren jedoch 
bereits in der \TUDScript-Version~v1.0 Relikte, um die Kompatibilität zur 
\Class{tudbook}-Klasse und ihren Derivaten zu gewährleisten. Mit \TUDScript in 
der Version~v2.00 wurden einige aus Gründen der Konsistenz lediglich umbenannt, 
andere wiederum wurden vollständig entfernt oder über neue Befehle und Optionen 
in ihrer Funktionalität ersetzt und erweitert. 

Das Paket \Package{tudscrcomp} dient der Überführung von alten Dokumenten, die 
entweder mit der \Class{tudbook}-Klasse, ihren Derivaten oder mit \TUDScript in 
der Version~v1.0 erstellt wurden, auf \TUDScript~\vTUDScript. Es werden einige 
Optionen und Befehle bereitgestellt, welche von den alten Klassen definiert 
wurden und das entsprechende Verhalten nachahmen. Damit soll die Kompatibilität 
bei der Änderung der Dokumentklasse sichergestellt werden. Die Intention ist, 
alte Dokumente möglichst schnell und einfach auf die \TUDScript-Klassen 
portieren zu können. Des Weiteren ist beschrieben, wie sich die Funktionalität 
ohne die Verwendung des Paketes \Package{tudscrcomp} mit den Mitteln von 
\TUDScript umsetzen lassen. Für den Satz neuer Dokumente wird empfohlen, auf 
den Einsatz dieses Paketes komplett zu verzichten und stattdessen die neuen 
Befehle zu nutzen.

\begin{Declaration}{\Macro{einrichtung}\Parameter{Fakultät}}{%
  identisch zu \Macro*{faculty}%
}
\begin{Declaration}{\Macro{fachrichtung}\Parameter{Einrichtung}}{%
  identisch zu \Macro*{department}%
}
\begin{Declaration}{\Macro{institut}\Parameter{Institut}}{%
  identisch zu \Macro*{institute}%
}
\begin{Declaration}{\Macro{professur}\Parameter{Lehrstuhl}}{%
  identisch zu \Macro*{chair}%
}
\printdeclarationlist%
%
Dies sind die deutschsprachigen Befehle für den Kopf im \CD.
\end{Declaration}
\end{Declaration}
\end{Declaration}
\end{Declaration}

\begin{Declaration}{\Option{serifmath}}{%
  identisch zu \Option*{cdmath}[false]%
}
\printdeclarationlist%
%
Die Funktionalität wird durch die Option \Option*{cdmath} bereitgestellt.
\end{Declaration}

\begin{Declaration}{\Macro{tudfont}\Parameter{Scriftart}}{%
  identisch zu \Macro*{cdfont}%
}
\printdeclarationlist%
%
Die direkte Auswahl der Schriftart sollte mit \Macro*{cdfont} erfolgen. 
Zusätzlich gibt es den Befehl \Macro*{textcdfont}, mit dem die Auszeichnung 
eines bestimmten Textes in einer anderen Schriftart erfolgen kann, ohne die 
Dokumentschrift umzuschalten.
\end{Declaration}
\ToDo[imp]{Unterstützung für \Class{tudmathposter} ENDE}[v2.05]
\index{Kompatibilität!\Class{tudmathposter}|)}%


\subsection{Optionen und Befehle aus \Class*{tudbook} \& Co.}
\begin{Declaration}{\Option{colortitle}}{%
  identisch zu \Option*{cdtitle}[color]%
}
\begin{Declaration}{\Option{nocolortitle}}{%
  identisch zu \Option*{cdtitle}[true]%
}
\printdeclarationlist%
%
Die Funktionalität wird durch die Option \Option*{cdtitle} bereitgestellt.
\end{Declaration}
\end{Declaration}

\begin{Declaration}{\Macro{moreauthor}\Parameter{Autorenzusatz}}{%
  identisch zu \Macro*{authormore}%
}
\printdeclarationlist%
%
Ursprünglich war diese Befehl für das Unterbringen aller möglichen, 
zusätzlichen Autoreninformationen gedacht. Auch der Befehl \Macro*{authormore} 
ist ein Rudiment davon. Empfohlen wird die Verwendung der Befehle 
\Macro*{dateofbirth}, \Macro*{placeofbirth}, \Macro*{matriculationnumber} und 
\Macro*{matriculationyear} sowie \Macro*{course}(\Package{tudscrsupervisor}) 
und \Macro*{discipline}(\Package{tudscrsupervisor}) aus dem Paket 
\Package{tudscrsupervisor} für die Aufgabenstellung einer wissenschaftlichen 
Arbeit.
\end{Declaration}

\begin{Declaration}{\Macro{submitdate}\Parameter{Datum}}{%
  identisch zu \Macro*{date}%
}
\printdeclarationlist%
%
Die Funktionalität wird durch den erweiterten Standardbefehl \Macro*{date} 
abgedeckt.
\end{Declaration}

\begin{Declaration}{\Macro{supervisorII}\Parameter{Name}}{%
  identisch zur Verwendung von \Macro*{and} innerhalb von \Macro*{supervisor}%
}
\printdeclarationlist%
%
Es ist \Macro*{supervisor}\PParameter{\PName{Name} \Macro*{and} \PName{Name}}
statt \Macro*{supervisorII}\Parameter{Name} zu verwenden.
\end{Declaration}

\begin{Declaration}{\Macro{supervisedby}\Parameter{Bezeichnung}}{%
  siehe \Term*{supervisorname}%
}
\begin{Declaration}{\Macro{supervisedIIby}\Parameter{Bezeichnung}}{%
  siehe \Term*{supervisorothername}%
}
\begin{Declaration}{\Macro{submittedon}\Parameter{Bezeichnung}}{%
  siehe \Term*{datetext}%
}
\printdeclarationlist%
%
Zur Änderung der Bezeichnung der Betreuer sollten die sprachabhängigen 
Bezeichner wie in \autoref{sec:localization} beschrieben angepasst werden. Eine 
Verwendung der alten Befehle entfernt die Abhängigkeit der Bezeichner von der 
verwendeten Sprache.
\end{Declaration}
\end{Declaration}
\end{Declaration}

\begin{Declaration}{\Option{ddcfooter}}{%
  identisch zu \Option*{ddcfoot}[true]%
}
\printdeclarationlist%
%
Die Funktionalität wird durch die Option \Option*{ddcfoot} bereitgestellt.
\end{Declaration}

\begin{Declaration}{\Macro{dissertation}}
\printdeclarationlist%
%
Die Funktionalität kann durch die Befehle \Macro*{thesis}\PParameter{diss} und 
\Macro*{referee} sowie die Bezeichner \Term*{refereename} und 
\Term*{refereeothername} dargestellt werden.
\end{Declaration}

\begin{Declaration}{\Macro{chapterpage}}
\printdeclarationlist%
%
Durch diesen Befehl können Kapitelseiten konträr zur eigentlichen Einstellung 
aktiviert oder deaktiviert werden. Prinzipiell ist dies auch durch eine 
Änderung der Option \Option*{chapterpage} möglich. Allerdings wird davon 
abgeraten, da dies zu einem inkonsistenten Layout innerhalb des Dokumentes 
führt.
\end{Declaration}

\begin{Declaration}{\Environment{theglossary}[\OParameter{Präambel}]}
\begin{Declaration}{\Macro{glossitem}\Parameter{Begriff}}
\printdeclarationlist%
%
Die \Class{tudbook}-Klasse stellt eine rudimentäre Umgebung für ein Glossar 
bereit. Allerdings gibt es dafür bereits zahlreiche und besser implementierte 
Pakete. Daher wird für diese Umgebung keine Portierung vorgenommen, sondern 
lediglich die ursprüngliche Definition übernommen. Allerdings sein an dieser 
Stelle auf wesentlich bessere Lösungen wie beispielsweise das Paket 
\Package{glossaries} oder~-- mit Abstrichen~-- das nicht ganz so umfangreiche 
Paket \Package{nomencl} verwiesen.
\end{Declaration}
\end{Declaration}
\index{Kompatibilität!\Class{tudbook}|)}%
%
%
%\subsection{Optionen und Befehle aus \Class{tudmathposter}}
\ToDo[imp]{Unterstützung für \Class{tudmathposter}}[v2.05]
\index{Kompatibilität!\Class{tudmathposter}|(}%
%\ToDo[imp]{interne Schalter beim Laden von tudbook bzw. tudmathposter}
%\ToDo[imp]{Optionen und Befehle für tudmathposter}
%\ToDo[doc]{Dokumentation tudmathposter}
\ToDo[imp]{Unterstützung für \Class{tudmathposter} ENDE}[v2.05]
\index{Kompatibilität!\Class{tudmathposter}|)}%
\end{Bundle!}

\chapter{Identifikation von \TUDScript}
Im \TUDScript-Bundle gibt es neben den Klassen selbst auch noch zusätzliche 
Pakete. Ein Teil dieser Pakete~-- genauer \Package{tudscrsupervisor} und 
\Package{tudscrcomp}~-- sind ausschließlich mit den \TUDScript-Klassen nutzbar, 
andere wiederum~-- die beiden Pakete für Belange des \CDs \Package{tudscrfonts} 
(Schriften) und \Package{tudscrcolor} (Farben) sowie die davon vollkommen 
unabhängigen Pakete \Package{mathswap} und \Package{twocolfix}~-- können mit 
allen existierenden \hologo{LaTeXe}-Dokumentklassen genutzt werden. Sämtliche 
Klassen und Pakete aus dem \TUDScript-Bundle enthalten Befehle, welche diese 
als Bestandteil identifizieren.

\begin{Declaration}[v2.04]{\Macro{TUDScript}}
\printdeclarationlist%
%
Diese Anweisung setzt das Logo respektive die Wortmarke \enquote{\TUDScript{}} 
in serifenloser Schrift und mit leichter Sperrung des in Versalien gesetzten 
Teils. Dieser Befehl wird von allen Klassen und Paketen des \TUDScript-Bundles 
mit \Macro*{DeclareRobustCommand}.
\end{Declaration}

\begin{Declaration}[v2.04]{\Macro{TUDVersion}}
\printdeclarationlist%
%
In dieser Anweisung ist die Hauptversion von \TUDScript in der Form
\begin{quoting}
\PName{Datum}~\PName{Version}~\PValue{TUD-KOMA-Script}
\end{quoting}
abgelegt. Diese Version ist im \TUDScript-Bundle für alle Klassen und Pakete 
gleich und kann daher nach dem Laden einer Klasse oder eines Paketes abgefragt 
werden. Diese Anleitung wurde beispielsweise mit \enquote{\TUDVersion{}} 
erstellt.
\end{Declaration}

\begin{Declaration}[v2.04]{\Macro{TUDClassName}}
\printdeclarationlist%
%
Im Makro \Macro{TUDClassName} ist die Bezeichnung der im jeweiligen Dokument 
verwendeten \TUDScript-Klasse abgelegt. Soll also in Erfahrung gebracht werden, 
welche oder ob überhaupt eine \TUDScript-Klasse verwendet wird, so kann einfach 
auf diese Anweisung getestet werden. \KOMAScript{} selbst stellt zusätzlich 
noch die beiden Anweisungen \Macro*{KOMAClassName} und \Macro*{ClassName} 
bereit, welche den Namen der zugrundeliegenden \KOMAScript-Klasse sowie die 
durch diese ersetzte Standardklasse enthalten.
\end{Declaration}

\CrossIndex{Optionen}{options}%
\CrossIndex*{Befehle}{macros}%
\CrossIndex{Bezeichner}{terms}%
\CrossIndex{Seitenstile,Schriftelemente,Farben}{elements}%
\CrossIndex*{Längen}{misc}%
\CrossIndex{Zähler}{misc}%
\CrossIndex*{Klassen,Pakete,Dateien}[Index der Dateien etc.]{files}%
\CrossIndex*{Änderungen,Changelog,Version}[Änderungsliste]{changelog}%
%
\SeeRef{Distribution}{\Distribution{Mac\hologo{TeX}}}
\SeeRef{Distribution}{\Distribution{\hologo{TeX}~Live}}
\SeeRef{Distribution}{\Distribution{\hologo{MiKTeX}}}
%
\index{Abbildungen|see{Grafiken}}%
\index{Abschlussarbeit|see{Typisierung}}%
\index{Aktualisierung|see{Update}}%
\index{Aufzählungen|see{Listen}}%
\index{Autorenangaben|see{Titel}}%
\index{Bindekorrektur|see{Satzspiegel}}%
\index{Cover|see{Umschlagseite}}%
\index{Dezimaltrennzeichen|see{Zifferngruppierung}}%
\index{doppelseitiger Satz|see{Satzspiegel}}%
\index{Dresden-concept-Logo@\DDC-Logo|see{Layout}}%
\index{Drittlogo|see{Layout}}%
\index{Fachreferent|see{Referent}}%
\index{Farben|see{Layout}}%
\index{Fußzeile|see{Layout}}%
\index{Installation|see{Update}}%
\index{Gliederung|see{Layout!Überschriften}}%
\index{Grafiken|see{Gleitobjekte}}%
\index{Großbuchstaben|see{Schriftauszeichnung}}%
\index{Kapitelseiten|see{Layout}}%
\index{Kapitelüberschriften|see{Layout}}%
\index{Klassenoptionen|see{Optionen}}%
\index{Kleinbuchstaben|see{Schriftauszeichnung}}%
\index{Kolumnentitel|see{Layout}}%
\index{Kopfzeile|see{Layout}}%
\index{Kurzfassung|see{Zusammenfassung}}%
\index{Layout!Seitenränder|see{Satzspiegel}}%
\index{Layout!Titel|see{Titel}}%
\index{Layout!Umschlagseite|see{Umschlagseite}}%
\index{Leerseiten|see{Vakatseiten}}%
\index{Lokalisierung|see{Bezeichner}}%
\index{Majuskeln|see{Schriftauszeichnung}}%
\index{Makros|see{Befehle}}%
\index{Mathematiksatz|see{Einheiten}}%
\index{Mathematiksatz|see{Griechische Buchstaben}}%
\index{Mathematiksatz|see{Zifferngruppierung}}%
\index{Minuskeln|see{Schriftauszeichnung}}%
\index{Nutzerinstallation|see{Installation}}%
\index{Outline-Eintrag|see{Lesezeichen}}%
\index{Parameter|see{Befehle}}%
\index{Professor|see{Hochschullehrer}}%
\index{Querbalken|see{Layout}}%
\index{Seitenränder|see{Satzspiegel}}%
\index{Seitenstile|see{Layout}}%
\index{Silbentrennung|see{Worttrennung}}%
\index{Sprachunterstützung|see{Bezeichner}}%
\index{Sprachunterstützung|see{Worttrennung}}%
\index{Sprungmarken|see{Lesezeichen}}%
\index{Tabellen|see{Gleitobjekte}}%
\index{Tausendertrennzeichen|see{Zifferngruppierung}}%
\index{Teileseiten|see{Layout}}%
\index{Teileüberschriften|see{Layout}}%
\index{Titel!Umschlagseite|see{Umschlagseite}}%
\index{Umgebungen|see{Befehle}}%
\index{Trennmuster|see{Worttrennung}}%
\index{Vektorgrafiken|see{Grafiken}}%
\index{Versalien|see{Schriftauszeichnung}}%
\index{zweiseitiger Satz|see{Satzspiegel}}%
\index{zweispaltiger Satz|see{Satzspiegel}}%
\index{Zweitlogo|see{Layout}}%


\setchapterpreamble{%
  \addparttocentry{}{\indexname}%
  \begin{abstract}
    \noindent Die Formatierung der Einträge in allen aufgeführten Indexen ist 
    folgendermaßen aufzufassen: \textbf{Zahlen in fetter Schrift} verweisen auf 
    die \textbf{Erklärung} zu einem Stichwort, wobei in der digitalen Fassung 
    dieses Handbuchs dieser Eintrag selbst ein Hyperlink zu seiner Erläuterung 
    ist. Seitenzahlen in normaler Schriftstärke hingegen deuten auf zusätzliche 
    Informationen, wobei diese für \textit{kursiv hervorgehobene Zahlen} als 
    besonders \textit{wichtig} erachtet werden.
    
    Bei Einträgen für \hyperref[idx:options]{Klassen- und Paketoptionen} 
    beziehungsweise für \hyperref[idx:macros]{Umgebungen und Befehle}, zu denen 
    keine direkte \textbf{Erklärung} gegeben ist sondern lediglich zusätzliche 
    Hinweise vorhanden sind, handelt es sich um \KOMAScript"=Optionen. Diese
    sind gegebenenfalls im \scrguide*[dazugehörigen Handbuch] nachzulesen.
  \end{abstract}
}
\chapter*{\indexname}
\PrintIndex
\PrintChangelog




\clearpage
\ToDo[doc]{%
  Markup in Tutorials ändern, insbesondere \Macro*{Option} und \Macro*{Macro}%
}[v2.05]
\ToDo[doc]{Überprüfung Label und overfull box in Tutorials, log prüfen}[v2.05]
\ToDo[imp]{alle Links auf http://www.ctan.org/pkg automatisieren}[v2.05]

\ToDo[imp]{\Macro*{DeclareSectionCommand}}[v2.06]
\ToDo[imp]{Unterstützung für algorithm2e o.ä. (Überschriften?!)}[v2.06]

\ToDo[rls]{KOMA-Script-Version anpassen (LoadClass) v3.18}

\ToDo[imp]{TeXstudio: tudscrposter.cwl erstellen}[v2.05]
\ToDo[imp]{TeXstudio: tudscrmanual.cwl/tudscrtutorial.cwl erneuern}[v2.05]
\ToDo[imp]{TeXstudio: alle cwl über docstrip separate .ins-Datei}[v2.05]

\ToDo[rls]{TeXstudio: alle *.cwl erneuern}
\ToDo[rls]{\emph{alle} dtx-Dateien der vorherigen Version mit WinMerge sichten}
\ToDo[rls]{Layout und Umbrüche kontrollieren}
\ToDo[rls]{Datum in tudscr-version.dtx, Handbuch und README aktualisieren}
\ToDo[rls]{%
  Release auf GitHub und CTAN, Tag für Homepage und im Forum ändern 
  \url{http://latex.wcms-file3.tu-dresden.de/phpBB3/viewtopic.php?t=303}%
}
\ListOfToDo
\end{document}
