\setchapterpreamble{%
  \begin{abstract}
    Zusätzlich zu den eigentlichen Hauptklassen werden im \TUDScript-Bundle 
    weitere Paket bereitgestellt. Diese werden im Folgenden vorgestellt.%
  \end{abstract}
}
\chapter{Zusätzliche Pakete im \TUDScript-Bundle}
\label{sec:bundle}
\section{Das Paket \Package*{tudscrsupervisor} -- Studentische Betreuung}
\begin{Bundle!}{\Package{tudscrsupervisor}}
%
Dieses Paket stellt für das Erstellen von Aufgabenstellungen und Gutachten  
wissenschaftlicher Arbeiten sowie offiziellen Aushängen im \CD passende 
Umgebungen und Befehle für den Anwender bereit. Deshalb richtet es sich 
vornehmlich an Mitarbeiter an der \TnUD, kann jedoch natürlich auch von 
Studenten genutzt werden.


\subsection{Aufgabenstellung für eine wissenschaftliche Arbeit}
\index{Aufgabenstellung|!(}%
\begin{Declaration}{\Environment{task}[\OLParameter{Überschrift}]}{%
  \Environment{tudpage}'auto'%
}
\begin{Declaration}{\Key{\Environment{task}}{headline=\PName{Überschrift}}}
\printdeclarationlist%
%
Mit der \Environment{task}"=Umgebung kann ein Aufgabenstellung für eine 
wissenschaftliche Arbeit ausgegeben werden. Diese basiert auf der Umgebung 
\Environment{tudpage} und akzeptiert deshalb im optionalen Argument alle 
Parameter, welche bei der Beschreibung von \Environment{tudpage}'full' 
erläutert wurden.

Für die Aufgabenstellung wird normalerweise eine Überschrift gesetzt, welche 
sich aus \Term{taskname} und~-- falls der Typ der Abschlussarbeit angegeben 
wurde~-- noch aus \Term{tasktext} und \Macro{thesis} zusammensetzt. Der 
Parameter \Key{\Environment{task}}{headline} kann genutzt werden, um diese 
automatisch generierte Überschrift anzupassen.

Zu Beginn der Aufgabenstellung erscheint eine Tabelle mit den angegebenen 
Informationen zum Autor respektive zu den Autoren der Abschlussarbeit. Zwingend 
anzugeben sind dafür lediglich der Name des oder der Verfasser (\Macro{author}) 
sowie der Titel der Arbeit (\Macro{title}), welcher am Ende der Tabelle in 
fetter Schrift aufgeführt wird. Optional werden noch die Felder für den 
Studiengang (\Macro{course}), die Fachrichtung (\Macro{discipline}) sowie die 
Matrikelnummer (\Macro{matriculationnumber}) und das Immatrikulationsjahr 
(\Macro{matriculationyear}) ausgegeben, wobei nicht angegebene Felder bei der 
Ausgabe ignoriert werden. Der eigentliche Inhalt der Umgebung~-- sprich die 
Aufgabenstellung selbst~-- wird nach dem generierten Kopf ausgegeben

Nach der Ausgabe des Inhaltes der Aufgabenstellung werden der oder die mit 
\Macro{supervisor} definierten Betreuer aufgelistet. Dabei wird unter dem 
jeweiligen Namen selbst der sprachabhängige Bezeichner (\Term{supervisorname}, 
\Term{supervisorothername}) gesetzt. Darauf folgend erscheint das Ausgabedatum 
(\Macro{issuedate}) und der verpflichtende Abgabetermin (\Macro{duedate}). Zum 
Schluss wird die Unterschriftzeile für den Prüfungsausschussvorsitzenden 
(\Macro{chairman}) und den betreuenden Hochschullehrer (\Macro{professor}) 
gesetzt. Für genannte Personen werden unter dem Namen selbst die Bezeichner 
ausgegeben (\Term{chairmanname} und \Term{professorname}).
\end{Declaration}
\end{Declaration}

\begin{Declaration}{\Macro{taskform}\LParameter%
  \Parameter{Ziele}\Parameter{Schwerpunkte}%
}
\printdeclarationlist%
%
Zusätzlich zur der frei gestaltbaren Umgebung \Environment{task} zur Erstellung
einer Aufgabenstellung wird ein separater Befehl für eine standardisierte 
Ausgabe zur Verfügung gestellt. Dieser strukturiert die Aufgabenstellung in die 
zwei Bereiche \emph{Ziele} und \emph{Schwerpunkte} der Arbeit mit dazugehörigen 
Überschriften (\Term{objectivesname}, \Term{focusname}).

Im optionalen Argument können alle Parameter der Umgebung \Environment{task} 
verwendet werden. Im ersten obligatorischen Argument sollte ein Text mit einer 
kurzen thematischen Einordnung und dem eigentlichen Ziel der Arbeit erscheinen. 
im zweiten Argument sollen die thematischen Schwerpunkte in Stichpunkten 
benannt werden. Der Inhalt des zweiten notwendigen Argumentes wird in einer 
\Environment{itemize}"=Umgebung gesetzt. Deshalb \emph{muss} jedem Stichpunkt 
\Macro*{item} vorangestellt werden.
\end{Declaration}
\index{Aufgabenstellung|!)}%
%
\begin{Example}
Die empfohlene Verwendung des Befehls \Macro{taskform} ist wie folgt:
\begin{Code}[escapechar=§]
\taskform{%
  Motivation der Arbeit im ersten Absatz§\dots§
  
  Ziele der Arbeit im zweiten Absatz§\dots§
}{%
  \item Schwerpunkt 1
  \item Schwerpunkt 2
}
\end{Code}
Hierzu sei auch auf das Minimalbeispiel in \autoref{sec:exmpl:task} verwiesen.
\end{Example}

\begin{Declaration}{\Macro{course}\Parameter{Studiengang}}
\begin{Declaration}[v2.02]{\Macro{discipline}\Parameter{Studienrichtung}}
\printdeclarationlist%
\index{Kollaboratives Schreiben}%
%
Mit diesen beiden Befehlen kann der Studiengang sowie die Studienrichtung für 
den Autor oder die Autoren angegeben werden. Diese Informationen werden zu 
Beginn der \Environment{task}"=Umgebung gesetzten Tabelle ausgegeben. Werden 
diese Befehle innerhalb des Makros \Macro{author} verwendet, können auch 
unterschiedliche Angaben für mehrere Autoren gemacht werden. Dabei sind die 
Autoren mit \Macro{and} voneinander zu trennen.
\end{Declaration}
\end{Declaration}

\begin{Declaration}{\Macro{chairman}\Parameter{Prüfungsausschussvorsitzender}}
\printdeclarationlist%
%
Wird dieses Feld genutzt, wird neben dem betreuenden Hochschullehrer 
(\Macro{professor}) auch der Vorsitzende des Prüfungsausschusses am Ende der 
Aufgabenstellung aufgeführt. Dies wird zumeist für Abschlussarbeiten wie 
beispielsweise \masterthesisname{} oder \diplomathesisname{} benötigt.
\end{Declaration}

\begin{Declaration}{\Macro{issuedate}\Parameter{Ausgabedatum}}
\begin{Declaration}{\Macro{duedate}\Parameter{Abgabetermin}}
\printdeclarationlist%
%
Mit diesen beiden Befehlen sollte das Datum der Ausgabe der Aufgabenstellung 
sowie der spätest mögliche Abgabetermin angegeben werden. Ist das Paket 
\Package{isodate} geladen, wird die damit eingestellte Formatierung des Datums 
durch den Befehl \Macro{printdate} aus diesem Paket für \Macro{issuedate} und 
\Macro{duedate} verwendet.
\end{Declaration}
\end{Declaration}


\subsection{Gutachten für wissenschaftliche Arbeiten}
\index{Gutachten|!(}%
\begin{Declaration}{\Environment{evaluation}[\OLParameter{Überschrift}]}{%
  \Environment{tudpage}'auto'%
}
\begin{Declaration}{%
  \Key{\Environment{evaluation}}{headline=\PName{Überschrift}}%
}
\begin{Declaration}{\Key{\Environment{evaluation}}{grade=\PName{Note}}}
\printdeclarationlist%
%
Diese Umgebung wird für das Erstellen eines Gutachtens einer wissenschaftlichen 
Arbeit bereitgestellt. Auch diese unterstützt alle Parameter, welche für die 
Umgebung \Environment{tudpage}'full' beschrieben wurden.

Für ein Gutachten wird gewöhnlich eine Überschrift aus \Term{evaluationname} 
und~-- falls der Abschlussarbeitstyp angegeben wurde~-- \Term{evaluationtext} 
sowie \Macro{thesis} generiert. Diese automatisch generierte Überschrift kann 
mit dem Parameter \Key{\Environment{evaluation}}{headline} ersetzt werden. Am 
Ende des Gutachtens wird die mit \Key{\Environment{evaluation}}{grade} 
gegebene Note in fetter Schrift ausgezeichnet.

Am Anfang der \Environment{evaluation}"=Umgebung wird die gleiche Tabelle mit 
Autorenangaben ausgegeben, wie dies bei der \Environment{task}"=Umgebung der 
Fall ist. Nach dem Tabellenkopf folgt auch hier der eigentliche Inhalt, sprich 
das Gutachten der Abschlussarbeit. Abgeschlossen wird die Umgebung mit der 
gegebenen Note~-- welche innerhalb von \Term{gradetext} ausgegeben wird~-- 
sowie der Orts- und Datumsangabe (\Macro{place}, \Macro{date}) und der 
darauffolgenden Unterschriftzeile für den oder die Gutachter (\Macro{referee}), 
welche wiederum mit den entsprechenden sprachabhängigen Bezeichner 
(\Term{refereename}, \Term{refereeothername}) ergänzt werden.
\end{Declaration}
\end{Declaration}
\end{Declaration}

\begin{Declaration}{\Macro{evaluationform}\LParameter%
  \Parameter{Aufgabe}\Parameter{Inhalt}\Parameter{Bewertung}\Parameter{Note}%
}
\printdeclarationlist%
%
Neben der individuell nutzbaren Umgebung \Environment{evaluation} wird ein 
separater Befehl zur Erstellung eines standardisierten Gutachtens 
bereitgestellt. Dieser strukturiert die Ausgabe in die vier Bereiche 
\emph{Aufgabe}, \emph{Inhalt}, \emph{Bewertung} und \emph{Note} und versieht 
diese jeweils mit der dazugehörigen Überschrift beziehungsweise Textausgabe 
(\Term{taskname}, \Term{contentname}, \Term{assessmentname} und 
\Term{gradetext}). Das optionale Argument unterstützt alle Parameter der 
\Environment{evaluation}"=Umgebung.
\end{Declaration}
\index{Gutachten|!)}%
%
\begin{Example}
Die empfohlene Verwendung des Befehls \Macro{evaluationform} ist wie folgt:
\begin{Code}[escapechar=§]
\evaluationform{%
  Kurzbeschreibung der Aufgabenstellung§\dots§
}{%
  Zusammenfassung von Inhalt und Struktur§\dots§
}{%
  Bewertung der schriftlichen Abschlussarbeit§\dots§
}{%
  Zahl (Note)
}
\end{Code}
Hierzu sei auch auf das Minimalbeispiel in \autoref{sec:exmpl:evaluation} 
verwiesen.
\end{Example}

\begin{Declaration}{\Macro{grade}\Parameter{Note}}
\printdeclarationlist%
%
Neben der Angabe der Note für ein Gutachten über den Parameter 
\Key{\Environment{evaluation}}{grade} kann dafür auch dieser global wirkende 
Befehl verwendet werden.
\end{Declaration}


\subsection{Aushang}
\index{Aushang|!(}%
\begin{Declaration}{\Environment{notice}[\OLParameter{Überschrift}]}{%
  \Environment{tudpage}'auto'%
}
\begin{Declaration}{\Key{\Environment{notice}}{headline=\PName{Überschrift}}}
\printdeclarationlist%
%
Für das Anfertigen eines Aushangs kann diese Umgebung verwendet werden. Diese 
basiert abermals auf der Umgebung \Environment{tudpage} und unterstützt alle 
deren Parameter.

Wurde ein Datum angegeben, wird dieses in der oberen rechten Ecke gesetzt. 
Anschließend wird die Überschrift ausgegeben, welche für gewöhnlich dem Inhalt 
von \Term{noticename} entspricht und mit \Key{\Environment{notice}}{headline} 
geändert werden kann. Nach der Überschrift wird bereits der Inhalt der Umgebung 
ausgegeben. Wurde mit \Macro{contactperson} ein oder mehrere Ansprechpartner 
angegeben, werden diese Informationen am Ende der Umgebung ausgegeben.
\end{Declaration}
\end{Declaration}

\begin{Declaration}{\Macro{noticeform}\LParameter%
  \Parameter{Inhalt}\Parameter{Schwerpunkte}%
}
\printdeclarationlist%
%
Auch für diese Umgebung gibt es einen Befehl für eine normierte Form. Diese 
soll vor allem Verwendung für den Aushang studentischer Arbeitsthemen finden. 
Für das optionale Argument können sämtliche Parameter verwendet werden, die 
auch die \Environment{notice}"=Umgebung unterstützt.

Das erste obligatorische Argument sollte für eine kurze Inhaltsbeschreibung 
verwendet werden. Neben dem textuellen Teil sollte hier wenn möglich eine 
thematisch passende Abbildung eingebunden werden (\Macro{includegraphics}). Das 
zweite Argument wird~--wie schon bei \Macro{taskform}~-- dazu verwendet, einige 
Schwerpunkte aufzuzählen. Auch hier kommt nach der gliedernden Überschrift 
(\Term{focusname}) eine \Environment{itemize}"=Umgebung zum Einsatz, allen 
Schwerpunkten muss ein \Macro*{item} vorangestellt werden.
\end{Declaration}
%
\begin{Example}
Die empfohlene Verwendung des Befehls \Macro{noticeform} ist wie folgt:
\begin{Code}[escapechar=§]
\noticeform{%
  Kurzbeschreibung des Inhaltes der studentischen Arbeit§\dots§
  
  Bild (optional), einzubinden mit:
    \includegraphics[§\PName{Optionen}§]{§\PName{Datei}§}
}{%
  \item Schwerpunkt 1
  \item Schwerpunkt 2
}
\end{Code}
Hierzu sei auch auf das Minimalbeispiel in \autoref{sec:exmpl:notice} verwiesen.
\end{Example}
\index{Aushang|!)}%

\begin{Declaration}[v2.02]{\Macro{contactperson}\Parameter{Kontaktperson(en)}}
\begin{Declaration}{\Macro{office}\Parameter{Dienstsitz}}
\begin{Declaration}[v2.02]{\Macro{telephone}\Parameter{Telefonnummer}}
\begin{Declaration}[v2.02]{\Macro{emailaddress}\Parameter{E-Mail-Adresse}}
\printdeclarationlist%
%
Am Ende eines Aushangs können mit \Macro{contactperson} Kontaktinformationen 
für eine oder mehrere Ansprechpartner angegeben werden. Soll mehr als eine 
Kontaktperson genannt werden, so müssen diese innerhalb des Befehls
\Macro{contactperson} mit dem Befehl \Macro{and} getrennt werden. Für jede 
Person kann innerhalb des Argumentes von \Macro{contactperson} der Dienstsitz 
(\Macro{office}), die dienstliche Telefonnummer (\Macro{telephone}) sowie die 
geschäftliche E"~Mail"=Adresse (\Macro{emailaddress}) angegeben werden. Sollte 
das Paket \Package{hyperref} geladen werden, wird die gegebene E"~Mail"=Adresse 
direkt in einen entsprechenden Link gewandelt.
\end{Declaration}
\end{Declaration}
\end{Declaration}
\end{Declaration}


\subsection{Zusätzliche sprachabhängige Bezeichner}
\index{Bezeichner|!(}%
%
Für das Paket \Package{tudscrsupervisor} werden für die zusätzlichen Befehle 
und Umgebungen weitere Bezeichner definiert. Für eine etwaige Anpassung dieser 
sei auf \autoref{sec:localization} verwiesen.

\begin{Declaration}{\Term{taskname}}
\begin{Declaration}{\Term{tasktext}}
\printdeclarationlist%
%
Die Bezeichnung der Aufgabenstellung selbst ist in \Term{taskname} enthalten. 
Für die Generierung einer Überschrift wird dieser verwendet. Wurde außerdem mit 
\Macro{thesis} oder \Macro{subject} der Typ der Abschlussarbeit%
\footnote{%
  \Option{subjectthesis} oder spezieller Wert aus \autoref{tab:thesis}
}
angegeben, wird die Überschrift zusammen mit dem Bezeichner \Term{tasktext}
um die Typisierung erweitert. Falls gewünscht, kann die automatisch generierte 
Überschrift mit dem Parameter \Key{\Environment{task}}{headline} der Umgebung 
\Environment{task} überschrieben werden.
\TermTable{taskname,tasktext}
\end{Declaration}
\end{Declaration}

\begin{Declaration}[v2.04]{\Term{namesname}}
\begin{Declaration}{\Term{titlename}}
\begin{Declaration}{\Term{coursename}}
\begin{Declaration}[v2.02]{\Term{disciplinename}}
\printdeclarationlist%
%
Diese Bezeichner werden in der Tabelle mit den Autoreninformationen zu Beginn 
der Aufgabenstellung verwendet. Dabei werden \Term{coursename} und 
\Term{disciplinename} nur genutzt, wenn für mindestens einen Autor die Befehle 
\Macro{course} und/oder \Macro{discipline} verwendet wurden.
\TermTable{namesname,titlename,coursename,disciplinename}
\end{Declaration}
\end{Declaration}
\end{Declaration}
\end{Declaration}

\begin{Declaration}{\Term{issuedatetext}}
\begin{Declaration}{\Term{duedatetext}}
\printdeclarationlist%
%
Am Ende der Aufgabenstellung wird nach dem oder der Betreuer das Ausgabedatum 
und der Abgabetermin (\Macro{issuedate}, \Macro{duedate}) der Abschlussarbeit 
mit folgenden Bezeichner erläutert.
\TermTable{issuedatetext,duedatetext}
\end{Declaration}
\end{Declaration}

\begin{Declaration}{\Term{chairmanname}}
\printdeclarationlist%
%
Wurde der Prüfungsausschussvorsitzende (\Macro{chairman}) angegeben, erfolgt 
unter dem Namen selbst die Ausgabe des Bezeichners.
\TermTable{chairmanname}
\end{Declaration}

\begin{Declaration}{\Term{focusname}}
\begin{Declaration}{\Term{objectivesname}}
\printdeclarationlist%
%
Die Standardformen für Aufgabenstellung (\Macro{taskform}) respektive Aushang 
(\Macro{noticeform}) nutzen für die gesetzten Überschriften diese Bezeichner.
\TermTable{focusname,objectivesname}
\end{Declaration}
\end{Declaration}

\begin{Declaration}{\Term{evaluationname}}
\begin{Declaration}{\Term{evaluationtext}}
\printdeclarationlist%
%
Die Bezeichnung des Gutachten selbst ist in \Term{evaluationname} enthalten. 
Für die Generierung der Überschrift wird der Bezeichner \Term{evaluationtext} 
sowie der mit \Macro{thesis} oder gegebenenfalls mit \Macro{subject} gegebenen 
Typ der Abschlussarbeit verwendet. Diese automatisch generierte Überschrift 
kann mit dem Parameter \Key{\Environment{evaluation}}{headline} der 
Umgebung \Environment{evaluation} durch den Anwender überschrieben werden.
\TermTable{evaluationname,evaluationtext}
\end{Declaration}
\end{Declaration}

\begin{Declaration}{\Term{contentname}}
\begin{Declaration}{\Term{assessmentname}}
\printdeclarationlist%
%
Bei der standardisierten Form des Gutachten (\Macro{evaluationform}) werden die 
darin~-- zur strukturierter Gliederung~-- erzeugten Überschriften mit den 
Bezeichnern \Term{taskname}, \Term{contentname} und \Term{assessmentname} 
gesetzt.
\TermTable{taskname,contentname,assessmentname}
\end{Declaration}
\end{Declaration}

\begin{Declaration}{\Term{gradetext}}
\printdeclarationlist%
%
Wird für das Gutachten einer wissenschaftlichen Arbeit die erzielte Note 
entweder mit dem Befehl \Macro{grade}\Parameter{Note} oder alternativ dazu mit 
dem Parameter \Key{\Environment{evaluation}}{grade=\PName{Note}} angegeben, so 
wird diese innerhalb von \Term{gradetext} verwendet.
\grade{\PName{Note}}
\TermTable*{gradetext}{.7\textwidth}
\end{Declaration}

\begin{Declaration}{\Term{noticename}}
\begin{Declaration}[v2.02]{\Term{contactpersonname}}
\printdeclarationlist%
%
Die Bezeichnung des Aushangs selbst ist in \Term{noticename} enthalten. Für 
die Generierung einer Überschrift wird dieser verwendet. Falls gewünscht, kann 
diese mit dem Parameter \Key{\Environment{notice}}{headline} der Umgebung 
\Environment{notice} überschrieben werden. Wurde eine Kontaktperson mit dem 
Befehl \Macro{contactperson} angegeben, wird als Überschrift der Kontaktdaten 
der Bezeichner \Term{contactpersonname} verwendet.

\TermTable{noticename,contactpersonname}
\end{Declaration}
\end{Declaration}
\index{Bezeichner|!)}%
\end{Bundle!}


\section{Das Paket \Package*{tudscrcolor} -- Farben im \CD}%
\begin{Bundle!}{\Package{tudscrcolor}}
\index{Farben|(}%
\index{Layout!Farben|(?}%
%
Zur Verwendung der Farben des \CDs wird das Paket \Package{tudscrcolor} 
genutzt. Falls dieses nicht in der Präambel geladen wird~-- um beispielsweise 
zusätzliche Optionen aufzurufen~-- binden die \TUDScript"=Klassen dieses 
automatisch ein.

Für das \CD sind mehrere Farben vorgesehen. Die prägnanteste aller ist die 
Hausfarbe \Color{HKS41}, danach folgen die Farben für Auszeichnungen der ersten
(\Color{HKS44}) und der zweiten Kategorie (\Color{HKS36}, \Color{HKS33}, 
\Color{HKS57}, \Color{HKS65}) sowie eine Ausnahmefarbe (\Color{HKS07}). 
Diese Farben dürfen sowohl in ihrer Grundform als auch in helleren Tönen mit 
einer Abstufung in 10\,\%"~Schritten verwendet werden. Das ohnehin verwendete 
Paket \Package{xcolor} stellt genau diese Funktionalität zur Verfügung. Jede 
der Farben kann sowohl mit \Color*{HKS}()\PName{Zahl} als auch über ein 
Pseudonym \Color*{cd}()\PName{Farbe} genutzt werden.
%
\begin{Example*}
Die Grundfarbe \Color*{HKS44} soll in der auf 20\% reduzierten, helleren 
Abstufung genutzt werden. Innerhalb eines Befehls, der als Argument eine 
gültige Farbe erwartet, muss lediglich \Color*{HKS44}\PValue{!20} angegeben 
werden. Dies wird hier exemplarisch mit dieser \colorbox{HKS44!20}{%
  Box \Macro*{colorbox}\PParameter{HKS44!20}\PParameter{Box}%
} demonstriert.
\end{Example*}
%
Bei der farbigen Gestaltung des \CDs (\Option{cd=color}) ist der Hintergrund 
von Umschlagseite, Titel sowie Teilen in \Color*{HKS41} und die Schrift auf 
selbigen in \Color*{HKS41}\PValue{!30} gehalten. Der Kapitelseitenhintergrund
erscheint in \Color*{HKS41}\PValue{!10}, die Schrift in \Color*{HKS41}. Bei 
geringerem Farbeinsatz werden lediglich die Schriften der Gliederungsseiten auf 
\Color*{HKS41} gesetzt.

Sollen bestimmte Optionen an das Paket \Package{xcolor} weitergereicht werden, 
gibt es dafür zwei Möglichkeiten. Diese kann entweder vor dem Laden der Klasse 
direkt an \Package{xcolor} übergeben werden%
\footnote{%
  \Macro{PassOptionsToPackage}\Parameter{Option}\PParameter{xcolor} gefolgt von
  \Macro*{documentclass}\OParameter{Klassenoptionen}\PParameter{tudscr\dots}
} oder es wird \Package{tudscrcolor} mit der entsprechenden Option geladen.%
\footnote{
  \Macro*{usepackage}\OParameter{Option}\PParameter{\Package{tudscrcolor}};
  \Package{tudscrcolor} reicht \PName{Option} an \Package{xcolor} weiter
}
\newcommand*\cdcolorcalc{}
\newcommand*\cdcolorname{}
\newcommand*\cdcolorvalue{}
\newcommand*\cdcolortext{}
\newcommand*\cdcolor[2][0]{%
  \noindent%
  \begin{tikzpicture}[every node/.style={%
    rectangle,minimum height=.1\linewidth,minimum width=25mm%
  }]%
  \def\cdcolorcalc##1##2{%
    \pgfmathparse{100-##1*10}%
    \xdef\cdcolorname{HKS##2!\pgfmathresult}%
    \xdef\cdcolorvalue{\pgfmathresult}%
    \pgfmathparse{10+##1*10}%
  }%
  \foreach \x in {0,1,...,9}{%
    \cdcolorcalc{\x}{#2}%
    \ifnum\x<#1%
      \def\cdcolortext{white}%
    \else%
      \def\cdcolortext{black}%
    \fi%
    \node [fill=\cdcolorname,rotate=90] at (.\x\linewidth,0)%
      {\textcolor{\cdcolortext}{HKS#2!\pgfmathprintnumber\cdcolorvalue}};%
  };%
  \end{tikzpicture}%
}

\subsection{Generelle Farbdefinitionen}
\minisec{Primäre Hausfarbe}
\begin{Declaration}{\Color{HKS41}[cddarkblue]}
\printdeclarationlist%
\cdcolor[6]{41}
\end{Declaration}

\minisec{Sekundäre Hausfarbe (Geschäftsausstattung)}
\begin{Declaration}{\Color{HKS92}[cdgray]}
\printdeclarationlist%
\cdcolor[4]{92}
\end{Declaration}

\minisec{Auszeichnungsfarbe 1.Kategorie}
\begin{Declaration}{\Color{HKS44}[cdblue]}
\printdeclarationlist%
\cdcolor[4]{44}
\end{Declaration}

\minisec{Auszeichnungsfarbe 2.Kategorie}
\begin{Declaration}{\Color{HKS36}[cdindigo]}
\begin{Declaration}{\Color{HKS33}[cdpurple]}
\begin{Declaration}{\Color{HKS57}[cddarkgreen]}
\begin{Declaration}{\Color{HKS65}[cdgreen]}
\printdeclarationlist%
\cdcolor[4]{36}\vskip\baselineskipglue
\cdcolor[4]{33}\vskip\baselineskipglue
\cdcolor[2]{57}\vskip\baselineskipglue
\cdcolor{65}
\end{Declaration}
\end{Declaration}
\end{Declaration}
\end{Declaration}

\minisec{Ausnahmefarbe}
\begin{Declaration}{\Color{HKS07}[cdorange]}
\printdeclarationlist%
\cdcolor{07}
\end{Declaration}


\subsection{Zusätzliche Farbdefinitionen}
Das Paket \Package{tudscrcolor} definiert im Normalfall lediglich die zuvor 
beschriebenen Grundfarben \Color*{HKS41}, \Color*{HKS92}, \Color*{HKS44}, 
\Color*{HKS36}, \Color*{HKS33}, \Color*{HKS57}, \Color*{HKS65} sowie 
\Color*{HKS07}. Alle anderen farblichen Abstufungen können mit den beschrieben 
Möglichkeiten des Paketes \Package{xcolor} generiert werden.

\begin{Declaration}{\Option{oldcolors}}
\printdeclarationlist%
%
In den letzten Jahren sind viele verschiedene Klassen und Pakete für das \CD 
der \TnUD entstanden. Innerhalb dieser existieren abweichende Farbdefinitionen. 
Um eine Migration von den benannten Klassen und Paketen auf \TUDScript zu 
ermöglichen, existiert die Paketoption \Option{oldcolors}. Wird diese genutzt, 
so werden zusätzliche Farben nach dem Schema \Color*{HKS41K}()\PName{Zahl} und 
\Color*{HKS41-}()\PName{Zahl} definiert, wobei der hinten angestellte 
Zahlenwert aus der 10er-Reihe kommen muss.
\end{Declaration}



\subsection{Umstellung des Farbmodells}
\index{Farben!Farbmodell}%
%
Normalerweise verwendet \Package{tudscrcolor} das CMYK"=Farbmodell. Außerdem 
wird weiterhin noch der RGB"=Farbraum unterstützt. Eine Umschaltung des 
Farbmodells ist beispielsweise für gewisse Funktionen des Paketes 
\Package{tikz} notwendig.

\begin{Declaration}{\Option{RGB}}
\printdeclarationlist%
%
Mit dieser Option werden bereits beim Laden des Paketes \Package{tudscrcolor} 
die Farben nicht nach dem CMYK"=Farbmodell sondern im RGB"=Farbraum global 
definiert.
\end{Declaration}

\begin{Declaration}{\Macro{setcdcolors}\Parameter{Farbmodell}}
\printdeclarationlist%
%
Mit diesem Befehl kann innerhalb des Dokumentes das verwendete Farbmodell 
angepasst werden. Damit ist es möglich, lokal innerhalb einer Umgebung den 
Farbmodus zu ändern und so nur in bestimmten Situationen beispielsweise aus dem 
CMYK"=Farbmodell in den RGB"=Farbraum zu wechseln. Unterstützte Werte für 
\PName{Farbmodell} sind \PValue{CMYK} und \PValue{RGB} beziehungsweise 
\PValue{rgb}.
\end{Declaration}

\bigskip\noindent
\Attention{%
  Beachten Sie, dass die Darstellung der Farben im jeweiligen Farbmodus 
  (\PValue{CMYK} oder \PValue{RGB}) je nach verwendeter Bildschirm- Drucker- 
  und Softwarekonfiguration verschieden ausfallen kann. Die verwendeten 
  RGB-Werte entstammen aus dem Handbuch zum \CD und sind lediglich 
  Näherungswerte. Abweichungen vom gedruckten \Color*{HKS}()-Farbregister und 
  selbst ermittelten Werten sind technisch nicht zu vermeiden.
}%
\index{Farben|)}%
\index{Layout!Farben|?)}%
\end{Bundle!}



\section{Das Paket \Package*{tudscrfonts} -- Schriften im \CD}
\begin{Bundle!}[v2.02]{\Package{tudscrfonts}}
\printchangedatlist%
%
Dieses Paket stellt die Schriften des \CDs für \hologo{LaTeXe}-Klassen bereit, 
welche \emph{nicht} zum \TUDScript-Bundle gehören. Das Paket unterstützt einen 
Großteil der in \fullref{sec:fonts} beschriebenen Optionen und Befehle. Die 
nutzbaren Paketoptionen sind für den Fließtext \Option{cdfont}~-- ohne die 
Einstellungsmöglichkeiten für den Querbalken des \CDs (\Option{cdhead})~-- und 
für die mathematischen Schriften \Option{cdmath} sowie \Option{slantedgreek}. 
Da von \Package{tudscrfonts} intern das Paket \Package{tudscrbase}<> geladen 
wird, können diese entweder als Paketoptionen im optionalen Argument von 
\Macro*{usepackage}\OParameter{Paketoption}\PParameter{tudscrfonts} oder direkt 
als Klassenoption angegeben werden. Zusätzlich ist nach dem Laden des Paketes 
die späte Optionenwahl mit \Macro{TUDoption} beziehungsweise \Macro{TUDoptions} 
möglich.

Des Weiteren wird das Paket \Package{textcase} geladen, welches die Befehle 
\Macro{MakeTextUppercase} und \Macro{NoCaseChange} zur Verfügung stellt. Der 
Befehl \Macro{ifdin} wird ebenso wie die in \autoref{sec:fonts} beschriebenen 
Textschalter und "~kommandos sowie die Befehle für die griechischen Buchstaben 
bereitgestellt.

Das Paket \Package{tudscrfonts} ist insbesondere für die Verwendung zusammen 
mit einer der Klassen \Class{tudbook}, \Class{tudbeamer}, \Class{tudletter}, 
\Class{tudfax}, \Class{tudhaus} sowie \Class{tudform} vorgesehen. Zusätzlich 
werden seit der Version~v2.04 
\ChangedAt{%
  v2.04:Unterstützung der Klassen \Class{tudposter} und \Class{tudmathposter}%
}
auch \Class{tudmathposter} und \Class{tudposter} unterstützt. Das Paket kann 
prinzipiell mit jeder beliebigen \hologo{LaTeXe}-Klasse verwendet werden. Wird 
keine der zuvor genannten Klassen genutzt, muss der Anwender das Setzen der 
Gliederungsüberschriften in Majuskeln der \DIN~-- wie es im \CD vorgesehen 
ist~-- selbst umsetzen. Dafür sei abermals auf die Textauswahlbefehle 
\Macro{textdbn} und \Macro{dinbn} sowie den Befehl \Macro{MakeTextUppercase} 
zur automatisierten Großschreibung hingewiesen.

Die Schriftinstallation für das \TUDScript-Bundle unterscheidet sich von der 
für die gerade genannten Klassen sehr stark. Dabei wurde auch die Bezeichnung 
der Schriftfamilien geändert. Dies hatte zwei Gründe, wobei der letztere der 
entscheidende ist:
%
\begin{enumerate}
\item
  Die bisherige Schriftbenennung entsprach nicht dem offiziellen     
  \hrfn{http://mirrors.ctan.org/info/fontname/fontname.pdf}%
  {\hologo{TeX}-Namensschema}
\item
  Bei der Installation für das \TUDScript-Bundle werden sowohl die Metriken
  als auch das Kerning der Schriften für Fließtext und den Mathematikmodus 
  angepasst, was das Ergebnis der erzeugten Ausgabe beeinflusst. Damit jedoch
  Dokumente, die mit den zuvor genannten, älteren Klassen erstellt wurden, 
  weiterhin genauso ausgegeben werden wie bisher, mussten die Schriftfamilien 
  einen neuen Namen erhalten.
\end{enumerate}
%
Wird nun das Paket \Package{tudscrfonts} einer der zuvor genannten, älteren 
Klassen verwendet, hat dies den Vorteil, dass auch in diesen das angepasste 
Kerning der Schriften sowie der stark verbesserte Mathematiksatz zum Tragen 
kommen. Außerdem kann bei der Verwendung von \Package{tudscrfonts} auf eine 
Installation der Schriften des \CDs in der alten Variante verzichtet werden.
\Attention{%
  In diesem Fall kann sich das Ausgabeergebnis im Vergleich zu der Varianten 
  mit den alten Schriften ändern. Alternativ zur Verwendung des Paketes 
  \Package{tudscrfonts} können die alten Schriftfamilien auch parallel zu den 
  neuen installiert werden. Hierfür werden die Skripte
  \hrfn{https://github.com/tud-cd/tudscrold/releases/download/fonts/tudfonts_install.bat}{\File{tudfonts\_install.bat}}
  beziehungsweise
  \hrfn{https://github.com/tud-cd/tudscrold/releases/download/fonts/tudfonts_install.sh}{\File{tudfonts\_install.sh}}
  bereitgestellt.
}%
\end{Bundle!}
\ToDo[doc]{%
  Optimale Verwendung des Paketes \Package{tudscrfonts} mit alten Klassen,
  Laden des Paketes \Package{rescalefont} bei \Class{tudmathposter} unterbinden
%  \RequirePackage{scrlfile}\PreventPackageFromLoading{rescalefont}%
}[v2.05]
%title: tudmathposter wird unterstützt 
%\sectionfont: tudmathposter wird unterstützt 
%\subsectionfont: tudmathposter wird unterstützt 
%\subtitlefont: tudmathposter wird unterstützt 
%\tudfont: tudmathposter wird unterstützt 
\ToDo[imp]{%
  Unterstützung für \emph{alle} Klassen von Klaus Bergmann implementieren%
}[v2.05]



\section{Das Paket \Package*{mathswap}}
\begin{Bundle!}{\Package{mathswap}}
\index{Mathematiksatz|(}%
\index{Zifferngruppierung|(}%
%
Die Verwendung von Dezimal- und Tausendertrennzeichen im mathematischen Satz 
sind regional sehr unterschiedlich. In den meisten englischsprachigen Ländern 
wird der Punkt als Dezimaltrennzeichen und das Komma zur Zifferngruppierung 
verwendet, im restlichen Europa wird dies genau entgegengesetzt praktiziert.
Dieses Paket soll dazu dienen, beliebige formatierte Zahlen in ihrer Ausgabe 
anzupassen. Dafür werden die Zeichen Punkt (\ .\ ) und Komma (\ ,\ ) als 
aktive Zeichen im Mathematikmodus definiert.

Ähnliche Funktionalitäten werden bereits durch die Pakete \Package{icomma} und 
\Package{ziffer} bereitgestellt. Bei \Package{icomma} muss jedoch beim
Verfassen des Dokumentes durch den Autor beachtet werden, ob das verwendete
Komma einem Dezimaltrennzeichen entspricht ($t=1,\!2$) oder einem normalen 
Komma im Mathematiksatz ($z=f(x,y)$), wo ein gewisser Abstand nach dem Komma 
durchaus gewünscht ist. Das Paket \Package{ziffer} liefert dafür die gewünschte 
Funktionalität,%
\footnote{kein Leerraum nach Komma, wenn direkt danach eine Ziffer folgt}
ist allerdings etwas unflexibel, was den Umgang mit den Trennzeichen anbelangt.
Als Alternative zu diesem Paket kann außerdem \Package{ionumbers} verwendet 
werden.

Das Paket \Package{mathswap} sorgt dafür, dass Trennzeichen direkt vor einer 
Ziffer erkannt und nach bestimmten Vorgaben ersetzt werden. Sollte sich jedoch 
zwischen Trennzeichen und Ziffer Leerraum befinden, wird dieser als solcher
auch gesetzt. Für ein Beispiel zur Verwendung des Paketes sei auf das Tutorial 
\Tutorial{mathswap} in \autoref{sec:exmpl:mathswap} hingewiesen.

\begin{Declaration}{\Macro{commaswap}\Parameter{Trennzeichen}}
\begin{Declaration}{\Macro{dotswap}\Parameter{Trennzeichen}}
\printdeclarationlist%
%
Die beiden Befehle \Macro{commaswap} und \Macro{dotswap} sind die zentrale 
Benutzerschnittstelle des Paketes. Das Makro \Macro{commaswap} definiert das 
Trennzeichen oder den Inhalt, wodurch ein Komma ersetzt werden soll, auf 
welches direkt danach eine Ziffer folgt. Normalerweise setzt \hologo{LaTeX}
nach einem Komma im mathematischen Satz zusätzlich einen horizontalen Abstand.
Bei der Ersetzung durch \Macro{commaswap} entfällt dieser. Die Voreinstellung
für \Macro{commaswap} ist deshalb auf ein Komma (,) gesetzt. Mit dem Makro 
\Macro{dotswap} kann definiert werden, wodurch der Punkt im mathematischen 
Satz ersetzt werden soll, wenn auf diesen direkt anschließend eine Ziffer 
folgt. Da der Punkt im deutschsprachigem Raum zur Gruppierung von Ziffern 
genutzt wird, ist hierfür standardmäßig ein halbes geschütztes Leerzeichen 
definiert (\bsc,).
\end{Declaration}
\end{Declaration}

\ToDo[imp]{Änderungsliste in subdeclarations splitten}[v2.05]
\ChangedAt*{v2.02:Funktionalität im Dokument umschaltbar}%
\begin{Declaration}[v2.02]{\Macro{mathswapon}}
\begin{Declaration}[v2.02]{\Macro{mathswapoff}}
\printdeclarationlist%
%
Die Funktionalität von \Package{mathswap} kann innerhalb des Dokumentes mit 
diesen beiden Befehlen an- und abgeschaltet werden. Beim Laden des Paketes ist 
es standardmäßig aktiviert.
\end{Declaration}
\end{Declaration}
\index{Mathematiksatz|)}%
\index{Zifferngruppierung|)}%
\end{Bundle!}



\section{Das Paket \Package*{twocolfix}}
\begin{Bundle!}{\Package{twocolfix}}
\index{Satzspiegel!zweispaltig|?}%
%
Der \hologo{LaTeXe}-Kernel enthält einen Fehler, der Kapitelüberschriften im
zweispaltigen Layout höher setzt, als im einspaltigen. Der 
\hrfn{http://www.latex-project.org/cgi-bin/ltxbugs2html?pr=latex/3126}{Fehler}
ist zwar schon länger bekannt, allerdings bisher noch nicht in den 
\hologo{LaTeXe}-Kernel übernommen worden. Das Paket \Package{twocolfix} behebt 
das Problem. Eine Integration dieses Bugfixes in \KOMAScript{} wurde bereits 
bei Markus Kohm angefragt, jedoch von ihm bis jetzt 
\hrfn{http://www.komascript.de/node/1681}{nicht weiter verfolgt}.
\end{Bundle!}


\addsec*{Zukünftige Arbeiten}
Diese Dinge sollen langfristig in das \TUDScript-Bundle eingearbeitet werden:

%\chapter{Das Paket \Package{tudscrletter} -- Briefe im \CD}
%\begin{Bundle!}{\Package{tudscrletter}}
\ToDo[imp]{Paket \Package*{tudscrletter} für Briefe im \CD}[v2.07]
Es soll das Paket \Package*{tudscrletter}<tudscr> für Briefe im \CD der \TnUD 
entstehen. Auch Klassen für Fax und Hausmitteilungen sollen dabei abfallen.
%\end{Bundle!}
%
%\chapter{Das Paket \Package{tudscrbeamer} -- Präsentationen im \CD}
%\begin{Bundle!}{\Package{tudscrbeamer}}
\ToDo[imp]{Paket \Package*{tudscrbeamer} für Präsentationen im \CD}[v2.08]
Mit \Package*{tudscrbeamer}<tudscr> soll ein Paket entwickelt werden, mit dem 
sich \hologo{LaTeX}-Beamer-Präsentationen im Stil der \TnUD erstellt werden 
können.
%\end{Bundle!}
%
%\chapter{Das Paket \Package{tudscrlayout} -- Seitenstil und Satzspiegel im \CD}
%\begin{Bundle!}{\Package{tudscrlayout}}
\ToDo[imp]{%
  Paket \Package*{tudscrlayout} \url{https://github.com/tud-cd/tud-cd/issues/6}
  \Option*{cdgeometry=forced} für Satzspiegel
}[v2.09]
Außerdem ist ein Paket \Package*{tudscrlayout}<tudscr> vorstellbar, mit dem 
entweder die \PageStyle{tudheadings}-Seitenstile in anderen Klassen genutzt 
werden können oder der durch das \CD vorgegebene Satzspiegel ohne den 
Seitenstil selbst verwendet wird, um vorgedrucktes Papier verwenden zu können.
%\end{Bundle!}
