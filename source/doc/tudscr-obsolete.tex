\chapter{Obsolete sowie vollständig entfernte Optionen und Befehle}
\label{sec:obsolete}%
%
Einige Optionen und Befehle waren während der Weiterentwicklung von \TUDScript
in ihrer ursprünglichen Form nicht mehr umsetzbar oder wurden schlichtweg 
verworfen. Dennoch wird hier \emph{teilweise} gezeigt, wie die Funktionalität 
mit \TUDScript in der Version \vTUDScript{} darstellbar ist.

\begin{Declaration}{\Option{cd}[alternative]}{entfällt}
\begin{Declaration}{\Option{cdtitle}[alternative]}{entfällt}
\begin{Declaration}{\Length{titlecolwidth}}{entfällt}
\begin{Declaration}{\Term{authortext}}{entfällt}
\printdeclarationlist*%
\index{Titel!alternativer}%
%
Die alternative Titelseite ist komplett aus dem \TUDScript-Bundle entfernt 
worden. Dementsprechend entfallen auch die dazugehörigen Optionen sowie Länge 
und Bezeichner.
\end{Declaration}
\end{Declaration}
\end{Declaration}
\end{Declaration}

\begin{Declaration}{\Option{color}[\PBoolean]}{siehe \Option*{cd}[color]}
\printdeclarationlist*%
%
Die Einstellungen der farbigen Ausprägung des Dokumentes erfolgt über die 
Option \Option*{cd}.
\end{Declaration}

\begin{Declaration}{\Option{tudfonts}[\PBoolean]}{siehe \Option*{cdfont}[\PSet]}
\printdeclarationlist*%
%
Die Option zur Schrifteinstellung ist wesentlich erweitert worden. Aus Gründen 
der Konsistenz wurde diese umbenannt.
\end{Declaration}

\begin{Declaration}{\Option{tudfoot}[\PBoolean]}{
  siehe \Option*{cdfoot}[\PBoolean]%
}
\printdeclarationlist*%
%
Ebenso wurde diese Option umbenannt, um dem Namensschema der restlichen 
Optionen zu entsprechen.
\end{Declaration}

\begin{Declaration}{\Option{headfoot}[\PSet]}{entfällt}
\printdeclarationlist*%
%
Diese Option war in der Version~v1.0 notwendig, um die parallele Verwendung von 
\Package{typearea} und \Package{geometry} zu ermöglichen. Dies wurde komplett 
überarbeitet, an das Paket \Package{geometry} werden die Einstellungen für die 
\KOMAScript"=Optionen \Option{headinclude} und \Option{footinclude} jetzt 
direkt weitergereicht. Damit ist die Option \Option*{headfoot} nicht mehr 
notwendig und wurde entfernt.
\end{Declaration}

\begin{Declaration}{\Option{partclear}[\PBoolean]}{%
  entfällt, siehe \Option*{cleardoublespecialpage}%
}
\begin{Declaration}{\Option{chapterclear}[\PBoolean]}{%
  entfällt, siehe \Option*{cleardoublespecialpage}%
}
\printdeclarationlist*%
%
Beide Optionen sind in der neuen Option \Option*{cleardoublespecialpage} 
aufgegangen, womit ein konsistentes Layout erreicht wird. Die ursprünglichen 
Optionen entfallen. 
\end{Declaration}
\end{Declaration}

\begin{Declaration}{\Option{abstracttotoc}[\PBoolean]}{%
  entfällt, siehe \Option*{abstract}[\PSet]%
}
\begin{Declaration}{\Option{abstractdouble}[\PBoolean]}{%
  entfällt, siehe \Option*{abstract}[\PSet]%
}
\printdeclarationlist*%
%
Beide Optionen wurden in die Option \Option*{abstract} integriert und sind 
deshalb überflüssig.
\end{Declaration}
\end{Declaration}

\begin{Declaration}{\Macro{confirmationandrestriction}}{%
  entfällt, siehe \Macro*{declaration}%
}
\begin{Declaration}{\Macro{restrictionandconfirmation}}{%
  entfällt, siehe \Macro*{declaration}%
}
\begin{Declaration}{\Macro{location}\Parameter{Ort}}{%
  siehe \Macro*{place} sowie auch Parameter \Key*{\Macro{declaration}}{place}%
}
\printdeclarationlist*%
%
Die ersten beiden Befehle entfallen, \Macro*{declaration} kann alternativ dazu 
verwendet werden. In Anlehnung an andere \hologo{LaTeX}-Pakete und "~Klassen 
wurde \Macro*{location} in \Macro*{place} umbenannt.
\end{Declaration}
\end{Declaration}
\end{Declaration}

\begin{Declaration}{\Macro{logofile}\Parameter{Dateiname}}%
  {siehe \Macro*{headlogo}\Parameter{Dateiname}%
}
\begin{Declaration}{\Macro{logofilename}\Parameter{Dateiname}}%
  {siehe \Macro*{headlogo}\Parameter{Dateiname}%
}
\printdeclarationlist*%
%
Der Befehl \Macro*{logofile} wurde in \Macro*{headlogo} umbenannt.
\end{Declaration}
\end{Declaration}

\begin{Declaration}{\Length{chapterheadingvskip}}{%
  siehe \Length*{pageheadingsvskip} sowie \Length*{headingsvskip}
}
\begin{Declaration}{\Length{signatureheight}}{entfällt}
\printdeclarationlist*%
%
Die vertikale Positionierung von Überschriften wurde zweigeteilt. Die Höhe für 
die Zeile der Unterschriften wurde dehnbar gestaltet. Eine Anpassung durch den 
Anwender ist nicht vonnöten.
\end{Declaration}
\end{Declaration}

\begin{Declaration}{\Term{titlecoldelim}}{%
  entfällt, siehe \Macro*{titledelimiter}%
}
\printdeclarationlist*%
%
Das Trennzeichen für Bezeichnungen beziehungsweise beschreibende Texte und dem 
eigentlichen Feld auf der Titelseite ist nicht mehr sprachabhängig und wurde 
umbenannt.
\end{Declaration}

\begin{Declaration}{\Macro{submissiondate}\Parameter{Datum}}{%
  Alias für \Macro*{date}%
}
\begin{Declaration}{\Macro{birthday}\Parameter{Geburtsdatum}}{%
  Alias für \Macro*{dateofbirth}%
}
\begin{Declaration}{\Macro{birthplace}\Parameter{Geburtsort}}{%
  Alias für \Macro*{placeofbirth}
}
\begin{Declaration}{\Macro{studentid}\Parameter{Matrikelnummer}}{%
  Alias für \Macro*{matriculationnumber}
}
\begin{Declaration}{\Macro{enrolmentyear}\Parameter{Immatrikulationsjahr}}{%
  Alias für \Macro*{matriculationyear}%
}
\printdeclarationlist*%
%
Alle Befehle wurden umbenannt und sind jetzt für die Titelseite im \CD nutzbar.
\end{Declaration}
\end{Declaration}
\end{Declaration}
\end{Declaration}
\end{Declaration}

\begin{Declaration}{\Term{submissiontext}}{umbenannt, siehe \Term*{datetext}}
\begin{Declaration}{\Term{birthdaytext}}{%
  umbenannt, siehe \Term*{dateofbirthtext}%
}
\begin{Declaration}{\Term{birthplacetext}}{%
  umbenannt, siehe \Term*{placeofbirthtext}%
}
\begin{Declaration}{\Term{studentidname}}{%
  umbenannt, siehe \Term*{matriculationnumbername}%
}
\begin{Declaration}{\Term{enrolmentname}}{%
  umbenannt, siehe \Term*{matriculationyearname}%
}
\begin{Declaration}{\Term{supervisorIIname}}{%
  umbenannt, siehe \Term*{supervisorothername}%
}
\begin{Declaration}{\Term{defensetext}}{%
  umbenannt, siehe \Term*{defensedatetext}%
}
\printdeclarationlist*%
%
Die Bezeichner wurden in Anlehnung an die dazugehörigen Befehlsnamen umbenannt.
\end{Declaration}
\end{Declaration}
\end{Declaration}
\end{Declaration}
\end{Declaration}
\end{Declaration}
\end{Declaration}


\minisec{Aufgabenstellung}
Die Umgebung für die Erstellung einer Aufgabenstellung für eine 
wissenschaftliche Arbeit wurde in das Paket \Package{tudscrsupervisor} 
ausgelagert. Dieses muss für die Verwendung der Umgebung \Environment*{task} 
und der daraus abgeleiteten standardisierten Form zwingend geladen werden.

\begin{Declaration}{\Option{cdtask}[\PSet]}{entfällt, siehe \Environment*{task}}
\begin{Declaration}{\Option{taskcompact}[\PBoolean]}{entfällt}
\begin{Declaration}{\Macro{tasks}\Parameter{Ziele}\Parameter{Schwerpunkte}}{%
  umbenannt, siehe \Macro*{taskform}%
}
\begin{Declaration}{\Length{taskcolwidth}}{entfällt}
\printdeclarationlist*%
%
Die Klassenoption \Option*{cdtask} ist komplett entfernt worden, alle 
Einstellungen, welche \Environment*{task} betreffen erfolgen direkt über das 
optionale Argument der Umgebung. Die Variante eines kompakten Kopfes mit der 
Option \Option*{taskcompact} wird nicht mehr bereitgestellt. Der Befehl 
\Macro*{tasks} wurde in \Macro*{taskform} umbenannt und in der Funktionalität 
erweitert. Die manuelle Einstellung der Spaltenbreite für den Kopf der 
Aufgabenstellung mit \Length*{taskcolwidth} wurde aufgrund der verbesserten 
automatischen Berechnung entfernt.
\end{Declaration}
\end{Declaration}
\end{Declaration}
\end{Declaration}

\begin{Declaration}{\Macro{startdate}\Parameter{Ausgabedatum}}{%
  Alias für \Macro*{issuedate}%
}
\begin{Declaration}{\Macro{enddate}\Parameter{Abgabetermin}}{%
  Alias für \Macro*{duedate}%
}
\begin{Declaration}{\Term{starttext}}{umbenannt, siehe \Term*{issuedatetext}}
\begin{Declaration}{\Term{duetext}}{umbenannt, siehe \Term*{duedatetext}}
\begin{Declaration}{\Term{focustext}}{umbenannt, siehe \Term*{focusname}}
\begin{Declaration}{\Term{objectivestext}}{%
  umbenannt, siehe \Term*{objectivesname}%
}
\printdeclarationlist*%
%
Alle genannten Befehle und Bezeichner wurden umbenannt.
\end{Declaration}
\end{Declaration}
\end{Declaration}
\end{Declaration}
\end{Declaration}
\end{Declaration}
