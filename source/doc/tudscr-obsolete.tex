\chapter{Obsolete sowie vollständig entfernte Optionen und Befehle}
\label{sec:obsolete}%
%
\ToDo[doc]{Eintragen der Änderungen beenden}[v2.03]
\ToDo[doc]{Änderungen der Version~v2.03 dokumentieren}[v2.03]
\ToDo[doc]{\enquote{siehe} durch \Macro*{see} ersetzen}[v2.03]
\section{Veraltete Optionen und Befehle in \TUDScript}
Einige Optionen und Befehle waren während der Weiterentwicklung von \TUDScript
in ihrer ursprünglichen Form nicht mehr umsetzbar oder wurden~-- unter anderem 
aus Gründen der Kompatibilität zu anderen Paketen~-- schlichtweg verworfen. 
Dennoch besteht für die meisten entfallenen Direktiven eine Möglichkeit, deren 
Funktionalität ohne größere Aufwände mit \TUDScript in der aktuellen Version 
\vTUDScript{} darzustellen. Ist dies der Fall, wird hier entsprechend kurz 
darauf hingewiesen.


\subsection{Änderungen für \TUDScript~v2.00}
\begin{Declaration}[v2.00]{\Option*{cd}[alternative]}
\begin{Declaration}[v2.00]{\Option*{cdtitle}[alternative]}
\begin{Declaration}[v2.00]{\Length{titlecolwidth}}
\begin{Declaration}[v2.00]{\Term{authortext}}
\printdeclarationlist*[\emph{entfällt}]%
\index{Titel!alternativer}%
%
Die alternative Titelseite ist komplett aus dem \TUDScript-Bundle entfernt 
worden. Dementsprechend entfallen auch die dazugehörigen Optionen sowie Länge 
und Bezeichner.
\end{Declaration}
\end{Declaration}
\end{Declaration}
\end{Declaration}

\Replace*{v2.00}{\Option{cd}}{\Option{color}}
\begin{Declaration}[v2.00]{\Option{color}[\PBoolean]}
\printdeclarationlist*%
%
Die Einstellungen der farbigen Ausprägung des Dokumentes erfolgt über die 
Option \Option*{cd}.
\end{Declaration}

\Replace{v2.00}{\Option{cdfont}}{\Option{tudfonts}}
\begin{Declaration}[v2.00]{\Option{tudfonts}[\PBoolean]}
\begin{Declaration}[v2.00]{\Option*{cdfonts}[\PBoolean]}
\printdeclarationlist*%
%
Die Option zur Schrifteinstellung ist wesentlich erweitert worden. Aus Gründen 
der Konsistenz wurde diese umbenannt.
\end{Declaration}
\end{Declaration}

\Replace{v2.00}{\Option{cdfoot}}{\Option{tudfoot}}
\begin{Declaration}[v2.00]{\Option{tudfoot}[\PBoolean]}
\printdeclarationlist*%
%
Ebenso wurde die Option \Option*{tudfoot} umbenannt, um dem Namensschema der 
restlichen Optionen von \TUDScript zu entsprechen.
\end{Declaration}

\begin{Declaration}[v2.00]{\Option{headfoot}[\PSet]}
\printdeclarationlist*[\emph{entfällt}, 
  \KOMAScript-Optionen \Option*{headinclude} und \Option*{footinclude}%
]%
%
Diese Option war für \TUDScript in der Version~v1.0 notwendig, um die parallele 
Verwendung der beiden Pakete \Package{typearea} und \Package{geometry} zu 
ermöglichen. Die Erstellung des Satzspiegels wurde komplett überarbeitet. 
Mittlerweile werden an das Paket \Package{geometry} die Einstellungen für die 
\KOMAScript"=Optionen \Option{headinclude} und \Option{footinclude} direkt 
weitergereicht, so dass die Option \Option*{headfoot} nicht mehr notwendig ist 
und deshalb entfernt wurde.
\end{Declaration}

\Replace{v2.00}{\Option{cleardoublespecialpage}}[die Optionen]{%
  \Option{partclear},\Option{chapterclear}%
}
\begin{Declaration}[v2.00]{\Option{partclear}[\PBoolean]}
\begin{Declaration}[v2.00]{\Option{chapterclear}[\PBoolean]}
\printdeclarationlist*[\emph{entfällt}]%
Beide Optionen sind in der neuen Option \Option*{cleardoublespecialpage} 
aufgegangen, womit ein konsistentes Layout erreicht wird. Die ursprünglichen 
Optionen entfallen. 
\end{Declaration}
\end{Declaration}

\Replace*{v2.00}{\Option{abstract}}{%
  \Option{abstracttotoc},\Option{abstractdouble}%
}
\begin{Declaration}[v2.00]{\Option{abstracttotoc}[\PBoolean]}
\begin{Declaration}[v2.00]{\Option{abstractdouble}[\PBoolean]}
\printdeclarationlist*[\emph{entfällt}]%
%
Beide Optionen wurden in die Option \Option*{abstract} integriert und sind 
deshalb überflüssig.
\end{Declaration}
\end{Declaration}

\Replace{v2.00}{\Macro{headlogo}}{\Macro{logofile},\Macro{logofilename}}
\begin{Declaration}[v2.00]{\Macro{logofile}\Parameter{Dateiname}}
\begin{Declaration}[v2.00]{\Macro{logofilename}\Parameter{Dateiname}}
\printdeclarationlist*%
%
Der Befehl \Macro*{logofile} sowie der dazugehörige Alias \Macro*{logofilename} 
wurden in \Macro*{headlogo} umbenannt. Die Funktionalität bleibt weiterhin 
bestehen.
\end{Declaration}
\end{Declaration}

\begin{Declaration}[v2.00]{\Length{signatureheight}}
\printdeclarationlist*[\emph{entfällt}]%
%
Die Höhe für die Zeile der Unterschriften wurde dehnbar gestaltet, eine etwaige 
Anpassung durch den Anwender ist nicht vonnöten.
\end{Declaration}

\Replace{v2.00}{\Macro{titledelimiter}}{\Term{titlecoldelim}}
\begin{Declaration}[v2.00]{\Term{titlecoldelim}}%
\printdeclarationlist*[\emph{entfällt}]%
%
Das Trennzeichen für Bezeichnungen beziehungsweise beschreibende Texte und dem 
eigentlichen Feld auf der Titelseite ist nicht mehr sprachabhängig und wurde 
umbenannt.
\end{Declaration}

\minisec{\taskname}
Die Umgebung für die Erstellung einer Aufgabenstellung für eine 
wissenschaftliche Arbeit wurde in das Paket \Package{tudscrsupervisor} 
ausgelagert. Dieses muss für die Verwendung der Umgebung \Environment*{task} 
und der daraus abgeleiteten standardisierten Form zwingend geladen werden.

\Replace*{v2.00}{\Environment{task}}{\Option{cdtask}}
\begin{Declaration}[v2.00]{\Option{cdtask}[\PSet]}
\begin{Declaration}[v2.00]{\Option{taskcompact}[\PBoolean]}
\begin{Declaration}[v2.00]{\Length{taskcolwidth}}
\printdeclarationlist*[\emph{entfällt}]%
%
Die Klassenoption \Option*{cdtask} ist komplett entfernt worden, alle 
Einstellungen, welche \Environment*{task} betreffen erfolgen direkt über das 
optionale Argument der Umgebung. Die Variante eines kompakten Kopfes mit der 
Option \Option*{taskcompact} wird nicht mehr bereitgestellt. Die manuelle 
Einstellmöglichkeit der Spaltenbreite für den Kopf der Aufgabenstellung mit 
\Length*{taskcolwidth} wurde aufgrund der verbesserten automatischen Berechnung 
entfernt.
\end{Declaration}
\end{Declaration}
\end{Declaration}

\Replace{v2.00}{\Macro{taskform}}{\Macro{tasks}}
\begin{Declaration}[v2.00]{%
  \Macro{tasks}\Parameter{Ziele}\Parameter{Schwerpunkte}%
}
\printdeclarationlist*%
%
Der Befehl \Macro*{tasks} wurde in \Macro*{taskform} umbenannt und in der 
Funktionalität erweitert.
\end{Declaration}

\Replace*{v2.00}{\Macro{date}}{\Macro{submissiondate}}
\begin{Declaration}[v2.00]{\Macro{submissiondate}\Parameter{Datum}}{%
  \see*{\Macro*{date}}%
}
\Replace*{v2.00}{\Macro{dateofbirth}}{\Macro{birthday}}
\begin{Declaration}[v2.00]{\Macro{birthday}\Parameter{Geburtsdatum}}{%
  \see*{\Macro*{dateofbirth}}%
}
\Replace*{v2.00}{\Macro{placeofbirth}}{\Macro{birthplace}}
\begin{Declaration}[v2.00]{\Macro{birthplace}\Parameter{Geburtsort}}{%
  \see*{\Macro*{placeofbirth}}%
}
\printdeclarationlist*%
%
Alle Befehle wurden umbenannt und sind jetzt neben der \taskname{} auch für die 
Titelseite im \CD nutzbar.
\end{Declaration}
\end{Declaration}
\end{Declaration}
\
\subsection{Änderungen für \TUDScript~v2.02}
\Replace{v2.02}{\Length{pageheadingsvskip}}{\Length{chapterheadingvskip}}
\begin{Declaration}[v2.02]{\Length{chapterheadingvskip}}
\printdeclarationlist*%
Die vertikale Positionierung von Überschriften wurde aufgeteilt. Zum einen kann 
diese für Titel-, Teile- und Kapitelseiten (\Option*{chapterpage}[true]) über 
die Länge \Length*{pageheadingsvskip} geändert werden. Für Kapitelüberschriften 
(\Option*{chapterpage}[false]) sowie den Titelkopf (\Option*{titlepage}[false]) 
kann dies unabhängig davon mit \Length*{headingsvskip} erfolgen.
\end{Declaration}

\begin{Declaration}[v2.00]{\Macro{studentid}\Parameter{Matrikelnummer}}{%
  \see*{\Macro*{matriculationnumber}}%
}
\begin{Declaration}[v2.00]{\Macro{matriculationid}\Parameter{Matrikelnummer}}{%
  \see*{\Macro*{matriculationnumber}}%
}
\begin{Declaration}{\Macro{enrolmentyear}\Parameter{Immatrikulationsjahr}}{%
  \see*{\Macro*{matriculationyear}}%
}
\printdeclarationlist*%
%
Alle Befehle wurden umbenannt und sind auch für die Titelseite im \CD nutzbar.
\end{Declaration}
\end{Declaration}
\end{Declaration}


% §§§ hier geht's weiter
\begin{Declaration}{\Term{submissiontext}}{umbenannt, siehe \Term*{datetext}}
\begin{Declaration}{\Term{birthdaytext}}{%
  umbenannt, siehe \Term*{dateofbirthtext}%
}
\begin{Declaration}{\Term{birthplacetext}}{%
  umbenannt, siehe \Term*{placeofbirthtext}%
}
\begin{Declaration}{\Term{studentidname}}{%
  umbenannt, siehe \Term*{matriculationnumbername}%
}
\begin{Declaration}{\Term{enrolmentname}}{%
  umbenannt, siehe \Term*{matriculationyearname}%
}
\begin{Declaration}{\Term{supervisorIIname}}{%
  umbenannt, siehe \Term*{supervisorothername}%
}
\begin{Declaration}{\Term{defensetext}}{%
  umbenannt, siehe \Term*{defensedatetext}%
}
\printdeclarationlist*%
%
Die Bezeichner wurden in Anlehnung an die dazugehörigen Befehlsnamen umbenannt.
\end{Declaration}
\end{Declaration}
\end{Declaration}
\end{Declaration}
\end{Declaration}
\end{Declaration}
\end{Declaration}

\begin{Declaration}{\Macro{degree}\OParameter{Kurzform}\Parameter{Grad}}{%
  umbenannt, siehe \Macro*{graduation}%
}
\begin{Declaration}{\Term{degreetext}}{umbenannt, siehe \Term*{graduationtext}}
\printdeclarationlist*%
%
Der Befehl wurde zur Erhöhung der Kompatibilität mit anderen Paketen umbenannt, 
der dazugehörige Bezeichner dahingehend angepasst.
\end{Declaration}
\end{Declaration}

\begin{Declaration}{\Macro{restriction}\OLParameter{Firma}}{%
  umbenannt, siehe \Macro*{blocking}%
}
\begin{Declaration}{\Term{restrictionname}}{%
  umbenannt, siehe \Term*{blockingname}
}
\begin{Declaration}{\Term{restrictiontext}}{%
  umbenannt, siehe \Term*{blockingtext}%
}
\printdeclarationlist*%
%
Der Befehl wurde zur Erhöhung der Kompatibilität mit anderen Paketen umbenannt, 
die dazugehörigen Bezeichner dahingehend angepasst.
\end{Declaration}
\end{Declaration}
\end{Declaration}

\begin{Declaration}{\Macro{confirmationandrestriction}}{%
  entfällt, siehe \Macro*{declaration}%
}
\begin{Declaration}{\Macro{restrictionandconfirmation}}{%
  entfällt, siehe \Macro*{declaration}%
}
\printdeclarationlist*%
%
Die beiden Befehle entfallen, stattdessen sollte entweder der Befehl 
\Macro*{declaration} oder die Umgebung \Environment{declarations} zusammen mit 
den Befehlen \Macro*{confirmation} und \Macro*{blocking} verwendet werden, 
wobei sich diese in der Umgebung in beliebiger Reihenfolge anordnen lassen.
\end{Declaration}
\end{Declaration}

\begin{Declaration}{\Macro{location}\Parameter{Ort}}{%
  siehe \Macro*{place} sowie auch Parameter \Key*{\Macro{declaration}}{place}%
}
\printdeclarationlist*%
In Anlehnung an andere \hologo{LaTeX}-Pakete und "~Klassen wurde 
\Macro*{location} in \Macro*{place} umbenannt.
\end{Declaration}



\minisec{\taskname{} und \noticename}
\begin{Declaration}{\Macro{branch}\Parameter{Studienrichtung}}{%
  umbenannt, siehe \Macro*{discipline}%
}
\begin{Declaration}{\Term{branchname}}{umbenannt, siehe \Term*{disciplinename}}
\begin{Declaration}{\Macro{startdate}\Parameter{Ausgabedatum}}{%
  Alias für \Macro*{issuedate}%
}
\begin{Declaration}{\Macro{enddate}\Parameter{Abgabetermin}}{%
  Alias für \Macro*{duedate}%
}
\begin{Declaration}{\Term{starttext}}{umbenannt, siehe \Term*{issuedatetext}}
\begin{Declaration}{\Term{duetext}}{umbenannt, siehe \Term*{duedatetext}}
\begin{Declaration}{\Term{focustext}}{umbenannt, siehe \Term*{focusname}}
\begin{Declaration}{\Term{objectivestext}}{%
  umbenannt, siehe \Term*{objectivesname}%
}
\printdeclarationlist*%
%
Alle genannten Befehle und Bezeichner wurden für die \taskname{} umbenannt.
\end{Declaration}
\end{Declaration}
\end{Declaration}
\end{Declaration}
\end{Declaration}
\end{Declaration}
\end{Declaration}
\end{Declaration}

\begin{Declaration}{\Macro{contact}\Parameter{Kontaktperson(en)}}{%
  umbenannt, siehe \Macro*{contactperson}%
}
\begin{Declaration}{\Term{contactname}}{%
  umbenannt, siehe \Term*{contactpersonname}%
}
\begin{Declaration}{\Macro{phone}\Parameter{Telefonnummer}}{%
  umbenannt, siehe \Macro*{telephone}%
}
\begin{Declaration}{\Macro{email}\Parameter{E-Mail-Adresse}}{%
  umbenannt, siehe \Macro*{emailaddress}%
}
\printdeclarationlist*%
%
Alle genannten Befehle und Bezeichner wurden für den \noticename{} umbenannt.
\end{Declaration}
\end{Declaration}
\end{Declaration}
\end{Declaration}



\section{Das Paket \Package{tudscrcomp} -- Umstieg von anderen Klassen}
\begin{Declaration*}{\Package{tudscrcomp}}
\index{Version!v1.0|(}%
\index{Kompatibilität!Version v1.0|(}%
\index{Kompatibilität!\Class{tudbook}|(}%
%
\noindent\Attention{%
  Sollten Sie eine der Klassen \Class{tudbook}, \Class{tudbeamer}, 
  \Class{tudletter}, \Class{tudfax}, \Class{tudhaus} sowie \Class{tudform} oder 
  \TUDScript in der Version~v1.0 nie genutzt haben, können Sie dieses 
  \autorefname ohne Weiteres überspringen. Sämtliche hier vorgestellten 
  Optionen und Befehle sind in der aktuellen Version von \TUDScript obsolet.
}

\bigskip\noindent
Zu Beginn der Entwicklung von \TUDScript bildete die Klasse \Class{tudbook}
die Basis. Ziel war es, sämtliche Funktionalitäten dieser beizubehalten und 
zusätzlich den vollen Funktionsumfang der \KOMAScript-Klassen nutzbar zu 
machen. Bei der kompletten Neuimplementierung der \TUDScript-Klassen wurde sehr 
viel verändert und verbessert. Einige der Optionen und Befehle waren jedoch 
bereits in der \TUDScript-Version~v1.0 Relikte, um die Kompatibilität zur 
\Class{tudbook}-Klasse und ihren Derivaten zu gewährleisten. Mit \TUDScript in 
der Version~v2.00 wurden einige aus Gründen der Konsistenz lediglich umbenannt, 
andere wiederum wurden vollständig entfernt oder über neue Befehle und Optionen 
in ihrer Funktionalität ersetzt und erweitert. 

Das Paket \Package{tudscrcomp} dient der Überführung von alten Dokumenten, die 
entweder mit der \Class{tudbook}-Klasse, ihren Derivaten oder mit \TUDScript in 
der Version~v1.0 erstellt wurden, auf \TUDScript~\vTUDScript. Es werden einige 
Optionen und Befehle bereitgestellt, welche von den alten Klassen definiert 
wurden und das entsprechende Verhalten nachahmen. Damit soll die Kompatibilität 
bei der Änderung der Dokumentklasse sichergestellt werden. Die Intention ist, 
alte Dokumente möglichst schnell und einfach auf die \TUDScript-Klassen 
portieren zu können. Des Weiteren ist beschrieben, wie sich die Funktionalität 
ohne die Verwendung des Paketes mit den Mitteln von \TUDScript umsetzen lassen. 
Für den Satz neuer Dokumente wird empfohlen, auf den Einsatz dieses Paketes 
komplett zu verzichten und stattdessen die neuen Befehle zu nutzen.

\begin{Declaration}{\Option{serifmath}}{%
  identisch zu \Option{cdmath}[false]%
}
\printdeclarationlist%
%
Die Funktionalität wird durch die Option \Option{cdmath} bereitgestellt.
\end{Declaration}

\begin{Declaration}{\Option{colortitle}}{%
  identisch zu \Option{cdtitle}[color]%
}
\begin{Declaration}{\Option{nocolortitle}}{%
  identisch zu \Option{cdtitle}[true]%
}
\printdeclarationlist%
%
Die Funktionalität wird durch die Option \Option{cdtitle} bereitgestellt.
\end{Declaration}
\end{Declaration}

\begin{Declaration}{\Macro{einrichtung}\Parameter{Fakultät}}{%
  identisch zu \Macro{faculty}
}
\begin{Declaration}{\Macro{fachrichtung}\Parameter{Einrichtung}}{%
  identisch zu \Macro{department}
}
\begin{Declaration}{\Macro{institut}\Parameter{Institut}}{%
  identisch zu \Macro{institute}
}
\begin{Declaration}{\Macro{professur}\Parameter{Lehrstuhl}}{%
  identisch zu \Macro{chair}
}
\printdeclarationlist%
%
Dies sind die deutschsprachigen Befehle für den Kopf im \CD.
\end{Declaration}
\end{Declaration}
\end{Declaration}
\end{Declaration}

\begin{Declaration}{\Macro{moreauthor}\Parameter{Autorenzusatz}}{%
  identisch zu \Macro{authormore}%
}
\printdeclarationlist%
%
Ursprünglich war diese Befehl für das Unterbringen aller möglichen, 
zusätzlichen Autoreninformationen gedacht. Auch der Befehl \Macro{authormore} 
ist ein Rudiment davon. Empfohlen wird die Verwendung der Befehle 
\Macro{dateofbirth}, \Macro{placeofbirth}, \Macro{matriculationnumber} und 
\Macro{matriculationyear} sowie für die Aufgabenstellung einer 
wissenschaftlichen Arbeit \Macro{course} und \Macro{discipline} aus dem Paket 
\Package{tudscrsupervisor}.
\end{Declaration}

\begin{Declaration}{\Macro{submitdate}\Parameter{Datum}}{%
  identisch zu \Macro{date}%
}
\printdeclarationlist%
%
Die Funktionalität wird durch den erweiterten Standardbefehl \Macro{date} 
abgedeckt.
\end{Declaration}

\begin{Declaration}{\Macro{supervisorII}\Parameter{Name}}{%
  identisch zur Verwendung von \Macro{and} innerhalb von \Macro{supervisor}%
}
\printdeclarationlist%
%
Es ist \Macro{supervisor}\PParameter{\PName{Name} \Macro{and} \PName{Name}}
statt \Macro{supervisorII}\Parameter{Name} zu verwenden.
\end{Declaration}

\begin{Declaration}{\Macro{supervisedby}\Parameter{Bezeichnung}}{%
  siehe \Term{supervisorname}%
}
\begin{Declaration}{\Macro{supervisedIIby}\Parameter{Bezeichnung}}{%
  siehe \Term{supervisorothername}%
}
\begin{Declaration}{\Macro{submittedon}\Parameter{Bezeichnung}}{%
  siehe \Term{datetext}%
}
\printdeclarationlist%
%
Zur Änderung der Bezeichnung der Betreuer sollten die sprachabhängigen 
Bezeichner wie in \autoref{sec:localization} beschrieben angepasst werden. Eine 
Verwendung der alten Befehle entfernt die Abhängigkeit der Bezeichner von der 
verwendeten Sprache.
\end{Declaration}
\end{Declaration}
\end{Declaration}

\begin{Declaration}{\Option{ddcfooter}}{%
  identisch zu \Option{ddcfoot}[true]%
}
\printdeclarationlist%
%
Die Funktionalität wird durch die Option \Option{ddcfoot} bereitgestellt.
\end{Declaration}

\begin{Declaration}{\Macro{dissertation}}
\printdeclarationlist%
%
Die Funktionalität kann durch die Befehle \Macro{thesis}\PParameter{diss} und 
\Macro{referee} sowie die Bezeichner \Term{refereename} und 
\Term{refereeothername} dargestellt werden.
\end{Declaration}

\begin{Declaration}{\Macro{chapterpage}}
\printdeclarationlist%
%
Durch diesen Befehl können Kapitelseiten konträr zur eigentlichen Einstellung 
aktiviert oder deaktiviert werden. Prinzipiell ist dies auch durch eine 
Änderung der Option \Option{chapterpage} möglich. Allerdings wird davon 
abgeraten, da dies zu einem inkonsistenten Layout innerhalb des Dokumentes 
führt.
\end{Declaration}

\begin{Declaration}{\Environment{theglossary}[\OParameter{Präambel}]}
\begin{Declaration}{\Macro{glossitem}\Parameter{Begriff}}
\printdeclarationlist%
%
Die \Class{tudbook}-Klasse stellt eine rudimentäre Umgebung für ein Glossar 
bereit. Allerdings gibt es dafür bereits zahlreiche und besser implementierte 
Pakete. Daher wird für diese Umgebung keine Portierung vorgenommen, sondern 
lediglich die ursprüngliche Definition übernommen. Allerdings sein an dieser 
Stelle auf wesentlich bessere Lösungen wie beispielsweise das Paket 
\Package{glossaries} oder~-- mit Abstrichen~-- das nicht ganz so umfangreiche 
Paket \Package{nomencl} verwiesen. 
\end{Declaration}
\end{Declaration}
\index{Version!v1.0|)}
\index{Kompatibilität!Version v1.0|)}
\index{Kompatibilität!\Class{tudbook}|)}
\end{Declaration*}