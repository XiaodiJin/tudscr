\RequirePackage[ngerman=ngerman-x-latest]{hyphsubst}
\documentclass[english,ngerman,final,parskip=no]{tudscrman}
\usepackage{selinput}
\SelectInputMappings{adieresis={ä},germandbls={ß}}
\usepackage[T1]{fontenc}
\lstset{%
  inputencoding=utf8,extendedchars=true,
  literate=%
    {ä}{{\"a}}1 {ö}{{\"o}}1 {ü}{{\"u}}1
    {Ä}{{\"A}}1 {Ö}{{\"O}}1 {Ü}{{\"U}}1
    {~}{{\textasciitilde}}1 {ß}{{\ss}}1
}
\begin{document}
Die meisten Anwender der \TUDScript-Klassen sind Studenten oder angehörige der 
\TnUD, die ihre ersten Schritte mit \hologo{LaTeXe} beim Verfassen einer 
wissenschaftlichen Arbeit oder ähnlichem machen. Während der Einstiegsphase in 
\hologo{LaTeXe} kann ein Anfänger sehr schnell aufgrund der großen Anzahl an 
empfohlenen Pakete sowie der teilweise diametral zueinander stehenden Hinweise 
überfordert sein. Mit dem Tutorial \Tutorial{treatise} soll versucht werden, 
ein wenig Licht ins Dunkel zu bringen. Es erhebt jedoch keinerlei Anspruch, 
vollständig oder perfekt zu sein. Einige der darin vorgestellten Möglichkeiten 
lassen sich mit Sicherheit auch anders, einfacher und/oder besser lösen. 
Dennoch ist es gerade für Neulinge~-- vielleicht auch für den einen oder 
anderen \hologo{LaTeX}"=Veteran~-- als Leitfaden für die Erstellung einer 
wissenschaftlichen Arbeit gedacht.
%\setchapterpreamble{%
  \begin{abstract}
    \hypersetup{linkcolor=red}
    Dieses Kapitel soll den Einstieg und den ersten Umgang mit \TUDScript 
    erleichtern. Dafür werden einige Minimalbeispiele gegeben, die einzelne 
    Funktionalitäten darstellen. Diese sind so reduziert ausgeführt, dass sie 
    sich dem Anwender direkt erschließen sollten. Des Weiteren werden 
    weiterführende und kommentierte Anwendungsbeispiele bereitgestellt. Diese 
    Tutorials sind nicht unmittelbar im Handbuch enthalten sondern werden als 
    externe Dateien bereitgehalten, welche direkt via Hyperlink geöffnet werden 
    können.
  \end{abstract}
}
\chapter{Minimalbeispiele und Tutorials}
\label{sec:exmpl}
\index{Minimalbeispiel|(}
\section{Dokument}
\index{Minimalbeispiel!Dokument}
Hier wird gezeigt, wie die Präambel eines minimalen \hologo{LaTeXe}-Dokumentes 
gestaltet werden sollte. Dieser Ausschnitt kann prinzipiell als Grundlage für 
ein neu zu erstellendes Dokument verwendet werden. Lediglich das Einbinden des 
Paketes \Package*{blindtext} mit \Macro*{usepackage}\PParameter{blindtext} und 
der daraus stammende Befehl \Macro*{blinddocument} kann weggelassen werden.
\includeexample{document}
\section{Dissertation}
\label{sec:exmpl:dissertation}
\index{Minimalbeispiel!Dissertation}
Eine Abschlussarbeit oder ähnliches könnte wie hier gezeigt begonnen werden.
\includeexample{dissertation}
\section{Abschlussarbeit (kollaborativ)}
\label{sec:exmpl:thesis}
\index{Minimalbeispiel!Abschlussarbeit}
\index{Minimalbeispiel!Kollaboratives Schreiben}
Alle zusätzlichen Angaben außerhalb des Argumentes von \Macro{author} werden 
für beide Autoren gleichermaßen übernommen.%
\footnote{In diesem Beispiel \Macro{matriculationyear}.}
Die Angaben innerhalb des Argumentes von \Macro{author} werden den jeweiligen, 
mit \Macro{and} getrennten Autoren zugeordnet.%
\footnote{%
  In diesem Beispiel \Macro{matriculationnumber}, \Macro{dateofbirth} und 
  \Macro{placeofbirth}.
}
Ohne die Verwendung von \Macro{and} kann natürlich auch nur ein Autor 
aufgeführt werden. Außerdem sei auf die Verwendung von \Macro{subject} anstelle 
von \Macro{thesis} mit einem speziellen Wert aus \autoref{tab:thesis} 
hingewiesen.
\includeexample{thesis}
\section{Aufgabenstellung (kollaborativ)}
\label{sec:exmpl:task}
\index{Minimalbeispiel!Aufgabenstellung}
\index{Minimalbeispiel!Kollaboratives Schreiben}
Eine Aufgabenstellung für eine wissenschaftliche Arbeit ist mithilfe der 
Umgebung \Environment{task} oder dem Befehl \Macro{taskform} aus dem Paket 
\Package{tudscrsupervisor} folgendermaßen dargestellt werden.
\includeexample{task}
\section{Gutachten}
\label{sec:exmpl:evaluation}
\index{Minimalbeispiel!Gutachten}
Nach dem Laden des Paketes \Package{tudscrsupervisor} kann ein Gutachten für 
eine wissenschaftliche Arbeit mit der \Environment{evaluation}"=Umgebung oder 
dem Befehl \Macro{evaluationform} erstellt werden.
\includeexample{evaluation}
\section{Aushang}
\label{sec:exmpl:notice}
\index{Minimalbeispiel!Aushang}
Das Paket \Package{tudscrsupervisor} stellt die Umgebung \Environment{notice} 
für das Anfertigen allgemeiner Aushänge sowie den Befehl \Macro*{noticeform} 
für die Ausschreibung wissenschaftlicher Arbeiten bereit.
\includeexample{notice}
\index{Minimalbeispiel|)}
\index{Tutorial|(}
\section{Vorlage für eine wissenschaftlichen Arbeit}
\label{sec:exmpl:treatise}
\index{Tutorial!Abschlussarbeit}
Die meisten Anwender der \TUDScript-Klassen sind Studenten oder angehörige der 
\TnUD, die ihre ersten Schritte mit \hologo{LaTeXe} beim Verfassen einer 
wissenschaftlichen Arbeit oder ähnlichem machen. Während der Einstiegsphase in 
\hologo{LaTeXe} kann ein Anfänger sehr schnell aufgrund der großen Anzahl an 
empfohlenen Pakete sowie der teilweise diametral zueinander stehenden Hinweise 
überfordert sein. Mit dem Tutorial \Tutorial{treatise} soll versucht werden, 
ein wenig Licht ins Dunkel zu bringen. Es erhebt jedoch keinerlei Anspruch, 
vollständig oder perfekt zu sein. Einige der darin vorgestellten Möglichkeiten 
lassen sich mit Sicherheit auch anders, einfacher und/oder besser lösen. 
Dennoch ist es gerade für Neulinge~-- vielleicht auch für den einen oder 
anderen \hologo{LaTeX}"=Veteran~-- als Leitfaden für die Erstellung einer 
wissenschaftlichen Arbeit gedacht.
\ToDo[imp]{Tutorial für eine wissenschaftlichen Arbeit}[v2.02]
\ToDo[imp]{Einarbeiten von paragraph in treatise}[v2.02]
\section{Ein Beitrag zum mathematischen Satz in \NoCaseChange{\hologo{LaTeXe}}}
\label{sec:exmpl:mathtype}
\index{Tutorial!Mathematiksatz}
Das Tutorial \Tutorial{mathtype} richtet sich an alle Anwender, die in ihrem 
\hologo{LaTeX}"=Dokument mathematische Formeln setzen wollen. In diesem wird 
ausführlich darauf eingegangen, wie mit wenigen Handgriffen ein typographisch 
sauberer Mathematiksatz zu bewerkstelligen ist.
\section{Änderung der Trennzeichen im Mathematikmodus}
\label{sec:exmpl:mathswap}
\index{Tutorial!Trennzeichen Mathematikmodus}
Sollen beim Verfassen eines \hologo{LaTeX}"=Dokumentes Daten in einem 
Zahlenformat importiert werden, welches nicht den Gepflogenheiten der 
Dokumentsprache entspricht, kommt es meist zu unschönen Ergebnissen bei der 
Ausgabe. Einfachstes Beispiel sind Daten, in denen als Dezimaltrennzeichen ein 
Punkt verwendet wird, wie es im englischsprachigen Raum der Fall ist. Sollen 
diese in einem Dokument deutscher Sprache eingebunden werden, müssten diese 
normalerweise allesamt angepasst und das ursprüngliche Dezimaltrennzeichen 
durch ein Komma ersetzt werden. Dieser Schritt wird mit dem \TUDScript-Paket 
\Package{mathswap} automatisiert. Wie dies genau funktioniert, wird im Tutorial 
\Tutorial{mathswap} erläutert.
\index{Tutorial|)}
%\chapter{Weiterführende Installationshinweise}
\label{sec:install:ext}
%
Bis zur Version~v2.01 wurde \TUDScript ausschließlich über das \Forum zur 
lokalen Nutzerinstallation angeboten. In erster Linie hat das historische 
Hintergründe und hängt mit der Entstehungsgeschichte von \TUDScript zusammen. 

Eine lokale Nutzerinstallation bietet mehr oder weniger genau einen Vorteil. 
Treten bei der Verwendung von \TUDScript Probleme auf, können diese im Forum 
gemeldet und diskutiert werden. Ist für ein solches Problem tatsächlich eine 
Fehlerkorrektur respektive eine Aktualisierung von \TUDScript nötig, kann 
diese schnell und unkompliziert über das 
\hrfn{https://github.com/tud-cd/tudscr}{GitHub-Repository \Package*{tudscr}} 
bereitgestellt und durch den Anwender sofort genutzt werden.

Dies hat allerdings für Anwender, welche das Forum relativ wenig oder gar 
nicht besuchen, den großen Nachteil, dass diese nicht von Aktualisierungen, 
Verbesserungen und Fehlerkorrekturen neuer Versionen profitieren können. Auch 
sämtliche nachfolgenden Bugfixes und Aktualisierungen des \TUDScript-Bundles 
müssen durch den Anwender manuell durchgeführt werden. Daher wird in Zukunft 
die Verbreitung via \hrfn{http://www.ctan.org/pkg/tudscr}{CTAN} präferiert, so 
dass \TUDScript stets in der aktuellen Version verfügbar ist~-- eine durch den 
Anwender aktuell gehaltene \hologo{LaTeX}"=Distribution vorausgesetzt. Der 
einzige Nachteil bei diesem Ansatz ist, dass die Verbreitung eines Bugfixes 
über das \hrfn{http://www.ctan.org/}{Comprehensive TeX Archive Network (CTAN)} 
und die anschließende Bereitstellung durch die verwendete Distribution für 
gewöhnlich mehrere Tage dauert.

Die gängigen \hologo{LaTeX}"=Distributionen durchsuchen im Regelfall zuerst das 
lokale \Path{texmf}"=Nutzerverzeichnis nach Klassen und Paketen und erst daran 
anschließend den \Path{texmf}"=Pfad der Distribution selbst. Dabei spielt es 
keine Rolle, in welchem Pfad die neuere Version einer Klasse oder eines Paketes 
liegt. Sobald im Nutzerverzeichnis die gesuchte Datei gefunden wurde, wird die 
Suche beendet.
\Attention{%
  In der Konsequenz bedeutet dies, dass sämtliche Aktualisierungen über CTAN 
  nicht zum Tragen kommen, falls \TUDScript als lokale Nutzerversion  
  installiert wurde.
}

Deshalb wird Anwendern, die \TUDScript in der Version~v2.01 oder älter nutzen 
und sich nicht \emph{bewusst} für eine lokale Nutzerinstallation entschieden 
haben, empfohlen, diese zu deinstallieren. Der Prozess der Deinstallation wird 
in \autoref{sec:local:uninstall} erläutert. Wird diese einmalig durchgeführt, 
können Updates des \TUDScript-Bundles durch die Aktualisierungsfunktion der 
Distribution erfolgen. Wie das \TUDScript-Bundle trotzdem als lokale 
Nutzerversion installiert oder aktualisiert werden kann, ist in 
\autoref{sec:local:install} beziehungsweise \autoref{sec:local:update} zu 
finden. Der Anwender sollte in diesem Fall allerdings genau wissen, was er 
damit bezweckt, da er in diesem Fall für die Aktualisierung von \TUDScript 
selbst verantwortlich ist.



\section{Lokale Deinstallation von \TUDScript}
\label{sec:local:uninstall}
\index{Deinstallation}
%
Um die lokale Nutzerinstallation zu entfernen, kann für Windows
\hrfn{https://github.com/tud-cd/tudscr/releases/download/uninstall/tudscr\_uninstall.bat}%
{\File{tudscr\_uninstall.bat}} sowie für unixartige Betriebssysteme
\hrfn{https://github.com/tud-cd/tudscr/releases/download/uninstall/tudscr\_uninstall.sh}%
{\File{tudscr\_uninstall.sh}} verwendet werden. Nach der Ausführung des 
jeweiligen Skriptes kann in der Konsole beziehungsweise im Terminal mit
%
\begin{quoting}
\Path{kpsewhich --all tudscrbase.sty}
\end{quoting}
%
überprüft werden, ob die Deinstallation erfolgreich war oder immer noch eine 
lokale Nutzerinstallation vorhanden ist. Es werden alle Pfade ausgegeben, in 
welchen die Datei \File*{tudscrbase.sty} gefunden wird. Sollte nur noch der 
Pfad der Distribution erscheinen, ist ab sofort die \TUDScript-Version von CTAN 
aktiv und der Anwender kann mit dem \TUDScript-Bundle arbeiten. Falls es nicht 
schon passiert ist, müssen dafür lediglich die Schriften des \CDs installiert 
werden (\autoref{sec:install}).

Wird \emph{nur} das lokale Nutzerverzeichnis oder gar kein Verzeichnis 
gefunden, so wird höchstwahrscheinlich eine veraltete Distribution 
verwendet. In diesem Fall \emph{muss} das \TUDScript-Bundle lokal aktualisiert 
(\autoref{sec:local:update}) beziehungsweise bei der erstmaligen Verwendung 
lokal installiert (\autoref{sec:local:install}) werden. Sollte neben dem 
Pfad der Distribution immer noch mindestens ein weiterer Pfad angezeigt werden,
so ist weiterhin eine lokale Nutzerversion installiert. In diesem Fall hat der 
Anwender zwei Möglichkeiten:
%
\begin{enumerate}
\item Entfernen der lokalen Nutzerinstallation (manuell)
\item Update der lokalen Nutzerversion
\end{enumerate}
%
Die erste Variante wird nachfolgend erläutert, die zweite Möglichkeit wird in 
\autoref{sec:local:update} beschrieben. Nur die manuelle Deinstallation der 
lokalen Nutzerversion \TUDScript ermöglicht dabei die Verwendung der jeweils 
aktuellen CTAN"=Version. Hierfür ist etwas Handarbeit vonnöten. Der in der 
Konsole beziehungsweise im Terminal mit \Path{kpsewhich --all tudscrbase.sty} 
gefundene~-- zum Ordner der Distribution \emph{zusätzliche}~-- Pfad hat die 
folgende Struktur:
%
\begin{quoting}
\Path{\emph{<Installationspfad>}/tex/latex/tudscr/tudscrbase.sty}
\end{quoting}
%
Um die Nutzerinstallation vollständig zu entfernen, muss als erstes zu 
\Path{\emph{<Installationspfad>}} navigiert werden. Anschließend ist in diesem 
Pfad Folgendes durchzuführen:
%
\settowidth{\tempdim}{\Path{tex/latex/tudscr/}}%
\begin{description}[labelwidth=\tempdim,labelsep=1em]
\item[\Path{tex/latex/tudscr/}]alle .cls- und .sty-Dateien löschen
\item[\Path{tex/latex/tudscr/}]Ordner \Path{logo} vollständig löschen
\item[\Path{doc/latex/}] Ordner \Path{tudscr} vollständig löschen
\item[\Path{source/latex/}] Ordner \Path{tudscr} vollständig löschen
\end{description}
%
Das Verzeichnis \Path{\emph{<Installationspfad>}/tex/latex/tudscr/fonts} 
\textbf{sollte erhalten bleiben}. Andernfalls müssen die Schriften des \CDs 
abermals wie unter \autoref{sec:install} beschrieben installiert werden.
Zum Abschluss ist in der Kommandozeile beziehungsweise im Terminal der Befehl 
\Path{texhash} aufzurufen. Damit wurde die lokale Version entfernt und es wird 
von nun an die Version von \TUDScript genutzt, welche durch die verwendete 
Distribution bereitgestellt wird.

\section{Lokale Installation des \TUDScript-Bundles}
\label{sec:local:install}
\index{Installation!Nutzerinstallation}\index{Nutzerinstallation (lokal)}
\subsection{Lokale Installation von \TUDScript unter Windows}
\index{Installation!Nutzerinstallation}\index{Nutzerinstallation (lokal)}
Für eine lokale Installation sowohl des \TUDScript-Bundles als auch der 
dazugehörigen Schriften für die Distributionen \Distribution{\hologo{TeX}~Live} 
oder \Distribution{\hologo{MiKTeX}} werden neben Schriftarchiven die Dateien aus
\hrfn{https://github.com/tud-cd/tudscr/releases/download/\vTUDScript/TUD-KOMA-Script\_\vTUDScript\_Windows\_full.zip}%
{\File*{TUD-KOMA-Script\_\vTUDScript\_Windows\_full.zip}} benötigt. Vor der 
Verwendung des Skripts \File{tudscr\_\vTUDScript\_install.bat} sollte 
sichergestellt werden, dass sich \emph{alle} der folgenden Dateien im selben 
Verzeichnis befinden:
%
\settowidth{\tempdim}{\File{tudscr\_\vTUDScript\_install.bat}}%
\begin{description}[labelwidth=\tempdim,labelsep=1em]
  \item[\File{tudscr\_\vTUDScript.zip}]Archiv mit Klassen- und Paketdateien
  \item[\File{tudscr\_\vTUDScript\_install.bat}]Installationsskript
  \item[\File{Univers\_PS.zip}]Archiv mit Schriftdateien für \Univers
  \item[\File{DIN\_Bd\_PS.zip}]Archiv mit Schriftdateien für \DIN
  \item[\File{tudscrfonts.zip}]Archiv mit Metriken für die
    Schriftinstallation via \Package{fontinst}
  \item[\File{7za.exe}]Stand-Alone-Version von 7-zip zum Entpacken der Archive%
    \footnote{%
      Windows stellt keine Bordmittel zum Extrahieren von Archiven auf 
      Kommandozeilen-/Skript-Ebene zur Verfügung.%
    }%
\end{description}
%
Beim Ausführen des Installationsskripts werden alle Schriften in das lokale 
Nutzerverzeichnis der jeweiligen Distribution installiert, falls kein anderes 
Verzeichnis explizit angegeben wird. Für Hinweise bei Problemen mit der 
Schriftinstallation sei auf \autoref{sec:install:fonts:win} verwiesen.



\subsection{Lokale Installation von \TUDScript unter Linux und OS~X}
\index{Installation!Nutzerinstallation}\index{Nutzerinstallation (lokal)}
Für eine lokale Installation sowohl des \TUDScript-Bundles als auch der 
dazugehörigen Schriften für die Distributionen \Distribution{\hologo{TeX}~Live} 
oder \Distribution{Mac\hologo{TeX}} werden neben Schriftarchiven die Dateien aus
\hrfn{https://github.com/tud-cd/tudscr/releases/download/\vTUDScript/TUD-KOMA-Script\_\vTUDScript\_Unix\_full.zip}%
{\File*{TUD-KOMA-Script\_\vTUDScript\_Unix\_full.zip}} benötigt. Vor der 
Verwendung des Skripts \File{tudscr\_\vTUDScript\_install.sh} sollte 
sichergestellt werden, dass sich \emph{alle} der folgenden Dateien im selben 
Verzeichnis befinden:
%
\begin{description}[labelwidth=\tempdim,labelsep=1em]
\settowidth{\tempdim}{\File{tudscr\_\vTUDScript\_install.sh}}%
  \item[\File{tudscr\_\vTUDScript.zip}]Archiv mit Klassen- und Paketdateien
  \item[\File{tudscr\_\vTUDScript\_install.sh}]Installationsskript
  \item[\File{Univers\_PS.zip}]Archiv mit Schriftdateien für \Univers
  \item[\File{DIN\_Bd\_PS.zip}]Archiv mit Schriftdateien für \DIN
  \item[\File{tudscrfonts.zip}]Archiv mit Metriken für die
    Schriftinstallation via \Package{fontinst}
\end{description}
%
Beim Ausführen des Installationsskripts werden alle Schriften in das lokale 
Nutzerverzeichnis der jeweiligen Distribution installiert. Für Hinweise bei 
Problemen mit der Schriftinstallation sei auf \autoref{sec:install:fonts:unix} 
verwiesen.



\section{Lokales Update des \TUDScript-Bundles}
\label{sec:local:update}
\index{Update!Nutzerinstallation}\index{Nutzerinstallation (lokal)}
\subsection{Update des \TUDScript-Bundles ab Version~\NoCaseChange{v}2.00}
Mit der Version~v2.02 gab es unter anderem auch einige Änderungen an den 
verwendeten Logos. Deshalb wird für diese Version kein dediziertes Update 
angeboten. Es können entweder, wie in \autoref{sec:local:install} erläutert, 
lokal via Skript neu installiert werden oder das Update erfolgt manuell. 
Hierfür muss der Inhalt des Archivs
\hrfn{https://github.com/tud-cd/tudscr/releases/download/v2.02/tudscr\_v2.02.zip}%
{\File{tudscr\_v2.02.zip}} in das lokale \Path{texmf}"=Nutzerverzeichnis 
kopiert werden.
%Für eine lokale Aktualisierung von \TUDScript auf \vTUDScript{} muss das Archiv
%\hrfn{https://github.com/tud-cd/tudscr/releases/download/\vTUDScript/TUD-KOMA-Script\_\vTUDScript\_Windows\_update.zip}%
%{TUD-KOMA-Script\_\vTUDScript\_Windows\_update.zip} respektive 
%\hrfn{https://github.com/tud-cd/tudscr/releases/download/\vTUDScript/TUD-KOMA-Script\_\vTUDScript\_Unix\_update.zip}%
%{TUD-KOMA-Script\_\vTUDScript\_Unix\_update.zip} entpackt und anschließend
%\File{tudscr\_\vTUDScript\_update.bat} oder 
%\File{tudscr\_\vTUDScript\_update.sh} ausgeführt werden.
%\Attention{%
%  Das lokale Update funktioniert nur, wenn bereits mindestens Version~v2.02 
%  entweder lokal oder über die Distribution installiert ist.%
%}



\DeclareClass{tudscrbookold}%
\DeclareClass{tudscrreprtold}%
\DeclareClass{tudscrartclold}%
\subsection{Update des \TUDScript-Bundles von Version \NoCaseChange{v}1.0}
\index{Update!Version v1.0}%
\index{Hauptklassen}
\index{Version!v1.0}%
Ist \TUDScript in der Version~v1.0 installiert, so wird dringend zu einer 
Deinstallation dieser geraten. Geschieht dies nicht, wird es zu Problemen 
kommen. Dafür werden die Skripte 
\hrfn{https://github.com/tud-cd/tudscr/releases/download/uninstall/tudscr\_uninstall.bat}{\File{tudscr\_uninstall.bat}}
beziehungsweise
\hrfn{https://github.com/tud-cd/tudscr/releases/download/uninstall/tudscr_uninstall.sh}{\File{tudscr\_uninstall.sh}}
bereitgestellt. Die aktuelle Version~\vTUDScript{} kann nach der Deinstallation 
der Version~v1.0 wie unter \autoref{sec:install} beschrieben installiert werden.

Sollen die obsoleten \TUDScript-Klassen in der Version~v1.0 nach einer 
Aktualisierung weiterhin genutzt werden, so müssen diese erst wie zuvor 
beschrieben de"~~und anschließend neu installiert werden. Dafür kann das Archiv 
\hrfn{https://github.com/tud-cd/tudscrold/releases/download/v1.0/TUD-KOMA-Script_v1.0old.zip}%
{\File*{TUD-KOMA-Script\_v1.0old.zip}} verwendet werden, welches sowohl die 
genannten Skripte zur Deinstallation als auch die zur neuerlichen Installation 
der veralteten Klassen benötigten \File{tudscr\_v1.0old\_install.bat} oder 
\File{tudscr\_v1.0old\_install.sh} enthält. Nach Abschluss des Vorgangs sind 
die alten Klassen der Version~v1.0 mit \Class*{tudscrbookold}, 
\Class*{tudscrreprtold} und \Class*{tudscrartclold} verwendbar und können 
parallel zur aktuellen Version~\vTUDScript{} genutzt werden.

Im Vergleich zur Version~v1.0 hat sich an der Benutzerschnittstelle nicht sehr 
viel verändert. Treten nach dem Umstieg von der Version~v1.0 auf die 
Version~\vTUDScript{} dennoch Probleme auf, sollte der Anwender als erstes in 
\autoref{sec:comp} sehen. Hier werden die gemachten Änderungen erläutert und im 
alten Dokument gegebenenfalls notwendige Anpassungen beschrieben. Sollten 
dennoch Fehler oder Probleme beim Umstieg auf die neue \TUDScript-Version 
auftreten, ist eine Meldung im Forum die beste Möglichkeit, um Hilfe zu 
erhalten.
%\Index{Optionen}{options}%
\Index{Befehle}[Befehle etc.]{macros}%
\Index{Befehle!Parameter}[Parameter]{keys}%
\Index{Makro}[Befehle etc.]{macros}%
\Index{Umgebungen}[Befehle etc.]{macros}%
\Index{Umgebungen!Parameter}[Parameter]{keys}%
\Index{Parameter}{keys}%
\Index{Bezeichner}{terms}%
\Index{Schriftelemente}{fonts}%
\Index{Farben}{colors}%
\Index{Klassen}[Dateien etc.]{files}%
\Index{Pakete}[Dateien etc.]{files}%
\Index{Dateien}[Dateien etc.]{files}%
\Changelog{Änderungen}
\Changelog{Changelog}
\Changelog{Version}
\index{Abbildungen|see{Grafiken}}%
\index{Aktualisierung|see{Update}}%
\index{Aufzählungen|see{Listen}}%
\index{Cover|see{Umschlagseite}}%
\index{Dezimaltrennzeichen|see{Trennzeichen}}%
\index{Distribution!\hologo{TeX}~Live|see{\hologo{TeX}~Live}}
\index{Distribution!\hologo{MiKTeX}|see{\hologo{MiKTeX}}}
\index{Fachreferent|see{Referent}}%
\index{Farbraum|see{Farben!Farbmodell}}%
\index{Grafiken!Beschriftung|see{Gleitobjekte}}
\index{Großbuchstaben|see{Schriftauszeichnung}}%
\index{Klassenoptionen|see{Optionen}}%
\index{Kleinbuchstaben|see{Schriftauszeichnung}}%
\index{Kurzfassung|see{Zusammenfassung}}%
\index{Lokalisierung|see{Bezeichner}}%
\index{Majuskeln|see{Schriftauszeichnung}}%
\index{Mathematiksatz!Einheiten|see{Einheiten}}
\index{Mathematiksatz!Trennzeichen|see{Trennzeichen}}
\index{Minuskeln|see{Schriftauszeichnung}}%
\index{Outline-Eintrag|see{Lesezeichen}}%
\index{Professor|see{Hochschullehrer}}%
\index{Seitenränder|see{Satzspiegel}}
\index{Silbentrennung|see{Worttrennung}}%
\index{Sprachunterstützung!Lokalisierung|see{Bezeichner}}%
\index{Sprachunterstützung!Worttrennung|see{Worttrennung}}%
\index{Sprungmarken|see{Lesezeichen}}
\index{Tabellen!Beschriftung|see{Gleitobjekte}}
\index{Tausendertrennzeichen|see{Trennzeichen}}%
\index{Trennmuster|see{Worttrennung}}%
\index{Vakatseiten|see{Leerseiten}}%
\index{Vektorgrafiken|see{Grafiken}}%
\setchapterpreamble{%
  \begin{abstract}
    \noindent Die Formatierung der Einträge in allen aufgeführten Indizes ist 
    folgendermaßen aufzufassen: \textbf{Zahlen in fetter Schrift} verweisen auf 
    die \textbf{Erklärung} zu einem Stichwort, wobei in der digitalen Fassung 
    dieses Handbuchs dieser Eintrag selbst ein Hyperlink zu seiner Erläuterung 
    ist. Seitenzahlen in normaler Schriftstärke hingegen deuten auf zusätzliche 
    Informationen, wobei diese für \textit{kursiv hervorgehobene Zahlen} als 
    besonders \textit{wichtig} erachtet werden.
    
    Bei Einträgen für \hyperref[idx:options]{Klassen- und Paketoptionen} 
    beziehungsweise für \hyperref[idx:macros]{Umgebungen und Befehle}, zu denen 
    keine direkte \textbf{Erklärung} gegeben ist sondern lediglich zusätzliche 
    Hinweise vorhanden sind, handelt es sich um \KOMAScript"=Optionen. Diese
    sind gegebenenfalls im dazugehörigen Handbuch nachzulesen 
    (\File{scrguide.pdf}).
  \end{abstract}
}
\addchap*{\indexname}
\addxcontentsline{toc}{part}{\indexname}
\PrintIndexPrologue{%
  Die aufgelisteten Schlagworte sollen sowohl Antworten bei generellen Fragen 
  als auch Lösungen für typische Probleme beim Umgang mit \hologo{LaTeX} sowie 
  dem \TUDScript-Bundle liefern. 
%  Falls ein gesuchter Begriff hier nicht auftaucht, ist das Forum erster 
%  Anlaufpunkt.
}
\PrintIndex
\PrintChangelog
\end{document}