\documentclass[ngerman]{tudscrreprt}
\usepackage{selinput}
\SelectInputMappings{adieresis={ä},germandbls={ß}}
\usepackage[T1]{fontenc}
\usepackage{babel}
\usepackage{mathswap}
\begin{document}
\pagestyle{empty}\noindent
Es ist eine Zahl im deutschen Layout gegeben (4.523,58). Die Ausgabe im 
Mathematikmodus:
\[4.523,58\]
Sollte die gleiche Zahl in englischer Formatierung gegeben sein (4,523.58), 
funktioniert das nicht mehr so gut:
\[4,523.58\]
Mit der Verwendung von \verb|\commaswap{\,}| und \verb|\dotswap{,}| können die 
Substitutionen für Komma und Punkt im Mathematikmodus geändert und damit die 
Ausgabe korrigiert werden:
\begingroup
\commaswap{\,}
\dotswap{,}
\[4,523.58\]
\endgroup
\end{document}