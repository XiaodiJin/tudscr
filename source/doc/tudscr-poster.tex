\chapter[Die Klasse tudscrposter]{Die Posterklasse}
\tudhyperdef*{sec:poster}%
%
\begin{Bundle*}[v2.05]{\Class{tudscrposter}}
\index{Posterklasse|!}%
\printchangedatlist%
%
Ergänzend zu den Hauptklassen, welche für das Setzen von Dokumenten im \TUDCD 
angeboten werden, wird die Klasse \Class{tudscrposter} bereitgestellt. Mit 
dieser wird das Erstellen von Postern mit \hologo{LaTeX} ermöglicht. Die Basis 
hierfür ist \Class{tudscrartcl} und alle durch diese Klasse angebotenen Befehle 
und Optionen können mit \Class{tudscrposter} gleichermaßen verwendet werden. 

\section{Layout und Formatierung eines Posters}
\index{Papierformat|!}%
\index{Schriftgröße|!}%
%
Der größte Unterschied zu den Hauptklassen ist offensichtlich der vereinfachte 
Titel. Der Befehl \Macro{makecover} für eine Umschlagseite steht nicht zur 
Verfügung, der Titel selbst kann mit \Macro{maketitle} nur als Titelkopf 
gesetzt werden, eine separat Titelseite existiert nicht. Ebenso ist eine 
Vielzahl der Befehle für die Titelseite aus \autoref{sec:title} logischer Weise 
nicht verfügbar. Die wichtigsten Optionen für die Gestaltung sind sicherlich 
\Option{cd} für das Layout sowie \Option{cdfont} für die Schriftauswahl. Diese 
können wie auch bei den Hauptklassen verwendet werden. 

Des Weiteren ist insbesondere die Festlegung von \emph{Papierformat} und 
\emph{Schriftgröße} essentiell für das Erstellen eines Posters. 
\Attention{Beide Einstellungen sollten zwingend als Klassenoption erfolgen.}
Dabei ist darauf zu achten, ob der Satz des Posters mehrspaltig erfolgen soll. 
Für dieses Unterfangen ist die \Environment{multicols}-Umgebung aus dem Paket 
\Package{multicol} sehr empfehlenswert. Für eine passende und gut abgestimmte 
Auswahl soll \autoref{tab:font+paper} als Referenz dienen.

Für die Anpassung respektive die Einstellung des Papierformats ist die 
\KOMAScript-Option \Option{paper=\PName{Format}}|declare| zu verwenden. Dabei 
lassen sich unter anderem die gängigen Klassen der ISO/DIN-Reihen A bis D als 
auch Quer- oder Längsformat auswählen. Zusätzlich kann mit \PValue{Höhe:Breite} 
ein beliebig freies Format eingestellt werden. Für genauere Hinweise ist das 
\scrguide zu Rate zu ziehen. Anschließend sollte unbedingt die Schriftgröße 
mit \Option{fontsize=\PName{Schriftgröße}}|declare| angegeben werden. 

\begin{table}
  \index{Papierformat|!}%
  \index{Schriftgröße|!}%
  \newcommand*\mtm{\small min\dots{}max}%
  \newcommand*\rng[2]{\small #1\dots{}#2pt}%
  \ttabbox[\linewidth]{%
    \setlength\tabcolsep{5pt}%
    \centering%
    \begin{subfloatrow}%
      \ttabbox{%
\begin{tabular}{r*{7}c}
  \toprule
        &\multicolumn{7}{c}{Klasse}\tabularnewline\midrule
        & 6    & 5    & 4    & 3    & 2    & 1    & 0    \tabularnewline\midrule
  Reihe & \mtm & \mtm & \mtm & \mtm & \mtm & \mtm & \mtm \tabularnewline\midrule
D&\rng{05}{07}&\rng{06}{09}&\rng{10}{14}&\rng{14}{20}&\rng{20}{29}&\rng{28}{40}&\rng{40}{60}\tabularnewline\midrule
A&\rng{06}{08}&\rng{07}{10}&\rng{11}{16}&\rng{16}{23}&\rng{23}{33}&\rng{32}{46}&\rng{45}{66}\tabularnewline\midrule
C&\rng{07}{09}&\rng{08}{11}&\rng{12}{18}&\rng{18}{26}&\rng{26}{37}&\rng{36}{52}&\rng{50}{72}\tabularnewline\midrule
B&\rng{08}{10}&\rng{09}{12}&\rng{13}{20}&\rng{20}{29}&\rng{29}{41}&\rng{40}{58}&\rng{55}{78}\tabularnewline\bottomrule
\end{tabular}
      }{\caption{Einspaltiges Layout}}%
    \end{subfloatrow}%
    \par\medskip
    \begin{subfloatrow}%
      \ttabbox{%
\begin{tabular}{r*{4}c}
  \toprule
        &\multicolumn{4}{c}{Klasse}\tabularnewline\midrule
        & 3    & 2    & 1    & 0    \tabularnewline\midrule
  Reihe & \mtm & \mtm & \mtm & \mtm \tabularnewline\midrule
D&\rng{07}{10}&\rng{10}{14}&\rng{14}{19}&\rng{19}{28}\tabularnewline\midrule
A&\rng{08}{11}&\rng{11}{16}&\rng{16}{22}&\rng{22}{32}\tabularnewline\midrule
C&\rng{09}{12}&\rng{12}{18}&\rng{18}{25}&\rng{25}{36}\tabularnewline\midrule
B&\rng{10}{13}&\rng{13}{20}&\rng{20}{28}&\rng{28}{40}\tabularnewline\bottomrule
\end{tabular}
      }{\caption{Zweispaltiges Layout}}%
      \ttabbox{%
\begin{tabular}{r*{2}c}
  \toprule
        &\multicolumn{2}{c}{Klasse}\tabularnewline\midrule
        & 1    & 0    \tabularnewline\midrule
  Reihe & \mtm & \mtm \tabularnewline\midrule
D&\rng{07}{10}&\rng{10}{14}\tabularnewline\midrule
A&\rng{08}{11}&\rng{11}{16}\tabularnewline\midrule
C&\rng{09}{12}&\rng{12}{18}\tabularnewline\midrule
B&\rng{10}{13}&\rng{13}{20}\tabularnewline\bottomrule
\end{tabular}
      }{\caption{Dreispaltiges Layout}}%
    \end{subfloatrow}%
  }{%
    \caption{%
      Empfohlene Kombinationen von Papierformat (\Option{paper}) und 
      Schriftgröße (\Option{fontsize})%
    }\label{tab:font+paper}%
  }%
\end{table}


\section{Angepasste und neue Befehle}
%
Für die Klasse \Class{tudscrposter} sind für die Gestaltung des Fußbereichs 
einige Befehle neu definiert oder bereits existierende Makros aus den 
\TUDScript-Hauptklassen angepasst respektive erweitert worden.


\ToDo[doc]{Doku für \Class{tudscrposter}}[v2.05]

\begin{Declaration}{%
  \Macro{faculty}[\OParameter{Fußzeile}\Parameter{Fakultät}]%
}
\begin{Declaration}{%
  \Macro{department}[\OParameter{Fußzeile}\Parameter{Einrichtung}]%
}
\begin{Declaration}{%
  \Macro{institute}[\OParameter{Fußzeile}\Parameter{Institut}]%
}
\begin{Declaration}{%
  \Macro{chair}[\OParameter{Fußzeile}\Parameter{Lehrstuhl}]%
}
\printdeclarationlist%
\end{Declaration}
\end{Declaration}
\end{Declaration}
\end{Declaration}


\begin{Declaration}{%
  \Macro{professor}[\OParameter{Fußzeile}\Parameter{Name}]%
}
\printdeclarationlist%
\end{Declaration}

\begin{Declaration}{\Macro{webpage}[\OParameter{Kram}\Parameter{URL}]}
\printdeclarationlist%
\end{Declaration}

\begin{Declaration}{\Macro{contactperson}[\Parameter{Name(n)}]}
\begin{Declaration}{\Macro{office}[\Parameter{Adresse/Gebäude}]}
\begin{Declaration}{\Macro{telephone}[\Parameter{Telefonnummer}]}
\begin{Declaration}{\Macro{emailaddress}[\Parameter{E-Mail-Adresse}]}
\printdeclarationlist%
\end{Declaration}
\end{Declaration}
\end{Declaration}
\end{Declaration}

\begin{Declaration}{\Term{authorname}}
\begin{Declaration}{\Term{contactname}}
\begin{Declaration}{\Term{contactpersonname}}
\printdeclarationlist%
%
\TermTable{authorname,contactname,contactpersonname}
\end{Declaration}
\end{Declaration}
\end{Declaration}

\section{Beispiele für Poster}
%
Hinweis auf entsprechendes Kapitel oder direkt hier?
\ToDo[doc]{Beispiel für \Class*{tudscrposter}}[v2.05]

\section{Der Umstieg von \Class{tudmathposter}}
%
\ToDo[doc]{\Class*{tudscrposter} als Ersatz für \Class{tudmathposter}}[v2.05]

\end{Bundle*}
