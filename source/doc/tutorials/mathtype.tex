\RequirePackage[ngerman=ngerman-x-latest]{hyphsubst}
\documentclass[english,ngerman]{tudscrartcl}
\usepackage{selinput}
\SelectInputMappings{adieresis={ä},germandbls={ß}}
\usepackage[T1]{fontenc}
\usepackage{tudscrman}
\lstset{%
  inputencoding=utf8,extendedchars=true,
  literate=%
    {ä}{{\"a}}1 {ö}{{\"o}}1 {ü}{{\"u}}1
    {Ä}{{\"A}}1 {Ö}{{\"O}}1 {Ü}{{\"U}}1
    {~}{{\textasciitilde}}1 {ß}{{\ss}}1
}

\usepackage{amsmath}

\begin{document}
\title{Ein Beitrag zum mathematischen Satz in \NoCaseChange{\hologo{LaTeXe}}}
\author{Falk Hanisch}
\date{21.08.2014}
\makeatletter
\begingroup%
  \def\and{, }%
  \let\thanks\@gobble%
  \let\footnote\@gobble%
  \hypersetup{%
    pdfauthor = {\@author},%
    pdftitle = {\@title},%
    pdfsubject = {Mathematiksatz in \hologo{LaTeXe}},%
    pdfkeywords = {LaTeX, \TUDScript, Tutorial, Mathematiksatz},%
  }%
\endgroup%
\markright{\@title}
\makeatother
\StartTutorial
%
\begin{Tutorial*}
\RequirePackage[ngerman=ngerman-x-latest]{hyphsubst}
\documentclass[ngerman]{tudscrartcl}% andere Klassen sind auch möglich
\usepackage{selinput}\SelectInputMappings{adieresis={ä},germandbls={ß}}
\usepackage[T1]{fontenc}
\usepackage{babel}

\usepackage{amsmath}

\begin{document}
\end{Tutorial*}
%
Ein guter Mathematiksatz ist in \hologo{LaTeX} durchaus Sisyphusarbeit. Wenn 
mikrotypographisch alles richtig gemacht werden soll, gibt es einiges zu 
beachten. Generell gilt, dass Variablen kursiv, Bezeichnungen und Konstanten
aufrecht gesetzt werden.

Um beschreibende Indizes bei Formelzeichen richtig zu setzen, ist ohne weitere 
Maßnahmen die exzessive Nutzung der Befehle \Macro*{mathrm}\PParameter{\dots} 
und \Macro*{mathit}\PParameter{\dots} wohl oder übel notwendig. So wird aus:
%
\begin{Tutorial}
\begin{equation*}
\begin{gathered}
M_{EM} = \frac{M_{Rad}}{i_g \cdot i_A} - M_{VM} \\
\textrm{für }
\begin{aligned}
0\leq M_{VM}\leq M_{VMmax} \\ 
M_{EMmin}\leq M_{EM}\leq M_{EMmax}
\end{aligned}
\end{gathered}
\end{equation*}
\end{Tutorial}
%
mit ziemlich viel Anpassungsarbeit:
%
\begin{Tutorial}
\begin{equation*}
\begin{gathered}
M_\mathrm{EM} = \frac{M_\mathrm{Rad}}{i_g \cdot i_A} - M_\mathrm{VM} \\
\textrm{für }
\begin{aligned}
0\leq M_\mathrm{VM}\leq M_\mathrm{VM_{max}} \\ 
M_\mathrm{EM_{min}}\leq M_\mathrm{EM}\leq M_\mathrm{EM_{max}}
\end{aligned}
\end{gathered}
\end{equation*}
\end{Tutorial}
%
Wie man sieht, ist dabei eine ganze Menge Handarbeit vonnöten. Allerdings lässt 
sich das relativ gut vereinfachen lassen. Zu diesem Zwecke wird ein Befehl 
\Macro*{ind}\PParameter{\dots} für das Setzen von Indizes bei Formelzeichen 
definiert. Danach kann man sich~-- wenn man das für nötig und sinnvoll 
erachtet~-- noch zusätzliche Befehle für häufig verwendete Ausdrücke schnitzen. 
Als Beispiel wird das schon eben genutzte Drehmoment \ensuremath{M} verwendet. 
Hierfür könnte man Folgendes definieren:
%
\TutorialHook{\let\newcommand\renewcommand}
\begin{Tutorial}
\newcommand*{\ind}[1]{\ensuremath{_\mathrm{#1}}}
\newcommand*{\M}[1]{\ensuremath{M\ind{#1}}}
\end{Tutorial}
%
und damit diese Ausgabe erzeugen:
%
\begin{Tutorial}
\begin{equation*}
\begin{gathered}
\M{EM} = \frac{\M{Rad}}{i_g \cdot i_A} - \M{VM} \\
\textrm{für }
\begin{aligned}
0\leq \M{VM}\leq \M{VM_{max}} \\ 
\M{EM_{min}}\leq \M{EM}\leq \M{EM_{max}}
\end{aligned}
\end{gathered}
\end{equation*}
\end{Tutorial}
%
Netter Nebeneffekt ist, dass man den Befehl aufgrund der Verwendung von 
\Macro*{ensuremath}\PParameter{\dots} nun auch im Fließtext verwenden kann, 
beispielsweise wie hier \M{VM_{ind}} (\Macro*{M}\PParameter{VM\_\{ind\}}) für 
das induzierte Moment einer Verbrennungskraftmaschine.

Möchte man es sich noch bequemer machen, strikt man sich noch eine Lösung, in
der man~-- im Gegensatz zum \hologo{LaTeX}"=Standardfall~-- \textbf{nach} dem
obligatorischen Argument noch ein optionales für einen weiteren Index angeben
kann, um damit der natürlichen Schreibweise zu entsprechen. Es wird der Befehl 
\Macro*{M} so definiert, das dieser entweder mit \Macro*{M}\Parameter{Index} 
oder in der Variante \Macro*{M}\Parameter{Index}\OParameter{Indexindex} 
genutzt 
werden kann.
%
\TutorialHook{\let\newcommand\renewcommand}
\begin{Tutorial}
\renewcommand*{\ind}[1]{\ensuremath{_\mathrm{#1}}}
\makeatletter
\renewcommand*{\M}[1]{\@ifnextchar[{\o@M{#1}}{\n@M{#1}}}
\newcommand*{\n@M}{}
\newcommand*{\o@M}{}
\def\n@M#1{\ensuremath{M\ind{#1}}}
\def\o@M#1[#2]{\ensuremath{M\ind{#1_{#2}}}}
\makeatother
\end{Tutorial}
%
Somit vereinfacht sich das zu Beginn vorgestellte Beispiel recht deutlich:
%
\begin{Tutorial}
\begin{equation*}
\begin{gathered}
\M{EM} = \frac{\M{Rad}}{i_g \cdot i\_A} - \M{VM} \\
\textrm{für }
\begin{aligned}
0\leq \M{VM}\leq \M{VM}[max] \\ 
\M{EM}[min]\leq \M{EM}\leq \M{EM}[max]
\end{aligned}
\end{gathered}
\end{equation*}
\end{Tutorial}
%
Das Definieren von \Macro*{M}\Parameter{Index}\OParameter{Indexindex} mit 
angehängtem optionalen Argument ist ehrlich gesagt nur ein wenig Spielerei und 
soll zeigen, wie dies prinzipiell mit \hologo{LaTeXe}"=Mitteln funktioniert. 
Das Paket \Package{xparse} könnte alternativ zum hier vorgestellten Vorgehen 
für die Befehlsdeklaration des optionalen \textbf{nach} dem obligatorischen 
Argument genutzt werden. Damit würde die Befehlsdeklaration für \Macro*{M} 
folgendermaßen lauten:
\begin{Tutorial-}
\NewDocumentCommand \M { m o } {%
  \ensuremath{%
    M\ind{%
      #1
      \IfNoValueTF{#2}{}{_{#2}}
    }%
  }%
}
\end{Tutorial-}
%
Das Ende des Dokumentes nicht vergessen.
%
\begin{Tutorial*}
\end{document}
\end{Tutorial*}
%
\FinishTutorial
\end{document}
