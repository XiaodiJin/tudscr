\RequirePackage[ngerman=ngerman-x-latest]{hyphsubst}
\documentclass[%
  english,ngerman,%
  geometry=no,DIV=12,automark,%
]{tudscrartcl}
\usepackage{selinput}\SelectInputMappings{adieresis={ä},germandbls={ß}}
\usepackage[T1]{fontenc}
\usepackage{lmodern}

\usepackage{tudscrman}
\lstset{%
  inputencoding=utf8,extendedchars=true,
  literate=%
    {ä}{{\"a}}1 {ö}{{\"o}}1 {ü}{{\"u}}1
    {Ä}{{\"A}}1 {Ö}{{\"O}}1 {Ü}{{\"U}}1
    {~}{{\textasciitilde}}1 {ß}{{\ss}}1
}
\lstset{escapechar=§}
\TUDoptions{cdfont=false}
\KOMAoptions{headings=normal}

\usepackage{tudscrsupervisor}
\usepackage{array}
\usepackage{tabu}
\usepackage{tabularx}
\usepackage{tabulary}
\usepackage{booktabs}
\usepackage{siunitx}
\sisetup{%
  locale = DE,%
  input-decimal-markers={,},%
  input-ignore={.},%
  group-separator={\,},%
  group-minimum-digits=3%
}
\usepackage{xfrac}
\AfterPackage*{hyperref}{%
  \usepackage[%
    automake,%
    xindy={language=german-din},%
    acronym,%
    symbols,%
    nomain,%
    translate=babel,%
    nogroupskip,%
  ]{glossaries}
  \setStyleFile{\jobname-temp}
  \renewcommand*{\glsglossarymark}[1]{}
  \newignoredglossary{abbreviation}
  \newcommand*\newabbreviation[4][]{%
    \newacronym[type=abbreviation,#1]{#2}{#3}{#4}%
  }%
  \makeglossaries
}
\usepackage{csquotes}
\usepackage[backend=biber,style=alphabetic]{biblatex}
\addbibresource{\jobname.bib}
\DefineBibliographyStrings{ngerman}{urlseen = {Am:}}

\usepackage{enumitem}

\usepackage{caption}
\DeclareCaptionSubType[alph]{figure}
\DeclareCaptionSubType[alph]{table}
\usepackage{floatrow}
\renewcommand{\floatpagefraction}{0.7}

\ifpdf
  \usepackage{tikz}
  \usetikzlibrary{chains}
  \usetikzlibrary{decorations.markings}
  \tikzset{on grid}
\fi

\usepackage{pstricks,pst-node}

\makeatletter
\newcommand*\pcolumnfuzz[1]{\pretocmd{\@endpbox}{\hfuzz=#1}{}{}}
\makeatother

\usepackage[open,openlevel=0]{bookmark}[2011/12/02]


\ActivateWarningFilters[Tutorial]%
\begin{filecontents}{\jobname.bib}
@book{talbot2012,
  author    = {Nicola L. C. Talbot},
  title     = {{\LaTeX} for Complete Novices},
  publisher = {Dickimaw Books},
  series    = {Dickimaw {\LaTeX} Series},
  volume    = {1},
  isbn      = {978-1-909440-00-5},
  address   = {Norfolk, UK},
  year      = {2012},
  url       = {http://www.dickimaw-books.com/latex/novices/},
  urldate   = {2014-12-01},
}
@book{talbot2013,
  author    = {Nicola L. C. Talbot},
  title     = {Using {\LaTeX} to Write a PhD Thesis},
  publisher = {Dickimaw Books},
  series    = {Dickimaw {\LaTeX} Series},
  volume    = {2},
  isbn      = {978-1-909440-02-9},
  address   = {Norfolk, UK},
  year      = {2013},
  url       = {http://www.dickimaw-books.com/latex/thesis/},
  urldate   = {2014-12-01},
}
@online{reichert2012,
  author       = {Reichert, Axel and Voß, Herbert},
  title        = {\LaTeX~-- Satz von Tabellen},
  organization = {Freie Universität Berlin},
  date         = {2012-01-12},
  url          = {http://userpage.fu-berlin.de/latex/Materialien/tabsatz.pdf},
  urldate      = {2014-12-01}
}
@article{neubauer1996,
  author       = {Marion Neubauer},
  title        = {Feinheiten bei wissenschaftlichen Publikationen~--
                   Mikrotypographie"=Regeln, Teil~I},
  journaltitle = {Die \TeX{}nische Komödie},
  year         = {1997},
  month        = {2},
  volume       = {4/96},
  pages        = {23-40},
  url          = {http://www.dante.de/tex/Dokumente/dtk-neubauer.pdf},
  urldate      = {2014-12-01},
}
@article{neubauer1997,
  author       = {Marion Neubauer},
  title        = {Feinheiten bei wissenschaftlichen Publikationen~--
                   Mikrotypographie"=Regeln, Teil~II},
  journaltitle = {Die \TeX{}nische Komödie},
  year         = {1997},
  month        = {5},
  volume       = {1/97},
  pages        = {25--44},
  url          = {http://www.dante.de/tex/Dokumente/dtk-neubauer.pdf},
  urldate      = {2014-12-01},
}
@online{struckmann2007,
  author  = {Struckmann, Werner},
  title   = {Einige typographische Grundregeln und ihre Umsetzung in \LaTeX},
  date    = {2007-09-03},
  url     = {http://www2.informatik.hu-berlin.de/sv/lehre/typographie.pdf},
  urldate = {2014-12-01}
}
@online{bier2009,
  author  = {Bier, Christoph},
  title   = {typokurz~-- Einige wichtige typografische Regeln},
  date    = {2009-05-21},
  url     = {http://zvisionwelt.wordpress.com/downloads/#typokurz},
  urldate = {2014-12-01}
}
\end{filecontents}
\DeactivateWarningFilters[Tutorial]%

\begin{document}
\title{%
  Ein Anwenderleitfaden für das Erstellen einer wissenschaftlichen Abhandlung%
}
\author{Falk Hanisch\thanks{\noexpand\scriptsize\noexpand\Email{\tudscrmail}}}
\date{01.12.2014}
\makeatletter
\begingroup%
  \def\and{, }%
  \let\thanks\@gobble%
  \let\footnote\@gobble%
  \hypersetup{%
    pdfauthor = {\@author},%
    pdftitle = {\@title},%
    pdfsubject = {Tutorial für \hologo{LaTeXe}},%
    pdfkeywords = {LaTeX, \TUDScript, Tutorial, Anwenderleitfaden},%
  }%
\endgroup%
\makeatother
\StartTutorial[%
  \begin{abstract}\noindent
  Der Versuch, ein allumfassendes Tutorial für eine wissenschaftliche Arbeit 
  zur Verfügung zu stellen gleicht der beschwerlichen Suche nach einer 
  eierlegenden Wollmilchsau. Es ist quasi nicht möglich, alle möglichen 
  Anforderungen an eine wissenschaftliche Arbeit in einem Dokument abzudecken. 
  Dennoch soll hier versucht werden, einen Großteil der für gewöhnlich 
  auftretenden Erfordernisse zu bearbeiten.
  
  Dieses Tutorial hat \emph{nicht} die Intention, \hologo{LaTeX}-Einsteigern 
  sämtliche Grundlagen zu erläutern. Vielmehr wird davon ausgegangen, dass Sie 
  bereits erste Erfahrungen mit \hologo{LaTeXe} gesammelt haben. Dennoch wird 
  versucht, alle Schritte möglichst leicht nachvollziehbar zu gestalten. Sollte 
  Ihnen beim Lesen und Durcharbeiten des Tutorials etwas auf- oder missfallen, 
  so dürfen Sie mich gerne per E-Mail kontaktieren. Auch Anregungen und Wünsche 
  dürfen sie mir gegenüber gerne kommunizieren.
  
  Für absolute Neueinsteiger gibt es einige freie Tutorials, welche die ersten 
  Schritte mit \hologo{LaTeXe} stark erleichtern. Sehr empfehlenswert ist die 
  ausführliche \hrfn{http://www.fadi-semmo.de/latex/workshop/}{Workshop-Reihe} 
  von Fadi~Semmo. Außerdem stellt Nicola~L.~C.~Talbot sehr gute Tutorials für 
  \hrfn{http://www.dickimaw-books.com/latex/novices/}{\hologo{LaTeX}-Novizen} 
  \cite{talbot2012} sowie 
  \hrfn{http://www.dickimaw-books.com/latex/thesis/}{Dissertationen} 
  \cite{talbot2013} zur freien Verfügung.
  
  In erster Linie ist dieser Leitfaden für Anwender gedacht, die für ihre
  wissenschaftliche Arbeit eine \TUDScript"=Dokumentklasse verwenden wollen. 
  Das vorgestellte Vorgehen kann jedoch~-- natürlich mit gewissen Abstrichen~-- 
  auch mit anderen Klassen, insbesondere denen aus dem \KOMAScript"=Bundle, 
  umgesetzt werden. Viele der hier verwendeten Optionen und Befehle aus dem 
  \TUDScript-Bundle werden nur sporadisch in ihrer Grundfunktion erläutert. 
  Eine detaillierte Erläuterung lässt sich jedoch jederzeit sehr einfach über 
  die farbigen Hyperlinks im \manualhyperref{}{\TUDScript-Handbuch} öffnen.
  Des Weiteren wird im Tutorial auf eine Vielzahl von Pakete verwiesen, deren 
  Dokumentation sich entweder über den für jedes Paket erzeugten Hyperlink auf 
  das \href{http://www.ctan.org/}{Comprehensive TeX Archive Network (CTAN)} 
  oder alternativ über die Kommandozeile respektive das Terminal mit dem Aufruf 
  \PValue{texdoc }\PName{Paket} direkt öffnen lässt.
  
  Der Anwenderleitfaden muss nicht zwingend vollständig nachvollzogen werden. 
  Dieser ist in einzelne Abschnitte untergliedert, damit Sie sich bestimmte 
  Aspekte erarbeiten können. Sollten Querbezüge zu den einzelnen Abschnitten 
  bestehen, werden diese auch genannt. Zu guter Letzt findet sich am Ende 
  dieses Dokumentes das komplette Tutorial als ausführbarer Quelltext. 
  \end{abstract}
]
\tableofcontents
\listoffigures
\listoftables

\addsec{Index}
Da das Tutorial etwas umfangreicher ist, wird für alle erläuterten Optionen, 
Umgebungen und Befehle ein Index erstellt.
\PrintIndex

\FinishTutorial[%
  Um das im kopierten Beispiel erstellte Literaturverzeichnis in das Dokument 
  einbinden zu können, bedarf es dem einmaligen Aufruf von \Application{biber} 
  nach dem ersten Durchlauf von \hologo{pdfLaTeX}. Dies erfolgt mit dem Aufruf 
  \PValue{biber}~\PName{Dateiname}. Danach ist ein weiteres mal die Verwendung 
  von \PValue{pdflatex}~\PName{Dateiname} notwendig.

  Für das Erstellen von Abkürzungs- und Symbolverzeichnis sollte das mehrmalige 
  Ausführen von \PValue{pdflatex}~\PName{Dateiname} vollkommen ausreichen. In 
  diesem Fall werden die Einträge mit \Application{makeindex} sortiert. Soll 
  stattdessen \Application{xindy} die Sortierung durchführen, muss beim Laden 
  von \Package{glossaries} die entsprechende Paketoption aktiviert werden. Für
  diesen Fall sollte nach der Verwendung von \hologo{pdfLaTeX} der Aufruf des 
  Perl"=Skriptes \PValue{makeglossaries}~\PName{Dateiname} erfolgen, was 
  allerdings nur mit \Distribution{\hologo{TeX}~Live} ohne weiteres Zutun 
  möglich ist. Nutzer von \Distribution{\hologo{MiKTeX}} müssen das Sortieren 
  mit \Application{xindy} händisch anstoßen.
]
\end{document}
