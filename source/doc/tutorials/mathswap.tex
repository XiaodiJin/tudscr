\RequirePackage[ngerman=ngerman-x-latest]{hyphsubst}
\documentclass[english,ngerman]{tudscrartcl}
\usepackage{selinput}\SelectInputMappings{adieresis={ä},germandbls={ß}}
\usepackage[T1]{fontenc}
\usepackage{tudscrman}
\lstset{%
  inputencoding=utf8,extendedchars=true,
  literate=%
    {ä}{{\"a}}1 {ö}{{\"o}}1 {ü}{{\"u}}1
    {Ä}{{\"A}}1 {Ö}{{\"O}}1 {Ü}{{\"U}}1
    {~}{{\textasciitilde}}1 {ß}{{\ss}}1
}

\usepackage{mathswap}

\begin{document}
\date{16.12.2014}
\author{Falk Hanisch\thanks{\noexpand\scriptsize\noexpand\Email{\tudscrmail}}}
\title{Änderung der Trennzeichen im Mathematikmodus}
\makeatletter
\begingroup%
  \def\and{, }%
  \let\thanks\@gobble%
  \let\footnote\@gobble%
  \hypersetup{%
    pdfauthor = {\@author},%
    pdftitle = {\@title},%
    pdfsubject = {Mathematiksatz in \hologo{LaTeXe}},%
    pdfkeywords = {LaTeX, \TUDScript, Tutorial, Mathematiksatz},%
  }%
\endgroup%
\markright{\@title}
\makeatother
\StartTutorial
%
\begin{Preamble}
\RequirePackage[ngerman=ngerman-x-latest]{hyphsubst}
\documentclass[english,ngerman]{tudscrartcl}% andere Klassen sind möglich
\usepackage{selinput}\SelectInputMappings{adieresis={ä},germandbls={ß}}
\usepackage[T1]{fontenc}
\usepackage{babel}
\usepackage{microtype}

\end{Preamble}
%
Zusätzlich wird das Paket \Package{mathswap} geladen, welches die Änderung der 
Gruppierungs- und Dezimaltrennzeichen von Zahlen im Mathematikmodus ermöglicht.
%
\begin{Preamble}
\usepackage{mathswap}
\end{Preamble}
%
Werden in einer wissenschaftlichen Abhandlung vielerlei Daten importiert und 
beispielsweise tabellarisch dargestellt, kann es durchaus sein, dass diese 
importierten Daten nicht dem Zahlenformat entsprechen, welches für die 
verwendete Dokumentsprache normalerweise notwendig wäre. Um sich die manuelle 
Konvertierung der Daten zu ersparen, kann alternativ dazu die Möglichkeit 
genutzt werden, dies mit Hilfe von \hologo{LaTeXe} selbst zu erledigen. In 
diesem Tutorial wird hierfür das Paket \Package*{mathswap} genutzt. Alternativ 
dazu kann auch das Paket \Package{ionumbers} verwendet werden.

Ohne die Verwendung eines speziellen Paketes zur Zahlenformatierung wird eine 
Zahl im mathematischen Modus durch \hologo{LaTeXe} normalerweise so 
ausgegeben: 
%
\CodePreamble{%
  Ausgabe mit deutscher Zifferngruppierung ohne \Package*{mathswap}%
}
\begin{Trunk*}
\mathswapoff
\(4.523,58\)
\mathswapon
\end{Trunk*}
%
Da in der Präambel dieses Dokumentes das Paket \Package*{mathswap} bereits 
geladen wurde, musste mit \Macro*{mathswapoff} auf das Standardverhalten von 
\hologo{LaTeXe} geschaltet werden. Der Befehl \Macro*{mathswapon} aktiviert 
wiederum die Funktionalität von \Package*{mathswap}.

Für diese Tutorial wird angenommen, dass die wissenschaftlichen Abhandlung in 
deutscher Sprache verfasst wird. Ist nun eine Zahl im deutschen Zahlenformat 
gegeben, kann diese einfach im Mathematikmodus angegeben werden. Diese wird 
den hierzulande existierenden Konventionen entsprechend gruppiert ausgegeben:
%
\CodePreamble{%
  Ausgabe mit deutscher Zifferngruppierung und Sprache \PValue{ngerman}%
}
\begin{Trunk*}
\(4.523,58\)
\end{Trunk*}
%
Sollte die gleiche Zahl in englischer Formatierung gegeben sein, funktioniert 
dies nicht mehr:
%
\CodePreamble{%
  Ausgabe mit englischer Zifferngruppierung und Sprache \PValue{ngerman}%
}
\begin{Trunk*}
\(4,523.58\)
\end{Trunk*}
%
Mit der Verwendung der beiden Befehle \Macro*{commaswap}\PParameter{\bsc,}
sowie \Macro*{dotswap}\PParameter{,} können die Substitutionen für Komma und 
Punkt im Mathematikmodus geändert:
%
\CodePreamble{%
  Ausgabe mit englischer Zifferngruppierung und geänderten Trennzeichen%
}
\begin{Trunk*}
\begingroup
  \commaswap{\,}
  \dotswap{,}
  \(4,523.58\)
\endgroup
\end{Trunk*}
%
Die Verwendung von \Macro{begingroup} und \Macro{endgroup} führt hierbei 
dazu, dass die Änderungen der beiden Trennzeichen nur lokal innerhalb dieser 
Gruppe erfolgt.

Wird das Paket \Package*{mathswap} zusammen mit einer \TUDScript-Klasse 
verwendet, werden die Dezimaltrennzeichen im Mathematikmodus sprachabhängig 
definiert. Wird die Dokumentsprache auf \PValue{english} gesetzt, so werden die
Dezimaltrennzeichen standardmäßig für die englischsprachige Zifferngruppierung 
definiert. In diesem Fall müssen Zahlen, welche im deutschsprachigen Format 
gruppiert sind, angepasst werden:
%
\CodePreamble{
  Ausgabe bei \PValue{english} als gewählter Sprache%
}
\begin{Trunk*}
\begingroup%
  \selectlanguage{english}%
  \(4,523.58\)\\
  \(4.523,58\)\\
  \commaswap{.}
  \dotswap{\,}
  \(4.523,58\)%
\endgroup
\end{Trunk*}
%
\FinishTutorial
\end{document}
