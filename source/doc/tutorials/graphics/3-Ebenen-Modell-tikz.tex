\RequirePackage[ngerman=ngerman-x-latest]{hyphsubst}
\documentclass[ngerman,DIV=12]{scrreprt}
\usepackage{selinput}\SelectInputMappings{adieresis={ä},germandbls={ß}}
\usepackage[T1]{fontenc}
\usepackage{babel}

\usepackage{showframe}

\usepackage{tikz}
\usetikzlibrary{arrows}
\usetikzlibrary{trees}
\usetikzlibrary{shapes.misc}
\usetikzlibrary{decorations.markings}
\usetikzlibrary{arrows}
\usetikzlibrary{calc}
\usetikzlibrary{chains}
\usetikzlibrary{decorations.markings}
\usetikzlibrary{positioning}
\usetikzlibrary{shapes.misc}
\usetikzlibrary{trees}

\tikzset{on grid}

% pstricks-Pfeile in TikZ
\tikzstyle{pstarrow->}=[%
	decoration={markings,
		mark=at position 1 with {\arrow[xscale=1.5]{stealth}};
	},
  postaction={decorate},
  shorten >=0.7pt
]
\tikzstyle{pstarrow<-}=[%
	decoration={markings,
		mark=at position 4.8pt with {\arrow[xscale=1.5]{stealth reversed}};
	},
  postaction={decorate},
  shorten <=0.7pt
]
\tikzstyle{pstarrow<->}=[%
	decoration={markings,
		mark=at position 4.8pt with {\arrow[xscale=1.5]{stealth reversed}},
		mark=at position 1 with {\arrow[xscale=1.5]{stealth}};
	},
  postaction={decorate},
  shorten <=0.7pt,
  shorten >=0.7pt
]

% abgerundete Boxen
\tikzstyle{rounded box}=[%
	minimum height=3.5ex, minimum width=\myGlength,rounded rectangle,draw
]

% abgewinkelte Baumdiagramme
\tikzstyle{edge description}=[%
	edge from parent path={%
  	(\tikzparentnode\tikzparentanchor)
  	|- 
  	+(0pt,-\tikzsiblingdistance-\tikznumberofcurrentchild\tikzsiblingdistance)
  	-- (\tikzchildnode\tikzchildanchor)
  },
	grow via three points={%
		one child at (\tikzleveldistance,-\tikzsiblingdistance)
		and two children at (\tikzleveldistance,-\tikzsiblingdistance)
		and (\tikzleveldistance,-2\tikzsiblingdistance)
	}
]


\begin{document}
\newlength{\tikzunit}
\setlength{\tikzunit}{.01\textwidth}
\tikzset{x=\tikzunit,y=\tikzunit}

\tikzstyle{inner box}=[%
  text width=17\tikzunit,%
  align=center,
  rectangle,
  minimum height=8\tikzunit,
  font=\hspace{0pt},
  draw
]
\tikzstyle{inner label}=[align=center, font=\scriptsize]
\tikzstyle{inner box chain}=[every node/.style={on chain}]
\tikzstyle{inner box chain below}=[%
  inner box chain, node distance=8\tikzunit,continue chain=going below
]
\tikzstyle{inner box chain right}=[%
  inner box chain,node distance=35\tikzunit,continue chain=going right
]
\tikzstyle{inner box chain above}=[%
  inner box chain,node distance=16\tikzunit,continue chain=going above
]
\begin{figure}
\centering
\begin{tikzpicture}
  \begin{scope}[start chain]
    \begin{scope}[inner box chain below]
      \node(NE)[inner box]{Navigations\-ebene};
      \node(NB)[inner label]{gewählte Fahrtroute\\zeitlicher Ablauf};
      \node(BE)[inner box]{{Bahnführungs\-ebene}};
      \node(BS)[inner label]{%
        gewählte Führungsgrößen:\\Sollspur, Sollgeschwindigkeit%
      };
      \node(SE)[inner box]{Stabilisierungs\-ebene};
    \end{scope}
    \begin{scope}[inner box chain right]
      \node(LQ)[inner box]{Längs- und Querdynamik};
      \node(FO)[inner box]{Fahrbahn\-oberfläche};
    \end{scope}
    \begin{scope}[inner box chain above]
      \node(FR)[inner box]{Fahrraum\\\smallskip{\scriptsize Straße 
      und\\\vspace{-1.5ex}Verkehrssituation}};
      \node(SN)[inner box]{Straßennetz};
    \end{scope}
  \end{scope}
  \begin{scope}[inner label,minimum size=0pt]
    \draw [pstarrow->] (FO) -| ++(13,-12) to node [above]{Istspur, 
    Istgeschwindigkeit} ++(-96,0) |- (SE);
    \draw [pstarrow->] (FR) -| ++(14  ,-33) to node [above]{Bereich sicherer 
    Führungsgrößen} ++(-98,0) |- (BE);
    \draw [pstarrow->] (SN) -| ++(15  ,-54) to node [above]{mögliche 
    Fahrtroute} ++(-100,0) |- (NE);
  \end{scope}
  \begin{scope}[inner label]
    \draw              (NE) to (NB);
    \draw [pstarrow->] (NB) to (BE);
    \draw              (BE) to (BS);
    \draw [pstarrow->] (BS) to (SE);
    \draw [pstarrow->] (SE) to
      node[above] {\parbox{9\tikzunit}{\centering Stell\-größen}}
      node[below] {\parbox{9\tikzunit}{\centering Lenken Gasgeben Bremsen}}
    (LQ);
    \draw [pstarrow->] (LQ) to
      node[above]{\parbox{9\tikzunit}{\centering Regel\-größen}}
      node[below]{\parbox{9\tikzunit}{\centering\hspace{0pt}Fahrzeugbewegung}}
    (FO);
    \draw [pstarrow->] (LQ)+(24,0) |- (FR);
    \draw [pstarrow->] (LQ)+(24,0) |- (SN);
  \end{scope}
  \begin{scope}[very thick,rounded corners=5\tikzunit]
    \draw (-12.5,-40) rectangle (12.5,12);
    \draw ( 22.5,-40) rectangle (47.5,-20);
    \draw ( 57.5,-40) rectangle (82.5,12);
  \end{scope}
  \begin{scope}[font=\bfseries]
    \node at (0,8) {Fahrer};
    \node at (35,-24) {Fahrzeug};
    \node at (70,8) {Umwelt};
  \end{scope}
\end{tikzpicture}
\end{figure}
\end{document}
