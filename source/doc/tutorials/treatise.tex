\RequirePackage[ngerman=ngerman-x-latest]{hyphsubst}
\documentclass[english,ngerman]{tudscrartcl}
\usepackage{selinput}
\SelectInputMappings{adieresis={ä},germandbls={ß}}
\usepackage[T1]{fontenc}
\usepackage{lmodern}
\usepackage{tudscrman}
\lstset{%
  inputencoding=utf8,extendedchars=true,
  literate=%
    {ä}{{\"a}}1 {ö}{{\"o}}1 {ü}{{\"u}}1
    {Ä}{{\"A}}1 {Ö}{{\"O}}1 {Ü}{{\"U}}1
    {~}{{\textasciitilde}}1 {ß}{{\ss}}1
}

\usepackage{atbegshi}

\usepackage{tudscrsupervisor}

\begin{document}
\TUDoption{cdfont}{no,nodin}
\title{Eine Vorlage für eine Abschlussarbeit}
\author{Falk Hanisch}
\date{09.09.2014}
\StartTutorial[\tableofcontents]
%
\section{Einleitung}
Der Versuch, ein allumfassendes Tutorial für eine Abschlussarbeit zur Verfügung 
zu stellen gleicht der Suche nach einer eierlegenden Wollmilchsau. Es ist quasi 
nicht möglich, alle möglichen Anforderungen an eine wissenschaftliche Arbeit in 
einem Dokument abzudecken. Dennoch soll hier versucht werden, einen Großteil 
der für gewöhnlich auftretenden Erfordernisse zu bearbeiten. Am Ende dieses 
Tutorials findet sich der resultierende, ausführbare Quelltext.

Zu Beginn werden allerhand Pakete geladen, auf die im späteren Verlauf noch 
genauer eingegangen wird. Den Start macht das Paket \Package{hyphsubst}. Dieses 
wird für eine verbesserte Silbentrennung für die deutsche Sprache benötigt und 
muss bereits \emph{vor} der Klasse geladen werden, damit es richtig 
funktioniert. Die Option \Option*{ngerman} führt dabei zur Verwendung der neuen 
deutschen Rechtschreibung. Für die alte Orthographie ist stattdessen die Option 
\Option*{german} zu verwenden.

Beim Laden der Klasse mit \Macro*{documentclass} sollten die im Dokument 
verwendeten Sprachen als Klassenoption angegeben werden, wobei die zuletzt 
angegebene als aktuelle Sprache aktiviert wird. Dadurch werden diese nicht nur 
an das Paket \Package{babel} sondern auch an andere Pakete weitergereicht, 
welche sprachspezifische Einstellungen vornehmen.
%
\begin{Tutorial*}
\RequirePackage[ngerman=ngerman-x-latest]{hyphsubst}
\documentclass[ngerman]{tudscrreprt}% andere Klassen sind bedingt möglich
\usepackage{babel}
\end{Tutorial*}
%
Bei der Verwendung von \hologo{LaTeXe} sollte zum einen die Eingabekodierung 
des erstellten Datei spezifiziert werden. Das Paket \Package{selinput} erkennt 
automatisch, welche Kodierung der genutzte Editor verwendet. Zum anderen werden 
die Schriften in der Ausgabe ebenfalls kodiert. Mit dem Paket \Package{fontenc} 
lässt sich die Schriftkodierung spezifizieren, wobei die europäischen Zeichen 
mit der Option~\Option*{T1} aktiviert werden.
%
\begin{Tutorial*}
\usepackage{selinput}\SelectInputMappings{adieresis={ä},germandbls={ß}}
\usepackage[T1]{fontenc}
\end{Tutorial*}
%
Das Paket \Package{fixltx2e} behebt einige Fehler im \hologo{LaTeXe}-Kernel. In 
neuen Dokumenten kann es bedenkenlos geladen werden.
%
\begin{Tutorial*}
\usepackage{fixltx2e}
\end{Tutorial*}
%
Für die Aufgabenstellung wird das Paket \Package{tudscrsupervisor} benötigt.
%
\begin{Tutorial*}
\usepackage{tudscrsupervisor}
\end{Tutorial*}
%
Damit sind alle notwendigen Pakete eingebunden und es das eigentliche Dokument 
kann begonnen werden.
\begin{Tutorial*}
\begin{document}
\end{Tutorial*}

\section{Umschlagseite und Titel}

\begin{Tutorial}
\faculty{Juristische Fakultät}
\department{Fachrichtung Strafrecht}
\institute{Institut für Kriminologie}
\chair{Lehrstuhl für Kriminalprognose}
\title{%
  Entwicklung eines optimalen Verfahrens zur Eroberung des
  Geldspeichers in Entenhausen
}
\thesis{master}
\graduation[M.Sc.]{Master of Science}
\author{%
  Mickey Mouse\matriculationnumber{12345678}
  \dateofbirth{2.1.1990}\placeofbirth{Dresden}
  \course{Klinische Prognostik}\discipline{Individualprognose}
  \and%
  Donald Duck\matriculationnumber{87654321}
  \dateofbirth{1.2.1990}\placeofbirth{Berlin}
  \course{Statistische Prognostik}\discipline{Makrosoziologische Prognosen}
}
\date{20.04.2014}
\issuedate{1.2.2012}
\duedate{1.8.2012}
\matriculationyear{2010}
\supervisor{Dagobert Duck \and Mac Moneysac}
\professor{Prof. Dr. Kater Karlo}
\chairman{Prof. Dr. Primus von Quack}
\end{Tutorial}
\begin{Tutorial*}
\makecover
\maketitle
\end{Tutorial*}




\section{Aufgabenstellung}

\section{Erklärungen}

\section{Inhalts-, Abbildungs-, und Tabellenverzeichnis}

%%Anfangszitat: Der Text erscheint jetzt kursiv (\itshape) und kleiner als der 
%%normale Text (\footnotesize). Mit \dictumwidth wird die Breite des Textfeldes 
%%gesteuert. Sie nimmt jetzt 65% der gesamten Textbreite ein. Mit 
%%%\raggeddictumtext wird die Ausrichtung des Textes gesteuert. \raggedleft 
%%%heißt, 
%%%dass der Text links „ausflattert”, also rechtsbündig gesetzt wird.
%\setkomafont{dictumtext}{\itshape}
%\renewcommand*{\dictumwidth}{.75\textwidth}
%\renewcommand*{\raggeddictumtext}{\raggedleft}


\section{Gleitumgebungen}
\subsection{Grafiken und Untergrafiken}
\subsection{Tabellen}

\section{Literaturverzeichnis}

\section{Abkürzungsverzeichnis}

\section{Symbolverzeichnis}

\section{Listen}

\section{Mathematiksatz}

\section{Typographie}
\subsection{Abkürzungen}
\subsection{Einheiten}

\section{Querverweise}

\section{Quelltexte}

\section{Silbentrennung}


\dots % §§§
%
\begin{Tutorial*}
\usepackage{xparse}% §§§ bereits geladen
\usepackage{microtype}% §§§ bereits geladen
\usepackage{textcomp}% §§§ bereits geladen
\usepackage{setspace}% §§§ bereits geladen
\usepackage{csquotes}% §§§ bereits geladen
\usepackage{quoting}% §§§ bereits geladen
\usepackage{isodate}% §§§ bereits geladen
\usepackage{listings}% §§§ bereits geladen
\usepackage{hyperref}% §§§ bereits geladen
\usepackage{xspace}% §§§ bereits geladen
\usepackage{scrhack}% §§§ bereits geladen
\end{Tutorial*}
%
\begin{Tutorial*}
\usepackage{tudscrsupervisor}
\usepackage{caption}
\usepackage{floatrow}
\usepackage{tikz}
\usepackage{biblatex}
\usepackage{glossaries}
\usepackage{array}
\usepackage{booktabs}
\usepackage{tabularx}
\usepackage{ltxtable}
\usepackage{tabu}
\usepackage{enumitem}
\setlist{noitemsep}
\usepackage{mathtools}
\usepackage{units}
\usepackage{siunitx}
\usepackage{xpunctuate}
\usepackage{ellipsis}
\usepackage{varioref}
\usepackage{chngcntr}
\usepackage{bookmark}
\end{Tutorial*}
%
\begin{Tutorial*}
\taskform[pagestyle=empty]{%
  Momentan ist das besagte Thema in aller Munde. Insbesondere wird es
  gerade in vielen~-- wenn nicht sogar in allen~-- Medien diskutiert.
  Es ist momentan noch nicht abzusehen, ob und wann sich diese Situation
  ändert. Eine kurzfristige Verlagerung aus dem Fokus der Öffentlichkeit
  wird nicht erwartet.
  Als Ziel dieser Arbeit soll identifiziert werden, warum das Thema
  gerade so omnipräsent ist und wie man diesen Effekt abschwächen
  könnte. Zusätzlich sollen Methoden entwickelt werden, wie sich ein
  ähnlicher Vorgang zukünftig vermeiden ließe.
}{%
  \item Recherche
  \item Analyse
  \item Entwicklung eines Konzeptes
  \item Anwendung der entwickelten Methodik
  \item Dokumentation und grafische Aufbereitung der Ergebnisse
}
\end{document}
\end{Tutorial*}

\FinishTutorial
\clearpage
  \begin{itemize}
  \item Cover
  \item Tiel
  \item Aufgabenstellung
  \item inhaltsverzeichnis, tabellen und abbildungsverzeichnis
  \item abstract
  \item grafiken und untergrafiken?
  \item tabellen
  \item literaturverzteichnis
  \item abkürzungsverzeichnis
  \item floatrow
  \item mathematikmodus
  \item tabbing-umgebung
  \end{itemize}
\end{document}