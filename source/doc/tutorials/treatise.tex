\RequirePackage[ngerman=ngerman-x-latest]{hyphsubst}
\documentclass[%
  english,ngerman,%
  geometry=no,DIV=12,automark,%
]{tudscrartcl}
\usepackage{selinput}
\SelectInputMappings{adieresis={ä},germandbls={ß}}
\usepackage[T1]{fontenc}
\usepackage{lmodern}

\usepackage{tudscrman}
\lstset{%
  inputencoding=utf8,extendedchars=true,
  literate=%
    {ä}{{\"a}}1 {ö}{{\"o}}1 {ü}{{\"u}}1
    {Ä}{{\"A}}1 {Ö}{{\"O}}1 {Ü}{{\"U}}1
    {~}{{\textasciitilde}}1 {ß}{{\ss}}1
}
\usepackage{tudscrsupervisor}
\usepackage{tabu,booktabs}
\usepackage{units}
\AfterPackage*{hyperref}{%
  \usepackage[%
    automake,%
    acronym,%
    symbols,%
    nomain,%
    translate=babel,%
    nogroupskip,%
    section=subsubsection,%
  ]{glossaries}
  \setStyleFile{\jobname-temp}
  \renewcommand*{\glsglossarymark}[1]{}
  \makeglossaries
}
\usepackage{csquotes}
\usepackage[backend=biber,style=alphabetic]{biblatex}
\usepackage{filecontents}
\addbibresource{treatise-temp.bib}
\renewcommand{\floatpagefraction}{0.7}

\ifpdf
  \usepackage{tikz}
  \usetikzlibrary{chains}
  \usetikzlibrary{decorations.markings}
  \tikzset{on grid}
\fi

\usepackage{pstricks,pst-node}

\begin{document}
\TUDoptions{cdfont=false}
\KOMAoptions{headings=normal}
\title{%
  Ein Anwenderleitfaden für das Erstellen einer wissenschaftlichen Abhandlung%
}
\author{Falk Hanisch}
\date{10.09.2014}
\makeatletter
\begingroup%
  \def\and{, }%
  \let\thanks\@gobble%
  \let\footnote\@gobble%
  \hypersetup{%
    pdfauthor = {\@author},%
    pdftitle = {\@title},%
    pdfsubject = {Tutorial für \hologo{LaTeXe}},%
    pdfkeywords = {LaTeX, \TUDScript, Tutorial, Anwenderleitfaden},%
  }%
\endgroup%
\makeatother
\StartTutorial[%
  \begin{abstract}\noindent
  Der Versuch, ein allumfassendes Tutorial für eine wissenschaftliche Arbeit 
  zur Verfügung zu stellen gleicht der beschwerlichen Suche nach einer 
  eierlegenden Wollmilchsau. Es ist quasi nicht möglich, alle möglichen 
  Anforderungen an eine wissenschaftliche Arbeit in einem Dokument abzudecken. 
  Dennoch soll hier versucht werden, einen Großteil der für gewöhnlich 
  auftretenden Erfordernisse zu bearbeiten.
  
  Dieses Tutorial hat \emph{nicht} die Intention, \hologo{LaTeX}-Einsteigern 
  sämtliche Grundlagen zu erläutern. Vielmehr wird davon ausgegangen, das der 
  Leser die ersten Versuche mit \hologo{LaTeXe} bereits hinter sich hat. 
  Dennoch wird versucht, alle Schritte möglichst leicht nachvollziehbar zu 
  gestalten.
  
  Für absolute Neueinsteiger gibt es einige freie Tutorials, welche die ersten 
  Schritte mit \hologo{LaTeXe} stark erleichtern. Sehr empfehlenswert ist die 
  ausführliche \hrfn{http://www.fadi-semmo.de/latex/workshop/}{Workshop-Reihe} 
  von Fadi~Semmo. Außerdem stellt Nicola~L.~C.~Talbot sehr gute Tutorials für 
  \hrfn{http://www.dickimaw-books.com/latex/novices/}{\hologo{LaTeX}-Novizen} 
  sowie \hrfn{http://www.dickimaw-books.com/latex/thesis/}{Dissertationen} zur 
  freien Verfügung.
  
  In erster Linie ist dieser Leitfaden für Anwender gedacht, die für ihre
  wissenschaftliche Arbeit eine \TUDScript"=Dokumentklasse verwenden wollen. 
  Das vorgestellte Vorgehen kann jedoch~-- natürlich mit gewissen Abstrichen~-- 
  auch mit anderen Klassen, insbesondere denen aus dem \KOMAScript"=Bundle, 
  umgesetzt werden. Viele der hier verwendeten Optionen und Befehle aus dem 
  \TUDScript-Bundle werden nur sporadisch in ihrer Grundfunktion erläutert. 
  Eine detaillierte Erläuterung lässt sich jedoch jederzeit sehr einfach über 
  die farbigen Hyperlinks im \href{run:../tudscr.pdf}{\TUDScript-Handbuch} 
  öffnen. Des Weiteren wird hier sehr häufig auf die Dokumentation zu einzelnen 
  Paketen hingewiesen. Diese lassen sich sehr einfach über die Kommandozeile 
  respektive das Terminal über den Aufruf \PValue{texdoc }\PName{Paket} öffnen.
  
  Der Anwenderleitfaden muss nicht zwingend vollständig nachvollzogen werden. 
  Dieser ist in einzelne Abschnitte untergliedert, damit der Anwender sich 
  bestimmte Aspekte erarbeiten kann. Sollten Querbezüge zu den einzelnen 
  Abschnitten bestehen, werden diese auch genannt. Zu guter Letzt findet sich 
  am Ende dieses Dokumentes das komplette Tutorial als ausführbarer Quelltext. 
  \end{abstract}
]
%\clearpage
\tableofcontents
\listoffigures
\listoftables
%\clearpage



\section{Einleitung}
\label{sec:introduction}
Zu Beginn werden allerhand Pakete geladen. Bei einigen wird in der Einleitung 
nicht weiter darauf eingegangen, diese werden in den relevanten Abschnitten in 
diesem Tutorial genauer erläutert. 

Den Anfang macht das Paket \Package{hyphsubst}. Dieses wird für eine wesentlich 
verbesserte Silbentrennung für die deutsche Sprache benötigt und muss bereits 
\emph{vor} der Klasse geladen werden, damit es problemlos funktioniert. Die 
Option \Option*{ngerman} führt dabei zur Verwendung der Trennmuster für die 
neue deutsche Rechtschreibung, wobei der Wert \PValue{ngerman-x-latest} die 
neuesten lädt. Für die alte Orthographie ist stattdessen die Option 
\Option*{german} und der angepasste Wert zu verwenden.
%
\begin{Tutorial*}
\RequirePackage[ngerman=ngerman-x-latest]{hyphsubst}
\end{Tutorial*}
%
Beim Laden der Klasse mit \Macro*{documentclass} sollten die im Dokument 
verwendeten Sprachen als Klassenoption angegeben werden, wobei die zuletzt 
angegebene als aktuelle Sprache aktiviert wird. Dadurch werden diese nicht nur 
an das Paket \Package{babel} sondern auch an andere Pakete weitergereicht, 
welche sprachspezifische Einstellungen vornehmen.
%
\begin{Tutorial*}
\documentclass[english,ngerman]{tudscrreprt}
\usepackage{babel}
\end{Tutorial*}
%
Bei der Verwendung von \hologo{LaTeXe} sollte zum einen die Eingabekodierung 
des erstellten Datei spezifiziert werden. Das Paket \Package{selinput} erkennt 
automatisch, welche Kodierung der genutzte Editor verwendet. Zum anderen werden 
die Schriften in der Ausgabe ebenfalls kodiert. Mit dem Paket \Package{fontenc} 
lässt sich die Schriftkodierung spezifizieren, wobei die europäischen Zeichen 
mit der Option~\Option*{T1} aktiviert werden.
%
\begin{Tutorial*}
\usepackage{selinput}\SelectInputMappings{adieresis={ä},germandbls={ß}}
\usepackage[T1]{fontenc}
\end{Tutorial*}
%
Das Paket \Package{fixltx2e} behebt einige Fehler im \hologo{LaTeXe}-Kernel. 
In neuen Dokumenten kann es bedenkenlos geladen werden.
%
\begin{Tutorial*}
\usepackage{fixltx2e}
\end{Tutorial*}
%
Das Paket \Package{tabu} ist relativ neu und versucht, viele Funktionalitäten 
ganz unterschiedlicher Pakete für den Tabellensatz in sich zu vereinen. Damit 
die erstellten Tabellen auch gut aussehen, ist das Paket \Package{booktabs} 
sehr zu empfehlen. Hinweise zum guten und korrekten Tabellensatz werden in 
\autoref{sec:tables} gegeben.
%
\begin{Tutorial*}
\usepackage{tabu}
\usepackage{booktabs}
\end{Tutorial*}
%
Mit \Package{units} lassen sich Einheiten typographisch korrekt setzen. 
Außerdem stellt diese Paket den Befehl \Macro*{nicefrac} zur Verfügung, mit dem 
sich eine schönere Darstellungen von mathematischen Brüchen~-- insbesondere für 
den Fließtext~-- setzen lässt.
%
\begin{Tutorial*}
\usepackage{units}
\end{Tutorial*}
%
Für eine Aufgabenstellung im Corporate Design der Technischen Universität 
Dresden, wie sie in \autoref{sec:task} vorgestellt wird, muss das Paket 
\Package*{tudscrsupervisor} geladen werden.
%
\begin{Tutorial*}
\usepackage{tudscrsupervisor}
\end{Tutorial*}
%
Für das Literaturverzeichnis wird das Paket \Package{biblatex} geladen. Für 
dieses wird außerdem die Verwendung von \Package{csquotes} sehr empfohlen. Mit 
der Umgebung \Environment*{filecontents} wird eine Literaturdatenbank 
\PValue{treatise-temp.bib} direkt beim Kompilieren mit \hologo{LaTeX} für ein 
kleines Beispiel erzeugt. Damit dies im Dokument selbst und nicht nur in der 
Präambel erfolgen kann, wird das Paket \Package{filecontents} geladen. Alles 
Weitere dazu ist in \autoref{sec:biblatex} zu finden.
%
\begin{Tutorial*}
\usepackage{csquotes}
\usepackage[backend=biber,style=alphabetic]{biblatex}
\usepackage{filecontents}
\addbibresource{treatise-temp.bib}
\end{Tutorial*}
%
Damit alle möglichen Querverweise in einem PDF-Dokument automatisch verlinkt 
werden, sollte das Paket \Package{hyperref} geladen werden. Um die erzeugten 
Links verträglich aussehen zu lassen, werden die Optionen \Option*{colorlinks} 
sowie \Option*{linkcolor}[blue] verwendet. Da \Package{hyperref} allerhand 
Veränderungen an vielen Standardbefehlen vornimmt, sollte dieses als letztes in 
der Präambel eingebunden werden. Nur Pakete, bei denen in der Dokumentation 
explizit darauf hingewiesen wird, dass diese nach \Package{hyperref} zu laden 
sind, sollten auch danach folgen.
%
\begin{Tutorial*}
\usepackage[colorlinks,linkcolor=blue]{hyperref}
\end{Tutorial*}
%
Eines dieser wenigen Pakete ist \Package{glossaries}. Dieses wird in diesem 
Tutorial für die Erstellung von Abkürzung- und Symbolverzeichnis verwendet. 
Genaueres hierzu ist in \autoref{sec:glossaries} zu finden. Dort wird auch 
genauer auf die hier genutzten Paketoptionen eingegangen.
%
\begin{Tutorial*}
\usepackage[%
  automake,%
%  xindy,\%={language=german-din},\% mit Tex Live einfach verwendbar
  acronym,% Abkürzungen
  symbols,% Formelzeichen
  nomain,% kein Glossar
  translate=babel,%
  nogroupskip,%
  toc,%
  section=section,%
]{glossaries}
\makeglossaries
\end{Tutorial*}
%
Damit sind alle notwendigen Pakete eingebunden und es das eigentliche Dokument 
kann begonnen werden.
\begin{Tutorial*}
\begin{document}
\end{Tutorial*}



\section{Satzspiegel und Bindekorrektur}
Gleich zu Beginn und bevor das eigentliche Verfassen der Arbeit beginnt, sollte 
man sich Gedanken über das zu nutzenden Layout und den Satzspiegel machen, um 
bei der Finalisierung keine böse Überraschung bei Seitenumbrüchen oder der 
Position von Gleitobjekten zu erleben.

Zuallererst gilt zu entscheiden, ob das Dokument einseitig oder beidseitig 
gesetzt werden soll. Ist Letzteres der Fall, so sollte \Option*{twoside} als 
Klassenoption angegeben werden. Im nächsten Schritt ist der zu verwendenden 
Satzspiegel festzulegen. Hierfür kann die Option \Option{geometry} verwendet 
werden, welche im \TUDScript-Handbuch beschrieben wird. Normalerweise wird das 
Dokument im asymmetrischen Layout des \CDs gesetzt. Dieses Verhalten kann mit 
\Option{geometry}[false] deaktiviert werden und der Satzspiegel wird durch das 
Paket \Package{typearea} nach typographischen Gesichtspunkten konstruiert.

Falls die Arbeit nach der Fertigstellung gebunden werden soll, so ist auf den 
notwendigen Binderand zu achten, quasi der Teil einer Seite, welcher durch die 
Bindung \enquote{verschwindet} und nicht mehr als sichtbarer Teil der Seite 
vorhanden ist. Als Faustregel gilt, dass die erforderliche Bindekorrektur in 
etwa der halben Höhe des Buchblocks entsprechen sollte. Dessen Höhe wiederum 
ist abhängig von der Anzahl der Seiten sowie der Papierdichte. Wird qualitativ 
höherwertiges Papier mit einer Dichte von \unit[100]{g/m²} verwendet, so 
entsprechen 100~Blatt in etwa einer Höhe von \unit[12]{mm}. Dementsprechend 
wäre bei diesem Beispiel eine Bindekorrektur \unit[6]{mm} notwendig. Diese 
kann mit der Klassenoption \Option{BCOR}[6mm] eingestellt werden.



\section{Umschlagseite und Titel}
Umschlagseite und Titel sind sich in ihrer Gestalt sehr ähnlich. Allerdings 
gibt es ein paar kleine Unterschiede. Zum einen werden auf dem Cover weniger 
Informationen als auf der Titelseite ausgegeben. Zum anderen wird der Titel 
immer im Satzspiegel des restlichen Dokumentes ausgegeben, wohingegen die 
Umschlagseite ohne weitere Optionen im asymmetrischen Layout des \CDs der \TnUD 
erscheint. Wie dies geändert werden kann, ist im Handbuch für \Macro{makecover} 
erläutert. Die resultierende Ausgabe des nachfolgenden Quelltextauszugs ist in 
\autoref{fig:title} zu sehen.
%
\begin{Tutorial!}{Title}
\faculty{Juristische Fakultät}
\department{Fachrichtung Strafrecht}
\institute{Institut für Kriminologie}
\chair{Lehrstuhl für Kriminalprognose}
\title{%
  Entwicklung eines optimalen Verfahrens zur Eroberung des
  Geldspeichers in Entenhausen
}
\thesis{master}
\graduation[M.Sc.]{Master of Science}
\author{%
  Mickey Mouse
  \matriculationnumber{12345678}
  \dateofbirth{2.1.1990}
  \placeofbirth{Dresden}
  \and%
  Donald Duck
  \matriculationnumber{87654321}
  \dateofbirth{1.2.1990}
  \placeofbirth{Berlin}
}
\matriculationyear{2010}
\supervisor{Dagobert Duck \and Mac Moneysac}
\professor{Prof. Dr. Kater Karlo}
\date{10.09.2014}
\makecover
\maketitle
\end{Tutorial!}
%
\begin{figure}
\IncludeTutorial{Title}[1,2]
\caption{Umschlagseite und Titel}
\label{fig:title}
\end{figure}



\section{Vor- und Nachspann}
In den folgenden Unterabschnitten werden Elemente vorgestellt, welche häufig 
als Bestandteil einer wissenschaftlichen (Abschluss"~)Arbeit gefordert werden, 
wobei meistens nur eine Teilmenge verlangt wird. Die Platzierung oder Position 
der vorgestellten Elemente innerhalb der Arbeit ist nicht eindeutig durch eine 
Norm oder dergleichen festgelegt. Vielmehr gibt es meist eine Richtlinie vom 
verantwortlichen Prüfungsamt oder eine konkrete Vorgabe des wissenschaftlichen 
Mitarbeiters respektive betreuenden Hochschullehrers.


\subsection{Aufgabenstellung}
\label{sec:task}
Wird eine Abschlussarbeit an der \TnUD geschrieben, kann entweder die Umgebung 
\Environment{task} oder der Befehl \Macro{taskform} für die Erstellung einer 
Aufgabenstellung verwendet werden. Bei beiden Varianten wird zu Beginn eine 
Tabelle mit Informationen zum Autor angegeben und am Ende der oder die Betreuer 
der Arbeit sowie Professor und gegebenenfalls der Prüfungsausschussvorsitzende. 
Dazwischen kann bei der Umgebung \Environment{task} ein freier Inhalt angegeben 
werden. Der Befehl \Macro{taskform} erzeugt eine standardisierte Ausgabe. Dabei 
ist darauf zu achten, dass das zweite obligatorische Argument innerhalb einer 
\Environment*{itemize}"=Umgebung verwendet wird und somit \Macro*{item} zu 
nutzen ist. Das Resultat des folgenden Quelltextes ist in \autoref{fig:task} 
dargestellt.
%
\begin{Tutorial!}{Task}
\faculty{Juristische Fakultät}
\department{Fachrichtung Strafrecht}
\institute{Institut für Kriminologie}
\chair{Lehrstuhl für Kriminalprognose}
\title{%
  Entwicklung eines optimalen Verfahrens zur Eroberung des
  Geldspeichers in Entenhausen
}
\thesis{master}
\graduation[M.Sc.]{Master of Science}
\author{%
  Mickey Mouse
  \matriculationnumber{12345678}
  \dateofbirth{2.1.1990}
  \placeofbirth{Dresden}
  \course{Klinische Prognostik}
  \discipline{Individualprognose}
\and%
  Donald Duck\matriculationnumber{87654321}
  \dateofbirth{1.2.1990}
  \placeofbirth{Berlin}
  \course{Statistische Prognostik}
  \discipline{Makrosoziologische Prognosen}
}
\matriculationyear{2010}
\issuedate{1.2.2015}
\duedate{1.8.2015}
\supervisor{Dagobert Duck \and Mac Moneysac}
\professor{Prof. Dr. Kater Karlo}
\chairman{Prof. Dr. Primus von Quack}
\newcommand\taskcontent{%
  Momentan ist das besagte Thema in aller Munde. Insbesondere wird es
  gerade in vielen~-- wenn nicht sogar in allen~-- Medien diskutiert.
  Es ist momentan noch nicht abzusehen, ob und wann sich diese Situation
  ändert. Eine kurzfristige Verlagerung aus dem Fokus der Öffentlichkeit
  wird nicht erwartet.
  
  Als Ziel dieser Arbeit soll identifiziert werden, warum das Thema
  gerade so omnipräsent ist und wie man diesen Effekt abschwächen
  könnte. Zusätzlich sollen Methoden entwickelt werden, wie sich ein
  ähnlicher Vorgang zukünftig vermeiden ließe.
}
\begin{task}
\smallskip
\par\noindent
\taskcontent
\end{task}
\taskform[pagestyle=empty]{\taskcontent}{%
  \item Recherche
  \item Analyse
  \item Entwicklung eines Konzeptes
  \item Anwendung der entwickelten Methodik
  \item Dokumentation und grafische Aufbereitung der Ergebnisse
}
\end{Tutorial!}
%
\begin{figure}
\IncludeTutorial{Task}[1,2]
\caption{Aufgabenstellung in freier und standardisierter Form}
\label{fig:task}
\end{figure}


\subsection{Zusammenfassung}
Häufig wird zu Beginn einer wissenschaftliche Arbeit die Motivation und der 
Inhalt dieser zusammengefasst, um den Leser die Thematik der Abhandlung 
vorzustellen. in den meisten Fällen wird diese dabei in deutscher und 
englischer Sprache verfasst. Hierfür stellt \KOMAScript{} bereits die Umgebung 
\Environment{abstract} bereit.

Vielfach wird der Wunsch geäußert, sowohl die deutsche als auch die englische 
Zusammenfassung auf derselben Seite zu setzen. Diese Variante kann mithilfe 
der \TUDScript-Klassen sehr einfach umgesetzt werden, wie der nachfolgende 
Quelltextauszug zeigt. Die resultierende Ausgabe ist in \autoref{fig:abstr} zu 
sehen.
%
\begin{Tutorial!}{Abstract}
\TUDoption{abstract}{multiple,section}
\begin{abstract}
  Dies ist der deutschsprachige Teil der Zusammenfassung, in dem die
  Motivation sowie der Inhalt der nachfolgenden wissenschaftlichen
  Abhandlung kurz dargestellt werden.
\nextabstract[english]
  This is the english part of the summary, in which the motivation and
  the content of the following academic treatise are briefly presented.
\end{abstract}
\end{Tutorial!}
%
\begin{figure}
\centering
\IncludeTutorial[width=.5\textwidth]{Abstract}
\caption{Zusammenfassung in deutscher und englischer Sprache}
\label{fig:abstr}
\end{figure}


\subsection{Erklärungen}
Für die meisten Abschlussarbeiten an der \TnUD wird vom Verfasser eine 
Selbstständigkeitserklärung verlangt. Für diese wird ein Standardtext 
bereitgestellt. Dieser kann mit dem Befehl \Macro{confirmation} ausgegeben 
werden. Wurde das Thema in Kooperation mit einem Unternehmen bearbeitet, so 
wird zumeist auch ein Sperrvermerk gefordert, welcher mit \Macro{blocking} 
erzeugt werden kann. Mit \Macro{declaration} werden beide Erklärungen direkt 
nacheinander erzeugt. Die verwendete Überschrift und ein möglicher Eintrag in 
das Inhaltsverzeichnis können über die Option \Option{declaration} reguliert 
werden. Eine mögliche Ausprägung der Erklärungen ist in \autoref{fig:decl} 
abgebildet.
%
\begin{Tutorial!}{Declaration}
\title{%
  Entwicklung eines optimalen Verfahrens zur Eroberung des
  Geldspeichers in Entenhausen
}
\author{Mickey Mouse\and Donald Duck}
\declaration[company=FIRMA]
\end{Tutorial!}
%
\begin{figure}
\centering
\IncludeTutorial[width=.5\textwidth]{Declaration}
\caption{Selbstständigkeitserklärung und Sperrvermerk}
\label{fig:decl}
\end{figure}


\subsection{Inhalts-, Abbildungs-, und Tabellenverzeichnis}
Zum Inhaltsverzeichnis ist nicht allzu viel zu sagen. Dieses wird mit dem
\hologo{LaTeX}"=Standardbefehl \Macro*{tableofcontents} erzeugt und führt die 
Gliederung des erstellten Dokumentes entsprechend der verwendeten Befehle 
(\Macro*{part}, \Macro*{addpart}, \Macro*{chapter}, \Macro*{addchap}, 
\Macro*{section}, \Macro*{addsec} etc.) auf. Wurde das Paket \Package{hyperref} 
geladen, so werden im Inhaltsverzeichnis PDF-Hyperlinks auf die einzelnen 
Abschnitte erzeugt.

Sowohl Abbildungen als auch Tabellen werden in \hologo{LaTeX} normalerweise mit 
speziellen Umgebungen~-- \Environment*{figure} und \Environment*{table}~-- 
eingebunden. Innerhalb dieser sogenannten Gleitumgebungen kann der Befehl 
\Macro*{caption}\OParameter{Verzeichniseintrag}\Parameter{Bezeichnung} genutzt 
werden, um diesen eine Bezeichnung hinzuzufügen. Mit \Macro*{listoffigures} 
beziehungsweise \Macro*{listoftables} lassen sich Verzeichnisse erstellen, in 
denen alle Gleitobjekte des jeweiligen Typs ausgegeben werden, falls diese denn 
eine Bezeichnung hinzugefügt wurde. Sollen Abbildungen oder Tabellen außerhalb 
ihrer angestammten Gleitumgebung genutzt und benannt werden, kann dies mit
\Macro*{captionof}\Parameter{Typ}\OParameter{Verzeichniseintrag}\Parameter{Bezeichnung}
erfolgen. Weitere Informationen hierzu sind der \KOMAScript"=Anleitung 
\scrguide zu entnehmen. Außerdem wird in \autoref{sec:floats} genauer auf die 
Verwendung von Gleitumgebungen eingegangen.
%
\begin{Tutorial*}
\tableofcontents
\listoffigures
\listoftables
\clearpage
\end{Tutorial*}


\subsection{Literaturverzeichnis}
\label{sec:biblatex}
Für das Erstellen eines Literaturverzeichnisses wurde in der Vergangenheit fast 
ausschließlich \hologo{BibTeX} verwendet. Lieder wird auch heute immer noch 
darauf verwiesen, obwohl es seit einigen Jahren das Paket \Package{biblatex} 
gibt, welches insbesondere für neue Dokumente definitiv den Vorzug erhalten 
sollte. Auch die Umstellung älterer \hologo{BibTeX}"=Datenbanken ist mit 
wenigen Handgriffen realisierbar. Für \Package{biblatex} existieren eine Menge 
unterschiedlicher, vordefinierter Zitierstile, welche sich im Vergleich zu 
\hologo{BibTeX} auch wesentlich leichter an die individuellen Bedürfnisse 
anpassen lassen. Ein weiterer Vorteil ist die Unterstützung von Datenbanken, 
welche eine UTF"~8"~Kodierung nutzen, wenn \Application{biber} zur Sortierung 
der Einträge verwendet wird. Welcher Stil und welches Backend zur Sortierung 
genutzt werden soll, lässt sich durch das optionale Argument beim Laden des 
Paketes festlegen.
%
\begin{Tutorial-}
\usepackage[backend=biber,style=alphabetic]{biblatex}
\end{Tutorial-}
%
Die Erstellung einer Literaturdatenbank kann entweder von Hand oder mithilfe 
einer externen Anwendung erfolgen. Für die letztgenannte Variante sind die 
Programme \Application{Citavi} respektive \Application{JabRef} empfehlenswert. 
Eine Einführung in die beiden Anwendungen würde allerdings zu weit führen. 
Deshalb wird nachfolgend eine Literaturdatenbank mit drei Einträgen manuell 
erzeugt, wobei hierfür die \Environment*{filecontents}"=Umgebung verwendet 
wird, mit der innerhalb eines \hologo{LaTeX}"=Dokumentes externe Textdateien 
erstellt werden können. Zuerst die Ausgabe, darunter folgt die Erläuterung:
%
\begin{Tutorial}
\begin{filecontents}{treatise-temp.bib}
@book{goossens94,
  author    = {Goossens, Michel and Mittelbach, Frank
               and Samarin, Alexander},
  title     = {The LaTeX Companion},
  date      = {1994},
  publisher = {Addison-Wesley},
  location  = {Reading, Massachusetts},
  language  = {english},
}
@book{knuth84,
  author    = {Knuth, Donald E.},
  title     = {The \TeX book},
  date      = {1984},
  maintitle = {Computers \& Typesetting},
  volume    = {A},
  publisher = {Addison-Wesley},
  location  = {Reading, Massachusetts},
  language  = {english},
}
@manual{hanisch14,
  author    = {Hanisch, Falk},
  title     = {Ein \LaTeX"=Bundle für Dokumente
               im neuen Corporate Design 
               der Technischen Universität Dresden},
  date      = {2014},
  subtitle  = {Benutzerhandbuch},
  location  = {Dresden},
  language  = {german},
}
\end{filecontents}
\end{Tutorial}
%
Es wurden drei Einträge definiert. Jeder Eintrag einer \PValue{.bib}-Datei 
beginnt mit \PValue{@}\PName{Eintragstyp}. Direkt danach erstes ist für den 
Eintrag ein \emph{eindeutiges} \PParameter{Label} festzulegen. Anschließend 
können für unterschiedliche Felder die dazugehörige Werte eingetragen werden. 
Die verwendbaren Eintragstypen und die für diesen benötigten und zusätzlich 
nutzbaren Felder sind in der Dokumentation von \Package{biblatex} zu finden.

Im einfachsten Fall werden die Einträge mit \Macro*{cite}\Parameter{macro} im 
Dokument referenziert. Für die Referenzierung werden durch \Package{biblatex} 
weitere Befehle angeboten.
%
\begin{Tutorial}
In diesem Textabschnitt werden \cite{knuth84} sowie \cite{goossens94} und 
\cite{hanisch14} zitiert.
\end{Tutorial}
%
Das Literaturverzeichnis wird mit \Macro*{printbibliography} ausgegeben, wobei 
nicht alle Einträge der Literaturdatenbank sondern lediglich die tatsächlich 
referenzierten verwendet werden.
%
\begin{Tutorial*}
\printbibliography[heading=subbibliography]
\end{Tutorial*}
\begin{quoting}[rightmargin=0pt]
\vspace*{-\baselineskip}
\printbibliography[heading=subbibliography]
\end{quoting}


\subsection{Abkürzungs- und Symbolverzeichnis}
\label{sec:glossaries}
Für die Auszeichnung von Abkürzungen gibt es zwei sehr gute Pakete, die dieses 
Unterfangen stark vereinfachen. Die einfachere~-- jedoch nicht so mächtige~-- 
der beiden Varianten ist die Nutzung des Paketes \Package{acro}. Sollen nur 
Abkürzungen und gegebenenfalls eine sortierte Liste dieser gesetzt werden, ist 
dieses allerdings absolut ausreichend. Für ein Symbolverzeichnis lässt sich in 
dieser Variante das Paket \Package{nomencl} nutzen. Dieses bietet meiner 
Meinung nach jedoch keine großen Vorteile, stattdessen kann auch einfach eine 
Tabelle händisch erzeugt werden. Das Paket \Package{acro} ist sehr gut und 
ausführlich dokumentiert. Deshalb wird hier auf eine exemplarische Erläuterung 
verzichtet und stattdessen auf dessen Dokumentationen verwiesen.

Die andere Möglichkeit ist die Nutzung des Paketes \Package{glossaries}, das 
eine große Zahl an Einstellmöglichkeiten und Optionen besitzt, allerdings auch 
etwas Zeit für die Einarbeitung und Studium der Dokumentation benötigt. Der 
ursprüngliche Einsatzzweck dieses Paketes ist das Setzen eines fachsprachlichen 
oder technischen Glossars. Es bietet zusätzlich die Mittel zum Erzeugen eines 
Abkürzungs- sowie Symbolverzeichnis. Die Dokumentation von \Package{glossaries} 
lässt ebenfalls keine Wünsche offen. Dennoch soll folgend hier kurz erläutert 
werden, wie das Paket zu verwenden ist. Für weiterführende Beispiele sollte 
die Dokumentationen zu Rate gezogen werden. In \autoref{sec:introduction} wurde 
das Paket \Package{glossaries} bereits geladen. Die hierfür genutzten Optionen 
werden kurz erläutert.
%
\begin{Tutorial-}
\usepackage[%
\end{Tutorial-}
%
Das Programm \Application{makeindex} wird im Normalfall durch die genutzte 
\hologo{LaTeX}"=Distribution bereitgestellt und für das alphabetische Sortieren 
der erstellten Listen verwendet. Mit der Paketoption \Option*{automake} erfolgt 
der automatische Aufruf von \Application{makeindex} mit den passenden 
Einstellungen für alle Verzeichnisse. Alternativ dazu kann die Option 
\Option{xindy} aktiviert werden, welche \Application{xindy} anstelle von 
\Application{makeindex} für das Sortieren verwendet. Diese Programm bietet 
unter anderem eine Unterstützung von Unicode sowie die Möglichkeit, nach 
sprachabhängigen Regeln zu sortieren. Allein für die deutsche Sprache gibt es 
beispielsweise zwei verschiedene Varianten~-- nach DIN und nach Duden~-- zum 
alphabetischen Sortieren. Allerdings wird das Programm \Application{xindy} 
lediglich mit der Distribution \Distribution{\hologo{TeX}~Live} jedoch nicht 
mit \Distribution{\hologo{MiKTeX}} geliefert. Wenn ohnehin die Distribution 
\Distribution{\hologo{TeX}~Live} verwendet wird, würde ich persönlich die 
Verwendung von \Application{xindy} vorziehen. Für das Erstellen der Glossare 
sollte das Perl"=Skript \Application{makeglossaries} verwendet werden, welches 
alle notwendigen Optionen an \Application{xindy} weiterleitet. Treten Probleme 
bei der Erzeugung der einzelnen Glossare auf, sollte die Dokumentation von 
\Package{glossaries} weiterhelfen können.
%
\begin{Tutorial-}
  automake,%
%  xindy,\%={language=german-din},\% mit Tex Live einfach verwendbar
\end{Tutorial-}
%
Die Optionen \Option*{acronym} und \Option*{symbols} erzeugen die Glossare 
beziehungsweise die Verzeichnisse für Abkürzungen und Symbole. Die Option  
\Option*{nomain} wird verwendet, weil in diesem Tutorial kein zusätzliches 
technisches Glossar erzeugt werden soll.
%
\begin{Tutorial-}
  acronym,% Abkürzungen
  symbols,% Formelzeichen
  nomain,% kein Glossar
\end{Tutorial-}
%
Mit der Option \Option*{translate}[babel] werden die Überschriften der Glossare 
in der Dokumentsprache gesetzt, mit \Option*{nogroupskip} kann der automatische 
Abstand zwischen den Einträgen zur Gruppierung innerhalb eines Glossars 
entfernt werden. Die Option \Option*{toc} fügt die Verzeichnisse dem 
Inhaltsverzeichnis hinzu, mit \Option*{section} kann die Gliederungsebene für 
die Überschrift angegeben werden.
%
\begin{Tutorial-}
  translate=babel,%
  nogroupskip,%
  toc,%
  section=section,%
\end{Tutorial-}
%
Damit sind alle verwendeten Optionen erläutert. Schließlich sorgt der Befehl 
\Macro*{makeglossaries} für das Erstellen der optionsabhängigen Stildateien für 
\Application{makeindex} respektive \Application{xindy} sowie das Erzeugen der 
benötigten Hilfsdateien.
%
\begin{Tutorial-}
]{glossaries}
\makeglossaries
\end{Tutorial-}
%
Damit wäre der erste Teil zur Initialisierung überstanden und wir können zum 
eigentlichen Problem kommen. Wie erstellt man nun ein Abkürzungs- und/oder 
Symbolverzeichnis?

\subsubsection{Abkürzungsverzeichnis}
Das Paket \Package{glossaries} stellt für die Definition von Abkürzungen einen 
speziellen Befehl bereit. Mit\Macro*{newacronym}\LParameter\Parameter{Label}%
\Parameter{Abkürzung}\Parameter{Wortgruppe} wird eine Abkürzung definiert und 
kann später über \Parameter{Label} genutzt werden. Die möglichen optionalen 
Parameter können in der Dokumentation zu \Package{glossaries} nachgeschlagen 
werden. Für ein kleines Beispiel werden drei Abkürzungen erstellt\dots
%
\begin{Tutorial}
\newacronym{apsp}{APSP}{All-Pairs Shortest Path}
\newacronym{spsp}{SPSP}{Single-Pair Shortest Path}
\newacronym{sssp}{SSSP}{Single-Source Shortest Path}
\end{Tutorial}
%
\dots und diese in einer kurzen Textpassage mit \Macro*{gls}\Parameter{Label} 
verwendet.
%
\begin{Tutorial}
In der Graphentheorie wird häufig die Lösung des Problems des kürzesten
Pfades zwischen zwei Knoten gesucht. Dieses Problem wird häugig auch
mit \gls{spsp} bezeichnet. Es lässt sich auf die Variationen \gls{sssp}
und \gls{apsp} erweitern. Für die Lösung von \gls{spsp}, \gls{sssp} oder 
\gls{apsp} kommen unterschiedliche Algorithmen zum Einsatz.
\end{Tutorial}
%
Gut zu sehen ist, dass sich die Ausgabe der Abkürzung bei der ersten Verwendung 
mit \Macro*{gls} von der zweiten~-- und jeder weiteren~-- unterscheidet. Das 
Verhalten lässt sich über verschiedene Stile mit \Macro*{setacronymstyle} 
anpassen. Die Ausgabe einer Liste aller Abkürzungen erfolgt mit:
%
\begin{Tutorial-}
\printacronyms
\end{Tutorial-}
\begin{quoting}[rightmargin=0pt]
\vspace*{-\baselineskip}
\glsdisablehyper
\printacronyms
\end{quoting}
%
Dabei werden die Abkürzungen in einer \Environment*{description}"=Umgebung 
gesetzt. Dies ist absolut ausreichend. Mir persönlich ist die Darstellung in 
einer quasi-tabellarischen Form jedoch lieber. Dabei soll eigentliche Stil mit 
fettgedruckter Abkürzung beibehalten werden. Das \Package{glossaries}"=Paket 
stellt zwar auch eine Menge Stilen in Tabellenform bereit, allerdings nicht in 
dem gewünschten Stil. Deshalb wird hier gezeigt, wie ein eigener Stil kreiert 
werden kann. Dafür wird der Befehl \Macro*{newglossarystyle} verwendet. Die 
Definition des neuen Stils \PValue{acronymstabu} wird nachfolgend ausgegeben, 
die Erläuterung dazu schließt sich daran an:
%
\TutorialHook{\let\newglossarystyle\renewglossarystyle}
\begin{Tutorial}
\newglossarystyle{acronymstabu}{%
  \renewenvironment{theglossary}{%
    \begin{tabu}spread 0pt{@{}lX<{\strut}l@{}}%
  }{%
    \end{tabu}\par\bigskip%
  }%
  \renewcommand*{\glossaryheader}{}%
  \renewcommand*{\glsgroupheading}[1]{}%
  \renewcommand*{\glsgroupskip}{}%
  \renewcommand*{\glossentry}[2]{%
    \glsentryitem{##1}% Entry number if required
    \glstarget{##1}{\sffamily\bfseries\glossentryname{##1}} &
    \glsentrydesc{##1} &
    ##2\tabularnewline
  }
}
\end{Tutorial}
%
Als erstes wird die Umgebung \Environment*{theglossary} umdefiniert, wobei 
innerhalb dieser \Environment*{tabu} aus dem Paket \Package{tabu} verwendet 
wird. Dabei werden drei Spalten definiert. Die erste und letzte Spalte sind 
schlicht linksbündig~(\PValue{l}). In diesen werden die Abkürzung selbst 
beziehungsweise die Seitenzahl eingetragen. Die Verwendung von~\PValue{@\{\}} 
führt dazu, dass der normalerweise vor der ersten und nach der letzten Spalte 
eingefügte Abstand von \Length*{tabcolsep} entfällt. Die mittlere Spalte vom 
Typ~\PValue{X} verhält sich prinzipiell wie der gleichnamige Spaltentyp aus dem 
Paket \Package{tabularx}. Allerdings gibt es hier eine Besonderheit.

Für \Environment*{tabularx}"=Tabellen muss generell eine feste Tabellenbreite 
angegeben werden. Die Breite der \PValue{X}"~Spalten wird anhand dieser und 
dem bereits für andere Spalten vom Typ~\PValue{l},~\PValue{r}~und~\PValue{c} 
benötigten Platz berechnet. Für \Environment*{tabu}"=Tabellen kann anstelle 
einer festen Breite auch \PValue{spread 0pt} angegeben werden. Dadurch werden 
Spalten vom Typ~\PValue{X} anfänglich in ihrer natürlichen Breite gesetzt. 
Sobald jedoch die Gesamtbreite der Tabelle die Seitenbreite überschreiten 
würde, werden die \PValue{X}"~Spalten automatisch umbrochen. Geschieht dies 
tatsächlich, gibt es beim Paket \Package{tabu} jedoch ein kleineres Problem. 
In umbrochenen Spalten setzt \Package{tabu} zu wenig vertikalen Freiraum am 
unteren Ende. Um dies zu beheben wird am Schluss jeder \PValue{X}"~Spalte mit
\PValue{X<\{\Macro*{strut}\}} einfach der Befehl \Macro*{strut} angehängt, der 
vertikalen Freiraum ober- und unterhalb der aktuellen Grundlinie einfügt.

Der Rest des Stils ist recht schnell erläutert. Zunächst wird auf Tabellenköpfe 
sowie Gruppierungen (Abstände und Überschriften) verzichtet. Schließlich ist 
der Befehl \Macro*{glossentry} verantwortlich für die Formatierung der Einträge 
im Abkürzungsverzeichnis. Dieser wird intern durch \Package{glossaries} mit 
zwei Argumenten aufgerufen. Das erste enthält das entsprechende Label, das 
zweite ein kommaseparierte Liste der Seitenzahlen. Dabei stehen verschiedene 
Makros zur Auswahl, um anhand eines Labels die gewünschte Information zu 
extrahieren.%
\footnote{%
  bspw. mit \Macro*{glossentryname} die Bezeichnung oder mit 
  \Macro*{glsentrydesc} die dazugehörige Beschreibung%
}
Der Befehl \Macro*{glossentry} so definiert, dass für jeden Eintrag eine Zeile 
in der Tabellen erzeugt wird, wo in der ersten Spalte die Abkürzung selbst, in 
der zweiten die Langform und in der dritten Spalte schließlich die Liste der 
Seiten, auf welchen die jeweilige Abkürzung mit \Macro*{gls}\Parameter{Label} 
verwendet wurde, ausgegeben wird. Zum Abschluss die resultierende Ausgabe des 
neuen Stils.
%
\begin{Tutorial*}
\printacronyms[style=acronymstabu]
\end{Tutorial*}
\begin{quoting}[rightmargin=0pt]
\vspace*{-\baselineskip}
\printacronyms[type=acronym,style=acronymstabu]
\end{quoting}

\subsubsection{Symbolverzeichnis}
Für das Erzeugen eines Symbolverzeichnisses kann ebenfalls \Package{glossaries} 
verwendet werden. Allerdings muss hierfür ein wenig mehr Aufwand getrieben 
werden, da das Paket hierfür keine dedizierte Schnittstelle bereitstellt. Wurde 
die Paketoption \Option*{symbols} angegeben wird jedoch zumindest das dazu 
notwendige Glossar erstellt.

Als erstes sollte ein gut nutzbarer Befehl zum Definieren eines neuen Symbols 
erstellt werden. In Anlehnung an den Befehl für Abkürzungen wird dieser 
\Macro*{newsymbol} genannt. Dieser hat ein optionales und vier obligatorische 
Argumente, wobei das optionale Argument prinzipiell alle Schlüssel-Wert-Paare 
enthalten kann, die durch \Package{glossaries} akzeptiert werden. Welche davon 
letztlich auch Auswirkungen haben, hängt allerdings von der Gestaltung des 
Stils durch den Anwender ab. Der nachfolgend definierte Befehl hat folgende 
Gestalt:
%
\par\vspace{\baselineskipglue}\noindent
\Macro*{newsymbol}\LParameter\Parameter{Label}\Parameter{Bezeichnung}%
\Parameter{Symbol}\Parameter{Einheit}
\par\vspace{\baselineskipglue}\noindent
%
Mit \PName{Label} erfolgt die eindeutige Kennzeichnung des Symbols. Außerdem 
wird dies für die Sortierung verwendet, was unter Umständen etwas problematisch 
sein könnte. Eine manuelle Festlegung für den dazugehörigen Schlüssel durch den 
Anwender über das optionale Argument mit \PValue{sort=}\PName{Bezeichnung} ist 
eventuell sinnvoll. Nach dem \PName{Label} folgt die \PName{Bezeichnung} für 
das Formelzeichen beziehungsweise das \PName{Symbol} sowie die dazugehörige 
physikalische \PName{Einheit}.
%
\TutorialHook{\let\newcommand\renewcommand}
\begin{Tutorial}
\newcommand*{\newsymbol}[5][]{%
  \newglossaryentry{#2}{%
    type=symbols,%
    description={},%
    name={#3},%
    symbol={\ensuremath{#4}},%
    user1={\ensuremath{\mathrm{#5}}},%
    sort={#2},%
    #1%
  }%
}
\end{Tutorial}
%
Da es sich zumeist um mathematische Symbole handelt, wird für das Symbol und 
die Einheit mit \Macro*{ensuremath} sorge getragen, dass diese auch im 
Textmodus ohne Probleme verwendet werden können. 
Für ein kleines Beispiel werden fünf Formelzeichen definiert\dots
%
\begin{Tutorial}
\newsymbol{l}{Länge}{l}{m}
\newsymbol{m}{Masse}{m}{kg}
\newsymbol{a}{Beschleunigung}{a}{\nicefrac{m}{s^2}}
\newsymbol{t}{Zeit}{t}{s}
\newsymbol{f}{Frequenz}{f}{s^{-1}}
\newsymbol{F}{Kraft}{F}{m \cdot kg \cdot s^{-2}=\nicefrac{J}{m}}
\end{Tutorial}
%
\dots und diese in einer kurzen Textpassage mit \Macro*{gls}\Parameter{Label} 
verwendet.
%
\begin{Tutorial'}
Die Einheiten für die \gls{f} sowie die \gls{F} werden aus den 
SI"=Einheiten der Basisgrößen \gls{l}, \gls{m} und \gls{t} abgeleitet.
Und dann gibt es noch die Grundgleichung der Mechanik, welche für den
Fall einer konstanten Kraftwirkung in die Bewegungsrichtung einer
Punktmasse lautet:
$\gls{F} = \gls{m} \cdot \gls{a}$
\end{Tutorial'}
%
Damit wären zumindest die Symbole definiert. Allerdings ist noch nicht 
festgelegt, wie genau die Ausgabe des Symbolverzeichnisses aussehen soll. 
Momentan erzeugt der Befehl \Macro*{printsymbols} jedenfalls kein sinnvolles 
Verzeichnis:
%
\begin{Tutorial-}
\printsymbols
\end{Tutorial-}
\begin{quoting}[rightmargin=0pt]
\vspace*{-\baselineskip}
\glsdisablehyper
\printsymbols
\end{quoting}
%
Für dieses muss erst ein Stil definiert werden, was hier ähnlich zum Stil 
\PValue{acronymstabu} geschieht. Einziger Unterschied hier ist, neben der 
Anzahl der Spalten, dass eine \Environment*{longtabu}"=Umgebung verwendet wird. 
Damit diese linksbündig im Text erscheint muss vor dem obligatorischen Argument 
mit den Spaltendefinitionen noch das optionale Argument \PValue{l} angegeben 
werden.
%
\TutorialHook{\let\newglossarystyle\renewglossarystyle}
\begin{Tutorial}
\newglossarystyle{symbolslongtabu}{%
  \renewenvironment{theglossary}{%
    \begin{longtabu}spread 0pt[l]{ccX<{\strut}l}%
  }{%
    \end{longtabu}%
  }%
  \renewcommand*{\glossaryheader}{%
    \toprule
    \bfseries Symbol & \bfseries Einheit &
    \bfseries Name & \bfseries Seite(n)
    \tabularnewline\midrule\endhead%
    \bottomrule\endfoot%
  }%
  \renewcommand*{\glsgroupheading}[1]{}%
  \renewcommand*{\glsgroupskip}{}%
  \renewcommand*{\glossentry}[2]{%
    \glsentryitem{##1}% Entry number if required
    \glstarget{##1}{\glossentrysymbol{##1}} &
    \glsentryuseri{##1} &
    \glossentryname{##1} &
    ##2\tabularnewline%
  }%
}
\end{Tutorial}
%
Innerhalb von \Macro*{newglossarystyle} wird \Macro*{glossaryheader} für einen
Tabellenkopf definiert, wie er auch für eine \Environment*{longtable}"=Umgebung 
erscheinen würde. In hier vorgestellten Fall werden Kopf und Fuß mit 
\Macro*{endhead} und \Macro*{endfoot} terminiert. Diese werden beim einem 
möglichen Seitenumbruch zu Beginn und am Ende auf jeder Seite gesetzt.

Damit man die Symbole auch wirklich sinnvoll nutzen kann, sollte das 
Erscheinungsbild der Einträge mit \Macro*{defglsentryfmt} angepasst werden.
%
\begin{Tutorial}
\defglsentryfmt[symbols]{%
  \ifmmode%
    \glssymbol{\glslabel}%
  \else%
    \glsgenentryfmt~\glsentrysymbol{\glslabel}%
  \fi%
}
\end{Tutorial}
%
Diese Definition führt dazu, dass bei der Verwendung eines Symbols mit
\Macro*{gls}\Parameter{Label} im Text diesem die Bezeichnung vorangestellt 
wird, im Mathematikmodus allerdings allein das Symbol verwendet wird. Das 
nachfolgende Beispiel macht dies deutlich.
%
\begin{Tutorial}
Die Einheiten für die \gls{f} sowie die \gls{F} werden aus den
SI"=Einheiten der Basisgrößen \gls{l}, \gls{m} und \gls{t} abgeleitet.
Und dann gibt es noch die Grundgleichung der Mechanik, welche für den
Fall einer konstanten Kraftwirkung in die Bewegungsrichtung einer
Punktmasse lautet:
$\gls{F} = \gls{m} \cdot \gls{a}$
\end{Tutorial}
%
Das Symbolverzeichnis kann sich nun durchaus sehen lassen.
%
\begin{Tutorial*}
\printsymbols[style=symbolslongtabu]
\end{Tutorial*}
\begin{quoting}[rightmargin=0pt]
\newglossarystyle{symbolstabu}{%
  \setglossarystyle{symbolslongtabu}%
  \renewenvironment{theglossary}{%
    \begin{tabu}spread 0pt{ccX<{\strut}l}%
  }{%
    \bottomrule\end{tabu}%
  }%
  \renewcommand*{\glossaryheader}{%
    \toprule
    \bfseries Symbol & \bfseries Einheit &
    \bfseries Name & \bfseries Seite(n)
    \tabularnewline\midrule%
  }%
}
\printsymbols[style=symbolstabu]
\end{quoting}


\section{Anfangszitat oder Schlauer Spruch}
\dictum[Johann Wolfgang von Goethe]{Es irrt der Mensch, solang er strebt.}
\bigskip\noindent
Oftmals möchte der Autor einer wissenschaftlichen Arbeit für das erste oder 
auch jedes Kapitel ein Zitat oder ähnliches voranstellen. Dies kann mit dem 
Befehl \Macro*{dictum}\OParameter{Autor}\Parameter{Text} erfolgen. Damit wird 
der im obligatorischen Argument angegeben Ausspruch in einer \Macro*{parbox} 
ausgegeben. Das optionale Argument kann für die Angabe des Autors verwendet 
werden. Soll das Ganze für einen Teil oder ein Kapitel erfolgen, sollte der 
Befehl \Macro*{dictum} innerhalb von \Macro*{setpartpreamble} beziehungsweise 
\Macro*{setchapterpreamble} verwendet werden. Genaueres hierzu und zu den 
Möglichkeiten, die Gestalt des Zitats zu beeinflussen, ist in der \scrguide zu 
finden.
\begin{Tutorial*}
\setchapterpreamble{%
  \dictum[Johann Wolfgang von Goethe]{%
    Es irrt der Mensch, solang er strebt.%
  }%
}
\chapter{Einleitung}
\end{Tutorial*}


\section{Erstellen von Abbildungen}
\label{sec:figures}


\subsection{tikz}
\begin{Tutorial*}
\begin{figure}
\begin{tikzpicture}
  \newlength{\tikzunit}
  \setlength{\tikzunit}{.01\textwidth}
  \tikzset{x=\tikzunit,y=\tikzunit}
  \tikzstyle{inner box}=[%
    text width=18\tikzunit,
    align=center,
    rectangle,
    inner sep=0pt,
    minimum height=8\tikzunit,
    font=\hspace{0pt},
    draw
  ]
  \tikzstyle{inner label}=[align=center, font=\scriptsize]
  \tikzstyle{inner box chain}=[every node/.style={on chain}]
  \tikzstyle{inner box chain below}=[%
    inner box chain, node distance=8\tikzunit,continue chain=going below
  ]
  \tikzstyle{inner box chain right}=[%
    inner box chain,node distance=35\tikzunit,continue chain=going right
  ]
  \tikzstyle{inner box chain above}=[%
    inner box chain,node distance=16\tikzunit,continue chain=going above
  ]
  \tikzstyle{pstarrow->}=[%
    decoration={markings,
      mark=at position 1 with {\arrow[xscale=1.5]{stealth}};
    },
    postaction={decorate},
    shorten >=0.7pt
  ]
  \newcommand\tikzparbox[2][9]{%
    \parbox{#1\tikzunit}{\centering\hspace{0pt}#2}%
  }
  \begin{scope}[start chain]
    \begin{scope}[inner box chain below]
      \node(NE)[inner box]{Navigations\-ebene};
      \node(NB)[inner label]{gewählte Fahrtroute\\zeitlicher Ablauf};
      \node(BE)[inner box]{{Bahnführungs\-ebene}};
      \node(BS)[inner label]{%
        gewählte Führungsgrößen:\\Sollspur, Sollgeschwindigkeit%
      };
      \node(SE)[inner box]{Stabilisierungs\-ebene};
    \end{scope}
    \begin{scope}[inner box chain right]
      \node(LQ)[inner box]{Längs- und Querdynamik};
      \node(FO)[inner box]{Fahrbahn\-oberfläche};
    \end{scope}
    \begin{scope}[inner box chain above]
      \node(FR)[inner box]{Fahrraum\\\smallskip{\scriptsize Straße 
      und\\\vspace{-1.5ex}Verkehrssituation}};
      \node(SN)[inner box]{Straßennetz};
    \end{scope}
  \end{scope}
  \begin{scope}[inner label,minimum size=0pt]
    \draw [pstarrow->] (FO) -| ++(13.5,-12) to
      node [above]{Istspur, Istgeschwindigkeit} ++(-97,0) |- (SE);
    \draw [pstarrow->] (FR) -| ++(14  ,-32) to 
      node [above]{Bereich sicherer Führungsgrößen} ++(-98,0) |- (BE);
    \draw [pstarrow->] (SN) -| ++(14.5  ,-52) to 
      node [above]{mögliche Fahrtroute} ++(-99,0) |- (NE);
  \end{scope}
  \begin{scope}[inner label]
    \draw              (NE) to (NB);
    \draw [pstarrow->] (NB) to (BE);
    \draw              (BE) to (BS);
    \draw [pstarrow->] (BS) to (SE);
    \draw [pstarrow->] (SE) to
      node[above] {\tikzparbox{Stell\-größen}}
      node[below] {\tikzparbox{Lenken Gasgeben Bremsen}}
    (LQ);
    \draw [pstarrow->] (LQ) to
      node[above]{\tikzparbox{Regel\-größen}}
      node[below]{\tikzparbox{Fahrzeugbewegung}}
    (FO);
    \draw [pstarrow->] (LQ)+(24,0) |- (FR);
    \draw [pstarrow->] (LQ)+(24,0) |- (SN);
  \end{scope}
  \begin{scope}[very thick,rounded corners=5\tikzunit]
    \draw (-12.5,-40) rectangle (12.5,14);
    \draw ( 22.5,-40) rectangle (47.5,-18);
    \draw ( 57.5,-40) rectangle (82.5,14);
  \end{scope}
  \begin{scope}[font=\bfseries]
    \node at (0,9) {Fahrer};
    \node at (35,-23) {Fahrzeug};
    \node at (70,9) {Umwelt};
  \end{scope}
\end{tikzpicture}
\caption{Eine mit TikZ erstellte Grafik}\label{fig:tikz}
\end{figure}
\end{Tutorial*}
\InputTutorial

\subsection{pstricks}
\begin{Tutorial*}
\begin{figure}
\psset{%
  unit=.01\textwidth,%
  cornersize=absolute,%
  labelsep=.8ex,%
  linewidth=.4pt,%
  arrowscale=1.5,%
}
\begin{pspicture}(0,-2)(100,64)
\newcommand\fnodetext{}
\def\fnodetext(#1)#2#3{%
  \fnode[framesize=18 8](#1){#2}%
  \rput(#1){\parbox{17\psunit}{\centering\hspace{0pt}#3}}%
}
\newcommand\scriptbox[2][24]{%
  \parbox{#1\psunit}{\scriptsize\centering\hspace{0pt}#2}%
}
\rput(15,10){%
  \rput(0,49){\textbf{Fahrer}}
  \fnodetext(0,40){NE}{Navigations\-ebene}
  \fnodetext(0,24){BE}{Bahnführungsebene}
  \fnodetext(0,08){SE}{Stabilisierungsebene}
  \ncline{->}{NE}{BE}
  \ncput*{\scriptbox{gewählte Fahrtroute\\zeitlicher Ablauf}}
  \ncline{->}{BE}{SE}
  \ncput*{%
    \scriptbox{gewählte Führungsgrößen:\\Sollspur,~Sollgeschwindigkeit}%
  }
  \psframe[dimen=middle,linewidth=1.2pt,linearc=5](-12.5,0)(12.5,54)
}
\rput(50,10){%
  \rput(0,17){\textbf{Fahrzeug}}
  \fnodetext(0,8){FZ}{Längs- und\\Querdynamik}
  \psframe[dimen=middle,linewidth=1.2pt,linearc=5](-12.5,0)(12.5,22)
}
\rput(85,10){%
  \rput(0,49){\textbf{Umwelt}}
  \fnodetext(0,40){SN}{Straßennetz}
  \fnodetext(0,24){FR}{%
    Fahrraum\\\smallskip\scriptsize{Straße und Verkehrssituation}%
  }
  \fnodetext(0,08){FO}{Fahrbahn\-oberfläche}
  \psframe[dimen=middle,linewidth=1.2pt,linearc=5](-12.5,0)(12.5,54)
}

\ncline{->}{SE}{FZ}
\naput{\scriptbox[9]{Stell\-größen}}
\nbput{\scriptbox[9]{Lenken Gasgeben Bremsen}}
\ncline{->}{FZ}{FO}
\naput{\scriptbox[9]{Regel\-größen}}
\nbput{\scriptbox[9]{Fahrzeugbewegung}}

\psset{armA=15,armB=0,angleA=0,angleB=180}
\ncangles{->}{FZ}{FR}
\ncangles{->}{FZ}{SN}

\psset{angleA=180,angleB=0}
\ncloop[loopsize=12,arm=4.5]{<-}{SE}{FO}
\naput{\scriptbox{Istspur, Istgeschwindigkeit}}
\ncloop[loopsize=32,arm=5]{<-}{BE}{FR}
\naput{\scriptbox[30]{Bereich sicherer Führungsgrößen}}
\ncloop[loopsize=52,arm=5.5]{<-}{NE}{SN}
\naput{\scriptbox{mögliche Fahrtroute}}
\end{pspicture}
\caption{Eine mit pstricks erstellte Grafik}\label{fig:pstricks}
\end{figure}
\end{Tutorial*}
\InputTutorial


\section{Gleitumgebungen}
\label{sec:floats}


\subsection{Grafiken und Untergrafiken}
\label{sec:graphics}


\subsection{Tabellen}
\label{sec:tables}


\section{Listen}


\section{Mathematiksatz}


\section{Typographie}

\subsection{Abkürzungen}

\subsection{Einheiten}


\section{Querverweise}


\section{Silbentrennung}


\section{Quelltexte}
%
%\begin{Tutorial-}
%\usepackage{xparse}% §§§ bereits geladen
%\usepackage{microtype}% §§§ bereits geladen
%\usepackage{textcomp}% §§§ bereits geladen
%\usepackage{setspace}% §§§ bereits geladen
%\usepackage{csquotes}% §§§ bereits geladen
%\usepackage{quoting}% §§§ bereits geladen
%\usepackage{isodate}% §§§ bereits geladen
%\usepackage{listings}% §§§ bereits geladen
%\usepackage{xspace}% §§§ bereits geladen
%\usepackage{scrhack}% §§§ bereits geladen
%\end{Tutorial-}
%%
%\begin{Tutorial-}
%\usepackage{caption}
%\usepackage{floatrow}
%\usepackage{tikz}
%\usepackage{biblatex}
%\usepackage{array}
%\usepackage{booktabs}
%\usepackage{tabularx}
%\usepackage{ltxtable}
%\usepackage{tabu}
%\usepackage{enumitem}
%\setlist{noitemsep}
%\usepackage{mathtools}
%\usepackage{units}
%\usepackage{siunitx}
%\usepackage{xpunctuate}
%\usepackage{ellipsis}
%\usepackage{varioref}
%\usepackage{chngcntr}
%\usepackage{bookmark}
%\end{Tutorial-}
%%
Damit ist das Tutorial beendet.
\begin{Tutorial*}
\end{document}
\end{Tutorial*}

  \begin{itemize}
  \item grafiken und untergrafiken?
  \item tabellen
  \item literaturverzteichnis
  \item floatrow
  \item mathematikmodus
  \item tabbing-umgebung
  \end{itemize}
\ListOfToDo
\ToDo[imp]{Einarbeiten von paragraph in treatise}[v2.02]

\FinishTutorial[%
  Um das im kopierten Beispiel erstellte Literaturverzeichnis in das Dokument 
  einbinden zu können, bedarf es dem einmaligen Aufruf von \Application{biber}. 
  nach dem ersten Durchlauf von \hologo{pdfLaTeX}. Dies erfolgt mit dem Aufruf 
  \PValue{biber}~\PName{Dateiname}. Danach ist ein weiteres mal die Verwendung 
  von \PValue{pdflatex}~\PName{Dateiname} notwendig.

  Für das Erstellen von Abkürzungs- und Symbolverzeichnis sollte das mehrmalige 
  Ausführen von \PValue{pdflatex}~\PName{Dateiname} vollkommen ausreichen. In 
  diesem Fall werden die Einträge mit \Application{makeindex} sortiert. Soll 
  stattdessen \Application{xindy} die Sortierung durchführen, muss beim Laden 
  von \Package{glossaries} die entsprechende Paketoption aktiviert werden. Im 
  einfachsten Fall sollte nach der Verwendung von \hologo{pdfLaTeX} der Aufruf 
  des Perl"=Skriptes \PValue{makeglossaries}~\PName{Dateiname} erfolgen. Dies 
  ist ohne weiteres Zutun nur mit \Distribution{\hologo{TeX}~Live} möglich.
]
\end{document}
