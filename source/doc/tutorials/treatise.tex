\RequirePackage[ngerman=ngerman-x-latest]{hyphsubst}
\documentclass[english,ngerman]{tudscrartcl}
\usepackage{selinput}
\SelectInputMappings{adieresis={ä},germandbls={ß}}
\usepackage[T1]{fontenc}
\usepackage{lmodern}
\usepackage{tudscrman}
\lstset{%
  inputencoding=utf8,extendedchars=true,
  literate=%
    {ä}{{\"a}}1 {ö}{{\"o}}1 {ü}{{\"u}}1
    {Ä}{{\"A}}1 {Ö}{{\"O}}1 {Ü}{{\"U}}1
    {~}{{\textasciitilde}}1 {ß}{{\ss}}1
}
\usepackage{tudscrsupervisor}
\usepackage{tabu}
\AfterPackage*{hyperref}{%
  \usepackage[%
    automake,%
  %  xindy,%={language=german-din},%
    acronym,% Abkürzungen
    symbols,% Formelzeichen
    nomain,% kein Glossar
    translate=babel,%
    nogroupskip,%
  ]{glossaries}
  \setStyleFile{\jobname-temp}
  \makeglossaries
}
\usepackage{units}

%\usepackage{tocstyle}
%\newtocstyle[KOMAlike]{compressed}{
%    \settocfeature[1]{entryvskip}{.5em plus 1pt}%  % Die Stellschraube
%}
%\usetocstyle{compressed}

\begin{document}
\TUDoptions{cdfont=no}
\KOMAoptions{headings=normal}
\title{%
  Ein Anwenderleitfaden für das Erstellen einer wissenschaftlichen Abhandlung%
}
\author{Falk Hanisch}
\date{10.09.2014}
\makeatletter
\begingroup%
  \def\and{, }%
  \let\thanks\@gobble%
  \let\footnote\@gobble%
  \hypersetup{%
    pdfauthor = {\@author},%
    pdftitle = {\@title},%
    pdfsubject = {Tutorial für \hologo{LaTeXe}},%
    pdfkeywords = {LaTeX, \TUDScript, Tutorial, Anwenderleitfaden},%
  }%
\endgroup%
\makeatother
\StartTutorial[%
  \begin{abstract}\noindent
  Der Versuch, ein allumfassendes Tutorial für eine wissenschaftliche Arbeit 
  zur Verfügung zu stellen gleicht der beschwerlichen Suche nach einer 
  eierlegenden Wollmilchsau. Es ist quasi nicht möglich, alle möglichen 
  Anforderungen an eine wissenschaftliche Arbeit in einem Dokument abzudecken. 
  Dennoch soll hier versucht werden, einen Großteil der für gewöhnlich 
  auftretenden Erfordernisse zu bearbeiten.
  
  Dieses Tutorial hat \emph{nicht} die Intention, \hologo{LaTeX}-Einsteigern 
  sämtliche Grundlagen zu erläutern. Vielmehr wird davon ausgegangen, das der 
  Leser die ersten Versuche mit \hologo{LaTeXe} bereits hinter sich hat. 
  Dennoch wird versucht, alle Schritte möglichst leicht nachvollziehbar zu 
  gestalten.
  
  Für absolute Neueinsteiger gibt es einige freie Tutorials, welche die ersten 
  Schritte mit \hologo{LaTeXe} stark erleichtern. Sehr empfehlenswert ist die 
  ausführliche \hrfn{http://www.fadi-semmo.de/latex/workshop/}{Workshop-Reihe} 
  von  Fadi~Semmo. Außerdem stellt Nicola~L.~C.~Talbot sehr gute Tutorials für 
  \hrfn{http://www.dickimaw-books.com/latex/novices/}{\hologo{LaTeX}-Novizen} 
  sowie \hrfn{http://www.dickimaw-books.com/latex/thesis/}{Dissertationen} zur 
  freien Verfügung.
  
  In erster Linie ist dieser Leitfaden für Anwender gedacht, die für ihre
  wissenschaftliche Arbeit eine \TUDScript"=Dokumentklasse verwenden wollen. 
  Das vorgestellte Vorgehen kann jedoch~-- natürlich mit gewissen Abstrichen~-- 
  auch mit anderen Klassen, insbesondere denen aus dem \KOMAScript"=Bundle, 
  umgesetzt werden. Viele der verwendeten Optionen und Befehle werden hier nur 
  sporadisch in ihrer Grundfunktion erläutert. Eine detaillierte Erläuterung 
  lässt sich jedoch jederzeit sehr einfach über die farbigen Hyperlinks zu den 
  entsprechenden Stellen im \href{run:../tudscr.pdf}{\TUDScript-Handbuch} 
  öffnen.
  
  Der Anwenderleitfaden muss nicht zwingend vollständig nachvollzogen werden. 
  Dieser ist in einzelne Abschnitte untergliedert, damit der Anwender sich 
  bestimmte Aspekte erarbeiten kann. Sollten Querbezüge zu den einzelnen 
  Abschnitten bestehen, werden diese auch genannt. Zu guter Letzt findet sich 
  am Ende dieses Dokumentes das komplette Tutorial als ausführbarer Quelltext. 
  \end{abstract}
]
\clearpage
\tableofcontents
\listoffigures
\listoftables
\clearpage



\section{Einleitung}
\label{sec:introduction}
Zu Beginn werden allerhand Pakete geladen, auf die im späteren Verlauf noch 
genauer eingegangen wird. Den Anfang macht das Paket \Package{hyphsubst}. 
Dieses wird für eine verbesserte Silbentrennung für die deutsche Sprache 
benötigt und muss bereits \emph{vor} der Klasse geladen werden, damit es 
problemlos funktioniert. Die Option \Option*{ngerman} führt dabei zur 
Verwendung der neuen deutschen Rechtschreibung. Für die alte Orthographie ist 
stattdessen die Option \Option*{german} zu verwenden.

Beim Laden der Klasse mit \Macro*{documentclass} sollten die im Dokument 
verwendeten Sprachen als Klassenoption angegeben werden, wobei die zuletzt 
angegebene als aktuelle Sprache aktiviert wird. Dadurch werden diese nicht nur 
an das Paket \Package{babel} sondern auch an andere Pakete weitergereicht, 
welche sprachspezifische Einstellungen vornehmen.
%
\begin{Tutorial*}
\RequirePackage[ngerman=ngerman-x-latest]{hyphsubst}
\documentclass[english,ngerman]{tudscrreprt}
\usepackage{babel}
\end{Tutorial*}
%
Bei der Verwendung von \hologo{LaTeXe} sollte zum einen die Eingabekodierung 
des erstellten Datei spezifiziert werden. Das Paket \Package{selinput} erkennt 
automatisch, welche Kodierung der genutzte Editor verwendet. Zum anderen werden 
die Schriften in der Ausgabe ebenfalls kodiert. Mit dem Paket \Package{fontenc} 
lässt sich die Schriftkodierung spezifizieren, wobei die europäischen Zeichen 
mit der Option~\Option*{T1} aktiviert werden.
%
\begin{Tutorial*}
\usepackage{selinput}\SelectInputMappings{adieresis={ä},germandbls={ß}}
\usepackage[T1]{fontenc}
\end{Tutorial*}
%
Das Paket \Package{fixltx2e} behebt einige Fehler im \hologo{LaTeXe}-Kernel. 
In neuen Dokumenten kann es bedenkenlos geladen werden.
%
\begin{Tutorial*}
\usepackage{fixltx2e}
\end{Tutorial*}
%
Für die Aufgabenstellung wird das Paket \Package{tudscrsupervisor} benötigt.
%
\begin{Tutorial*}
\usepackage{tudscrsupervisor}
\end{Tutorial*}
%
Das Paket \Package{tabu} ist relativ neu und versucht, viele Funktionalitäten 
ganz unterschiedlicher Pakete für den Tabellensatz in sich zu vereinen.
%
\begin{Tutorial*}
\usepackage{tabu}
\end{Tutorial*}
%
Damit alle möglichen Querverweise in einem PDF-Dokument automatisch verlinkt 
werden, sollte das Paket \Package{hyperref} geladen werden. Da dieses allerhand 
Veränderungen an vielen Standardbefehlen vornimmt, sollte dieses als letztes in 
der Präambel eingebunden werden. Nur Pakete, bei denen in der Dokumentation 
explizit darauf hingewiesen wird, dass diese nach \Package{hyperref} zu laden 
sind, sollten auch danach folgen.
%
\begin{Tutorial*}
\usepackage{hyperref}
\end{Tutorial*}
%
Eines dieser wenigen Pakete ist \Package{glossaries}. Dieses wird in diesem 
Tutorial für die Erstellung von Abkürzung- und Symbolverzeichnis verwendet. 
Genaueres hierzu ist in \autoref{sec:glossaries} zu finden. Dort wird auch 
genauer auf die hier genutzten Paketoptionen eingegangen.
%
\begin{Tutorial*}
\usepackage[%
  automake,%
%  xindy,\%={language=german-din},\% mit Tex Live einfach verwendbar
  acronym,% Abkürzungen
  symbols,% Formelzeichen
  nomain,% kein Glossar
  translate=babel,%
  nogroupskip,% 
]{glossaries}
\makeglossaries
\end{Tutorial*}
%
Damit sind alle notwendigen Pakete eingebunden und es das eigentliche Dokument 
kann begonnen werden.
\begin{Tutorial*}
\begin{document}
\end{Tutorial*}



\section{Satzspiegel und Bindekorrektur}
Gleich zu Beginn und bevor das eigentliche Verfassen der Arbeit beginnt, sollte 
man sich Gedanken über das zu nutzenden Layout und den Satzspiegel machen, um 
bei der Finalisierung keine böse Überraschung bei Seitenumbrüchen oder der 
Position von Gleitobjekten zu erleben.

Zuallererst gilt zu entscheiden, ob das Dokument einseitig oder beidseitig 
gesetzt werden soll. Ist Letzteres der Fall, so sollte \Option*{twoside} als 
Klassenoption angegeben werden. Im nächsten Schritt ist der zu verwendenden 
Satzspiegel festzulegen. Hierfür kann die Option \Option{geometry} verwendet 
werden, welche im \TUDScript-Handbuch beschrieben wird. Normalerweise wird das 
Dokument im asymmetrischen Layout des \CDs gesetzt. Dieses Verhalten kann mit 
\Option{geometry}[false] deaktiviert werden und der Satzspiegel wird durch das 
Paket \Package{typearea} nach typographischen Gesichtspunkten konstruiert.

Falls die Arbeit nach der Fertigstellung gebunden werden soll, so ist auf den 
notwendigen Binderand zu achten, quasi der Teil einer Seite, welcher durch die 
Bindung \enquote{verschwindet} und nicht mehr als sichtbarer Teil der Seite 
vorhanden ist. Als Faustregel gilt, dass die erforderliche Bindekorrektur in 
etwa der halben Höhe des Buchblocks entsprechen sollte. Dessen Höhe wiederum 
ist abhängig von der Anzahl der Seiten sowie der Papierdichte. Wird qualitativ 
höherwertiges Papier mit einer Dichte von \unit[100]{g/m²} verwendet, so 
entsprechen 100~Blatt in etwa einer Höhe von \unit[12]{mm}. Dementsprechend 
wäre bei diesem Beispiel eine Bindekorrektur \unit[6]{mm} notwendig. Diese 
kann mit der Klassenoption \Option{BCOR}[6mm] eingestellt werden.



\section{Umschlagseite und Titel}
Umschlagseite und Titel sind sich in ihrer Gestalt sehr ähnlich. Allerdings 
gibt es ein paar kleine Unterschiede. Zum einen werden auf dem Cover weniger 
Informationen als auf der Titelseite ausgegeben. Zum anderen wird der Titel 
immer im Satzspiegel des restlichen Dokumentes ausgegeben, wohingegen die 
Umschlagseite ohne weitere Optionen im asymmetrischen Layout des \CDs der \TnUD 
erscheint. Wie dies geändert werden kann, ist im Handbuch für \Macro{makecover} 
erläutert. Die resultierende Ausgabe des nachfolgenden Quelltextauszugs ist in 
\autoref{fig:title} zu sehen.
\begin{figure}
\includetutorial{Title}[1,2]
\caption{Umschlagseite und Titel}
\label{fig:title}
\end{figure}
\begin{Tutorial+}{Title}
\faculty{Juristische Fakultät}
\department{Fachrichtung Strafrecht}
\institute{Institut für Kriminologie}
\chair{Lehrstuhl für Kriminalprognose}
\title{%
  Entwicklung eines optimalen Verfahrens zur Eroberung des
  Geldspeichers in Entenhausen
}
\thesis{master}
\graduation[M.Sc.]{Master of Science}
\author{%
  Mickey Mouse
  \matriculationnumber{12345678}
  \dateofbirth{2.1.1990}
  \placeofbirth{Dresden}
  \and%
  Donald Duck
  \matriculationnumber{87654321}
  \dateofbirth{1.2.1990}
  \placeofbirth{Berlin}
}
\matriculationyear{2010}
\supervisor{Dagobert Duck \and Mac Moneysac}
\professor{Prof. Dr. Kater Karlo}
\date{10.09.2014}
\makecover
\maketitle
\end{Tutorial+}



\section{Vor- und Nachspann}
In den folgenden Unterabschnitten werden Elemente vorgestellt, die bei den 
meisten wissenschaftlichen (Abschluss"~)Arbeiten gefordert werden. Dabei ist 
die Platzierung der einzelnen Elemente innerhalb der Arbeit nicht eindeutig 
durch eine Norm oder dergleichen festgelegt. Vielmehr gibt es meist eine 
Richtlinie vom verantwortlichen Prüfungsamt oder eine konkrete Vorgabe des 
betreuenden Hochschullehrer respektive wissenschaftlichen Mitarbeiters. 


\subsection{Aufgabenstellung}
Wird eine Abschlussarbeit an der \TnUD geschrieben, kann entweder die Umgebung 
\Environment{task} oder der Befehl \Macro{taskform} für die Erstellung einer 
Aufgabenstellung verwendet werden. Bei beiden Varianten wird zu Beginn eine 
Tabelle mit Informationen zum Autor angegeben und am Ende der oder die Betreuer 
der Arbeit sowie Professor und gegebenenfalls der Prüfungsausschussvorsitzende. 
Dazwischen kann bei der Umgebung \Environment{task} ein freier Inhalt angegeben 
werden. Der Befehl \Macro{taskform} erzeugt eine standardisierte Ausgabe. Das 
Resultat des folgenden Quelltextes ist in \autoref{fig:task} dargestellt.
\begin{figure}
\includetutorial{Task}[1,2]
\caption{Aufgabenstellung in freier und standardisierter Form}
\label{fig:task}
\end{figure}
\begin{Tutorial+}{Task}
\faculty{Juristische Fakultät}
\department{Fachrichtung Strafrecht}
\institute{Institut für Kriminologie}
\chair{Lehrstuhl für Kriminalprognose}
\title{%
  Entwicklung eines optimalen Verfahrens zur Eroberung des
  Geldspeichers in Entenhausen
}
\thesis{master}
\graduation[M.Sc.]{Master of Science}
\author{%
  Mickey Mouse
  \matriculationnumber{12345678}
  \dateofbirth{2.1.1990}
  \placeofbirth{Dresden}
  \course{Klinische Prognostik}
  \discipline{Individualprognose}
\and%
  Donald Duck\matriculationnumber{87654321}
  \dateofbirth{1.2.1990}
  \placeofbirth{Berlin}
  \course{Statistische Prognostik}
  \discipline{Makrosoziologische Prognosen}
}
\matriculationyear{2010}
\issuedate{1.2.2015}
\duedate{1.8.2015}
\supervisor{Dagobert Duck \and Mac Moneysac}
\professor{Prof. Dr. Kater Karlo}
\chairman{Prof. Dr. Primus von Quack}
\newcommand\taskcontent{%
  Momentan ist das besagte Thema in aller Munde. Insbesondere wird es
  gerade in vielen~-- wenn nicht sogar in allen~-- Medien diskutiert.
  Es ist momentan noch nicht abzusehen, ob und wann sich diese Situation
  ändert. Eine kurzfristige Verlagerung aus dem Fokus der Öffentlichkeit
  wird nicht erwartet.
  
  Als Ziel dieser Arbeit soll identifiziert werden, warum das Thema
  gerade so omnipräsent ist und wie man diesen Effekt abschwächen
  könnte. Zusätzlich sollen Methoden entwickelt werden, wie sich ein
  ähnlicher Vorgang zukünftig vermeiden ließe.
}
\begin{task}
\smallskip\par\noindent
\taskcontent
\end{task}
\taskform[pagestyle=empty]{\taskcontent}{%
  \item Recherche
  \item Analyse
  \item Entwicklung eines Konzeptes
  \item Anwendung der entwickelten Methodik
  \item Dokumentation und grafische Aufbereitung der Ergebnisse
}
\end{Tutorial+}


\subsection{Zusammenfassung}
Häufig wird zu Beginn einer wissenschaftliche Arbeit die Motivation und der 
Inhalt dieser zusammengefasst, um den Leser die Thematik der Abhandlung 
vorzustellen. in den meisten Fällen wird diese dabei in deutscher und 
englischer Sprache verfasst. Hierfür stellt \KOMAScript{} bereits die Umgebung 
\Environment{abstract} bereit.

Vielfach wird der Wunsch geäußert, sowohl die deutsche als auch die englische 
Zusammenfassung auf derselben Seite zu setzen. Diese Variante kann mithilfe 
der \TUDScript-Klassen sehr einfach umgesetzt werden, wie der nachfolgende 
Quelltextauszug zeigt. Die resultierende Ausgabe ist in \autoref{fig:abstr} zu 
sehen.
\begin{figure}
\centering
\includetutorial[width=.5\textwidth]{Abstract}
\caption{Zusammenfassung in deutscher und englischer Sprache}
\label{fig:abstr}
\end{figure}
\begin{Tutorial+}{Abstract}
\TUDoption{abstract}{multiple,section}
\begin{abstract}
  Dies ist der deutschsprachige Teil der Zusammenfassung, in dem die
  Motivation sowie der Inhalt der nachfolgenden wissenschaftlichen
  Abhandlung kurz dargestellt werden.
\nextabstract[english]
  This is the english part of the summary, in which the motivation and
  the content of the following academic treatise are briefly presented.
\end{abstract}
\end{Tutorial+}


\subsection{Erklärungen}
Für die meisten Abschlussarbeiten an der \TnUD wird vom Verfasser eine 
Selbstständigkeitserklärung verlangt. Für diese wird ein Standardtext 
bereitgestellt. Dieser kann mit dem Befehl \Macro{confirmation} ausgegeben 
werden. Wurde das Thema in Kooperation mit einem Unternehmen bearbeitet, so 
wird zumeist auch ein Sperrvermerk gefordert, welcher mit \Macro{blocking} 
erzeugt werden kann. Mit \Macro{declaration} werden beide Erklärungen direkt 
nacheinander erzeugt. Die verwendete Überschrift und ein möglicher Eintrag in 
das Inhaltsverzeichnis können über die Option \Option{declaration} reguliert 
werden. Eine mögliche Ausprägung der Erklärungen ist in \autoref{fig:decl} 
abgebildet.
\begin{figure}
\centering
\includetutorial[width=.5\textwidth]{Declaration}
\caption{Selbstständigkeitserklärung und Sperrvermerk}
\label{fig:decl}
\end{figure}
\begin{Tutorial+}{Declaration}
\title{%
  Entwicklung eines optimalen Verfahrens zur Eroberung des
  Geldspeichers in Entenhausen
}
\author{Mickey Mouse\and Donald Duck}
\declaration[company=FIRMA]
\end{Tutorial+}


\subsection{Inhalts-, Abbildungs-, und Tabellenverzeichnis}
Zum Inhaltsverzeichnis ist nicht allzu viel zu sagen. Dieses wird mit dem
\hologo{LaTeX}"=Standardbefehl \Macro*{tableofcontents} erzeugt und führt die 
Gliederung des erstellten Dokumentes entsprechend der verwendeten Befehle 
(\Macro*{part}, \Macro*{addpart}, \Macro*{chapter}, \Macro*{addchap}, 
\Macro*{section}, \Macro*{addsec} etc.) auf. Wurde das Paket \Package{hyperref} 
geladen, so werden im Inhaltsverzeichnis PDF-Hyperlinks auf die einzelnen 
Abschnitte erzeugt.

Abbildungen und Tabellen werden mit \hologo{LaTeX} normalerweise innerhalb 
sogenannter Gleitumgebungen eingebunden (\Environment*{figure}"=Umgebung und 
\Environment*{table}"=Umgebung). Innerhalb dieser Umgebungen kann der Befehl 
\Macro*{caption}\OParameter{Verzeichniseintrag}\Parameter{Bezeichnung} genutzt 
werden, um diesen eine Bezeichnung hinzuzufügen. Mit \Macro*{listoffigures} 
beziehungsweise \Macro*{listoftables} lassen sich Verzeichnisse erstellen, in 
denen alle Gleitobjekte des jeweiligen Typs ausgegeben werden, falls diese denn 
eine Bezeichnung hinzugefügt wurde. Sollen Abbildungen oder Tabellen außerhalb 
ihrer angestammten Gleitumgebung genutzt und benannt werden, kann dies mit
\Macro*{captionof}\Parameter{Typ}\OParameter{Verzeichniseintrag}\Parameter{Bezeichnung}
erfolgen. Weitere Informationen hierzu sind der \KOMAScript"=Anleitung 
\scrguide zu 
entnehmen. Außerdem wird in diesem Tutorial in \autoref{sec:floats} genauer auf 
die Verwendung von Gleitumgebungen eingegangen.
%
\begin{Tutorial*}
\tableofcontents
\listoffigures
\listoftables
\end{Tutorial*}


\subsection{Abkürzungs- und Symbolverzeichnis}
\label{sec:glossaries}
Für die Auszeichnung von Abkürzungen gibt es zwei sehr gute Pakete, die dieses 
Unterfangen stark vereinfachen. Die einfachere~-- jedoch nicht so mächtige~-- 
der beiden Varianten ist die Nutzung des Paketes \Package{acro}. Sollen nur 
Abkürzungen und gegebenenfalls eine sortierte Liste dieser gesetzt werden, ist 
dieses allerdings absolut ausreichend. Für ein Symbolverzeichnis lässt sich in 
dieser Variante das Paket \Package*{nomencl} nutzen. Dieses bietet meiner 
Meinung nach jedoch keine großen Vorteile, stattdessen kann auch einfach eine 
Tabelle händisch erzeugt werden. Das Paket \Package{acro} ist sehr gut und 
ausführlich dokumentiert. Deshalb wird hier auf eine exemplarische Erläuterung 
verzichtet und stattdessen auf dessen Dokumentationen verwiesen.

Die andere Möglichkeit ist die Nutzung des Paketes \Package{glossaries}, das 
eine große Zahl an Einstellmöglichkeiten und Optionen besitzt, allerdings auch 
etwas Zeit für die Einarbeitung und Studium der Dokumentation benötigt. Der 
ursprüngliche Einsatzzweck dieses Paketes ist das Setzen eines fachsprachlichen 
oder technischen Glossars. Es bietet zusätzlich die Mittel zum Erzeugen eines 
Abkürzungs- sowie Symbolverzeichnis. Die Dokumentation von \Package{glossaries} 
lässt ebenfalls keine Wünsche offen. Dennoch soll folgend hier kurz erläutert 
werden, wie das Paket zu verwenden ist. Für weiterführende Beispiele sollte 
die Dokumentationen zu Rate gezogen werden. In \autoref{sec:introduction} wurde 
das Paket \Package{glossaries} bereits geladen. Die hierfür genutzten Optionen 
werden kurz erläutert.
%
\begin{Tutorial-}
\usepackage[%
\end{Tutorial-}
%
Das Programm \Application{makeindex} wird im Normalfall durch die genutzte 
\hologo{LaTeX}"=Distribution bereitgestellt und für das alphabetische Sortieren 
der erstellten Listen verwendet. Mit der Paketoption \Option*{automake} erfolgt 
der automatische Aufruf von \Application{makeindex} mit den passenden 
Einstellungen für alle Verzeichnisse. Alternativ dazu kann die Option 
\Option{xindy} aktiviert werden, welche \Application{xindy} anstelle von 
\Application{makeindex} für das Sortieren verwendet. Diese Programm bietet 
unter anderem eine Unterstützung von Unicode sowie die Möglichkeit, nach 
sprachabhängigen Regeln zu sortieren. Allein für die deutsche Sprache gibt es 
beispielsweise zwei verschiedene Varianten~-- nach DIN und nach Duden~-- zum 
alphabetischen Sortieren. Allerdings wird das Programm \Application{xindy} 
lediglich mit der Distribution \Distribution{\hologo{TeX}~Live} jedoch nicht 
mit \Distribution{\hologo{MiKTeX}} geliefert. Wenn ohnehin die Distribution 
\Distribution{\hologo{TeX}~Live} verwendet wird, würde ich persönlich die 
Verwendung von \Application{xindy} vorziehen. Für das Erstellen der Glossare 
sollte das Perl"=Skript \Application{makeglossaries} verwendet werden, welches 
alle notwendigen Optionen an \Application{xindy} weiterleitet. Treten Probleme 
bei der Erzeugung der einzelnen Glossare auf, sollte die Dokumentation von 
\Package{glossaries} weiterhelfen können.
%
\begin{Tutorial-}
  automake,%
%  xindy,\%={language=german-din},\% mit Tex Live einfach verwendbar
\end{Tutorial-}
%
Die Optionen \Option*{acronym} und \Option*{symbols} erzeugen die Glossare 
beziehungsweise die Verzeichnisse für Abkürzungen und Symbole. Die Option  
\Option*{nomain} wird verwendet, weil in diesem Tutorial kein zusätzliches 
technisches Glossar erzeugt werden soll.
%
\begin{Tutorial-}
  acronym,% Abkürzungen
  symbols,% Formelzeichen
  nomain,% kein Glossar
\end{Tutorial-}
%
Mit der Option \Option*{translate}[babel] werden die Überschriften der Glossare 
in der Dokumentsprache gesetzt, mit \Option*{nogroupskip} kann der automatische 
Abstand zwischen den Einträgen zur Gruppierung innerhalb eines Glossars 
entfernt werden.
%
\begin{Tutorial-}
  translate=babel,%
  nogroupskip,% 
\end{Tutorial-}
%
Damit sind alle verwendeten Optionen erläutert. Schließlich sorgt der Befehl 
\Macro*{makeglossaries} für das Erstellen der optionsabhängigen Stildateien für 
\Application{makeindex} respektive \Application{xindy} sowie das Erzeugen der 
benötigten Hilfsdateien.
%
\begin{Tutorial-}
]{glossaries}
\makeglossaries
\end{Tutorial-}
%
Damit wäre der erste Teil zur Initialisierung überstanden und wir können zum 
eigentlichen Problem kommen. Wie erstellt man nun ein Abkürzungs- und/oder 
Symbolverzeichnis?

\subsubsection{Abkürzungsverzeichnis}
Das Paket \Package{glossaries} stellt für die Definition von Abkürzungen einen 
speziellen Befehl bereit. Mit\Macro*{newacronym}\LParameter\Parameter{Label}%
\Parameter{Abkürzung}\Parameter{Wortgruppe} wird eine Abkürzung definiert und 
kann später über \Parameter{Label} genutzt werden. Für ein kleines Beispiel 
werden drei Abkürzungen erstellt\dots
%
\begin{Tutorial}
\newacronym{spsp}{SPSP}{Single-Pair Shortest Path}
\newacronym{sssp}{SSSP}{Single-Source Shortest Path}
\newacronym{apsp}{APSP}{All-Pairs Shortest Path}
\end{Tutorial}
\vspace*{-1.5\baselineskip}\par\noindent%
%
\dots und diese in einer kurzen Textpassage mit \Macro*{gls}\Parameter{Label} 
verwendet.
%
\TutorialPreamble{\minisec{Ausgabe der Textpassage}}
\begin{Tutorial}
In der Graphentheorie wird häufig die Lösung des Problems des kürzesten
Pfades zwischen zwei Knoten gesucht, welches auch als \gls{spsp}
bezeichnet werden kann. Dieses Problem lässt sich auf die Variationen
\gls{sssp} und \gls{apsp} erweitern. Für die Lösung von \gls{spsp},
\gls{sssp} oder \gls{apsp} kommen unterschiedliche Algorithmen zum Einsatz.
\end{Tutorial}
%
Gut zu sehen ist, dass sich die Ausgabe der Abkürzung bei der ersten Verwendung 
mit \Macro*{gls} von der zweiten~-- und jeder weiteren~-- unterscheidet. Das 
Verhalten lässt sich über verschiedene Stile mit \Macro*{setacronymstyle} 
anpassen. Die Ausgabe einer Liste aller Abkürzungen erfolgt mit:
%
\begin{Tutorial*}
\printacronyms
\end{Tutorial*}
\vspace*{-1.5\baselineskip}\par\noindent%
\begin{quoting}[rightmargin=0pt]
\printacronyms
\end{quoting}
%
Dabei werden die Abkürzungen in einer \Environment*{description}"=Umgebung 
gesetzt. Dies ist absolut ausreichend. Mir persönlich ist die Darstellung in 
einer quasi-tabellarischen Form jedoch lieber. Dabei soll eigentliche Stil mit 
fettgedruckter Abkürzung beibehalten werden. Das \Package{glossaries}"=Paket 
stellt zwar auch eine Menge Stilen in Tabellenform bereit, allerdings nicht in 
dem gewünschten Stil. Deshalb wird hier gezeigt, wie ein eigener Stil kreiert 
werden kann. Dafür wird der Befehl \Macro*{newglossarystyle} verwendet. Die 
Definition des neuen Stils \PValue{acronymstabu} wird nachfolgend ausgegeben, 
die Erläuterung dazu schließt sich daran an.
%
\begin{Tutorial}
\newglossarystyle{acronymstabu}{%
  \renewenvironment{theglossary}{
    \begin{tabu}spread 0pt{@{}lX<{\strut}l@{}}
  }{
    \end{tabu}
  }
  \renewcommand*{\glossaryheader}{}%
  \renewcommand*{\glsgroupheading}[1]{}%
  \renewcommand*{\glsgroupskip}{}%
  \renewcommand*{\glossentry}[2]{%
    \glsentryitem{##1}% Entry number if required
    \glstarget{##1}{\sffamily\bfseries\glossentryname{##1}} &
    \glsentrydesc{##1} &
    ##2\tabularnewline
  }
}
\end{Tutorial}
%
Als erstes wird die Umgebung \Environment*{theglossary} umdefiniert, wobei 
innerhalb dieser \Environment*{tabu} aus dem Paket \Package{tabu} verwendet 
wird. Dabei werden drei Spalten definiert. Die erste und letzte Spalte sind 
schlicht linksbündig~(\PValue{l}). In diesen werden die Abkürzung selbst 
beziehungsweise die Seitenzahl eingetragen. Die Verwendung von~\PValue{@\{\}} 
führt dazu, dass der normalerweise vor der ersten und nach der letzten Spalte 
eingefügte Abstand von \Length*{tabcolsep} entfällt. Die mittlere Spalte vom 
Typ~\PValue{X} verhält sich prinzipiell wie der gleichnamige Spaltentyp aus dem 
Paket \Package{tabularx}. Allerdings gibt es hier eine Besonderheit.

Für \Environment*{tabularx}"=Tabellen muss generell eine feste Tabellenbreite 
angegeben werden. Die Breite der \PValue{X}"~Spalten wird anhand dieser und 
dem bereits für andere Spalten vom Typ~\PValue{l},~\PValue{r}~und~\PValue{c} 
benötigten Platz berechnet. Für \Environment*{tabu}"=Tabellen kann anstelle 
einer festen Breite auch \PValue{spread 0pt} angegeben werden. Dadurch werden 
Spalten vom Typ~\PValue{X} anfänglich in ihrer natürlichen Breite gesetzt. 
Sobald jedoch die Gesamtbreite der Tabelle die Seitenbreite überschreiten 
würde, werden die \PValue{X}"~Spalten automatisch umbrochen.

Der Rest des Stils ist recht schnell erläutert. Zunächst wird auf Tabellenköpfe 
sowie Gruppierungen (Abstände und Überschriften) verzichtet. Schließlich ist 
der Befehl \Macro*{glossentry} verantwortlich für die Formatierung der Einträge 
im Abkürzungsverzeichnis. Dieser wird intern durch \Package{glossaries} mit 
zwei Argumenten aufgerufen. Das erste enthält das entsprechende Label, das 
zweite ein kommaseparierte Liste der Seitenzahlen. Dabei stehen verschiedene 
Makros zur Auswahl, um anhand eines Labels die gewünschte Information zu 
extrahieren.%
\footnote{%
  bspw. mit \Macro*{glossentryname} die Bezeichnung oder mit 
  \Macro*{glsentrydesc} die dazugehörige Beschreibung%
}
Der Befehl \Macro*{glossentry} so definiert, dass für jeden Eintrag eine Zeile 
in der Tabellen erzeugt wird, wo in der ersten Spalte die Abkürzung selbst, in 
der zweiten die Langform und in der dritten Spalte schließlich die Liste der 
Seiten, auf welchen die jeweilige Abkürzung mit \Macro*{gls}\Parameter{Label} 
verwendet wurde, ausgegeben wird. Zum Abschluss die resultierende Ausgabe des 
neuen Stils.
%
\begin{Tutorial*}
\printacronyms[style=acronymstabu]
\end{Tutorial*}
\vspace*{-1.5\baselineskip}\par\noindent%
\begin{quoting}[rightmargin=0pt]
\printacronyms[style=acronymstabu]
\end{quoting}

\subsubsection{Symbolverzeichnis}
Für das Erstellen 





\section{Anfangszitat oder Schlauer Spruch}
%%Anfangszitat: Der Text erscheint jetzt kursiv (\itshape) und kleiner als der 
%%normale Text (\footnotesize). Mit \dictumwidth wird die Breite des Textfeldes 
%%gesteuert. Sie nimmt jetzt 65% der gesamten Textbreite ein. Mit 
%%%\raggeddictumtext wird die Ausrichtung des Textes gesteuert. \raggedleft 
%%%heißt, 
%%%dass der Text links „ausflattert”, also rechtsbündig gesetzt wird.
%\setkomafont{dictumtext}{\itshape}
%\renewcommand*{\dictumwidth}{.75\textwidth}
%\renewcommand*{\raggeddictumtext}{\raggedleft}


\section{Gleitumgebungen}
\label{sec:floats}

\subsection{Grafiken und Untergrafiken}

\subsection{Tabellen}


\section{Listen}


\section{Mathematiksatz}


\section{Typographie}

\subsection{Abkürzungen}

\subsection{Einheiten}


\section{Querverweise}


\section{Silbentrennung}


\section{Literaturverzeichnis}


\section{Quelltexte}



\dots % §§§
%
%\begin{Tutorial-}
%\usepackage{xparse}% §§§ bereits geladen
%\usepackage{microtype}% §§§ bereits geladen
%\usepackage{textcomp}% §§§ bereits geladen
%\usepackage{setspace}% §§§ bereits geladen
%\usepackage{csquotes}% §§§ bereits geladen
%\usepackage{quoting}% §§§ bereits geladen
%\usepackage{isodate}% §§§ bereits geladen
%\usepackage{listings}% §§§ bereits geladen
%\usepackage{xspace}% §§§ bereits geladen
%\usepackage{scrhack}% §§§ bereits geladen
%\end{Tutorial-}
%%
%\begin{Tutorial-}
%\usepackage{caption}
%\usepackage{floatrow}
%\usepackage{tikz}
%\usepackage{biblatex}
%\usepackage{array}
%\usepackage{booktabs}
%\usepackage{tabularx}
%\usepackage{ltxtable}
%\usepackage{tabu}
%\usepackage{enumitem}
%\setlist{noitemsep}
%\usepackage{mathtools}
%\usepackage{units}
%\usepackage{siunitx}
%\usepackage{xpunctuate}
%\usepackage{ellipsis}
%\usepackage{varioref}
%\usepackage{chngcntr}
%\usepackage{bookmark}
%\end{Tutorial-}
%%
%Damit ist das Tutorial beendet.
%\begin{Tutorial*}
%\end{document}
%\end{Tutorial*}
\FinishTutorial
\clearpage
  \begin{itemize}
  \item grafiken und untergrafiken?
  \item tabellen
  \item literaturverzteichnis
  \item floatrow
  \item mathematikmodus
  \item tabbing-umgebung
  \end{itemize}
\ListOfToDo
\ToDo[imp]{Einarbeiten von paragraph in treatise}[v2.02]
\end{document}