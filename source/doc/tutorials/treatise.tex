\RequirePackage[ngerman=ngerman-x-latest]{hyphsubst}
\documentclass[english,ngerman]{tudscrartcl}
\usepackage{selinput}
\SelectInputMappings{adieresis={ä},germandbls={ß}}
\usepackage[T1]{fontenc}
\usepackage{lmodern}
\usepackage{tudscrman}
\lstset{%
  inputencoding=utf8,extendedchars=true,
  literate=%
    {ä}{{\"a}}1 {ö}{{\"o}}1 {ü}{{\"u}}1
    {Ä}{{\"A}}1 {Ö}{{\"O}}1 {Ü}{{\"U}}1
    {~}{{\textasciitilde}}1 {ß}{{\ss}}1
}
\usepackage{tudscrsupervisor}
\usepackage{units}
%\usepackage{tocstyle}
%\newtocstyle[KOMAlike]{compressed}{
%    \settocfeature[1]{entryvskip}{.5em plus 1pt}%  % Die Stellschraube
%}
%\usetocstyle{compressed}
\begin{document}
\TUDoption{cdfont}{no,nodin}
\KOMAoptions{headings=normal}
\title{%
  Ein Anwenderleitfaden für das Erstellen einer wissenschaftlichen Abhandlung%
}
\author{Falk Hanisch}
\date{10.09.2014}
\StartTutorial[%
  \begin{abstract}\noindent
  Der Versuch, ein allumfassendes Tutorial für eine wissenschaftliche Arbeit 
  zur Verfügung zu stellen gleicht der beschwerlichen Suche nach einer 
  eierlegenden Wollmilchsau. Es ist quasi nicht möglich, alle möglichen 
  Anforderungen an eine wissenschaftliche Arbeit in einem Dokument abzudecken. 
  Dennoch soll hier versucht werden, einen Großteil der für gewöhnlich 
  auftretenden Erfordernisse zu bearbeiten.
  
  Dieses Tutorial hat \emph{nicht} die Intention, \hologo{LaTeX}-Einsteigern 
  sämtliche Grundlagen zu erläutern. Vielmehr wird davon ausgegangen, das der 
  Leser die ersten Versuche mit \hologo{LaTeXe} bereits hinter sich hat. 
  Dennoch wird versucht, alle Schritte möglichst leicht nachvollziehbar zu 
  gestalten.
  
  In erster Linie ist dieser Anwenderleitfaden für die praktische Umsetzung mit 
  einer \TUDScript-Dokumentklasse gedacht, kann jedoch~-- natürlich mit 
  gewissen Abstrichen~-- auch als Grundlage für andere Klassen, insbesondere 
  die aus dem \KOMAScript"=Bundle genutzt werden. Viele der verwendeten 
  Optionen und Befehle werden hier nur sporadisch erläutert. Eine detaillierte 
  Erklärung lässt sich jedoch jederzeit sehr einfach über die Hyperlinks zu den 
  entsprechenden Stellen im \href{run:../tudscr.pdf}{\TUDScript-Handbuch} 
  öffnen.
  
  Das Tutorial muss nicht zwingend vollständig nachvollzogen werden. Es ist in 
  einzelne Abschnitte untergliedert, damit der Anwender sich bestimmte Aspekte 
  erarbeiten kann. Nichtsdestotrotz findet sich am Ende dieses Dokumentes das 
  komplette Tutorial als ausführbarer Quelltext.
  \end{abstract}
]
\clearpage\tableofcontents\clearpage

\section{Einleitung}
Zu Beginn werden allerhand Pakete geladen, auf die im späteren Verlauf noch 
genauer eingegangen wird. Den Anfang macht das Paket \Package{hyphsubst}. 
Dieses wird für eine verbesserte Silbentrennung für die deutsche Sprache 
benötigt und muss bereits \emph{vor} der Klasse geladen werden, damit es 
problemlos funktioniert. Die Option \Option*{ngerman} führt dabei zur 
Verwendung der neuen deutschen Rechtschreibung. Für die alte Orthographie ist 
stattdessen die Option \Option*{german} zu verwenden.

Beim Laden der Klasse mit \Macro*{documentclass} sollten die im Dokument 
verwendeten Sprachen als Klassenoption angegeben werden, wobei die zuletzt 
angegebene als aktuelle Sprache aktiviert wird. Dadurch werden diese nicht nur 
an das Paket \Package{babel} sondern auch an andere Pakete weitergereicht, 
welche sprachspezifische Einstellungen vornehmen.
%
\begin{Tutorial*}
\RequirePackage[ngerman=ngerman-x-latest]{hyphsubst}
\documentclass[ngerman]{tudscrreprt}% andere Klassen sind bedingt möglich
\usepackage{babel}
\end{Tutorial*}
%
Bei der Verwendung von \hologo{LaTeXe} sollte zum einen die Eingabekodierung 
des erstellten Datei spezifiziert werden. Das Paket \Package{selinput} erkennt 
automatisch, welche Kodierung der genutzte Editor verwendet. Zum anderen werden 
die Schriften in der Ausgabe ebenfalls kodiert. Mit dem Paket \Package{fontenc} 
lässt sich die Schriftkodierung spezifizieren, wobei die europäischen Zeichen 
mit der Option~\Option*{T1} aktiviert werden.
%
\begin{Tutorial*}
\usepackage{selinput}\SelectInputMappings{adieresis={ä},germandbls={ß}}
\usepackage[T1]{fontenc}
\end{Tutorial*}
%
Das Paket \Package{fixltx2e} behebt einige Fehler im \hologo{LaTeXe}-Kernel. 
In neuen Dokumenten kann es bedenkenlos geladen werden.
%
\begin{Tutorial*}
\usepackage{fixltx2e}
\end{Tutorial*}
%
Für die Aufgabenstellung wird das Paket \Package{tudscrsupervisor} benötigt.
%
\begin{Tutorial*}
\usepackage{tudscrsupervisor}
\end{Tutorial*}
%
Damit sind alle notwendigen Pakete eingebunden und es das eigentliche Dokument 
kann begonnen werden.
\begin{Tutorial*}
\begin{document}
\end{Tutorial*}


\section{Satzspiegel und Bindekorrektur}
Gleich zu Beginn und bevor das eigentliche Verfassen der Arbeit beginnt, sollte 
man sich Gedanken über das zu nutzenden Layout und den Satzspiegel machen, um 
bei der Finalisierung keine böse Überraschung bei Seitenumbrüchen oder der 
Position von Gleitobjekten zu erleben.

Zuallererst gilt zu entscheiden, ob das Dokument einseitig oder beidseitig 
gesetzt werden soll. Ist Letzteres der Fall, so sollte \Option*{twoside} als 
Klassenoption angegeben werden. Im nächsten Schritt ist der zu verwendenden 
Satzspiegel festzulegen. Hierfür kann die Option \Option{geometry} verwendet 
werden, welche im \TUDScript-Handbuch beschrieben wird. Normalerweise wird das 
Dokument im asymmetrischen Layout des \CDs gesetzt. Dieses Verhalten kann mit 
\Option{geometry}[false] deaktiviert werden und der Satzspiegel wird durch das 
Paket \Package{typearea} nach typographischen Gesichtspunkten konstruiert.

Falls die Arbeit nach der Fertigstellung gebunden werden soll, so ist auf den 
notwendigen Binderand zu achten, quasi der Teil einer Seite, welcher durch die 
Bindung \enquote{verschwindet} und nicht mehr als sichtbarer Teil der Seite 
vorhanden ist. Als Faustregel gilt, dass die erforderliche Bindekorrektur in 
etwa der halben Höhe des Buchblocks entsprechen sollte. Dessen Höhe wiederum 
ist abhängig von der Anzahl der Seiten sowie der Papierdichte. Wird qualitativ 
höherwertiges Papier mit einer Dichte von \unit[100]{g/m²} verwendet, so 
entsprechen 100~Blatt in etwa einer Höhe von \unit[12]{mm}. Dementsprechend 
wäre bei diesem Beispiel eine Bindekorrektur \unit[6]{mm} notwendig. Diese 
kann mit der Klassenoption \Option{BCOR}[6mm] eingestellt werden.


\section{Umschlagseite und Titel}
Umschlagseite und Titel sind sich in ihrer Gestalt sehr ähnlich. Allerdings 
gibt es ein paar kleine Unterschiede. Zum einen werden auf dem Cover weniger 
Informationen als auf der Titelseite ausgegeben. Zum anderen wird der Titel 
immer im Satzspiegel des restlichen Dokumentes ausgegeben, wohingegen die 
Umschlagseite ohne weitere Optionen im asymmetrischen Layout des \CDs der \TnUD 
erscheint. Wie dies geändert werden kann, ist im Handbuch für \Macro{makecover} 
erläutert. Die resultierende Ausgabe des nachfolgenden Quelltextauszugs ist in 
\autoref{fig:title} zu sehen.
\begin{figure}
\includetutorial{Title}[1,2]
\caption{Umschlagseite und Titel}
\label{fig:title}
\end{figure}
\begin{Tutorial+}{Title}
\faculty{Juristische Fakultät}
\department{Fachrichtung Strafrecht}
\institute{Institut für Kriminologie}
\chair{Lehrstuhl für Kriminalprognose}
\title{%
  Entwicklung eines optimalen Verfahrens zur Eroberung des
  Geldspeichers in Entenhausen
}
\thesis{master}
\graduation[M.Sc.]{Master of Science}
\author{%
  Mickey Mouse
  \matriculationnumber{12345678}
  \dateofbirth{2.1.1990}
  \placeofbirth{Dresden}
  \and%
  Donald Duck
  \matriculationnumber{87654321}
  \dateofbirth{1.2.1990}
  \placeofbirth{Berlin}
}
\matriculationyear{2010}
\supervisor{Dagobert Duck \and Mac Moneysac}
\professor{Prof. Dr. Kater Karlo}
\date{10.09.2014}
\makecover
\maketitle
\end{Tutorial+}


\section{Aufgabenstellung}
Wird eine Abschlussarbeit an der \TnUD geschrieben, kann entweder die Umgebung 
\Environment{task} oder der Befehl \Macro{taskform} für die Erstellung einer 
Aufgabenstellung verwendet werden. Bei beiden Varianten wird zu Beginn eine 
Tabelle mit Informationen zum Autor angegeben und am Ende der oder die Betreuer 
der Arbeit sowie Professor und gegebenenfalls der Prüfungsausschussvorsitzende. 
Dazwischen kann bei der Umgebung \Environment{task} ein freier Inhalt angegeben 
werden. Der Befehl \Macro{taskform} erzeugt eine standardisierte Ausgabe. Das 
Resultat des folgenden Quelltextes ist in \autoref{fig:task} dargestellt.
\begin{figure}
\includetutorial{Task}[1,2]
\caption{Aufgabenstellung in freier und standardisierter Form}
\label{fig:task}
\end{figure}
\begin{Tutorial+}{Task}
\faculty{Juristische Fakultät}
\department{Fachrichtung Strafrecht}
\institute{Institut für Kriminologie}
\chair{Lehrstuhl für Kriminalprognose}
\title{%
  Entwicklung eines optimalen Verfahrens zur Eroberung des
  Geldspeichers in Entenhausen
}
\thesis{master}
\graduation[M.Sc.]{Master of Science}
\author{%
  Mickey Mouse
  \matriculationnumber{12345678}
  \dateofbirth{2.1.1990}
  \placeofbirth{Dresden}
  \course{Klinische Prognostik}
  \discipline{Individualprognose}
\and%
  Donald Duck\matriculationnumber{87654321}
  \dateofbirth{1.2.1990}
  \placeofbirth{Berlin}
  \course{Statistische Prognostik}
  \discipline{Makrosoziologische Prognosen}
}
\matriculationyear{2010}
\issuedate{1.2.2015}
\duedate{1.8.2015}
\supervisor{Dagobert Duck \and Mac Moneysac}
\professor{Prof. Dr. Kater Karlo}
\chairman{Prof. Dr. Primus von Quack}
\newcommand\taskcontent{%
  Momentan ist das besagte Thema in aller Munde. Insbesondere wird es
  gerade in vielen~-- wenn nicht sogar in allen~-- Medien diskutiert.
  Es ist momentan noch nicht abzusehen, ob und wann sich diese Situation
  ändert. Eine kurzfristige Verlagerung aus dem Fokus der Öffentlichkeit
  wird nicht erwartet.
  
  Als Ziel dieser Arbeit soll identifiziert werden, warum das Thema
  gerade so omnipräsent ist und wie man diesen Effekt abschwächen
  könnte. Zusätzlich sollen Methoden entwickelt werden, wie sich ein
  ähnlicher Vorgang zukünftig vermeiden ließe.
}
\begin{task}
\bigskip\taskcontent
\end{task}
\taskform[pagestyle=empty]{\taskcontent}{%
  \item Recherche
  \item Analyse
  \item Entwicklung eines Konzeptes
  \item Anwendung der entwickelten Methodik
  \item Dokumentation und grafische Aufbereitung der Ergebnisse
}
\end{Tutorial+}


\section{Erklärungen}
Für die meisten Abschlussarbeiten an der \TnUD wird vom Verfasser eine 
Selbstständigkeitserklärung verlangt. Für diese wird ein Standardtext 
bereitgestellt. Dieser kann mit dem Befehl \Macro{confirmation} ausgegeben 
werden. Wurde das Thema in Kooperation mit einem Unternehmen bearbeitet, so 
wird zumeist auch ein Sperrvermerk gefordert, welcher mit \Macro{blocking} 
erzeugt werden kann. Mit \Macro{declaration} werden beide Erklärungen direkt 
nacheinander erzeugt. Die verwendete Überschrift und ein möglicher Eintrag in 
das Inhaltsverzeichnis können über die Option \Option{declaration} reguliert 
werden. Eine mögliche Ausprägung der Erklärungen ist in \autoref{fig:decl} 
abgebildet.
\begin{figure}
\centering
\includetutorial[width=.5\textwidth]{Declaration}
\caption{Selbstständigkeitserklärung und Sperrvermerk}
\label{fig:decl}
\end{figure}
\begin{Tutorial+}{Declaration}
\title{%
  Entwicklung eines optimalen Verfahrens zur Eroberung des
  Geldspeichers in Entenhausen
}
\author{Mickey Mouse\and Donald Duck}
\declaration[company=FIRMA]
\end{Tutorial+}


\section{Zusammenfassung}
Häufig wird zu Beginn einer wissenschaftliche Arbeit die Motivation und der 
Inhalt dieser zusammengefasst, um den Leser die Thematik der Abhandlung 
vorzustellen. in den meisten Fällen wird diese dabei in deutscher und 
englischer Sprache verfasst. Hierfür stellt \KOMAScript bereits die Umgebung 
\Environment{abstract} bereit. Vielfach wird der Wunsch geäußert, sowohl die 
deutsche als auch die englische Zusammenfassung auf derselben Seite zu setzen. 
Diese Variante wird durch die \TUDScript-Klassen ermöglicht, die resultierende 
Ausgabe ist in \autoref{fig:abstr} zu sehen.
\begin{figure}
\centering
\includetutorial[width=.5\textwidth]{Abstract}
\caption{Zusammenfassung in deutscher und englischer Sprache}
\label{fig:abstr}
\end{figure}
\begin{Tutorial+}{Abstract}
\TUDoption{abstract}{multiple,section}
\begin{abstract}
  Dies ist der deutschsprachige Teil der Zusammenfassung, in dem die
  Motivation sowie der Inhalt der nachfolgenden wissenschaftlichen
  Abhandlung kurz dargestellt werden.
\nextabstract[english]
  This is the english part of the summary, in which the motivation and
  the content of the following academic treatise are briefly presented.
\end{abstract}
\end{Tutorial+}


\section{Inhalts-, Abbildungs-, und Tabellenverzeichnis}

\section{Abkürzungsverzeichnis}

\section{Symbolverzeichnis}
%%Anfangszitat: Der Text erscheint jetzt kursiv (\itshape) und kleiner als der 
%%normale Text (\footnotesize). Mit \dictumwidth wird die Breite des Textfeldes 
%%gesteuert. Sie nimmt jetzt 65% der gesamten Textbreite ein. Mit 
%%%\raggeddictumtext wird die Ausrichtung des Textes gesteuert. \raggedleft 
%%%heißt, 
%%%dass der Text links „ausflattert”, also rechtsbündig gesetzt wird.
%\setkomafont{dictumtext}{\itshape}
%\renewcommand*{\dictumwidth}{.75\textwidth}
%\renewcommand*{\raggeddictumtext}{\raggedleft}

%\documentclass{tudscrbook}
%\usepackage{selinput}
%\SelectInputMappings{adieresis={ä},germandbls={ß}}
%\usepackage[T1]{fontenc}
%\newcommand*\listofloaname{List of Algorithms} 
%\newcommand*{\listofalgorithms}{\listoftoc{loa}}
%\addtotoclist[float]{loa}
%\begin{document}
%\title{Mein Titel}
%\author{Mein Name}
%\KOMAoption{listof}{leveldown}
%\addchap{Lists of Figures, Tables and Algorithms}
%\listoffigures
%\listoftables
%\listofalgorithms
%\cleardoublepage
%\end{document}



\section{Gleitumgebungen}
\subsection{Grafiken und Untergrafiken}
\subsection{Tabellen}

\section{Listen}

\section{Mathematiksatz}

\section{Typographie}
\subsection{Abkürzungen}
\subsection{Einheiten}

\section{Querverweise}

\section{Silbentrennung}

\section{Literaturverzeichnis}

\section{Quelltexte}



\dots % §§§
%
\begin{Tutorial*}
\usepackage{xparse}% §§§ bereits geladen
\usepackage{microtype}% §§§ bereits geladen
\usepackage{textcomp}% §§§ bereits geladen
\usepackage{setspace}% §§§ bereits geladen
\usepackage{csquotes}% §§§ bereits geladen
\usepackage{quoting}% §§§ bereits geladen
\usepackage{isodate}% §§§ bereits geladen
\usepackage{listings}% §§§ bereits geladen
\usepackage{hyperref}% §§§ bereits geladen
\usepackage{xspace}% §§§ bereits geladen
\usepackage{scrhack}% §§§ bereits geladen
\end{Tutorial*}
%
\begin{Tutorial*}
\usepackage{caption}
\usepackage{floatrow}
\usepackage{tikz}
\usepackage{biblatex}
\usepackage{glossaries}
\usepackage{array}
\usepackage{booktabs}
\usepackage{tabularx}
\usepackage{ltxtable}
\usepackage{tabu}
\usepackage{enumitem}
\setlist{noitemsep}
\usepackage{mathtools}
\usepackage{units}
\usepackage{siunitx}
\usepackage{xpunctuate}
\usepackage{ellipsis}
\usepackage{varioref}
\usepackage{chngcntr}
\usepackage{bookmark}
\end{Tutorial*}
%
\FinishTutorial
\clearpage
  \begin{itemize}
  \item inhaltsverzeichnis, tabellen und abbildungsverzeichnis
  \item abstract
  \item grafiken und untergrafiken?
  \item tabellen
  \item literaturverzteichnis
  \item abkürzungsverzeichnis
  \item floatrow
  \item mathematikmodus
  \item tabbing-umgebung
  \end{itemize}
\ListOfToDo
\ToDo[imp]{Einarbeiten von paragraph in treatise}[v2.02]
\end{document}