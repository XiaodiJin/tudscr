\RequirePackage[ngerman=ngerman-x-latest]{hyphsubst}
\documentclass[%
  english,ngerman,%
  geometry=no,DIV=12,automark,%
]{tudscrartcl}
\usepackage{selinput}
\SelectInputMappings{adieresis={ä},germandbls={ß}}
\usepackage[T1]{fontenc}
\usepackage{lmodern}
\usepackage{tudscrman}
\lstset{%
  inputencoding=utf8,extendedchars=true,
  literate=%
    {ä}{{\"a}}1 {ö}{{\"o}}1 {ü}{{\"u}}1
    {Ä}{{\"A}}1 {Ö}{{\"O}}1 {Ü}{{\"U}}1
    {~}{{\textasciitilde}}1 {ß}{{\ss}}1
}
\usepackage{tudscrsupervisor}
\usepackage{tabu,booktabs}
\usepackage{units}
\AfterPackage*{hyperref}{%
  \usepackage[%
    automake,%
    acronym,%
    symbols,%
    nomain,%
    translate=babel,%
    nogroupskip,%
    section=subsubsection,%
  ]{glossaries}
  \setStyleFile{\jobname-temp}
  \makeglossaries
}
\usepackage{csquotes}
\usepackage[backend=biber,style=alphabetic]{biblatex}
\usepackage{filecontents}
\addbibresource{treatise-temp.bib}
\renewcommand{\floatpagefraction}{0.7}

\begin{document}
\TUDoptions{cdfont=no}
\KOMAoptions{headings=normal}
\title{%
  Ein Anwenderleitfaden für das Erstellen einer wissenschaftlichen Abhandlung%
}
\author{Falk Hanisch}
\date{10.09.2014}
\makeatletter
\begingroup%
  \def\and{, }%
  \let\thanks\@gobble%
  \let\footnote\@gobble%
  \hypersetup{%
    pdfauthor = {\@author},%
    pdftitle = {\@title},%
    pdfsubject = {Tutorial für \hologo{LaTeXe}},%
    pdfkeywords = {LaTeX, \TUDScript, Tutorial, Anwenderleitfaden},%
  }%
\endgroup%
\makeatother
%Das Paket \Package{glossaries} stellt für die Definition von Abkürzungen einen 
%speziellen Befehl bereit. Mit\Macro*{newacronym}\LParameter\Parameter{Label}%
%\Parameter{Abkürzung}\Parameter{Wortgruppe} wird eine Abkürzung definiert und 
%kann später über \Parameter{Label} genutzt werden. Die möglichen optionalen 
%Parameter können in der Dokumentation zu \Package{glossaries} nachgeschlagen 
%werden. Für ein kleines Beispiel werden drei Abkürzungen erstellt\dots
%
\begin{Tutorial}
\newacronym{apsp}{APSP}{All-Pairs Shortest Path}
\newacronym{spsp}{SPSP}{Single-Pair Shortest Path}
\newacronym{sssp}{SSSP}{Single-Source Shortest Path}
\end{Tutorial}
%
\dots und diese in einer kurzen Textpassage mit \Macro*{gls}\Parameter{Label} 
verwendet.
%
\begin{Tutorial}
\gls{apsp} und \gls{spsp} sowie \gls{sssp} werden beim zweiten mal zu
\gls{apsp} und \gls{spsp} sowie \gls{sssp} was absolut großartig ist
\end{Tutorial}
%
\dots und diese in einer kurzen Textpassage mit \Macro*{gls}\Parameter{Label} 
verwendet.
\gls{apsp} und \gls{spsp} sowie \gls{sssp} werden beim zweiten mal zu
\gls{apsp} und \gls{spsp} sowie \gls{sssp} was absolut großartig ist
%
\TutorialPreamble{%
und diese in einer kurzen Textpassage mit  und diese in einer 
kurzen Textpassage mit und diese in einer kurzen Textpassage mit
%\gls{apsp} und \gls{spsp} sowie \gls{sssp} werden beim zweiten mal zu
%\gls{apsp} und \gls{spsp} sowie \gls{sssp} was absolut großartig ist
}
\begin{Tutorial}
\gls{apsp} und \gls{spsp} sowie \gls{sssp} werden beim zweiten mal zu
\gls{apsp} und \gls{spsp} sowie \gls{sssp} was absolut großartig ist
\end{Tutorial}
%
\dots und diese in einer kurzen Textpassage mit \Macro*{gls}\Parameter{Label} 
verwendet.
\gls{apsp} und \gls{spsp} sowie \gls{sssp} werden beim zweiten mal zu
\gls{apsp} und \gls{spsp} sowie \gls{sssp} was absolut großartig ist
%
\printacronyms

\end{document}
