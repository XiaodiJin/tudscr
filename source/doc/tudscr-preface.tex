\addchap{\prefacename}
Die im Folgenden beschriebenen Klassen und Pakete wurden für das Erstellen von 
\hologo{LaTeX}"=Dokumenten im \CD der \TnUD entwickelt.%
\footnote{%
  \url{http://tu-dresden.de/cd}\hfill
  \url{http://tu-dresden.de/service/publizieren/cd/6_handbuch/index.html}%
}
Sie basieren auf den gerade im deutschsprachigen Raum häufig verwendeten 
\KOMAScript"=Klassen, welche eine Vielzahl von Einstellmöglichkeiten bieten, 
die weit über die Möglichkeiten der \hologo{LaTeX}"=Standardklassen 
hinausgehen. Zusätzlich bietet das hier dokumentierten \TUDScript-Bundle 
weitere, insbesondere das Dokumentlayout betreffende Auswahlmöglichkeiten.

Es sei angemerkt, dass die hier beschriebenen Klassen eine Abweichung vom \CD 
der \TnUD zulassen, da dieses gerade unter typographischen Gesichtspunkten 
durchaus als diskussionswürdig zu erachten ist. Mit den entsprechenden 
Einstellungen kann bis auf das Standardlayout der \KOMAScript"=Klassen 
zurückgestellt werden. Inwieweit der Nutzer der \TUDScript"=Klassen von diesen 
Möglichkeiten Gebrauch macht, bleibt ihm selbst überlassen. Ohne die gezielte 
Verwendung der entsprechenden Optionen werden standardmäßig alle Vorgaben des 
\CDs umgesetzt.

Dieses Handbuch ist die Anwenderdokumentation der Klassen und Pakete aus dem 
\TUDScript"=Bundle. Es werden Hinweise für eine einfache Installation und ein 
Überblick über die zusätzlich zu den \KOMAScript"=Klassen nutzbaren Optionen 
und Befehle gegeben. Dies bedeutet, dass Grundkenntnisse in der Verwendung von 
\hologo{LaTeX} vorausgesetzt werden. Sollten diese nicht vorhanden sein, wird 
dem Nutzer zumindest das Lesen der Kurzbeschreibung von \hologo{LaTeXe}
\hrfn{http://mirrors.ctan.org/info/lshort/german/l2kurz.pdf}{\File{l2kurz.pdf}}
dringend empfohlen. Für den stärker vertieften Einstieg in \hologo{LaTeX} gibt 
es eine \hrfn{http://www.fadi-semmo.de/latex/workshop/}{Workshop-Reihe} von 
Fadi~Semmo. Außerdem stellt Nicola~L.~C.~Talbot sehr ausführliche Tutorials für 
\hrfn{http://www.dickimaw-books.com/latex/novices/}{\hologo{LaTeX}-Novizen} und 
\hrfn{http://www.dickimaw-books.com/latex/thesis/}{Dissertationen} zur freien 
Verfügung. Außerdem werden in \autoref{part:additional} dieses Handbuchs 
Minimalbeispiele sowie etwas ausführlichere Tutorials angeboten.

Des Weiteren sollte \emph{jeder} Anwender das \hologo{LaTeXe}"=Sündenregister 
\hrfn{http://mirrors.ctan.org/info/l2tabu/german/l2tabu.pdf}{\File{l2tabu.pdf}}
kennen, um typische Fehler zu vermeiden. Antworten auf häufig gestellte Fragen 
liefert \hrfn{http://projekte.dante.de/DanteFAQ/WebHome}{DANTE"~FAQ}. Falls der 
Nutzer unerfahren bei der Verwendung von \KOMAScript{} sein sollte, so ist ein 
Blick in das dazugehörige Handbuch \scrguide sehr zu empfehlen, wenn nicht 
sogar unumgänglich.

Der aktuelle Stand der Klassen und Pakete aus dem \TUDScript-Bundle in der 
Version~\vTUDScript{} wurde nach bestem Wissen und Gewissen auf Herz und Nieren 
getestet. Dennoch kann nicht für das Ausbleiben von Fehlern garantiert werden. 
Beim Auftreten eines Problems sollte dieses genauso wie Inkompatibilitäten mit 
anderen Paketen im Forum unter
\begin{quoting}
\Forum*%
\end{quoting}
gemeldet werden. Für eine schnelle und erfolgreiche Fehlersuche sollte bitte 
ein \hrfn{http://www.komascript.de/minimalbeispiel}{\textbf{Minimalbeispiel}} 
bereitgestellt werden. Auf Anfragen ohne dieses werde ich gegebenenfalls 
verspätet oder gar nicht reagieren. Ebenso sind im genannten Forum auch 
\emph{Fragen}, \emph{Kritik} und \emph{Verbesserungsvorschläge}~-- sowohl das 
Bundle selbst als auch die Dokumentation betreffend~-- gerne gesehen. Da dieses 
Bundle in meiner Freizeit entstanden ist und auch gepflegt wird, bitte ich um 
Nachsicht, falls ich nicht sofort antworte und/oder eine Fehlerkorrektur 
vornehmen kann.

\makeatletter
\medskip
\noindent Falk Hanisch\newline
Dresden, \@date
\makeatother
