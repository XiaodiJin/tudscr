\newcommand*\TaT{\hyperref[sec:tips]{Tipps \& Tricks}:\xspace}
\chapter{Praktische Tipps \& Tricks}
\tudhyperdef*{sec:tips}%
\section{\NoCaseChange{\hologo{LaTeX}}-Editoren}
\tudhyperdef*{sec:tips:editor}%
%
Hier werden die gängigsten Editoren zum Erzeugen von \hologo{LaTeX}"=Dateien 
genannt. Ich persönlich bin mittlerweile sehr überzeugter Nutzer von 
\Application{\hologo{TeX}studio}, da dieser viele Unterstützungs- und 
Assistenzfunktionen bietet. Neben diesen gibt es noch weitere, gut nutzbare 
\hologo{LaTeX}-Editoren. Unabhängig von der Auswahl des Editors, sollte dieser 
auf jeden Fall eine Unicode"=Unterstützung~(UTF"~8) enthalten:
%
\begin{itemize}
\item \Application{\hologo{TeX}maker}
\item \Application{Kile}
\item \Application{\hologo{TeX}works}
\item \Application{\hologo{TeX}lipse}~-- Plug-in für \Application{Eclipse}
\item \Application{\hologo{TeX}nicCenter}
\item \Application{WinEdt}
\item \Application{LEd}~-- früher \hologo{LaTeX}~Editor
\item \Application{\hologo{LyX}}~-- grafisches Front"~End für \hologo{LaTeX}
\end{itemize}
%
Für \Application{\hologo{TeX}studio} wird im \GitHubRepo* das Archiv 
\hrfn{https://github.com/tud-cd/tudscr/releases/download/TeXstudio/tudscr4texstudio.zip}%
{\File{tudscr4texstudio.zip}} bereitgestellt, welches Dateien zur Erweiterung 
der automatischen Befehlsvervollständigung für \TUDScript enthält. Diese müssen 
unter Windows in \Path{\%APPDATA\%\textbackslash texstudio} beziehungsweise 
unter unixoiden Betriebssystemen in \Path{.config/texstudio} eingefügt werden.

Möchten Sie das grafische \hologo{LaTeX}-Frontend~\Application{\hologo{LyX}} 
für das Erstellen eines Dokumentes mit \TUDScript nutzen, so werden dafür 
spezielle Layout-Dateien benötigt, um die Klassendateien verwenden zu können. 
Diese sind zusammen mit einem \Application{\hologo{LyX}}-Dokument als Archiv 
\hrfn{https://github.com/tud-cd/tudscr/releases/download/LyX/tudscr4lyx.zip}%
{\File{tudscr4lyx.zip}} im \GitHubRepo* verfügbar. Die Layout-Dateien müssen 
dafür im \Application{\hologo{LyX}}"=Installationspfad in den passenden 
Unterordner kopiert werden. Dieser ist bei Windows
\Path{\%PROGRAMFILES(X86)\%\textbackslash{}LyX~2.1\textbackslash{}Resources\textbackslash{}layouts}
beziehungsweise bei unixoiden Betriebssystemen \Path{/usr/share/lyx/layouts}.
Anschließend muss LyX über den Menüpunkt \emph{Werkzeuge} neu konfiguriert 
werden. 



\section{Literaturverwaltung in \NoCaseChange{\hologo{LaTeX}}}
\ChangedAt{v2.02:\TaT Literaturverwaltung}
%
Die simpelste Variante, eine \hologo{LaTeX}-Literaturdatenbank zu verwalten, 
ist dies mit dem Editor manuell zu erledigen. Wesentlich komfortabler ist es 
jedoch, die Referenzverwaltung mit einer darauf spezialisierten Anwendung zu 
bewerkstelligen. Dafür gibt es zwei sehr gute Programme:
%
\begin{itemize}
\item \Application{Citavi}
\item \Application{JabRef}
\end{itemize}
%
Das Programm \Application{Citavi} ermöglicht den Import von bibliografischen 
Informationen aus dem Internet. Allerdings sind diese teilweise unvollständig 
oder mangelhaft. Mit \Application{JabRef} hingegen muss die Literaturdatenbank 
manuell erstellt werden. Allerdings lassen sich einzelne Einträge aus 
.bib-Dateien sehr importieren. Beide Anwendungen unterstützen den Export 
beziehungsweise die Erstellung von Datenbanken im Stil von \Package{biblatex}. 
Für \Application{JabRef} muss diese durch den Anwender explizit aktiviert 
werden.\footnote{Optionen/Einstellungen/Erweitert/BibLaTeX-Modus} 
Zur Verwendung der beiden Programme in Verbindung mit \Package{biblatex} und 
\Application{biber} gibt es ein gutes Tutorial unter diesem
\href{http://www.suedraum.de/latex/stammtisch/degenkolb_latex_biblatex_folien-final.pdf}{Link}.



\section{Worttrennungen in deutschsprachigen Texten}
\tudhyperdef*{sec:tips:hyphenation}
\ChangedAt{v2.02:\TaT Worttrennungen}
%
Die möglichen Trennstellen von Wörtern werden von \hologo{LaTeXe} mithilfe 
eines Algorithmus berechnet. Dieser ist jedoch in seiner ursprünglichen Form 
für die englische Sprache konzipiert worden. Für deutschsprachige Texte wird 
die Worttrennung~-- insbesondere bei zusammengeschriebenen Wörtern~-- mit dem 
Paket \Package{hyphsubst} entscheidend verbessert. Dafür wird ein um vielerlei 
Trennungsmuster ergänztes Wörterbuch aus dem Paket \Package{dehyph-exptl} 
genutzt. 

Das Paket \Package{hyphsubst} muss bereits \emph{vor} der Dokumentklasse selbst 
geladen werden. Außerdem wird das Paket \Package{babel} benötigt. Damit auch 
Wörter mit Umlauten richtig getrennt werden, ist zusätzlich die Verwendung des 
Paketes \Package{fontenc} mit der \PValue{T1}"~Schriftkodierung erforderlich. 
Der Beginn einer Dokumentpräambel könnte folgendermaßen aussehen:
%
\begin{quoting}[rightmargin=0pt]
\begin{Code}[escapechar=§]
\RequirePackage[ngerman=ngerman-x-latest]{hyphsubst}
\documentclass[ngerman,§\PName{Klassenoptionen}§]§\Parameter{Dokumentklasse}§
\usepackage{selinput}\SelectInputMappings{adieresis={ä},germandbls={ß}}
\usepackage[T1]{fontenc}
\usepackage{babel}
§\dots§
\end{Code}
\end{quoting}
%
Eine Anmerkung noch zur Trennung von Wörtern mit Bindestrichen. Normalerweise 
sind die beiden von \hologo{LaTeXe} verwendeten Zeichen für Bindestrich und 
Trennstrich identisch. Leider wird der Trennungsalgorithmus von \hologo{LaTeXe} 
bei Wörtern, welche bereits einen Bindestrich enthalten, außer Kraft gesetzt. 
In der Folge werden~-- in der deutschen Sprache durchaus öfter anzutreffende~-- 
Wortungetüme wie die \enquote{Donaudampfschifffahrts-Gesellschafterversammlung} 
normalerweise nur direkt nach dem angegebenen Bindestrich getrennt. 

Allerdings gibt es die Möglichkeit, das genutzte Zeichen für den Trennstrich 
zu ändern. Dafür ist das Laden der \PValue{T1}"~Schriftkodierung mit dem Paket 
\Package{fontenc} zwingend erforderlich. Wenn von der verwendeten Schrift 
nichts anderes eingestellt ist, liegen sowohl Binde- als auch Trennstrich auf 
Position~\PValue{45} der Zeichentabelle. In der \PValue{T1}"~Schriftkodierung 
befindet sich auf der Position~\PValue{127} glücklicherweise für gewöhnlich das 
gleiche Zeichen noch einmal. Dies ist jedoch von der verwendeten Schrift 
abhängig. Wird der Ausdruck \Macro{defaulthyphenchar}[\PValue{=127}] in der 
Dokumentpräambel verwendet, kann dieses Zeichen für den Trennstrich genutzt 
werden. Bei den Schriften des \TUDCDs ist dies bereits automatisch eingestellt.

Sollte trotz aller Maßnahmen dennoch einmal ein bestimmtes Wort falsch getrennt 
werden, so kann die Worttrennung dieses Wortes manuell und global geändert 
werden. Dies wird mit \Macro{hyphenation}[\PParameter{Sil-ben-tren-nung}] 
gemacht. Es ist zu beachten, dass dies für alle Flexionsformen des Wortes 
erfolgen sollte. Für eine lokale/temporäre Worttrennung kann mit Befehlen aus 
dem Paket \Package{babel} gearbeitet werden. Diese sind: 

\vskip\medskipamount\noindent
\begingroup
\newcommand*\listhyphens[2]{#1 & \PValue{#2} \tabularnewline}%
\begin{tabular}{@{}ll}
  \textbf{Beschreibung} & \textbf{Befehl} \tabularnewline
  \listhyphens{ausschließliche Trennstellen}{\textbackslash-}
  \listhyphens{zusätzliche Trennstellen}{"'-} 
  \listhyphens{Umbruch ohne Trennstrich}{"'"'}
  \listhyphens{Bindestrich ohne Umbruch}{"'\textasciitilde} 
  \listhyphens{Bindestrich, der weitere Trennstellen erlaubt}{"'=}
\end{tabular}
\endgroup



\section{Lokale Änderungen von Befehlen und Einstellungen}
\index{Befehlsdeklaration!Geltungsbereich}%
\ChangedAt{v2.02:\TaT Lokale Änderungen}
%
Ein zentraler Bestandteil von \hologo{LaTeX} ist die Verwendung von Gruppen 
oder Gruppierungen. Innerhalb dieser bleiben alle vorgenommenen Änderungen an 
Befehlen, Umgebungen oder Einstellungen lokal. Dies kann sehr nützlich sein, 
wenn beispielsweise das Verhalten eines bestimmten Makros einmalig oder 
innerhalb von selbst definierten Befehlen oder Umgebungen geändert werden, im 
Normalfall jedoch die ursprüngliche Funktionalität behalten soll.
\begin{Example}
\index{Schriftauszeichnung}%
Der Befehl \Macro{emph} wird von \hologo{LaTeX} für Hervorhebungen im Text 
bereitgestellt und führt normalerweise zu einer kursiven oder~-- falls kein 
Schriftschnitt mit echten Kursiven vorhanden ist~-- kursivierten Auszeichnung. 
Soll nun in einem bestimmten Abschnitt die Auszeichnung mit fetter Schrift 
erfolgen, kann der Befehl \Macro{emph} innerhalb einer Gruppierung geändert 
und verändert werden. Wird diese beendet, verhält sich der Befehl wie gewohnt.
\begin{Code}
In diesem Text wird ein bestimmtes \emph{Wort} hervorgehoben.

\begingroup
  \renewcommand*\emph[1]{\textbf{#1}}%
  In diesem Text wird ein bestimmtes \emph{Wort} hervorgehoben.
\endgroup

In diesem Text wird ein bestimmtes \emph{Wort} hervorgehoben.
\end{Code}
\end{Example}
Eine Gruppierung kann entweder mit \Macro*{begingroup} und \Macro*{endgroup} 
oder einfach mit einem geschweiften Klammerpaar \PParameter{\dots} definiert 
werden.



\section{Bezeichnung der Gliederungsebenen durch \Package{hyperref}}
\index{Querverweise}%
\ChangedAt{v2.02:\TaT Bezeichnungen der Gliederungsebenen}
%
Das Paket \Package{hyperref} stellt für Querverweise unter anderem den Befehl 
\Macro{autoref}[\Parameter{label}](\Package{hyperref})'none' zur Verfügung. Mit 
diesem wird~-- im Gegensatz zur Verwendung von \Macro{ref}~-- bei einer 
Referenz nicht nur die Nummerierung selber sondern auch das entsprechende 
Element wie Kapitel oder Abbildung vorangestellt. Bei der Benennung des 
referenzierten Elementes wird sequentiell geprüft, ob das Makro 
\Macro*{\PName{Element}\PValue{autorefname}} oder 
\Macro*{\PName{Element}\PValue{name}} existiert. Soll die Bezeichnung eines 
Elementes geändert werden, muss der entsprechende Bezeichner angepasst werden.
%
\begin{Example}
Bezeichnungen von Gliederungsebenen können folgendermaßen verändert werden.
\begin{Code}
\renewcaptionname{ngerman}{\sectionautorefname}{Unterkapitel}
\renewcaptionname{ngerman}{\subsectionautorefname}{Abschnitt}
\renewcaptionname{ngerman}{\subsubsectionautorefname}{Unterabschnitt}
\end{Code}
\end{Example}



\section{URL-Umbrüche im Literaturverzeichnis mit \Package{biblatex}}
\index{Literaturverzeichnis}%
\ChangedAt{v2.02:\TaT URL-Umbrüche im Literaturverzeichnis}
%
Wird das Paket \Package{biblatex} verwendet, kann es unter Umständen dazu 
kommen, das eine URL nicht vernünftig umbrochen werden. Ist dies der Fall, 
können die Zählern \Counter{biburlnumpenalty}(\Package{biblatex})'none', 
\Counter{biburlucpenalty}(\Package{biblatex})'none' und 
\Counter{biburllcpenalty}(\Package{biblatex})'none' erhöht werden. Die 
möglichen Werte liegen zwischen 0~und~10000, wobei es bei höheren Werte der 
Zähler zu mehr URL-Umbrüchen an 
Ziffern~(\Counter{biburlnumpenalty}(\Package{biblatex})'none'), 
Groß-~(\Counter{biburlucpenalty}(\Package{biblatex})'none') und 
Kleinbuchstaben~(\Counter{biburllcpenalty}(\Package{biblatex})'none') kommt. 
Genaueres hierzu ist der Dokumentation des \Package{biblatex}"=Paketes zu 
entnehmen.



\section{Zeilenabstände in Überschriften}
\tudhyperdef*{sec:tips:headings}%
%
Mit dem Paket \Package{setspace} kann der Zeilenabstand beziehungsweise der 
Durchschuss innerhalb des Dokumentes geändert werden. Sollte dieser erhöht 
worden sein, können die Abstände bei mehrzeiligen Überschriften als zu groß 
erscheinen. Um dies zu korrigieren kann mit dem Befehl 
\Macro{addtokomafont}[%
  \PParameter{disposition}\PParameter{%
    \Macro{setstretch}[\PParameter{1}](\Package{setspace})'none'%
  }%
] der Zeilenabstand aller Überschriften auf einzeilig zurückgeschaltet werden. 
Soll dies nur für eine bestimmte Gliederungsebene erfolgen, so ist der 
Parameter \PValue{disposition} durch das dazugehörige Schriftelement zu 
ersetzen.



\section{Warnung wegen zu geringer Höhe der Kopf-/Fußzeile}
\tudhyperdef*{sec:tips:headline}%
%
Wird das Paket \Package{setspace} verwendet, kann es passieren, dass nach der 
Änderung des Zeilenabstandes \emph{innerhalb} des Dokumentes eine oder beide 
der folgenden Warnungen erscheinen:
%
\begin{quoting}
\begin{Code}
scrlayer-scrpage Warning: \headheight to low.
scrlayer-scrpage Warning: \footheight to low.
\end{Code}
\end{quoting}
%
Dies liegt an dem durch den vergrößerten Zeilenabstand erhöhten Bedarf für die
Kopf- und Fußzeile, die Höhen können in diesem Fall direkt mit der Verwendung 
von \Macro{recalctypearea} angepasst werden. Allerdings ändert das den 
Satzspiegel im Dokument, was eine andere und durchaus berechtigte Warnung von 
\Package{typearea} zur Folge hat. Falls die Änderung des Durchschusses wirklich 
nötig ist, sollte dies in der Präambel des Dokumentes einmalig passieren. Dann 
entfallen auch die Warnungen.



\section{Einrückung von Tabellenspalten verhindern}%
\tudhyperdef*{sec:tips:table}
\index{Tabellen}%
%
Normalerweise wird in einer Tabelle vor \emph{und} nach jeder Spalte durch 
\hologo{LaTeXe} etwas horizontaler Raum mit \Macro*{hskip}\Length{tabcolsep} 
eingefügt.%
\footnote{%
  Der Abstand zweier Spalten beträgt folglich \PValue{2}\Length{tabcolsep}.%
}
Dies geschieht auch \emph{vor} der ersten und \emph{nach} der letzten Spalte. 
Diese optische Einrückung an den äußeren Rändern kann unter Umständen stören, 
insbesondere bei Tabellen, die willentlich~-- beispielsweise mit den Paketen 
\Package{tabularx}, \Package{tabulary} oder auch \Package{tabu}~-- über die 
komplette Seitenbreite aufgespannt werden.

Das Paket \Package{tabularborder} versucht, dieses Problem automatisiert zu 
beheben, ist jedoch nicht zu allen \hologo{LaTeXe}-Paketen für den Tabellensatz 
kompatibel, unter anderem auch nicht zu den drei zuvor genannten. Allerdings 
lässt sich dieses Problem manuell durch den Anwender lösen. 

Bei der Deklaration einer Tabelle kann mit~\PValue{@}\PParameter{\dots} vor und 
nach dem Spaltentyp angegeben werden, was anstelle von \Length{tabcolsep} vor 
beziehungsweise nach der eigentlichen Spalte eingeführt werden soll. Dies kann 
für das Entfernen der Einrückungen genutzt werden, indem an den entsprechenden 
Stellen~\PValue{@\PParameter{}} bei der Angabe der Spaltentypen vor der ersten 
und nach der letzten Tabellenspalte verwendet wird.
%
\begin{Example}
Eine Tabelle mit zwei Spalten, wobei bei einer die Breite automatisch berechnet 
wird, soll über die komplette Textbreite gesetzt werden. Dabei soll der Rand 
vor der ersten und nach der letzten entfernt werden.
\begin{Code}[escapechar=§]
\begin{tabularx}{\textwidth}{@{}lX@{}}
§\dots§ & §\dots§ \tabularnewline
§\dots§
\end{tabularx}
\end{Code}
\end{Example}




\section{Unterdrückung des Einzuges eines Absatzes}
\index{Absatzauszeichnung}%
%
Werden zur Auszeichnung von Absätzen im Dokument~-- wie es aus typografischer 
Sicht zumeist sinnvoll ist~-- Einzüge und keine vertikalen Abstände verwendet
(\KOMAScript-Option \Option{parskip=false}), kann es vorkommen, dass ein ganz 
bestimmter Absatz~-- beispielsweise der nach einer zuvor genutzten Umgebung 
folgende~-- ungewollt eingerückt ist. Dies kann sehr einfach behoben werden, 
indem direkt zu Beginn des Absatzes das Makro \Macro{noindent} aufgerufen wird. 
Soll das Einrücken von Absätzen nach ganz bestimmten Umgebungen oder Befehlen 
automatisiert unterbunden werden, ist das Paket \Package{noindentafter} zu 
empfehlen.



\section{Unterbinden des Zurücksetzens von Fußnoten}%
\tudhyperdef*{sec:tips:counter}%
\index{Fußnoten}%
%
Oft taucht die Frage auf, wie sich über Kapitel fortlaufende Fußnoten 
realisieren lassen. Dies ist sehr einfach mit dem Paket \Package{chngcntr} 
möglich. Nach dem Laden des Paketes, kann das Zurücksetzen des Zählers nach 
einem Kapitel mit \Macro{counterwithout*}[%
  \PParameter{footnote}\PParameter{chapter}%
](\Package{chngcntr})'none' deaktiviert werden. Auch andere 
\hologo{LaTeXe}-Zähler~-- wie beispielsweise der bereits vorgestellte 
\Counter{symbolheadings}~-- lassen sich mit diesem Paket manipulieren.



\section{Setzen von Einheiten mit \Package{siunitx}}
\tudhyperdef*{sec:tips:siunitx}%
\index{Einheiten}%
%
Wenn \Package{siunitx} in einem deutschsprachigen Dokument genutzt soll
werden, muss zumindest die richtige Lokalisierung mit
\Macro{sisetup}[\PParameter{locale = DE}](\Package{siunitx})'none' angegeben 
werden. Sollen auch die Zahlen richtig formatiert sein, müssen weitere 
Einstellungen vorgenommen werden. Die meiner Meinung nach besten sind die 
folgenden.
%
\begin{quoting}
\begin{Code}
\sisetup{%
  locale = DE,%
  input-decimal-markers={,},input-ignore={.},%
  group-separator={\,},group-minimum-digits=3%
}
\end{Code}
\end{quoting}
%
Das Komma kommt als Dezimaltrennzeichen zum Einsatz. Des Weiteren werden Punkte 
innerhalb der Zahlen ignoriert und eine Gruppierung von jeweils drei Ziffern 
vorgenommen. Alternativ zu diesem Paket kann übrigens auch \Package{units} 
verwendet werden.



\section{Warnung beim Erzeugen des Inhaltsverzeichnisses}
\index{Inhaltsverzeichnis}%
\ChangedAt{v2.02:\TaT Warnung beim Erzeugen des Inhaltsverzeichnisses}
\ChangedAt{v2.02:\TaT Hinweis auf Rand bei mehrzeiligen Einträgen ergänzt}
%
Wird mit \Macro{tableofcontents} das Inhaltsverzeichnis für ein Dokument mit 
einer dreistelligen Seitenanzahl erstellt, so erscheinen unter Umständen viele 
Warnungen mit der Meldung:
%
\begin{quoting}
\begin{Code}
overfull \hbox
\end{Code}
\end{quoting}
%
Das liegt daran, dass die Seitenzahlen in Verzeichnis in einer Box mit der 
Breite \Macro*{@pnumwidth} gesetzt werden. Der hierfür standardmäßig verwendete 
Wert von~\PValue{1.55em} ist in diesem Fall zu klein und sollte vergrößert 
werden. Dabei ist auch der rechte Rand für mehrzeilige Einträge im Verzeichnis 
\Macro*{@pnumwidth} zu vergrößern, welcher mit~\PValue{2.55em} voreingestellt 
ist. Die Anpassungen können folgendermaßen durchgeführt werden:
%
\begin{quoting}
\begin{Code}
\makeatletter
\renewcommand*\@pnumwidth{1.7em}
\renewcommand*\@tocrmarg{2.7em}
\makeatother
\end{Code}
\end{quoting}
%
Dabei sollte der eingesetzte Wert nicht zu groß ausfallen.



\section{Leer- und Satzzeichen nach \NoCaseChange{\hologo{LaTeX}}-Befehlen}%
\tudhyperdef*{sec:tips:xspace}%
\index{Typografie}%
%
Normalerweise \enquote{schluckt} \hologo{LaTeX} die Leerzeichen nach einem 
Makro ohne Argumente. Dies ist jedoch nicht immer~-- genau genommen in den 
seltensten Fällen~-- erwünscht. Für dieses Handbuch ist beispielsweise der 
Befehl \Macro*{TUD} definiert worden, um \enquote{\TUD{}} nicht ständig 
ausschreiben zu müssen. Um sich bei der Verwendung des Befehl innerhalb eines 
Satzes für den Erhalt eines folgenden Leerzeichens das Setzen der geschweiften 
Klammer nach dem Befehl zu sparen (\Macro*{TUD}[\PParameter{}]), kann 
\Macro{xspace}(\Package{xspace})'none' aus dem Paket \Package{xspace} genutzt 
werden. Damit wird ein folgendes Leerzeichen erhalten. Der Befehl \Macro*{TUD} 
ist wie folgt definiert:
%
\begin{quoting}
\begin{Code}
\newcommand*\TUD{Technische Universit\"at Dresden\xspace}
\end{Code}
\end{quoting}
%
Das Paket \Package{xpunctuate} erweitert die Funktionalität nochmals. Damit 
können auch Abkürzungen so definiert werden, dass ein versehentlicher Punkt 
ignoriert wird:
%
\begin{quoting}
\begin{Code}
\newcommand*\zB{z.\,B\xperiod}
\end{Code}
\end{quoting}



\section{Das Setzen von Auslassungspunkten}
\tudhyperdef*{sec:tips:dots}%
\index{Typografie}%
\ChangedAt{v2.02:\TaT Das Setzen von Auslassungspunkten}
%
Auslassungspunkte werden mit \hologo{LaTeXe} mit den Befehlen \Macro{dots} oder 
\Macro{textellipsis} gesetzt. Für gewöhnlich folgt diesen \emph{immer} ein 
Leerzeichen, was nicht in jedem Fall gewollt ist. Das Paket \Package{ellipsis} 
schafft hier Abhilfe, wobei die Option \Option*{xspace} führt dazu, dass nach 
der Verwendung eines der beiden Befehle automatisch ein Leerzeichen gesetzt 
wird. 
%
\begin{quoting}
\begin{Code}
\usepackage[xspace]{ellipsis}
\end{Code}
\end{quoting}
%
Im Ursprung ist es für das Setzen englischsprachiger Texte gedacht, wo zwischen 
Auslassungspunkten und Satzzeichen ein Leerzeichen gesetzt wird. Im Deutschen 
ist dies anders:
%
\begin{quoting}
\enquote{%
  Um eine Auslassung in einem Text zu kennzeichnen, werden drei Punkte gesetzt. 
  Vor und nach den Auslassungspunkten wird jeweils ein Wortzwischenraum 
  gesetzt, wenn sie für ein selbständiges Wort oder mehrere Wörter stehen. Bei 
  Auslassung eines Wortteils werden sie unmittelbar an den Rest des Wortes 
  angeschlossen. Am Satzende wird kein zusätzlicher Schlusspunkt gesetzt. 
  Satzzeichen werden ohne Zwischenraum angeschlossen.%
}~[Duden, 23. Aufl.]
\end{quoting} 
%
Um dieses Verhalten zu erreichen, sollte noch Folgendes in der Präambel 
eingefügt werden:
%
\begin{quoting}
\begin{Code}
\let\ellipsispunctuation\relax
\newcommand*\qdots{[\dots{}]\xspace}
\end{Code}
\end{quoting}
%
Der Befehl \Macro*{qdots} wird definiert, um Auslassungspunkte in eckigen 
Klammern (\POParameter{\dots}) setzen zu können, wie sie für das Kürzen von 
wörtlichen Zitaten häufig verwendet werden.



\section{Finden von unbekannten \NoCaseChange{\hologo{LaTeX}}-Symbolen}
\index{Symbole}%
%
Für \hologo{LaTeX} stehen jede Menge Symbole zur Verfügung, die allerdings 
nicht immer einfach zu finden sind. In der Zusammenfassung
\hrfn{http://mirrors.ctan.org/info/symbols/comprehensive/symbols-a4.pdf}%
{\File{symbols-a4.pdf}} werden viele Symbole aus mehreren Paketen aufgeführt. 
Alternativ kann \hrfn{http://detexify.kirelabs.org/classify.html}{Detexify} 
verwendet werden. Auf dieser Web-Seite wird das gesuchte Symbol einfach 
gezeichnet, die dazu ähnlichsten werden zurückgegeben.



\section{Änderung des Papierformates}
\index{Papierformat}%
%
Es kann vorkommen, dass innerhalb eines Dokumentes kurzzeitig das Papierformat 
geändert werden soll, um beispielsweise eine Konstruktionsskizze in der 
digitalen PDF"~Datei einzubinden. Dabei ist es mit der \KOMAScript-Option 
\Option{paper=\PSet} sowohl möglich, lediglich die Ausrichtung in ein
Querformat zu ändern, als auch die Größe des Papierformates selber.
%
\begin{Example}
Ein Dokument im A4"~Format soll kurzzeitig auf ein A3"=Querformat geändert 
werden. Das folgende Minimalbeispiel zeigt, wie dies mit den Mitteln von 
\KOMAScript{} über die beiden Einstellungen \Option{paper=landscape} und 
\Option{paper=A3} geändert werden kann.
\begin{Code}
\documentclass[paper=a4,pagesize]{tudscrreprt}
\usepackage{selinput}
\SelectInputMappings{adieresis={ä},germandbls={ß}}
\usepackage[T1]{fontenc}
\usepackage[ngerman]{babel}
\usepackage{blindtext}

\begin{document}
\chapter{Überschrift Eins}
\Blindtext

\cleardoublepage
\storeareas\PotraitArea% speichert den aktuellen Satzspiegel
\KOMAoptions{paper=A3,paper=landscape,DIV=current}
\chapter{Überschrift Zwei}
\Blindtext

\cleardoublepage
\PotraitArea% lädt den gespeicherten Satzspiegel
\chapter{Überschrift Drei}
\Blindtext
\end{document}
\end{Code}
\end{Example}

\ToDo[doc]{Schnittmarken mit \Package{crop} dokumentieren}[v2.05]
%\section{Schnittmarken}


\section{Vermeiden des Skalierens einer PDF-Datei beim Druck}
\ChangedAt{v2.04:\TaT Vermeiden des Skalierens einer PDF-Datei beim Druck}
%
Beim Erzeugen eines Druckauftrages einer PDF-Datei kann es unter Umständen dazu 
führen, dass diese durch den verwendeten PDF-Betrachter unnötigerweise vorher 
skaliert wird und dabei die Seitenränder vergrößert werden. Um dieses Verhalten 
für Dokumente, die mit \Engine{pdfTeX} erzeugt werden, zu unterdrücken, gibt es 
zwei Möglichkeiten:
%
\begin{enumerate}
\item Es kann \Macro*{hypersetup}[\PParameter{pdfprintscaling=None}] verwendet 
  werden, wenn im Dokument ohnehin das Paket \Package{hyperref} geladen wird.
\item Der Low-Level-Befehl
  \Macro*{pdfcatalog}[\PParameter{/ViewerPreferences<{}</PrintScaling/None>{}>}]
  hat das gleiche Verhalten und kann auch ohne das Paket \Package{hyperref} 
  genutzt werden.
\end{enumerate}
%
Weitere Informationen dazu sind unter \url{http://www.komascript.de/node/1897} 
zu finden.


\section{Warnung bei der Schriftgrößenwahl}
\ChangedAt{v2.04:\TaT Warnung bei der Schriftgrößenwahl}
%
Die im Dokument verwendete Schriftgröße kann bei den \KOMAScript-Klassen sehr 
einfach über die Option~\Option{fontsize} eingestellt werden, wobei diese immer 
als Klassenoption angegeben werden sollte. Bei relativ großen und kleinen 
Schriftgrößen kann dabei eine Warnung in der Gestalt 
%
\begin{quoting}
\begin{Code}[escapechar=§]
LaTeX Font Warning: Font shape `§\dots§' in size <xx> not available
\end{Code}
\end{quoting}
%
auftreten. Dies daran, dass zum Zeitpunkt des Ladens einer Klasse immer nach 
den Computer-Modern-Standardschriften gesucht wird, unabhängig davon, ob im 
Nachhinein ein anderes Schriftpaket geladen wird. Diese sind de-facto nicht in 
alle Größen skalierbar. Um die Warnungen zu beseitigen, sollte das Paket 
\Package{fix-cm} mit \Macro*{RequirePackage} \emph{vor} der Dokumentklasse 
geladen werden: 
\begin{quoting}[rightmargin=0pt]
\begin{Code}[escapechar=§]
\RequirePackage{fix-cm}
\documentclass§\OParameter{Optionen}\Parameter{Klasse}§
\usepackage{selinput}\SelectInputMappings{adieresis={ä},germandbls={ß}}
\usepackage[T1]{fontenc}
§\dots§
\begin{document}
§\dots§
\end{document}
\end{Code}
\end{quoting}
%
Damit werden die Warnungen behoben.


\section{Platzierung von Gleitobjekten}
\tudhyperdef*{sec:tips:floats}{}%
\index{Gleitobjekte!Platzierung|?}%
%
Mit den beiden Paketen \Package{flafter} sowie \Package{placeins} gibt es die 
Möglichkeit, den für \hologo{LaTeX} zur Verfügung stehenden Raum für die 
Platzierung von Gleitobjekten einzuschränken. Darüber hinaus kann diese auch 
durch die im Folgenden aufgezählten Befehle beeinflusst werden. Die Makros 
lassen sich mit \Macro*{renewcommand*}[\Parameter{Befehl}\Parameter{Wert}] 
sehr einfach ändern.

\begin{Declaration}{\Macro*{floatpagefraction}}[0\floatpagefraction]
\begin{Declaration}{\Macro*{dblfloatpagefraction}}[0\dblfloatpagefraction]
\printdeclarationlist%
%
Der Wert gibt die relative Größe eines Gleitobjektes bezogen auf die Texthöhe 
(\Length*{textheight}) an, die mindestens erreicht sein muss, damit für dieses 
gegebenenfalls vor dem Beginn eines neuen Kapitels eine separate Seite erzeugt 
wird. Dabei wird einspaltiges (\Macro*{floatpagefraction}) und zweispaltiges 
(\Macro*{dblfloatpagefraction}) Layout unterschieden. Der Wert für beide 
Befehle sollte im Bereich von~\PValue{0.5}\dots\PValue{0.8} liegen.
\end{Declaration}
\end{Declaration}

\begin{Declaration}{\Macro*{topfraction}}[0\topfraction]
\begin{Declaration}{\Macro*{dbltopfraction}}[0\dbltopfraction]
\printdeclarationlist%
%
Diese Werte geben den maximalen Seitenanteil für Gleitobjekte, die am oberen 
Seitenrand platziert werden, für einspaltiges und zweispaltiges Layout an. Er 
sollte im Bereich von \PValue{0.5}\dots\PValue{0.8} liegen und größer als 
\Macro*{floatpagefraction} beziehungsweise \Macro*{dblfloatpagefraction} sein.
\end{Declaration}
\end{Declaration}

\begin{Declaration}{\Macro*{bottomfraction}}[0\bottomfraction]
\printdeclarationlist%
%
Dies ist der maximale Seitenanteil für Gleitobjekte, die am unteren Seitenrand 
platziert werden. Er sollte zwischen~\PValue{0.2} und~\PValue{0.5} betragen.
\end{Declaration}

\begin{Declaration}{\Macro*{textfraction}}[0\textfraction]
\printdeclarationlist%
%
Dies ist der Mindestanteil an Text, der auf einer Seite mit Gleitobjekten 
vorhanden sein muss, wenn diese nicht auf einer eigenen Seite ausgegeben 
werden. Er sollte im Bereich von~\PValue{0.1}\dots\PValue{0.3} liegen.
\end{Declaration}

\begin{Declaration}{\Counter*{totalnumber}}[\arabic{totalnumber}]
\begin{Declaration}{\Counter*{topnumber}}[\arabic{topnumber}]
\begin{Declaration}{\Counter*{dbltopnumber}}[\arabic{dbltopnumber}]
\begin{Declaration}{\Counter*{bottomnumber}}[\arabic{bottomnumber}]
\printdeclarationlist%
%
Außerdem gibt es noch Zähler, welche die maximale Anzahl an Gleitobjekten pro 
Seite insgesamt (\Counter*{totalnumber}), am oberen (\Counter*{topnumber}) und 
am unteren Rand der Seite (\Counter*{bottomnumber}) sowie im zweispaltigen Satz
beide Spalten überspannend (\Counter*{dbltopnumber}) festlegen. Die Werte 
können mit \Macro*{setcounter}[\Parameter{Zähler}\Parameter{Wert}] geändert 
werden.
\end{Declaration}
\end{Declaration}
\end{Declaration}
\end{Declaration}

\begin{Declaration}{\Length*{@fptop}}
\begin{Declaration}{\Length*{@fpsep}}
\begin{Declaration}{\Length*{@fpbot}}
\begin{Declaration}{\Length*{@dblfptop}}
\begin{Declaration}{\Length*{@dblfpsep}}
\begin{Declaration}{\Length*{@dblfpbot}}
\printdeclarationlist%
%
Sind vor Beginn eines Kapitels noch Gleitobjekte verblieben, so werden diese 
durch \hologo{LaTeX} normalerweise auf einer separaten vertikal zentriert Seite 
ausgegeben. Dabei bestimmen diese Längen jeweils den Abstand vor dem ersten 
Gleitobjekt zum oberen Seitenrand (\Length*{@fptop}, \Length*{@dblfptop}), 
zwischen den einzelnen Objekten (\Length*{@fpsep}, \Length*{@dblfpsep}) sowie 
zum unteren Seitenrand (\Length*{@fpbot}, \Length*{@dblfpbot}). Soll dies nicht 
geschehen, können die Längen durch den Anwender geändert werden.
\end{Declaration}
\end{Declaration}
\end{Declaration}
\end{Declaration}
\end{Declaration}
\end{Declaration}
%
\begin{Example}
Alle Gleitobjekte auf einer dafür speziell gesetzten Seite sollen direkt zu 
Beginn dieser ausgegeben werden. In der Dokumentpräambel lässt sich für dieses 
Unterfangen Folgendes nutzen:
\begin{Code}
\makeatletter
\setlength{\@fptop}{0pt}
\setlength{\@dblfptop}{0pt}% twocolumn
\makeatother
\end{Code}
\end{Example}



\section{Automatisiertes Einbinden von \Application{Inkscape}-Grafiken }
\tudhyperdef*{sec:tips:svg}%
\index{Grafiken}%
%
Das automatisierte Einbinden von \Application{Inkscape}-Grafiken wird auf 
\CTAN[pkg/svg-inkscape]{\Package*{svg-inkscape}'none'} erläutert. Hier wird ein 
daraus abgeleiteter und verbesserter Ansatz vorgestellt, um 
\Application{Inkscape}-Grafiken \textbf{automatisiert} in ein 
\hologo{LaTeX}"=Dokument einzubinden. Nutzer von unixartigen Systemen können 
alternativ zu diesem auch das Paket \Package{svg} nutzen, welches den folgend 
erläuterten Befehl \Macro{includesvg} definiert.

Die mit \Application{Inkscape}|?| erstellte Grafik soll automatisch kompiliert 
und eingebunden werden. Dies soll allerdings nicht bei jeder Kompilierung des 
Hauptdokumentes erfolgen, sondern lediglich, wenn die originale Bilddatei 
geändert beziehungsweise aktualisiert wurde. Hierfür wird \Package{filemod} 
verwendet. Die automatisierte Übersetzung einer Grafik im SVG"~Format in eine 
PDF"~Datei und die daran anschließende Einbindung dieser in das Dokument ist 
mit der Definition von \Macro{includesvg}[\OParameter{Breite}\Parameter{Datei}] 
in der Präambel des Dokumentes wie folgt möglich:
%
\begin{Declaration*}{\Macro{includesvg}}
\begin{quoting}
\begin{Code}[escapechar=§]
\usepackage{filemod}
\newcommand*{\includesvg}[2][\textwidth]{%
  \def\svgwidth{#1}
  \filemodCmp{#2.pdf}{#2.svg}{}{%
    \immediate\write18{%
      inkscape -z -D --file=#2.svg --export-pdf=#2.pdf --export-latex
    }%
  }%
  \input{#2.pdf_tex}%
}
\end{Code}
\end{quoting}
\end{Declaration*}
%
Mit \Macro*{immediate}\Macro*{write18}[\Parameter{externer Aufruf}] wird das 
zwischenzeitliche Ausführen eines externen Programms beim Durchlauf von 
\Engine{pdfLaTeX}~-- in diesem Fall von \File{inkscape.exe}~-- möglich. Damit 
der externe Aufruf auch tatsächlich durchgeführt wird, ist die Ausführung von 
\Engine{pdfLaTeX} mit der Option \Path{-{}-shell-escape} beziehungsweise bei 
der Verwendung von \Distribution{\hologo{MiKTeX}} mit \Path{-{}-enable-write18} 
zwingend notwendig. Außerdem muss der Pfad zur Datei \File{inkscape.exe} dem 
System bekannt sein.%
\footnote{%
  Der Pfad zu \File{inkscape.exe} in der Umgebungsvariable \Path{PATH} des 
  Betriebssystems enthalten sein.
}
Bei der Verwendung des Befehls \Macro{includesvg} \emph{muss} der Dateiname 
\emph{ohne Endung} angegeben werden. Die einzubindende SVG"~Datei sollte sich 
hierbei im gleichen Pfad wie das Hauptdokument befinden. Ist die SVG"~Datei in 
einem Unterordner relativ zum Pfad des Hauptdokumentes, kann dieser einfach mit 
\Macro{includesvg}[\PParameter{\PName{Ordner}/\PName{Datei}}] im Argument 
angegeben werden.



\section{Fehlermeldung: ! No room for a new \textbackslash write}
\ChangedAt{v2.02:\TaT Fehler beim Schreiben von Hilfsdateien}
%
Für das Erstellen und Schreiben externer Hilfsdateien steht \hologo{LaTeXe} nur 
eine begrenzte Anzahl sogenannter Ausgabe-Streams zur Verfügung. Allein für 
jedes zu erstellende Verzeichnis reserviert \hologo{LaTeX} selbst jeweils einen 
neuen Stream. Auch einige bereits zuvor in diesem Handbuch vorgestellte, sehr 
hilfreiche Pakete~-- wie beispielsweise \Package{hyperref}, \Package{biblatex}, 
\Package{glossaries}, \Package{todonotes} oder auch \Package{filecontents}~-- 
benötigen eigene Hilfsdateien und öffnen für das Erstellen dieser einen 
Ausgabe-Stream oder mehr. Lädt der Anwender mehrere, in eine Hilfsdatei 
schreibende Pakete, kann es zur der Fehlermeldung
%
\begin{quoting}
\begin{Code}
! No room for a new \write .
\end{Code}
\end{quoting}
%
kommen. Abhilfe schafft hier das Paket \Package{scrwfile}<koma-script>, welches 
einige Änderungen am \hologo{LaTeX}-Kern vornimmt, um die Anzahl der benötigten 
Hilfsdateien für das Schreiben aller Verzeichnisse zu reduzieren. Es muss 
einfach in der Präambel eingebunden werden. Sollten mit diesem Paket 
unerwarteter Weise Probleme auftreten, ist dessen Anleitung im \scrguide zu 
finden. Eine weitere Möglichkeit, das beschriebene Problem der geringen Menge 
an Ausgabe-Streams zu umgehen, stellt das Paket \Package{morewrites} dar. 
Allerdings ist dessen Verwendung nicht in allen Fällen von Erfolg gekrönt.



\section{Fehlermeldung beim Laden eines Paketes mit Optionen}
\ChangedAt{v2.05:\TaT Fehler beim Laden eines Paketes mit Optionen}
%
Es kann unter Umständen passieren, dass beim Laden eines Paketes mit bestimmten 
Optionen via \Macro*{usepackage}[\OParameter{Option}\Parameter{Paket}] 
folgender Fehler ausgegeben wird:
%
\begin{quoting}
\begin{Code}[escapechar=§]
! LaTeX Error: Option clash for package <§\dots§>.
\end{Code}
\end{quoting}
%
Mit großer Sicherheit wird das angeforderte Paket bereits durch die verwendete 
Dokumentklasse oder ein anderes Paket geladen. Im einfachsten Fall genügt es, 
bereits vor dem Laden der Dokumentklasse mit \Macro*{documentclass} durch 
\Macro{PassOptionsToPackage}[\Parameter{Optionen}\Parameter{Paket}] die 
gewünschten Optionen an das Paket weiterzureichen.



\section{Probleme bei der Verwendung von \Package{auto-pst-pdf}}
\tudhyperdef*{sec:tips:auto-pst-pdf}
\ChangedAt{v2.02:\TaT Hinweise zum Paket \Package{auto-pst-pdf}}
%
Bei der Verwendung von \Engine{pdfLaTeX} liest das Paket \Package{auto-pst-pdf} 
die Präambel ein und erstellt anschließend über den PostScript"=Pfad 
\Path{latex \textrightarrow{} dvips \textrightarrow{} ps2pdf} eine PDF-Datei, 
welche lediglich alle in den vorhandenen 
\Environment{pspicture}(\Package{pstricks})'none'"=Umgebungen erstellten 
Grafiken enthält. Das Paket \Package{ifpdf} stellt das Makro 
\Macro{ifpdf}(\Package{ifpdf})'none' bereit, mit welchem unterschieden werden 
kann, ob \Engine{pdfLaTeX} als Textsatzsystem verwendet wird. Abhängig davon 
können unterschiedliche Quelltexte ausgeführt werden, was genutzt wird, um die 
nachfolgend beschriebenen Probleme zu beheben.
%
\begin{quoting}
\begin{Code}
\usepackage{ifpdf}
\end{Code}
\end{quoting}

\minisec{Die gleichzeitige Verwendung von \Package{floatrow}}
Das Paket \Package{floatrow} stellt Befehle bereit, mit denen die Beschriftung 
von Gleitobjekten sehr bequem gesetzt werden können. Diese Setzen ihren Inhalt 
erst in einer Box, um deren Breite zu ermitteln und diese anschließend 
auszugeben. In Kombination mit \Package{auto-pst-pdf} führt das zu einer 
doppelten Erstellung der gewünschten Abbildung. Um dies zu vermeiden, müssen 
die durch \Package{floatrow} bereitgestellten Befehle \enquote{unschädlich} 
gemacht werden. Die fraglichen Befehlen akzeptieren allerdings bis zu drei 
optionale Argumente \emph{vor} den beiden obligatorischen, was für die 
Benutzerschnittstelle für die (Re-)Definition durch \hologo{LaTeXe} 
normalerweise nicht vorgesehen ist. Deshalb wird das Paket \Package{xparse} 
geladen, mit welchem dies möglich wird. Genaueres dazu ist der dazugehörigen 
Paketdokumentation zu entnehmen. Mit folgendem Quelltextauszug lassen sich die 
\Package{floatrow}-Befehle zusammen mit der 
\Environment{pspicture}(\Package{pstricks})'none'"=Umgebung wie gewohnt 
verwenden.
%
\begin{quoting}
\begin{Code}
\usepackage{floatrow}
\usepackage{xparse}
\ifpdf\else
  \RenewDocumentCommand{\fcapside}{ooo+m+m}{#4#5}
  \RenewDocumentCommand{\ttabbox}{ooo+m+m}{#4#5}
  \RenewDocumentCommand{\ffigbox}{ooo+m+m}{#4#5}
\fi
\end{Code}
\end{quoting}

\minisec{Die parallele Nutzung von \Package{tikz} und \Package{todonotes}}
Mit dem Paket \Package{tikz}~-- und auch allen anderen Paketen die 
selbiges nutzen wie beispielsweise \Package{todonotes}~-- gibt es in Verbindung 
mit \Package{auto-pst-pdf} ebenfalls Probleme. Lösen lässt sich dieses Dilemma, 
indem die fraglichen Pakete lediglich geladen werden, wenn \Engine{pdfLaTeX} 
aktiv ist.
%
\begin{quoting}[rightmargin=0pt]
\begin{Code}
\ifpdf
  \usepackage{tikz}%\dots gegebenenfalls weitere auf tikz basierende Pakete
\fi
\end{Code}
\end{quoting}
